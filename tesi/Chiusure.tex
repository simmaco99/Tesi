\section{Chiusure}
Come gi\`a osservato nell'esempio introduttivo, il modello botton-up su una rete richiede un elevato numero di equazioni differenziali: le equazioni per i nodi dipendono dalle coppie, le coppie dalle triple e cos\`i via.\\
Tale approccio, al crescere del numero di nodi, risulterebbe computazionalmente intrattabile. Per risolvere questo problema dobbiamo trovare alcune semplificazioni che ci permettano di esprimere le coppie in termini dei singoli nodi, le triple in termini delle coppie e dei nodi e cos\`i via.\\
Se riusciamo a fare ci\`o, possiamo rompere la ``cascata" nella quale ogni struttura dipende da tutte le strutture di ordine superiore.\\ \\
Andiamo a presentare un approccio formale basato sul lavoro di Keeling~\cite{keeling1995ecology}  e van Baalen~\cite{van2000pair}.  A tal fine introduciamo i  \textit{coefficienti di correlazione}.\\
Siano $A, B\in \{ S, I,R\}$ e $(i,j)\in E$ allora 
$$C_{A_i B_j} = \frac{\angol{A_i B_j}}{\angol{A_i} \angol{B_j}}.$$
Tali coefficienti quantificano la propensione che due nodi adiacenti abbiamo stato differente o identico.\\
Se $C_{A_iB_j}=1$ allora possiamo, in modo equivalente, assuemere  $A_i$ e $B_j$ indipendenti.\\
Osserviamo che nel modello $SIR$ gli stati non sono per\`o indipendenti. I nodi infetti possono infettare i loro vicini dunque hanno una maggior probabilit\`a di essere uniti ad altri nodi infetti,  di conseguenza  $C_{I_i I_j}\geq 1$.  Diremo che $I_i$ e $I_j$ sono correlati positivamente.\\
Con medesime argomentazioni si arriva a dire che $S_i$ e $I_j$ sono correlati negativamente: $C_{S_i I_j}\leq 1$.\\
 In un certo senso, sapere che $j$ \`e infetto aumenta la nostra aspettativa  che $i$ sia infetto e diminuisce quella che sia sano.\\
  
\subsection{Chiusura al livello delle coppie}
Da quanto osservato nella parte introduttiva sulle chiusure, possiamo scrivere:
$$ \angol{ A_i B_j} = \angol{ A_i} \angol{B_j} C_{A_i B_j} \text{ dove } A,B\in \{S,I,R\} \text{ e } (i,j)\in E.$$ 
Assumendo, in prima approssimazione, l'indipendenza a livello delle coppie abbiamo 
$$ \angol{ A_i B_j} \approx \angol{A_i}\angol{B_j}.$$ 
Per enfatizzare che non abbiamo identit\`a esatte, quando andremo a risolvere un sistema ottenuto usando le chiusure denoteremo  con $\angol{X_i}$ l'approssimazione di $\angol{ S_i}$ e con $\angol{ Y_i}$ quella di $\angol{ I_i}$.\\ \\
Presentiamo il modello botton-up generale su una rete con $N$ nodi senza loop e mostriamo come usando l'indipendenza a livello delle coppie si riesca ad ottenere un sistema di equazioni differenziali chiuse.\\
Sia $G=(g_{ij})$ la matrice di adiacenza del grafo $G$. Assumiamo che il tasso di trasmissione da $i$ a $j$ sia $\tau g_{ij}$ e che il tasso di rimozione per il nodo $i$ sia $\gamma_i$ indipendentemente dallo stato di ogni altro nodo.\\
Dunque le equazione per il sistema diventano
\begin{equation}
\begin{aligned}
	 \angol{ S_i} =& - \sum_{j=1\atop{j\neq i }}^N g_{ij} \angol{ S_i I_j},\\
	 \angol{I_i} =&\spa \tau \sum_{j=1\atop{j\neq i}}^N  g_{ij} \angol{ S_i I_j} -\gamma_i \angol{I_i},
\end{aligned}
\end{equation}
dove ricordiamo che $\angol{R_i}=1-\angol{S_i}- \angol{I_i}$.\\
Usando l'indipendenza a livello delle coppie possiamo un sistema chiuso: 
\begin{equation}
\begin{aligned}
	 \angol{ X_i} =& -\tau \sum_{j=1\atop{j\neq i }}^N g_{ij}\angol{ X_i} \angol{Y_j},\\
	 \angol{Y_i} =&\spa \tau \sum_{j=1\atop{j\neq i}}^N  g_{ij}\angol{ X_i}\angol{Y_j} -\gamma_i \angol{Y_i}.
\end{aligned}
\label{Coppie3nodi}
\end{equation}
Possiamo scrivere il sistema precedente in forma vettoriale:
\begin{equation}
	\begin{aligned}
	\dot{\angol{X}} = & - \tau G X * Y,\\
	\dot{\angol{Y}}=& \spa \tau G X * Y - \Gamma Y,	
	\end{aligned}
\end{equation}
dove $\Gamma=diag(\gamma_1, \dots, \gamma_n)$ e il prodotto tra vettori \`e inteso elemento per elemento.\\

Nella Figura~\ref{fig::coppie3nodi} possiamo confrontare le soluzioni del modello esatto con l'approssimazione ottenuta  dall'indipendenza a livello delle coppie per il grafo~\ref{fig::3nodi}. Per non appesantire i grafici abbiamo tracciato solamente  $\angol{S_i}$ e $\angol{I_i}$ in funzione  del tempo: $\angol{ R_i}$ pu\`o essere ricavata.  
Dalla Figura~\ref{fig::coppie3nodi} (d) possiamo notare che il modello semplificato tende a sovrastimare la prevalenza della malattia. Tale sovrastima risulta sconveniente se volessimo usare queste  predizioni per intervenire con delle politiche di contenimento. Nelle sezioni successive ci domanderemo se sia possibile trovare una rappresentazione con un numero minore di variabili che sia pi\`u accurato o, meglio ancora, esatta. 

\begin{figure}[!h]
	\centering
\subfloat[][Nodo 2]
{\resizebox{0.4\textwidth}{!}{% This file was created by matlab2tikz.
%
\definecolor{mycolor1}{rgb}{0.00000,0.44700,0.74100}%
\definecolor{mycolor2}{rgb}{0.85000,0.32500,0.09800}%
%
\begin{tikzpicture}

\begin{axis}[%
width=6.028in,
height=4.754in,
at={(1.011in,0.642in)},
scale only axis,
xmin=0,
xmax=60,
xlabel style={font=\color{white!15!black}},
xlabel={T},
ymin=0,
ymax=1,
axis background/.style={fill=white},
legend style={legend cell align=left, align=left, draw=white!15!black}
]
\addplot [color=mycolor1, line width=2.0pt]
  table[row sep=crcr]{%
0	1\\
0.000167459095433972	0.999949763953885\\
0.000334918190867944	0.999899531272651\\
0.000502377286301916	0.999849301956071\\
0.000669836381735888	0.999799076003922\\
0.00150713185890575	0.999547996701732\\
0.00234442733607561	0.999297001476487\\
0.00318172281324547	0.999046090300034\\
0.00401901829041533	0.998795263144227\\
0.00820549567626463	0.99754238669048\\
0.0123919730621139	0.996291606536599\\
0.0165784504479632	0.99504291917508\\
0.0207649278338125	0.993796321104278\\
0.041697314763059	0.987594547951493\\
0.0626297016923055	0.981444485184988\\
0.083562088621552	0.975345701651817\\
0.104494475550799	0.969297769768593\\
0.209156410197031	0.939806128919933\\
0.313818344843264	0.91152368113044\\
0.418480279489496	0.884400880611941\\
0.523142214135729	0.858390124885504\\
0.693500300195606	0.818313029330461\\
0.863858386255483	0.780876134190323\\
1.03421647231536	0.745905882432082\\
1.20457455837524	0.713239208200554\\
1.44051447786383	0.671518451609108\\
1.67645439735243	0.633555768395972\\
1.91239431684103	0.599014330293434\\
2.14833423632963	0.567583625933305\\
2.44722092525848	0.531792785970123\\
2.74610761418733	0.500036637585254\\
3.04499430311618	0.471864642818689\\
3.34388099204503	0.446867232882206\\
3.7078444355742	0.420187626925138\\
4.07180787910338	0.397125358000084\\
4.43577132263256	0.37719800996968\\
4.79973476616174	0.359970808884354\\
5.20292074647974	0.343584549055943\\
5.60610672679773	0.329641377827843\\
6.00929270711573	0.317785061805884\\
6.41247868743373	0.307695074640984\\
6.85758807700814	0.298279020146722\\
7.30269746658256	0.290400815197997\\
7.74780685615697	0.283816683396636\\
8.19291624573139	0.278306887206634\\
8.68220109096914	0.273270062909561\\
9.17148593620689	0.269130237284364\\
9.66077078144463	0.26573389992122\\
10.1500556266824	0.262941774813527\\
10.6416653734136	0.260629008549993\\
11.1332751201447	0.258729901399203\\
11.6248848668759	0.257173395775351\\
12.1164946136071	0.255894991586587\\
12.6079732666127	0.254841779768896\\
13.0994519196183	0.253976898623833\\
13.5909305726239	0.253268003848913\\
14.0824092256295	0.252685736828123\\
14.573895267778	0.252205890773349\\
15.0653813099266	0.251811849840116\\
15.5568673520752	0.251488877503596\\
16.0483533942238	0.251223597539249\\
16.5398390198718	0.251004984161706\\
17.0313246455198	0.250825462765357\\
17.5228102711679	0.250678319536887\\
18.0142958968159	0.250557460402728\\
18.5057815459406	0.250457862043894\\
18.9972671950653	0.250376073653432\\
19.48875284419	0.250309036475977\\
19.9802384933148	0.250253974104222\\
20.5017696853463	0.250206086979697\\
21.0233008773778	0.250167237410805\\
21.5448320694093	0.250135800555606\\
22.0663632614408	0.25011029017459\\
22.6746017709939	0.250086404450525\\
23.282840280547	0.250067696703614\\
23.8910787901001	0.250053130020746\\
24.4993172996532	0.25004171845769\\
25.0869400675273	0.250032958718024\\
25.6745628354014	0.250026040102264\\
26.2621856032755	0.250020602126255\\
26.8498083711495	0.250016305998746\\
27.5306427936611	0.250012399912303\\
28.2114772161728	0.25000943014995\\
28.8923116386844	0.25000719673085\\
29.573146061196	0.250005498735476\\
30.3571240725376	0.250004004549794\\
31.1411020838792	0.250002916085038\\
31.9250800952207	0.250002142131246\\
32.7090581065623	0.250001579100913\\
33.6393150845161	0.250001077720724\\
34.5695720624699	0.250000734525865\\
35.4998290404237	0.250000514703531\\
36.4300860183775	0.250000365483857\\
37.5675806203718	0.250000223876343\\
38.7050752223662	0.250000135575646\\
39.8425698243606	0.250000092288829\\
40.980064426355	0.250000066799481\\
42.4285780340909	0.250000031279727\\
43.8770916418269	0.250000012344854\\
45.3256052495628	0.250000011743775\\
46.7741188572988	0.250000013124446\\
48.2741188572988	0.25000000571389\\
49.7741188572988	0.250000001873181\\
51.2741188572988	0.250000002198961\\
52.7741188572988	0.250000002800139\\
54.2741188572988	0.250000001219075\\
55.7741188572988	0.250000000399649\\
57.2741188572988	0.250000000469155\\
58.7741188572988	0.250000000597418\\
59.0805891429741	0.250000000528484\\
59.3870594286494	0.250000000467506\\
59.6935297143247	0.250000000413576\\
60	0.250000000365868\\
};
\addlegendentry{$\langle\text{ S}_\text{2}\text{ }\rangle\text{(t)}$}

\addplot [color=mycolor2, line width=2.0pt]
  table[row sep=crcr]{%
0	1\\
0.000167459095433972	0.999949763953857\\
0.000334918190867944	0.999899531272425\\
0.000502377286301916	0.999849301955311\\
0.000669836381735888	0.999799076002119\\
0.00150713185890575	0.999547996681205\\
0.00234442733607561	0.999297001399249\\
0.00318172281324547	0.999046090107033\\
0.00401901829041533	0.998795262755379\\
0.00820549567626463	0.997542383386964\\
0.0123919730621139	0.996291595178081\\
0.0165784504479632	0.995042892025075\\
0.0207649278338125	0.993796267848315\\
0.041697314763059	0.987594120420592\\
0.0626297016923055	0.981443049142888\\
0.083562088621552	0.975342320785816\\
0.104494475550799	0.969291216183545\\
0.209156410197031	0.939755849652793\\
0.313818344843264	0.911360973769593\\
0.418480279489496	0.884031201510103\\
0.523142214135729	0.857698167570851\\
0.693500300195606	0.81680944243538\\
0.863858386255483	0.778163206698545\\
1.03421647231536	0.741557574425101\\
1.20457455837524	0.706819568750158\\
1.44051447786383	0.661520154084226\\
1.67645439735243	0.619190189816724\\
1.91239431684103	0.579558910208721\\
2.14833423632963	0.542402918667076\\
2.44722092525848	0.498599669712652\\
2.74610761418733	0.458157040342809\\
3.04499430311618	0.420813606357304\\
3.34388099204503	0.386345848519706\\
3.7078444355742	0.347972902082767\\
4.07180787910338	0.31325430231197\\
4.43577132263256	0.28189855399673\\
4.79973476616174	0.253633521758195\\
5.20292074647974	0.225620214495843\\
5.60610672679773	0.200753539499827\\
6.00929270711573	0.178721689929983\\
6.41247868743373	0.159232440119444\\
6.85758807700814	0.140343689946708\\
7.30269746658256	0.12388731981209\\
7.74780685615697	0.109559462335178\\
8.19291624573139	0.0970876854824286\\
8.68220109096914	0.0852318595654657\\
9.17148593620689	0.0750446855847971\\
9.66077078144463	0.0662821438209136\\
10.1500556266824	0.05873444568597\\
10.6416653734136	0.0521933804030884\\
11.1332751201447	0.0465424373231981\\
11.6248848668759	0.0416498643809277\\
12.1164946136071	0.037403972127709\\
12.6079732666127	0.0337111086868088\\
13.0994519196183	0.0304900896694505\\
13.5909305726239	0.0276732370812083\\
14.0824092256295	0.0252032509689746\\
14.573895267778	0.0230315494528747\\
15.0653813099266	0.0211169831688443\\
15.5568673520752	0.0194245619476555\\
16.0483533942238	0.0179245171666944\\
16.5398390198718	0.0165914776707145\\
17.0313246455198	0.0154037713069589\\
17.5228102711679	0.0143428582550358\\
18.0142958968159	0.0133928379709634\\
18.5057815459406	0.0125400450735501\\
18.9972671950653	0.0117727126576549\\
19.48875284419	0.0110806775888913\\
19.9802384933148	0.0104551522036236\\
20.5017696853463	0.00985562655117116\\
21.0233008773778	0.00931453062680311\\
21.5448320694093	0.00882504593325778\\
22.0663632614408	0.0083812721669376\\
22.6746017709939	0.00791465958461995\\
23.282840280547	0.00749626561710675\\
23.8910787901001	0.00712010602625066\\
24.4993172996532	0.00678105769059476\\
25.0869400675273	0.00648460726236425\\
25.6745628354014	0.00621525084903502\\
26.2621856032755	0.005970008377041\\
26.8498083711495	0.00574628074914531\\
27.5306427936611	0.00551099661871109\\
28.2114772161728	0.00529849326113302\\
28.8923116386844	0.00510613274889579\\
29.573146061196	0.00493163410296571\\
30.3571240725376	0.00475026671642952\\
31.1411020838792	0.00458737106367863\\
31.9250800952207	0.0044407235341731\\
32.7090581065623	0.00430841512066994\\
33.6393150845161	0.00416776865761866\\
34.5695720624699	0.00404265530493508\\
35.4998290404237	0.00393108663632639\\
36.4300860183775	0.00383137175427228\\
37.5675806203718	0.00372346054685316\\
38.7050752223662	0.00362889444570628\\
39.8425698243606	0.00354581178045761\\
40.980064426355	0.00347265037523595\\
42.4285780340909	0.00339178645080207\\
43.8770916418269	0.00332262732256281\\
45.3256052495628	0.00326332456346975\\
46.7741188572988	0.00321235771514442\\
48.2741188572988	0.00316702460979045\\
49.7741188572988	0.00312812143222317\\
51.2741188572988	0.00309468254080362\\
52.7741188572988	0.00306590043139326\\
54.2741188572988	0.00304109675586004\\
55.7741188572988	0.00301969934143004\\
57.2741188572988	0.00300122381818026\\
58.7741188572988	0.00298525882615117\\
59.0805891429741	0.00298227318141278\\
59.3870594286494	0.00297937483573661\\
59.6935297143247	0.00297656116518997\\
60	0.00297382963035769\\
};
\addlegendentry{$\langle\text{ X}_\text{2}\text{ }\rangle\text{ (t)}$}

\end{axis}
\end{tikzpicture}%} 
 \quad 
\resizebox{0.4\textwidth}{!}{ % This file was created by matlab2tikz.
%
\definecolor{mycolor1}{rgb}{0.00000,0.44700,0.74100}%
\definecolor{mycolor2}{rgb}{0.85000,0.32500,0.09800}%
%
\begin{tikzpicture}

\begin{axis}[%
width=0.39\columnwidth,
height=1.9in,
at={(1.011in,0.642in)},
scale only axis,
xmin=0,
xmax=60,
xlabel style={font=\color{white!15!black}},
xlabel={T},
ymin=0,
ymax=0.6,
axis background/.style={fill=white},
legend style={legend cell align=left, align=left, draw=none,fill=none}
]
\addplot [color=mycolor1, line width=2.0pt]
  table[row sep=crcr]{%
0	0\\
0.000167459095433972	5.02356254883503e-05\\
0.000334918190867944	0.000100467044890239\\
0.000502377286301916	0.000150694258501483\\
0.000669836381735888	0.000200917266617879\\
0.00150713185890575	0.000451969235128613\\
0.00234442733607561	0.000702916110626103\\
0.00318172281324547	0.000953757930062376\\
0.00401901829041533	0.00120449473037694\\
0.00820549567626463	0.00245660473713581\\
0.0123919730621139	0.00370609479958259\\
0.0165784504479632	0.00495296952109098\\
0.0207649278338125	0.00619723349725102\\
0.041697314763059	0.0123795525149475\\
0.0626297016923055	0.0184972877685076\\
0.083562088621552	0.0245510050044134\\
0.104494475550799	0.0305412652287822\\
0.209156410197031	0.0595600686924977\\
0.313818344843264	0.0870738896256172\\
0.418480279489496	0.113147672208371\\
0.523142214135729	0.137843781066386\\
0.693500300195606	0.175249338581645\\
0.863858386255483	0.209405409857356\\
1.03421647231536	0.240538830760215\\
1.20457455837524	0.268862525001243\\
1.44051447786383	0.303818596471941\\
1.67645439735243	0.334245921772327\\
1.91239431684103	0.360582367596924\\
2.14833423632963	0.383230967745073\\
2.44722092525848	0.407198365602763\\
2.74610761418733	0.426485640869426\\
3.04499430311618	0.441672920138603\\
3.34388099204503	0.453286293297944\\
3.7078444355742	0.463275846874522\\
4.07180787910338	0.46935668981669\\
4.43577132263256	0.472139376138478\\
4.79973476616174	0.472172061936486\\
5.20292074647974	0.469567505094857\\
5.60610672679773	0.46467180684465\\
6.00929270711573	0.457921782162667\\
6.41247868743373	0.449708082295428\\
6.85758807700814	0.439335806446126\\
7.30269746658256	0.427911208540696\\
7.74780685615697	0.415715901892051\\
8.19291624573139	0.403001249816413\\
8.68220109096914	0.388671106840877\\
9.17148593620689	0.374149950243233\\
9.66077078144463	0.359594516257495\\
10.1500556266824	0.345144730282224\\
10.6416653734136	0.330843251166074\\
11.1332751201447	0.31682425223781\\
11.6248848668759	0.30314250445849\\
12.1164946136071	0.289845851051411\\
12.6079732666127	0.27697222726519\\
13.0994519196183	0.264532277469983\\
13.5909305726239	0.252536316282153\\
14.0824092256295	0.240992167355439\\
14.573895267778	0.229902115890175\\
15.0653813099266	0.219260240094611\\
15.5568673520752	0.209059232287995\\
16.0483533942238	0.199291195161469\\
16.5398390198718	0.189946340036884\\
17.0313246455198	0.181011536407207\\
17.5228102711679	0.172473590770781\\
18.0142958968159	0.164319488934899\\
18.5057815459406	0.156535802900637\\
18.9972671950653	0.14910805738424\\
19.48875284419	0.142022137868879\\
19.9802384933148	0.135264379521748\\
20.5017696853463	0.128437338075173\\
21.0233008773778	0.121948428479072\\
21.5448320694093	0.115782041394707\\
22.0663632614408	0.109923234113274\\
22.6746017709939	0.10345968067617\\
23.282840280547	0.0973725428757864\\
23.8910787901001	0.0916406186898327\\
24.4993172996532	0.0862438769737739\\
25.0869400675273	0.0813304947215163\\
25.6745628354014	0.0766957392759583\\
26.2621856032755	0.0723240733458858\\
26.8498083711495	0.068200783364252\\
27.5306427936611	0.0637157111733428\\
28.2114772161728	0.0595249505264587\\
28.8923116386844	0.0556093554280656\\
29.573146061196	0.051950953723168\\
30.3571240725376	0.0480350394326654\\
31.1411020838792	0.0444140402140323\\
31.9250800952207	0.0410658652957561\\
32.7090581065623	0.0379699469731079\\
33.6393150845161	0.0345974316000959\\
34.5695720624699	0.0315244003882625\\
35.4998290404237	0.0287244006816334\\
36.4300860183775	0.0261730527001414\\
37.5675806203718	0.0233588440636795\\
38.7050752223662	0.0208472681352205\\
39.8425698243606	0.0186060465411244\\
40.980064426355	0.0166057804025651\\
42.4285780340909	0.0143659448183652\\
43.8770916418269	0.0124283537428185\\
45.3256052495628	0.0107528736684299\\
46.7741188572988	0.00930335312641603\\
48.2741188572988	0.00800707471067509\\
49.7741188572988	0.00689150112952988\\
51.2741188572988	0.00593188939887653\\
52.7741188572988	0.00510596293416252\\
54.2741188572988	0.00439452344759861\\
55.7741188572988	0.00378226174883817\\
57.2741188572988	0.00325559841873243\\
58.7741188572988	0.00280230566423442\\
59.0805891429741	0.00271772603687272\\
59.3870594286494	0.00263569921449369\\
59.6935297143247	0.00255614817289934\\
60	0.00247899814359053\\
};
\addlegendentry{$\langle\text{ I}_\text{2}\text{ }\rangle\text{ (t)}$}

\addplot [color=mycolor2, line width=2.0pt]
  table[row sep=crcr]{%
0	0\\
0.000167459095433972	5.02356255165241e-05\\
0.000334918190867944	0.000100467045115613\\
0.000502377286301916	0.000150694259262062\\
0.000669836381735888	0.000200917268420599\\
0.00150713185890575	0.000451969255655102\\
0.00234442733607561	0.000702916187860346\\
0.00318172281324547	0.000953758123048095\\
0.00401901829041533	0.00120449511918631\\
0.00820549567626463	0.00245660803997705\\
0.0123919730621139	0.00370610615458687\\
0.0165784504479632	0.00495299665983842\\
0.0207649278338125	0.0061972867255334\\
0.041697314763059	0.0123799796075772\\
0.0626297016923055	0.0184987215786687\\
0.083562088621552	0.0245543787936758\\
0.104494475550799	0.0305478015879279\\
0.209156410197031	0.0596100683199002\\
0.313818344843264	0.0872352755037915\\
0.418480279489496	0.1135133727281\\
0.523142214135729	0.13852638857525\\
0.693500300195606	0.176725562555453\\
0.863858386255483	0.212056305065258\\
1.03421647231536	0.244767200155724\\
1.20457455837524	0.275073975429707\\
1.44051447786383	0.313422975959112\\
1.67645439735243	0.347943646809329\\
1.91239431684103	0.378992749240041\\
2.14833423632963	0.406871933412632\\
2.44722092525848	0.438037519492424\\
2.74610761418733	0.464975195482736\\
3.04499430311618	0.488066904527981\\
3.34388099204503	0.507646031039421\\
3.7078444355742	0.52717345030486\\
4.07180787910338	0.542415181193204\\
4.43577132263256	0.553810589679287\\
4.79973476616174	0.561764343985809\\
5.20292074647974	0.567010515859882\\
5.60610672679773	0.568966391920642\\
6.00929270711573	0.568067535257039\\
6.41247868743373	0.564713102425068\\
6.85758807700814	0.558593524173596\\
7.30269746658256	0.550362480383516\\
7.74780685615697	0.54040898211884\\
8.19291624573139	0.529074361767942\\
8.68220109096914	0.515373936774946\\
9.17148593620689	0.500700274220661\\
9.66077078144463	0.485337849105587\\
10.1500556266824	0.469524160995857\\
10.6416653734136	0.453379120670631\\
11.1332751201447	0.437140566465681\\
11.6248848668759	0.420941463771214\\
12.1164946136071	0.404888834318843\\
12.6079732666127	0.389072053748551\\
13.0994519196183	0.373553697130149\\
13.5909305726239	0.358385449135512\\
14.0824092256295	0.343606395048581\\
14.573895267778	0.329245001459219\\
15.0653813099266	0.315321618177901\\
15.5568673520752	0.301849360219801\\
16.0483533942238	0.288835620596475\\
16.5398390198718	0.27628316162386\\
17.0313246455198	0.264190983473947\\
17.5228102711679	0.252555090392891\\
18.0142958968159	0.24136910546898\\
18.5057815459406	0.230624777425767\\
18.9972671950653	0.220312399840216\\
19.48875284419	0.210421168130167\\
19.9802384933148	0.200939453944673\\
20.5017696853463	0.19131231472422\\
21.0233008773778	0.182117491735684\\
21.5448320694093	0.173339655520351\\
22.0663632614408	0.164963361661959\\
22.6746017709939	0.155681157696078\\
23.282840280547	0.146899954888701\\
23.8910787901001	0.138595994242475\\
24.4993172996532	0.130746073432756\\
25.0869400675273	0.123572286631831\\
25.6745628354014	0.11678159902841\\
26.2621856032755	0.110355047942582\\
26.8498083711495	0.104274367549387\\
27.5306427936611	0.0976386133862052\\
28.2114772161728	0.0914175980091737\\
28.8923116386844	0.0855865781167584\\
29.573146061196	0.0801220824785545\\
30.3571240725376	0.0742549188941794\\
31.1411020838792	0.0688124058708863\\
31.9250800952207	0.0637646639877352\\
32.7090581065623	0.0590837589691989\\
33.6393150845161	0.0539695326304018\\
34.5695720624699	0.0492947872074972\\
35.4998290404237	0.0450223790195588\\
36.4300860183775	0.0411181796559852\\
37.5675806203718	0.0367990869202093\\
38.7050752223662	0.0329317807381826\\
39.8425698243606	0.0294694424179177\\
40.980064426355	0.0263700045170798\\
42.4285780340909	0.0228891693906661\\
43.8770916418269	0.0198668481966104\\
45.3256052495628	0.0172429274273263\\
46.7741188572988	0.014965095315772\\
48.2741188572988	0.0129225931149068\\
49.7741188572988	0.01115863648395\\
51.2741188572988	0.00963532091069012\\
52.7741188572988	0.00831987518172544\\
54.2741188572988	0.00718397327506719\\
55.7741188572988	0.00620313644631283\\
57.2741188572988	0.00535621424249753\\
58.7741188572988	0.00462493465861914\\
59.0805891429741	0.00448828422601655\\
59.3870594286494	0.00435567233296861\\
59.6935297143247	0.0042269795528441\\
60	0.00410209004092822\\
};
\addlegendentry{$\langle\text{ Y}_\text{2}\text{ }\rangle\text{ (t)}$}

\end{axis}
\end{tikzpicture}%}} \\
\subfloat[][Nodo 3]
{\resizebox{0.4\textwidth}{!}{% This file was created by matlab2tikz.
%
\definecolor{mycolor1}{rgb}{0.00000,0.44700,0.74100}%
\definecolor{mycolor2}{rgb}{0.85000,0.32500,0.09800}%
%
\begin{tikzpicture}

\begin{axis}[%
width=0.39\columnwidth,
height=1.7in,
at={(1.011in,0.642in)},
scale only axis,
xmin=0,
xmax=60,
xlabel style={font=\color{white!15!black}},
xlabel={T},
ymin=0,
ymax=1,
axis background/.style={fill=white},
legend style={legend cell align=left, align=left, draw=none,fill=none}
]
\addplot [color=mycolor1, line width=2.0pt]
  table[row sep=crcr]{%
0	1\\
0.000167459095433972	0.999999998738142\\
0.000334918190867944	0.999999994952792\\
0.000502377286301916	0.999999988644289\\
0.000669836381735888	0.999999979812971\\
0.00150713185890575	0.999999897825981\\
0.00234442733607561	0.999999752819296\\
0.00318172281324547	0.99999954483513\\
0.00401901829041533	0.999999273915676\\
0.00820549567626463	0.999996976764418\\
0.0123919730621139	0.999993112547874\\
0.0165784504479632	0.999987686516463\\
0.0207649278338125	0.999980703907444\\
0.041697314763059	0.999922624618094\\
0.0626297016923055	0.999826408925982\\
0.083562088621552	0.99969269680636\\
0.104494475550799	0.999522120290137\\
0.209156410197031	0.998137960827395\\
0.313818344843264	0.995922465165433\\
0.418480279489496	0.992946027807028\\
0.523142214135729	0.989274937612289\\
0.693500300195606	0.981973940722739\\
0.863858386255483	0.973240038754978\\
1.03421647231536	0.963292530931682\\
1.20457455837524	0.952330841025336\\
1.44051447786383	0.935806342228975\\
1.67645439735243	0.918080997911094\\
1.91239431684103	0.899500127085142\\
2.14833423632963	0.880367859680095\\
2.44722092525848	0.85574171145015\\
2.74610761418733	0.83103916350853\\
3.04499430311618	0.806578409348801\\
3.34388099204503	0.782631978004113\\
3.7078444355742	0.75447211159862\\
4.07180787910338	0.727603684649032\\
4.43577132263256	0.702174223241644\\
4.79973476616174	0.678307316278175\\
5.20292074647974	0.653774724370522\\
5.60610672679773	0.631205361841917\\
6.00929270711573	0.610542558139064\\
6.41247868743373	0.591735043728731\\
6.85758807700814	0.57303680236404\\
7.30269746658256	0.556333540005082\\
7.74780685615697	0.541460456252127\\
8.19291624573139	0.528275695887157\\
8.68220109096914	0.51556125386222\\
9.17148593620689	0.504497996616038\\
9.66077078144463	0.494894174784991\\
10.1500556266824	0.486586354120359\\
10.6416653734136	0.479392347597474\\
11.1332751201447	0.473202420173063\\
11.6248848668759	0.46788601528397\\
12.1164946136071	0.463329681682318\\
12.6079732666127	0.459433513843843\\
13.0994519196183	0.456105361253136\\
13.5909305726239	0.453266757288285\\
14.0824092256295	0.450848775743308\\
14.573895267778	0.44879122631073\\
15.0653813099266	0.447042976012409\\
15.5568673520752	0.445559600087142\\
16.0483533942238	0.444301821295448\\
16.5398390198718	0.443235739886372\\
17.0313246455198	0.442333582646524\\
17.5228102711679	0.441571158188601\\
18.0142958968159	0.440926985076838\\
18.5057815459406	0.440382661675318\\
18.9972671950653	0.439923504076921\\
19.48875284419	0.439536690438452\\
19.9802384933148	0.439210799744039\\
20.5017696853463	0.438920754658941\\
21.0233008773778	0.438679302761171\\
21.5448320694093	0.438478627706432\\
22.0663632614408	0.43831175482993\\
22.6746017709939	0.438152012143916\\
23.282840280547	0.438023420740633\\
23.8910787901001	0.437920237430922\\
24.4993172996532	0.437837273729477\\
25.0869400675273	0.437772357070206\\
25.6745628354014	0.437719844553824\\
26.2621856032755	0.437677488337093\\
26.8498083711495	0.437643254426811\\
27.5306427936611	0.437611553445292\\
28.2114772161728	0.437586827964751\\
28.8923116386844	0.437567662580957\\
29.573146061196	0.437552729805634\\
30.3571240725376	0.437539406333623\\
31.1411020838792	0.43752943285221\\
31.9250800952207	0.437522069969169\\
32.7090581065623	0.437516567914547\\
33.6393150845161	0.437511657612987\\
34.5695720624699	0.437508192616622\\
35.4998290404237	0.437505840190485\\
36.4300860183775	0.437504189241777\\
37.5675806203718	0.437502680495585\\
38.7050752223662	0.437501704404973\\
39.8425698243606	0.437501154963458\\
40.980064426355	0.437500810027397\\
42.4285780340909	0.437500423216806\\
43.8770916418269	0.437500205882951\\
45.3256052495628	0.437500155093005\\
46.7741188572988	0.437500139033043\\
48.2741188572988	0.437500070492889\\
49.7741188572988	0.437500032582013\\
51.2741188572988	0.437500025442845\\
52.7741188572988	0.437500024043849\\
54.2741188572988	0.437500012593451\\
55.7741188572988	0.437500006149462\\
57.2741188572988	0.437500004486816\\
58.7741188572988	0.43750000393094\\
59.0805891429741	0.437500003525987\\
59.3870594286494	0.437500003162157\\
59.6935297143247	0.437500002835375\\
60	0.437500002541907\\
};
\addlegendentry{$\langle\text{ S}_\text{3}\text{ }\rangle\text{(t)}$}

\addplot [color=mycolor2, line width=2.0pt]
  table[row sep=crcr]{%
0	1\\
0.000167459095433972	0.999999998738121\\
0.000334918190867944	0.999999994952623\\
0.000502377286301916	0.999999988643719\\
0.000669836381735888	0.999999979811619\\
0.00150713185890575	0.999999897810584\\
0.00234442733607561	0.999999752761357\\
0.00318172281324547	0.999999544690347\\
0.00401901829041533	0.999999273623959\\
0.00820549567626463	0.999996974285392\\
0.0123919730621139	0.99999310402177\\
0.0165784504479632	0.999987666130806\\
0.0207649278338125	0.999980663908501\\
0.041697314763059	0.999922303080207\\
0.0626297016923055	0.999825327388393\\
0.083562088621552	0.999690146760756\\
0.104494475550799	0.999517169946184\\
0.209156410197031	0.998099562083633\\
0.313818344843264	0.995797602736572\\
0.418480279489496	0.992660787241997\\
0.523142214135729	0.988737569824434\\
0.693500300195606	0.980791887673879\\
0.863858386255483	0.971084816672113\\
1.03421647231536	0.959804485451343\\
1.20457455837524	0.947129575840581\\
1.44051447786383	0.92758521760773\\
1.67645439735243	0.906106445832667\\
1.91239431684103	0.883075108589426\\
2.14833423632963	0.858837590253955\\
2.44722092525848	0.826886007743043\\
2.74610761418733	0.794054411348015\\
3.04499430311618	0.760820010065433\\
3.34388099204503	0.727585063894323\\
3.7078444355742	0.687591104586465\\
4.07180787910338	0.648565972587078\\
4.43577132263256	0.610869108045145\\
4.79973476616174	0.574757882885561\\
5.20292074647974	0.536811750289604\\
5.60610672679773	0.501154822235437\\
6.00929270711573	0.46783841094304\\
6.41247868743373	0.436850883694166\\
6.85758807700814	0.405275435185886\\
7.30269746658256	0.376344399323862\\
7.74780685615697	0.349904623141994\\
8.19291624573139	0.325786226657295\\
8.68220109096914	0.301742096377738\\
9.17148593620689	0.280056160791516\\
9.66077078144463	0.260503229035094\\
10.1500556266824	0.242871160246053\\
10.6416653734136	0.226891693201811\\
11.1332751201447	0.212470740859802\\
11.6248848668759	0.199443155444072\\
12.1164946136071	0.187660352686063\\
12.6079732666127	0.176992040603435\\
13.0994519196183	0.167316401231543\\
13.5909305726239	0.15852801642726\\
14.0824092256295	0.150533283579751\\
14.573895267778	0.143249079022458\\
15.0653813099266	0.136601930856374\\
15.5568673520752	0.130526600867737\\
16.0483533942238	0.124965248309587\\
16.5398390198718	0.119866568704441\\
17.0313246455198	0.115185000892432\\
17.5228102711679	0.110880064930525\\
18.0142958968159	0.106915746746887\\
18.5057815459406	0.103259964703636\\
18.9972671950653	0.0998841105748442\\
19.48875284419	0.0967626241214158\\
19.9802384933148	0.093872652366401\\
20.5017696853463	0.0910363694187017\\
21.0233008773778	0.0884158587940551\\
21.5448320694093	0.0859915788622176\\
22.0663632614408	0.0837460650816602\\
22.6746017709939	0.0813322583853868\\
23.282840280547	0.0791182798112997\\
23.8910787901001	0.0770846044770127\\
24.4993172996532	0.0752139465164833\\
25.0869400675273	0.0735470960341475\\
25.6745628354014	0.0720057390561177\\
26.2621856032755	0.0705788449983809\\
26.8498083711495	0.0692565149973051\\
27.5306427936611	0.06784353618056\\
28.2114772161728	0.066546673051544\\
28.8923116386844	0.0653549652901753\\
29.573146061196	0.0642586554629632\\
30.3571240725376	0.063103159557281\\
31.1411020838792	0.0620508751398579\\
31.9250800952207	0.0610914208358915\\
32.7090581065623	0.0602156212256792\\
33.6393150845161	0.0592737881763883\\
34.5695720624699	0.0584263864507068\\
35.4998290404237	0.0576629977791319\\
36.4300860183775	0.0569745079611135\\
37.5675806203718	0.0562228313805171\\
38.7050752223662	0.0555585438870261\\
39.8425698243606	0.0549707401692344\\
40.980064426355	0.0544500217669509\\
42.4285780340909	0.0538713152654743\\
43.8770916418269	0.0533739986675399\\
45.3256052495628	0.0529460801788547\\
46.7741188572988	0.0525774669931346\\
48.2741188572988	0.0522491769173989\\
49.7741188572988	0.0519673459336875\\
51.2741188572988	0.0517252105971922\\
52.7741188572988	0.0515170392194709\\
54.2741188572988	0.0513379628371783\\
55.7741188572988	0.0511838366660678\\
57.2741188572988	0.0510511258845694\\
58.7741188572988	0.0509368110338632\\
59.0805891429741	0.0509154776538754\\
59.3870594286494	0.0508947831556744\\
59.6935297143247	0.050874708138684\\
60	0.0508552338163797\\
};
\addlegendentry{$\langle\text{ X}_\text{3}\text{ }\rangle\text{ (t)}$}

\end{axis}
\end{tikzpicture}%} 
 \quad 
\resizebox{0.4\textwidth}{!}{ % This file was created by matlab2tikz.
%
\definecolor{mycolor1}{rgb}{0.00000,0.44700,0.74100}%
\definecolor{mycolor2}{rgb}{0.85000,0.32500,0.09800}%
%
\begin{tikzpicture}

\begin{axis}[%
width=0.39\columnwidth,
height=1.9in,
at={(1.011in,0.642in)},
scale only axis,
xmin=0,
xmax=60,
xlabel style={font=\color{white!15!black}},
xlabel={T},
ymin=0,
ymax=0.5,
axis background/.style={fill=white},
legend style={legend cell align=left, align=left, draw=none,fill=none}
]
\addplot [color=mycolor1, line width=2.0pt]
  table[row sep=crcr]{%
0	0\\
0.000167459095433972	1.26185129491217e-09\\
0.000334918190867944	5.04715161691134e-09\\
0.000502377286301916	1.13555206521399e-08\\
0.000669836381735888	2.01865781270714e-08\\
0.00150713185890575	1.02168885342488e-07\\
0.00234442733607561	2.47161385709955e-07\\
0.00318172281324547	4.55116590395451e-07\\
0.00401901829041533	7.25987035736679e-07\\
0.00820549567626463	3.02240839189113e-06\\
0.0123919730621139	6.88460568276832e-06\\
0.0165784504479632	1.23066742198438e-05\\
0.0207649278338125	1.92827249369075e-05\\
0.041697314763059	7.7267642315786e-05\\
0.0626297016923055	0.000173227718853331\\
0.083562088621552	0.000306444247749785\\
0.104494475550799	0.000476207943859229\\
0.209156410197031	0.00184891650923723\\
0.313818344843264	0.00403428386881361\\
0.418480279489496	0.00695384907760906\\
0.523142214135729	0.0105339891355051\\
0.693500300195606	0.0175971136151213\\
0.863858386255483	0.0259614923743754\\
1.03421647231536	0.0353880052833885\\
1.20457455837524	0.0456606118346815\\
1.44051447786383	0.0609285917757482\\
1.67645439735243	0.0770270262868401\\
1.91239431684103	0.0935965284435017\\
2.14833423632963	0.110324114104919\\
2.44722092525848	0.13133553630602\\
2.74610761418733	0.151803393776639\\
3.04499430311618	0.17143301715889\\
3.34388099204503	0.189977555508\\
3.7078444355742	0.210833991668167\\
4.07180787910338	0.229677134951398\\
4.43577132263256	0.246438421785966\\
4.79973476616174	0.261066449116606\\
5.20292074647974	0.274785754990105\\
5.60610672679773	0.286039368883271\\
6.00929270711573	0.294983433116465\\
6.41247868743373	0.30175643196477\\
6.85758807700814	0.306899090107461\\
7.30269746658256	0.309867303731916\\
7.74780685615697	0.310918643619185\\
8.19291624573139	0.310273875840171\\
8.68220109096914	0.307859497934145\\
9.17148593620689	0.30394952537288\\
9.66077078144463	0.298802849798728\\
10.1500556266824	0.292638032935273\\
10.6416653734136	0.285615796252768\\
11.1332751201447	0.277950926739595\\
11.6248848668759	0.269801213188547\\
12.1164946136071	0.261301267061574\\
12.6079732666127	0.252569321908963\\
13.0994519196183	0.243702027756301\\
13.5909305726239	0.234781977080018\\
14.0824092256295	0.225879536525819\\
14.573895267778	0.217053038249217\\
15.0653813099266	0.208348121918181\\
15.5568673520752	0.199801976342069\\
16.0483533942238	0.191445697640121\\
16.5398390198718	0.183303671016761\\
17.0313246455198	0.175392222310533\\
17.5228102711679	0.167723699389534\\
18.0142958968159	0.160307635901842\\
18.5057815459406	0.153150116044107\\
18.9972671950653	0.146252792268445\\
19.48875284419	0.139615587093586\\
19.9802384933148	0.133237287300585\\
20.5017696853463	0.126749085509615\\
21.0233008773778	0.120543262208316\\
21.5448320694093	0.114613005008405\\
22.0663632614408	0.108951184514624\\
22.6746017709939	0.102676735601474\\
23.282840280547	0.0967423452585563\\
23.8910787901001	0.0911334321360161\\
24.4993172996532	0.0858358971254951\\
25.0869400675273	0.0810003106792662\\
25.6745628354014	0.0764286533064571\\
26.2621856032755	0.0721080243560162\\
26.8498083711495	0.0680260834605836\\
27.5306427936611	0.0635793731585896\\
28.2114772161728	0.0594186100567405\\
28.8923116386844	0.0555263353843065\\
29.573146061196	0.0518861460511328\\
30.3571240725376	0.0479865022042955\\
31.1411020838792	0.0443777124961245\\
31.9250800952207	0.0410385808014444\\
32.7090581065623	0.0379494355212916\\
33.6393150845161	0.0345829658368376\\
34.5695720624699	0.0315142114252991\\
35.4998290404237	0.0287171284645176\\
36.4300860183775	0.0261678325282954\\
37.5675806203718	0.0233554939459096\\
38.7050752223662	0.0208451311709022\\
39.8425698243606	0.0186045988786763\\
40.980064426355	0.0166047671655161\\
42.4285780340909	0.0143654118090185\\
43.8770916418269	0.0124280915770715\\
45.3256052495628	0.0107526786215321\\
46.7741188572988	0.00930318087347226\\
48.2741188572988	0.00800698643404652\\
49.7741188572988	0.00689145956002702\\
51.2741188572988	0.00593185767404391\\
52.7741188572988	0.00510593367583665\\
54.2741188572988	0.00439450787540565\\
55.7741188572988	0.00378225394920435\\
57.2741188572988	0.00325559290546576\\
58.7741188572988	0.00280230102039895\\
59.0805891429741	0.00271772186404062\\
59.3870594286494	0.00263569546579028\\
59.6935297143247	0.00255614480597557\\
60	0.00247899512024932\\
};
\addlegendentry{$\langle\text{ I}_\text{3}\text{ }\rangle\text{ (t)}$}

\addplot [color=mycolor2, line width=2.0pt]
  table[row sep=crcr]{%
0	0\\
0.000167459095433972	1.26187242549581e-09\\
0.000334918190867944	5.04732065095519e-09\\
0.000502377286301916	1.13560911062091e-08\\
0.000669836381735888	2.01879302295867e-08\\
0.00150713185890575	1.02184281803407e-07\\
0.00234442733607561	2.47219320713395e-07\\
0.00318172281324547	4.55261361324533e-07\\
0.00401901829041533	7.26278723323254e-07\\
0.00820549567626463	3.02488691034271e-06\\
0.0123919730621139	6.89312914165404e-06\\
0.0165784504479632	1.23270514159998e-05\\
0.0207649278338125	1.93227030901108e-05\\
0.041697314763059	7.75888461837317e-05\\
0.0626297016923055	0.00017430755083688\\
0.083562088621552	0.000308988928504255\\
0.104494475550799	0.000481145277362125\\
0.209156410197031	0.00188713998878568\\
0.313818344843264	0.00415819235297196\\
0.418480279489496	0.00723605935851084\\
0.523142214135729	0.0110641401707356\\
0.693500300195606	0.0187582501128445\\
0.863858386255483	0.0280685939090168\\
1.03421647231536	0.038781348702539\\
1.20457455837524	0.050695670725046\\
1.44051447786383	0.0688333053117084\\
1.67645439735243	0.0884590045160488\\
1.91239431684103	0.109160760888305\\
2.14833423632963	0.130571221115777\\
2.44722092525848	0.158207981143474\\
2.74610761418733	0.185896662878587\\
3.04499430311618	0.213165780463282\\
3.34388099204503	0.239631557052223\\
3.7078444355742	0.270339304796926\\
4.07180787910338	0.298996850299779\\
4.43577132263256	0.325324823585313\\
4.79973476616174	0.349154022298033\\
5.20292074647974	0.372540604911139\\
5.60610672679773	0.392759067286727\\
6.00929270711573	0.409884549124148\\
6.41247868743373	0.424050853489826\\
6.85758807700814	0.436463050104125\\
7.30269746658256	0.445748970833454\\
7.74780685615697	0.452194594924938\\
8.19291624573139	0.456089769321328\\
8.68220109096914	0.457766673464304\\
9.17148593620689	0.457062988320793\\
9.66077078144463	0.454312115034089\\
10.1500556266824	0.449818784655169\\
10.6416653734136	0.443826478157626\\
11.1332751201447	0.436601503850688\\
11.6248848668759	0.428364081717156\\
12.1164946136071	0.419307707507557\\
12.6079732666127	0.409604053214299\\
13.0994519196183	0.399397555767262\\
13.5909305726239	0.388815151472685\\
14.0824092256295	0.377966163528239\\
14.573895267778	0.366944181775427\\
15.0653813099266	0.355829440028067\\
15.5568673520752	0.344690017892647\\
16.0483533942238	0.333583512446607\\
16.5398390198718	0.322558397412516\\
17.0313246455198	0.311655195698201\\
17.5228102711679	0.300907540097816\\
18.0142958968159	0.290343098921674\\
18.5057815459406	0.279984375468896\\
18.9972671950653	0.269849406585556\\
19.48875284419	0.259952393792112\\
19.9802384933148	0.25030421247411\\
20.5017696853463	0.240347267136109\\
21.0233008773778	0.23068614253712\\
21.5448320694093	0.221324840198831\\
22.0663632614408	0.212265130705681\\
22.6746017709939	0.20207997883171\\
23.282840280547	0.192302092044039\\
23.8910787901001	0.182926369836573\\
24.4993172996532	0.173945839215214\\
25.0869400675273	0.165637103962906\\
25.6745628354014	0.157680717688887\\
26.2621856032755	0.150067281055099\\
26.8498083711495	0.142786828909496\\
27.5306427936611	0.134754242803134\\
28.2114772161728	0.127138152884103\\
28.8923116386844	0.119921737176536\\
29.573146061196	0.113088113835243\\
30.3571240725376	0.105671559434967\\
31.1411020838792	0.0987150371633935\\
31.9250800952207	0.0921938384173302\\
32.7090581065623	0.0860839551570198\\
33.6393150845161	0.0793357364963272\\
34.5695720624699	0.0730968726927723\\
35.4998290404237	0.0673320550305249\\
36.4300860183775	0.062007867055897\\
37.5675806203718	0.0560505407170007\\
38.7050752223662	0.0506512402712961\\
39.8425698243606	0.045760260874588\\
40.980064426355	0.0413318198027856\\
42.4285780340909	0.0362960168768387\\
43.8770916418269	0.0318636645068222\\
45.3256052495628	0.0279645282209824\\
46.7741188572988	0.0245360808627083\\
48.2741188572988	0.0214226728395207\\
49.7741188572988	0.018699880307068\\
51.2741188572988	0.0163195611727412\\
52.7741188572988	0.0142393231839026\\
54.2741188572988	0.0124218796892963\\
55.7741188572988	0.0108344668021829\\
57.2741188572988	0.00944831909937817\\
58.7741188572988	0.00823820038111376\\
59.0805891429741	0.0080105617745301\\
59.3870594286494	0.00778916478021219\\
59.6935297143247	0.00757383996796787\\
60	0.0073644224945157\\
};
\addlegendentry{$\langle\text{ Y}_\text{3}\text{ }\rangle\text{ (t)}$}

\end{axis}
\end{tikzpicture}%}}
\\
\subfloat[][Prevalenza.]
{\resizebox{0.4\textwidth}{!}{% This file was created by matlab2tikz.
%
%The latest updates can be retrieved from
%  http://www.mathworks.com/matlabcentral/fileexchange/22022-matlab2tikz-matlab2tikz
%where you can also make suggestions and rate matlab2tikz.
%
\begin{tikzpicture}

\begin{axis}[%
width=6.028in,
height=4.754in,
at={(1.011in,0.642in)},
scale only axis,
xmin=0,
xmax=5,
ymin=0,
ymax=0.35,
axis background/.style={fill=white},
axis x line*=bottom,
axis y line*=left,
legend style={legend cell align=left, align=left, draw=white!15!black}
]
\addplot [color=red, line width=2.0pt]
  table[row sep=crcr]{%
0	0.333333333333333\\
0.000100475457260383	0.333327750896148\\
0.000200950914520766	0.333322167524441\\
0.00030142637178115	0.333316583218443\\
0.000401901829041533	0.333310997978383\\
0.000904279115343449	0.333283057775202\\
0.00140665640164536	0.333255094254932\\
0.00190903368794728	0.333227107446242\\
0.0024114109742492	0.333199097377757\\
0.00492329740575878	0.333058699137463\\
0.00743518383726836	0.332917723672439\\
0.00994707026877794	0.332776174514818\\
0.0124589567002875	0.332634055171426\\
0.0250183888578354	0.331915026764786\\
0.0375778210153833	0.331182255384855\\
0.0501372531729312	0.330436152327952\\
0.0626966853304791	0.329677114738795\\
0.102108236390274	0.327215481818432\\
0.141519787450068	0.324641174996368\\
0.180931338509863	0.321964171028335\\
0.220342889569658	0.31919350535293\\
0.274870838360705	0.315220730809272\\
0.329398787151753	0.311103336256022\\
0.383926735942801	0.306858491220908\\
0.438454684733849	0.302501546535542\\
0.511698843017727	0.296496384111662\\
0.584943001301606	0.290342511122817\\
0.658187159585485	0.284066091588779\\
0.731431317869363	0.277690592664632\\
0.823962581336954	0.269526726473224\\
0.916493844804545	0.261275455659342\\
1.00902510827214	0.252971216615607\\
1.10155637173973	0.244645176809034\\
1.21569652246265	0.23438574594883\\
1.32983667318558	0.224183113794988\\
1.44397682390851	0.214080766948463\\
1.55811697463144	0.204117072879612\\
1.68311697463144	0.193403784285744\\
1.80811697463144	0.182936514271466\\
1.93311697463144	0.172749616737409\\
2.05811697463144	0.162871139436702\\
2.18311697463144	0.153323565735268\\
2.30811697463144	0.144126198180392\\
2.43311697463144	0.135293531088611\\
2.55811697463144	0.126835090981413\\
2.68311697463144	0.118756329228822\\
2.80811697463144	0.11106042130276\\
2.93311697463144	0.10374712141988\\
3.05811697463144	0.0968128382965037\\
3.18311697463144	0.0902514673020522\\
3.30811697463144	0.0840557450763619\\
3.43311697463144	0.0782164406459592\\
3.55811697463144	0.0727224755499841\\
3.68311697463144	0.0675615941436991\\
3.80811697463144	0.0627213396997781\\
3.93311697463144	0.0581884346441867\\
4.05811697463144	0.0539488620496409\\
4.18311697463144	0.0499883521752367\\
4.30811697463144	0.0462930527442834\\
4.43311697463144	0.0428490255007505\\
4.55811697463144	0.0396422733863442\\
4.66858773097358	0.0369947840804212\\
4.77905848731572	0.034512898505142\\
4.88952924365786	0.0321877719670461\\
5	0.0300107450038088\\
};
\addlegendentry{Approssimata}

\addplot [color=blue, line width=2.0pt]
  table[row sep=crcr]{%
0	0.333333333333333\\
0.000100475457260383	0.33332775117652\\
0.000200950914520766	0.333322168645711\\
0.00030142637178115	0.333316585740814\\
0.000401901829041533	0.333311002461736\\
0.000904279115343449	0.333283080450326\\
0.00140665640164536	0.333255149070309\\
0.00190903368794728	0.333227208309998\\
0.0024114109742492	0.333199258157732\\
0.00492329740575878	0.333059366110503\\
0.00743518383726836	0.332919237528702\\
0.00994707026877794	0.332778870981109\\
0.0124589567002875	0.332638265051006\\
0.0250183888578354	0.331931595902953\\
0.0375778210153833	0.331218738109183\\
0.0501372531729312	0.330499530838155\\
0.0626966853304791	0.329773821516115\\
0.0991412937643315	0.327629833725129\\
0.135585902198184	0.32542682709702\\
0.172030510632036	0.323162271980266\\
0.208475119065889	0.320834051099024\\
0.257093885813804	0.317626142783061\\
0.305712652561719	0.314299781871326\\
0.354331419309635	0.310853927218438\\
0.40295018605755	0.307288328954497\\
0.46627950019307	0.302465571513947\\
0.52960881432859	0.297446337787991\\
0.592938128464109	0.292238061528472\\
0.656267442599629	0.286849067385365\\
0.733658471248229	0.280032591279204\\
0.811049499896829	0.272984735524843\\
0.888440528545428	0.265730903942892\\
0.965831557194028	0.258296309197627\\
1.05793596976172	0.249249184589584\\
1.15004038232942	0.240033136540703\\
1.24214479489711	0.230697304828548\\
1.3342492074648	0.221287878081579\\
1.44286725998794	0.210158037529048\\
1.55148531251107	0.199063059012029\\
1.66010336503421	0.188071612582134\\
1.76872141755734	0.177244701902424\\
1.88893196521442	0.165520374303466\\
2.00914251287149	0.154131833198533\\
2.12935306052856	0.143134740860959\\
2.24956360818563	0.132574608346779\\
2.37456360818563	0.12209553959032\\
2.49956360818563	0.112155781376778\\
2.62456360818563	0.102773341814187\\
2.74956360818563	0.0939584472228666\\
2.87456360818563	0.0857132435851595\\
2.99956360818563	0.0780307810161572\\
3.12456360818563	0.0708988244967405\\
3.24956360818563	0.0643015118334588\\
3.37456360818563	0.0582191503589672\\
3.49956360818563	0.0526277151834682\\
3.62456360818563	0.0475016394160471\\
3.74956360818563	0.0428146697023513\\
3.87456360818563	0.0385398175357336\\
3.99956360818563	0.0346492430664406\\
4.12456360818563	0.031115623106694\\
4.24956360818563	0.0279124562820615\\
4.37456360818563	0.0250140892127293\\
4.49956360818563	0.0223957818462954\\
4.62456360818563	0.0200340851137771\\
4.74956360818563	0.0179068527747512\\
4.81217270613922	0.0169229578724849\\
4.87478180409282	0.015990178489161\\
4.93739090204641	0.0151061256192865\\
5	0.0142684938646879\\
};
\addlegendentry{Esatta}

\end{axis}
\end{tikzpicture}%}}
\caption[Confronto tra modello esatto e chiuso alle coppie per~\ref{fig::3nodi}]{Divisione in classi dei  singoli nodi (a)(b) e grafico della prevalenza (c). Non abbiamo riportato i grafici relativi al nodo $1$ poich\`e in questo caso il modello esatto e il modello chiuso alle coppie coincidono.Per ottenere i grafici abbiamo risolto numericamente,  usando MATLAB,  il problema di Cauchy derivato dal modello esatto~\eqref{3nodi} (in blu) con quello approssimato  ottenuto   assumendo l'indipendenza delle coppie~\eqref{Coppie3nodi} (in rosso).  Per entrambi i modelli le condizioni iniziali sono stati puri~\eqref{statipuri},  inoltre,  nel modello esatto abbiamo supposto l'indipendenza statica delle condizioni iniziali. Per la sperimentazione abbiamo usato come parametri $\tau=0.3$ e $\gamma=0.1$.\\}
\label{fig::coppie3nodi}
\end{figure}

