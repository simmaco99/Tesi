%Formattazione e codifica caratteri
\usepackage[italian]{babel}
\usepackage[utf8]{inputenc}
\usepackage{microtype}
\usepackage{libertine}
% Tavola dei contenuti
\usepackage{tocloft}
\renewcommand{\cftdot}{}
% Figure
\usepackage{graphicx}
\usepackage{caption}
\usepackage{subfig}
% Matematica
\usepackage{amsmath,amsfonts,amssymb,amsthm}
\usepackage{color}
% Tabelle
\usepackage{booktabs}
%Listati
\usepackage{listings}
\lstloadlanguages{MATLAB}
% Testatine
\author{S. Di Lillo}
\usepackage{fancyhdr}
\pagestyle{fancy}
\setlength{\headheight}{15pt}
\fancyhf{}
\fancyhead[LE,RO]{S. Di Lillo}
\fancyhead[RE,LO]{\leftmark}
\fancyfoot[CE,CO]{}
\fancyfoot[LE,RO]{\thepage}
\renewcommand{\headrulewidth}{1.5pt}
% Titoli dei capitoli che "rubano" meno spazio
\usepackage{titlesec, blindtext}
\usepackage[dvipsnames]{xcolor}
\definecolor{gray75}{gray}{0.75}
\newcommand{\hsp}{\hspace{15pt}}
\titleformat{\chapter}[hang]{\LARGE\bfseries}{\thechapter\hsp\textcolor{gray75}{|}\hsp}{0pt}{\Large\bfseries}
\titleformat{\section}[hang]{\Large\bfseries}{\thesection\hsp\textcolor{gray75}{|}\hsp}{0pt}{\large\bfseries}

%Grafi
\usepackage{tikz-network}
\SetVertexStyle[MinSize=1\DefaultUnit,FillColor=white]
% Hyperref e metadati
\usepackage[linktocpage,
			colorlinks,
			citecolor=ForestGreen,
			bookmarks,
			pdftitle={Il Modello Epidemiologico SIR sulle Reti Complesse},
			pdfauthor={S. Di Lillo},
			pdfsubject={Tesi di Laurea Triennale},
			pdfkeywords={SIR,Epidemiologia,Reti Complesse},
			]{hyperref}


% Per importare le figure in MATLAB
\usepackage{pgfplots}
 \pgfplotsset{compat=newest}
  \usetikzlibrary{plotmarks}
  \usetikzlibrary{arrows.meta}
  \usepgfplotslibrary{patchplots}
  \usepackage{grffile}
\usepackage{tikz}
\pgfplotsset{plot coordinates/math parser=false}
  \newlength\figureheight
 \newlength\figurewidth

%Per le osservazioni
\theoremstyle{plain}
\newtheorem{oss}{Osservazione} 
\newtheorem{thm}{Teorema}[section] 
\newtheorem{prop}[thm]{Proposizione} 
%COMANDI PERSONALIZZATI 
\newcommand{\ro}{\mathcal R_0}
\newcommand{\di}{\,\,\mathrm{d} }
\newcommand{\tonde}[1]{\left( #1 \right)}
\newcommand{\ses}{\Leftrightarrow} 
\newcommand{\abs}[1]{\left\vert#1\right\vert}
\newcommand{\angol}[1]{\langle #1 \rangle}
\newcommand{\spa}{\mathbin{\textcolor{white}{-}}} %per indentare le equazioni in o-o-o serve pacchetto color

