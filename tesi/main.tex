%Carattere dimensione 12
\documentclass[12pt,a4paper,twoside]{report}

%Formattazione e codifica caratteri
\usepackage[italian]{babel}
\usepackage[utf8]{inputenc}
\usepackage{microtype}
\usepackage{libertine}
% Tavola dei contenuti
\usepackage{tocloft}
\renewcommand{\cftdot}{}
% Figure
\usepackage{graphicx}
% Matematica
\usepackage{amsmath,amsfonts,amssymb,amsthm}
% Tabelle
\usepackage{booktabs}
% Testatine
\author{S. Di Lillo}
\usepackage{fancyhdr}
\pagestyle{fancy}
\setlength{\headheight}{15pt}
\fancyhf{}
\fancyhead[LE,RO]{S. Di Lillo}
\fancyhead[RE,LO]{\leftmark}
\fancyfoot[CE,CO]{}
\fancyfoot[LE,RO]{\thepage}
\renewcommand{\headrulewidth}{1.5pt}
% Titoli dei capitoli che "rubano" meno spazio
\usepackage{titlesec, blindtext}
\usepackage[dvipsnames]{xcolor}
\definecolor{gray75}{gray}{0.75}
\newcommand{\hsp}{\hspace{15pt}}
\titleformat{\chapter}[hang]{\LARGE\bfseries}{\thechapter\hsp\textcolor{gray75}{|}\hsp}{0pt}{\Large\bfseries}
\titleformat{\section}[hang]{\Large\bfseries}{\thesection\hsp\textcolor{gray75}{|}\hsp}{0pt}{\large\bfseries}

% Hyperref e metadati
\usepackage[linktocpage,
			colorlinks,
			citecolor=ForestGreen,
			bookmarks,
			pdftitle={Il Modello Epidemiologico SIR sulle Reti Complesse},
			pdfauthor={S. Di Lillo},
			pdfsubject={Tesi di Laurea Triennale},
			pdfkeywords={SIR,Epidemiologia,Reti Complesse},
			]{hyperref}

% Comando per il testo riempitivo: rimuovere prima della fine
\usepackage{lipsum}

\begin{document}

\begin{titlepage}
\begin{figure}[t]
	\centering\includegraphics[width=0.35\textwidth]{figure/stemma_unipi.png}\\[1em]
	{\LARGE \textsc{Università di Pisa}}\\
	\vspace*{-0.5em}\rule{0.6\textwidth}{0.4pt}
\end{figure}
\begin{center}
	\textbf{ Dipartimento di Matematica\\ Corso di Laurea Triennale in Matematica \\}
	\vspace{15mm}
	{\centering Tesi di Laurea}\\[1em]
    {\LARGE{\bf Il Modello Epidemiologico SIR \\ sulle Reti Complesse}}\\
	\vspace{3mm}
	%{\LARGE{\bf Secondo Titolo}}\\
\end{center}

\vspace{36mm}

\begin{minipage}[t]{0.47\textwidth}
	{\large{\bf Relatore:\\Dott. Fabio Durastante}}
\end{minipage}\hfill\begin{minipage}[t]{0.47\textwidth}\raggedleft
	{\large{\bf Candidato:\\Simmaco Di Lillo\\ }}
\end{minipage}

\vfill
\centering{\rule{0.5\textwidth}{0.4pt}}\\
\centering{\large{\bf Anno Accademico 2020/2021 }}

\end{titlepage}

\begin{abstract}
\addcontentsline{toc}{chapter}{Sommario}
% Da scrivere alla fine, un sommario breve del contenuto finale della tesi
\cite{KISS}
\end{abstract}

\chapter{Epidemiologia matematica e reti}

\lipsum[1-5]

\section{Il Modello SIR}

\lipsum[6-10]

\section{Reti complesse e grafi}

\lipsum[6-10]

\chapter{Il modello SIR \textit{bottom-up} su una rete}

\lipsum[6-10]

\section{Chiusure}

\lipsum[6-10]

\subsection{Chiusura al livello delle coppie}

\lipsum[6-10]

\subsection{Chiusura al livello delle triple}

\lipsum[6-10]

\subsection{Approccio generale alla chiusura}

\lipsum[6-10]

\section{Un esempio completo}

\lipsum[6-10]

\chapter{Integrazione numerica del sistema di ODE}

\lipsum[6-7]

\section{Esempi numerici}

\lipsum[8-10]

\chapter*{Conclusioni}
\addcontentsline{toc}{chapter}{Conclusioni}

\lipsum[11-14]

\bibliography{references}
\addcontentsline{toc}{chapter}{Bibliografia}
\bibliographystyle{plain}

\cleardoublepage
\listoffigures
\addcontentsline{toc}{chapter}{Elenco delle figure}
\listoftables
\addcontentsline{toc}{chapter}{Elenco delle tabelle}
\tableofcontents

\end{document}