\chapter{Epidemiologia matematica e reti}
Introduciamo in questo capitolo il modello compartimentale di Kermack-McKendrick~\cite{kermack} e ne analizziamo le propriet\`a principali. Ci focalizzeremo in particolare sui suoi aspetti costitutivi, le propriet\`a analitiche e l'applicazione ad un primo caso realistico.
\section{Il modello SIR}{\label{modellosir}}
Il modello SIR \`e un modello compartimentale: la popolazione  viene suddivisa in tre classi: \begin{itemize}
	\item $S$: i \textit{suscettibili}  ovvero individui che possono contrarre la malattia;
	\item $I$: gli \textit{infetti} ovvero coloro che sono ammalati;
	\item $R$: i \textit{rimossi} ovvero  quelli tolti dalle classi precedenti perch\`e completamente guariti (dunque immuni).
	\end{itemize}
	Tale modello si basa su alcune assunzioni.
	\begin{itemize}
		\item[A1] Il numero della popolazione \`e costante nel tempo e verr\`a indicato con $N$, ovvero non si considerano nuove nascite o morti. Inoltre stiamo assumendo che $N$ sia sufficientemente grande da poter considerare la variazione in ogni classe continua e non discreta.
\item[A2] Esiste un fattore di contatto  $\beta$. Tale rapporto indica il valor medio di contatti per infettivo per unit\`a di tempo.
		\item[A3] Gli infetti lasciano la classe al tasso $\alpha$ per unit\`a di tempo e vanno nella classe $R$, un individuo che entra nella classe $R$ non uscir\`a da tale classe.
	\end{itemize}
Da queste considerazioni segue che 
\begin{equation}
\label{SIR}
\begin{aligned}
  S'&=-\beta S I,\\
  I'&= \beta S I -\alpha I,\\
  R'&=\alpha I.
\end{aligned}
\end{equation}
Poich\`e abbiamo assunto che $N=S+I+R$ sia costante, il sistema precedente risulta equivalente a 
\begin{equation}
\label{SI}
\begin{aligned}
  S'&=-\beta S I,\\
  I'&= \beta S I -\alpha I.
\end{aligned}
\end{equation}\\ 
Studiamo cosa succede se introduciamo un piccolo numero di infetti in una popolazione di suscettibili, ovvero consideriamo il sistema~\eqref{SI} con le condizioni iniziali
\begin{equation*}
\begin{aligned}
  I(0)&=I_0>0 ,\, I_0 \ll N \text{ e } \\
  S(0)&=S_0= N-I_0.
\end{aligned}
\end{equation*}
Da~\eqref{SI} osserviamo che $S'<0$ per ogni tempo $t$,  mentre $I'>0$ se~e~solo~se~$\frac{\beta S }{\alpha}~>~1 $.\\
Definiamo  
$$ \ro  = \frac{\beta S_0 }{\alpha}$$ 
il \textit{ numero  di riproduttivit\`a di base}. Esso  rappresenta il numero di individui infettati all'interno di una popolazione di suscettibili.\\
Tale valore ci permette di descrivere la fase iniziale dell'epidemia:
\begin{itemize}
	\item se $\ro<1$ allora l'epidemia si estingue infatti sotto queste condizioni $I'(t)<0$ per ogni tempo $t$,
	\item se $\ro>1$ allora $I$ inizialmente aumenta e dunque l'epidemia ha inizio.
\end{itemize}
Studiamo ora  gli equilibri del sistema autonomo di equazioni differenziali~\eqref{SI}.\\
Sommando le due equazioni  che definiscono tale  sistema otteniamo 
\begin{equation}
	\label{(S+I)'}(S+I)' = -\alpha I.
\end{equation}
Ora $S+I$ \`e una funzione non negativa, decrescente dunque ammette un limite.\\
Poich\`e la derivata di una funzione decrescente e limitata deve tendere a $0$ si ha $$I(t)\to 0.$$
Da queste due osservazioni si ha $$S(t) \to S_\infty. $$ 
Integrando  da $0$ a $+\infty$ in~\eqref{(S+I)'} otteniamo 
$$ \alpha \int_0^{+\infty} I(t) \di t = -\int_0^{+\infty} \tonde{ S(t) + I(t)}' = N -S_\infty.$$ 
In~\eqref{SI}, dividendo per $S$ e integrando da $0$ a $T$ otteniamo 
\begin{equation}\label{rel_finale} \log \frac{S_0}{S_\infty} = \beta \int_0^{+\infty} I(t) \di t =  \frac{\beta}{\alpha} \tonde{ N - S_\infty }= \ro \tonde{ 1 - \frac{S_\infty}{N}}.
\end{equation}
L'equazione~\eqref{rel_finale} prende il nome di \textit{relazione di dimensione finale} infatti fornisce una relazione tra il numero $\ro$ e la dimensione dell'epidemia,  ovvero il numero di membri che sono stati infetti nel corso dell'epidemia: $ N-S_\infty$.
\begin{oss}
Poich\`e il lato destro della~\eqref{rel_finale} \`e finito lo \`e anche il lato sinistro:  $S_\infty >0$ ovvero finita l'epidemia esisteranno ancora degli individui suscettibili.	
\end{oss}

\begin{prop}
La relazione finale~\eqref{rel_finale}  ha un'unica soluzione.
\proof 
Sia 
$$ g(x) =\log \frac{S_0}{x} -\ro \tonde{ 1 -\frac{x}{N}}, $$
ora 
$$\lim_{x\to 0^+} g(x)>0 \text{  e }  g(N) =\log\frac{S_0}{N} < 0, $$ 
mentre
$$g'(x)=-\frac{1}{x} + \frac{\ro}{N} < 0\quad \ses x < \frac{N}{\ro}. $$
\begin{itemize}
	\item Se $\ro\leq 1 $,   allora $N < \frac{N}{\ro}$ dunque $g$ decresce da un valore positivo in $0^+$  fino ad un valore negativo in $N$.\\
	In questo caso  $g(x)$ ammette un'unico zero   $S_\infty$ con $S_\infty< N $.
	\item Se $\ro >1$, allora $g(x)$ \`e monotona decrescente da un valore positivo in $0^+$ fino al minimo in $\frac{N}{\ro}$.\\
Ora
$$ g\tonde{ \frac{S_0}{\ro}} =\log \ro - \ro +\frac{S_0}{N}\leq \log \ro \leq  \ro + 1 <0,$$
dove la penultima disuguaglianza deriva da $\log x < x - 1$ se $x>0$.\\
Dunque $g(x)$ ha un unico zero in $S_\infty$ con $S_\infty< \frac{N}{\ro}$.
\end{itemize}
\endproof	
\end{prop}

\vspace{0.5 cm}

Possiamo inoltre  ora a descrivere le orbite delle soluzioni nel piano $(S,I)$ con
$$ \log \frac{S_0}{S(t)} =\beta \int_0^{+\infty } I(t) \di t = \frac{\beta}{\alpha} \tonde{ N- S(t) - I(t)}
$$
ottenuta dividendo per $S$ l'equazione~\eqref{SI} e integrando tra $0$ a $t$. 

\newpage

In generale, per poter studiare ed applicare il modello occorre inoltre stimare i parametri.\\ Il fattore $\beta$ \`e di difficile stima: dipende dalla malattia in esame e,  in particolar modo, da fattori sociali e comportamentali.\\
I valori di $S_0$ e $S_\infty$ possono essere ricavati tramite test sierologici (misurazione della risposta immunitaria tramite analisi del sangue): da questi valori usando la~\eqref{rel_finale} possiamo stimare $\ro$.\\
Questa stima tuttavia \`e a posteriori, cio\`e pu\`o essere ricavata solamente dopo che l'epidemia ha fatto il suo corso.\\ \\
Un altro modo per stimare $\beta$ pu\`o essere ricavato dalla seguente approssimazione
$$ I' = \tonde{ \beta N -\alpha } I,$$ 
da cui si ottiene che il numero degli infetti cresce esponenzialmente con un tasso di crescita 
$$ r =\beta N - \alpha = \alpha \tonde{ \ro -1},$$
ricavabile dall'incidenza della malattia  all'inizio dell'epidemia.\\Otteniamo cos\`i 
$$ \beta = \frac{r + \alpha } {N}.$$
Oltre ai parametri $\alpha,\beta$ due misure fondamentali nello studio dell'epidemia sono l'incidenza e la prevalenza di cui usiamo la definizione data dal comitato di esperti dell'OMS~\cite{world1959expert}.\\ \\ 
La \textit{prevalenza} indica il numero di individui malati in una determinata popolazione, senza nessuna distinzione tra nuovi e vecchi casi.La prevalenza può essere registrata in un determinato momento (prevalenza puntuale) o durante un determinato periodo di tempo (periodo di prevalenza). La ``prevalenza puntiforme"  è solitamente espressa come una frazione, il denominatore \`e il numero della popolazione.\\ \\
L'\textit{incidenza} \`e il numeri di  individui che si ammalano durante un dato periodo in una specifica popolazione. L'incidenza \`e solitamente espressa come frazione, il denomitare \`e il numero medio di persone nella popolazione.
\newpage

\begin{oss}[Immunizzazione]
Se un gruppo di infetti viene introdotto in una popolazione,  \`e possibile ridurre l'impatto dell'epidemia diminuedo $\ro$.\\
Una strategia pu\`o essere tramite l'immunizzazione, il cui scopo \`e quello di trasferire membri della popolazione della classe $S$ a quella $R$, cos\`i facendo viene ridotto il numero $S_0$ e dunque anche $\ro$.\\
Supponiamo che una frazione $p$ della popolazione sia  immunizzata: il numeri dei suscettibili passa da $S_0$ a $S_0(1-p)$.\\
Se inizialmente il numero di riproduzione di base era $\ro = \frac{\beta N}{\alpha}$, nella nuova situazione diventa $\ro''=\frac{\beta N(1-p)}{\alpha}$  da cui si ha che  
$$\frac{\beta N(1-p)}{\alpha}< 1\quad \ses \quad p> 1-\frac{\alpha}{\beta N} = 1 -\frac{1}{\ro}.
$$
\end{oss}
\newpage
\subsection{Un esempio}
Vediamo ora un esempio di applicazione di questo modello.\\
Analizziamo i dati della peste bubonica del $1665-66$ nel villaggio di Eyam in Inghilterra~\cite{raggett1982stochastic}. \\
I membri del villaggio hanno annotato giorno per giorno il numero di decessi. Per appianare alcune significative variazioni giornaliera nel tasso di mortalit\`a,  Raggett~\cite{raggett1982stochastic} ha raccolto  i dati con una cadenza di $15 \frac{1}{2}$ giorni a partire dal $18$ Giugno del $1666$ (Tabella~\ref{table::1}).
\begin{table}[!ht]
\centering
\caption{Popolazione di deceduti e rimossi.}
\label{table::1}
\begin{tabular}{l | c | c }

Periodo (1666) & Deceduti & Rimossi (alla fine del periodo) \\
\hline

19 Giugno-3/4 Luglio & 11.5 & 11.5\\
4/5 Luglio-19 Luglio & 26.5 & 38\\ 
20 Luglio-3/4 Agosto & 40.5 & 78.5\\
4/5 Agosto-19 Agosto & 41.5 & 120\\
20 Agosto-3/4 Settembre & 25 & 145\\
4/5 Settembre-19 Settembre & 11 & 156\\
20 Settembre-4/5 Ottobre & 11.5 & 167.5\\
5/6 Ottobre-20 Ottobre & 10.5 & 178\\

\end{tabular}

\end{table}

Prendendo come periodo medio di infezione  $11$ giorni, possiamo stimare il numero degli infetti. Alla fine di ogni intervallo di tempo il numero di infetti \`e dato analizzando il diario dei decessi degli $11$ giorni successivi.\\
Sfruttando la relazione 
$$ N = S(t) + I(t) + R(t),$$ 
otteniamo la Tabella~\ref{table::2}.
\begin{table}[!ht]
\centering
\caption{Numero di suscettibili ed infetti. Dati reali.}

 $S(0) = 254$ $I(0) = 7$ e $N = 261$ 	\\
\label{table::2}
\begin{tabular}{l|c|c}

Data (1666) & S & I\\
\cline{1-3}
3/4 Luglio  & 235& 14.5\\
19 Luglio  & 201 & 22\\
3/4 Agosto  & 153.5& 29\\
19  Agosto  & 121& 20\\
3/4 Settembre  & 108&  8\\
19 Settembre  & 97& 8\\
4/5 Ottobre  & Sconosciuti& Sconosciuti\\
20 Ottobre  & 83& 0\\
 
\end{tabular}
\end{table}

Ora sfruttando ~\eqref{rel_finale} otteniamo $\frac{\alpha}{\beta} \simeq 159$.
\newpage

Simulando il modello~\eqref{SI} con condizioni iniziali $S(0) = 254$ e  $I(0)=7$,   utilizzando come parametri $\alpha =2.73 $ e $\beta  = 0.0178 $ troviamo i risultati in Tabella~\ref{table::3} e i grafici di $S,I$ e $R$ come funzioni del tempo (Figura~
\ref{fig::sirsemplice}(a)).\\  \\
La Figura~\ref{fig::sirsemplice}(b) mostra i punti reali insieme al ritratto di fase per il modello~\eqref{SI}: notiamo che i  dati effettivi sono notevolmente vicini alle previsioni ottenute. \begin{table}[ht]
\centering
\caption{Numero di suscettibili ed infetti. Dati sperimentali.}
$S(0) = 254$ $I(0) = 7$ e $N = 261$ \\
\label{table::3}
\begin{tabular}{l|c|c}

Data (1666) & S & I\\
\cline{1-3}
3/4 Luglio  & 230& 15\\
19 Luglio  & 190 & 26\\
3/4 Agosto  & 147& 30\\
19  Agosto  & 115& 24\\
3/4 Settembre  & 96&  15\\
19 Settembre  & 86& 9\\
4/5 Ottobre  & 81& 4\\
20 Ottobre  & 78& 2\\
 
\end{tabular}
\end{table}

\begin{figure}[ht]
\centering
\subfloat[][]{% This file was created by matlab2tikz.
%
%The latest updates can be retrieved from
%  http://www.mathworks.com/matlabcentral/fileexchange/22022-matlab2tikz-matlab2tikz
%where you can also make suggestions and rate matlab2tikz.
%
\definecolor{mycolor1}{rgb}{0.00000,0.44700,0.74100}%
\definecolor{mycolor2}{rgb}{0.85000,0.32500,0.09800}%
\definecolor{mycolor3}{rgb}{0.92900,0.69400,0.12500}%
%
\begin{tikzpicture}

\begin{axis}[%
width=6.028in,
height=4.754in,
at={(1.011in,0.642in)},
scale only axis,
xmin=0,
xmax=5,
xlabel style={font=\color{white!15!black}},
xlabel={t},
ymin=0,
ymax=300,
axis background/.style={fill=white},
axis x line*=bottom,
axis y line*=left,
legend style={legend cell align=left, align=left, draw=white!15!black}
]
\addplot [color=mycolor1, dashed]
  table[row sep=crcr]{%
0	254\\
0.005005005005005	253.840937647464\\
0.01001001001001	253.680546129765\\
0.015015015015015	253.51881756491\\
0.02002002002002	253.355744085814\\
0.025025025025025	253.191317841079\\
0.03003003003003	253.025530995004\\
0.035035035035035	252.858375789304\\
0.04004004004004	252.689844438122\\
0.045045045045045	252.519929104995\\
0.0500500500500501	252.348621991913\\
0.0550550550550551	252.175915339314\\
0.0600600600600601	252.001801426083\\
0.0650650650650651	251.826272569559\\
0.0700700700700701	251.649321125527\\
0.0750750750750751	251.470939488222\\
0.0800800800800801	251.291120090331\\
0.0850850850850851	251.109855402987\\
0.0900900900900901	250.927137935775\\
0.0950950950950951	250.742960236727\\
0.1001001001001	250.557314892327\\
0.105105105105105	250.370194527507\\
0.11011011011011	250.181591805649\\
0.115115115115115	249.991499428584\\
0.12012012012012	249.799910136592\\
0.125125125125125	249.606816708404\\
0.13013013013013	249.412211961199\\
0.135135135135135	249.216088750607\\
0.14014014014014	249.018439970705\\
0.145145145145145	248.819258554022\\
0.15015015015015	248.618537471535\\
0.155155155155155	248.41626973267\\
0.16016016016016	248.212459056064\\
0.165165165165165	248.007107747517\\
0.17017017017017	247.800206857156\\
0.175175175175175	247.591747609933\\
0.18018018018018	247.381721405627\\
0.185185185185185	247.170119818838\\
0.19019019019019	246.956934598992\\
0.195195195195195	246.742157670341\\
0.2002002002002	246.52578113196\\
0.205205205205205	246.307797257747\\
0.21021021021021	246.088198496428\\
0.215215215215215	245.866977471551\\
0.22022022022022	245.644126981489\\
0.225225225225225	245.41963999944\\
0.23023023023023	245.193509673427\\
0.235235235235235	244.965729326295\\
0.24024024024024	244.736292455717\\
0.245245245245245	244.505192734189\\
0.25025025025025	244.27242400903\\
0.255255255255255	244.037980302386\\
0.26026026026026	243.801855811225\\
0.265265265265265	243.564044907343\\
0.27027027027027	243.324542137357\\
0.275275275275275	243.083342222711\\
0.28028028028028	242.840440059671\\
0.285285285285285	242.59583071933\\
0.29029029029029	242.349509447605\\
0.295295295295295	242.101471665236\\
0.3003003003003	241.851712967789\\
0.305305305305305	241.600229125655\\
0.31031031031031	241.347016084047\\
0.315315315315315	241.092069963006\\
0.32032032032032	240.835387057393\\
0.325325325325325	240.576963836899\\
0.33033033033033	240.316796946035\\
0.335335335335335	240.054883204139\\
0.34034034034034	239.791219605373\\
0.345345345345345	239.525803318722\\
0.35035035035035	239.258631687998\\
0.355355355355355	238.989702231836\\
0.36036036036036	238.719012643696\\
0.365365365365365	238.446560791861\\
0.37037037037037	238.172344719442\\
0.375375375375375	237.896362644371\\
0.38038038038038	237.618612959406\\
0.385385385385385	237.33909423213\\
0.39039039039039	237.05780520495\\
0.395395395395395	236.774744795097\\
0.4004004004004	236.489912094627\\
0.405405405405405	236.203306370421\\
0.41041041041041	235.914927064184\\
0.415415415415415	235.624773792445\\
0.42042042042042	235.332846346558\\
0.425425425425425	235.039144692703\\
0.43043043043043	234.743668971881\\
0.435435435435435	234.446419499922\\
0.44044044044044	234.147396767476\\
0.445445445445445	233.846601440022\\
0.45045045045045	233.544034357859\\
0.455455455455455	233.239696536114\\
0.46046046046046	232.933589164737\\
0.465465465465465	232.625713608502\\
0.47047047047047	232.31607140701\\
0.475475475475475	232.004664274683\\
0.48048048048048	231.69149410077\\
0.485485485485485	231.376562949345\\
0.49049049049049	231.059873059304\\
0.495495495495495	230.741426844369\\
0.500500500500501	230.421226893088\\
0.505505505505506	230.09927596883\\
0.510510510510511	229.775577009791\\
0.515515515515516	229.450133128991\\
0.520520520520521	229.122947614275\\
0.525525525525526	228.794023928312\\
0.530530530530531	228.463365708595\\
0.535535535535536	228.130976767442\\
0.540540540540541	227.796861091996\\
0.545545545545546	227.461022844225\\
0.550550550550551	227.123466360919\\
0.555555555555556	226.784196153695\\
0.560560560560561	226.443216908994\\
0.565565565565566	226.10053348808\\
0.570570570570571	225.756150927044\\
0.575575575575576	225.4100744368\\
0.580580580580581	225.062309403086\\
0.585585585585586	224.712861386466\\
0.590590590590591	224.361736122328\\
0.595595595595596	224.008939520883\\
0.600600600600601	223.65447766717\\
0.605605605605606	223.298356821048\\
0.610610610610611	222.940583417205\\
0.615615615615616	222.58116406515\\
0.620620620620621	222.220105549219\\
0.625625625625626	221.85741482857\\
0.630630630630631	221.493099037188\\
0.635635635635636	221.127165483882\\
0.640640640640641	220.759621652284\\
0.645645645645646	220.390475200852\\
0.650650650650651	220.019733962868\\
0.655655655655656	219.647405840782\\
0.660660660660661	219.273487420567\\
0.665665665665666	218.897982213279\\
0.670670670670671	218.520902989138\\
0.675675675675676	218.142262632548\\
0.680680680680681	217.762074142097\\
0.685685685685686	217.380350630557\\
0.690690690690691	216.997105324881\\
0.695695695695696	216.612351566211\\
0.700700700700701	216.226102809867\\
0.705705705705706	215.838372625357\\
0.710710710710711	215.44917469637\\
0.715715715715716	215.058522820782\\
0.720720720720721	214.666430910649\\
0.725725725725726	214.272912992213\\
0.730730730730731	213.877983205899\\
0.735735735735736	213.481655806316\\
0.740740740740741	213.083945162257\\
0.745745745745746	212.684865756699\\
0.750750750750751	212.284432186801\\
0.755755755755756	211.882659163907\\
0.760760760760761	211.479561513546\\
0.765765765765766	211.075154175428\\
0.770770770770771	210.66945220345\\
0.775775775775776	210.262470765689\\
0.780780780780781	209.854225144409\\
0.785785785785786	209.444730736056\\
0.790790790790791	209.034003051261\\
0.795795795795796	208.622057714837\\
0.800800800800801	208.208910465782\\
0.805805805805806	207.794577157278\\
0.810810810810811	207.379073756689\\
0.815815815815816	206.962416345566\\
0.820820820820821	206.54462111964\\
0.825825825825826	206.125704388828\\
0.830830830830831	205.70568257723\\
0.835835835835836	205.284572223131\\
0.840840840840841	204.862389978998\\
0.845845845845846	204.439152611482\\
0.850850850850851	204.01487700142\\
0.855855855855856	203.589580143829\\
0.860860860860861	203.163279147913\\
0.865865865865866	202.735991237058\\
0.870870870870871	202.307733748834\\
0.875875875875876	201.878524134996\\
0.880880880880881	201.448379961481\\
0.885885885885886	201.01731890841\\
0.890890890890891	200.58535877009\\
0.895895895895896	200.152517455008\\
0.900900900900901	199.718812985838\\
0.905905905905906	199.284263499437\\
0.910910910910911	198.848887246844\\
0.915915915915916	198.412702593283\\
0.920920920920921	197.975728018163\\
0.925925925925926	197.537982115075\\
0.930930930930931	197.099483591794\\
0.935935935935936	196.660251270279\\
0.940940940940941	196.220304086673\\
0.945945945945946	195.779661091303\\
0.950950950950951	195.338341448678\\
0.955955955955956	194.896364437493\\
0.960960960960961	194.453749450626\\
0.965965965965966	194.010515995137\\
0.970970970970971	193.566683692274\\
0.975975975975976	193.122272277463\\
0.980980980980981	192.677301600319\\
0.985985985985986	192.231791624638\\
0.990990990990991	191.7857624284\\
0.995995995995996	191.339234203769\\
1.001001001001	190.892227257094\\
1.00600600600601	190.444762008904\\
1.01101101101101	189.996858993917\\
1.01601601601602	189.54853886103\\
1.02102102102102	189.099822373327\\
1.02602602602603	188.650730408074\\
1.03103103103103	188.201283956722\\
1.03603603603604	187.751504124905\\
1.04104104104104	187.30141213244\\
1.04604604604605	186.851029313328\\
1.05105105105105	186.400377115757\\
1.05605605605606	185.949477102093\\
1.06106106106106	185.498350948891\\
1.06606606606607	185.047020446887\\
1.07107107107107	184.595507501001\\
1.07607607607608	184.143834130337\\
1.08108108108108	183.692022468183\\
1.08608608608609	183.24009476201\\
1.09109109109109	182.788073373475\\
1.0960960960961	182.335980778415\\
1.1011011011011	181.883839566854\\
1.10610610610611	181.431672442999\\
1.11111111111111	180.97950222524\\
1.11611611611612	180.52735184615\\
1.12112112112112	180.075244352488\\
1.12612612612613	179.623202905195\\
1.13113113113113	179.171250779397\\
1.13613613613614	178.719411364403\\
1.14114114114114	178.267708163705\\
1.14614614614615	177.81616479498\\
1.15115115115115	177.364804990089\\
1.15615615615616	176.913651688434\\
1.16116116116116	176.462698527341\\
1.16616616616617	176.011953449083\\
1.17117117117117	175.561438548038\\
1.17617617617618	175.111175727597\\
1.18118118118118	174.661186700154\\
1.18618618618619	174.211492987116\\
1.19119119119119	173.762115918894\\
1.1961961961962	173.31307663491\\
1.2012012012012	172.864396083591\\
1.20620620620621	172.416095022375\\
1.21121121121121	171.968194017706\\
1.21621621621622	171.520713445037\\
1.22122122122122	171.073673488829\\
1.22622622622623	170.62709414255\\
1.23123123123123	170.180995208676\\
1.23623623623624	169.735396298694\\
1.24124124124124	169.290316833095\\
1.24624624624625	168.845776041381\\
1.25125125125125	168.40179296206\\
1.25625625625626	167.958386442649\\
1.26126126126126	167.515575139673\\
1.26626626626627	167.073377518665\\
1.27127127127127	166.631811854166\\
1.27627627627628	166.190896229724\\
1.28128128128128	165.750648537897\\
1.28628628628629	165.311086480249\\
1.29129129129129	164.872227567354\\
1.2962962962963	164.434089118792\\
1.3013013013013	163.996688263153\\
1.30630630630631	163.560041938034\\
1.31131131131131	163.124166890039\\
1.31631631631632	162.689079674782\\
1.32132132132132	162.254796656884\\
1.32632632632633	161.821334009973\\
1.33133133133133	161.388707716688\\
1.33633633633634	160.956933568673\\
1.34134134134134	160.526027166581\\
1.34634634634635	160.096003920073\\
1.35135135135135	159.66687904782\\
1.35635635635636	159.238667577497\\
1.36136136136136	158.81138434579\\
1.36636636636637	158.385043998393\\
1.37137137137137	157.959660990007\\
1.37637637637638	157.535249584341\\
1.38138138138138	157.111823854112\\
1.38638638638639	156.689397681045\\
1.39139139139139	156.267984755875\\
1.3963963963964	155.847598578342\\
1.4014014014014	155.428252457196\\
1.40640640640641	155.009959510194\\
1.41141141141141	154.592732664102\\
1.41641641641642	154.176584654693\\
1.42142142142142	153.761528026748\\
1.42642642642643	153.347575134057\\
1.43143143143143	152.934738139418\\
1.43643643643644	152.523029014636\\
1.44144144144144	152.112459540525\\
1.44644644644645	151.703041306906\\
1.45145145145145	151.294785712608\\
1.45645645645646	150.88770396547\\
1.46146146146146	150.481807082337\\
1.46646646646647	150.077105889062\\
1.47147147147147	149.673611020508\\
1.47647647647648	149.271332920543\\
1.48148148148148	148.870281842047\\
1.48648648648649	148.470467846903\\
1.49149149149149	148.071900806007\\
1.4964964964965	147.674590399259\\
1.5015015015015	147.27854611557\\
1.50650650650651	146.883777252858\\
1.51151151151151	146.490292918047\\
1.51651651651652	146.098102027073\\
1.52152152152152	145.707213304876\\
1.52652652652653	145.317635285407\\
1.53153153153153	144.929376311624\\
1.53653653653654	144.542444535492\\
1.54154154154154	144.156847917986\\
1.54654654654655	143.772594229087\\
1.55155155155155	143.389691047785\\
1.55655655655656	143.008145762079\\
1.56156156156156	142.627965568974\\
1.56656656656657	142.249157474484\\
1.57157157157157	141.871728293632\\
1.57657657657658	141.495684650447\\
1.58158158158158	141.121032977967\\
1.58658658658659	140.747779518239\\
1.59159159159159	140.375930322317\\
1.5965965965966	140.005491250262\\
1.6016016016016	139.636467971146\\
1.60660660660661	139.268865963045\\
1.61161161161161	138.902690513047\\
1.61661661661662	138.537946717245\\
1.62162162162162	138.174639480742\\
1.62662662662663	137.812773517648\\
1.63163163163163	137.452353351081\\
1.63663663663664	137.093383313168\\
1.64164164164164	136.735867545042\\
1.64664664664665	136.379809996846\\
1.65165165165165	136.025214427731\\
1.65665665665666	135.672085095867\\
1.66166166166166	135.320435891758\\
1.66666666666667	134.970272585802\\
1.67167167167167	134.621596990895\\
1.67667667667668	134.274410831652\\
1.68168168168168	133.928715744413\\
1.68668668668669	133.584513277236\\
1.69169169169169	133.241804889901\\
1.6966966966967	132.900591953912\\
1.7017017017017	132.560875752489\\
1.70670670670671	132.222657480579\\
1.71171171171171	131.885938244845\\
1.71671671671672	131.550719063674\\
1.72172172172172	131.217000867174\\
1.72672672672673	130.884784497174\\
1.73173173173173	130.554070707224\\
1.73673673673674	130.224860162596\\
1.74174174174174	129.897153440281\\
1.74674674674675	129.570951028994\\
1.75175175175175	129.246253329169\\
1.75675675675676	128.923060652962\\
1.76176176176176	128.601373224251\\
1.76676676676677	128.281191178635\\
1.77177177177177	127.962514563433\\
1.77677677677678	127.645343337686\\
1.78178178178178	127.329677372156\\
1.78678678678679	127.015516449326\\
1.79179179179179	126.702860263402\\
1.7967967967968	126.391708420309\\
1.8018018018018	126.082060437693\\
1.80680680680681	125.773915744923\\
1.81181181181181	125.467273683088\\
1.81681681681682	125.162133505\\
1.82182182182182	124.858494375189\\
1.82682682682683	124.556355369908\\
1.83183183183183	124.255715477133\\
1.83683683683684	123.956573596557\\
1.84184184184184	123.658928539599\\
1.84684684684685	123.362779029394\\
1.85185185185185	123.068123700804\\
1.85685685685686	122.774961100407\\
1.86186186186186	122.483289686505\\
1.86686686686687	122.193107829121\\
1.87187187187187	121.904413809999\\
1.87687687687688	121.617205822603\\
1.88188188188188	121.33148197212\\
1.88688688688689	121.047240275457\\
1.89189189189189	120.764478661243\\
1.8968968968969	120.483194969828\\
1.9019019019019	120.203386953283\\
1.90690690690691	119.925052275399\\
1.91191191191191	119.648188511691\\
1.91691691691692	119.372793149393\\
1.92192192192192	119.09886358746\\
1.92692692692693	118.826397136571\\
1.93193193193193	118.555391019123\\
1.93693693693694	118.285842369235\\
1.94194194194194	118.017748232749\\
1.94694694694695	117.751105567226\\
1.95195195195195	117.48591124195\\
1.95695695695696	117.222162037923\\
1.96196196196196	116.959854647873\\
1.96696696696697	116.698985676246\\
1.97197197197197	116.439551639209\\
1.97697697697698	116.181548964652\\
1.98198198198198	115.924973992185\\
1.98698698698699	115.669822973139\\
1.99199199199199	115.416092070568\\
1.996996996997	115.163777359244\\
2.002002002002	114.912874825664\\
2.00700700700701	114.663380368043\\
2.01201201201201	114.41528979632\\
2.01701701701702	114.168598832152\\
2.02202202202202	113.92330310892\\
2.02702702702703	113.679398171725\\
2.03203203203203	113.436879477389\\
2.03703703703704	113.195742394456\\
2.04204204204204	112.95598220319\\
2.04704704704705	112.717594095577\\
2.05205205205205	112.480573175325\\
2.05705705705706	112.244914457862\\
2.06206206206206	112.010612870338\\
2.06706706706707	111.777663251623\\
2.07207207207207	111.546060352309\\
2.07707707707708	111.315798834709\\
2.08208208208208	111.086873272858\\
2.08708708708709	110.859278152512\\
2.09209209209209	110.633007871146\\
2.0970970970971	110.40805673796\\
2.1021021021021	110.184418973872\\
2.10710710710711	109.962088711522\\
2.11211211211211	109.741059995273\\
2.11711711711712	109.521326781207\\
2.12212212212212	109.302882937129\\
2.12712712712713	109.085722242562\\
2.13213213213213	108.869838388754\\
2.13713713713714	108.655224978673\\
2.14214214214214	108.441875527006\\
2.14714714714715	108.229783460165\\
2.15215215215215	108.018942116281\\
2.15715715715716	107.80934667716\\
2.16216216216216	107.601008448571\\
2.16716716716717	107.393925688062\\
2.17217217217217	107.188092369329\\
2.17717717717718	106.983502481988\\
2.18218218218218	106.780150031569\\
2.18718718718719	106.578029039523\\
2.19219219219219	106.377133543215\\
2.1971971971972	106.177457595931\\
2.2022022022022	105.978995266873\\
2.20720720720721	105.781740641159\\
2.21221221221221	105.585687819828\\
2.21721721721722	105.390830919834\\
2.22222222222222	105.197164074049\\
2.22722722722723	105.004681431262\\
2.23223223223223	104.813377156181\\
2.23723723723724	104.623245429431\\
2.24224224224224	104.434280447554\\
2.24724724724725	104.24647642301\\
2.25225225225225	104.059827584175\\
2.25725725725726	103.874328175346\\
2.26226226226226	103.689972456734\\
2.26726726726727	103.50675470447\\
2.27227227227227	103.3246692106\\
2.27727727727728	103.143710283091\\
2.28228228228228	102.963872245824\\
2.28728728728729	102.7851494386\\
2.29229229229229	102.607536217136\\
2.2972972972973	102.431026953067\\
2.3023023023023	102.255616033947\\
2.30730730730731	102.081297863245\\
2.31231231231231	101.908066860349\\
2.31731731731732	101.735917460566\\
2.32232232232232	101.564844115116\\
2.32732732732733	101.394841291141\\
2.33233233233233	101.225903471699\\
2.33733733733734	101.058025155766\\
2.34234234234234	100.891200858233\\
2.34734734734735	100.725425109913\\
2.35235235235235	100.560692457532\\
2.35735735735736	100.396997463737\\
2.36236236236236	100.23433470709\\
2.36736736736737	100.072698782073\\
2.37237237237237	99.9120842990831\\
2.37737737737738	99.7524858844365\\
2.38238238238238	99.5938981803665\\
2.38738738738739	99.4363158450239\\
2.39239239239239	99.279733552477\\
2.3973973973974	99.1241459927118\\
2.4024024024024	98.9695478716316\\
2.40740740740741	98.8159339110574\\
2.41241241241241	98.6632988487278\\
2.41741741741742	98.5116374382987\\
2.42242242242242	98.3609444493437\\
2.42742742742743	98.211214667354\\
2.43243243243243	98.0624428937382\\
2.43743743743744	97.9146239458225\\
2.44244244244244	97.7677526568505\\
2.44744744744745	97.6218238759835\\
2.45245245245245	97.4768324683004\\
2.45745745745746	97.3327733147974\\
2.46246246246246	97.1896413123883\\
2.46746746746747	97.0474313739047\\
2.47247247247247	96.9061384280953\\
2.47747747747748	96.7657574196267\\
2.48248248248248	96.6262833090829\\
2.48748748748749	96.4877110729654\\
2.49249249249249	96.3500357036932\\
2.4974974974975	96.213252209603\\
2.5025025025025	96.0773556149489\\
2.50750750750751	95.9423409599026\\
2.51251251251251	95.8082033005533\\
2.51751751751752	95.6749377089077\\
2.52252252252252	95.5425392728902\\
2.52752752752753	95.4110030963425\\
2.53253253253253	95.280324299024\\
2.53753753753754	95.1504980166116\\
2.54254254254254	95.0215194006996\\
2.54754754754755	94.8933836188002\\
2.55255255255255	94.7660858543427\\
2.55755755755756	94.6396213066742\\
2.56256256256256	94.5139851910593\\
2.56756756756757	94.38917273868\\
2.57257257257257	94.265179196636\\
2.57757757757758	94.1419998279444\\
2.58258258258258	94.0196299115401\\
2.58758758758759	93.8980647422751\\
2.59259259259259	93.7772996309194\\
2.5975975975976	93.6573299041602\\
2.6026026026026	93.5381509046023\\
2.60760760760761	93.4197579907682\\
2.61261261261261	93.3021465370977\\
2.61761761761762	93.1853119339484\\
2.62262262262262	93.0692495875952\\
2.62762762762763	92.9539549202306\\
2.63263263263263	92.8394233699648\\
2.63763763763764	92.7256503908252\\
2.64264264264264	92.6126314527571\\
2.64764764764765	92.5003620416231\\
2.65265265265265	92.3888376592034\\
2.65765765765766	92.2780533143526\\
2.66266266266266	92.1680015025977\\
2.66766766766767	92.0586773042921\\
2.67267267267267	91.9500763288572\\
2.67767767767768	91.8421942043318\\
2.68268268268268	91.7350265773725\\
2.68768768768769	91.6285691132534\\
2.69269269269269	91.5228174958661\\
2.6976976976977	91.4177674277201\\
2.7027027027027	91.3134146299422\\
2.70770770770771	91.2097548422768\\
2.71271271271271	91.1067838230861\\
2.71771771771772	91.0044973493497\\
2.72272272272272	90.902891216665\\
2.72772772772773	90.8019612392467\\
2.73273273273273	90.7017032499274\\
2.73773773773774	90.6021131001571\\
2.74274274274274	90.5031866600034\\
2.74774774774775	90.4049198181517\\
2.75275275275275	90.3073084819047\\
2.75775775775776	90.2103485771829\\
2.76276276276276	90.1140360485243\\
2.76776776776777	90.0183668590845\\
2.77277277277277	89.9233369906369\\
2.77777777777778	89.8289424435721\\
2.78278278278278	89.7351792368987\\
2.78778778778779	89.6420434082425\\
2.79279279279279	89.5495310138473\\
2.7977977977978	89.4576381285742\\
2.8028028028028	89.366360845902\\
2.80780780780781	89.2756952779271\\
2.81281281281281	89.1856375553635\\
2.81781781781782	89.0961838275427\\
2.82282282282282	89.007330262414\\
2.82782782782783	88.9190730465441\\
2.83283283283283	88.8314083851174\\
2.83783783783784	88.7443325019357\\
2.84284284284284	88.6578416394188\\
2.84784784784785	88.5719320586037\\
2.85285285285285	88.4866000391451\\
2.85785785785786	88.4018418793154\\
2.86286286286286	88.3176538960046\\
2.86786786786787	88.2340324247202\\
2.87287287287287	88.1509738195872\\
2.87787787787788	88.0684744533484\\
2.88288288288288	87.9865307173642\\
2.88788788788789	87.9051390216124\\
2.89289289289289	87.8242957946885\\
2.8978978978979	87.7439974838056\\
2.9029029029029	87.6642405547945\\
2.90790790790791	87.5850214921033\\
2.91291291291291	87.5063367987981\\
2.91791791791792	87.4281829965622\\
2.92292292292292	87.3505566256968\\
2.92792792792793	87.2734542451204\\
2.93293293293293	87.1968724323694\\
2.93793793793794	87.1208077835977\\
2.94294294294294	87.0452569135766\\
2.94794794794795	86.9702164556953\\
2.95295295295295	86.8956830619603\\
2.95795795795796	86.8216534029959\\
2.96296296296296	86.748124168044\\
2.96796796796797	86.6750920649639\\
2.97297297297297	86.6025538202327\\
2.97797797797798	86.530506178945\\
2.98298298298298	86.458945904813\\
2.98798798798799	86.3878697801666\\
2.99299299299299	86.3172746059531\\
2.997997997998	86.2471572017374\\
3.003003003003	86.1775144057023\\
3.00800800800801	86.1083430746479\\
3.01301301301301	86.039640083992\\
3.01801801801802	85.9714023277699\\
3.02302302302302	85.9036267186346\\
3.02802802802803	85.8363101878568\\
3.03303303303303	85.7694496853245\\
3.03803803803804	85.7030421795435\\
3.04304304304304	85.6370846576371\\
3.04804804804805	85.5715741253464\\
3.05305305305305	85.5065076070298\\
3.05805805805806	85.4418821456635\\
3.06306306306306	85.3776948028413\\
3.06806806806807	85.3139426587744\\
3.07307307307307	85.2506228122917\\
3.07807807807808	85.1877323808398\\
3.08308308308308	85.1252685004828\\
3.08808808808809	85.0632283259024\\
3.09309309309309	85.0016090303979\\
3.0980980980981	84.9404078058862\\
3.1031031031031	84.8796218629018\\
3.10810810810811	84.8192484305968\\
3.11311311311311	84.7592847567408\\
3.11811811811812	84.6997281077211\\
3.12312312312312	84.6405757685426\\
3.12812812812813	84.5818250428278\\
3.13313313313313	84.5234732528167\\
3.13813813813814	84.465517739367\\
3.14314314314314	84.407955861954\\
3.14814814814815	84.3507849986704\\
3.15315315315315	84.2940025462267\\
3.15815815815816	84.237605508701\\
3.16316316316316	84.1815897129953\\
3.16816816816817	84.12595261461\\
3.17317317317317	84.0706918667145\\
3.17817817817818	84.0158051320201\\
3.18318318318318	83.96129008278\\
3.18818818818819	83.9071444007894\\
3.19319319319319	83.8533657773851\\
3.1981981981982	83.7999519134463\\
3.2032032032032	83.7469005193936\\
3.20820820820821	83.6942093151899\\
3.21321321321321	83.6418760303398\\
3.21821821821822	83.58989840389\\
3.22322322322322	83.5382741844288\\
3.22822822822823	83.4870011300867\\
3.23323323323323	83.4360770085359\\
3.23823823823824	83.3854995969908\\
3.24324324324324	83.3352666822075\\
3.24824824824825	83.2853760604839\\
3.25325325325325	83.2358255376601\\
3.25825825825826	83.1866129291179\\
3.26326326326326	83.1377360597811\\
3.26826826826827	83.0891927641154\\
3.27327327327327	83.0409808861283\\
3.27827827827828	82.9930982793694\\
3.28328328328328	82.9455428069302\\
3.28828828828829	82.8983123414439\\
3.29329329329329	82.8514047650858\\
3.2982982982983	82.804817969573\\
3.3033033033033	82.7585498561646\\
3.30830830830831	82.7125983356617\\
3.31331331331331	82.666961328407\\
3.31831831831832	82.6216367642854\\
3.32332332332332	82.5766225827236\\
3.32832832832833	82.5319167326903\\
3.33333333333333	82.4875171726959\\
3.33833833833834	82.4434218707929\\
3.34334334334334	82.3996288045757\\
3.34834834834835	82.3561359611805\\
3.35335335335335	82.3129413372855\\
3.35835835835836	82.2700429391109\\
3.36336336336336	82.2274387824185\\
3.36836836836837	82.1851268925124\\
3.37337337337337	82.1431053042383\\
3.37837837837838	82.101372061984\\
3.38338338338338	82.0599252196792\\
3.38838838838839	82.0187628407953\\
3.39339339339339	81.977882998346\\
3.3983983983984	81.9372837748865\\
3.4034034034034	81.8969632625141\\
3.40840840840841	81.8569195628681\\
3.41341341341341	81.8171507871295\\
3.41841841841842	81.7776550560215\\
3.42342342342342	81.7384304998088\\
3.42842842842843	81.6994752582985\\
3.43343343343343	81.6607874808392\\
3.43843843843844	81.6223653263216\\
3.44344344344344	81.5842069631784\\
3.44844844844845	81.5463105693839\\
3.45345345345345	81.5086743324547\\
3.45845845845846	81.4712964494489\\
3.46346346346346	81.4341751269669\\
3.46846846846847	81.3973085811509\\
3.47347347347347	81.3606950376847\\
3.47847847847848	81.3243327317945\\
3.48348348348348	81.2882199082481\\
3.48848848848849	81.2523548213552\\
3.49349349349349	81.2167357349676\\
3.4984984984985	81.1813609224789\\
3.5035035035035	81.1462286668246\\
3.50850850850851	81.1113372604821\\
3.51351351351351	81.0766850054707\\
3.51851851851852	81.0422702133518\\
3.52352352352352	81.0080912052285\\
3.52852852852853	80.9741463117459\\
3.53353353353353	80.9404338730909\\
3.53853853853854	80.9069522389924\\
3.54354354354354	80.8736997687213\\
3.54854854854855	80.8406748310904\\
3.55355355355355	80.8078758044541\\
3.55855855855856	80.775301076709\\
3.56356356356356	80.7429490452937\\
3.56856856856857	80.7108181171885\\
3.57357357357357	80.6789067089157\\
3.57857857857858	80.6472132465394\\
3.58358358358358	80.6157361656657\\
3.58858858858859	80.5844739114427\\
3.59359359359359	80.5534249385603\\
3.5985985985986	80.5225877112503\\
3.6036036036036	80.4919607032865\\
3.60860860860861	80.4615423979845\\
3.61361361361361	80.4313312882019\\
3.61861861861862	80.4013258763381\\
3.62362362362362	80.3715246743347\\
3.62862862862863	80.3419262036748\\
3.63363363363363	80.3125289953836\\
3.63863863863864	80.2833315900284\\
3.64364364364364	80.2543325377181\\
3.64864864864865	80.2255303981037\\
3.65365365365365	80.1969237403781\\
3.65865865865866	80.168511162213\\
3.66366366366366	80.1402914116174\\
3.66866866866867	80.1122633012586\\
3.67367367367367	80.0844256481456\\
3.67867867867868	80.0567772738041\\
3.68368368368368	80.0293170042763\\
3.68868868868869	80.0020436701214\\
3.69369369369369	79.9749561064152\\
3.6986986986987	79.9480531527501\\
3.7037037037037	79.9213336532355\\
3.70870870870871	79.8947964564972\\
3.71371371371371	79.8684404156779\\
3.71871871871872	79.8422643884369\\
3.72372372372372	79.8162672369505\\
3.72872872872873	79.7904478279113\\
3.73373373373373	79.764805032529\\
3.73873873873874	79.7393377265297\\
3.74374374374374	79.7140447901565\\
3.74874874874875	79.6889251081689\\
3.75375375375375	79.6639775698435\\
3.75875875875876	79.6392010689733\\
3.76376376376376	79.6145945038681\\
3.76876876876877	79.5901567773546\\
3.77377377377377	79.5658867967758\\
3.77877877877878	79.541783473992\\
3.78378378378378	79.5178457253796\\
3.78878878878879	79.4940724718323\\
3.79379379379379	79.47046263876\\
3.7987987987988	79.4470151560896\\
3.8038038038038	79.4237289582648\\
3.80880880880881	79.4006029842458\\
3.81381381381381	79.3776361775095\\
3.81881881881882	79.3548274860497\\
3.82382382382382	79.3321758623769\\
3.82882882882883	79.3096802635182\\
3.83383383383383	79.2873396510174\\
3.83883883883884	79.2651529909352\\
3.84384384384384	79.2431192538488\\
3.84884884884885	79.2212374148523\\
3.85385385385385	79.1995064535564\\
3.85885885885886	79.1779253540886\\
3.86386386386386	79.1564931050929\\
3.86886886886887	79.1352086997304\\
3.87387387387387	79.1140711356786\\
3.87887887887888	79.0930794151319\\
3.88388388388388	79.0722325448012\\
3.88888888888889	79.0515295359143\\
3.89389389389389	79.0309694042158\\
3.8988988988989	79.0105511699667\\
3.9039039039039	78.990273857945\\
3.90890890890891	78.9701364974454\\
3.91391391391391	78.9501381222791\\
3.91891891891892	78.9302777707742\\
3.92392392392392	78.9105544857755\\
3.92892892892893	78.8909673146445\\
3.93393393393393	78.8715153092594\\
3.93893893893894	78.8521975260151\\
3.94394394394394	78.8330130258232\\
3.94894894894895	78.8139608741122\\
3.95395395395395	78.795040140827\\
3.95895895895896	78.7762499004295\\
3.96396396396396	78.7575892318982\\
3.96896896896897	78.7390572187283\\
3.97397397397397	78.7206529489318\\
3.97897897897898	78.7023755150373\\
3.98398398398398	78.6842240140901\\
3.98898898898899	78.6661975476525\\
3.99399399399399	78.6482952218032\\
3.998998998999	78.6305161471377\\
4.004004004004	78.6128594387682\\
4.00900900900901	78.5953242163238\\
4.01401401401401	78.5779096039502\\
4.01901901901902	78.5606147303096\\
4.02402402402402	78.5434387285812\\
4.02902902902903	78.5263807364609\\
4.03403403403403	78.5094398961612\\
4.03903903903904	78.4926153544114\\
4.04404404404404	78.4759062624574\\
4.04904904904905	78.4593117760619\\
4.05405405405405	78.4428310555044\\
4.05905905905906	78.426463265581\\
4.06406406406406	78.4102075756045\\
4.06906906906907	78.3940631594045\\
4.07407407407407	78.3780291953273\\
4.07907907907908	78.3621048662359\\
4.08408408408408	78.34628935951\\
4.08908908908909	78.330581867046\\
4.09409409409409	78.3149815852571\\
4.0990990990991	78.2994877150732\\
4.1041041041041	78.2840994619407\\
4.10910910910911	78.2688160358232\\
4.11411411411411	78.2536366512004\\
4.11911911911912	78.2385605270693\\
4.12412412412412	78.2235868869432\\
4.12912912912913	78.2087149588523\\
4.13413413413413	78.1939439753435\\
4.13913913913914	78.1792731734804\\
4.14414414414414	78.1647017948433\\
4.14914914914915	78.1502290855293\\
4.15415415415415	78.1358542961521\\
4.15915915915916	78.1215768198477\\
4.16416416416416	78.1073962785595\\
4.16916916916917	78.0933120863975\\
4.17417417417417	78.0793236519808\\
4.17917917917918	78.0654303860962\\
4.18418418418418	78.051631701698\\
4.18918918918919	78.0379270139086\\
4.19419419419419	78.0243157400178\\
4.1991991991992	78.0107972994834\\
4.2042042042042	77.9973711139306\\
4.20920920920921	77.9840366071527\\
4.21421421421421	77.9707932051104\\
4.21921921921922	77.9576403359324\\
4.22422422422422	77.9445774299148\\
4.22922922922923	77.9316039195217\\
4.23423423423423	77.9187192393849\\
4.23923923923924	77.9059228263037\\
4.24424424424424	77.8932141192455\\
4.24924924924925	77.880592559345\\
4.25425425425425	77.8680575899049\\
4.25925925925926	77.8556086563955\\
4.26426426426426	77.843245206455\\
4.26926926926927	77.8309666898892\\
4.27427427427427	77.8187725586715\\
4.27927927927928	77.8066622669432\\
4.28428428428428	77.7946352710134\\
4.28928928928929	77.7826910293587\\
4.29429429429429	77.7708290026235\\
4.2992992992993	77.7590486536199\\
4.3043043043043	77.747349447328\\
4.30930930930931	77.7357308508952\\
4.31431431431431	77.7241923336369\\
4.31931931931932	77.7127333670362\\
4.32432432432432	77.7013534247437\\
4.32932932932933	77.6900519825781\\
4.33433433433433	77.6788285185255\\
4.33933933933934	77.66768251274\\
4.34434434434434	77.656613447543\\
4.34934934934935	77.6456208074242\\
4.35435435435435	77.6347040790405\\
4.35935935935936	77.6238627512168\\
4.36436436436436	77.6130963149458\\
4.36936936936937	77.6024042633876\\
4.37437437437437	77.5917860918704\\
4.37937937937938	77.5812412978898\\
4.38438438438438	77.5707693811094\\
4.38938938938939	77.5603698433603\\
4.39439439439439	77.5500421886415\\
4.3993993993994	77.5397859231195\\
4.4044044044044	77.5296005551288\\
4.40940940940941	77.5194855951715\\
4.41441441441441	77.5094405559174\\
4.41941941941942	77.4994649522039\\
4.42442442442442	77.4895583010365\\
4.42942942942943	77.479720121588\\
4.43443443443443	77.4699499351992\\
4.43943943943944	77.4602472653785\\
4.44444444444444	77.4506116378022\\
4.44944944944945	77.441042580314\\
4.45445445445445	77.4315396229256\\
4.45945945945946	77.4221022978164\\
4.46446446446446	77.4127301393334\\
4.46946946946947	77.4034226839914\\
4.47447447447447	77.3941794704728\\
4.47947947947948	77.3850000396281\\
4.48448448448448	77.375883934475\\
4.48948948948949	77.3668307001993\\
4.49449449449449	77.3578398841545\\
4.4994994994995	77.3489110358616\\
4.5045045045045	77.3400437070095\\
4.50950950950951	77.3312374514547\\
4.51451451451451	77.3224918252217\\
4.51951951951952	77.3138063865024\\
4.52452452452452	77.3051806956566\\
4.52952952952953	77.2966143152118\\
4.53453453453453	77.2881068098632\\
4.53953953953954	77.2796577464736\\
4.54454454454454	77.2712666940739\\
4.54954954954955	77.2629332238623\\
4.55455455455455	77.254656909205\\
4.55955955955956	77.2464373256358\\
4.56456456456456	77.2382740508562\\
4.56956956956957	77.2301666647355\\
4.57457457457457	77.2221147493108\\
4.57957957957958	77.2141178887868\\
4.58458458458458	77.2061756695359\\
4.58958958958959	77.1982876800983\\
4.59459459459459	77.1904535111819\\
4.5995995995996	77.1826727556623\\
4.6046046046046	77.174945008583\\
4.60960960960961	77.1672698671548\\
4.61461461461461	77.1596469307568\\
4.61961961961962	77.1520758009353\\
4.62462462462462	77.1445560814047\\
4.62962962962963	77.1370873780469\\
4.63463463463463	77.1296692989116\\
4.63963963963964	77.1223014542163\\
4.64464464464464	77.1149834563461\\
4.64964964964965	77.1077149198538\\
4.65465465465465	77.1004954614601\\
4.65965965965966	77.0933249912197\\
4.66466466466466	77.0862035882881\\
4.66966966966967	77.079130935327\\
4.67467467467467	77.0721067130121\\
4.67967967967968	77.0651306035333\\
4.68468468468468	77.0582022905947\\
4.68968968968969	77.0513214594144\\
4.69469469469469	77.044487796725\\
4.6996996996997	77.0377009907729\\
4.7047047047047	77.0309607313189\\
4.70970970970971	77.024266709638\\
4.71471471471471	77.0176186185193\\
4.71971971971972	77.011016152266\\
4.72472472472472	77.0044590066955\\
4.72972972972973	76.9979468791395\\
4.73473473473473	76.9914794684437\\
4.73973973973974	76.9850564749682\\
4.74474474474474	76.978677600587\\
4.74974974974975	76.9723425486884\\
4.75475475475475	76.9660510241749\\
4.75975975975976	76.9598027334632\\
4.76476476476476	76.953597384484\\
4.76976976976977	76.9474346866824\\
4.77477477477477	76.9413143510175\\
4.77977977977978	76.9352360899626\\
4.78478478478478	76.9291996175053\\
4.78978978978979	76.9232046491472\\
4.79479479479479	76.9172509019041\\
4.7997997997998	76.9113380943061\\
4.8048048048048	76.9054659463973\\
4.80980980980981	76.8996341797362\\
4.81481481481481	76.8938425173953\\
4.81981981981982	76.8880906839612\\
4.82482482482482	76.8823784055349\\
4.82982982982983	76.8767054097313\\
4.83483483483483	76.8710714256798\\
4.83983983983984	76.8654761840237\\
4.84484484484484	76.8599194169205\\
4.84984984984985	76.8544008580421\\
4.85485485485485	76.8489202425743\\
4.85985985985986	76.8434773072172\\
4.86486486486486	76.838071790185\\
4.86986986986987	76.8327034312062\\
4.87487487487487	76.8273719715234\\
4.87987987987988	76.8220771538933\\
4.88488488488488	76.8168187225869\\
4.88988988988989	76.8115964233893\\
4.89489489489489	76.8064100035998\\
4.8998998998999	76.8012592120318\\
4.9049049049049	76.796143799013\\
4.90990990990991	76.7910635163851\\
4.91491491491491	76.7860181175042\\
4.91991991991992	76.7810073572403\\
4.92492492492492	76.7760309919779\\
4.92992992992993	76.7710887796153\\
4.93493493493493	76.7661804795653\\
4.93993993993994	76.7613058527546\\
4.94494494494494	76.7564646616243\\
4.94994994994995	76.7516566701296\\
4.95495495495495	76.7468816437398\\
4.95995995995996	76.7421393494384\\
4.96496496496496	76.7374295557231\\
4.96996996996997	76.7327520326057\\
4.97497497497497	76.7281065516124\\
4.97997997997998	76.7234928857833\\
4.98498498498498	76.7189108096727\\
4.98998998998999	76.7143600993494\\
4.99499499499499	76.7098405323958\\
5	76.7053518879091\\
};
\addlegendentry{Sani}

\addplot [color=mycolor2, dashdotted]
  table[row sep=crcr]{%
0	7\\
0.005005005005005	7.06298691984465\\
0.01001001001001	7.12643919517239\\
0.015015015015015	7.19035833906985\\
0.02002002002002	7.25474582965249\\
0.025025025025025	7.31960310933302\\
0.03003003003003	7.38493158482141\\
0.035035035035035	7.45073257836408\\
0.04004004004004	7.51700740673299\\
0.045045045045045	7.5837574039862\\
0.0500500500500501	7.65098385396419\\
0.0550550550550551	7.71868799028986\\
0.0600600600600601	7.78687099636853\\
0.0650650650650651	7.85553400538797\\
0.0700700700700701	7.92467810031835\\
0.0750750750750751	7.99430431391229\\
0.0800800800800801	8.06441362870481\\
0.0850850850850851	8.13500697701339\\
0.0900900900900901	8.2060852409379\\
0.0950950950950951	8.27764925236066\\
0.1001001001001	8.3496997929464\\
0.105105105105105	8.4222375941423\\
0.11011011011011	8.49526333717794\\
0.115115115115115	8.56877765306534\\
0.12012012012012	8.64278112259895\\
0.125125125125125	8.71727427635563\\
0.13013013013013	8.79225759469467\\
0.135135135135135	8.86773150775781\\
0.14014014014014	8.94369639546918\\
0.145145145145145	9.02015258753537\\
0.15015015015015	9.09710036344537\\
0.155155155155155	9.17453995247061\\
0.16016016016016	9.25246721375046\\
0.165165165165165	9.330878532132\\
0.17017017017017	9.40977480350429\\
0.175175175175175	9.48915681138211\\
0.18018018018018	9.56902522690594\\
0.185185185185185	9.64938060884197\\
0.19019019019019	9.73022340358207\\
0.195195195195195	9.81155394514385\\
0.2002002002002	9.8933724551706\\
0.205205205205205	9.97567904293132\\
0.21021021021021	10.0584737053207\\
0.215215215215215	10.1417563268592\\
0.22022022022022	10.2255266796929\\
0.225225225225225	10.3097844235936\\
0.23023023023023	10.3945291059589\\
0.235235235235235	10.4797601618119\\
0.24024024024024	10.5654769138017\\
0.245245245245245	10.6516785722029\\
0.25025025025025	10.7383642349157\\
0.255255255255255	10.8255328874663\\
0.26026026026026	10.9131834030064\\
0.265265265265265	11.0013145423134\\
0.27027027027027	11.0899249537906\\
0.275275275275275	11.1790131734667\\
0.28028028028028	11.2685776249965\\
0.285285285285285	11.35861661966\\
0.29029029029029	11.4491283563634\\
0.295295295295295	11.5401109216383\\
0.3003003003003	11.6315622896421\\
0.305305305305305	11.7234803221579\\
0.31031031031031	11.8158627685945\\
0.315315315315315	11.9087072659865\\
0.32032032032032	12.0020113389939\\
0.325325325325325	12.0957723999027\\
0.33033033033033	12.1899877486246\\
0.335335335335335	12.2846545726968\\
0.34034034034034	12.3797699472825\\
0.345345345345345	12.4753308351703\\
0.35035035035035	12.5713340867746\\
0.355355355355355	12.6677764401357\\
0.36036036036036	12.7646545209193\\
0.365365365365365	12.8619648424171\\
0.37037037037037	12.9597038055462\\
0.375375375375375	13.0578676988496\\
0.38038038038038	13.156452698496\\
0.385385385385385	13.2554548682798\\
0.39039039039039	13.354870159621\\
0.395395395395395	13.4546944115655\\
0.4004004004004	13.5549233507846\\
0.405405405405405	13.6555525915757\\
0.41041041041041	13.7565776358615\\
0.415415415415415	13.8579938731908\\
0.42042042042042	13.9597965807377\\
0.425425425425425	14.0619809233023\\
0.43043043043043	14.1645419533103\\
0.435435435435435	14.2674746108132\\
0.44044044044044	14.370773723488\\
0.445445445445445	14.4744340066376\\
0.45045045045045	14.5784500631905\\
0.455455455455455	14.6828163837009\\
0.46046046046046	14.7875273463489\\
0.465465465465465	14.8925772169399\\
0.47047047047047	14.9979601489055\\
0.475475475475475	15.1036701833026\\
0.48048048048048	15.2097012488139\\
0.485485485485485	15.3160471617481\\
0.49049049049049	15.4227016260392\\
0.495495495495495	15.5296582332471\\
0.500500500500501	15.6369104625573\\
0.505505505505506	15.7444516807813\\
0.510510510510511	15.8522751423559\\
0.515515515515516	15.9603739893438\\
0.520520520520521	16.0687412514335\\
0.525525525525526	16.177369845939\\
0.530530530530531	16.2862525778001\\
0.535535535535536	16.3953821395825\\
0.540540540540541	16.5047511114772\\
0.545545545545546	16.6143519613012\\
0.550550550550551	16.7241770444972\\
0.555555555555556	16.8342186041334\\
0.560560560560561	16.9444687709039\\
0.565565565565566	17.0549195631284\\
0.570570570570571	17.1655628867524\\
0.575575575575576	17.276390535347\\
0.580580580580581	17.387394190109\\
0.585585585585586	17.4985654198612\\
0.590590590590591	17.6098956810516\\
0.595595595595596	17.7213763177542\\
0.600600600600601	17.8329985616689\\
0.605605605605606	17.9447535321208\\
0.610610610610611	18.0566322360611\\
0.615615615615616	18.1686255680666\\
0.620620620620621	18.2807243103397\\
0.625625625625626	18.3929191327088\\
0.630630630630631	18.5052005926276\\
0.635635635635636	18.6175591351757\\
0.640640640640641	18.7299850930586\\
0.645645645645646	18.8424686866071\\
0.650650650650651	18.9550000237781\\
0.655655655655656	19.0675692529414\\
0.660660660660661	19.1801837331534\\
0.665665665665666	19.2928432940739\\
0.670670670670671	19.4055350473286\\
0.675675675675676	19.5182461426963\\
0.680680680680681	19.6309637681091\\
0.685685685685686	19.7436751496524\\
0.690690690690691	19.8563675515643\\
0.695695695695696	19.9690282762366\\
0.700700700700701	20.0816446642137\\
0.705705705705706	20.1942040941936\\
0.710710710710711	20.3066939830273\\
0.715715715715716	20.4191017857187\\
0.720720720720721	20.5314149954253\\
0.725725725725726	20.6436211434573\\
0.730730730730731	20.7557077992785\\
0.735735735735736	20.8676625705055\\
0.740740740740741	20.9794731029081\\
0.745745745745746	21.0911270804094\\
0.750750750750751	21.2026122250855\\
0.755755755755756	21.3139162971658\\
0.760760760760761	21.4250270950327\\
0.765765765765766	21.5359324552219\\
0.770770770770771	21.6466202524221\\
0.775775775775776	21.7570783994751\\
0.780780780780781	21.8672948473762\\
0.785785785785786	21.9772575852734\\
0.790790790790791	22.0869546404681\\
0.795795795795796	22.1963740784149\\
0.800800800800801	22.3055040027214\\
0.805805805805806	22.4143325551485\\
0.810810810810811	22.52284791561\\
0.815815815815816	22.6310383021731\\
0.820820820820821	22.738891971058\\
0.825825825825826	22.8463972166382\\
0.830830830830831	22.9535423714402\\
0.835835835835836	23.0603158061438\\
0.840840840840841	23.1667059295817\\
0.845845845845846	23.2727011887399\\
0.850850850850851	23.3782900687577\\
0.855855855855856	23.4834610929273\\
0.860860860860861	23.5882028226943\\
0.865865865865866	23.6925038576571\\
0.870870870870871	23.7963528355675\\
0.875875875875876	23.8997384323305\\
0.880880880880881	24.0026493620041\\
0.885885885885886	24.1050743767995\\
0.890890890890891	24.2070022670811\\
0.895895895895896	24.3084218613663\\
0.900900900900901	24.4093220263258\\
0.905905905905906	24.5096916667834\\
0.910910910910911	24.609519725716\\
0.915915915915916	24.7087951842538\\
0.920920920920921	24.80750706168\\
0.925925925925926	24.905644415431\\
0.930930930930931	25.0031963410963\\
0.935935935935936	25.1001519724187\\
0.940940940940941	25.196500481294\\
0.945945945945946	25.2922310777711\\
0.950950950950951	25.3873330100522\\
0.955955955955956	25.4817955644927\\
0.960960960960961	25.5756080656009\\
0.965965965965966	25.6687598760385\\
0.970970970970971	25.7612403966202\\
0.975975975975976	25.8530390663139\\
0.980980980980981	25.9441453622406\\
0.985985985985986	26.0345487996746\\
0.990990990990991	26.1242389320431\\
0.995995995995996	26.2132053509267\\
1.001001001001	26.301437686059\\
1.00600600600601	26.3889256053267\\
1.01101101101101	26.4756588147699\\
1.01601601601602	26.5616270585816\\
1.02102102102102	26.6468201191081\\
1.02602602602603	26.7312278168487\\
1.03103103103103	26.814840010456\\
1.03603603603604	26.8976465967356\\
1.04104104104104	26.9796375106465\\
1.04604604604605	27.0608027253005\\
1.05105105105105	27.1411322519629\\
1.05605605605606	27.2206161400519\\
1.06106106106106	27.299244477139\\
1.06606606606607	27.3770073889486\\
1.07107107107107	27.4538950393587\\
1.07607607607608	27.5298976303999\\
1.08108108108108	27.6050054022565\\
1.08608608608609	27.6792086332655\\
1.09109109109109	27.7524976399173\\
1.0960960960961	27.8248627768554\\
1.1011011011011	27.8962944368764\\
1.10610610610611	27.9667830509301\\
1.11111111111111	28.0363190881194\\
1.11611611611612	28.1048930557004\\
1.12112112112112	28.1724954990823\\
1.12612612612613	28.2391170018275\\
1.13113113113113	28.3047481856516\\
1.13613613613614	28.3693797104231\\
1.14114114114114	28.433002274164\\
1.14614614614615	28.4956066130492\\
1.15115115115115	28.5571835014068\\
1.15615615615616	28.617723922402\\
1.16116116116116	28.6772252699722\\
1.16616616616617	28.7356845975781\\
1.17117117117117	28.7930967733901\\
1.17617617617618	28.8494568122452\\
1.18118118118118	28.9047598756466\\
1.18618618618619	28.9590012717638\\
1.19119119119119	29.012176455433\\
1.1961961961962	29.0642810281563\\
1.2012012012012	29.1153107381027\\
1.20620620620621	29.1652614801072\\
1.21121121121121	29.2141292956713\\
1.21621621621622	29.2619103729629\\
1.22122122122122	29.3086010468164\\
1.22622622622623	29.3541977987322\\
1.23123123123123	29.3986972568776\\
1.23623623623624	29.4420961960858\\
1.24124124124124	29.4843915378567\\
1.24624624624625	29.5255803503565\\
1.25125125125125	29.5656598484176\\
1.25625625625626	29.604627393539\\
1.26126126126126	29.6424804938861\\
1.26626626626627	29.6792168042904\\
1.27127127127127	29.7148341262501\\
1.27627627627628	29.7493304079297\\
1.28128128128128	29.7827037441598\\
1.28628628628629	29.8149523764378\\
1.29129129129129	29.8460746929272\\
1.2962962962963	29.876069228458\\
1.3013013013013	29.9049346645266\\
1.30630630630631	29.9326698292955\\
1.31131131131131	29.9592736975941\\
1.31631631631632	29.9847453909176\\
1.32132132132132	30.0090841774281\\
1.32632632632633	30.0322894719537\\
1.33133133133133	30.054360835989\\
1.33633633633634	30.0752979776951\\
1.34134134134134	30.0951007518994\\
1.34634634634635	30.1137691600955\\
1.35135135135135	30.1313033504436\\
1.35635635635636	30.1477036177704\\
1.36136136136136	30.1629704035685\\
1.36636636636637	30.1771042959974\\
1.37137137137137	30.1901060298826\\
1.37637637637638	30.2019764867163\\
1.38138138138138	30.2127166946568\\
1.38638638638639	30.222327828529\\
1.39139139139139	30.2308112098239\\
1.3963963963964	30.2381683066992\\
1.4014014014014	30.2444007339788\\
1.40640640640641	30.249510253153\\
1.41141141141141	30.2534987723785\\
1.41641641641642	30.2563683464783\\
1.42142142142142	30.258121176942\\
1.42642642642643	30.2587596119254\\
1.43143143143143	30.2582861462506\\
1.43643643643644	30.2567034214063\\
1.44144144144144	30.2540142255474\\
1.44644644644645	30.2502214934953\\
1.45145145145145	30.2453283067377\\
1.45645645645646	30.2393378934288\\
1.46146146146146	30.2322536283889\\
1.46646646646647	30.224079033105\\
1.47147147147147	30.2148177757304\\
1.47647647647648	30.2044736710846\\
1.48148148148148	30.1930506806537\\
1.48648648648649	30.18055291259\\
1.49149149149149	30.1669846217124\\
1.4964964964965	30.1523502095059\\
1.5015015015015	30.1366542241221\\
1.50650650650651	30.1199013603789\\
1.51151151151151	30.1020964597606\\
1.51651651651652	30.0832445104178\\
1.52152152152152	30.0633506471676\\
1.52652652652653	30.0424201514934\\
1.53153153153153	30.0204584515451\\
1.53653653653654	29.9974711221387\\
1.54154154154154	29.973463884757\\
1.54654654654655	29.9484426075487\\
1.55155155155155	29.9224133053292\\
1.55655655655656	29.8953821395803\\
1.56156156156156	29.86735541845\\
1.56656656656657	29.8383395967527\\
1.57157157157157	29.8083412759694\\
1.57657657657658	29.7773672042472\\
1.58158158158158	29.7454242763997\\
1.58658658658659	29.712519533907\\
1.59159159159159	29.6786601649153\\
1.5965965965966	29.6438535042374\\
1.6016016016016	29.6081070333524\\
1.60660660660661	29.5714283804058\\
1.61161161161161	29.5338253202096\\
1.61661661661662	29.4953057742419\\
1.62162162162162	29.4558778106474\\
1.62662662662663	29.415549644237\\
1.63163163163163	29.3743296364883\\
1.63663663663664	29.332226295545\\
1.64164164164164	29.2892482762172\\
1.64664664664665	29.2454043799814\\
1.65165165165165	29.2007035549806\\
1.65665665665666	29.1551543232042\\
1.66166166166166	29.1087568707419\\
1.66666666666667	29.0615173357625\\
1.67167167167167	29.0134449556093\\
1.67667667667668	28.9645489463463\\
1.68168168168168	28.9148385027591\\
1.68668668668669	28.8643227983545\\
1.69169169169169	28.8130109853605\\
1.6966966966967	28.7609121947263\\
1.7017017017017	28.7080355361226\\
1.70670670670671	28.6543900979409\\
1.71171171171171	28.5999849472944\\
1.71671671671672	28.5448291300172\\
1.72172172172172	28.4889316706649\\
1.72672672672673	28.4323015725142\\
1.73173173173173	28.3749478175631\\
1.73673673673674	28.3168793665308\\
1.74174174174174	28.2581051588579\\
1.74674674674675	28.1986341127059\\
1.75175175175175	28.138475124958\\
1.75675675675676	28.0776370712184\\
1.76176176176176	28.0161288058124\\
1.76676676676677	27.9539591617869\\
1.77177177177177	27.8911369509098\\
1.77677677677678	27.8276709636702\\
1.78178178178178	27.7635699692787\\
1.78678678678679	27.698842715667\\
1.79179179179179	27.633497929488\\
1.7967967967968	27.5675443161158\\
1.8018018018018	27.5009905596461\\
1.80680680680681	27.4338453228953\\
1.81181181181181	27.3661172474016\\
1.81681681681682	27.297814953424\\
1.82182182182182	27.228947039943\\
1.82682682682683	27.1595220846602\\
1.83183183183183	27.0895486439987\\
1.83683683683684	27.0190352531025\\
1.84184184184184	26.9479904258371\\
1.84684684684685	26.8764226547892\\
1.85185185185185	26.8043404112666\\
1.85685685685686	26.7317521452985\\
1.86186186186186	26.6586662856354\\
1.86686686686687	26.5850912397488\\
1.87187187187187	26.5110353938318\\
1.87687687687688	26.4365071127983\\
1.88188188188188	26.3615147402839\\
1.88688688688689	26.2860665986451\\
1.89189189189189	26.2101709889598\\
1.8968968968969	26.1338361910273\\
1.9019019019019	26.0570704633677\\
1.90690690690691	25.9798820432229\\
1.91191191191191	25.9022791465556\\
1.91691691691692	25.82426996805\\
1.92192192192192	25.7458626811115\\
1.92692692692693	25.6670654378666\\
1.93193193193193	25.5878863691632\\
1.93693693693694	25.5083335845705\\
1.94194194194194	25.4284151723787\\
1.94694694694695	25.3481391995995\\
1.95195195195195	25.2675137119657\\
1.95695695695696	25.1865467339315\\
1.96196196196196	25.1052462686721\\
1.96696696696697	25.0236202980842\\
1.97197197197197	24.9416767827856\\
1.97697697697698	24.8594236621153\\
1.98198198198198	24.7768688541338\\
1.98698698698699	24.6940202556225\\
1.99199199199199	24.6108857420844\\
1.996996996997	24.5274731677434\\
2.002002002002	24.4437903655449\\
2.00700700700701	24.3598451471554\\
2.01201201201201	24.2756453029628\\
2.01701701701702	24.1911986020761\\
2.02202202202202	24.1065127923256\\
2.02702702702703	24.021595600263\\
2.03203203203203	23.9364547311608\\
2.03703703703704	23.8510978690133\\
2.04204204204204	23.7655326765357\\
2.04704704704705	23.6797667951645\\
2.05205205205205	23.5938078450576\\
2.05705705705706	23.5076634250939\\
2.06206206206206	23.4213411128737\\
2.06706706706707	23.3348484647186\\
2.07207207207207	23.2481930156713\\
2.07707707707708	23.1613822794959\\
2.08208208208208	23.0744237486776\\
2.08708708708709	22.9873248944229\\
2.09209209209209	22.9000931666596\\
2.0970970970971	22.8127359940367\\
2.1021021021021	22.7252607839244\\
2.10710710710711	22.6376749224142\\
2.11211211211211	22.549985774319\\
2.11711711711712	22.4622006831726\\
2.12212212212212	22.3743269712304\\
2.12712712712713	22.2863719394687\\
2.13213213213213	22.1983428675854\\
2.13713713713714	22.1102470139995\\
2.14214214214214	22.022091615851\\
2.14714714714715	21.9338838890016\\
2.15215215215215	21.8456310280339\\
2.15715715715716	21.7573392995341\\
2.16216216216216	21.6690069470075\\
2.16716716716717	21.5806375471936\\
2.17217217217217	21.4922365724291\\
2.17717717717718	21.4038094454467\\
2.18218218218218	21.315361539375\\
2.18718718718719	21.2268981777388\\
2.19219219219219	21.1384246344587\\
2.1971971971972	21.0499461338514\\
2.2022022022022	20.9614678506293\\
2.20720720720721	20.8729949099012\\
2.21221221221221	20.7845323871715\\
2.21721721721722	20.6960853083408\\
2.22222222222222	20.6076586497056\\
2.22722722722723	20.5192573379583\\
2.23223223223223	20.4308862501875\\
2.23723723723724	20.3425502138775\\
2.24224224224224	20.2542540069088\\
2.24724724724725	20.1660023575578\\
2.25225225225225	20.0777999444967\\
2.25725725725726	19.989651396794\\
2.26226226226226	19.901561293914\\
2.26726726726727	19.813534165717\\
2.27227227227227	19.7255744924592\\
2.27727727727728	19.6376867047928\\
2.28228228228228	19.5498751837662\\
2.28728728728729	19.4621442608235\\
2.29229229229229	19.3744982178048\\
2.2972972972973	19.2869412869464\\
2.3023023023023	19.1994776508803\\
2.30730730730731	19.1121114426347\\
2.31231231231231	19.0248467456336\\
2.31731731731732	18.9376875936971\\
2.32232232232232	18.8506379710412\\
2.32732732732733	18.7637018122779\\
2.33233233233233	18.6768830024153\\
2.33733733733734	18.5901853768573\\
2.34234234234234	18.5036127214038\\
2.34734734734735	18.4171687722507\\
2.35235235235235	18.33085721599\\
2.35735735735736	18.2446816896094\\
2.36236236236236	18.1586457804929\\
2.36736736736737	18.0727530264202\\
2.37237237237237	17.9870069155672\\
2.37737737737738	17.9014108865057\\
2.38238238238238	17.8159683282033\\
2.38738738738739	17.7306825800239\\
2.39239239239239	17.645556931727\\
2.3973973973974	17.5605946234685\\
2.4024024024024	17.4757988457998\\
2.40740740740741	17.3911727396688\\
2.41241241241241	17.306719396419\\
2.41741741741742	17.2224418577899\\
2.42242242242242	17.1383431159172\\
2.42742742742743	17.0544261133323\\
2.43243243243243	16.9706937429628\\
2.43743743743744	16.8871488481322\\
2.44244244244244	16.80379422256\\
2.44744744744745	16.7206326103615\\
2.45245245245245	16.6376667060483\\
2.45745745745746	16.5548991545277\\
2.46246246246246	16.472332551103\\
2.46746746746747	16.3899694414737\\
2.47247247247247	16.307812321735\\
2.47747747747748	16.2258636383783\\
2.48248248248248	16.1441257882908\\
2.48748748748749	16.0626011187558\\
2.49249249249249	15.9812919274525\\
2.4974974974975	15.9002004624562\\
2.5025025025025	15.8193289222379\\
2.50750750750751	15.738679455665\\
2.51251251251251	15.6582541620004\\
2.51751751751752	15.5780550909034\\
2.52252252252252	15.498084242429\\
2.52752752752753	15.4183435670283\\
2.53253253253253	15.3388349655483\\
2.53753753753754	15.259560289232\\
2.54254254254254	15.1805213397185\\
2.54754754754755	15.1017198690427\\
2.55255255255255	15.0231575796355\\
2.55755755755756	14.944836124324\\
2.56256256256256	14.8667571063309\\
2.56756756756757	14.7889220792752\\
2.57257257257257	14.7113325471717\\
2.57757757757758	14.6339899644312\\
2.58258258258258	14.5568957358606\\
2.58758758758759	14.4800512166627\\
2.59259259259259	14.4034577124361\\
2.5975975975976	14.3271164791757\\
2.6026026026026	14.2510287232721\\
2.60760760760761	14.1751956015121\\
2.61261261261261	14.0996182210783\\
2.61761761761762	14.0242976395493\\
2.62262262262262	13.9492348648998\\
2.62762762762763	13.8744308555003\\
2.63263263263263	13.7998865201175\\
2.63763763763764	13.7256027179138\\
2.64264264264264	13.6515802584479\\
2.64764764764765	13.5778199016741\\
2.65265265265265	13.5043223579431\\
2.65765765765766	13.4310887901432\\
2.66266266266266	13.3581228150705\\
2.66766766766767	13.2854254954891\\
2.67267267267267	13.2129973744842\\
2.67767767767768	13.1408389791028\\
2.68268268268268	13.0689508203544\\
2.68768768768769	12.9973333932105\\
2.69269269269269	12.9259871766046\\
2.6976976976977	12.8549126334325\\
2.7027027027027	12.7841102105521\\
2.70770770770771	12.7135803387836\\
2.71271271271271	12.6433234329089\\
2.71771771771772	12.5733398916726\\
2.72272272272272	12.503630097781\\
2.72772772772773	12.4341944179027\\
2.73273273273273	12.3650332026685\\
2.73773773773774	12.2961467866713\\
2.74274274274274	12.2275354884662\\
2.74774774774775	12.1591996105702\\
2.75275275275275	12.0911394394626\\
2.75775775775776	12.0233552455851\\
2.76276276276276	11.955847283341\\
2.76776776776777	11.8886157910962\\
2.77277277277277	11.8216609911786\\
2.77777777777778	11.7549830898781\\
2.78278278278278	11.6885822774468\\
2.78778778778779	11.6224587280992\\
2.79279279279279	11.5566126000116\\
2.7977977977978	11.4910440353226\\
2.8028028028028	11.4257531601329\\
2.80780780780781	11.3607400845054\\
2.81281281281281	11.296004902465\\
2.81781781781782	11.231547691999\\
2.82282282282282	11.1673685150565\\
2.82782782782783	11.1034674175491\\
2.83283283283283	11.0398444293502\\
2.83783783783784	10.9764995642955\\
2.84284284284284	10.913432820183\\
2.84784784784785	10.8506441787725\\
2.85285285285285	10.7881336057863\\
2.85785785785786	10.7259010509085\\
2.86286286286286	10.6639464477856\\
2.86786786786787	10.6022697140261\\
2.87287287287287	10.5408707512007\\
2.87787787787788	10.4797494448422\\
2.88288288288288	10.4189056644456\\
2.88788788788789	10.358339263468\\
2.89289289289289	10.2980500793286\\
2.8978978978979	10.2380379334089\\
2.9029029029029	10.1783026310524\\
2.90790790790791	10.1188439615646\\
2.91291291291291	10.0596616982135\\
2.91791791791792	10.000755598229\\
2.92292292292292	9.94212540280317\\
2.92792792792793	9.88377083709024\\
2.93293293293293	9.82569161020659\\
2.93793793793794	9.76788741523074\\
2.94294294294294	9.71035792920333\\
2.94794794794795	9.65310281312716\\
2.95295295295295	9.59612171196717\\
2.95795795795796	9.53941425465043\\
2.96296296296296	9.48298005406616\\
2.96796796796797	9.4268187070657\\
2.97297297297297	9.37092979446257\\
2.97797797797798	9.31531288103239\\
2.98298298298298	9.25996751551295\\
2.98798798798799	9.20489323060417\\
2.99299299299299	9.1500895429681\\
2.997997997998	9.09555595322894\\
3.003003003003	9.04129194597303\\
3.00800800800801	8.98729698974887\\
3.01301301301301	8.93357053706706\\
3.01801801801802	8.88011202440038\\
3.02302302302302	8.82692087218372\\
3.02802802802803	8.77399648481413\\
3.03303303303303	8.7213382506508\\
3.03803803803804	8.66894554201504\\
3.04304304304304	8.61681771519032\\
3.04804804804805	8.56495411042227\\
3.05305305305305	8.5133540519186\\
3.05805805805806	8.46201684784922\\
3.06306306306306	8.41094179034616\\
3.06806806806807	8.36012815550359\\
3.07307307307307	8.30957520337781\\
3.07807807807808	8.25928217798727\\
3.08308308308308	8.20924830731257\\
3.08808808808809	8.15947280329643\\
3.09309309309309	8.10995486184374\\
3.0980980980981	8.0606936628215\\
3.1031031031031	8.01168837005886\\
3.10810810810811	7.96293813134712\\
3.11311311311311	7.91444207843972\\
3.11811811811812	7.86619932705222\\
3.12312312312312	7.81820897686236\\
3.12812812812813	7.77047011150997\\
3.13313313313313	7.72298179859707\\
3.13813813813814	7.67574308968778\\
3.14314314314314	7.62875302030839\\
3.14814814814815	7.58201060994733\\
3.15315315315315	7.53551486205513\\
3.15815815815816	7.48926502537071\\
3.16316316316316	7.44326123612801\\
3.16816816816817	7.39750278057533\\
3.17317317317317	7.35198883292373\\
3.17817817817818	7.30671856603044\\
3.18318318318318	7.26169115139882\\
3.18818818818819	7.2169057591784\\
3.19319319319319	7.17236155816487\\
3.1981981981982	7.12805771580008\\
3.2032032032032	7.08399339817201\\
3.20820820820821	7.04016777001482\\
3.21321321321321	6.99657999470881\\
3.21821821821822	6.95322923428045\\
3.22322322322322	6.91011464940235\\
3.22822822822823	6.86723539939329\\
3.23323323323323	6.82459064221819\\
3.23823823823824	6.78217953448814\\
3.24324324324324	6.74000123146037\\
3.24824824824825	6.69805488703829\\
3.25325325325325	6.65633965377143\\
3.25825825825826	6.61485468285551\\
3.26326326326326	6.57359912413237\\
3.26826826826827	6.53257212609005\\
3.27327327327327	6.49177283586271\\
3.27827827827828	6.45120039923067\\
3.28328328328328	6.41085396062042\\
3.28828828828829	6.37073266310459\\
3.29329329329329	6.33083564840197\\
3.2982982982983	6.29116205687752\\
3.3033033033033	6.25171102754233\\
3.30830830830831	6.21248169805367\\
3.31331331331331	6.17347320471494\\
3.31831831831832	6.13468468247571\\
3.32332332332332	6.09611526493171\\
3.32832832832833	6.05776408432482\\
3.33333333333333	6.01963027154307\\
3.33833833833834	5.98171295612066\\
3.34334334334334	5.94401126623792\\
3.34834834834835	5.90652432872136\\
3.35335335335335	5.86925126904363\\
3.35835835835836	5.83219121132355\\
3.36336336336336	5.79534327832608\\
3.36836836836837	5.75870659146234\\
3.37337337337337	5.72228027078962\\
3.37837837837838	5.68606343501134\\
3.38338338338338	5.65005520147709\\
3.38838838838839	5.61425468618261\\
3.39339339339339	5.57866100376981\\
3.3983983983984	5.54327326752674\\
3.4034034034034	5.5080905893876\\
3.40840840840841	5.47311207993277\\
3.41341341341341	5.43833684838876\\
3.41841841841842	5.40376400262824\\
3.42342342342342	5.36939264917005\\
3.42842842842843	5.33522189317918\\
3.43343343343343	5.30125083846676\\
3.43843843843844	5.26747858749009\\
3.44344344344344	5.23390424135263\\
3.44844844844845	5.20052689980397\\
3.45345345345345	5.1673456612399\\
3.45845845845846	5.13435962270231\\
3.46346346346346	5.10156787987929\\
3.46846846846847	5.06896952710506\\
3.47347347347347	5.03656365736\\
3.47847847847848	5.00434936227067\\
3.48348348348348	4.97232573210975\\
3.48848848848849	4.94049185579608\\
3.49349349349349	4.90884682089469\\
3.4984984984985	4.87738971361673\\
3.5035035035035	4.84611961881951\\
3.50850850850851	4.8150356200065\\
3.51351351351351	4.78413679932733\\
3.51851851851852	4.75342223757779\\
3.52352352352352	4.72289101419981\\
3.52852852852853	4.69254220728149\\
3.53353353353353	4.66237489355707\\
3.53853853853854	4.63238814840695\\
3.54354354354354	4.6025810458577\\
3.54854854854855	4.57295265858203\\
3.55355355355355	4.5435020578988\\
3.55855855855856	4.51422831377304\\
3.56356356356356	4.48513049481595\\
3.56856856856857	4.45620766828483\\
3.57357357357357	4.4274589000832\\
3.57857857857858	4.39888325476069\\
3.58358358358358	4.37047979551311\\
3.58858858858859	4.34224758418241\\
3.59359359359359	4.3141856812567\\
3.5985985985986	4.28629314587026\\
3.6036036036036	4.25856903580351\\
3.60860860860861	4.23101240748301\\
3.61361361361361	4.20362231598151\\
3.61861861861862	4.17639781501791\\
3.62362362362362	4.14933795695723\\
3.62862862862863	4.12244179281068\\
3.63363363363363	4.09570837223561\\
3.63863863863864	4.06913674353554\\
3.64364364364364	4.04272595366012\\
3.64864864864865	4.01647504820519\\
3.65365365365365	3.99038307141272\\
3.65865865865866	3.96444917707995\\
3.66366366366366	3.93867281741119\\
3.66866866866867	3.91305323726883\\
3.67367367367367	3.88758966547144\\
3.67867867867868	3.86228133240705\\
3.68368368368368	3.83712747003314\\
3.68868868868869	3.81212731187664\\
3.69369369369369	3.78728009303394\\
3.6986986986987	3.76258505017087\\
3.7037037037037	3.73804142152273\\
3.70870870870871	3.71364844689427\\
3.71371371371371	3.68940536765967\\
3.71871871871872	3.6653114267626\\
3.72372372372372	3.64136586871615\\
3.72872872872873	3.61756793960288\\
3.73373373373373	3.59391688707481\\
3.73873873873874	3.57041196035339\\
3.74374374374374	3.54705241022954\\
3.74874874874875	3.52383748906364\\
3.75375375375375	3.5007664507855\\
3.75875875875876	3.47783855089439\\
3.76376376376376	3.45505304645906\\
3.76876876876877	3.43240919611767\\
3.77377377377377	3.40990626007787\\
3.77877877877878	3.38754350011674\\
3.78378378378378	3.36532017958081\\
3.78878878878879	3.3432355633861\\
3.79379379379379	3.32128891801803\\
3.7987987987988	3.29947951153151\\
3.8038038038038	3.2778066135509\\
3.80880880880881	3.25626949526999\\
3.81381381381381	3.23486742945206\\
3.81881881881882	3.2135996904298\\
3.82382382382382	3.19246555410539\\
3.82882882882883	3.17146429795043\\
3.83383383383383	3.15059520100601\\
3.83883883883884	3.12985754388265\\
3.84384384384384	3.10925060876032\\
3.84884884884885	3.08877367938846\\
3.85385385385385	3.06842604108594\\
3.85885885885886	3.0482069807411\\
3.86386386386386	3.02811578681173\\
3.86886886886887	3.00815174932508\\
3.87387387387387	2.98831415987784\\
3.87887887887888	2.96860231163615\\
3.88388388388388	2.94901549933562\\
3.88888888888889	2.92955301928131\\
3.89389389389389	2.91021416934771\\
3.8988988988989	2.89099824897879\\
3.9039039039039	2.87190455918797\\
3.90890890890891	2.85293240255811\\
3.91391391391391	2.83408108324152\\
3.91891891891892	2.81534990695999\\
3.92392392392392	2.79673818100474\\
3.92892892892893	2.77824521423644\\
3.93393393393393	2.75987031708522\\
3.93893893893894	2.74161280155068\\
3.94394394394394	2.72347198120185\\
3.94894894894895	2.70544717117722\\
3.95395395395395	2.68753768818474\\
3.95895895895896	2.66974285050179\\
3.96396396396396	2.65206197797523\\
3.96896896896897	2.63449439202137\\
3.97397397397397	2.61703941562596\\
3.97897897897898	2.5996963733442\\
3.98398398398398	2.58246459130077\\
3.98898898898899	2.56534339718977\\
3.99399399399399	2.54833212027477\\
3.998998998999	2.5314300913888\\
4.004004004004	2.51463664293432\\
4.00900900900901	2.49795110888327\\
4.01401401401401	2.48137282477703\\
4.01901901901902	2.46490112772643\\
4.02402402402402	2.44853535641175\\
4.02902902902903	2.43227485108273\\
4.03403403403403	2.41611895355857\\
4.03903903903904	2.40006700722791\\
4.04404404404404	2.38411835704886\\
4.04904904904905	2.36827234954895\\
4.05405405405405	2.3525283328252\\
4.05905905905906	2.33688565654407\\
4.06406406406406	2.32134367194146\\
4.06906906906907	2.30590173182273\\
4.07407407407407	2.29055919056271\\
4.07907907907908	2.27531540410567\\
4.08408408408408	2.26016972996532\\
4.08908908908909	2.24512152722484\\
4.09409409409409	2.23017015653687\\
4.0990990990991	2.21531498012348\\
4.1041041041041	2.2005553617762\\
4.10910910910911	2.18589066685603\\
4.11411411411411	2.17132026229341\\
4.11911911911912	2.15684351658823\\
4.12412412412412	2.14245979980984\\
4.12912912912913	2.12816848359703\\
4.13413413413413	2.11396894115807\\
4.13913913913914	2.09986054727065\\
4.14414414414414	2.08584267828193\\
4.14914914914915	2.07191471210854\\
4.15415415415415	2.05807602823652\\
4.15915915915916	2.04432612266957\\
4.16416416416416	2.03066468364159\\
4.16916916916917	2.01709122848948\\
4.17417417417417	2.00360526962864\\
4.17917917917918	1.99020632089073\\
4.18418418418418	1.97689389752361\\
4.18918918918919	1.96366751619141\\
4.19419419419419	1.95052669497447\\
4.1991991991992	1.93747095336937\\
4.2042042042042	1.92449981228892\\
4.20920920920921	1.91161279406216\\
4.21421421421421	1.89880942243438\\
4.21921921921922	1.88608922256708\\
4.22422422422422	1.87345172103801\\
4.22922922922923	1.86089644584116\\
4.23423423423423	1.84842292638673\\
4.23923923923924	1.83603069350117\\
4.24424424424424	1.82371927942716\\
4.24924924924925	1.81148821782361\\
4.25425425425425	1.79933704376567\\
4.25925925925926	1.78726529374471\\
4.26426426426426	1.77527250566836\\
4.26926926926927	1.76335821886045\\
4.27427427427427	1.75152197406107\\
4.27927927927928	1.73976331342653\\
4.28428428428428	1.72808178052938\\
4.28928928928929	1.71647692035839\\
4.29429429429429	1.70494827931858\\
4.2992992992993	1.6934954052312\\
4.3043043043043	1.68211784733372\\
4.30930930930931	1.67081515627987\\
4.31431431431431	1.65958688413958\\
4.31931931931932	1.64843258439904\\
4.32432432432432	1.63735181196066\\
4.32932932932933	1.62634412314309\\
4.33433433433433	1.61540907568121\\
4.33933933933934	1.60454622872613\\
4.34434434434434	1.59375514284521\\
4.34934934934935	1.58303538002201\\
4.35435435435435	1.57238650365636\\
4.35935935935936	1.5618080785643\\
4.36436436436436	1.55129967097812\\
4.36936936936937	1.54086084854633\\
4.37437437437437	1.53049118033367\\
4.37937937937938	1.52019023682113\\
4.38438438438438	1.50995758990592\\
4.38938938938939	1.49979281290149\\
4.39439439439439	1.48969548053753\\
4.3993993993994	1.47966516895994\\
4.4044044044044	1.46970145573087\\
4.40940940940941	1.45980391982871\\
4.41441441441441	1.44997214164808\\
4.41941941941942	1.44020570299981\\
4.42442442442442	1.43050418711099\\
4.42942942942943	1.42086717862495\\
4.43443443443443	1.41129426360121\\
4.43943943943944	1.40178502951558\\
4.44444444444444	1.39233906526006\\
4.44944944944945	1.3829559611429\\
4.45445445445445	1.37363530888858\\
4.45945945945946	1.36437670163782\\
4.46446446446446	1.35517973394758\\
4.46946946946947	1.34604400179102\\
4.47447447447447	1.33696910255757\\
4.47947947947948	1.32795463505288\\
4.48448448448448	1.31900019949883\\
4.48948948948949	1.31010539753353\\
4.49449449449449	1.30126983221134\\
4.4994994994995	1.29249310800284\\
4.5045045045045	1.28377483079484\\
4.50950950950951	1.27511460789039\\
4.51451451451451	1.26651204800878\\
4.51951951951952	1.25796676128552\\
4.52452452452452	1.24947835927237\\
4.52952952952953	1.24104645493731\\
4.53453453453453	1.23267066266454\\
4.53953953953954	1.22435059825453\\
4.54454454454454	1.21608587892396\\
4.54954954954955	1.20787612330574\\
4.55455455455455	1.19972095144902\\
4.55955955955956	1.1916199848192\\
4.56456456456456	1.18357284629787\\
4.56956956956957	1.17557916018291\\
4.57457457457457	1.16763855218838\\
4.57957957957958	1.15975064944461\\
4.58458458458458	1.15191508049814\\
4.58958958958959	1.14413147531177\\
4.59459459459459	1.13639946526451\\
4.5995995995996	1.12871868315161\\
4.6046046046046	1.12108876318455\\
4.60960960960961	1.11350934099105\\
4.61461461461461	1.10598005361507\\
4.61961961961962	1.09850053951678\\
4.62462462462462	1.09107043857261\\
4.62962962962963	1.08368939207521\\
4.63463463463463	1.07635704273346\\
4.63963963963964	1.06907303467248\\
4.64464464464464	1.06183701343362\\
4.64964964964965	1.05464862597447\\
4.65465465465465	1.04750752066884\\
4.65965965965966	1.04041360323514\\
4.66466466466466	1.03336693049233\\
4.66966966966967	1.02636721249347\\
4.67467467467467	1.01941415747038\\
4.67967967967968	1.0125074749023\\
4.68468468468468	1.00564687551596\\
4.68968968968969	0.998832071285486\\
4.69469469469469	0.992062775432474\\
4.6996996996997	0.985338702425952\\
4.7047047047047	0.978659567982393\\
4.70970970970971	0.972025089065711\\
4.71471471471471	0.965434983887262\\
4.71971971971972	0.958888971905846\\
4.72472472472472	0.952386773827704\\
4.72972972972973	0.945928111606519\\
4.73473473473473	0.939512708443418\\
4.73973973973974	0.933140288786967\\
4.74474474474474	0.926810578333177\\
4.74974974974975	0.920523304025502\\
4.75475475475475	0.914278194054835\\
4.75975975975976	0.908074977859514\\
4.76476476476476	0.901913386125317\\
4.76976976976977	0.895793150785467\\
4.77477477477477	0.889714005020628\\
4.77977977977978	0.883675683258905\\
4.78478478478478	0.877677921175846\\
4.78978978978979	0.871720455694443\\
4.79479479479479	0.865803024985128\\
4.7997997997998	0.859925368465777\\
4.8048048048048	0.854087226801706\\
4.80980980980981	0.848288341905675\\
4.81481481481481	0.842528456937887\\
4.81981981981982	0.836807316305985\\
4.82482482482482	0.831124665665056\\
4.82982982982983	0.825480251917629\\
4.83483483483483	0.819873823213674\\
4.83983983983984	0.814305128950604\\
4.84484484484484	0.808773919773276\\
4.84984984984985	0.803279947573986\\
4.85485485485485	0.797822965492476\\
4.85985985985986	0.792402727915926\\
4.86486486486486	0.787018990478962\\
4.86986986986987	0.781671510063651\\
4.87487487487487	0.7763600447995\\
4.87987987987988	0.771084354063462\\
4.88488488488488	0.765844198479931\\
4.88988988988989	0.760639339920741\\
4.89489489489489	0.755469541505172\\
4.8998998998999	0.750334567599943\\
4.9049049049049	0.745234183819217\\
4.90990990990991	0.740168157024598\\
4.91491491491491	0.735136255325134\\
4.91991991991992	0.730138248077315\\
4.92492492492492	0.725173905885071\\
4.92992992992993	0.720243000599778\\
4.93493493493493	0.71534530532025\\
4.93993993993994	0.710480594392747\\
4.94494494494494	0.705648643410968\\
4.94994994994995	0.700849229216057\\
4.95495495495495	0.6960821298966\\
4.95995995995996	0.691347124788623\\
4.96496496496496	0.686643994475596\\
4.96996996996997	0.681972520788431\\
4.97497497497497	0.677332486805483\\
4.97997997997998	0.672723676852547\\
4.98498498498498	0.668145876502864\\
4.98998998998999	0.663598872577113\\
4.99499499499499	0.659082453143417\\
5	0.654596407517343\\
};
\addlegendentry{Infetti}

\addplot [color=mycolor3]
  table[row sep=crcr]{%
0	0\\
0.005005005005005	0.0960754326912481\\
0.01001001001001	0.19301467506287\\
0.015015015015015	0.290824096019784\\
0.02002002002002	0.389510084533857\\
0.025025025025025	0.48907904958774\\
0.03003003003003	0.58953742017484\\
0.035035035035035	0.690891632331687\\
0.04004004004004	0.793148155145168\\
0.045045045045045	0.896313491018354\\
0.0500500500500501	1.0003941541224\\
0.0550550550550551	1.10539667039663\\
0.0600600600600601	1.21132757754841\\
0.0650650650650651	1.3181934250533\\
0.0700700700700701	1.426000774155\\
0.0750750750750751	1.53475619786523\\
0.0800800800800801	1.64446628096397\\
0.0850850850850851	1.75513761999919\\
0.0900900900900901	1.86677682328711\\
0.0950950950950951	1.97939051091196\\
0.1001001001001	2.09298531472619\\
0.105105105105105	2.20756787835026\\
0.11011011011011	2.3231448571729\\
0.115115115115115	2.43972291835085\\
0.12012012012012	2.55730874080898\\
0.125125125125125	2.67590901524036\\
0.13013013013013	2.79553044410612\\
0.135135135135135	2.91617974163552\\
0.14014014014014	3.03786363382596\\
0.145145145145145	3.16058885844293\\
0.15015015015015	3.28436216502011\\
0.155155155155155	3.40919031485925\\
0.16016016016016	3.53507373018517\\
0.165165165165165	3.66201372035114\\
0.17017017017017	3.7900183393401\\
0.175175175175175	3.91909557868465\\
0.18018018018018	4.04925336746732\\
0.185185185185185	4.18049957232025\\
0.19019019019019	4.31284199742549\\
0.195195195195195	4.44628838451479\\
0.2002002002002	4.58084641286968\\
0.205205205205205	4.7165236993215\\
0.21021021021021	4.85332779825132\\
0.215215215215215	4.99126620159\\
0.22022022022022	5.13034633881822\\
0.225225225225225	5.27057557696633\\
0.23023023023023	5.41196122061456\\
0.235235235235235	5.55451051189286\\
0.24024024024024	5.69823063048096\\
0.245245245245245	5.84312869360838\\
0.25025025025025	5.98921175605443\\
0.255255255255255	6.13648681014809\\
0.26026026026026	6.28496078576824\\
0.265265265265265	6.43464055034351\\
0.27027027027027	6.58553290885222\\
0.275275275275275	6.73764460382258\\
0.28028028028028	6.89098231533249\\
0.285285285285285	7.04555266100965\\
0.29029029029029	7.20136219603158\\
0.295295295295295	7.35841741312546\\
0.3003003003003	7.51672474256839\\
0.305305305305305	7.6762905521871\\
0.31031031031031	7.8371211473582\\
0.315315315315315	7.99922277100804\\
0.32032032032032	8.16260160361273\\
0.325325325325325	8.32726376319817\\
0.33033033033033	8.49321530534004\\
0.335335335335335	8.66046222316378\\
0.34034034034034	8.82901044734457\\
0.345345345345345	8.99886584610748\\
0.35035035035035	9.17003422522721\\
0.355355355355355	9.34252132802834\\
0.36036036036036	9.51633283538516\\
0.365365365365365	9.69147436572177\\
0.37037037037037	9.86795147501203\\
0.375375375375375	10.0457696567796\\
0.38038038038038	10.2249343420979\\
0.385385385385385	10.40545089959\\
0.39039039039039	10.587324635429\\
0.395395395395395	10.7705607933376\\
0.4004004004004	10.9551645545883\\
0.405405405405405	11.1411410380034\\
0.41041041041041	11.3284952999548\\
0.415415415415415	11.5172323343646\\
0.42042042042042	11.7073570727042\\
0.425425425425425	11.8988743839951\\
0.43043043043043	12.0917890748083\\
0.435435435435435	12.286105889265\\
0.44044044044044	12.4818295090356\\
0.445445445445445	12.6789645533407\\
0.45045045045045	12.8775155789507\\
0.455455455455455	13.0774870801853\\
0.46046046046046	13.2788834889146\\
0.465465465465465	13.481709174558\\
0.47047047047047	13.685968444085\\
0.475475475475475	13.8916655420145\\
0.48048048048048	14.0988046504156\\
0.485485485485485	14.3073898889068\\
0.49049049049049	14.5174253146567\\
0.495495495495495	14.7289149223835\\
0.500500500500501	14.941862644355\\
0.505505505505506	15.1562723503892\\
0.510510510510511	15.3721478478536\\
0.515515515515516	15.5894928816654\\
0.520520520520521	15.8083111342917\\
0.525525525525526	16.0286062257495\\
0.530530530530531	16.2503817136053\\
0.535535535535536	16.4736410929756\\
0.540540540540541	16.6983877965265\\
0.545545545545546	16.9246251944741\\
0.550550550550551	17.152356594584\\
0.555555555555556	17.3815852421717\\
0.560560560560561	17.6123143201025\\
0.565565565565566	17.8445469487916\\
0.570570570570571	18.0782861862037\\
0.575575575575576	18.3135350278534\\
0.580580580580581	18.5502964068051\\
0.585585585585586	18.7885731936729\\
0.590590590590591	19.0283681966209\\
0.595595595595596	19.2696841613626\\
0.600600600600601	19.5125237711616\\
0.605605605605606	19.756889646831\\
0.610610610610611	20.0027843467341\\
0.615615615615616	20.2502103667835\\
0.620620620620621	20.4991701404417\\
0.625625625625626	20.7496660387212\\
0.630630630630631	21.0017003701841\\
0.635635635635636	21.2552753809422\\
0.640640640640641	21.5103932546573\\
0.645645645645646	21.7670561125407\\
0.650650650650651	22.0252660133538\\
0.655655655655656	22.2850249062763\\
0.660660660660661	22.5463288462793\\
0.665665665665666	22.8091744926469\\
0.670670670670671	23.0735619635332\\
0.675675675675676	23.3394912247553\\
0.680680680680681	23.6069620897935\\
0.685685685685686	23.875974219791\\
0.690690690690691	24.1465271235543\\
0.695695695695696	24.4186201575529\\
0.700700700700701	24.6922525259194\\
0.705705705705706	24.9674232804496\\
0.710710710710711	25.2441313206023\\
0.715715715715716	25.5223753934994\\
0.720720720720721	25.8021540939259\\
0.725725725725726	26.0834658643299\\
0.730730730730731	26.3663089948226\\
0.735735735735736	26.6506816231785\\
0.740740740740741	26.9365817348348\\
0.745745745745746	27.2240071628921\\
0.750750750750751	27.512955588114\\
0.755755755755756	27.8034245389272\\
0.760760760760761	28.0954113914216\\
0.765765765765766	28.3889133693499\\
0.770770770770771	28.6839275441283\\
0.775775775775776	28.9804508348359\\
0.780780780780781	29.2784800082148\\
0.785785785785786	29.5780116786704\\
0.790790790790791	29.8790423082711\\
0.795795795795796	30.1815682067483\\
0.800800800800801	30.4855855314966\\
0.805805805805806	30.7910902875739\\
0.810810810810811	31.0980783277008\\
0.815815815815816	31.4065453522612\\
0.820820820820821	31.7164869093023\\
0.825825825825826	32.027898394534\\
0.830830830830831	32.3407750513295\\
0.835835835835836	32.6551119707252\\
0.840840840840841	32.9709040914204\\
0.845845845845846	33.2881461997776\\
0.850850850850851	33.6068329298225\\
0.855855855855856	33.9269587632437\\
0.860860860860861	34.2485180293929\\
0.865865865865866	34.5715049052852\\
0.870870870870871	34.8959134155984\\
0.875875875875876	35.2217374326737\\
0.880880880880881	35.5489706765152\\
0.885885885885886	35.8776067147902\\
0.890890890890891	36.2076389628292\\
0.895895895895896	36.5390606836255\\
0.900900900900901	36.8718649878358\\
0.905905905905906	37.2060448337797\\
0.910910910910911	37.5415930274401\\
0.915915915915916	37.8785022224627\\
0.920920920920921	38.2167649201566\\
0.925925925925926	38.5563734694939\\
0.930930930930931	38.8973200671097\\
0.935935935935936	39.2395967573023\\
0.940940940940941	39.583195432033\\
0.945945945945946	39.9281078309264\\
0.950950950950951	40.2743255412699\\
0.955955955955956	40.6218399980144\\
0.960960960960961	40.9706424837735\\
0.965965965965966	41.3207241288241\\
0.970970970970971	41.6720759111061\\
0.975975975975976	42.0246886562227\\
0.980980980980981	42.3785530374399\\
0.985985985985986	42.7336595756871\\
0.990990990990991	43.0899986395566\\
0.995995995995996	43.4475604453039\\
1.001001001001	43.8063350568475\\
1.00600600600601	44.1663123857692\\
1.01101101101101	44.5274821913135\\
1.01601601601602	44.8898340803885\\
1.02102102102102	45.253357507565\\
1.02602602602603	45.6180417750771\\
1.03103103103103	45.9838760328219\\
1.03603603603604	46.3508492783597\\
1.04104104104104	46.7189503569139\\
1.04604604604605	47.088167961371\\
1.05105105105105	47.4584906322803\\
1.05605605605606	47.8299067578547\\
1.06106106106106	48.2024045739698\\
1.06606606606607	48.5759721641644\\
1.07107107107107	48.9505974596407\\
1.07607607607608	49.3262682392634\\
1.08108108108108	49.7029721295609\\
1.08608608608609	50.0806966047243\\
1.09109109109109	50.459428986608\\
1.0960960960961	50.8391564447295\\
1.1011011011011	51.2198659962692\\
1.10610610610611	51.6015445060708\\
1.11111111111111	51.984178686641\\
1.11611611611612	52.3677550981496\\
1.12112112112112	52.7522601484297\\
1.12612612612613	53.1376800929773\\
1.13113113113113	53.5240010349513\\
1.13613613613614	53.9112089251741\\
1.14114114114114	54.2992895621311\\
1.14614614614615	54.6882285919706\\
1.15115115115115	55.0780115085042\\
1.15615615615616	55.468624389164\\
1.16116116116116	55.8600762026863\\
1.16616616616617	56.2523619533393\\
1.17117117117117	56.6454646785717\\
1.17617617617618	57.0393674601582\\
1.18118118118118	57.4340534241991\\
1.18618618618619	57.82950574112\\
1.19119119119119	58.2257076256726\\
1.1961961961962	58.6226423369337\\
1.2012012012012	59.0202931783059\\
1.20620620620621	59.4186434975176\\
1.21121121121121	59.8176766866224\\
1.21621621621622	60.2173761819998\\
1.22122122122122	60.6177254643549\\
1.22622622622623	61.0187080587182\\
1.23123123123123	61.420307534446\\
1.23623623623624	61.8225075052201\\
1.24124124124124	62.2252916290479\\
1.24624624624625	62.6286436082624\\
1.25125125125125	63.0325471895223\\
1.25625625625626	63.4369861638119\\
1.26126126126126	63.8419443664408\\
1.26626626626627	64.2474056770446\\
1.27127127127127	64.6533540195843\\
1.27627627627628	65.0597733623466\\
1.28128128128128	65.4666477179436\\
1.28628628628629	65.8739611433132\\
1.29129129129129	66.2816977397189\\
1.2962962962963	66.6898416527496\\
1.3013013013013	67.0983770723201\\
1.30630630630631	67.5072882326705\\
1.31131131131131	67.9165594123668\\
1.31631631631632	68.3261749343004\\
1.32132132132132	68.7361191656882\\
1.32632632632633	69.1463765180731\\
1.33133133133133	69.5569314473231\\
1.33633633633634	69.9677684536323\\
1.34134134134134	70.3788720815199\\
1.34634634634635	70.7902269198312\\
1.35135135135135	71.2018176017367\\
1.35635635635636	71.6136288047327\\
1.36136136136136	72.0256452506411\\
1.36636636636637	72.4378517056093\\
1.37137137137137	72.8502329801104\\
1.37637637637638	73.262773928943\\
1.38138138138138	73.6754594512315\\
1.38638638638639	74.0882744904257\\
1.39139139139139	74.5012040343011\\
1.3963963963964	74.9142331149587\\
1.4014014014014	75.3273468088252\\
1.40640640640641	75.7405302366529\\
1.41141141141141	76.1537685635197\\
1.41641641641642	76.5670469988291\\
1.42142142142142	76.98035079631\\
1.42642642642643	77.3936652540173\\
1.43143143143143	77.8069757143311\\
1.43643643643644	78.2202675639575\\
1.44144144144144	78.6335262339278\\
1.44644644644645	79.0467371995991\\
1.45145145145145	79.4598859806542\\
1.45645645645646	79.8729581411013\\
1.46146146146146	80.2859392892744\\
1.46646646646647	80.6988150778329\\
1.47147147147147	81.1115712037619\\
1.47647647647648	81.5241934083721\\
1.48148148148148	81.9366674772998\\
1.48648648648649	82.3489792405069\\
1.49149149149149	82.7611145722809\\
1.4964964964965	83.173059391235\\
1.5015015015015	83.5847996603077\\
1.50650650650651	83.9963213867635\\
1.51151151151151	84.4076106221923\\
1.51651651651652	84.8186534625094\\
1.52152152152152	85.2294360479562\\
1.52652652652653	85.6399445630992\\
1.53153153153153	86.0501652368308\\
1.53653653653654	86.4600843423689\\
1.54154154154154	86.869688197257\\
1.54654654654655	87.2789631633642\\
1.55155155155155	87.6878956468853\\
1.55655655655656	88.0964720983405\\
1.56156156156156	88.5046790125759\\
1.56656656656657	88.9125029287629\\
1.57157157157157	89.3199304303987\\
1.57657657657658	89.726948145306\\
1.58158158158158	90.133542745633\\
1.58658658658659	90.5397009478538\\
1.59159159159159	90.9454095127678\\
1.5965965965966	91.3506552455003\\
1.6016016016016	91.7554249955019\\
1.60660660660661	92.159705656549\\
1.61161161161161	92.5634841667435\\
1.61661661661662	92.966747508513\\
1.62162162162162	93.3694827086106\\
1.62662662662663	93.771676838115\\
1.63163163163163	94.1733170124307\\
1.63663663663664	94.5743903912874\\
1.64164164164164	94.9748841787409\\
1.64664664664665	95.3747856231721\\
1.65165165165165	95.774082017288\\
1.65665665665666	96.172760580929\\
1.66166166166166	96.5708072374998\\
1.66666666666667	96.968210078435\\
1.67167167167167	97.364958053496\\
1.67667667667668	97.7610402220016\\
1.68168168168168	98.1564457528283\\
1.68668668668669	98.5511639244099\\
1.69169169169169	98.9451841247381\\
1.6966966966967	99.3384958513619\\
1.7017017017017	99.731088711388\\
1.70670670670671	100.12295242148\\
1.71171171171171	100.514076807861\\
1.71671671671672	100.904451806309\\
1.72172172172172	101.294067462161\\
1.72672672672673	101.682913930311\\
1.73173173173173	102.070981475212\\
1.73673673673674	102.458260470873\\
1.74174174174174	102.844741400861\\
1.74674674674675	103.230414858301\\
1.75175175175175	103.615271545873\\
1.75675675675676	103.99930227582\\
1.76176176176176	104.382497969936\\
1.76676676676677	104.764849659578\\
1.77177177177177	105.146348485657\\
1.77677677677678	105.526985698644\\
1.78178178178178	105.906752658565\\
1.78678678678679	106.285640835007\\
1.79179179179179	106.66364180711\\
1.7967967967968	107.040747263576\\
1.8018018018018	107.416949002661\\
1.80680680680681	107.792238932182\\
1.81181181181181	108.16660906951\\
1.81681681681682	108.540051541576\\
1.82182182182182	108.912558584868\\
1.82682682682683	109.284122545431\\
1.83183183183183	109.654735878869\\
1.83683683683684	110.02439115034\\
1.84184184184184	110.393081034564\\
1.84684684684685	110.760798315816\\
1.85185185185185	111.12753588793\\
1.85685685685686	111.493286754294\\
1.86186186186186	111.858044027859\\
1.86686686686687	112.22180093113\\
1.87187187187187	112.584550796169\\
1.87687687687688	112.946287064599\\
1.88188188188188	113.307003287596\\
1.88688688688689	113.666693125898\\
1.89189189189189	114.025350349797\\
1.8968968968969	114.382968839145\\
1.9019019019019	114.73954258335\\
1.90690690690691	115.095065681378\\
1.91191191191191	115.449532341754\\
1.91691691691692	115.802936882557\\
1.92192192192192	116.155273731428\\
1.92692692692693	116.506537425562\\
1.93193193193193	116.856722611714\\
1.93693693693694	117.205824046194\\
1.94194194194194	117.553836594872\\
1.94694694694695	117.900755233174\\
1.95195195195195	118.246575046085\\
1.95695695695696	118.591291228145\\
1.96196196196196	118.934899083455\\
1.96696696696697	119.27739402567\\
1.97197197197197	119.618771578005\\
1.97697697697698	119.959027373233\\
1.98198198198198	120.298157153682\\
1.98698698698699	120.636156771238\\
1.99199199199199	120.973022187348\\
1.996996996997	121.308749473012\\
2.002002002002	121.643334808791\\
2.00700700700701	121.976774484801\\
2.01201201201201	122.309064900717\\
2.01701701701702	122.640202565772\\
2.02202202202202	122.970184098754\\
2.02702702702703	123.299006228012\\
2.03203203203203	123.62666579145\\
2.03703703703704	123.953159736531\\
2.04204204204204	124.278485120275\\
2.04704704704705	124.602639109258\\
2.05205205205205	124.925618979617\\
2.05705705705706	125.247422117044\\
2.06206206206206	125.568046016788\\
2.06706706706707	125.887488283659\\
2.07207207207207	126.20574663202\\
2.07707707707708	126.522818885795\\
2.08208208208208	126.838702978464\\
2.08708708708709	127.153396953066\\
2.09209209209209	127.466898962194\\
2.0970970970971	127.779207268004\\
2.1021021021021	128.090320242204\\
2.10710710710711	128.400236366063\\
2.11211211211211	128.708954230408\\
2.11711711711712	129.01647253562\\
2.12212212212212	129.322790091641\\
2.12712712712713	129.627905817969\\
2.13213213213213	129.93181874366\\
2.13713713713714	130.234528007328\\
2.14214214214214	130.536032857143\\
2.14714714714715	130.836332650833\\
2.15215215215215	131.135426855685\\
2.15715715715716	131.433314023306\\
2.16216216216216	131.729984604422\\
2.16716716716717	132.025436764744\\
2.17217217217217	132.319671058241\\
2.17717717717718	132.612688072565\\
2.18218218218218	132.904488429056\\
2.18718718718719	133.195072782739\\
2.19219219219219	133.484441822326\\
2.1971971971972	133.772596270218\\
2.2022022022022	134.059536882498\\
2.20720720720721	134.345264448939\\
2.21221221221221	134.629779793\\
2.21721721721722	134.913083771825\\
2.22222222222222	135.195177276246\\
2.22722722722723	135.47606123078\\
2.23223223223223	135.755736593631\\
2.23723723723724	136.034204356691\\
2.24224224224224	136.311465545537\\
2.24724724724725	136.587521219433\\
2.25225225225225	136.862372471328\\
2.25725725725726	137.13602042786\\
2.26226226226226	137.408466249352\\
2.26726726726727	137.679711129813\\
2.27227227227227	137.949756296941\\
2.27727727727728	138.218603012116\\
2.28228228228228	138.48625257041\\
2.28728728728729	138.752706300577\\
2.29229229229229	139.01796556506\\
2.2972972972973	139.282031759987\\
2.3023023023023	139.544906315173\\
2.30730730730731	139.80659069412\\
2.31231231231231	140.067086394017\\
2.31731731731732	140.326394945737\\
2.32232232232232	140.584517913843\\
2.32732732732733	140.841456896581\\
2.33233233233233	141.097213525885\\
2.33733733733734	141.351789467377\\
2.34234234234234	141.605186420363\\
2.34734734734735	141.857406117836\\
2.35235235235235	142.108450326478\\
2.35735735735736	142.358320846654\\
2.36236236236236	142.607019512417\\
2.36736736736737	142.854548191507\\
2.37237237237237	143.10090878535\\
2.37737737737738	143.346103229058\\
2.38238238238238	143.59013349143\\
2.38738738738739	143.833001574952\\
2.39239239239239	144.074709515796\\
2.3973973973974	144.31525938382\\
2.4024024024024	144.554653282569\\
2.40740740740741	144.792893349274\\
2.41241241241241	145.029981754853\\
2.41741741741742	145.265920703911\\
2.42242242242242	145.500712434739\\
2.42742742742743	145.734359219314\\
2.43243243243243	145.966863363299\\
2.43743743743744	146.198227206045\\
2.44244244244244	146.428453120589\\
2.44744744744745	146.657543513655\\
2.45245245245245	146.885500825651\\
2.45745745745746	147.112327530675\\
2.46246246246246	147.338026136509\\
2.46746746746747	147.562599184622\\
2.47247247247247	147.78604925017\\
2.47747747747748	148.008378941995\\
2.48248248248248	148.229590902626\\
2.48748748748749	148.449687808279\\
2.49249249249249	148.668672368854\\
2.4974974974975	148.886547327941\\
2.5025025025025	149.103315462813\\
2.50750750750751	149.318979584432\\
2.51251251251251	149.533542537446\\
2.51751751751752	149.747007200189\\
2.52252252252252	149.959376484681\\
2.52752752752753	150.170653336629\\
2.53253253253253	150.380840735428\\
2.53753753753754	150.589941694156\\
2.54254254254254	150.797959259582\\
2.54754754754755	151.004896512157\\
2.55255255255255	151.210756566022\\
2.55755755755756	151.415542569002\\
2.56256256256256	151.61925770261\\
2.56756756756757	151.821905182045\\
2.57257257257257	152.023488256192\\
2.57757757757758	152.224010207624\\
2.58258258258258	152.423474352599\\
2.58758758758759	152.621884041062\\
2.59259259259259	152.819242656644\\
2.5975975975976	153.015553616664\\
2.6026026026026	153.210820372126\\
2.60760760760761	153.40504640772\\
2.61261261261261	153.598235241824\\
2.61761761761762	153.790390426502\\
2.62262262262262	153.981515547505\\
2.62762762762763	154.171614224269\\
2.63263263263263	154.360690109918\\
2.63763763763764	154.548746891261\\
2.64264264264264	154.735788288795\\
2.64764764764765	154.921818056703\\
2.65265265265265	155.106839982854\\
2.65765765765766	155.290857895504\\
2.66266266266266	155.473875682332\\
2.66766766766767	155.655897200219\\
2.67267267267267	155.836926296659\\
2.67767767767768	156.016966816565\\
2.68268268268268	156.196022602273\\
2.68768768768769	156.374097493536\\
2.69269269269269	156.551195327529\\
2.6976976976977	156.727319938847\\
2.7027027027027	156.902475159506\\
2.70770770770771	157.07666481894\\
2.71271271271271	157.249892744005\\
2.71771771771772	157.422162758978\\
2.72272272272272	157.593478685554\\
2.72772772772773	157.763844342851\\
2.73273273273273	157.933263547404\\
2.73773773773774	158.101740113172\\
2.74274274274274	158.26927785153\\
2.74774774774775	158.435880571278\\
2.75275275275275	158.601552078633\\
2.75775775775776	158.766296177232\\
2.76276276276276	158.930116668135\\
2.76776776776777	159.093017349819\\
2.77277277277277	159.255002018185\\
2.77777777777778	159.41607446655\\
2.78278278278278	159.576238485654\\
2.78778778778779	159.735497863658\\
2.79279279279279	159.893856386141\\
2.7977977977978	160.051317836103\\
2.8028028028028	160.207885993965\\
2.80780780780781	160.363564637567\\
2.81281281281281	160.518357542171\\
2.81781781781782	160.672268480458\\
2.82282282282282	160.825301222529\\
2.82782782782783	160.977459535907\\
2.83283283283283	161.128747185532\\
2.83783783783784	161.279167933769\\
2.84284284284284	161.428725540398\\
2.84784784784785	161.577423762624\\
2.85285285285285	161.725266355069\\
2.85785785785786	161.872257069776\\
2.86286286286286	162.01839965621\\
2.86786786786787	162.163697861254\\
2.87287287287287	162.308155429212\\
2.87787787787788	162.451776101809\\
2.88288288288288	162.59456361819\\
2.88788788788789	162.73652171492\\
2.89289289289289	162.877654125983\\
2.8978978978979	163.017964582785\\
2.9029029029029	163.157456814153\\
2.90790790790791	163.296134546332\\
2.91291291291291	163.434001502988\\
2.91791791791792	163.571061405209\\
2.92292292292292	163.7073179715\\
2.92792792792793	163.842774917789\\
2.93293293293293	163.977435957424\\
2.93793793793794	164.111304801172\\
2.94294294294294	164.24438515722\\
2.94794794794795	164.376680731178\\
2.95295295295295	164.508195226073\\
2.95795795795796	164.638932342354\\
2.96296296296296	164.76889577789\\
2.96796796796797	164.89808922797\\
2.97297297297297	165.026516385305\\
2.97797797797798	165.154180940023\\
2.98298298298298	165.281086579674\\
2.98798798798799	165.407236989229\\
2.99299299299299	165.532635851079\\
2.997997997998	165.657286845034\\
3.003003003003	165.781193648325\\
3.00800800800801	165.904359935603\\
3.01301301301301	166.026789378941\\
3.01801801801802	166.14848564783\\
3.02302302302302	166.269452409182\\
3.02802802802803	166.389693327329\\
3.03303303303303	166.509212064025\\
3.03803803803804	166.628012278442\\
3.04304304304304	166.746097627173\\
3.04804804804805	166.863471764231\\
3.05305305305305	166.980138341052\\
3.05805805805806	167.096101006487\\
3.06306306306306	167.211363406813\\
3.06806806806807	167.325929185722\\
3.07307307307307	167.439801984331\\
3.07807807807808	167.552985441173\\
3.08308308308308	167.665483192205\\
3.08808808808809	167.777298870801\\
3.09309309309309	167.888436107758\\
3.0980980980981	167.998898531292\\
3.1031031031031	168.108689767039\\
3.10810810810811	168.217813438056\\
3.11311311311311	168.32627316482\\
3.11811811811812	168.434072565227\\
3.12312312312312	168.541215254595\\
3.12812812812813	168.647704845662\\
3.13313313313313	168.753544948586\\
3.13813813813814	168.858739170945\\
3.14314314314314	168.963291117738\\
3.14814814814815	169.067204391382\\
3.15315315315315	169.170482591718\\
3.15815815815816	169.273129465928\\
3.16316316316316	169.375149050877\\
3.16816816816817	169.476544604815\\
3.17317317317317	169.577319300362\\
3.17817817817818	169.677476301949\\
3.18318318318318	169.777018765821\\
3.18818818818819	169.875949840032\\
3.19319319319319	169.97427266445\\
3.1981981981982	170.071990370754\\
3.2032032032032	170.169106082434\\
3.20820820820821	170.265622914795\\
3.21321321321321	170.361543974951\\
3.21821821821822	170.45687236183\\
3.22322322322322	170.551611166169\\
3.22822822822823	170.64576347052\\
3.23323323323323	170.739332349246\\
3.23823823823824	170.832320868521\\
3.24324324324324	170.924732086332\\
3.24824824824825	171.016569052478\\
3.25325325325325	171.107834808568\\
3.25825825825826	171.198532388027\\
3.26326326326326	171.288664816087\\
3.26826826826827	171.378235109795\\
3.27327327327327	171.467246278009\\
3.27827827827828	171.5557013214\\
3.28328328328328	171.643603232449\\
3.28828828828829	171.730954995452\\
3.29329329329329	171.817759586512\\
3.2982982982983	171.90401997355\\
3.3033033033033	171.989739116293\\
3.30830830830831	172.074919966285\\
3.31331331331331	172.159565466878\\
3.31831831831832	172.243678553239\\
3.32332332332332	172.327262152345\\
3.32832832832833	172.410319182985\\
3.33333333333333	172.492852555761\\
3.33833833833834	172.574865173086\\
3.34334334334334	172.656359929186\\
3.34834834834835	172.737339710098\\
3.35335335335335	172.817807393671\\
3.35835835835836	172.897765849566\\
3.36336336336336	172.977217939255\\
3.36836836836837	173.056166516025\\
3.37337337337337	173.134614424972\\
3.37837837837838	173.212564503005\\
3.38338338338338	173.290019578844\\
3.38838838838839	173.366982473022\\
3.39339339339339	173.443455997884\\
3.3983983983984	173.519442957587\\
3.4034034034034	173.594946148098\\
3.40840840840841	173.669968357199\\
3.41341341341341	173.744512364482\\
3.41841841841842	173.81858094135\\
3.42342342342342	173.892176851021\\
3.42842842842843	173.965302848522\\
3.43343343343343	174.037961680694\\
3.43843843843844	174.110156086188\\
3.44344344344344	174.181888795469\\
3.44844844844845	174.253162530812\\
3.45345345345345	174.323980006305\\
3.45845845845846	174.394343927849\\
3.46346346346346	174.464256993154\\
3.46846846846847	174.533721891744\\
3.47347347347347	174.602741304955\\
3.47847847847848	174.671317905935\\
3.48348348348348	174.739454359642\\
3.48848848848849	174.807153322849\\
3.49349349349349	174.874417444138\\
3.4984984984985	174.941249363904\\
3.5035035035035	175.007651714356\\
3.50850850850851	175.073627119511\\
3.51351351351351	175.139178195202\\
3.51851851851852	175.20430754907\\
3.52352352352352	175.269017780572\\
3.52852852852853	175.333311480973\\
3.53353353353353	175.397191233352\\
3.53853853853854	175.460659612601\\
3.54354354354354	175.523719185421\\
3.54854854854855	175.586372510328\\
3.55355355355355	175.648622137647\\
3.55855855855856	175.710470609518\\
3.56356356356356	175.77192045989\\
3.56856856856857	175.832974214527\\
3.57357357357357	175.893634391001\\
3.57857857857858	175.9539034987\\
3.58358358358358	176.013784038821\\
3.58858858858859	176.073278504375\\
3.59359359359359	176.132389380183\\
3.5985985985986	176.191119142879\\
3.6036036036036	176.24947026091\\
3.60860860860861	176.307445194533\\
3.61361361361361	176.365046395817\\
3.61861861861862	176.422276308644\\
3.62362362362362	176.479137368708\\
3.62862862862863	176.535632003515\\
3.63363363363363	176.591762632381\\
3.63863863863864	176.647531666436\\
3.64364364364364	176.702941508622\\
3.64864864864865	176.757994553691\\
3.65365365365365	176.812693188209\\
3.65865865865866	176.867039660707\\
3.66366366366366	176.921035770971\\
3.66866866866867	176.974683461473\\
3.67367367367367	177.027984686383\\
3.67867867867868	177.080941393789\\
3.68368368368368	177.133555525691\\
3.68868868868869	177.185829018002\\
3.69369369369369	177.237763800551\\
3.6986986986987	177.289361797079\\
3.7037037037037	177.340624925242\\
3.70870870870871	177.391555096609\\
3.71371371371371	177.442154216662\\
3.71871871871872	177.4924241848\\
3.72372372372372	177.542366894333\\
3.72872872872873	177.591984232486\\
3.73373373373373	177.641278080396\\
3.73873873873874	177.690250313117\\
3.74374374374374	177.738902799614\\
3.74874874874875	177.787237402767\\
3.75375375375375	177.835255979371\\
3.75875875875876	177.882960380132\\
3.76376376376376	177.930352449673\\
3.76876876876877	177.977434026528\\
3.77377377377377	178.024206943146\\
3.77877877877878	178.070673025891\\
3.78378378378378	178.11683409504\\
3.78878878878879	178.162691964782\\
3.79379379379379	178.208248443222\\
3.7987987987988	178.253505332379\\
3.8038038038038	178.298464428184\\
3.80880880880881	178.343127520484\\
3.81381381381381	178.387496393038\\
3.81881881881882	178.43157282352\\
3.82382382382382	178.475358583518\\
3.82882882882883	178.518855438531\\
3.83383383383383	178.562065147977\\
3.83883883883884	178.604989465182\\
3.84384384384384	178.647630137391\\
3.84884884884885	178.689988905759\\
3.85385385385385	178.732067505358\\
3.85885885885886	178.77386766517\\
3.86386386386386	178.815391108095\\
3.86886886886887	178.856639550944\\
3.87387387387387	178.897614704444\\
3.87887887887888	178.938318273232\\
3.88388388388388	178.978751955863\\
3.88888888888889	179.018917444804\\
3.89389389389389	179.058816426437\\
3.8988988988989	179.098450581054\\
3.9039039039039	179.137821582867\\
3.90890890890891	179.176931099997\\
3.91391391391391	179.215780794479\\
3.91891891891892	179.254372322266\\
3.92392392392392	179.29270733322\\
3.92892892892893	179.330787471119\\
3.93393393393393	179.368614373655\\
3.93893893893894	179.406189672434\\
3.94394394394394	179.443514992975\\
3.94894894894895	179.480591954711\\
3.95395395395395	179.517422170988\\
3.95895895895896	179.554007249069\\
3.96396396396396	179.590348790127\\
3.96896896896897	179.62644838925\\
3.97397397397397	179.662307635442\\
3.97897897897898	179.697928111619\\
3.98398398398398	179.733311394609\\
3.98898898898899	179.768459055158\\
3.99399399399399	179.803372657922\\
3.998998998999	179.838053761474\\
4.004004004004	179.872503918297\\
4.00900900900901	179.906724674793\\
4.01401401401401	179.940717571273\\
4.01901901901902	179.974484141964\\
4.02402402402402	180.008025915007\\
4.02902902902903	180.041344412456\\
4.03403403403403	180.07444115028\\
4.03903903903904	180.107317638361\\
4.04404404404404	180.139975380494\\
4.04904904904905	180.172415874389\\
4.05405405405405	180.20464061167\\
4.05905905905906	180.236651077875\\
4.06406406406406	180.268448752454\\
4.06906906906907	180.300035108773\\
4.07407407407407	180.33141161411\\
4.07907907907908	180.362579729658\\
4.08408408408408	180.393540910525\\
4.08908908908909	180.424296605729\\
4.09409409409409	180.454848258206\\
4.0990990990991	180.485197304803\\
4.1041041041041	180.515345176283\\
4.10910910910911	180.545293297321\\
4.11411411411411	180.575043086506\\
4.11911911911912	180.604595956342\\
4.12412412412412	180.633953313247\\
4.12912912912913	180.663116557551\\
4.13413413413413	180.692087083498\\
4.13913913913914	180.720866279249\\
4.14414414414414	180.749455526875\\
4.14914914914915	180.777856202362\\
4.15415415415415	180.806069675611\\
4.15915915915916	180.834097057483\\
4.16416416416416	180.861939037799\\
4.16916916916917	180.889596685113\\
4.17417417417417	180.917071078391\\
4.17917917917918	180.944363293013\\
4.18418418418418	180.971474400778\\
4.18918918918919	180.9984054699\\
4.19419419419419	181.025157565008\\
4.1991991991992	181.051731747147\\
4.2042042042042	181.07812907378\\
4.20920920920921	181.104350598785\\
4.21421421421421	181.130397372455\\
4.21921921921922	181.156270441501\\
4.22422422422422	181.181970849047\\
4.22922922922923	181.207499634637\\
4.23423423423423	181.232857834228\\
4.23923923923924	181.258046480195\\
4.24424424424424	181.283066601327\\
4.24924924924925	181.307919222831\\
4.25425425425425	181.332605366329\\
4.25925925925926	181.35712604986\\
4.26426426426426	181.381482287877\\
4.26926926926927	181.40567509125\\
4.27427427427427	181.429705467267\\
4.27927927927928	181.45357441963\\
4.28428428428428	181.477282948457\\
4.28928928928929	181.500832050283\\
4.29429429429429	181.524222718058\\
4.2992992992993	181.547455941149\\
4.3043043043043	181.570532705338\\
4.30930930930931	181.593453992825\\
4.31431431431431	181.616220782224\\
4.31931931931932	181.638834048565\\
4.32432432432432	181.661294763296\\
4.32932932932933	181.683603894279\\
4.33433433433433	181.705762405793\\
4.33933933933934	181.727771258534\\
4.34434434434434	181.749631409612\\
4.34934934934935	181.771343812554\\
4.35435435435435	181.792909417303\\
4.35935935935936	181.814329170219\\
4.36436436436436	181.835604014076\\
4.36936936936937	181.856734888066\\
4.37437437437437	181.877722727796\\
4.37937937937938	181.898568465289\\
4.38438438438438	181.919273028985\\
4.38938938938939	181.939837343738\\
4.39439439439439	181.960262330821\\
4.3993993993994	181.980548907921\\
4.4044044044044	182.00069798914\\
4.40940940940941	182.020710485\\
4.41441441441441	182.040587302435\\
4.41941941941942	182.060329344796\\
4.42442442442442	182.079937511853\\
4.42942942942943	182.099412699787\\
4.43443443443443	182.1187558012\\
4.43943943943944	182.137967705106\\
4.44444444444444	182.157049296938\\
4.44944944944945	182.176001458543\\
4.45445445445445	182.194825068186\\
4.45945945945946	182.213521000546\\
4.46446446446446	182.232090126719\\
4.46946946946947	182.250533314218\\
4.47447447447447	182.26885142697\\
4.47947947947948	182.287045325319\\
4.48448448448448	182.305115866026\\
4.48948948948949	182.323063902267\\
4.49449449449449	182.340890283634\\
4.4994994994995	182.358595856136\\
4.5045045045045	182.376181462196\\
4.50950950950951	182.393647940655\\
4.51451451451451	182.41099612677\\
4.51951951951952	182.428226852212\\
4.52452452452452	182.445340945071\\
4.52952952952953	182.462339229851\\
4.53453453453453	182.479222527472\\
4.53953953953954	182.495991655272\\
4.54454454454454	182.512647427002\\
4.54954954954955	182.529190652832\\
4.55455455455455	182.545622139346\\
4.55955955955956	182.561942689545\\
4.56456456456456	182.578153102846\\
4.56956956956957	182.594254175082\\
4.57457457457457	182.610246698501\\
4.57957957957958	182.626131461769\\
4.58458458458458	182.641909249966\\
4.58958958958959	182.65758084459\\
4.59459459459459	182.673147023554\\
4.5995995995996	182.688608561186\\
4.6046046046046	182.703966228233\\
4.60960960960961	182.719220791854\\
4.61461461461461	182.734373015628\\
4.61961961961962	182.749423659548\\
4.62462462462462	182.764373480023\\
4.62962962962963	182.779223229878\\
4.63463463463463	182.793973658355\\
4.63963963963964	182.808625511111\\
4.64464464464464	182.82317953022\\
4.64964964964965	182.837636454172\\
4.65465465465465	182.851997017871\\
4.65965965965966	182.866261405545\\
4.66466466466466	182.88042948122\\
4.66966966966967	182.894501852179\\
4.67467467467467	182.908479129517\\
4.67967967967968	182.922361921564\\
4.68468468468468	182.936150833889\\
4.68968968968969	182.9498464693\\
4.69469469469469	182.963449427843\\
4.6996996996997	182.976960306801\\
4.7047047047047	182.990379700699\\
4.70970970970971	183.003708201296\\
4.71471471471471	183.016946397593\\
4.71971971971972	183.030094875828\\
4.72472472472472	183.043154219477\\
4.72972972972973	183.056125009254\\
4.73473473473473	183.069007823113\\
4.73973973973974	183.081803236245\\
4.74474474474474	183.09451182108\\
4.74974974974975	183.107134147286\\
4.75475475475475	183.11967078177\\
4.75975975975976	183.132122288677\\
4.76476476476476	183.144489229391\\
4.76976976976977	183.156772162532\\
4.77477477477477	183.168971643962\\
4.77977977977978	183.181088226778\\
4.78478478478478	183.193122461319\\
4.78978978978979	183.205074895158\\
4.79479479479479	183.216946073111\\
4.7997997997998	183.228736537228\\
4.8048048048048	183.240446826801\\
4.80980980980981	183.252077478358\\
4.81481481481481	183.263629025667\\
4.81981981981982	183.275101999733\\
4.82482482482482	183.2864969288\\
4.82982982982983	183.297814338351\\
4.83483483483483	183.309054751107\\
4.83983983983984	183.320218687026\\
4.84484484484484	183.331306663306\\
4.84984984984985	183.342319194384\\
4.85485485485485	183.353256791933\\
4.85985985985986	183.364119964867\\
4.86486486486486	183.374909219336\\
4.86986986986987	183.38562505873\\
4.87487487487487	183.396267983677\\
4.87987987987988	183.406838492043\\
4.88488488488488	183.417337078933\\
4.88988988988989	183.42776423669\\
4.89489489489489	183.438120454895\\
4.8998998998999	183.448406220368\\
4.9049049049049	183.458622017168\\
4.90990990990991	183.46876832659\\
4.91491491491491	183.478845627171\\
4.91991991991992	183.488854394682\\
4.92492492492492	183.498795102137\\
4.92992992992993	183.508668219785\\
4.93493493493493	183.518474215114\\
4.93993993993994	183.528213552853\\
4.94494494494494	183.537886694965\\
4.94994994994995	183.547494100654\\
4.95495495495495	183.557036226364\\
4.95995995995996	183.566513525773\\
4.96496496496496	183.575926449801\\
4.96996996996997	183.585275446606\\
4.97497497497497	183.594560961582\\
4.97997997997998	183.603783437364\\
4.98498498498498	183.612943313824\\
4.98998998998999	183.622041028074\\
4.99499499499499	183.631077014461\\
5	183.640051704574\\
};
\addlegendentry{Rimossi}

\end{axis}
\end{tikzpicture}%}
\subfloat[][]{% This file was created by matlab2tikz.
%
%The latest updates can be retrieved from
%  http://www.mathworks.com/matlabcentral/fileexchange/22022-matlab2tikz-matlab2tikz
%where you can also make suggestions and rate matlab2tikz.
%
\definecolor{mycolor1}{rgb}{0.00000,0.44700,0.74100}%
\definecolor{mycolor2}{rgb}{0.85000,0.32500,0.09800}%
%
\begin{tikzpicture}

\begin{axis}[%
width=0.39\columnwidth,
height=1.8in,
at={(1.011in,0.642in)},
scale only axis,
xmin=60,
xmax=260,
xlabel style={font=\color{white!15!black}},
xlabel={S},
ymin=0,
ymax=35,
ylabel style={font=\color{white!15!black}},
ylabel={I},
axis background/.style={fill=white},
axis x line*=bottom,
axis y line*=left,
legend style={at={(axis cs:110,1)},anchor=south west,legend cell align=left, align=left, draw=none,fill=none}
]
\addplot [color=mycolor1, only marks, mark=asterisk, mark options={solid, mycolor1}]
  table[row sep=crcr]{%
235	14.5\\
201	22\\
153.5	29\\
121	20\\
108	8\\
97	8\\
83	0\\
};
\addlegendentry{Realt\`a}

\addplot [color=mycolor2, line width=2.0pt]
  table[row sep=crcr]{%
254	7\\
253.840937647464	7.06298691984465\\
253.680546129765	7.12643919517239\\
253.51881756491	7.19035833906985\\
253.355744085814	7.25474582965249\\
253.191317841079	7.31960310933302\\
253.025530995004	7.38493158482141\\
252.858375789304	7.45073257836408\\
252.689844438122	7.51700740673299\\
252.519929104995	7.5837574039862\\
252.348621991913	7.65098385396419\\
252.175915339314	7.71868799028986\\
252.001801426083	7.78687099636853\\
251.826272569559	7.85553400538797\\
251.649321125527	7.92467810031835\\
251.470939488222	7.99430431391229\\
251.291120090331	8.06441362870481\\
251.109855402987	8.13500697701339\\
250.927137935775	8.2060852409379\\
250.742960236727	8.27764925236066\\
250.557314892327	8.3496997929464\\
250.370194527507	8.4222375941423\\
250.181591805649	8.49526333717794\\
249.991499428584	8.56877765306534\\
249.799910136592	8.64278112259895\\
249.606816708404	8.71727427635563\\
249.412211961199	8.79225759469467\\
249.216088750607	8.86773150775781\\
249.018439970705	8.94369639546918\\
248.819258554022	9.02015258753537\\
248.618537471535	9.09710036344537\\
248.41626973267	9.17453995247061\\
248.212459056064	9.25246721375046\\
248.007107747517	9.330878532132\\
247.800206857156	9.40977480350429\\
247.591747609933	9.48915681138211\\
247.381721405627	9.56902522690594\\
247.170119818838	9.64938060884197\\
246.956934598992	9.73022340358207\\
246.742157670341	9.81155394514385\\
246.52578113196	9.8933724551706\\
246.307797257747	9.97567904293132\\
246.088198496428	10.0584737053207\\
245.866977471551	10.1417563268592\\
245.644126981489	10.2255266796929\\
245.41963999944	10.3097844235936\\
245.193509673427	10.3945291059589\\
244.965729326295	10.4797601618119\\
244.736292455717	10.5654769138017\\
244.505192734189	10.6516785722029\\
244.27242400903	10.7383642349157\\
244.037980302386	10.8255328874663\\
243.801855811225	10.9131834030064\\
243.564044907343	11.0013145423134\\
243.324542137357	11.0899249537906\\
243.083342222711	11.1790131734667\\
242.840440059671	11.2685776249965\\
242.59583071933	11.35861661966\\
242.349509447605	11.4491283563634\\
242.101471665236	11.5401109216383\\
241.851712967789	11.6315622896421\\
241.600229125655	11.7234803221579\\
241.347016084047	11.8158627685945\\
241.092069963006	11.9087072659865\\
240.835387057393	12.0020113389939\\
240.576963836899	12.0957723999027\\
240.316796946035	12.1899877486246\\
240.054883204139	12.2846545726968\\
239.791219605373	12.3797699472825\\
239.525803318722	12.4753308351703\\
239.258631687998	12.5713340867746\\
238.989702231836	12.6677764401357\\
238.719012643696	12.7646545209193\\
238.446560791861	12.8619648424171\\
238.172344719442	12.9597038055462\\
237.896362644371	13.0578676988496\\
237.618612959406	13.156452698496\\
237.33909423213	13.2554548682798\\
237.05780520495	13.354870159621\\
236.774744795097	13.4546944115655\\
236.489912094627	13.5549233507846\\
236.203306370421	13.6555525915757\\
235.914927064184	13.7565776358615\\
235.624773792445	13.8579938731908\\
235.332846346558	13.9597965807377\\
235.039144692703	14.0619809233023\\
234.743668971881	14.1645419533103\\
234.446419499922	14.2674746108132\\
234.147396767476	14.370773723488\\
233.846601440022	14.4744340066376\\
233.544034357859	14.5784500631905\\
233.239696536114	14.6828163837009\\
232.933589164737	14.7875273463489\\
232.625713608502	14.8925772169399\\
232.31607140701	14.9979601489055\\
232.004664274683	15.1036701833026\\
231.69149410077	15.2097012488139\\
231.376562949345	15.3160471617481\\
231.059873059304	15.4227016260392\\
230.741426844369	15.5296582332471\\
230.421226893088	15.6369104625573\\
230.09927596883	15.7444516807813\\
229.775577009791	15.8522751423559\\
229.450133128991	15.9603739893438\\
229.122947614275	16.0687412514335\\
228.794023928312	16.177369845939\\
228.463365708595	16.2862525778001\\
228.130976767442	16.3953821395825\\
227.796861091996	16.5047511114772\\
227.461022844225	16.6143519613012\\
227.123466360919	16.7241770444972\\
226.784196153695	16.8342186041334\\
226.443216908994	16.9444687709039\\
226.10053348808	17.0549195631284\\
225.756150927044	17.1655628867524\\
225.4100744368	17.276390535347\\
225.062309403086	17.387394190109\\
224.712861386466	17.4985654198612\\
224.361736122328	17.6098956810516\\
224.008939520883	17.7213763177542\\
223.65447766717	17.8329985616689\\
223.298356821048	17.9447535321208\\
222.940583417205	18.0566322360611\\
222.58116406515	18.1686255680666\\
222.220105549219	18.2807243103397\\
221.85741482857	18.3929191327088\\
221.493099037188	18.5052005926276\\
221.127165483882	18.6175591351757\\
220.759621652284	18.7299850930586\\
220.390475200852	18.8424686866071\\
220.019733962868	18.9550000237781\\
219.647405840782	19.0675692529414\\
219.273487420567	19.1801837331534\\
218.897982213279	19.2928432940739\\
218.520902989138	19.4055350473286\\
218.142262632548	19.5182461426963\\
217.762074142097	19.6309637681091\\
217.380350630557	19.7436751496524\\
216.997105324881	19.8563675515643\\
216.612351566211	19.9690282762366\\
216.226102809867	20.0816446642137\\
215.838372625357	20.1942040941936\\
215.44917469637	20.3066939830273\\
215.058522820782	20.4191017857187\\
214.666430910649	20.5314149954253\\
214.272912992213	20.6436211434573\\
213.877983205899	20.7557077992785\\
213.481655806316	20.8676625705055\\
213.083945162257	20.9794731029081\\
212.684865756699	21.0911270804094\\
212.284432186801	21.2026122250855\\
211.882659163907	21.3139162971658\\
211.479561513546	21.4250270950327\\
211.075154175428	21.5359324552219\\
210.66945220345	21.6466202524221\\
210.262470765689	21.7570783994751\\
209.854225144409	21.8672948473762\\
209.444730736056	21.9772575852734\\
209.034003051261	22.0869546404681\\
208.622057714837	22.1963740784149\\
208.208910465782	22.3055040027214\\
207.794577157278	22.4143325551485\\
207.379073756689	22.52284791561\\
206.962416345566	22.6310383021731\\
206.54462111964	22.738891971058\\
206.125704388828	22.8463972166382\\
205.70568257723	22.9535423714402\\
205.284572223131	23.0603158061438\\
204.862389978998	23.1667059295817\\
204.439152611482	23.2727011887399\\
204.01487700142	23.3782900687577\\
203.589580143829	23.4834610929273\\
203.163279147913	23.5882028226943\\
202.735991237058	23.6925038576571\\
202.307733748834	23.7963528355675\\
201.878524134996	23.8997384323305\\
201.448379961481	24.0026493620041\\
201.01731890841	24.1050743767995\\
200.58535877009	24.2070022670811\\
200.152517455008	24.3084218613663\\
199.718812985838	24.4093220263258\\
199.284263499437	24.5096916667834\\
198.848887246844	24.609519725716\\
198.412702593283	24.7087951842538\\
197.975728018163	24.80750706168\\
197.537982115075	24.905644415431\\
197.099483591794	25.0031963410963\\
196.660251270279	25.1001519724187\\
196.220304086673	25.196500481294\\
195.779661091303	25.2922310777711\\
195.338341448678	25.3873330100522\\
194.896364437493	25.4817955644927\\
194.453749450626	25.5756080656009\\
194.010515995137	25.6687598760385\\
193.566683692274	25.7612403966202\\
193.122272277463	25.8530390663139\\
192.677301600319	25.9441453622406\\
192.231791624638	26.0345487996746\\
191.7857624284	26.1242389320431\\
191.339234203769	26.2132053509267\\
190.892227257094	26.301437686059\\
190.444762008904	26.3889256053267\\
189.996858993917	26.4756588147699\\
189.54853886103	26.5616270585816\\
189.099822373327	26.6468201191081\\
188.650730408074	26.7312278168487\\
188.201283956722	26.814840010456\\
187.751504124905	26.8976465967356\\
187.30141213244	26.9796375106465\\
186.851029313328	27.0608027253005\\
186.400377115757	27.1411322519629\\
185.949477102093	27.2206161400519\\
185.498350948891	27.299244477139\\
185.047020446887	27.3770073889486\\
184.595507501001	27.4538950393587\\
184.143834130337	27.5298976303999\\
183.692022468183	27.6050054022565\\
183.24009476201	27.6792086332655\\
182.788073373475	27.7524976399173\\
182.335980778415	27.8248627768554\\
181.883839566854	27.8962944368764\\
181.431672442999	27.9667830509301\\
180.97950222524	28.0363190881194\\
180.52735184615	28.1048930557004\\
180.075244352488	28.1724954990823\\
179.623202905195	28.2391170018275\\
179.171250779397	28.3047481856516\\
178.719411364403	28.3693797104231\\
178.267708163705	28.433002274164\\
177.81616479498	28.4956066130492\\
177.364804990089	28.5571835014068\\
176.913651688434	28.617723922402\\
176.462698527341	28.6772252699722\\
176.011953449083	28.7356845975781\\
175.561438548038	28.7930967733901\\
175.111175727597	28.8494568122452\\
174.661186700154	28.9047598756466\\
174.211492987116	28.9590012717638\\
173.762115918894	29.012176455433\\
173.31307663491	29.0642810281563\\
172.864396083591	29.1153107381027\\
172.416095022375	29.1652614801072\\
171.968194017706	29.2141292956713\\
171.520713445037	29.2619103729629\\
171.073673488829	29.3086010468164\\
170.62709414255	29.3541977987322\\
170.180995208676	29.3986972568776\\
169.735396298694	29.4420961960858\\
169.290316833095	29.4843915378567\\
168.845776041381	29.5255803503565\\
168.40179296206	29.5656598484176\\
167.958386442649	29.604627393539\\
167.515575139673	29.6424804938861\\
167.073377518665	29.6792168042904\\
166.631811854166	29.7148341262501\\
166.190896229724	29.7493304079297\\
165.750648537897	29.7827037441598\\
165.311086480249	29.8149523764378\\
164.872227567354	29.8460746929272\\
164.434089118792	29.876069228458\\
163.996688263153	29.9049346645266\\
163.560041938034	29.9326698292955\\
163.124166890039	29.9592736975941\\
162.689079674782	29.9847453909176\\
162.254796656884	30.0090841774281\\
161.821334009973	30.0322894719537\\
161.388707716688	30.054360835989\\
160.956933568673	30.0752979776951\\
160.526027166581	30.0951007518994\\
160.096003920073	30.1137691600955\\
159.66687904782	30.1313033504436\\
159.238667577497	30.1477036177704\\
158.81138434579	30.1629704035685\\
158.385043998393	30.1771042959974\\
157.959660990007	30.1901060298826\\
157.535249584341	30.2019764867163\\
157.111823854112	30.2127166946568\\
156.689397681045	30.222327828529\\
156.267984755875	30.2308112098239\\
155.847598578342	30.2381683066992\\
155.428252457196	30.2444007339788\\
155.009959510194	30.249510253153\\
154.592732664102	30.2534987723785\\
154.176584654693	30.2563683464783\\
153.761528026748	30.258121176942\\
153.347575134057	30.2587596119254\\
152.934738139418	30.2582861462506\\
152.523029014636	30.2567034214063\\
152.112459540525	30.2540142255474\\
151.703041306906	30.2502214934953\\
151.294785712608	30.2453283067377\\
150.88770396547	30.2393378934288\\
150.481807082337	30.2322536283889\\
150.077105889062	30.224079033105\\
149.673611020508	30.2148177757304\\
149.271332920543	30.2044736710846\\
148.870281842047	30.1930506806537\\
148.470467846903	30.18055291259\\
148.071900806007	30.1669846217124\\
147.674590399259	30.1523502095059\\
147.27854611557	30.1366542241221\\
146.883777252858	30.1199013603789\\
146.490292918047	30.1020964597606\\
146.098102027073	30.0832445104178\\
145.707213304876	30.0633506471676\\
145.317635285407	30.0424201514934\\
144.929376311624	30.0204584515451\\
144.542444535492	29.9974711221387\\
144.156847917986	29.973463884757\\
143.772594229087	29.9484426075487\\
143.389691047785	29.9224133053292\\
143.008145762079	29.8953821395803\\
142.627965568974	29.86735541845\\
142.249157474484	29.8383395967527\\
141.871728293632	29.8083412759694\\
141.495684650447	29.7773672042472\\
141.121032977967	29.7454242763997\\
140.747779518239	29.712519533907\\
140.375930322317	29.6786601649153\\
140.005491250262	29.6438535042374\\
139.636467971146	29.6081070333524\\
139.268865963045	29.5714283804058\\
138.902690513047	29.5338253202096\\
138.537946717245	29.4953057742419\\
138.174639480742	29.4558778106474\\
137.812773517648	29.415549644237\\
137.452353351081	29.3743296364883\\
137.093383313168	29.332226295545\\
136.735867545042	29.2892482762172\\
136.379809996846	29.2454043799814\\
136.025214427731	29.2007035549806\\
135.672085095867	29.1551543232042\\
135.320435891758	29.1087568707419\\
134.970272585802	29.0615173357625\\
134.621596990895	29.0134449556093\\
134.274410831652	28.9645489463463\\
133.928715744413	28.9148385027591\\
133.584513277236	28.8643227983545\\
133.241804889901	28.8130109853605\\
132.900591953912	28.7609121947263\\
132.560875752489	28.7080355361226\\
132.222657480579	28.6543900979409\\
131.885938244845	28.5999849472944\\
131.550719063674	28.5448291300172\\
131.217000867174	28.4889316706649\\
130.884784497174	28.4323015725142\\
130.554070707224	28.3749478175631\\
130.224860162596	28.3168793665308\\
129.897153440281	28.2581051588579\\
129.570951028994	28.1986341127059\\
129.246253329169	28.138475124958\\
128.923060652962	28.0776370712184\\
128.601373224251	28.0161288058124\\
128.281191178635	27.9539591617869\\
127.962514563433	27.8911369509098\\
127.645343337686	27.8276709636702\\
127.329677372156	27.7635699692787\\
127.015516449326	27.698842715667\\
126.702860263402	27.633497929488\\
126.391708420309	27.5675443161158\\
126.082060437693	27.5009905596461\\
125.773915744923	27.4338453228953\\
125.467273683088	27.3661172474016\\
125.162133505	27.297814953424\\
124.858494375189	27.228947039943\\
124.556355369908	27.1595220846602\\
124.255715477133	27.0895486439987\\
123.956573596557	27.0190352531025\\
123.658928539599	26.9479904258371\\
123.362779029394	26.8764226547892\\
123.068123700804	26.8043404112666\\
122.774961100407	26.7317521452985\\
122.483289686505	26.6586662856354\\
122.193107829121	26.5850912397488\\
121.904413809999	26.5110353938318\\
121.617205822603	26.4365071127983\\
121.33148197212	26.3615147402839\\
121.047240275457	26.2860665986451\\
120.764478661243	26.2101709889598\\
120.483194969828	26.1338361910273\\
120.203386953283	26.0570704633677\\
119.925052275399	25.9798820432229\\
119.648188511691	25.9022791465556\\
119.372793149393	25.82426996805\\
119.09886358746	25.7458626811115\\
118.826397136571	25.6670654378666\\
118.555391019123	25.5878863691632\\
118.285842369235	25.5083335845705\\
118.017748232749	25.4284151723787\\
117.751105567226	25.3481391995995\\
117.48591124195	25.2675137119657\\
117.222162037923	25.1865467339315\\
116.959854647873	25.1052462686721\\
116.698985676246	25.0236202980842\\
116.439551639209	24.9416767827856\\
116.181548964652	24.8594236621153\\
115.924973992185	24.7768688541338\\
115.669822973139	24.6940202556225\\
115.416092070568	24.6108857420844\\
115.163777359244	24.5274731677434\\
114.912874825664	24.4437903655449\\
114.663380368043	24.3598451471554\\
114.41528979632	24.2756453029628\\
114.168598832152	24.1911986020761\\
113.92330310892	24.1065127923256\\
113.679398171725	24.021595600263\\
113.436879477389	23.9364547311608\\
113.195742394456	23.8510978690133\\
112.95598220319	23.7655326765357\\
112.717594095577	23.6797667951645\\
112.480573175325	23.5938078450576\\
112.244914457862	23.5076634250939\\
112.010612870338	23.4213411128737\\
111.777663251623	23.3348484647186\\
111.546060352309	23.2481930156713\\
111.315798834709	23.1613822794959\\
111.086873272858	23.0744237486776\\
110.859278152512	22.9873248944229\\
110.633007871146	22.9000931666596\\
110.40805673796	22.8127359940367\\
110.184418973872	22.7252607839244\\
109.962088711522	22.6376749224142\\
109.741059995273	22.549985774319\\
109.521326781207	22.4622006831726\\
109.302882937129	22.3743269712304\\
109.085722242562	22.2863719394687\\
108.869838388754	22.1983428675854\\
108.655224978673	22.1102470139995\\
108.441875527006	22.022091615851\\
108.229783460165	21.9338838890016\\
108.018942116281	21.8456310280339\\
107.80934667716	21.7573392995341\\
107.601008448571	21.6690069470075\\
107.393925688062	21.5806375471936\\
107.188092369329	21.4922365724291\\
106.983502481988	21.4038094454467\\
106.780150031569	21.315361539375\\
106.578029039523	21.2268981777388\\
106.377133543215	21.1384246344587\\
106.177457595931	21.0499461338514\\
105.978995266873	20.9614678506293\\
105.781740641159	20.8729949099012\\
105.585687819828	20.7845323871715\\
105.390830919834	20.6960853083408\\
105.197164074049	20.6076586497056\\
105.004681431262	20.5192573379583\\
104.813377156181	20.4308862501875\\
104.623245429431	20.3425502138775\\
104.434280447554	20.2542540069088\\
104.24647642301	20.1660023575578\\
104.059827584175	20.0777999444967\\
103.874328175346	19.989651396794\\
103.689972456734	19.901561293914\\
103.50675470447	19.813534165717\\
103.3246692106	19.7255744924592\\
103.143710283091	19.6376867047928\\
102.963872245824	19.5498751837662\\
102.7851494386	19.4621442608235\\
102.607536217136	19.3744982178048\\
102.431026953067	19.2869412869464\\
102.255616033947	19.1994776508803\\
102.081297863245	19.1121114426347\\
101.908066860349	19.0248467456336\\
101.735917460566	18.9376875936971\\
101.564844115116	18.8506379710412\\
101.394841291141	18.7637018122779\\
101.225903471699	18.6768830024153\\
101.058025155766	18.5901853768573\\
100.891200858233	18.5036127214038\\
100.725425109913	18.4171687722507\\
100.560692457532	18.33085721599\\
100.396997463737	18.2446816896094\\
100.23433470709	18.1586457804929\\
100.072698782073	18.0727530264202\\
99.9120842990831	17.9870069155672\\
99.7524858844365	17.9014108865057\\
99.5938981803665	17.8159683282033\\
99.4363158450239	17.7306825800239\\
99.279733552477	17.645556931727\\
99.1241459927118	17.5605946234685\\
98.9695478716316	17.4757988457998\\
98.8159339110574	17.3911727396688\\
98.6632988487278	17.306719396419\\
98.5116374382987	17.2224418577899\\
98.3609444493437	17.1383431159172\\
98.211214667354	17.0544261133323\\
98.0624428937382	16.9706937429628\\
97.9146239458225	16.8871488481322\\
97.7677526568505	16.80379422256\\
97.6218238759835	16.7206326103615\\
97.4768324683004	16.6376667060483\\
97.3327733147974	16.5548991545277\\
97.1896413123883	16.472332551103\\
97.0474313739047	16.3899694414737\\
96.9061384280953	16.307812321735\\
96.7657574196267	16.2258636383783\\
96.6262833090829	16.1441257882908\\
96.4877110729654	16.0626011187558\\
96.3500357036932	15.9812919274525\\
96.213252209603	15.9002004624562\\
96.0773556149489	15.8193289222379\\
95.9423409599026	15.738679455665\\
95.8082033005533	15.6582541620004\\
95.6749377089077	15.5780550909034\\
95.5425392728902	15.498084242429\\
95.4110030963425	15.4183435670283\\
95.280324299024	15.3388349655483\\
95.1504980166116	15.259560289232\\
95.0215194006996	15.1805213397185\\
94.8933836188002	15.1017198690427\\
94.7660858543427	15.0231575796355\\
94.6396213066742	14.944836124324\\
94.5139851910593	14.8667571063309\\
94.38917273868	14.7889220792752\\
94.265179196636	14.7113325471717\\
94.1419998279444	14.6339899644312\\
94.0196299115401	14.5568957358606\\
93.8980647422751	14.4800512166627\\
93.7772996309194	14.4034577124361\\
93.6573299041602	14.3271164791757\\
93.5381509046023	14.2510287232721\\
93.4197579907682	14.1751956015121\\
93.3021465370977	14.0996182210783\\
93.1853119339484	14.0242976395493\\
93.0692495875952	13.9492348648998\\
92.9539549202306	13.8744308555003\\
92.8394233699648	13.7998865201175\\
92.7256503908252	13.7256027179138\\
92.6126314527571	13.6515802584479\\
92.5003620416231	13.5778199016741\\
92.3888376592034	13.5043223579431\\
92.2780533143526	13.4310887901432\\
92.1680015025977	13.3581228150705\\
92.0586773042921	13.2854254954891\\
91.9500763288572	13.2129973744842\\
91.8421942043318	13.1408389791028\\
91.7350265773725	13.0689508203544\\
91.6285691132534	12.9973333932105\\
91.5228174958661	12.9259871766046\\
91.4177674277201	12.8549126334325\\
91.3134146299422	12.7841102105521\\
91.2097548422768	12.7135803387836\\
91.1067838230861	12.6433234329089\\
91.0044973493497	12.5733398916726\\
90.902891216665	12.503630097781\\
90.8019612392467	12.4341944179027\\
90.7017032499274	12.3650332026685\\
90.6021131001571	12.2961467866713\\
90.5031866600034	12.2275354884662\\
90.4049198181517	12.1591996105702\\
90.3073084819047	12.0911394394626\\
90.2103485771829	12.0233552455851\\
90.1140360485243	11.955847283341\\
90.0183668590845	11.8886157910962\\
89.9233369906369	11.8216609911786\\
89.8289424435721	11.7549830898781\\
89.7351792368987	11.6885822774468\\
89.6420434082425	11.6224587280992\\
89.5495310138473	11.5566126000116\\
89.4576381285742	11.4910440353226\\
89.366360845902	11.4257531601329\\
89.2756952779271	11.3607400845054\\
89.1856375553635	11.296004902465\\
89.0961838275427	11.231547691999\\
89.007330262414	11.1673685150565\\
88.9190730465441	11.1034674175491\\
88.8314083851174	11.0398444293502\\
88.7443325019357	10.9764995642955\\
88.6578416394188	10.913432820183\\
88.5719320586037	10.8506441787725\\
88.4866000391451	10.7881336057863\\
88.4018418793154	10.7259010509085\\
88.3176538960046	10.6639464477856\\
88.2340324247202	10.6022697140261\\
88.1509738195872	10.5408707512007\\
88.0684744533484	10.4797494448422\\
87.9865307173642	10.4189056644456\\
87.9051390216124	10.358339263468\\
87.8242957946885	10.2980500793286\\
87.7439974838056	10.2380379334089\\
87.6642405547945	10.1783026310524\\
87.5850214921033	10.1188439615646\\
87.5063367987981	10.0596616982135\\
87.4281829965622	10.000755598229\\
87.3505566256968	9.94212540280317\\
87.2734542451204	9.88377083709024\\
87.1968724323694	9.82569161020659\\
87.1208077835977	9.76788741523074\\
87.0452569135766	9.71035792920333\\
86.9702164556953	9.65310281312716\\
86.8956830619603	9.59612171196717\\
86.8216534029959	9.53941425465043\\
86.748124168044	9.48298005406616\\
86.6750920649639	9.4268187070657\\
86.6025538202327	9.37092979446257\\
86.530506178945	9.31531288103239\\
86.458945904813	9.25996751551295\\
86.3878697801666	9.20489323060417\\
86.3172746059531	9.1500895429681\\
86.2471572017374	9.09555595322894\\
86.1775144057023	9.04129194597303\\
86.1083430746479	8.98729698974887\\
86.039640083992	8.93357053706706\\
85.9714023277699	8.88011202440038\\
85.9036267186346	8.82692087218372\\
85.8363101878568	8.77399648481413\\
85.7694496853245	8.7213382506508\\
85.7030421795435	8.66894554201504\\
85.6370846576371	8.61681771519032\\
85.5715741253464	8.56495411042227\\
85.5065076070298	8.5133540519186\\
85.4418821456635	8.46201684784922\\
85.3776948028413	8.41094179034616\\
85.3139426587744	8.36012815550359\\
85.2506228122917	8.30957520337781\\
85.1877323808398	8.25928217798727\\
85.1252685004828	8.20924830731257\\
85.0632283259024	8.15947280329643\\
85.0016090303979	8.10995486184374\\
84.9404078058862	8.0606936628215\\
84.8796218629018	8.01168837005886\\
84.8192484305968	7.96293813134712\\
84.7592847567408	7.91444207843972\\
84.6997281077211	7.86619932705222\\
84.6405757685426	7.81820897686236\\
84.5818250428278	7.77047011150997\\
84.5234732528167	7.72298179859707\\
84.465517739367	7.67574308968778\\
84.407955861954	7.62875302030839\\
84.3507849986704	7.58201060994733\\
84.2940025462267	7.53551486205513\\
84.237605508701	7.48926502537071\\
84.1815897129953	7.44326123612801\\
84.12595261461	7.39750278057533\\
84.0706918667145	7.35198883292373\\
84.0158051320201	7.30671856603044\\
83.96129008278	7.26169115139882\\
83.9071444007894	7.2169057591784\\
83.8533657773851	7.17236155816487\\
83.7999519134463	7.12805771580008\\
83.7469005193936	7.08399339817201\\
83.6942093151899	7.04016777001482\\
83.6418760303398	6.99657999470881\\
83.58989840389	6.95322923428045\\
83.5382741844288	6.91011464940235\\
83.4870011300867	6.86723539939329\\
83.4360770085359	6.82459064221819\\
83.3854995969908	6.78217953448814\\
83.3352666822075	6.74000123146037\\
83.2853760604839	6.69805488703829\\
83.2358255376601	6.65633965377143\\
83.1866129291179	6.61485468285551\\
83.1377360597811	6.57359912413237\\
83.0891927641154	6.53257212609005\\
83.0409808861283	6.49177283586271\\
82.9930982793694	6.45120039923067\\
82.9455428069302	6.41085396062042\\
82.8983123414439	6.37073266310459\\
82.8514047650858	6.33083564840197\\
82.804817969573	6.29116205687752\\
82.7585498561646	6.25171102754233\\
82.7125983356617	6.21248169805367\\
82.666961328407	6.17347320471494\\
82.6216367642854	6.13468468247571\\
82.5766225827236	6.09611526493171\\
82.5319167326903	6.05776408432482\\
82.4875171726959	6.01963027154307\\
82.4434218707929	5.98171295612066\\
82.3996288045757	5.94401126623792\\
82.3561359611805	5.90652432872136\\
82.3129413372855	5.86925126904363\\
82.2700429391109	5.83219121132355\\
82.2274387824185	5.79534327832608\\
82.1851268925124	5.75870659146234\\
82.1431053042383	5.72228027078962\\
82.101372061984	5.68606343501134\\
82.0599252196792	5.65005520147709\\
82.0187628407953	5.61425468618261\\
81.977882998346	5.57866100376981\\
81.9372837748865	5.54327326752674\\
81.8969632625141	5.5080905893876\\
81.8569195628681	5.47311207993277\\
81.8171507871295	5.43833684838876\\
81.7776550560215	5.40376400262824\\
81.7384304998088	5.36939264917005\\
81.6994752582985	5.33522189317918\\
81.6607874808392	5.30125083846676\\
81.6223653263216	5.26747858749009\\
81.5842069631784	5.23390424135263\\
81.5463105693839	5.20052689980397\\
81.5086743324547	5.1673456612399\\
81.4712964494489	5.13435962270231\\
81.4341751269669	5.10156787987929\\
81.3973085811509	5.06896952710506\\
81.3606950376847	5.03656365736\\
81.3243327317945	5.00434936227067\\
81.2882199082481	4.97232573210975\\
81.2523548213552	4.94049185579608\\
81.2167357349676	4.90884682089469\\
81.1813609224789	4.87738971361673\\
81.1462286668246	4.84611961881951\\
81.1113372604821	4.8150356200065\\
81.0766850054707	4.78413679932733\\
81.0422702133518	4.75342223757779\\
81.0080912052285	4.72289101419981\\
80.9741463117459	4.69254220728149\\
80.9404338730909	4.66237489355707\\
80.9069522389924	4.63238814840695\\
80.8736997687213	4.6025810458577\\
80.8406748310904	4.57295265858203\\
80.8078758044541	4.5435020578988\\
80.775301076709	4.51422831377304\\
80.7429490452937	4.48513049481595\\
80.7108181171885	4.45620766828483\\
80.6789067089157	4.4274589000832\\
80.6472132465394	4.39888325476069\\
80.6157361656657	4.37047979551311\\
80.5844739114427	4.34224758418241\\
80.5534249385603	4.3141856812567\\
80.5225877112503	4.28629314587026\\
80.4919607032865	4.25856903580351\\
80.4615423979845	4.23101240748301\\
80.4313312882019	4.20362231598151\\
80.4013258763381	4.17639781501791\\
80.3715246743347	4.14933795695723\\
80.3419262036748	4.12244179281068\\
80.3125289953836	4.09570837223561\\
80.2833315900284	4.06913674353554\\
80.2543325377181	4.04272595366012\\
80.2255303981037	4.01647504820519\\
80.1969237403781	3.99038307141272\\
80.168511162213	3.96444917707995\\
80.1402914116174	3.93867281741119\\
80.1122633012586	3.91305323726883\\
80.0844256481456	3.88758966547144\\
80.0567772738041	3.86228133240705\\
80.0293170042763	3.83712747003314\\
80.0020436701214	3.81212731187664\\
79.9749561064152	3.78728009303394\\
79.9480531527501	3.76258505017087\\
79.9213336532355	3.73804142152273\\
79.8947964564972	3.71364844689427\\
79.8684404156779	3.68940536765967\\
79.8422643884369	3.6653114267626\\
79.8162672369505	3.64136586871615\\
79.7904478279113	3.61756793960288\\
79.764805032529	3.59391688707481\\
79.7393377265297	3.57041196035339\\
79.7140447901565	3.54705241022954\\
79.6889251081689	3.52383748906364\\
79.6639775698435	3.5007664507855\\
79.6392010689733	3.47783855089439\\
79.6145945038681	3.45505304645906\\
79.5901567773546	3.43240919611767\\
79.5658867967758	3.40990626007787\\
79.541783473992	3.38754350011674\\
79.5178457253796	3.36532017958081\\
79.4940724718323	3.3432355633861\\
79.47046263876	3.32128891801803\\
79.4470151560896	3.29947951153151\\
79.4237289582648	3.2778066135509\\
79.4006029842458	3.25626949526999\\
79.3776361775095	3.23486742945206\\
79.3548274860497	3.2135996904298\\
79.3321758623769	3.19246555410539\\
79.3096802635182	3.17146429795043\\
79.2873396510174	3.15059520100601\\
79.2651529909352	3.12985754388265\\
79.2431192538488	3.10925060876032\\
79.2212374148523	3.08877367938846\\
79.1995064535564	3.06842604108594\\
79.1779253540886	3.0482069807411\\
79.1564931050929	3.02811578681173\\
79.1352086997304	3.00815174932508\\
79.1140711356786	2.98831415987784\\
79.0930794151319	2.96860231163615\\
79.0722325448012	2.94901549933562\\
79.0515295359143	2.92955301928131\\
79.0309694042158	2.91021416934771\\
79.0105511699667	2.89099824897879\\
78.990273857945	2.87190455918797\\
78.9701364974454	2.85293240255811\\
78.9501381222791	2.83408108324152\\
78.9302777707742	2.81534990695999\\
78.9105544857755	2.79673818100474\\
78.8909673146445	2.77824521423644\\
78.8715153092594	2.75987031708522\\
78.8521975260151	2.74161280155068\\
78.8330130258232	2.72347198120185\\
78.8139608741122	2.70544717117722\\
78.795040140827	2.68753768818474\\
78.7762499004295	2.66974285050179\\
78.7575892318982	2.65206197797523\\
78.7390572187283	2.63449439202137\\
78.7206529489318	2.61703941562596\\
78.7023755150373	2.5996963733442\\
78.6842240140901	2.58246459130077\\
78.6661975476525	2.56534339718977\\
78.6482952218032	2.54833212027477\\
78.6305161471377	2.5314300913888\\
78.6128594387682	2.51463664293432\\
78.5953242163238	2.49795110888327\\
78.5779096039502	2.48137282477703\\
78.5606147303096	2.46490112772643\\
78.5434387285812	2.44853535641175\\
78.5263807364609	2.43227485108273\\
78.5094398961612	2.41611895355857\\
78.4926153544114	2.40006700722791\\
78.4759062624574	2.38411835704886\\
78.4593117760619	2.36827234954895\\
78.4428310555044	2.3525283328252\\
78.426463265581	2.33688565654407\\
78.4102075756045	2.32134367194146\\
78.3940631594045	2.30590173182273\\
78.3780291953273	2.29055919056271\\
78.3621048662359	2.27531540410567\\
78.34628935951	2.26016972996532\\
78.330581867046	2.24512152722484\\
78.3149815852571	2.23017015653687\\
78.2994877150732	2.21531498012348\\
78.2840994619407	2.2005553617762\\
78.2688160358232	2.18589066685603\\
78.2536366512004	2.17132026229341\\
78.2385605270693	2.15684351658823\\
78.2235868869432	2.14245979980984\\
78.2087149588523	2.12816848359703\\
78.1939439753435	2.11396894115807\\
78.1792731734804	2.09986054727065\\
78.1647017948433	2.08584267828193\\
78.1502290855293	2.07191471210854\\
78.1358542961521	2.05807602823652\\
78.1215768198477	2.04432612266957\\
78.1073962785595	2.03066468364159\\
78.0933120863975	2.01709122848948\\
78.0793236519808	2.00360526962864\\
78.0654303860962	1.99020632089073\\
78.051631701698	1.97689389752361\\
78.0379270139086	1.96366751619141\\
78.0243157400178	1.95052669497447\\
78.0107972994834	1.93747095336937\\
77.9973711139306	1.92449981228892\\
77.9840366071527	1.91161279406216\\
77.9707932051104	1.89880942243438\\
77.9576403359324	1.88608922256708\\
77.9445774299148	1.87345172103801\\
77.9316039195217	1.86089644584116\\
77.9187192393849	1.84842292638673\\
77.9059228263037	1.83603069350117\\
77.8932141192455	1.82371927942716\\
77.880592559345	1.81148821782361\\
77.8680575899049	1.79933704376567\\
77.8556086563955	1.78726529374471\\
77.843245206455	1.77527250566836\\
77.8309666898892	1.76335821886045\\
77.8187725586715	1.75152197406107\\
77.8066622669432	1.73976331342653\\
77.7946352710134	1.72808178052938\\
77.7826910293587	1.71647692035839\\
77.7708290026235	1.70494827931858\\
77.7590486536199	1.6934954052312\\
77.747349447328	1.68211784733372\\
77.7357308508952	1.67081515627987\\
77.7241923336369	1.65958688413958\\
77.7127333670362	1.64843258439904\\
77.7013534247437	1.63735181196066\\
77.6900519825781	1.62634412314309\\
77.6788285185255	1.61540907568121\\
77.66768251274	1.60454622872613\\
77.656613447543	1.59375514284521\\
77.6456208074242	1.58303538002201\\
77.6347040790405	1.57238650365636\\
77.6238627512168	1.5618080785643\\
77.6130963149458	1.55129967097812\\
77.6024042633876	1.54086084854633\\
77.5917860918704	1.53049118033367\\
77.5812412978898	1.52019023682113\\
77.5707693811094	1.50995758990592\\
77.5603698433603	1.49979281290149\\
77.5500421886415	1.48969548053753\\
77.5397859231195	1.47966516895994\\
77.5296005551288	1.46970145573087\\
77.5194855951715	1.45980391982871\\
77.5094405559174	1.44997214164808\\
77.4994649522039	1.44020570299981\\
77.4895583010365	1.43050418711099\\
77.479720121588	1.42086717862495\\
77.4699499351992	1.41129426360121\\
77.4602472653785	1.40178502951558\\
77.4506116378022	1.39233906526006\\
77.441042580314	1.3829559611429\\
77.4315396229256	1.37363530888858\\
77.4221022978164	1.36437670163782\\
77.4127301393334	1.35517973394758\\
77.4034226839914	1.34604400179102\\
77.3941794704728	1.33696910255757\\
77.3850000396281	1.32795463505288\\
77.375883934475	1.31900019949883\\
77.3668307001993	1.31010539753353\\
77.3578398841545	1.30126983221134\\
77.3489110358616	1.29249310800284\\
77.3400437070095	1.28377483079484\\
77.3312374514547	1.27511460789039\\
77.3224918252217	1.26651204800878\\
77.3138063865024	1.25796676128552\\
77.3051806956566	1.24947835927237\\
77.2966143152118	1.24104645493731\\
77.2881068098632	1.23267066266454\\
77.2796577464736	1.22435059825453\\
77.2712666940739	1.21608587892396\\
77.2629332238623	1.20787612330574\\
77.254656909205	1.19972095144902\\
77.2464373256358	1.1916199848192\\
77.2382740508562	1.18357284629787\\
77.2301666647355	1.17557916018291\\
77.2221147493108	1.16763855218838\\
77.2141178887868	1.15975064944461\\
77.2061756695359	1.15191508049814\\
77.1982876800983	1.14413147531177\\
77.1904535111819	1.13639946526451\\
77.1826727556623	1.12871868315161\\
77.174945008583	1.12108876318455\\
77.1672698671548	1.11350934099105\\
77.1596469307568	1.10598005361507\\
77.1520758009353	1.09850053951678\\
77.1445560814047	1.09107043857261\\
77.1370873780469	1.08368939207521\\
77.1296692989116	1.07635704273346\\
77.1223014542163	1.06907303467248\\
77.1149834563461	1.06183701343362\\
77.1077149198538	1.05464862597447\\
77.1004954614601	1.04750752066884\\
77.0933249912197	1.04041360323514\\
77.0862035882881	1.03336693049233\\
77.079130935327	1.02636721249347\\
77.0721067130121	1.01941415747038\\
77.0651306035333	1.0125074749023\\
77.0582022905947	1.00564687551596\\
77.0513214594144	0.998832071285486\\
77.044487796725	0.992062775432474\\
77.0377009907729	0.985338702425952\\
77.0309607313189	0.978659567982393\\
77.024266709638	0.972025089065711\\
77.0176186185193	0.965434983887262\\
77.011016152266	0.958888971905846\\
77.0044590066955	0.952386773827704\\
76.9979468791395	0.945928111606519\\
76.9914794684437	0.939512708443418\\
76.9850564749682	0.933140288786967\\
76.978677600587	0.926810578333177\\
76.9723425486884	0.920523304025502\\
76.9660510241749	0.914278194054835\\
76.9598027334632	0.908074977859514\\
76.953597384484	0.901913386125317\\
76.9474346866824	0.895793150785467\\
76.9413143510175	0.889714005020628\\
76.9352360899626	0.883675683258905\\
76.9291996175053	0.877677921175846\\
76.9232046491472	0.871720455694443\\
76.9172509019041	0.865803024985128\\
76.9113380943061	0.859925368465777\\
76.9054659463973	0.854087226801706\\
76.8996341797362	0.848288341905675\\
76.8938425173953	0.842528456937887\\
76.8880906839612	0.836807316305985\\
76.8823784055349	0.831124665665056\\
76.8767054097313	0.825480251917629\\
76.8710714256798	0.819873823213674\\
76.8654761840237	0.814305128950604\\
76.8599194169205	0.808773919773276\\
76.8544008580421	0.803279947573986\\
76.8489202425743	0.797822965492476\\
76.8434773072172	0.792402727915926\\
76.838071790185	0.787018990478962\\
76.8327034312062	0.781671510063651\\
76.8273719715234	0.7763600447995\\
76.8220771538933	0.771084354063462\\
76.8168187225869	0.765844198479931\\
76.8115964233893	0.760639339920741\\
76.8064100035998	0.755469541505172\\
76.8012592120318	0.750334567599943\\
76.796143799013	0.745234183819217\\
76.7910635163851	0.740168157024598\\
76.7860181175042	0.735136255325134\\
76.7810073572403	0.730138248077315\\
76.7760309919779	0.725173905885071\\
76.7710887796153	0.720243000599778\\
76.7661804795653	0.71534530532025\\
76.7613058527546	0.710480594392747\\
76.7564646616243	0.705648643410968\\
76.7516566701296	0.700849229216057\\
76.7468816437398	0.6960821298966\\
76.7421393494384	0.691347124788623\\
76.7374295557231	0.686643994475596\\
76.7327520326057	0.681972520788431\\
76.7281065516124	0.677332486805483\\
76.7234928857833	0.672723676852547\\
76.7189108096727	0.668145876502864\\
76.7143600993494	0.663598872577113\\
76.7098405323958	0.659082453143417\\
76.7053518879091	0.654596407517343\\
};
\addlegendentry{Sperimentazione}

\end{axis}
\end{tikzpicture}%}
\caption[Analisi dei dati della peste di Eyam]{Popolazione delle singole classi  in funzioni del tempo (a) e orbite nel piano $(S,I)$ (b).  Per ottenere i grafici abbiamo risolto  numericamente, usando MATLAB,  il problema di Cauchy~\eqref{SI} con  condizioni iniziali $S(0)=254, ~I(0)~=~7$.  Per quanto riguarda i parametri, abbiamo considerato    $\alpha=2.73$  e $\beta= 0.0178$.}
\label{fig::sirsemplice}
\end{figure}


%%% Local Variables:
%%% mode: latex
%%% TeX-master: "main"
%%% End:

