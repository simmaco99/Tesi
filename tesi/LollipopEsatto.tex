%per leggerlo come in KISS
% _x1 -> _{x3}
% _x3 -> _x1
% _x{3} -> _x3 
% _x2 -> _x{4}
% _x4 -> _x2
% _x{4} -> _x4

\documentclass{report}
%Formattazione e codifica caratteri
\usepackage[italian]{babel}
\usepackage[utf8]{inputenc}
\usepackage{microtype}
\usepackage{libertine}
% Tavola dei contenuti
\usepackage{tocloft}
\renewcommand{\cftdot}{}
% Figure
\usepackage{graphicx}
\usepackage{caption}
% Matematica
\usepackage{amsmath,amsfonts,amssymb,amsthm}
% Tabelle
\usepackage{booktabs}
%Listati
\usepackage{listings}
\lstloadlanguages{MATLAB}
% Testatine
\author{S. Di Lillo}
\usepackage{fancyhdr}
\pagestyle{fancy}
\setlength{\headheight}{15pt}
\fancyhf{}
\fancyhead[LE,RO]{S. Di Lillo}
\fancyhead[RE,LO]{\leftmark}
\fancyfoot[CE,CO]{}
\fancyfoot[LE,RO]{\thepage}
\renewcommand{\headrulewidth}{1.5pt}
% Titoli dei capitoli che "rubano" meno spazio
\usepackage{titlesec, blindtext}
\usepackage[dvipsnames]{xcolor}
\definecolor{gray75}{gray}{0.75}
\newcommand{\hsp}{\hspace{15pt}}
\titleformat{\chapter}[hang]{\LARGE\bfseries}{\thechapter\hsp\textcolor{gray75}{|}\hsp}{0pt}{\Large\bfseries}
\titleformat{\section}[hang]{\Large\bfseries}{\thesection\hsp\textcolor{gray75}{|}\hsp}{0pt}{\large\bfseries}

% Hyperref e metadati
\usepackage[linktocpage,
			colorlinks,
			citecolor=ForestGreen,
			bookmarks,
			pdftitle={Il Modello Epidemiologico SIR sulle Reti Complesse},
			pdfauthor={S. Di Lillo},
			pdfsubject={Tesi di Laurea Triennale},
			pdfkeywords={SIR,Epidemiologia,Reti Complesse},
			]{hyperref}


% Per importare le figure in MATLAB
\usepackage{pgfplots}
 \pgfplotsset{compat=newest}
  \usetikzlibrary{plotmarks}
  \usetikzlibrary{arrows.meta}
  \usepgfplotslibrary{patchplots}
  \usepackage{grffile}
\usepackage{tikz}
\pgfplotsset{plot coordinates/math parser=false}
  \newlength\figureheight
 \newlength\figurewidth


%Per le osservazioni
\theoremstyle{plain}
\newtheorem{oss}{Osservazione} 
%COMANDI PERSONALIZZATI 
\newcommand{\ro}{\mathcal R_0}
\newcommand{\di}{\,\,\mathrm{d} }
\newcommand{\tonde}[1]{\left( #1 \right)}
\newcommand{\ses}{\Leftrightarrow} 


\begin{document}
Le equazioni per i singoli nodi sono 
 \begin{equation*}
 \begin{aligned}
 \dot{\angol{S_1}}&= -\tau \tonde{ \angol{ S_1 I_3} + \angol{S_1 I_2}},  \quad 
 &\dot{\angol{S_3}}&= -\tau \tonde{ \angol{I_1 S_3} + \angol{I_2 S_3}+ \angol{S_3 I_4}},  \\
 \dot{\angol{I_1}}&= \spa \tau \tonde{ \angol{ S_1 I_2} + \angol{S_1 I_3}} - \gamma\angol{I_1} \quad 
 &\dot{\angol{I_3}}&= \spa \tau \tonde{ \angol{I_1 S_3} + \angol{I_2 S_3}+ \angol{S_3 I_4}}-\gamma\angol{I_3},\\
 \dot{\angol{S_2}}&= -\tau \tonde{ \angol{ I_1 S_2} + \angol{S_2 I_3}}, \quad &\dot{\angol{S_4}} &= -\tau \angol{I_3S_4},  \\
 \dot{\angol{I_2}}&= \spa \tau \tonde{ \angol{ I_1 S_2} + \angol{S_2 I_3}} - \gamma\angol{I_2},\quad & \dot{\angol{I_4}}&=\spa \tau\angol{I_3S_4}-\gamma\angol{I_4}. \\\
 \end{aligned}	
 \end{equation*}
 Tali equazioni dipendono da alcune coppie: tutte le disposizioni di archi con un nodo suscettibile ed un uno infetto. Abbiamo bisogno di equazioni addizionali
 \begin{equation*}
 \begin{aligned}
 \dot{\angol{S_1I_2}} =&\spa \tau\angol{S_1S_2I_3} - \tonde{ \tau + \gamma} \angol{S_1 I_2} -\tau\angol{S_1I_2I_3},\\
 \dot{\angol{S_1I_3}}=&\spa \tau\tonde{ \angol{S_1I_2S_3} + \angol{S_1S_3I_4}} - \tonde{ \tau +\gamma}\angol{S_1 I_3} -\tau \angol{ S_1 I_2 I_3},\\
 \dot{\angol{I_1S_2}} =&\spa\tau\angol{S_1S_2I_3} - \tonde{ \tau + \gamma} \angol{I_1 S_2} -\tau\angol{I_1S_2I_3},\\
\dot{\angol{S_2I_3}} =&\spa \tau\tonde{\angol{I_1S_2S_3} +{\angol{S_2S_3I_4}}} - \tonde{ \tau + \gamma} \angol{S_2 I_3} -\tau\angol{I_1S_2I_3},\\
 \dot{\angol{I_1S_3}} =&- \tau\tonde{\angol{I_1I_2S_3} \angol{I_1S_3I_4}} - \tonde{ \tau + \gamma} \angol{I_1 S_3} +\tau\angol{S_1I_2S_3},\\ 
\dot{\angol{I_2S_3}} =&- \tau\tonde{\angol{I_1I_2S_3} +\angol{I_2S_3I_4}} - \tonde{ \tau + \gamma} \angol{I_2 S_3} +\tau\angol{I_1S_2S_3},\\ 
 \dot{\angol{S_3I_4}} =&- \tau\tonde{\angol{I_1S_3I_4}+\angol{I_2S_3I_4}} - \tonde{ \tau + \gamma} \angol{S_3 I_4}, \\ 
    \dot{\angol{I_3S_4}} =& \spa \tau\tonde{\angol{I_1S_3S_4}+\angol{I_2S_3S_4}} - \tonde{ \tau + \gamma} \angol{I_3 S_4}.\\ 
\end{aligned}
 \end{equation*}
Per le triple
 \begin{equation*}
 \begin{aligned}
 		\dot{\angol{S_1S_2I_3}}=&\spa \tau \angol{S_1S_2S_3I_4} -\tonde{ 2\tau + \gamma} \angol{S_1S_2I_3},\\
 		\dot{\angol{S_1I_2I_3}} =&\spa \tau \tonde{ \angol{S_1S_2I_3}+\angol{S_1I_2S_3I_4} } -2\tonde{ \tau + \gamma} \angol{S_1I_2I_3} + \tau \angol{S_1 I_2 S_3},\\
 		\dot{\angol{S_1I_2S_3}} =&-\tau \angol{S_1I_2S_3I_4} -\tonde{ 2\tau + \gamma} \angol{S_1I_2S_3},\\
 		\dot{\angol{I_1S_2I_3}} =&\spa \tau  \angol{S_1S_2I_3} -2\tonde{ \tau + \gamma} \angol{I_1S_2I_3} + \tau \angol{I_1S_2S_3}+\tau \angol{I_1S_2S_3I_4},\\
 		\dot{\angol{I_1S_2S_3}}=&-\tau \angol{ I_1S_2S_3I_4}-\tonde{ 2 \tau + \gamma} \angol{I_1S_2S_3},\\
 		\dot{\angol{I_1I_2S_3}} =&\spa\tau \tonde{ \angol{S_1I_2S_3}+ \angol{I_1S_2S_3} - \angol{I_1I_2S_3I_4}} -2\tonde{ \tau + \gamma} \angol{I_1I_2S_3}.
 	\end{aligned}
 \end{equation*}
 Mancano 
 \begin{equation*}
 	\begin{aligned}
 	\angol{I_1S_3I_4}=&-2
 	\tonde{ \tau + \gamma} ciap\angol{ I_1 S_3 I_4} + \tau \angol{S_1I_2S_3I_4}-\tau \angol{ I_1I_2S_3I_4}\\
 	\angol{I_1S_3 S_4} =& - \tonde{ \tau + \gamma}  \angol{I_1 S_3 S_4} + \tau\angol{S_1I_2S_3S_4} -\tau \angol{I_1I_2S_3S_4}\\
\angol{S_2S_3I_4}=& 	-\tonde{ \tau + \gamma } \angol{S_2S_3I_4}  - 2\tau \angol{ I_1S_2S_3I_4} \\
 	\angol{I_2S_3I_4}= &-2\tonde{ \tau + \gamma}  	\angol{I_2S_3I_4} + \tau \angol{ I_1S_2S_3I_4} - \tau \angol{ I_1I_2S_3 I_4} \\
 	\angol{I_2S_3S_4}=&- \tonde{ \tau + \gamma}\angol{ I_2 S_3S_4} +\tau \angol{ I_1S_2S_3S_4} - \tau \angol{ I_1I_2 S_3S_4}\\
 	\angol{ S_1S_3I_4} =&- 2\tau \angol{ S_1I_2S_3I_4} -\tonde{ \tau + \gamma} \angol{ S_1S_3I_4}
 	\end{aligned}
 \end{equation*}
e le quadruple
 \begin{equation*}
 	\begin{aligned}	
 	\angol{I_1I_2S_3I_4} = &- 3\tonde{ \tau +\gamma} \angol{ I_1 I_2 S_3 I_4} + \tau \tonde{\angol{ S_1 I_2 S_3 I_4 } +\angol{ I_1 S_2 S_3 I_4}} \\
 	\angol{S_1I_2S_3I_4}=& -\tonde{ 3\tau + 2\gamma}  	\angol{S_1I_2S_3I_4}\\
 	\angol{I_1S_2S_3I_4}=&-\tonde{ 3\tau + 2\gamma}\angol{I_1S_2S_3I_4}\\
 	\angol{I_1I_2S_3S_4}=& \tau \tonde{\angol{S_1 I_2 S_3S_4} +\angol{ I_1 S_2 S_3S_4}} -2\tonde{ \tau+\gamma} \angol{I_1I_2S_3S_4}\\
 	\angol{I_1S_2S_3S_4}=&-\tonde{ 2\tau + \gamma} \angol{I_1S_2S_3S_4}\\
 	\angol{S_1I_2S_3S_4}=&-\tonde{ 2\tau + \gamma} \angol{S_1I_2S_3S_4}\\
 	\angol{S_1S_2S_3I_4}=&-\tonde{ \tau + \gamma} \angol{S_1S_2S_3I_4}\\	
 	\end{aligned}
\end{equation*}
\end{document}