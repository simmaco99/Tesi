\subsection{Approccio generale alla chiusura}
In questa sezione ci domanderemo se sia possibile trovare una rappresentazione con un numero minore di variabili che sia pi\`u accurata o,  meglio ancora, esatta.  Per farlo useremo delle propriet\`a di connessione del grafico.\\ \\
Sia $G=(V,E)$ un grafo connesso.
\begin{itemize}
\item Diremo che $v\in V$ \`e un \textit{cut-vertex} (vertice tagliato) se il grafo senza il nodo $v$ risulta sconnesso.
\item Diremo che $e\in E$ \`e un \textit{ponte} se il grafo $G'=(V, E\backslash\{e\})$ risulta sconnesso.
\end{itemize}
Per un esempio di cut-vertex si guardi la Figura~\ref{fig::cut-vertex}.
\begin{figure}[!ht]
\centering
\subfloat[]{
\begin{tikzpicture}[scale=0.8]

	\Vertex[y=0]{1}
	\Vertex[x=2,y=0,label=]{2} 
	
	\Vertex[x=1,y=-1.73,label=1]{3}
	\Vertex[x=3,y=-1.73]{4}
	
	\Vertex[x=-1, y=-3.73]{15}
	\Vertex[x=1,  y=-3.73,label=2]{5}
	\Vertex[x=3,  y=-3.73,label=4]{6}
	\Vertex[x=5,  y=-3.73,label=5]{7}
	\Vertex[x=7,  y=-3.73]{8}

    \Vertex[x=-1, y=-5.73]{14}
    \Vertex[x=1,  y=-5.73,label=3]{9}
	\Vertex[x=3,  y=-5.73]{10}
    \Vertex[x=5,  y=-5.73]{11}
    \Vertex[x=7,  y=-5.73]{12}
   
    \Vertex[x=0, y=-7.46]{16}
    \Vertex[x=2, y=-7.46]{17}
    
\Edge(9)(16)\Edge(9)(17) \Edge(5)(15) \Edge(14)(15) \Edge(2)(3) \Edge(1)(2) \Edge(1)(3) \Edge(3)(4) \Edge(3)(5) \Edge(5)(4) \Edge(6)(4) \Edge(5)(6) \Edge(6)(7) \Edge(8)(7) \Edge(14)(9) \Edge(9)(5) \Edge(6)(10) \Edge(7)(11) \Edge(7)(12) \Edge(8)(11) \Edge(11)(12) \Edge(7)(11) \Edge(8)(12) \Edge(16)(17) \Edge(16)(9) \Edge(17)(9)
          
\end{tikzpicture}	}\\
\subfloat[]{\begin{tikzpicture}[scale=0.8]
	\Vertex[x=-2,y=0]{1}
	\Vertex[x=0,y=0]{2}
\Vertex[x=-1,y=-1.73,label=1]{33}
	\Vertex[x=1,y=-1.73,label=1]{3}
	\Vertex[x=3,y=-1.73]{4}

	\Vertex[x=-3, y=-3.73,]{60}
	\Vertex[x=-1, y=-3.73,label=2]{55}
	\Vertex[x=1, y=-3.73,label=2]{5}
	\Vertex[x=3, y=-3.73,label=4]{6}
	\Vertex[x=5,y=-3.73,label=4]{66}
	\Vertex[x=7, y=-3.73,label=5]{7}
	
	\Vertex[x=-3, y=-5.73]{14}
    \Vertex[x=-1, y=-5.73,label=3]{20}
	\Vertex[x=3,  y=-5.73,label=4]{10}
	\Vertex[x=9,  y=-3.73,label=5]{11}
	\Vertex[x=11,  y=-3.73]{12}

    \Vertex[x=-1, y=-7.73,label=3]{9}
    \Vertex[x=3,  y=-7.73]{67}
	\Vertex[x=9,  y=-5.73]{77}
    \Vertex[x=11,  y=-5.73]{88}
     
    \Vertex[x=-2, y=-9.46]{16}
    \Vertex[x=0, y=-9.46]{17}
    
\Edge(2)(33) \Edge(1)(2) \Edge(1)(33) \Edge(3)(4) \Edge(3)(5) \Edge(5)(4) \Edge(6)(4) \Edge(5)(6) \Edge(66)(7) \Edge(14)(20) \Edge(55)(60) \Edge(55)(20) \Edge(60)(14) \Edge(67)(10) \Edge(77)(11) \Edge(77)(12) \Edge(88)(11) \Edge(88)(77) \Edge(11)(12) \Edge(77)(11) \Edge(88)(12) \Edge(16)(17) \Edge(16)(9) \Edge(17)(9)

\end{tikzpicture}	
}

\caption[Cut-vertex e decomposizione in sottoreti]{(a) Un esempio di rete con $5$ cut-vertices e (b) la rete equivalente dopo la decomposizioni in sottoreti.}
\label{fig::cut-vertex}
\end{figure}
\newpage

Prima di presentare il teorema cardine del capitolo  
ricordiamo per\`o, la seguente definizione.\\ \\
Siano $A,B$ due eventi di uno spazio di probabilit\`a con $\mathbb{P}(B)>0$. Si dice \textit{probabilit\`a condizionale} di $A$ dato $B$ la quantit\`a.
\begin{equation} 
\label{p_con}
\mathbb{P}(A\, \vert B) = \frac{ \mathbb{P}(A\cap B)}{\mathbb{P}(B)}.
\end{equation}

\begin{thm}\label{th_cut-vertex}
Sia $G=(V,E)$ un grafo e $F=\{ v_1, \dots, v_k\}$ un sottoinsieme connesso di vertici e sia $v_i$ un suo cut-vertex. Poniamo  $$ F_1 = \{ v_1, \dots, v_{i-1}\} \text{ e }  F_2 =\{ v_{i+1}, \dots, v_k\}.$$ 
Se ogni cammino che connette un nodo in $F_1$ ad uno in $F_2$ passa da $v_i$ allora: 
\begin{equation}\label{cut}\angol{ Z_{v_1}\dots Z_{v_{i-1}} S_{v_i} Z_{v_{i+1}} \dots Z_{v_k}} = \angol{ Z_{v_1}\dots Z_{v_{i-1}} S_{v_i}} \angol{S_{v_i}  Z_{v_{i+1}} \dots Z_{v_k}}	
\end{equation}
dove $Z\in \{ S,I,R\}$. 
\proof  
Se $\angol{S_{v_i}}=0$ allora l'uguaglianza~\eqref{cut} risulta banalmente vera.\\ \\ 
Sia $\angol{S_{v_i}}\neq 0 $. Utilizzando la definizione di  probabilit\`a condizionale~\eqref{p_con}
$$ \angol{ Z_{v_1}\dots Z_{v_{i-1}} S_{v_i} Z_{v_{i+1}}\dots Z_{v_k}} = \angol{ Z_{v_1}\dots Z_{v_{i-1}} S_{v_i} Z_{v_{i+1}}\dots Z_{v_k}\, \vert \, S_{v_i}} \angol{S_{v_i}}.$$
Notiamo che $$ \angol{ Z_{v_1}\dots Z_{v_{i-1}} S_{v_i} Z_{v_{i+1}}\dots Z_{v_k}\, \vert \, S_{v_i}} = \angol{ Z_{v_1}\dots Z_{v_{i-1}} S_{v_i} \, \vert\, S_{v_i}} \angol{S_{v_i}Z_{v_{i+1}}\dots Z_{v_k}\, \vert \, S_{v_i}},$$ 
infatti, ogni percorso da $F_1$ a $F_2$ deve passare attraverso $v_i$. Ora  $v_i$ \`e suscettibile dunque la trasmissione non pu\`o avvenire tra un nodo in $F_1$ ed uno in $F_2$. Per tale motivo i due eventi presenti nel membro di sinistra sono indipendenti.\\
Se riapplichiamo la definizione di probabilit\`a condizionale otteniamo  
$$ \angol{ Z_{v_1} \dots Z_{v_{i-1}}S_{v_i} \, \vert S_{v_i}} =\frac{ \angol{Z_{v_1} \dots Z_{v_{i-1} S_{v_i}}}}{\angol{ S_{v_i}}},$$
$$ \angol{ S_{v_i} Z_{v_{i+1}}\dots Z_{v_{k}} \, \vert S_{v_i}} =\frac{ \angol{S_{v_i}Z_{v_{i+1}} \dots Z_{v_{k}}}}{\angol{ S_{v_i}}}$$
da cui la tesi.
\endproof
\end{thm}
Grazie a questo teorema siamo pronti per presentare l'algoritmo generale per le chiusure. Ecco i passi dell'algoritmo.
\begin{enumerate}
	\item Si trovano tutti i cut-vertices di $G$ e si denotino con $C=\{ v_{i_1}, \dots, v_{i_L}\}$. Tale procedimento pu\`o essere fatto con  un costo in tempo di $O(\vert E \vert + \vert V \vert)$ utilizzando una visita DFS. 
	\item Si divide la rete originale in sottoreti connesse a due a due scollegate.
	\item Tale procedura conduce ad una famiglia di sottoreti distinte $G_1, \dots, G_P$. Le sottoreti vengono create in modo che i cut-vertices siano mantenuti in tutte le sottoreti generate. Si veda la Figura~\ref{fig::cut-vertex}.
	\item Per ogni nodo $i$ delle sottoreti, si ha 
	\begin{equation*}
	\begin{aligned}
\dot{\angol{S_i}} &= -\tau \sum_j g_{ij} \angol{S_iI_j},\\
\dot{\angol{I_i}} &= \tau \sum_j g_{ij} \angol{S_iI_j}-\gamma\angol{I_i},\\
\dot{\angol{R_i}} &= 1 -\angol{S_i} -\angol{I_i}.
		\end{aligned}
	\end{equation*}
	Si possono trovare  equazioni simili per le derivate di tutte le coppie che sorgono in queste equazioni. Queste coppie dipendono dalle triple. A loro volta le triple dipendono dalle quadruple. Si forma, cos\`i, una gerarchia di equazioni.
	\item Nella gerarchia che si verr\`a a creare, se appare un termine composto da  vertici di sottoreti diverse allora in esso \`e presente un cut-vertex suscettibile. Usando il Teorema (\ref{th_cut-vertex}) \`e possibile esprimere questo termine usando termini pi\`u semplici.
\end{enumerate}
\newpage
Concludiamo l'esempio, che abbiamo iniziato a inizio capitolo, applicando l'algoritmo appena presentato alla rete in Figura~\ref{fig::3nodi}.\\
Come visto nel modello completo, per i singoli nodi si ha \begin{equation*}
\begin{aligned}
	\dot{\angol {S_1}} = & -\tau \angol{ S_1 I_2}, 
\quad &
	\dot{\angol {I_1}} = & \tau \angol{S_1 I_2}-\gamma \angol{I_1}, 
\\
	\dot{\angol {S_2}} = & -\tau \tonde{ \angol{ I_1 S_2} + \angol{I_3S_2}},	
\quad & 
	\dot{\angol {I_2}} = & \tau \tonde{ \angol{ I_1 S_2} + \angol{I_3S_2}}-\gamma \angol{I_2},
\\
	\dot{\angol {S_3}} = & -\tau \angol{ I_2 S_3},
\quad & 
	\dot{\angol {I_3}} = & \tau \angol{ I_2 S_3}-\gamma \angol{I_3}.
 \end{aligned}
\end{equation*}
Nelle precedenti equazioni appaiono delle coppie: scriviamo le equazioni anche per loro.\\
$$\dot{\angol{S_1I_2}}=\tau
\angol{S_1S_2I_3} - \tonde{ \tau + \gamma}\angol{S_1 I_2} = \frac{
\angol{S_1S_2}\angol{S_2I_3}}{\angol{S_2}} - \tonde{ \tau + \gamma}\angol{S_1 I_2}$$ 
infatti  dal Teorema~\ref{th_cut-vertex} applicato al cut-vertex $2$ si ha  
$$ \angol{S_1S_2I_3} = \frac{ \angol{S_1S_2}\angol{S_2I_3}}{\angol{S_2}}.$$ 
In maniera analoga otteniamo 
\begin{equation*}
\begin{aligned}
\dot{\angol{I_1S_2}}=&-\tau\frac{\angol{I_1S_2}\angol{S_2I_3}}{\angol{S_2}} - \tonde{ \tau + \gamma}\angol{I_1 S_2},
\\
\dot{\angol{S_2I_3}}=&-\tau\frac{\angol{I_1S_2}\angol{S_2I_3}}{\angol{S_2}} - \tonde{ \tau + \gamma}\angol{S_2 I_3},
\\
\dot{\angol{I_2S_3}}=&\spa\tau\frac{\angol{I_1S_2}\angol{S_2S_3}}{\angol{S_2}} - \tonde{ \tau + \gamma}\angol{I_2 S_3}.
\end{aligned}
\end{equation*}
Le precedenti equazioni introducono altre due coppie. Aggiungendole otteniamo un modello chiuso: 
\begin{equation*}
\begin{aligned}
\dot{\angol{S_1S_2}} = &- \tau\frac{\angol{S_1S_2}\angol{S_2I_3}}{\angol{S_2}},\\
\dot{\angol{S_2S_3}}= & -\tau \frac{\angol{ I_1 S_2}\angol{S_2S_3}}{\angol{S_2}}.
	\end{aligned}	
\end{equation*}
In questo esempio, non abbiamo ottenuto un notevole vantaggio, infatti abbiamo un sistema di $12$ equazioni mentre il  modello esatto~\eqref{3nodi} $13$ equazioni. Nell'esempio, che vedremo successivamente, potremmo apprezzare meglio il risparmio computazionale.\\ \\
Andando a risolvere numericamente il sistema e tracciando le soluzioni otteniamo la  Figura~\ref{fig::3nodicut}. Possiamo notare, come previsto dalla teoria, che  tali grafici sono identici a quelli ottenuti nel modello esteso~\ref{fig::spe3nodi}. Inoltre, se andiamo a calcolare l'errore assoluto tra le soluzioni (Figura~\ref{fig::errori3nodi}) otteniamo che tali errori sono dell'ordine della tolleranza dell'integratore cosa che non accade confrontando il modello esatto con quello chiuso alle coppie (Figura~\ref{fig::errori3nodiPair}).
\begin{figure}[!htb]
\centering
\subfloat[][Nodo 1.]
	{\resizebox{0.45\textwidth}{!}{% This file was created by matlab2tikz.
%
%The latest updates can be retrieved from
%  http://www.mathworks.com/matlabcentral/fileexchange/22022-matlab2tikz-matlab2tikz
%where you can also make suggestions and rate matlab2tikz.
%
\definecolor{mycolor1}{rgb}{0.00000,0.44700,0.74100}%
\definecolor{mycolor2}{rgb}{0.85000,0.32500,0.09800}%
\definecolor{mycolor3}{rgb}{0.92900,0.69400,0.12500}%
%
\begin{tikzpicture}

\begin{axis}[%
width=6.028in,
height=4.754in,
at={(1.011in,0.642in)},
scale only axis,
xmin=0,
xmax=60,
xlabel style={font=\color{white!15!black}},
xlabel={T},
ymin=0,
ymax=1,
axis background/.style={fill=white},
legend style={legend cell align=left, align=left, draw=white!15!black}
]
\addplot [color=mycolor1, line width=2.0pt]
  table[row sep=crcr]{%
0	0\\
0.000167459095433972	0\\
0.000334918190867944	0\\
0.000502377286301916	0\\
0.000669836381735888	0\\
0.00150713185890575	0\\
0.00234442733607561	0\\
0.00318172281324547	0\\
0.00401901829041533	0\\
0.00820549567626463	0\\
0.0123919730621139	0\\
0.0165784504479632	0\\
0.0207649278338125	0\\
0.041697314763059	0\\
0.0626297016923055	0\\
0.083562088621552	0\\
0.104494475550799	0\\
0.209156410197031	0\\
0.313818344843264	0\\
0.418480279489496	0\\
0.523142214135729	0\\
0.693500300195606	0\\
0.863858386255483	0\\
1.03421647231536	0\\
1.20457455837524	0\\
1.44051447786383	0\\
1.67645439735243	0\\
1.91239431684103	0\\
2.14833423632963	0\\
2.44722092525848	0\\
2.74610761418733	0\\
3.04499430311618	0\\
3.34388099204503	0\\
3.7078444355742	0\\
4.07180787910338	0\\
4.43577132263256	0\\
4.79973476616174	0\\
5.20292074647974	0\\
5.60610672679773	0\\
6.00929270711573	0\\
6.41247868743373	0\\
6.85758807700814	0\\
7.30269746658256	0\\
7.74780685615697	0\\
8.19291624573139	0\\
8.68220109096914	0\\
9.17148593620689	0\\
9.66077078144463	0\\
10.1500556266824	0\\
10.6416653734136	0\\
11.1332751201447	0\\
11.6248848668759	0\\
12.1164946136071	0\\
12.6079732666127	0\\
13.0994519196183	0\\
13.5909305726239	0\\
14.0824092256295	0\\
14.573895267778	0\\
15.0653813099266	0\\
15.5568673520752	0\\
16.0483533942238	0\\
16.5398390198718	0\\
17.0313246455198	0\\
17.5228102711679	0\\
18.0142958968159	0\\
18.5057815459406	0\\
18.9972671950653	0\\
19.48875284419	0\\
19.9802384933148	0\\
20.5017696853463	0\\
21.0233008773778	0\\
21.5448320694093	0\\
22.0663632614408	0\\
22.6746017709939	0\\
23.282840280547	0\\
23.8910787901001	0\\
24.4993172996532	0\\
25.0869400675273	0\\
25.6745628354014	0\\
26.2621856032755	0\\
26.8498083711495	0\\
27.5306427936611	0\\
28.2114772161728	0\\
28.8923116386844	0\\
29.573146061196	0\\
30.3571240725376	0\\
31.1411020838792	0\\
31.9250800952207	0\\
32.7090581065623	0\\
33.6393150845161	0\\
34.5695720624699	0\\
35.4998290404237	0\\
36.4300860183775	0\\
37.5675806203718	0\\
38.7050752223662	0\\
39.8425698243606	0\\
40.980064426355	0\\
42.4285780340909	0\\
43.8770916418269	0\\
45.3256052495628	0\\
46.7741188572988	0\\
48.2741188572988	0\\
49.7741188572988	0\\
51.2741188572988	0\\
52.7741188572988	0\\
54.2741188572988	0\\
55.7741188572988	0\\
57.2741188572988	0\\
58.7741188572988	0\\
59.0805891429741	0\\
59.3870594286494	0\\
59.6935297143247	0\\
60	0\\
};
\addlegendentry{$\langle\text{ S}_\text{1}\rangle\text{(t)}$}

\addplot [color=mycolor2, line width=2.0pt]
  table[row sep=crcr]{%
0	1\\
0.000167459095433972	0.999983254230669\\
0.000334918190867944	0.999966508741758\\
0.000502377286301916	0.999949763533263\\
0.000669836381735888	0.99993301860518\\
0.00150713185890575	0.999849298170771\\
0.00234442733607561	0.999765584745943\\
0.00318172281324547	0.999681878330108\\
0.00401901829041533	0.99959817892268\\
0.00820549567626463	0.999179786991109\\
0.0123919730621139	0.998761570181715\\
0.0165784504479632	0.998343528421197\\
0.0207649278338125	0.997925661636288\\
0.041697314763059	0.995838949783605\\
0.0626297016923055	0.993756601348492\\
0.083562088621552	0.991678607206836\\
0.104494475550799	0.989604958253573\\
0.209156410197031	0.979301573919075\\
0.313818344843264	0.969105464466204\\
0.418480279489496	0.959015513024293\\
0.523142214135729	0.949030614247059\\
0.693500300195606	0.933000044355593\\
0.863858386255483	0.917240255444454\\
1.03421647231536	0.901746674002992\\
1.20457455837524	0.886514802601982\\
1.44051447786383	0.865843198617418\\
1.67645439735243	0.845653612966956\\
1.91239431684103	0.82593480798817\\
2.14833423632963	0.806675802322813\\
2.44722092525848	0.782922080229594\\
2.74610761418733	0.759867824316432\\
3.04499430311618	0.737492443896854\\
3.34388099204503	0.715775937140886\\
3.7078444355742	0.690192682774706\\
4.07180787910338	0.665523833816803\\
4.43577132263256	0.641736722764718\\
4.79973476616174	0.618799807115624\\
5.20292074647974	0.594346903836115\\
5.60610672679773	0.570860310615108\\
6.00929270711573	0.548301864570512\\
6.41247868743373	0.526634848483406\\
6.85758807700814	0.503707833308423\\
7.30269746658256	0.481778962138025\\
7.74780685615697	0.460804813087566\\
8.19291624573139	0.440743766091925\\
8.68220109096914	0.419697857386959\\
9.17148593620689	0.399656933289053\\
9.66077078144463	0.380573049485789\\
10.1500556266824	0.362400430033594\\
10.6416653734136	0.345015262566064\\
11.1332751201447	0.32846412077008\\
11.6248848668759	0.312707032158958\\
12.1164946136071	0.297705839833527\\
12.6079732666127	0.283427933623719\\
13.0994519196183	0.269834808968427\\
13.5909305726239	0.25689365474737\\
14.0824092256295	0.244573149792937\\
14.573895267778	0.232843303587973\\
15.0653813099266	0.221676039881433\\
15.5568673520752	0.21104440229279\\
16.0483533942238	0.200922658547134\\
16.5398390198718	0.191286318919159\\
17.0313246455198	0.182112153427683\\
17.5228102711679	0.173378016819964\\
18.0142958968159	0.165062769471869\\
18.5057815459406	0.157146285625828\\
18.9972671950653	0.149609488922149\\
19.48875284419	0.142434186503514\\
19.9802384933148	0.135603011660711\\
20.5017696853463	0.128712120714768\\
21.0233008773778	0.122171411693478\\
21.5448320694093	0.115963108802182\\
22.0663632614408	0.110070287679395\\
22.6746017709939	0.103574886610203\\
23.282840280547	0.0974628051472713\\
23.8910787901001	0.0917114587174947\\
24.4993172996532	0.086299501584027\\
25.0869400675273	0.0813744396788814\\
25.6745628354014	0.0767304594123107\\
26.2621856032755	0.0723515428475588\\
26.8498083711495	0.0682225246959141\\
27.5306427936611	0.0637322443897468\\
28.2114772161728	0.0595375240597255\\
28.8923116386844	0.0556189510691986\\
29.573146061196	0.0519582853704699\\
30.3571240725376	0.0480403788323907\\
31.1411020838792	0.0444179283274167\\
31.9250800952207	0.0410687214707513\\
32.7090581065623	0.0379720524409922\\
33.6393150845161	0.0345988685610616\\
34.5695720624699	0.0315253797560828\\
35.4998290404237	0.028725086953008\\
36.4300860183775	0.0261735400119507\\
37.5675806203718	0.0233591425654702\\
38.7050752223662	0.0208474489027482\\
39.8425698243606	0.0186061695928961\\
40.980064426355	0.0166058694685394\\
42.4285780340909	0.0143659865246682\\
43.8770916418269	0.0124283702026242\\
45.3256052495628	0.0107528893267973\\
46.7741188572988	0.00930337062567787\\
48.2741188572988	0.00800708232919516\\
49.7741188572988	0.00689150362710516\\
51.2741188572988	0.00593189233082438\\
52.7741188572988	0.00510596666768183\\
54.2741188572988	0.00439452507303242\\
55.7741188572988	0.00378226228170326\\
57.2741188572988	0.00325559904427218\\
58.7741188572988	0.00280230646079182\\
59.0805891429741	0.00271772674151796\\
59.3870594286494	0.00263569983783545\\
59.6935297143247	0.00255614872433423\\
60	0.00247899863141475\\
};
\addlegendentry{$\langle\text{ I}_\text{1}\rangle\text{(t)}$}

\addplot [color=mycolor3, line width=2.0pt]
  table[row sep=crcr]{%
0	0\\
0.000167459095433972	1.67457693314166e-05\\
0.000334918190867944	3.34912582420355e-05\\
0.000502377286301916	5.02364667366306e-05\\
0.000669836381735888	6.6981394819754e-05\\
0.00150713185890575	0.000150701829228828\\
0.00234442733607561	0.000234415254057319\\
0.00318172281324547	0.000318121669892091\\
0.00401901829041533	0.000401821077319897\\
0.00820549567626463	0.00082021300889068\\
0.0123919730621139	0.00123842981828526\\
0.0165784504479632	0.00165647157880289\\
0.0207649278338125	0.00207433836371185\\
0.041697314763059	0.00416105021639457\\
0.0626297016923055	0.00624339865150814\\
0.083562088621552	0.00832139279316391\\
0.104494475550799	0.0103950417464269\\
0.209156410197031	0.0206984260809253\\
0.313818344843264	0.0308945355337955\\
0.418480279489496	0.0409844869757066\\
0.523142214135729	0.0509693857529409\\
0.693500300195606	0.066999955644407\\
0.863858386255483	0.0827597445555461\\
1.03421647231536	0.0982533259970083\\
1.20457455837524	0.113485197398018\\
1.44051447786383	0.134156801382582\\
1.67645439735243	0.154346387033044\\
1.91239431684103	0.17406519201183\\
2.14833423632963	0.193324197677187\\
2.44722092525848	0.217077919770406\\
2.74610761418733	0.240132175683568\\
3.04499430311618	0.262507556103146\\
3.34388099204503	0.284224062859114\\
3.7078444355742	0.309807317225294\\
4.07180787910338	0.334476166183197\\
4.43577132263256	0.358263277235282\\
4.79973476616174	0.381200192884376\\
5.20292074647974	0.405653096163885\\
5.60610672679773	0.429139689384892\\
6.00929270711573	0.451698135429488\\
6.41247868743373	0.473365151516594\\
6.85758807700814	0.496292166691577\\
7.30269746658256	0.518221037861975\\
7.74780685615697	0.539195186912434\\
8.19291624573139	0.559256233908075\\
8.68220109096914	0.580302142613041\\
9.17148593620689	0.600343066710948\\
9.66077078144463	0.619426950514211\\
10.1500556266824	0.637599569966406\\
10.6416653734136	0.654984737433936\\
11.1332751201447	0.67153587922992\\
11.6248848668759	0.687292967841042\\
12.1164946136071	0.702294160166473\\
12.6079732666127	0.716572066376281\\
13.0994519196183	0.730165191031573\\
13.5909305726239	0.74310634525263\\
14.0824092256295	0.755426850207063\\
14.573895267778	0.767156696412027\\
15.0653813099266	0.778323960118567\\
15.5568673520752	0.78895559770721\\
16.0483533942238	0.799077341452866\\
16.5398390198718	0.808713681080841\\
17.0313246455198	0.817887846572317\\
17.5228102711679	0.826621983180036\\
18.0142958968159	0.834937230528131\\
18.5057815459406	0.842853714374172\\
18.9972671950653	0.850390511077851\\
19.48875284419	0.857565813496486\\
19.9802384933148	0.864396988339289\\
20.5017696853463	0.871287879285232\\
21.0233008773778	0.877828588306522\\
21.5448320694093	0.884036891197818\\
22.0663632614408	0.889929712320605\\
22.6746017709939	0.896425113389797\\
23.282840280547	0.902537194852729\\
23.8910787901001	0.908288541282505\\
24.4993172996532	0.913700498415973\\
25.0869400675273	0.918625560321119\\
25.6745628354014	0.923269540587689\\
26.2621856032755	0.927648457152441\\
26.8498083711495	0.931777475304086\\
27.5306427936611	0.936267755610253\\
28.2114772161728	0.940462475940275\\
28.8923116386844	0.944381048930801\\
29.573146061196	0.94804171462953\\
30.3571240725376	0.951959621167609\\
31.1411020838792	0.955582071672583\\
31.9250800952207	0.958931278529249\\
32.7090581065623	0.962027947559008\\
33.6393150845161	0.965401131438938\\
34.5695720624699	0.968474620243917\\
35.4998290404237	0.971274913046992\\
36.4300860183775	0.973826459988049\\
37.5675806203718	0.97664085743453\\
38.7050752223662	0.979152551097252\\
39.8425698243606	0.981393830407104\\
40.980064426355	0.983394130531461\\
42.4285780340909	0.985634013475332\\
43.8770916418269	0.987571629797376\\
45.3256052495628	0.989247110673203\\
46.7741188572988	0.990696629374322\\
48.2741188572988	0.991992917670805\\
49.7741188572988	0.993108496372895\\
51.2741188572988	0.994068107669176\\
52.7741188572988	0.994894033332318\\
54.2741188572988	0.995605474926968\\
55.7741188572988	0.996217737718297\\
57.2741188572988	0.996744400955728\\
58.7741188572988	0.997197693539208\\
59.0805891429741	0.997282273258482\\
59.3870594286494	0.997364300162165\\
59.6935297143247	0.997443851275666\\
60	0.997521001368585\\
};
\addlegendentry{$\langle\text{ R}_\text{1}\rangle\text{(t)}$}

\end{axis}
\end{tikzpicture}%}}
 \quad 
\subfloat[][Nodo 2.]
{\resizebox{0.45\textwidth}{!}{ % This file was created by matlab2tikz.
%
\definecolor{mycolor1}{rgb}{0.00000,0.44700,0.74100}%
\definecolor{mycolor2}{rgb}{0.85000,0.32500,0.09800}%
\definecolor{mycolor3}{rgb}{0.92900,0.69400,0.12500}%
%
\begin{tikzpicture}

\begin{axis}[%
width=6.028in,
height=4.754in,
at={(1.011in,0.642in)},
scale only axis,
xmin=0,
xmax=60,
xlabel style={font=\color{white!15!black}},
xlabel={T},
ymin=0,
ymax=1,
axis background/.style={fill=white},
legend style={legend cell align=left, align=left, draw=white!15!black}
]
\addplot [color=mycolor1, line width=2.0pt]
  table[row sep=crcr]{%
0	1\\
0.000167459095433972	0.999949763953885\\
0.000334918190867944	0.999899531272651\\
0.000502377286301916	0.999849301956071\\
0.000669836381735888	0.999799076003922\\
0.00150713185890575	0.999547996701732\\
0.00234442733607561	0.999297001476487\\
0.00318172281324547	0.999046090300034\\
0.00401901829041533	0.998795263144227\\
0.00820549567626463	0.997542386690479\\
0.0123919730621139	0.996291606536589\\
0.0165784504479632	0.995042919175067\\
0.0207649278338125	0.993796321104271\\
0.041697314763059	0.987594547947566\\
0.0626297016923055	0.981444485159969\\
0.083562088621552	0.975345701619957\\
0.104494475550799	0.969297769753325\\
0.209156410197031	0.939806168163308\\
0.313818344843264	0.911523738353683\\
0.418480279489496	0.88440090361763\\
0.523142214135729	0.858390119923957\\
0.693500300195606	0.818313433705788\\
0.863858386255483	0.780876732324173\\
1.03421647231536	0.745906111440154\\
1.20457455837524	0.713239122026401\\
1.44051447786383	0.671520079694493\\
1.67645439735243	0.633558229601977\\
1.91239431684103	0.599015201830698\\
2.14833423632963	0.567583098707322\\
2.44722092525848	0.531796402639468\\
2.74610761418733	0.500042313863994\\
3.04499430311618	0.471866419354097\\
3.34388099204503	0.446865510843087\\
3.7078444355742	0.420193398001698\\
4.07180787910338	0.397134929825741\\
4.43577132263256	0.377200512141872\\
4.79973476616174	0.359966886808121\\
5.20292074647974	0.343588256362086\\
5.60610672679773	0.329649082515803\\
6.00929270711573	0.317786029057104\\
6.41247868743373	0.307689877859709\\
6.85758807700814	0.298280993976801\\
7.30269746658256	0.290406643306569\\
7.74780685615697	0.283816554998098\\
8.19291624573139	0.278301271044437\\
8.68220109096914	0.273270646891322\\
9.17148593620689	0.269134229409004\\
9.66077078144463	0.265733070599577\\
10.1500556266824	0.262936476757549\\
10.6416653734136	0.260627097135723\\
11.1332751201447	0.258729980919448\\
11.6248848668759	0.257171531864214\\
12.1164946136071	0.255891292369106\\
12.6079732666127	0.25483985092165\\
13.0994519196183	0.253976064396396\\
13.5909305726239	0.253266440902041\\
14.0824092256295	0.252683466795874\\
14.573895267778	0.252204531825498\\
15.0653813099266	0.251811075235949\\
15.5568673520752	0.251487841268248\\
16.0483533942238	0.251222296926601\\
16.5398390198718	0.251004146132859\\
17.0313246455198	0.25082493005874\\
17.5228102711679	0.250677699773121\\
18.0142958968159	0.250556746569625\\
18.5057815459406	0.250457380612448\\
18.9972671950653	0.250375749099364\\
19.48875284419	0.250308686856743\\
19.9802384933148	0.250253593620033\\
20.5017696853463	0.25020584440604\\
21.0233008773778	0.250167085903632\\
21.5448320694093	0.250135625253087\\
22.0663632614408	0.250110088345249\\
22.6746017709939	0.250086313648461\\
23.282840280547	0.250067673341702\\
23.8910787901001	0.250053058602301\\
24.4993172996532	0.250041600070263\\
25.0869400675273	0.250032886207585\\
25.6745628354014	0.250025997604692\\
26.2621856032755	0.250020551922669\\
26.8498083711495	0.250016246937907\\
27.5306427936611	0.250012373669365\\
28.2114772161728	0.250009423801076\\
28.8923116386844	0.250007177183036\\
29.573146061196	0.250005466132144\\
30.3571240725376	0.250003994735166\\
31.1411020838792	0.25000291941862\\
31.9250800952207	0.250002133576394\\
32.7090581065623	0.250001559265355\\
33.6393150845161	0.250001074767862\\
34.5695720624699	0.250000740809287\\
35.4998290404237	0.250000510634068\\
36.4300860183775	0.250000351984084\\
37.5675806203718	0.250000223315542\\
38.7050752223662	0.250000141681369\\
39.8425698243606	0.250000089889573\\
40.980064426355	0.250000057031013\\
42.4285780340909	0.250000031950999\\
43.8770916418269	0.250000017900344\\
45.3256052495628	0.250000010028615\\
46.7741188572988	0.250000005618511\\
48.2741188572988	0.250000003083582\\
49.7741188572988	0.250000001692313\\
51.2741188572988	0.250000000928761\\
52.7741188572988	0.250000000509717\\
54.2741188572988	0.250000000279742\\
55.7741188572988	0.250000000153529\\
57.2741188572988	0.250000000084261\\
58.7741188572988	0.250000000046246\\
59.0805891429741	0.250000000040908\\
59.3870594286494	0.250000000036189\\
59.6935297143247	0.250000000032017\\
60	0.250000000028325\\
};
\addlegendentry{Suscettibili}

\addplot [color=mycolor2, line width=2.0pt]
  table[row sep=crcr]{%
0	0\\
0.000167459095433972	5.02356254883503e-05\\
0.000334918190867944	0.000100467044890239\\
0.000502377286301916	0.000150694258501483\\
0.000669836381735888	0.000200917266617879\\
0.00150713185890575	0.000451969235128613\\
0.00234442733607561	0.000702916110626107\\
0.00318172281324547	0.000953757930062382\\
0.00401901829041533	0.00120449473037694\\
0.00820549567626463	0.00245660473713794\\
0.0123919730621139	0.0037060947995958\\
0.0165784504479632	0.00495296952110829\\
0.0207649278338125	0.00619723349726048\\
0.041697314763059	0.0123795525201794\\
0.0626297016923055	0.0184972878018344\\
0.083562088621552	0.0245510050468531\\
0.104494475550799	0.0305412652491206\\
0.209156410197031	0.0595600164160656\\
0.313818344843264	0.0870738133968313\\
0.418480279489496	0.11314764156286\\
0.523142214135729	0.137843787680198\\
0.693500300195606	0.175248799950433\\
0.863858386255483	0.20940461311705\\
1.03421647231536	0.24053852573524\\
1.20457455837524	0.268862639870873\\
1.44051447786383	0.303816428270863\\
1.67645439735243	0.334242643902881\\
1.91239431684103	0.360581207071262\\
2.14833423632963	0.383231670501034\\
2.44722092525848	0.407193551056282\\
2.74610761418733	0.426478083727982\\
3.04499430311618	0.441670555903295\\
3.34388099204503	0.453288588451717\\
3.7078444355742	0.463268170334781\\
4.07180787910338	0.469343954321767\\
4.43577132263256	0.472136050361024\\
4.79973476616174	0.47217728850265\\
5.20292074647974	0.469562587129115\\
5.60610672679773	0.464661571875658\\
6.00929270711573	0.457920506318279\\
6.41247868743373	0.449715005531737\\
6.85758807700814	0.43933320847839\\
7.30269746658256	0.427903489864195\\
7.74780685615697	0.415716091047561\\
8.19291624573139	0.403008727934979\\
8.68220109096914	0.388670372122671\\
9.17148593620689	0.374144696564468\\
9.66077078144463	0.359595644501902\\
10.1500556266824	0.34515177823617\\
10.6416653734136	0.330845829479154\\
11.1332751201447	0.316824197584907\\
11.6248848668759	0.303145002044657\\
12.1164946136071	0.289850763480305\\
12.6079732666127	0.276974817156347\\
13.0994519196183	0.264533425905096\\
13.5909305726239	0.252538404627472\\
14.0824092256295	0.240995172317214\\
14.573895267778	0.229903937525938\\
15.0653813099266	0.219261297662104\\
15.5568673520752	0.209060612874046\\
16.0483533942238	0.199292906971955\\
16.5398390198718	0.189947461116218\\
17.0313246455198	0.181012262969611\\
17.5228102711679	0.172474412399053\\
18.0142958968159	0.164320418697294\\
18.5057815459406	0.156536444368864\\
18.9972671950653	0.14910850018116\\
19.48875284419	0.142022596979649\\
19.9802384933148	0.135264865726133\\
20.5017696853463	0.128437664343194\\
21.0233008773778	0.121948644966075\\
21.5448320694093	0.115782270354763\\
22.0663632614408	0.109923482519257\\
22.6746017709939	0.10345982332505\\
23.282840280547	0.0973726160305641\\
23.8910787901001	0.0916407202767276\\
24.4993172996532	0.0862440111821377\\
25.0869400675273	0.0813305960799503\\
25.6745628354014	0.0766958145095452\\
26.2621856032755	0.0723241359832878\\
26.8498083711495	0.0682008386504439\\
27.5306427936611	0.0637157698870477\\
28.2114772161728	0.0595250058105512\\
28.8923116386844	0.0556093884944659\\
29.573146061196	0.0519509701650054\\
30.3571240725376	0.048035099872587\\
31.1411020838792	0.0444141195590803\\
31.9250800952207	0.0410658960157937\\
32.7090581065623	0.0379699370569364\\
33.6393150845161	0.0345975319159539\\
34.5695720624699	0.0315245537416839\\
35.4998290404237	0.0287244495344917\\
36.4300860183775	0.0261730106284325\\
37.5675806203718	0.0233590495699716\\
38.7050752223662	0.0208475969715238\\
39.8425698243606	0.0186061438720866\\
40.980064426355	0.0166056706644438\\
42.4285780340909	0.0143664338020051\\
43.8770916418269	0.0124291459095339\\
45.3256052495628	0.010753093885223\\
46.7741188572988	0.0093030527860469\\
48.2741188572988	0.0080072141242616\\
49.7741188572988	0.006891874310315\\
51.2741188572988	0.00593189183887918\\
52.7741188572988	0.00510562696752385\\
54.2741188572988	0.0043944540448391\\
55.7741188572988	0.00378234175544107\\
57.2741188572988	0.00325549178767263\\
58.7741188572988	0.00280202781730321\\
59.0805891429741	0.00271745647006354\\
59.3870594286494	0.00263543777903238\\
59.6935297143247	0.00255589465143586\\
60	0.00247875226546527\\
};
\addlegendentry{Infetti}

\addplot [color=mycolor3, line width=2.0pt]
  table[row sep=crcr]{%
0	0\\
0.000167459095433972	4.2062652069497e-10\\
0.000334918190867944	1.68245901753647e-09\\
0.000502377286301916	3.78542716020659e-09\\
0.000669836381735888	6.72946041634635e-09\\
0.00150713185890575	3.40631394982888e-08\\
0.00234442733607561	8.24128865253067e-08\\
0.00318172281324547	1.51769903511704e-07\\
0.00401901829041533	2.42125395660559e-07\\
0.00820549567626463	1.00857238349156e-06\\
0.0123919730621139	2.29866381498559e-06\\
0.0165784504479632	4.11130382500357e-06\\
0.0207649278338125	6.4453984686668e-06\\
0.041697314763059	2.5899532254868e-05\\
0.0626297016923055	5.8227038196506e-05\\
0.083562088621552	0.000103293333190105\\
0.104494475550799	0.000160964997554532\\
0.209156410197031	0.000633815420626169\\
0.313818344843264	0.00140244824948524\\
0.418480279489496	0.00245145481951062\\
0.523142214135729	0.00376609239584499\\
0.693500300195606	0.00643776634377838\\
0.863858386255483	0.00971865455877702\\
1.03421647231536	0.0135553628246058\\
1.20457455837524	0.0178982381027259\\
1.44051447786383	0.0246634920346438\\
1.67645439735243	0.0321991264951424\\
1.91239431684103	0.04040359109804\\
2.14833423632963	0.0491852307916441\\
2.44722092525848	0.0610100463042492\\
2.74610761418733	0.073479602408024\\
3.04499430311618	0.0864630247426084\\
3.34388099204503	0.0998459007051964\\
3.7078444355742	0.116538431663522\\
4.07180787910338	0.133521115852492\\
4.43577132263256	0.150663437497104\\
4.79973476616174	0.167855824689228\\
5.20292074647974	0.186849156508798\\
5.60610672679773	0.205689345608539\\
6.00929270711573	0.224293464624617\\
6.41247868743373	0.242595116608554\\
6.85758807700814	0.262385797544809\\
7.30269746658256	0.281689866829236\\
7.74780685615697	0.30046735395434\\
8.19291624573139	0.318690001020583\\
8.68220109096914	0.338058980986007\\
9.17148593620689	0.356721074026528\\
9.66077078144463	0.374671284898521\\
10.1500556266824	0.391911745006281\\
10.6416653734136	0.408527073385124\\
11.1332751201447	0.424445821495645\\
11.6248848668759	0.439683466091129\\
12.1164946136071	0.454257944150589\\
12.6079732666127	0.468185331922002\\
13.0994519196183	0.481490509698508\\
13.5909305726239	0.494195154470487\\
14.0824092256295	0.506321360886912\\
14.573895267778	0.517891530648564\\
15.0653813099266	0.528927627101947\\
15.5568673520752	0.539451545857706\\
16.0483533942238	0.549484796101445\\
16.5398390198718	0.559048392750923\\
17.0313246455198	0.568162806971649\\
17.5228102711679	0.576847887827825\\
18.0142958968159	0.585122834733081\\
18.5057815459406	0.593006175018688\\
18.9972671950653	0.600515750719475\\
19.48875284419	0.607668716163609\\
19.9802384933148	0.614481540653834\\
20.5017696853463	0.621356491250766\\
21.0233008773778	0.627884269130292\\
21.5448320694093	0.63408210439215\\
22.0663632614408	0.639966429135494\\
22.6746017709939	0.646453863026489\\
23.282840280547	0.652559710627734\\
23.8910787901001	0.658306221120972\\
24.4993172996532	0.663714388747599\\
25.0869400675273	0.668636517712465\\
25.6745628354014	0.673278187885763\\
26.2621856032755	0.677655312094043\\
26.8498083711495	0.681782914411649\\
27.5306427936611	0.686271856443588\\
28.2114772161728	0.690465570388373\\
28.8923116386844	0.694383434322498\\
29.573146061196	0.698043563702851\\
30.3571240725376	0.701960905392247\\
31.1411020838792	0.705582961022299\\
31.9250800952207	0.708931970407813\\
32.7090581065623	0.712028503677709\\
33.6393150845161	0.715401393316184\\
34.5695720624699	0.718474705449029\\
35.4998290404237	0.72127503983144\\
36.4300860183775	0.723826637387483\\
37.5675806203718	0.726640727114487\\
38.7050752223662	0.729152261347108\\
39.8425698243606	0.73139376623834\\
40.980064426355	0.733394272304543\\
42.4285780340909	0.735633534246996\\
43.8770916418269	0.737570836190122\\
45.3256052495628	0.739246896086162\\
46.7741188572988	0.740696941595442\\
48.2741188572988	0.741992782792156\\
49.7741188572988	0.743108123997372\\
51.2741188572988	0.74406810723236\\
52.7741188572988	0.74489437252276\\
54.2741188572988	0.745605545675419\\
55.7741188572988	0.74621765809103\\
57.2741188572988	0.746744508128066\\
58.7741188572988	0.74719797213645\\
59.0805891429741	0.747282543489028\\
59.3870594286494	0.747364562184778\\
59.6935297143247	0.747444105316547\\
60	0.74752124770621\\
};
\addlegendentry{Rimossi}

\end{axis}
\end{tikzpicture}%}}
\\
\subfloat[][Nodo 3.]
{\resizebox{0.45\textwidth}{!}
{% This file was created by matlab2tikz.
%
\definecolor{mycolor1}{rgb}{0.00000,0.44700,0.74100}%
\definecolor{mycolor2}{rgb}{0.85000,0.32500,0.09800}%
\definecolor{mycolor3}{rgb}{0.92900,0.69400,0.12500}%
%
\begin{tikzpicture}

\begin{axis}[%
width=0.39\columnwidth,
height=1.7in,
at={(1.011in,0.642in)},
scale only axis,
xmin=0,
xmax=60,
xlabel style={font=\color{white!15!black}},
xlabel={T},
ymin=0,
ymax=1,
axis background/.style={fill=white},
legend style={legend cell align=left, align=left,draw=none,fill=none}
]
\addplot [color=mycolor1, line width=2.0pt]
  table[row sep=crcr]{%
0	1\\
0.000167459095433972	0.999999998738142\\
0.000334918190867944	0.999999994952792\\
0.000502377286301916	0.999999988644289\\
0.000669836381735888	0.999999979812971\\
0.00150713185890575	0.999999897825981\\
0.00234442733607561	0.999999752819296\\
0.00318172281324547	0.99999954483513\\
0.00401901829041533	0.999999273915677\\
0.00820549567626463	0.999996976764423\\
0.0123919730621139	0.999993112547904\\
0.0165784504479632	0.999987686516502\\
0.0207649278338125	0.999980703907465\\
0.041697314763059	0.999922624629933\\
0.0626297016923055	0.999826409001474\\
0.083562088621552	0.999692696902672\\
0.104494475550799	0.999522120336272\\
0.209156410197031	0.998137840759209\\
0.313818344843264	0.995922289251673\\
0.418480279489496	0.99294595806566\\
0.523142214135729	0.989274956406301\\
0.693500300195606	0.981972736387244\\
0.863858386255483	0.973238244928869\\
1.03421647231536	0.96329186479314\\
1.20457455837524	0.952331160031066\\
1.44051447786383	0.935801793669581\\
1.67645439735243	0.91807403785708\\
1.91239431684103	0.899497808251488\\
2.14833423632963	0.880369716738813\\
2.44722092525848	0.85573272075568\\
2.74610761418733	0.831024667672379\\
3.04499430311618	0.80657440850026\\
3.34388099204503	0.782637584956939\\
3.7078444355742	0.754460245699215\\
4.07180787910338	0.727582746559292\\
4.43577132263256	0.702170099088371\\
4.79973476616174	0.678318735322167\\
5.20292074647974	0.653770875562649\\
5.60610672679773	0.631193188152194\\
6.00929270711573	0.610543351294101\\
6.41247868743373	0.591747941764318\\
6.85758807700814	0.573038094092212\\
7.30269746658256	0.55632823114237\\
7.74780685615697	0.541463655325261\\
8.19291624573139	0.52828693489063\\
8.68220109096914	0.515565114984503\\
9.17148593620689	0.504497465365703\\
9.66077078144463	0.494897878606301\\
10.1500556266824	0.486594143827427\\
10.6416653734136	0.479397325430951\\
11.1332751201447	0.473205468386526\\
11.6248848668759	0.467889117778627\\
12.1164946136071	0.463333012080587\\
12.6079732666127	0.459436100987076\\
13.0994519196183	0.456107326754063\\
13.5909305726239	0.453268021934224\\
14.0824092256295	0.450849503017159\\
14.573895267778	0.448791983002455\\
15.0653813099266	0.447043667751192\\
15.5568673520752	0.445559725690352\\
16.0483533942238	0.444301478682893\\
16.5398390198718	0.443235634289295\\
17.0313246455198	0.44233359303735\\
17.5228102711679	0.441570836126662\\
18.0142958968159	0.44092637905947\\
18.5057815459406	0.440382291061817\\
18.9972671950653	0.439923273546552\\
19.48875284419	0.439536291641761\\
19.9802384933148	0.439210253494403\\
20.5017696853463	0.438920435673544\\
21.0233008773778	0.438679123594922\\
21.5448320694093	0.438478325927771\\
22.0663632614408	0.438311341046057\\
22.6746017709939	0.438151873497914\\
23.282840280547	0.438023443267924\\
23.8910787901001	0.437920082088796\\
24.4993172996532	0.437836951972973\\
25.0869400675273	0.437772168808728\\
25.6745628354014	0.437719741210863\\
26.2621856032755	0.437677335424623\\
26.8498083711495	0.43764305338053\\
27.5306427936611	0.437611476784113\\
28.2114772161728	0.437586825601741\\
28.8923116386844	0.437567592400895\\
29.573146061196	0.437552594775992\\
30.3571240725376	0.437539376666326\\
31.1411020838792	0.437529463762673\\
31.9250800952207	0.437522034519481\\
32.7090581065623	0.437516470005118\\
33.6393150845161	0.437511652392799\\
34.5695720624699	0.437508238472235\\
35.4998290404237	0.437505821188862\\
36.4300860183775	0.43750411076645\\
37.5675806203718	0.437502684273934\\
38.7050752223662	0.437501751375406\\
39.8425698243606	0.43750114183208\\
40.980064426355	0.437500743900978\\
42.4285780340909	0.437500430644641\\
43.8770916418269	0.437500249042731\\
45.3256052495628	0.437500143882113\\
46.7741188572988	0.437500083050415\\
48.2741188572988	0.437500046967414\\
49.7741188572988	0.437500026537861\\
51.2741188572988	0.437500014982207\\
52.7741188572988	0.437500008451793\\
54.2741188572988	0.437500004764356\\
55.7741188572988	0.437500002683857\\
57.2741188572988	0.437500001510883\\
58.7741188572988	0.437500000850036\\
59.0805891429741	0.437500000755694\\
59.3870594286494	0.437500000671844\\
59.6935297143247	0.437500000597315\\
60	0.437500000531033\\
};
\addlegendentry{Suscettibili}

\addplot [color=mycolor2, line width=2.0pt]
  table[row sep=crcr]{%
0	0\\
0.000167459095433972	1.26185129491355e-09\\
0.000334918190867944	5.0471516169101e-09\\
0.000502377286301916	1.13555206521346e-08\\
0.000669836381735888	2.01865781270664e-08\\
0.00150713185890575	1.02168885342189e-07\\
0.00234442733607561	2.47161385699258e-07\\
0.00318172281324547	4.55116590379602e-07\\
0.00401901829041533	7.25987035727091e-07\\
0.00820549567626463	3.02240838600845e-06\\
0.0123919730621139	6.88460564632564e-06\\
0.0165784504479632	1.23066741721052e-05\\
0.0207649278338125	1.92827249108082e-05\\
0.041697314763059	7.72676278348998e-05\\
0.0626297016923055	0.000173227626505261\\
0.083562088621552	0.000306444129914356\\
0.104494475550799	0.000476207887416148\\
0.209156410197031	0.00184906356709497\\
0.313818344843264	0.00403449941495126\\
0.418480279489496	0.00695393442627649\\
0.523142214135729	0.0105339657290871\\
0.693500300195606	0.0175985851398963\\
0.863858386255483	0.0259636855360641\\
1.03421647231536	0.0353888174512076\\
1.20457455837524	0.0456602151891865\\
1.44051447786383	0.0609341164059151\\
1.67645439735243	0.077035489695418\\
1.91239431684103	0.093599331233309\\
2.14833423632963	0.110321813556605\\
2.44722092525848	0.13134632602151\\
2.74610761418733	0.151820840695471\\
3.04499430311618	0.171437763923712\\
3.34388099204503	0.189970652685552\\
3.7078444355742	0.210847907404192\\
4.07180787910338	0.229701888735119\\
4.43577132263256	0.246443097051735\\
4.79973476616174	0.261052528214549\\
5.20292074647974	0.274789676074337\\
5.60610672679773	0.286053070188536\\
6.00929270711573	0.294982066983248\\
6.41247868743373	0.301740961039022\\
6.85758807700814	0.306896743665575\\
7.30269746658256	0.309872491647603\\
7.74780685615697	0.310914438945311\\
8.19291624573139	0.310260752458576\\
8.68220109096914	0.307854199034656\\
9.17148593620689	0.303948972152535\\
9.66077078144463	0.298798210293078\\
10.1500556266824	0.29262939655715\\
10.6416653734136	0.285609826040275\\
11.1332751201447	0.27794688732232\\
11.6248848668759	0.269797710204035\\
12.1164946136071	0.261298039741962\\
12.6079732666127	0.252566533428562\\
13.0994519196183	0.243699721296075\\
13.5909305726239	0.234780816283881\\
14.0824092256295	0.225879301756875\\
14.573895267778	0.217052492014829\\
15.0653813099266	0.20834748256313\\
15.5568673520752	0.199802153221824\\
16.0483533942238	0.191446565654698\\
16.5398390198718	0.18330409486335\\
17.0313246455198	0.175392402311884\\
17.5228102711679	0.167724330669958\\
18.0142958968159	0.160308659854292\\
18.5057815459406	0.153150770232222\\
18.9972671950653	0.146253217885122\\
19.48875284419	0.139616228314043\\
19.9802384933148	0.133238121353628\\
20.5017696853463	0.126749594517843\\
21.0233008773778	0.120543566076478\\
21.5448320694093	0.114613461037488\\
22.0663632614408	0.108951782803096\\
22.6746017709939	0.102676972309626\\
23.282840280547	0.0967423650150338\\
23.8910787901001	0.0911336694273004\\
24.4993172996532	0.0858363419551039\\
25.0869400675273	0.0810005905425645\\
25.6745628354014	0.0764288238330873\\
26.2621856032755	0.0721082406731258\\
26.8498083711495	0.0680263477476442\\
27.5306427936611	0.0635795078442614\\
28.2114772161728	0.0594186621426391\\
28.8923116386844	0.0555264424763088\\
29.573146061196	0.0518863099291597\\
30.3571240725376	0.0479865923859841\\
31.1411020838792	0.0443777539608027\\
31.9250800952207	0.0410386502328789\\
32.7090581065623	0.0379495363154676\\
33.6393150845161	0.0345830701600837\\
34.5695720624699	0.031514309921325\\
35.4998290404237	0.0287171985834105\\
36.4300860183775	0.0261678815905834\\
37.5675806203718	0.0233556938536014\\
38.7050752223662	0.0208454034856839\\
39.8425698243606	0.0186047113188866\\
40.980064426355	0.0166047358274855\\
42.4285780340909	0.0143658915601497\\
43.8770916418269	0.0124288317529024\\
45.3256052495628	0.0107529120710205\\
46.7741188572988	0.00930294767067081\\
48.2741188572988	0.00800715458462473\\
49.7741188572988	0.00689184061881361\\
51.2741188572988	0.0059318727913637\\
52.7741188572988	0.00510561620818277\\
54.2741188572988	0.00439444797210539\\
55.7741188572988	0.00378233833049425\\
57.2741188572988	0.00325548985742313\\
58.7741188572988	0.00280202673016872\\
59.0805891429741	0.00271745550338019\\
59.3870594286494	0.00263543691942989\\
59.6935297143247	0.00255589388703257\\
60	0.00247875158574684\\
};
\addlegendentry{Infetti}

\addplot [color=mycolor3, line width=2.0pt]
  table[row sep=crcr]{%
0	0\\
0.000167459095433972	6.98961562976097e-15\\
0.000334918190867944	5.62569549326689e-14\\
0.000502377286301916	1.90163984227985e-13\\
0.000669836381735888	4.50699268636422e-13\\
0.00150713185890575	5.13323695986094e-12\\
0.00234442733607561	1.93184278459606e-11\\
0.00318172281324547	4.82799703089654e-11\\
0.00401901829041533	9.72877164745839e-11\\
0.00820549567626463	8.27191095820724e-10\\
0.0123919730621139	2.84644965709305e-09\\
0.0165784504479632	6.80932573371384e-09\\
0.0207649278338125	1.33676238959192e-08\\
0.041697314763059	1.07742232456517e-07\\
0.0626297016923055	3.63372021057321e-07\\
0.083562088621552	8.58967413937272e-07\\
0.104494475550799	1.67177631188366e-06\\
0.209156410197031	1.3095673695947e-05\\
0.313818344843264	4.32113333760442e-05\\
0.418480279489496	0.000100107508063741\\
0.523142214135729	0.000191077864611911\\
0.693500300195606	0.000428678472859663\\
0.863858386255483	0.000798069535066736\\
1.03421647231536	0.00131931775565243\\
1.20457455837524	0.00200862477974788\\
1.44051447786383	0.0032640899245041\\
1.67645439735243	0.00489047244750238\\
1.91239431684103	0.00690286051520264\\
2.14833423632963	0.00930846970458167\\
2.44722092525848	0.0129209532228095\\
2.74610761418733	0.0171544916321505\\
3.04499430311618	0.0219878275760286\\
3.34388099204503	0.0273917623575095\\
3.7078444355742	0.0346918468965934\\
4.07180787910338	0.0427153647055895\\
4.43577132263256	0.0513868038598941\\
4.79973476616174	0.060628736463284\\
5.20292074647974	0.0714394483630145\\
5.60610672679773	0.0827537416592705\\
6.00929270711573	0.0944745817226505\\
6.41247868743373	0.10651109719666\\
6.85758807700814	0.120065162242213\\
7.30269746658256	0.133799277210027\\
7.74780685615697	0.147621905729427\\
8.19291624573139	0.161452312650794\\
8.68220109096914	0.176580685980841\\
9.17148593620689	0.191553562481763\\
9.66077078144463	0.206303911100621\\
10.1500556266824	0.220776459615423\\
10.6416653734136	0.234992848528774\\
11.1332751201447	0.248847644291154\\
11.6248848668759	0.262313172017338\\
12.1164946136071	0.275368948177451\\
12.6079732666127	0.287997365584362\\
13.0994519196183	0.300192951949862\\
13.5909305726239	0.311951161781895\\
14.0824092256295	0.323271195225966\\
14.573895267778	0.334155524982716\\
15.0653813099266	0.344608849685678\\
15.5568673520752	0.354638121087824\\
16.0483533942238	0.364251955662409\\
16.5398390198718	0.373460270847355\\
17.0313246455198	0.382274004650766\\
17.5228102711679	0.39070483320338\\
18.0142958968159	0.398764961086238\\
18.5057815459406	0.406466938705961\\
18.9972671950653	0.413823508568326\\
19.48875284419	0.420847480044196\\
19.9802384933148	0.427551625151969\\
20.5017696853463	0.434329969808614\\
21.0233008773778	0.4407773103286\\
21.5448320694093	0.446908213034741\\
22.0663632614408	0.452736876150846\\
22.6746017709939	0.45917115419246\\
23.282840280547	0.465234191717042\\
23.8910787901001	0.470946248483904\\
24.4993172996532	0.476326706071924\\
25.0869400675273	0.481227240648708\\
25.6745628354014	0.48585143495605\\
26.2621856032755	0.490214423902251\\
26.8498083711495	0.494330598871826\\
27.5306427936611	0.498809015371625\\
28.2114772161728	0.50299451225562\\
28.8923116386844	0.506905965122796\\
29.573146061196	0.510561095294848\\
30.3571240725376	0.51447403094769\\
31.1411020838792	0.518092782276524\\
31.9250800952207	0.52143931524764\\
32.7090581065623	0.524533993679415\\
33.6393150845161	0.527905277447117\\
34.5695720624699	0.53097745160644\\
35.4998290404237	0.533776980227728\\
36.4300860183775	0.536328007642967\\
37.5675806203718	0.539141621872465\\
38.7050752223662	0.54165284513891\\
39.8425698243606	0.543894146849034\\
40.980064426355	0.545894520271536\\
42.4285780340909	0.548133677795209\\
43.8770916418269	0.550070919204366\\
45.3256052495628	0.551746944046866\\
46.7741188572988	0.553196969278914\\
48.2741188572988	0.554492798447961\\
49.7741188572988	0.555608132843326\\
51.2741188572988	0.556568112226429\\
52.7741188572988	0.557394375340024\\
54.2741188572988	0.558105547263538\\
55.7741188572988	0.558717658985649\\
57.2741188572988	0.559244508631694\\
58.7741188572988	0.559697972419796\\
59.0805891429741	0.559782543740926\\
59.3870594286494	0.559864562408726\\
59.6935297143247	0.559944105515652\\
60	0.56002124788322\\
};
\addlegendentry{Rimossi}

\end{axis}
\end{tikzpicture}%}
}
\quad 
\subfloat[][Prevalenza.]
{\resizebox{0.45\textwidth}{!}{ % This file was created by matlab2tikz.
%
%The latest updates can be retrieved from
%  http://www.mathworks.com/matlabcentral/fileexchange/22022-matlab2tikz-matlab2tikz
%where you can also make suggestions and rate matlab2tikz.
%
\definecolor{mycolor1}{rgb}{0.00000,0.44700,0.74100}%
%
\begin{tikzpicture}

\begin{axis}[%
width=6.028in,
height=4.754in,
at={(1.011in,0.642in)},
scale only axis,
xmin=0,
xmax=60,
ymin=0,
ymax=0.6,
axis background/.style={fill=white},
axis x line*=bottom,
axis y line*=left
]
\addplot [color=mycolor1, line width=2.0pt, forget plot]
  table[row sep=crcr]{%
0	0.333333333333333\\
0.000167459095433972	0.333346357799813\\
0.000334918190867944	0.333359382006611\\
0.000502377286301916	0.333372405953696\\
0.000669836381735888	0.33338542964104\\
0.00150713185890575	0.333450544180612\\
0.00234442733607561	0.333515652222244\\
0.00318172281324547	0.333580753762272\\
0.00401901829041533	0.333645848797031\\
0.00820549567626463	0.333971226263809\\
0.0123919730621139	0.334296440551265\\
0.0165784504479632	0.334621491205169\\
0.0207649278338125	0.334946377773328\\
0.041697314763059	0.336568333634319\\
0.0626297016923055	0.33818612068879\\
0.083562088621552	0.339799684673782\\
0.104494475550799	0.341408972533459\\
0.209156410197031	0.34938946665612\\
0.313818344843264	0.357255789085086\\
0.418480279489496	0.365002602843887\\
0.523142214135729	0.372625132859308\\
0.693500300195606	0.38475568848826\\
0.863858386255483	0.39653142957701\\
1.03421647231536	0.407941031431478\\
1.20457455837524	0.418975299094782\\
1.44051447786383	0.433624794438073\\
1.67645439735243	0.447533637142379\\
1.91239431684103	0.460699895656536\\
2.14833423632963	0.473123944927263\\
2.44722092525848	0.487803848743562\\
2.74610761418733	0.501326422004098\\
3.04499430311618	0.513722744744762\\
3.34388099204503	0.525021947751027\\
3.7078444355742	0.53734651471454\\
4.07180787910338	0.548172156268076\\
4.43577132263256	0.55757753265428\\
4.79973476616174	0.565631517927584\\
5.20292074647974	0.573059119352328\\
5.60610672679773	0.579022490830576\\
6.00929270711573	0.583622708347856\\
6.41247868743373	0.586949280746428\\
6.85758807700814	0.589245817518974\\
7.30269746658256	0.59021550437947\\
7.74780685615697	0.589968512521601\\
8.19291624573139	0.588604520886292\\
8.68220109096914	0.585928312856786\\
9.17148593620689	0.582135329410503\\
9.66077078144463	0.577334567222661\\
10.1500556266824	0.571627548252987\\
10.6416653734136	0.565075967645812\\
11.1332751201447	0.557790677326464\\
11.6248848668759	0.54985109756196\\
12.1164946136071	0.541333503551458\\
12.6079732666127	0.532311188982125\\
13.0994519196183	0.522842721346191\\
13.5909305726239	0.51298544629886\\
14.0824092256295	0.502795214294804\\
14.573895267778	0.492323751703937\\
15.0653813099266	0.481615243092825\\
15.5568673520752	0.470711634011877\\
16.0483533942238	0.459653868755485\\
16.5398390198718	0.448479997446346\\
17.0313246455198	0.437222425317004\\
17.5228102711679	0.425912050817231\\
18.0142958968159	0.414578845656627\\
18.5057815459406	0.403250584696953\\
18.9972671950653	0.391951140591323\\
19.48875284419	0.380703112020529\\
19.9802384933148	0.369528133418102\\
20.5017696853463	0.357772098713114\\
21.0233008773778	0.346141209962941\\
21.5448320694093	0.334654425380947\\
22.0663632614408	0.323329522891469\\
22.6746017709939	0.310347602531035\\
23.282840280547	0.297628841593494\\
23.8910787901001	0.285191935057362\\
24.4993172996532	0.273053651900737\\
25.0869400675273	0.26162370596486\\
25.6745628354014	0.250495496367714\\
26.2621856032755	0.239677190944159\\
26.8498083711495	0.22917536145636\\
27.5306427936611	0.217409719439121\\
28.2114772161728	0.206079845103249\\
28.8923116386844	0.195187736872762\\
29.573146061196	0.184733168408286\\
30.3571240725376	0.173233522006174\\
31.1411020838792	0.16230395931065\\
31.9250800952207	0.151934828774627\\
32.7090581065623	0.142113892197481\\
33.6393150845161	0.131151668187394\\
34.5695720624699	0.120914545571232\\
35.4998290404237	0.111373434680736\\
36.4300860183775	0.102496709995189\\
37.5675806203718	0.0924969177596037\\
38.7050752223662	0.0833818295618134\\
39.8425698243606	0.075091428348473\\
40.980064426355	0.0675644223338172\\
42.4285780340909	0.0589891501052424\\
43.8770916418269	0.0514425847476294\\
45.3256052495628	0.0448184172275052\\
46.7741188572988	0.0390124577659239\\
48.2741188572988	0.0337602629542047\\
49.7741188572988	0.0291940254167815\\
51.2741188572988	0.0252314599893827\\
52.7741188572988	0.0217949251050357\\
54.2741188572988	0.0188152783318617\\
55.7741188572988	0.0162364315482906\\
57.2741188572988	0.0140072500712239\\
58.7741188572988	0.0120805138329808\\
59.0805891429741	0.0117201716201511\\
59.3870594286494	0.0113704541399994\\
59.6935297143247	0.0110310555073265\\
60	0.0107016779231368\\
};
\end{axis}
\end{tikzpicture}%}}
\\

\caption[Sperimentazione in MATLAB relativo al grafo~\ref{fig::3nodi}]{Divisione in classi nei singoli nodi (a)(b)(c) e grafico della prevalenza (d) per il grafo~\ref{fig::3nodi}.  Per ottenere i grafici abbiamo risolto numericamente,  usando MATLAB,  il problema di Cauchy ottenuto usando il Teorema~\ref{th_cut-vertex} con condizioni iniziali  di stati puri~\eqref{statipuri}.  Abbiamo inoltre supposto l'indipendenza statica di tali condizioni iniziali.\\
Per la sperimentazioni abbiamo usato come parametri $\tau= 0.3$ e $\gamma =0.1 $.}\label{fig::3nodicut}

\end{figure}
\begin{figure}[!htb]
	\centering
	\subfloat[][Nodo 1.]
	{\resizebox{0.45\textwidth}{!}{% This file was created by matlab2tikz.
%
\definecolor{mycolor1}{rgb}{0.00000,0.44700,0.74100}%
\definecolor{mycolor2}{rgb}{0.85000,0.32500,0.09800}%
%
\begin{tikzpicture}

\begin{axis}[%
width=6.028in,
height=4.754in,
at={(1.011in,0.642in)},
scale only axis,
xmin=0,
xmax=60,
xlabel style={font=\color{white!15!black}},
xlabel={T},
ymode=log,
ymin=1e-16,
ymax=1e-06,
yminorticks=true,
ylabel style={font=\color{white!15!black}},
ylabel={Errore assoluto},
axis background/.style={fill=white},
legend style={legend cell align=left, align=left, draw=white!15!black}
]
\addplot [color=mycolor1, line width=2.0pt]
  table[row sep=crcr]{%
0	0\\
0.000167459095433972	0\\
0.000334918190867944	0\\
0.000502377286301916	0\\
0.000669836381735888	0\\
0.00150713185890575	0\\
0.00234442733607561	0\\
0.00318172281324547	0\\
0.00401901829041533	0\\
0.00820549567626463	0\\
0.0123919730621139	0\\
0.0165784504479632	0\\
0.0207649278338125	0\\
0.041697314763059	0\\
0.0626297016923055	0\\
0.083562088621552	0\\
0.104494475550799	0\\
0.209156410197031	0\\
0.313818344843264	0\\
0.418480279489496	0\\
0.523142214135729	0\\
0.693500300195606	0\\
0.863858386255483	0\\
1.03421647231536	0\\
1.20457455837524	0\\
1.44051447786383	0\\
1.67645439735243	0\\
1.91239431684103	0\\
2.14833423632963	0\\
2.44722092525848	0\\
2.74610761418733	0\\
3.04499430311618	0\\
3.34388099204503	0\\
3.7078444355742	0\\
4.07180787910338	0\\
4.43577132263256	0\\
4.79973476616174	0\\
5.20292074647974	0\\
5.60610672679773	0\\
6.00929270711573	0\\
6.41247868743373	0\\
6.85758807700814	0\\
7.30269746658256	0\\
7.74780685615697	0\\
8.19291624573139	0\\
8.68220109096914	0\\
9.17148593620689	0\\
9.66077078144463	0\\
10.1500556266824	0\\
10.6416653734136	0\\
11.1332751201447	0\\
11.6248848668759	0\\
12.1164946136071	0\\
12.6079732666127	0\\
13.0994519196183	0\\
13.5909305726239	0\\
14.0824092256295	0\\
14.573895267778	0\\
15.0653813099266	0\\
15.5568673520752	0\\
16.0483533942238	0\\
16.5398390198718	0\\
17.0313246455198	0\\
17.5228102711679	0\\
18.0142958968159	0\\
18.5057815459406	0\\
18.9972671950653	0\\
19.48875284419	0\\
19.9802384933148	0\\
20.5017696853463	0\\
21.0233008773778	0\\
21.5448320694093	0\\
22.0663632614408	0\\
22.6746017709939	0\\
23.282840280547	0\\
23.8910787901001	0\\
24.4993172996532	0\\
25.0869400675273	0\\
25.6745628354014	0\\
26.2621856032755	0\\
26.8498083711495	0\\
27.5306427936611	0\\
28.2114772161728	0\\
28.8923116386844	0\\
29.573146061196	0\\
30.3571240725376	0\\
31.1411020838792	0\\
31.9250800952207	0\\
32.7090581065623	0\\
33.6393150845161	0\\
34.5695720624699	0\\
35.4998290404237	0\\
36.4300860183775	0\\
37.5675806203718	0\\
38.7050752223662	0\\
39.8425698243606	0\\
40.980064426355	0\\
42.4285780340909	0\\
43.8770916418269	0\\
45.3256052495628	0\\
46.7741188572988	0\\
48.2741188572988	0\\
49.7741188572988	0\\
51.2741188572988	0\\
52.7741188572988	0\\
54.2741188572988	0\\
55.7741188572988	0\\
57.2741188572988	0\\
58.7741188572988	0\\
59.0805891429741	0\\
59.3870594286494	0\\
59.6935297143247	0\\
60	0\\
};
\addlegendentry{Suscettibili}

\addplot [color=mycolor2, line width=2.0pt]
  table[row sep=crcr]{%
0	0\\
0.000167459095433972	0\\
0.000334918190867944	0\\
0.000502377286301916	0\\
0.000669836381735888	0\\
0.00150713185890575	1.11022302462516e-16\\
0.00234442733607561	1.11022302462516e-16\\
0.00318172281324547	0\\
0.00401901829041533	0\\
0.00820549567626463	0\\
0.0123919730621139	1.11022302462516e-16\\
0.0165784504479632	0\\
0.0207649278338125	1.11022302462516e-16\\
0.041697314763059	5.10702591327572e-15\\
0.0626297016923055	3.2085445411667e-14\\
0.083562088621552	4.06341627012807e-14\\
0.104494475550799	1.94289029309402e-14\\
0.209156410197031	4.80684381187757e-11\\
0.313818344843264	6.8871575109597e-11\\
0.418480279489496	2.87392332154468e-11\\
0.523142214135729	1.58528745686226e-12\\
0.693500300195606	5.35891220287965e-10\\
0.863858386255483	7.71493313678207e-10\\
1.03421647231536	3.19120618819113e-10\\
1.20457455837524	2.92404989110651e-11\\
1.44051447786383	2.57943622017365e-09\\
1.67645439735243	3.73856035018605e-09\\
1.91239431684103	1.52402301889509e-09\\
2.14833423632963	2.120162934105e-10\\
2.44722092525848	7.6793129633046e-09\\
2.74610761418733	1.12302089849692e-08\\
3.04499430311618	4.47856918217582e-09\\
3.34388099204503	8.98387253478461e-10\\
3.7078444355742	1.82290047501255e-08\\
4.07180787910338	2.69392855800632e-08\\
4.43577132263256	1.04521359345355e-08\\
4.79973476616174	2.868811743717e-09\\
5.20292074647974	2.51091157865702e-08\\
5.60610672679773	3.79482871792902e-08\\
6.00929270711573	1.38239060198941e-08\\
6.41247868743373	5.80539094574561e-09\\
6.85758807700814	3.38057017934545e-08\\
7.30269746658256	5.21349283744144e-08\\
7.74780685615697	1.79574599079579e-08\\
8.19291624573139	1.0097696512279e-08\\
8.68220109096914	4.39241408800584e-08\\
9.17148593620689	6.91540867903129e-08\\
9.66077078144463	2.24822157557369e-08\\
10.1500556266824	1.61206917259626e-08\\
10.6416653734136	2.97607190558047e-08\\
11.1332751201447	5.13740906482596e-08\\
11.6248848668759	1.23713179700502e-08\\
12.1164946136071	1.98610811685818e-08\\
12.6079732666127	1.80948284689286e-08\\
13.0994519196183	3.61318638342212e-08\\
13.5909305726239	4.41615638502313e-09\\
14.0824092256295	2.17478919595226e-08\\
14.573895267778	9.70529484556337e-09\\
15.0653813099266	2.4761935890627e-08\\
15.5568673520752	1.06108088981927e-09\\
16.0483533942238	2.23397129406333e-08\\
16.5398390198718	3.70753699963089e-09\\
17.0313246455198	1.62869140396538e-08\\
17.5228102711679	4.72341599166981e-09\\
18.0142958968159	2.20150752627823e-08\\
18.5057815459406	4.40367964316124e-10\\
18.9972671950653	1.00581628581509e-08\\
19.48875284419	7.04820915520266e-09\\
19.9802384933148	2.11078679535071e-08\\
20.5017696853463	2.83647849652624e-09\\
21.0233008773778	1.44774395111069e-08\\
21.5448320694093	4.77663730791278e-09\\
22.0663632614408	2.06998056612173e-08\\
22.6746017709939	2.15794610652909e-08\\
23.282840280547	4.2005561959324e-08\\
23.8910787901001	6.36229971484603e-09\\
24.4993172996532	2.36415390042488e-08\\
25.0869400675273	4.67784837254737e-09\\
25.6745628354014	1.85701572313901e-08\\
26.2621856032755	4.30071264712062e-09\\
26.8498083711495	2.34615943339733e-08\\
27.5306427936611	2.37231200794774e-08\\
28.2114772161728	4.6818926777159e-08\\
28.8923116386844	7.00264828762398e-09\\
29.573146061196	2.70292731946453e-08\\
30.3571240725376	4.73537500184085e-08\\
31.1411020838792	8.37898236835666e-08\\
31.9250800952207	1.93135666209932e-08\\
32.7090581065623	3.63635832834053e-08\\
33.6393150845161	9.63787081115219e-08\\
34.5695720624699	1.61731317026292e-07\\
35.4998290404237	4.34269074936489e-08\\
36.4300860183775	6.00714064880048e-08\\
37.5675806203718	2.04758556179852e-07\\
38.7050752223662	3.36977266465754e-07\\
39.8425698243606	9.41319543007468e-08\\
40.980064426355	1.22762745857763e-07\\
42.4285780340909	4.89878668727406e-07\\
43.8770916418269	7.99574034454631e-07\\
45.3256052495628	2.17929912333173e-07\\
46.7741188572988	3.1034828291282e-07\\
48.2741188572988	1.359065085315e-07\\
49.7741188572988	3.72939626926833e-07\\
51.2741188572988	7.46402028725668e-10\\
52.7741188572988	3.39020536290792e-07\\
54.2741188572988	7.06552050824816e-08\\
55.7741188572988	7.96784421401922e-08\\
57.2741188572988	1.07144251735615e-07\\
58.7741188572988	2.78581827250265e-07\\
59.0805891429741	2.70216910154466e-07\\
59.3870594286494	2.62010550898268e-07\\
59.6935297143247	2.54030209382618e-07\\
60	2.46328183056005e-07\\
};
\addlegendentry{Infetti}

\end{axis}
\end{tikzpicture}%}}
 \quad 
\subfloat[][Nodo 2.]
{\resizebox{0.45\textwidth}{!}{ % This file was created by matlab2tikz.
%
\definecolor{mycolor1}{rgb}{0.00000,0.44700,0.74100}%
\definecolor{mycolor2}{rgb}{0.85000,0.32500,0.09800}%
%
\begin{tikzpicture}

\begin{axis}[%
width=0.39\columnwidth,
height=1.7in,
at={(1.011in,0.642in)},
scale only axis,
xmin=0,
xmax=60,
xlabel style={font=\color{white!15!black}},
xlabel={T},
ymode=log,
ymin=1e-20,
ymax=0.0001,
yminorticks=true,
axis background/.style={fill=white},
legend style={at={(axis cs:25,1e-20)},anchor=south west,legend cell align=left, align=left, draw=none,fill=none}
]
\addplot [color=mycolor1, line width=2.0pt]
  table[row sep=crcr]{%
0	0\\
0.000167459095433972	0\\
0.000334918190867944	0\\
0.000502377286301916	0\\
0.000669836381735888	0\\
0.00150713185890575	1.11022302462516e-16\\
0.00234442733607561	0\\
0.00318172281324547	0\\
0.00401901829041533	1.11022302462516e-16\\
0.00820549567626463	1.4432899320127e-15\\
0.0123919730621139	9.76996261670138e-15\\
0.0165784504479632	1.28785870856518e-14\\
0.0207649278338125	6.88338275267597e-15\\
0.041697314763059	3.92752497191395e-12\\
0.0626297016923055	2.50189868822304e-11\\
0.083562088621552	3.18600701376681e-11\\
0.104494475550799	1.5268231123855e-11\\
0.209156410197031	3.92433755491695e-08\\
0.313818344843264	5.72232431439446e-08\\
0.418480279489496	2.30056883543384e-08\\
0.523142214135729	4.96154772822166e-09\\
0.693500300195606	4.04375327445194e-07\\
0.863858386255483	5.98133849782911e-07\\
1.03421647231536	2.29008072283321e-07\\
1.20457455837524	8.61741529245563e-08\\
1.44051447786383	1.62808538561876e-06\\
1.67645439735243	2.46120600488897e-06\\
1.91239431684103	8.71537263469335e-07\\
2.14833423632963	5.27225982649426e-07\\
2.44722092525848	3.61666934545202e-06\\
2.74610761418733	5.67627873981902e-06\\
3.04499430311618	1.77653540806011e-06\\
3.34388099204503	1.72203911968927e-06\\
3.7078444355742	5.77107655991282e-06\\
4.07180787910338	9.57182565647363e-06\\
4.43577132263256	2.50217219227578e-06\\
4.79973476616174	3.92207623223229e-06\\
5.20292074647974	3.70730614296377e-06\\
5.60610672679773	7.70468795957546e-06\\
6.00929270711573	9.67251220618248e-07\\
6.41247868743373	5.19678127469669e-06\\
6.85758807700814	1.9738300790495e-06\\
7.30269746658256	5.82810857208127e-06\\
7.74780685615697	1.28398538024044e-07\\
8.19291624573139	5.6161621967421e-06\\
8.68220109096914	5.83981760449159e-07\\
9.17148593620689	3.99212463914056e-06\\
9.66077078144463	8.29321642970005e-07\\
10.1500556266824	5.29805597787592e-06\\
10.6416653734136	1.9114142701393e-06\\
11.1332751201447	7.952024505542e-08\\
11.6248848668759	1.86391113604634e-06\\
12.1164946136071	3.69921748111235e-06\\
12.6079732666127	1.92884724620201e-06\\
13.0994519196183	8.34227436519175e-07\\
13.5909305726239	1.56294687159164e-06\\
14.0824092256295	2.27003224978484e-06\\
14.573895267778	1.35894785102675e-06\\
15.0653813099266	7.7460416791908e-07\\
15.5568673520752	1.03623534786212e-06\\
16.0483533942238	1.30061264885795e-06\\
16.5398390198718	8.38028847705008e-07\\
17.0313246455198	5.32706617595302e-07\\
17.5228102711679	6.1976376547479e-07\\
18.0142958968159	7.1383310218387e-07\\
18.5057815459406	4.81431445931157e-07\\
18.9972671950653	3.2455406773213e-07\\
19.48875284419	3.49619233674936e-07\\
19.9802384933148	3.80484189377306e-07\\
20.5017696853463	2.42573656938916e-07\\
21.0233008773778	1.51507172141407e-07\\
21.5448320694093	1.75302519223042e-07\\
22.0663632614408	2.01829340795712e-07\\
22.6746017709939	9.08020638124007e-08\\
23.282840280547	2.33619112943728e-08\\
23.8910787901001	7.14184458794342e-08\\
24.4993172996532	1.18387426639721e-07\\
25.0869400675273	7.25104387644393e-08\\
25.6745628354014	4.24975718504861e-08\\
26.2621856032755	5.02035855665106e-08\\
26.8498083711495	5.90608392192493e-08\\
27.5306427936611	2.62429382114071e-08\\
28.2114772161728	6.34887376005366e-09\\
28.8923116386844	1.95478135345262e-08\\
29.573146061196	3.26033325892539e-08\\
30.3571240725376	9.81462833227198e-09\\
31.1411020838792	3.33358218806978e-09\\
31.9250800952207	8.55485277062584e-09\\
32.7090581065623	1.9835558362491e-08\\
33.6393150845161	2.95286195495237e-09\\
34.5695720624699	6.28342217146027e-09\\
35.4998290404237	4.06946260023133e-09\\
36.4300860183775	1.34997727441188e-08\\
37.5675806203718	5.60801460824933e-10\\
38.7050752223662	6.10572281622979e-09\\
39.8425698243606	2.39925551648312e-09\\
40.980064426355	9.76846792344332e-09\\
42.4285780340909	6.71272093466513e-10\\
43.8770916418269	5.55548984593202e-09\\
45.3256052495628	1.71516006952643e-09\\
46.7741188572988	7.50593487364171e-09\\
48.2741188572988	2.63030802694075e-09\\
49.7741188572988	1.80868209298524e-10\\
51.2741188572988	1.27020000073585e-09\\
52.7741188572988	2.29042279586977e-09\\
54.2741188572988	9.3933377742772e-10\\
55.7741188572988	2.46120179792086e-10\\
57.2741188572988	3.84893561466981e-10\\
58.7741188572988	5.5117160835394e-10\\
59.0805891429741	4.87575313457e-10\\
59.3870594286494	4.31316760085565e-10\\
59.6935297143247	3.8155900661252e-10\\
60	3.37542938044777e-10\\
};
\addlegendentry{Suscettibili}

\addplot [color=mycolor2, line width=2.0pt]
  table[row sep=crcr]{%
0	0\\
0.000167459095433972	0\\
0.000334918190867944	1.35525271560688e-20\\
0.000502377286301916	5.42101086242752e-20\\
0.000669836381735888	1.62630325872826e-19\\
0.00150713185890575	0\\
0.00234442733607561	4.01154803819637e-18\\
0.00318172281324547	5.96311194867027e-18\\
0.00401901829041533	2.60208521396521e-18\\
0.00820549567626463	2.13284251371348e-15\\
0.0123919730621139	1.32155571208603e-14\\
0.0165784504479632	1.73047340346066e-14\\
0.0207649278338125	9.45597766754958e-15\\
0.041697314763059	5.23189824797043e-12\\
0.0626297016923055	3.33267753671596e-11\\
0.083562088621552	4.24396802423121e-11\\
0.104494475550799	2.03384357966296e-11\\
0.209156410197031	5.22764320859403e-08\\
0.313818344843264	7.62287859640276e-08\\
0.418480279489496	3.06455118781468e-08\\
0.523142214135729	6.61381191480181e-09\\
0.693500300195606	5.38631211716156e-07\\
0.863858386255483	7.96740306008292e-07\\
1.03421647231536	3.05024975416623e-07\\
1.20457455837524	1.14869630141179e-07\\
1.44051447786383	2.16820107779014e-06\\
1.67645439735243	3.27786944592789e-06\\
1.91239431684103	1.16052566156988e-06\\
2.14833423632963	7.02755960535484e-07\\
2.44722092525848	4.81454648076918e-06\\
2.74610761418733	7.55714144390351e-06\\
3.04499430311618	2.36423530802776e-06\\
3.34388099204503	2.29515377259126e-06\\
3.7078444355742	7.67653974159677e-06\\
4.07180787910338	1.27354949227554e-05\\
4.43577132263256	3.32577745348894e-06\\
4.79973476616174	5.22656616480655e-06\\
5.20292074647974	4.91796574125791e-06\\
5.60610672679773	1.02349689920511e-05\\
6.00929270711573	1.2758443878047e-06\\
6.41247868743373	6.92323630896441e-06\\
6.85758807700814	2.59796773666165e-06\\
7.30269746658256	7.71867650084523e-06\\
7.74780685615697	1.89155510865735e-07\\
8.19291624573139	7.4781185661621e-06\\
8.68220109096914	7.34718205941398e-07\\
9.17148593620689	5.25367876513805e-06\\
9.66077078144463	1.12824440673398e-06\\
10.1500556266824	7.04795394584901e-06\\
10.6416653734136	2.57831307975964e-06\\
11.1332751201447	5.46529023148778e-08\\
11.6248848668759	2.49758616654994e-06\\
12.1164946136071	4.9124288942215e-06\\
12.6079732666127	2.58989115736741e-06\\
13.0994519196183	1.14843511306306e-06\\
13.5909305726239	2.08834531911783e-06\\
14.0824092256295	3.00496177504939e-06\\
14.573895267778	1.8216357634826e-06\\
15.0653813099266	1.0575674938007e-06\\
15.5568673520752	1.38058605017588e-06\\
16.0483533942238	1.71181048608249e-06\\
16.5398390198718	1.12107933453176e-06\\
17.0313246455198	7.26562404740339e-07\\
17.5228102711679	8.21628271890917e-07\\
18.0142958968159	9.2976239496334e-07\\
18.5057815459406	6.41468227285946e-07\\
18.9972671950653	4.42796920546717e-07\\
19.48875284419	4.59110769651661e-07\\
19.9802384933148	4.86204385141686e-07\\
20.5017696853463	3.26268021710874e-07\\
21.0233008773778	2.16487003060206e-07\\
21.5448320694093	2.28960055609351e-07\\
22.0663632614408	2.48405982741784e-07\\
22.6746017709939	1.42648880110952e-07\\
23.282840280547	7.31547776244845e-08\\
23.8910787901001	1.01586894882266e-07\\
24.4993172996532	1.3420836382505e-07\\
25.0869400675273	1.01358433979293e-07\\
25.6745628354014	7.52335869019793e-08\\
26.2621856032755	6.26374020151754e-08\\
26.8498083711495	5.52861918606817e-08\\
27.5306427936611	5.87137049257169e-08\\
28.2114772161728	5.52840924705755e-08\\
28.8923116386844	3.30664003123093e-08\\
29.573146061196	1.64418374817843e-08\\
30.3571240725376	6.04399216300178e-08\\
31.1411020838792	7.93450480457425e-08\\
31.9250800952207	3.07200375346262e-08\\
32.7090581065623	9.91617149503909e-09\\
33.6393150845161	1.00315857995414e-07\\
34.5695720624699	1.53353421440683e-07\\
35.4998290404237	4.88528582830161e-08\\
36.4300860183775	4.20717088771283e-08\\
37.5675806203718	2.05506292084295e-07\\
38.7050752223662	3.2883630332603e-07\\
39.8425698243606	9.73309622342988e-08\\
40.980064426355	1.09738121305269e-07\\
42.4285780340909	4.88983639954088e-07\\
43.8770916418269	7.92166715330236e-07\\
45.3256052495628	2.20216793050954e-07\\
46.7741188572988	3.00340369133872e-07\\
48.2741188572988	1.39413586508486e-07\\
49.7741188572988	3.73180785121809e-07\\
51.2741188572988	2.44000264958133e-09\\
52.7741188572988	3.35966638674232e-07\\
54.2741188572988	6.94027595133512e-08\\
55.7741188572988	8.00066028974408e-08\\
57.2741188572988	1.06631059798752e-07\\
58.7741188572988	2.77846931210409e-07\\
59.0805891429741	2.69566809182713e-07\\
59.3870594286494	2.61435461314241e-07\\
59.6935297143247	2.53521463483043e-07\\
60	2.4587812526549e-07\\
};
\addlegendentry{Infetti}

\end{axis}
\end{tikzpicture}%}}
\\
\subfloat[][Nodo 3.]
{\resizebox{0.45\textwidth}{!}
{% This file was created by matlab2tikz.
%
\definecolor{mycolor1}{rgb}{0.00000,0.44700,0.74100}%
\definecolor{mycolor2}{rgb}{0.85000,0.32500,0.09800}%
%
\begin{tikzpicture}

\begin{axis}[%
width=0.39\columnwidth,
height=1.7in,
at={(1.011in,0.642in)},
scale only axis,
xmin=0,
xmax=60,
xlabel style={font=\color{white!15!black}},
xlabel={T},
ymode=log,
ymin=1e-17,
ymax=1e-12,
yminorticks=true,
axis background/.style={fill=white},
legend columns=2,
legend style={legend cell align=left, align=left, draw=none, fill=none }
]
\addplot [color=mycolor1, line width=2.0pt]
  table[row sep=crcr]{%
0	0\\
0.00199053585276749	3.33066907387547e-16\\
0.00398107170553497	2.22044604925031e-16\\
0.00597160755830246	9.99200722162641e-16\\
0.00796214341106994	9.99200722162641e-16\\
0.0143083073603051	6.90558721316847e-14\\
0.0206544713095403	7.99360577730113e-15\\
0.0270006352587754	5.39568389967826e-14\\
0.0333467992080106	2.38697950294409e-14\\
0.0396935645593046	6.81676937119846e-14\\
0.0460403299105986	7.66053886991358e-15\\
0.0523870952618926	5.36237720893951e-14\\
0.0587338606131866	2.36477504245158e-14\\
0.0650961133419323	6.81676937119846e-14\\
0.0714583660706779	7.66053886991358e-15\\
0.0778206187994235	5.35127497869325e-14\\
0.0841828715281692	2.36477504245158e-14\\
0.0905606411731996	6.82787160144471e-14\\
0.0969384108182301	7.66053886991358e-15\\
0.103316180463261	5.35127497869325e-14\\
0.109693950108291	2.36477504245158e-14\\
0.116087318034554	6.82787160144471e-14\\
0.122480685960817	7.88258347483861e-15\\
0.12887405388708	5.340172748447e-14\\
0.135267421813343	2.35367281220533e-14\\
0.141676469438281	6.81676937119846e-14\\
0.14808551706322	7.88258347483861e-15\\
0.154494564688158	5.340172748447e-14\\
0.160903612313097	2.35367281220533e-14\\
0.167328422015208	6.81676937119846e-14\\
0.173753231717319	7.88258347483861e-15\\
0.18017804141943	5.3179682879545e-14\\
0.186602851121541	2.34257058195908e-14\\
0.193043505744106	6.81676937119846e-14\\
0.199484160366671	7.7715611723761e-15\\
0.205924814989236	5.32907051820075e-14\\
0.212365469611801	2.34257058195908e-14\\
0.218822053394839	6.81676937119846e-14\\
0.225278637177877	7.88258347483861e-15\\
0.231735220960914	5.3179682879545e-14\\
0.238191804743952	2.34257058195908e-14\\
0.244664401671674	6.81676937119846e-14\\
0.251136998599397	7.99360577730113e-15\\
0.25760959552712	5.3179682879545e-14\\
0.264082192454842	2.33146835171283e-14\\
0.270570887963067	6.80566714095221e-14\\
0.277059583471292	7.99360577730113e-15\\
0.283548278979517	5.30686605770825e-14\\
0.290036974487742	2.34257058195908e-14\\
0.296541854059766	6.80566714095221e-14\\
0.303046733631789	7.99360577730113e-15\\
0.309551613203813	5.3179682879545e-14\\
0.316056492775836	2.33146835171283e-14\\
0.322577643612098	6.80566714095221e-14\\
0.32909879444836	7.99360577730113e-15\\
0.335619945284621	5.30686605770825e-14\\
0.342141096120883	2.32036612146658e-14\\
0.348678605036902	6.80566714095221e-14\\
0.35521611395292	7.99360577730113e-15\\
0.361753622868939	5.295763827462e-14\\
0.368291131784958	2.32036612146658e-14\\
0.374845086704082	6.80566714095221e-14\\
0.381399041623206	7.99360577730113e-15\\
0.387952996542331	5.295763827462e-14\\
0.394506951461455	2.32036612146658e-14\\
0.401077441531393	6.79456491070596e-14\\
0.407647931601331	7.88258347483861e-15\\
0.414218421671268	5.28466159721575e-14\\
0.420788911741206	2.32036612146658e-14\\
0.427376026319146	6.78346268045971e-14\\
0.433963140897085	7.88258347483861e-15\\
0.440550255475025	5.295763827462e-14\\
0.447137370052964	2.32036612146658e-14\\
0.453741199367022	6.79456491070596e-14\\
0.46034502868108	7.99360577730113e-15\\
0.466948857995137	5.28466159721575e-14\\
0.473552687309195	2.30926389122033e-14\\
0.480173322720616	6.79456491070596e-14\\
0.486793958132037	7.88258347483861e-15\\
0.493414593543458	5.27355936696949e-14\\
0.500035228954879	2.32036612146658e-14\\
0.506672761762064	6.78346268045971e-14\\
0.513310294569248	7.88258347483861e-15\\
0.519947827376433	5.27355936696949e-14\\
0.526585360183617	2.32036612146658e-14\\
0.533239883483442	6.78346268045971e-14\\
0.539894406783268	7.88258347483861e-15\\
0.546548930083094	5.27355936696949e-14\\
0.553203453382919	2.32036612146658e-14\\
0.559875060934079	6.77236045021345e-14\\
0.566546668485238	7.88258347483861e-15\\
0.573218276036397	5.27355936696949e-14\\
0.579889883587557	2.32036612146658e-14\\
0.586578669168294	6.77236045021345e-14\\
0.593267454749031	7.99360577730113e-15\\
0.599956240329768	5.26245713672324e-14\\
0.606645025910505	2.32036612146658e-14\\
0.613351084619436	6.7612582199672e-14\\
0.620057143328367	7.88258347483861e-15\\
0.626763202037298	5.26245713672324e-14\\
0.633469260746229	2.32036612146658e-14\\
0.640192689007476	6.75015598972095e-14\\
0.646916117268722	7.7715611723761e-15\\
0.653639545529968	5.26245713672324e-14\\
0.660362973791214	2.32036612146658e-14\\
0.667103867904104	6.75015598972095e-14\\
0.673844762016994	7.88258347483861e-15\\
0.680585656129883	5.26245713672324e-14\\
0.687326550242773	2.30926389122033e-14\\
0.694085008124096	6.75015598972095e-14\\
0.700843466005419	7.88258347483861e-15\\
0.707601923886742	5.26245713672324e-14\\
0.714360381768066	2.30926389122033e-14\\
0.721136501534817	6.75015598972095e-14\\
0.727912621301569	7.88258347483861e-15\\
0.73468874106832	5.26245713672324e-14\\
0.741464860835071	2.32036612146658e-14\\
0.748258742408364	6.7390537594747e-14\\
0.755052623981657	7.88258347483861e-15\\
0.761846505554949	5.26245713672324e-14\\
0.768640387128242	2.30926389122033e-14\\
0.775452130474897	6.72795152922845e-14\\
0.782263873821551	7.7715611723761e-15\\
0.789075617168206	5.25135490647699e-14\\
0.79588736051486	2.30926389122033e-14\\
0.80271706685068	6.72795152922845e-14\\
0.809546773186499	7.88258347483861e-15\\
0.816376479522319	5.25135490647699e-14\\
0.823206185858138	2.32036612146658e-14\\
0.830053957332632	6.72795152922845e-14\\
0.836901728807126	7.7715611723761e-15\\
0.84374950028162	5.25135490647699e-14\\
0.850597271756113	2.30926389122033e-14\\
0.857463211448599	6.72795152922845e-14\\
0.864329151141086	7.88258347483861e-15\\
0.871195090833572	5.25135490647699e-14\\
0.878061030526058	2.30926389122033e-14\\
0.884945241937476	6.7168492989822e-14\\
0.891829453348894	7.88258347483861e-15\\
0.898713664760312	5.24025267623074e-14\\
0.90559787617173	2.29816166097407e-14\\
0.912500464439821	6.7168492989822e-14\\
0.919403052707912	7.88258347483861e-15\\
0.926305640976003	5.22915044598449e-14\\
0.933208229244093	2.29816166097407e-14\\
0.940129300503941	6.70574706873595e-14\\
0.947050371763788	7.88258347483861e-15\\
0.953971443023636	5.22915044598449e-14\\
0.960892514283483	2.29816166097407e-14\\
0.967832174888143	6.70574706873595e-14\\
0.974771835492802	7.88258347483861e-15\\
0.981711496097461	5.22915044598449e-14\\
0.988651156702121	2.29816166097407e-14\\
0.995609514344775	6.69464483848969e-14\\
1.00256787198743	7.7715611723761e-15\\
1.00952622963008	5.21804821573824e-14\\
1.01648458727274	2.29816166097407e-14\\
1.0234617510246	6.69464483848969e-14\\
1.03043891477646	7.7715611723761e-15\\
1.03741607852832	5.21804821573824e-14\\
1.04439324228018	2.29816166097407e-14\\
1.05138932194451	6.69464483848969e-14\\
1.05838540160883	7.88258347483861e-15\\
1.06538148127316	5.20694598549198e-14\\
1.07237756093749	2.29816166097407e-14\\
1.07939266704796	6.68354260824344e-14\\
1.08640777315843	7.7715611723761e-15\\
1.0934228792689	5.20694598549198e-14\\
1.10043798537936	2.28705943072782e-14\\
1.10747222983544	6.68354260824344e-14\\
1.11450647429151	7.88258347483861e-15\\
1.12154071874758	5.20694598549198e-14\\
1.12857496320366	2.28705943072782e-14\\
1.13562845894337	6.68354260824344e-14\\
1.14268195468308	7.88258347483861e-15\\
1.14973545042278	5.19584375524573e-14\\
1.15678894616249	2.28705943072782e-14\\
1.16386180727142	6.67244037799719e-14\\
1.17093466838034	7.7715611723761e-15\\
1.17800752948927	5.19584375524573e-14\\
1.18508039059819	2.28705943072782e-14\\
1.19217273186328	6.67244037799719e-14\\
1.19926507312837	7.88258347483861e-15\\
1.20635741439346	5.19584375524573e-14\\
1.21344975565855	2.28705943072782e-14\\
1.22056169301535	6.66133814775094e-14\\
1.22767363037216	7.88258347483861e-15\\
1.23478556772896	5.18474152499948e-14\\
1.24189750508577	2.27595720048157e-14\\
1.2490291561515	6.66133814775094e-14\\
1.25616080721723	7.88258347483861e-15\\
1.26329245828296	5.17363929475323e-14\\
1.27042410934869	2.27595720048157e-14\\
1.27757559222201	6.66133814775094e-14\\
1.28472707509533	7.88258347483861e-15\\
1.29187855796865	5.17363929475323e-14\\
1.29903004084196	2.27595720048157e-14\\
1.30620147471517	6.65023591750469e-14\\
1.31337290858837	7.7715611723761e-15\\
1.32054434246157	5.18474152499948e-14\\
1.32771577633477	2.27595720048157e-14\\
1.33490728210935	6.63913368725844e-14\\
1.34209878788392	7.7715611723761e-15\\
1.3492902936585	5.17363929475323e-14\\
1.35648179943308	2.27595720048157e-14\\
1.36369349872651	6.63913368725844e-14\\
1.37090519801994	7.88258347483861e-15\\
1.37811689731337	5.16253706450698e-14\\
1.3853285966068	2.27595720048157e-14\\
1.39256061231053	6.63913368725844e-14\\
1.39979262801426	7.7715611723761e-15\\
1.40702464371799	5.17363929475323e-14\\
1.41425665942172	2.27595720048157e-14\\
1.42150911582444	6.62803145701218e-14\\
1.42876157222716	7.7715611723761e-15\\
1.43601402862988	5.16253706450698e-14\\
1.44326648503259	2.27595720048157e-14\\
1.45053950730217	6.62803145701218e-14\\
1.45781252957175	7.88258347483861e-15\\
1.46508555184133	5.16253706450698e-14\\
1.47235857411091	2.26485497023532e-14\\
1.47965228878223	6.62803145701218e-14\\
1.48694600345356	7.88258347483861e-15\\
1.49423971812488	5.16253706450698e-14\\
1.5015334327962	2.26485497023532e-14\\
1.50884796768823	6.60582699651968e-14\\
1.51616250258027	7.66053886991358e-15\\
1.5234770374723	5.15143483426073e-14\\
1.53079157236433	2.26485497023532e-14\\
1.53812705668821	6.60582699651968e-14\\
1.54546254101209	7.7715611723761e-15\\
1.55279802533598	5.14033260401447e-14\\
1.56013350965986	2.26485497023532e-14\\
1.56749007329995	6.59472476627343e-14\\
1.57484663694003	7.7715611723761e-15\\
1.58220320058012	5.15143483426073e-14\\
1.5895597642202	2.26485497023532e-14\\
1.59693753927215	6.59472476627343e-14\\
1.60431531432409	7.7715611723761e-15\\
1.61169308937604	5.14033260401447e-14\\
1.61907086442798	2.25375273998907e-14\\
1.6264699837717	6.59472476627343e-14\\
1.63386910311542	7.7715611723761e-15\\
1.64126822245913	5.14033260401447e-14\\
1.64866734180285	2.26485497023532e-14\\
1.65608793943564	6.58362253602718e-14\\
1.66350853706843	7.66053886991358e-15\\
1.67092913470122	5.12923037376822e-14\\
1.67834973233401	2.26485497023532e-14\\
1.68579194412116	6.57252030578093e-14\\
1.69323415590832	7.66053886991358e-15\\
1.70067636769548	5.12923037376822e-14\\
1.70811857948264	2.25375273998907e-14\\
1.71558254216844	6.57252030578093e-14\\
1.72304650485425	7.7715611723761e-15\\
1.73051046754005	5.12923037376822e-14\\
1.73797443022586	2.25375273998907e-14\\
1.74546028230335	6.57252030578093e-14\\
1.75294613438084	7.66053886991358e-15\\
1.76043198645833	5.12923037376822e-14\\
1.76791783853582	2.25375273998907e-14\\
1.77542572012108	6.56141807553468e-14\\
1.78293360170633	7.66053886991358e-15\\
1.79044148329159	5.11812814352197e-14\\
1.79794936487685	2.25375273998907e-14\\
1.80547941694215	6.56141807553468e-14\\
1.81300946900745	7.7715611723761e-15\\
1.82053952107275	5.10702591327572e-14\\
1.82806957313806	2.24265050974282e-14\\
1.8356219382731	6.56141807553468e-14\\
1.84317430340815	7.7715611723761e-15\\
1.85072666854319	5.09592368302947e-14\\
1.85827903367824	2.24265050974282e-14\\
1.86585385599785	6.56141807553468e-14\\
1.87342867831746	7.7715611723761e-15\\
1.88100350063707	5.09592368302947e-14\\
1.88857832295669	2.23154827949656e-14\\
1.89617574793629	6.55031584528842e-14\\
1.90377317291589	7.88258347483861e-15\\
1.9113705978955	5.08482145278322e-14\\
1.9189680228751	2.24265050974282e-14\\
1.92658819822989	6.53921361504217e-14\\
1.93420837358469	7.7715611723761e-15\\
1.94182854893948	5.08482145278322e-14\\
1.94944872429427	2.24265050974282e-14\\
1.95709179801755	6.52811138479592e-14\\
1.96473487174083	7.54951656745106e-15\\
1.97237794546411	5.08482145278322e-14\\
1.98002101918739	2.24265050974282e-14\\
1.98768714199827	6.51700915454967e-14\\
1.99535326480915	7.66053886991358e-15\\
2.00301938762003	5.08482145278322e-14\\
2.01068551043091	2.24265050974282e-14\\
2.0183748340895	6.50590692430342e-14\\
2.02606415774808	7.66053886991358e-15\\
2.03375348140667	5.07371922253697e-14\\
2.04144280506525	2.24265050974282e-14\\
2.0491554826466	6.50590692430342e-14\\
2.05686816022795	7.54951656745106e-15\\
2.06458083780929	5.08482145278322e-14\\
2.07229351539064	2.24265050974282e-14\\
2.08002970237887	6.50590692430342e-14\\
2.08776588936711	7.54951656745106e-15\\
2.09550207635534	5.07371922253697e-14\\
2.10323826334358	2.23154827949656e-14\\
2.11099811595816	6.50590692430342e-14\\
2.11875796857274	7.66053886991358e-15\\
2.12651782118732	5.06261699229071e-14\\
2.1342776738019	2.23154827949656e-14\\
2.14206135075448	6.49480469405717e-14\\
2.14984502770706	7.66053886991358e-15\\
2.15762870465964	5.05151476204446e-14\\
2.16541238161222	2.22044604925031e-14\\
2.17322004315878	6.49480469405717e-14\\
2.18102770470534	7.7715611723761e-15\\
2.1888353662519	5.04041253179821e-14\\
2.19664302779846	2.22044604925031e-14\\
2.2044748353169	6.48370246381091e-14\\
2.21230664283533	7.7715611723761e-15\\
2.22013845035377	5.04041253179821e-14\\
2.2279702578722	2.20934381900406e-14\\
2.23582637510447	6.48370246381091e-14\\
2.24368249233675	7.7715611723761e-15\\
2.25153860956902	5.04041253179821e-14\\
2.25939472680129	2.22044604925031e-14\\
2.26727531955162	6.47260023356466e-14\\
2.27515591230195	7.66053886991358e-15\\
2.28303650505228	5.04041253179821e-14\\
2.29091709780261	2.22044604925031e-14\\
2.29882233260671	6.47260023356466e-14\\
2.30672756741082	7.7715611723761e-15\\
2.31463280221492	5.01820807130571e-14\\
2.32253803701902	2.22044604925031e-14\\
2.33046808291417	6.46149800331841e-14\\
2.33839812880931	7.66053886991358e-15\\
2.34632817470446	5.02931030155196e-14\\
2.3542582205996	2.20934381900406e-14\\
2.36221324911394	6.45039577307216e-14\\
2.37016827762829	7.7715611723761e-15\\
2.37812330614264	5.01820807130571e-14\\
2.38607833465698	2.20934381900406e-14\\
2.39405851771369	6.43929354282591e-14\\
2.40203870077039	7.66053886991358e-15\\
2.4100188838271	5.00710584105946e-14\\
2.4179990668838	2.20934381900406e-14\\
2.42600457974877	6.45039577307216e-14\\
2.43401009261373	7.7715611723761e-15\\
2.4420156054787	5.00710584105946e-14\\
2.45002111834367	2.19824158875781e-14\\
2.45805213720175	6.45039577307216e-14\\
2.46608315605982	7.7715611723761e-15\\
2.4741141749179	4.98490138056695e-14\\
2.48214519377598	2.19824158875781e-14\\
2.4902018978345	6.43929354282591e-14\\
2.49825860189303	7.7715611723761e-15\\
2.50631530595155	4.98490138056695e-14\\
2.51437201001007	2.18713935851156e-14\\
2.52245457948208	6.42819131257966e-14\\
2.53053714895408	7.7715611723761e-15\\
2.53861971842609	4.98490138056695e-14\\
2.54670228789809	2.18713935851156e-14\\
2.55481090545363	6.42819131257966e-14\\
2.56291952300918	7.7715611723761e-15\\
2.57102814056472	4.9737991503207e-14\\
2.57913675812026	2.17603712826531e-14\\
2.58727160888194	6.40598685208715e-14\\
2.59540645964362	7.66053886991358e-15\\
2.6035413104053	4.96269692007445e-14\\
2.61167616116698	2.18713935851156e-14\\
2.61983743200028	6.3948846218409e-14\\
2.62799870283359	7.7715611723761e-15\\
2.63615997366689	4.9737991503207e-14\\
2.6443212445002	2.18713935851156e-14\\
2.65250912437636	6.3948846218409e-14\\
2.66069700425252	7.66053886991358e-15\\
2.66888488412869	4.96269692007445e-14\\
2.67707276400485	2.18713935851156e-14\\
2.6852874441124	6.3948846218409e-14\\
2.69350212421995	7.66053886991358e-15\\
2.7017168043275	4.96269692007445e-14\\
2.70993148443506	2.18713935851156e-14\\
2.71817315820996	6.38378239159465e-14\\
2.72641483198486	7.66053886991358e-15\\
2.73465650575977	4.9515946898282e-14\\
2.74289817953467	2.17603712826531e-14\\
2.75116704265512	6.38378239159465e-14\\
2.75943590577556	7.7715611723761e-15\\
2.767704768896	4.9293902293357e-14\\
2.77597363201644	2.16493489801906e-14\\
2.78426988288509	6.38378239159465e-14\\
2.79256613375373	7.7715611723761e-15\\
2.80086238462237	4.91828799908944e-14\\
2.80915863549101	2.1538326677728e-14\\
2.81748247408502	6.3726801613484e-14\\
2.82580631267902	7.7715611723761e-15\\
2.83413015127303	4.91828799908944e-14\\
2.84245398986704	2.1538326677728e-14\\
2.85080561880438	6.36157793110215e-14\\
2.85915724774172	7.7715611723761e-15\\
2.86750887667906	4.91828799908944e-14\\
2.8758605056164	2.1538326677728e-14\\
2.88424013037944	6.3504757008559e-14\\
2.89261975514247	7.7715611723761e-15\\
2.90099937990551	4.90718576884319e-14\\
2.90937900466855	2.1538326677728e-14\\
2.91778683316272	6.33937347060964e-14\\
2.9261946616569	7.66053886991358e-15\\
2.93460249015107	4.90718576884319e-14\\
2.94301031864525	2.16493489801906e-14\\
2.95144656047954	6.32827124036339e-14\\
2.95988280231384	7.66053886991358e-15\\
2.96831904414814	4.90718576884319e-14\\
2.97675528598243	2.16493489801906e-14\\
2.98522015378253	6.31716901011714e-14\\
2.99368502158263	7.54951656745106e-15\\
3.00214988938273	4.90718576884319e-14\\
3.01061475718283	2.16493489801906e-14\\
3.01910846625213	6.30606677987089e-14\\
3.02760217532142	7.43849426498855e-15\\
3.03609588439071	4.89608353859694e-14\\
3.04458959346	2.16493489801906e-14\\
3.05311236220025	6.29496454962464e-14\\
3.0616351309405	7.66053886991358e-15\\
3.07015789968076	4.89608353859694e-14\\
3.07868066842101	2.1538326677728e-14\\
3.0872327162858	6.30606677987089e-14\\
3.0957847641506	7.54951656745106e-15\\
3.10433681201539	4.87387907810444e-14\\
3.11288885988019	2.14273043752655e-14\\
3.12147041092284	6.28386231937839e-14\\
3.13005196196549	7.66053886991358e-15\\
3.13863351300815	4.86277684785819e-14\\
3.1472150640508	2.14273043752655e-14\\
3.15582634421903	6.27276008913213e-14\\
3.16443762438727	7.54951656745106e-15\\
3.1730489045555	4.87387907810444e-14\\
3.18166018472374	2.14273043752655e-14\\
3.19030142291735	6.26165785888588e-14\\
3.19894266111096	7.54951656745106e-15\\
3.20758389930457	4.86277684785819e-14\\
3.21622513749818	2.14273043752655e-14\\
3.22489656499851	6.26165785888588e-14\\
3.23356799249884	7.54951656745106e-15\\
3.24223941999916	4.85167461761193e-14\\
3.25091084749949	2.14273043752655e-14\\
3.25961269965928	6.25055562863963e-14\\
3.26831455181906	7.43849426498855e-15\\
3.27701640397885	4.85167461761193e-14\\
3.28571825613863	2.14273043752655e-14\\
3.29445076992401	6.23945339839338e-14\\
3.30318328370939	7.54951656745106e-15\\
3.31191579749477	4.85167461761193e-14\\
3.32064831128015	2.1316282072803e-14\\
3.32941172759282	6.22835116814713e-14\\
3.33817514390549	7.43849426498855e-15\\
3.34693856021816	4.84057238736568e-14\\
3.35570197653083	2.1316282072803e-14\\
3.36449653919211	6.21724893790088e-14\\
3.37329110185339	7.43849426498855e-15\\
3.38208566451467	4.84057238736568e-14\\
3.39088022717596	2.1316282072803e-14\\
3.39970618282859	6.20614670765463e-14\\
3.40853213848122	7.43849426498855e-15\\
3.41735809413385	4.81836792687318e-14\\
3.42618404978648	2.12052597703405e-14\\
3.43504164855909	6.19504447740837e-14\\
3.44389924733171	7.43849426498855e-15\\
3.45275684610433	4.81836792687318e-14\\
3.46161444487695	2.12052597703405e-14\\
3.47050394016004	6.19504447740837e-14\\
3.47939343544313	7.54951656745106e-15\\
3.48828293072622	4.80726569662693e-14\\
3.49717242600932	2.12052597703405e-14\\
3.50609407425931	6.18394224716212e-14\\
3.5150157225093	7.43849426498855e-15\\
3.52393737075929	4.79616346638068e-14\\
3.53285901900928	2.1094237467878e-14\\
3.54181308032811	6.18394224716212e-14\\
3.55076714164694	7.54951656745106e-15\\
3.55972120296577	4.79616346638068e-14\\
3.56867526428461	2.1094237467878e-14\\
3.57766200230417	6.16173778666962e-14\\
3.58664874032374	7.43849426498855e-15\\
3.5956354783433	4.78506123613442e-14\\
3.60462221636287	2.1094237467878e-14\\
3.61364189770245	6.15063555642337e-14\\
3.62266157904204	7.43849426498855e-15\\
3.63168126038162	4.78506123613442e-14\\
3.64070094172121	2.1094237467878e-14\\
3.64975383695641	6.13953332617712e-14\\
3.65880673219161	7.43849426498855e-15\\
3.66785962742681	4.77395900588817e-14\\
3.67691252266202	2.09832151654155e-14\\
3.6859989062531	6.15063555642337e-14\\
3.69508528984418	7.54951656745106e-15\\
3.70417167343526	4.75175454539567e-14\\
3.71325805702634	2.09832151654155e-14\\
3.72237820615531	6.12843109593086e-14\\
3.73149835528429	7.43849426498855e-15\\
3.74061850441326	4.76285677564192e-14\\
3.74973865354223	2.09832151654155e-14\\
3.75889285078234	6.11732886568461e-14\\
3.76804704802246	7.43849426498855e-15\\
3.77720124526257	4.75175454539567e-14\\
3.78635544250268	2.09832151654155e-14\\
3.79554397276389	6.09512440519211e-14\\
3.80473250302509	7.32747196252603e-15\\
3.8139210332863	4.75175454539567e-14\\
3.8231095635475	2.09832151654155e-14\\
3.83233271658159	6.08402217494586e-14\\
3.84155586961568	7.32747196252603e-15\\
3.85077902264978	4.74065231514942e-14\\
3.86000217568387	2.09832151654155e-14\\
3.86926024490616	6.08402217494586e-14\\
3.87851831412845	7.32747196252603e-15\\
3.88777638335074	4.72955008490317e-14\\
3.89703445257303	2.08721928629529e-14\\
3.90632773542166	6.07291994469961e-14\\
3.91562101827029	7.43849426498855e-15\\
3.92491430111892	4.71844785465692e-14\\
3.93420758396755	2.07611705604904e-14\\
3.94353638235509	6.06181771445335e-14\\
3.95286518074264	7.32747196252603e-15\\
3.96219397913019	4.70734562441066e-14\\
3.97152277751774	2.08721928629529e-14\\
3.9808873974673	6.0507154842071e-14\\
3.99025201741687	7.32747196252603e-15\\
3.99961663736644	4.69624339416441e-14\\
4.008981257316	2.07611705604904e-14\\
4.01838200905393	6.03961325396085e-14\\
4.02778276079185	7.32747196252603e-15\\
4.03718351252977	4.68514116391816e-14\\
4.0465842642677	2.07611705604904e-14\\
4.05602146260214	6.0285110237146e-14\\
4.06545866093658	7.32747196252603e-15\\
4.07489585927102	4.68514116391816e-14\\
4.08433305760547	2.06501482580279e-14\\
4.09380702196714	6.0285110237146e-14\\
4.10328098632881	7.32747196252603e-15\\
4.11275495069048	4.67403893367191e-14\\
4.12222891505216	2.06501482580279e-14\\
4.13173996944755	5.99520433297585e-14\\
4.14125102384295	7.32747196252603e-15\\
4.15076207823835	4.67403893367191e-14\\
4.16027313263375	2.06501482580279e-14\\
4.16982160537755	5.99520433297585e-14\\
4.17937007812135	7.21644966006352e-15\\
4.18891855086514	4.65183447317941e-14\\
4.19846702360894	2.05391259555654e-14\\
4.20805324837695	5.98410210272959e-14\\
4.21763947314496	7.32747196252603e-15\\
4.22722569791297	4.64073224293315e-14\\
4.23681192268098	2.04281036531029e-14\\
4.24643623800907	5.98410210272959e-14\\
4.25606055333716	7.32747196252603e-15\\
4.26568486866525	4.6296300126869e-14\\
4.27530918399335	2.03170813506404e-14\\
4.28497193298002	5.97299987248334e-14\\
4.29463468196669	7.43849426498855e-15\\
4.30429743095337	4.61852778244065e-14\\
4.31396017994004	2.03170813506404e-14\\
4.32366171143296	5.95079541199084e-14\\
4.33336324292588	7.21644966006352e-15\\
4.3430647744188	4.61852778244065e-14\\
4.35276630591172	2.03170813506404e-14\\
4.3625069732434	5.93969318174459e-14\\
4.37224764057509	7.21644966006352e-15\\
4.38198830790678	4.6074255521944e-14\\
4.39172897523846	2.03170813506404e-14\\
4.40150913802251	5.91748872125208e-14\\
4.41128930080656	7.21644966006352e-15\\
4.42106946359061	4.5852210917019e-14\\
4.43084962637466	2.02060590481778e-14\\
4.44066964916844	5.91748872125208e-14\\
4.45048967196222	7.43849426498855e-15\\
4.460309694756	4.57411886145564e-14\\
4.47012971754978	2.00950367457153e-14\\
4.47998997038358	5.90638649100583e-14\\
4.48985022321737	7.32747196252603e-15\\
4.49971047605117	4.56301663120939e-14\\
4.50957072888496	1.99840144432528e-14\\
4.51947158785354	5.90638649100583e-14\\
4.52937244682212	7.32747196252603e-15\\
4.5392733057907	4.54081217071689e-14\\
4.54917416475928	1.99840144432528e-14\\
4.55911601157961	5.89528426075958e-14\\
4.56905785839994	7.32747196252603e-15\\
4.57899970522028	4.54081217071689e-14\\
4.58894155204061	1.99840144432528e-14\\
4.59892477418835	5.88418203051333e-14\\
4.6089079963361	7.43849426498855e-15\\
4.61889121848384	4.51860771022439e-14\\
4.62887444063159	1.98729921407903e-14\\
4.6388994324499	5.86197757002083e-14\\
4.64892442426821	7.32747196252603e-15\\
4.65894941608653	4.50750547997814e-14\\
4.66897440790484	1.98729921407903e-14\\
4.67904156948105	5.85087533977457e-14\\
4.68910873105727	7.32747196252603e-15\\
4.69917589263348	4.49640324973188e-14\\
4.7092430542097	1.97619698383278e-14\\
4.71935279189367	5.83977310952832e-14\\
4.72946252957763	7.43849426498855e-15\\
4.7395722672616	4.47419878923938e-14\\
4.74968200494557	1.96509475358653e-14\\
4.75983473174927	5.83977310952832e-14\\
4.76998745855297	7.54951656745106e-15\\
4.78014018535667	4.46309655899313e-14\\
4.79029291216037	1.95399252334028e-14\\
4.8004890483408	5.81756864903582e-14\\
4.81068518452122	7.43849426498855e-15\\
4.82088132070165	4.45199432874688e-14\\
4.83107745688208	1.95399252334028e-14\\
4.84131742876219	5.80646641878957e-14\\
4.85155740064229	7.43849426498855e-15\\
4.8617973725224	4.44089209850063e-14\\
4.87203734440251	1.94289029309402e-14\\
4.88232158523569	5.78426195829707e-14\\
4.89260582606888	7.43849426498855e-15\\
4.90289006690206	4.42978986825437e-14\\
4.91317430773525	1.94289029309402e-14\\
4.92350325885451	5.77315972805081e-14\\
4.93383220997378	7.32747196252603e-15\\
4.94416116109305	4.41868763800812e-14\\
4.95449011221231	1.93178806284777e-14\\
4.96486422170276	5.77315972805081e-14\\
4.97523833119322	7.43849426498855e-15\\
4.98561244068367	4.39648317751562e-14\\
4.99598655017412	1.93178806284777e-14\\
5.00640627395839	5.73985303731206e-14\\
5.01682599774266	7.32747196252603e-15\\
5.02724572152694	4.39648317751562e-14\\
5.03766544531121	1.93178806284777e-14\\
5.04813124666974	5.71764857681956e-14\\
5.05859704802827	7.21644966006352e-15\\
5.0690628493868	4.38538094726937e-14\\
5.07952865074533	1.92068583260152e-14\\
5.09004100143523	5.7065463465733e-14\\
5.10055335212514	7.32747196252603e-15\\
5.11106570281504	4.37427871702312e-14\\
5.12157805350494	1.92068583260152e-14\\
5.13213743298625	5.69544411632705e-14\\
5.14269681246755	7.32747196252603e-15\\
5.15325619194885	4.35207425653061e-14\\
5.16381557143015	1.90958360235527e-14\\
5.17442246764722	5.67323965583455e-14\\
5.1850293638643	7.32747196252603e-15\\
5.19563626008137	4.35207425653061e-14\\
5.20624315629844	1.90958360235527e-14\\
5.21689806557204	5.65103519534205e-14\\
5.22755297484563	7.21644966006352e-15\\
5.23820788411923	4.32986979603811e-14\\
5.24886279339282	1.90958360235527e-14\\
5.25956622085953	5.65103519534205e-14\\
5.27026964832624	7.21644966006352e-15\\
5.28097307579295	4.31876756579186e-14\\
5.29167650325966	1.89848137210902e-14\\
5.30242896340108	5.62883073484954e-14\\
5.31318142354251	7.32747196252603e-15\\
5.32393388368394	4.29656310529936e-14\\
5.33468634382537	1.88737914186277e-14\\
5.34548835981846	5.60662627435704e-14\\
5.35629037581155	7.105427357601e-15\\
5.36709239180464	4.29656310529936e-14\\
5.37789440779773	1.88737914186277e-14\\
5.38874651272417	5.58442181386454e-14\\
5.39959861765061	7.105427357601e-15\\
5.41045072257705	4.2854608750531e-14\\
5.42130282750349	1.88737914186277e-14\\
5.43220556367259	5.55111512312578e-14\\
5.44310829984169	6.99440505513849e-15\\
5.45401103601079	4.27435864480685e-14\\
5.46491377217989	1.89848137210902e-14\\
5.47586769235055	5.52891066263328e-14\\
5.48682161252121	6.99440505513849e-15\\
5.49777553269188	4.27435864480685e-14\\
5.50872945286254	1.88737914186277e-14\\
5.51973511980852	5.50670620214078e-14\\
5.5307407867545	6.99440505513849e-15\\
5.54174645370048	4.25215418431435e-14\\
5.55275212064646	1.88737914186277e-14\\
5.5638101076295	5.48450174164827e-14\\
5.57486809461254	6.88338275267597e-15\\
5.58592608159558	4.25215418431435e-14\\
5.59698406857862	1.88737914186277e-14\\
5.60809495979115	5.46229728115577e-14\\
5.61920585100368	6.77236045021345e-15\\
5.63031674221621	4.2410519540681e-14\\
5.64142763342874	1.88737914186277e-14\\
5.65259202401047	5.42899059041702e-14\\
5.66375641459219	6.77236045021345e-15\\
5.67492080517392	4.21884749357559e-14\\
5.68608519575564	1.87627691161651e-14\\
5.69730369280843	5.41788836017076e-14\\
5.70852218986121	6.66133814775094e-15\\
5.719740686914	4.20774526332934e-14\\
5.73095918396678	1.87627691161651e-14\\
5.74223240540217	5.39568389967826e-14\\
5.75350562683755	6.66133814775094e-15\\
5.76477884827294	4.19664303308309e-14\\
5.77605206970832	1.86517468137026e-14\\
5.78738064638032	5.36237720893951e-14\\
5.79870922305231	6.66133814775094e-15\\
5.81003779972431	4.17443857259059e-14\\
5.8213663763963	1.86517468137026e-14\\
5.83275095107969	5.35127497869325e-14\\
5.84413552576308	6.66133814775094e-15\\
5.85552010044646	4.15223411209809e-14\\
5.86690467512985	1.84297022087776e-14\\
5.87834590369998	5.32907051820075e-14\\
5.88978713227011	6.66133814775094e-15\\
5.90122836084025	4.14113188185183e-14\\
5.91266958941038	1.84297022087776e-14\\
5.92416814082324	5.30686605770825e-14\\
5.9356666922361	6.66133814775094e-15\\
5.94716524364896	4.13002965160558e-14\\
5.95866379506182	1.83186799063151e-14\\
5.97022035244649	5.295763827462e-14\\
5.98177690983116	6.66133814775094e-15\\
5.99333346721583	4.09672296086683e-14\\
6.00489002460049	1.82076576038526e-14\\
6.01650528373054	5.28466159721575e-14\\
6.02812054286059	6.77236045021345e-15\\
6.03973580199063	4.07451850037432e-14\\
6.05135106112068	1.79856129989275e-14\\
6.06302573298457	5.25135490647699e-14\\
6.07470040484846	6.66133814775094e-15\\
6.08637507671235	4.05231403988182e-14\\
6.09804974857625	1.79856129989275e-14\\
6.10978455915361	5.21804821573824e-14\\
6.12151936973097	6.66133814775094e-15\\
6.13325418030833	4.04121180963557e-14\\
6.14498899088569	1.79856129989275e-14\\
6.15678468114828	5.19584375524573e-14\\
6.16858037141086	6.55031584528842e-15\\
6.18037606167345	4.01900734914307e-14\\
6.19217175193603	1.7874590696465e-14\\
6.20402907883544	5.17363929475323e-14\\
6.21588640573484	6.55031584528842e-15\\
6.22774373263424	4.00790511889682e-14\\
6.23960105953364	1.7874590696465e-14\\
6.25152079648947	5.14033260401447e-14\\
6.2634405334453	6.43929354282591e-15\\
6.27536027040113	3.98570065840431e-14\\
6.28728000735696	1.77635683940025e-14\\
6.2992629439367	5.11812814352197e-14\\
6.31124588051644	6.43929354282591e-15\\
6.32322881709618	3.97459842815806e-14\\
6.33521175367592	1.765254609154e-14\\
6.34725869756642	5.08482145278322e-14\\
6.35930564145691	6.43929354282591e-15\\
6.3713525853474	3.94129173741931e-14\\
6.3833995292379	1.75415237890775e-14\\
6.39551130573695	5.06261699229071e-14\\
6.40762308223601	6.43929354282591e-15\\
6.41973485873506	3.93018950717305e-14\\
6.43184663523411	1.75415237890775e-14\\
6.44402408779825	5.01820807130571e-14\\
6.45620154036239	6.21724893790088e-15\\
6.46837899292652	3.90798504668055e-14\\
6.48055644549066	1.7430501486615e-14\\
6.49280043746748	5.00710584105946e-14\\
6.5050444294443	6.21724893790088e-15\\
6.51728842142112	3.88578058618805e-14\\
6.52953241339793	1.73194791841524e-14\\
6.5418438273366	4.9737991503207e-14\\
6.55415524127526	6.32827124036339e-15\\
6.56646665521392	3.86357612569554e-14\\
6.57877806915258	1.72084568816899e-14\\
6.59115780864942	4.94049245958195e-14\\
6.60353754814627	6.21724893790088e-15\\
6.61591728764311	3.84137166520304e-14\\
6.62829702713995	1.72084568816899e-14\\
6.64074601624377	4.91828799908944e-14\\
6.65319500534758	6.21724893790088e-15\\
6.66564399445139	3.81916720471054e-14\\
6.67809298355521	1.70974345792274e-14\\
6.69061216875419	4.87387907810444e-14\\
6.70313135395318	6.10622663543836e-15\\
6.71565053915216	3.79696274421804e-14\\
6.72816972435114	1.69864122767649e-14\\
6.74076007448507	4.85167461761193e-14\\
6.75335042461899	6.10622663543836e-15\\
6.76594077475292	3.77475828372553e-14\\
6.77853112488684	1.69864122767649e-14\\
6.79119363328311	4.81836792687318e-14\\
6.80385614167938	6.10622663543836e-15\\
6.81651865007565	3.74145159298678e-14\\
6.82918115847192	1.67643676718399e-14\\
6.84191684190172	4.77395900588817e-14\\
6.85465252533151	6.10622663543836e-15\\
6.86738820876131	3.73034936274053e-14\\
6.88012389219111	1.67643676718399e-14\\
6.8929337927896	4.74065231514942e-14\\
6.90574369338808	5.88418203051333e-15\\
6.91855359398656	3.70814490224802e-14\\
6.93136349458505	1.66533453693773e-14\\
6.94424868139473	4.69624339416441e-14\\
6.95713386820441	5.88418203051333e-15\\
6.97001905501408	3.68594044175552e-14\\
6.98290424182376	1.65423230669148e-14\\
6.99586581007284	4.66293670342566e-14\\
7.00882737832193	5.88418203051333e-15\\
7.02178894657101	3.65263375101677e-14\\
7.03475051482009	1.64313007644523e-14\\
7.04778958943027	4.64073224293315e-14\\
7.06082866404044	5.88418203051333e-15\\
7.07386773865062	3.61932706027801e-14\\
7.08690681326079	1.63202784619898e-14\\
7.10002454751536	4.59632332194815e-14\\
7.11314228176992	5.77315972805081e-15\\
7.12626001602448	3.59712259978551e-14\\
7.13937775027905	1.62092561595273e-14\\
7.15257532859946	4.55191440096314e-14\\
7.16577290691987	5.6621374255883e-15\\
7.17897048524028	3.56381590904675e-14\\
7.19216806356068	1.60982338570648e-14\\
7.20544670103548	4.50750547997814e-14\\
7.21872533851028	5.6621374255883e-15\\
7.23200397598508	3.54161144855425e-14\\
7.24528261345988	1.59872115546023e-14\\
7.2586435601875	4.47419878923938e-14\\
7.27200450691511	5.6621374255883e-15\\
7.28536545364272	3.50830475781549e-14\\
7.29872640037034	1.58761892521397e-14\\
7.31217093962371	4.42978986825437e-14\\
7.32561547887708	5.55111512312578e-15\\
7.33906001813046	3.48610029732299e-14\\
7.35250455738383	1.57651669496772e-14\\
7.36603400873681	4.38538094726937e-14\\
7.37956346008979	5.44009282066327e-15\\
7.39309291144277	3.44169137633799e-14\\
7.40662236279575	1.56541446472147e-14\\
7.42023808324719	4.34097202628436e-14\\
7.43385380369863	5.44009282066327e-15\\
7.44746952415006	3.41948691584548e-14\\
7.4610852446015	1.55431223447522e-14\\
7.47478862987629	4.29656310529936e-14\\
7.48849201515108	5.44009282066327e-15\\
7.50219540042587	3.37507799486048e-14\\
7.51589878570066	1.54321000422897e-14\\
7.52969127264357	4.2410519540681e-14\\
7.54348375958647	5.21804821573824e-15\\
7.55727624652937	3.35287353436797e-14\\
7.57106873347227	1.53210777398272e-14\\
7.58495180146776	4.19664303308309e-14\\
7.59883486946326	5.32907051820075e-15\\
7.61271793745875	3.31956684362922e-14\\
7.62660100545424	1.50990331349021e-14\\
7.64057617834747	4.15223411209809e-14\\
7.6545513512407	5.21804821573824e-15\\
7.66852652413393	3.27515792264421e-14\\
7.68250169702716	1.49880108324396e-14\\
7.69657054527904	4.10782519111308e-14\\
7.71063939353093	5.10702591327572e-15\\
7.72470824178282	3.24185123190546e-14\\
7.7387770900347	1.47659662275146e-14\\
7.75294123299068	4.04121180963557e-14\\
7.76710537594666	5.10702591327572e-15\\
7.78126951890264	3.2085445411667e-14\\
7.79543366185862	1.46549439250521e-14\\
7.80969476969516	3.99680288865056e-14\\
7.8239558775317	5.10702591327572e-15\\
7.83821698536823	3.15303338993544e-14\\
7.85247809320477	1.4432899320127e-14\\
7.86683788984999	3.94129173741931e-14\\
7.8811976864952	4.9960036108132e-15\\
7.89555748314041	3.11972669919669e-14\\
7.90991727978563	1.43218770176645e-14\\
7.9243775449477	3.88578058618805e-14\\
7.93883781010978	4.88498130835069e-15\\
7.95329807527185	3.06421554796543e-14\\
7.96775834043393	1.40998324127395e-14\\
7.98232091364819	3.83026943495679e-14\\
7.99688348686244	4.88498130835069e-15\\
8.0114460600767	3.01980662698043e-14\\
8.02600863329096	1.37667655053519e-14\\
8.04067541501396	3.77475828372553e-14\\
8.05534219673697	4.9960036108132e-15\\
8.07000897845997	2.96429547574917e-14\\
8.08467576018298	1.36557432028894e-14\\
8.0994487168151	3.70814490224802e-14\\
8.11422167344722	4.88498130835069e-15\\
8.12899463007934	2.91988655476416e-14\\
8.14376758671146	1.35447209004269e-14\\
8.1586487531091	3.64153152077051e-14\\
8.17352991950675	4.66293670342566e-15\\
8.1884110859044	2.88657986402541e-14\\
8.20329225230204	1.32116539930394e-14\\
8.21828373575118	3.574918139293e-14\\
8.23327521920032	4.66293670342566e-15\\
8.24826670264945	2.83106871279415e-14\\
8.26325818609859	1.29896093881143e-14\\
8.27836217017066	3.51940698806175e-14\\
8.29346615424273	4.66293670342566e-15\\
8.3085701383148	2.76445533131664e-14\\
8.32367412238687	1.27675647831893e-14\\
8.33889287092472	3.44169137633799e-14\\
8.35411161946257	4.55191440096314e-15\\
8.36933036800042	2.70894418008538e-14\\
8.38454911653827	1.25455201782643e-14\\
8.39988497972518	3.37507799486048e-14\\
8.41522084291208	4.55191440096314e-15\\
8.43055670609898	2.65343302885412e-14\\
8.44589256928588	1.22124532708767e-14\\
8.46134798619136	3.30846461338297e-14\\
8.47680340309684	4.44089209850063e-15\\
8.49225882000231	2.58681964737661e-14\\
8.50771423690779	1.19904086659517e-14\\
8.52329174240573	3.23074900165921e-14\\
8.53886924790367	4.44089209850063e-15\\
8.55444675340161	2.52020626589911e-14\\
8.57002425889955	1.17683640610267e-14\\
8.58572649044944	3.14193115968919e-14\\
8.60142872199933	4.32986979603811e-15\\
8.61713095354922	2.4757973449141e-14\\
8.63283318509911	1.15463194561016e-14\\
8.64866288641169	3.05311331771918e-14\\
8.66449258772426	4.21884749357559e-15\\
8.68032228903684	2.39808173319034e-14\\
8.69615199034942	1.12132525487141e-14\\
8.71211202044306	2.96429547574917e-14\\
8.7280720505367	4.10782519111308e-15\\
8.74403208063034	2.33146835171283e-14\\
8.75999211072399	1.08801856413265e-14\\
8.77608545081172	2.8643754035329e-14\\
8.79217879089946	3.99680288865056e-15\\
8.80827213098719	2.26485497023532e-14\\
8.82436547107493	1.06581410364015e-14\\
8.84059523179885	2.77555756156289e-14\\
8.85682499252278	3.88578058618805e-15\\
8.87305475324671	2.19824158875781e-14\\
8.88928451397064	1.0325074129014e-14\\
8.9056539454559	2.67563748934663e-14\\
8.92202337694115	3.77475828372553e-15\\
8.93839280842641	2.12052597703405e-14\\
8.95476223991167	9.99200722162641e-15\\
8.97127474100433	2.56461518688411e-14\\
8.987787242097	3.66373598126302e-15\\
9.00429974318966	2.03170813506404e-14\\
9.02081224428233	9.65894031423886e-15\\
9.0374713729445	2.46469511466785e-14\\
9.05413050160667	3.66373598126302e-15\\
9.07078963026884	1.95399252334028e-14\\
9.08744875893101	9.2148511043888e-15\\
9.10425824450065	2.34257058195908e-14\\
9.12106773007028	3.5527136788005e-15\\
9.13787721563991	1.85407245112401e-14\\
9.15468670120955	8.88178419700125e-15\\
9.17165045698708	2.22044604925031e-14\\
9.18861421276461	3.44169137633799e-15\\
9.20557796854214	1.75415237890775e-14\\
9.22254172431967	8.43769498715119e-15\\
9.23966386191763	2.1094237467878e-14\\
9.25678599951559	3.33066907387547e-15\\
9.27390813711355	1.65423230669148e-14\\
9.29103027471151	7.99360577730113e-15\\
9.30828551504259	1.95399252334028e-14\\
9.32554075537366	3.33066907387547e-15\\
9.34279599570473	1.53210777398272e-14\\
9.36005123603581	7.43849426498855e-15\\
9.37740098280311	1.7874590696465e-14\\
9.39475072957042	3.10862446895044e-15\\
9.41210047633773	1.38777878078145e-14\\
9.42945022310504	6.82787160144471e-15\\
9.44689582322271	1.6153745008296e-14\\
9.46434142334038	2.94209101525666e-15\\
9.48178702345805	1.24344978758018e-14\\
9.49923262357573	6.16173778666962e-15\\
9.51677513419308	1.4432899320127e-14\\
9.53431764481043	2.72004641033163e-15\\
9.55186015542778	1.10467190950203e-14\\
9.56940266604513	5.55111512312578e-15\\
9.58704316415657	1.26565424807268e-14\\
9.60468366226802	2.4980018054066e-15\\
9.62232416037946	9.71445146547012e-15\\
9.6399646584909	4.94049245958195e-15\\
9.65770423902654	1.08246744900953e-14\\
9.67544381956217	2.33146835171283e-15\\
9.6931834000978	8.32667268468867e-15\\
9.71092298063344	4.32986979603811e-15\\
9.72876275712183	9.04831765069503e-15\\
9.74660253361022	2.1094237467878e-15\\
9.76444231009861	6.88338275267597e-15\\
9.782282086587	3.66373598126302e-15\\
9.80022319117994	7.27196081129478e-15\\
9.81816429577288	1.94289029309402e-15\\
9.83610540036582	5.38458166943201e-15\\
9.85404650495876	2.99760216648792e-15\\
9.87209008914481	5.49560397189452e-15\\
9.89013367333085	1.77635683940025e-15\\
9.9081772575169	3.94129173741931e-15\\
9.92622084170295	2.33146835171283e-15\\
9.94436807654274	3.66373598126302e-15\\
9.96251531138254	1.55431223447522e-15\\
9.98066254622234	2.55351295663786e-15\\
9.99880978106213	1.72084568816899e-15\\
10.0170618577606	1.83186799063151e-15\\
10.0353139344591	1.33226762955019e-15\\
10.0535660111575	1.11022302462516e-15\\
10.071818087856	1.0547118733939e-15\\
10.0901762182373	5.55111512312578e-17\\
10.1085343486186	1.11022302462516e-15\\
10.1268924789999	3.88578058618805e-16\\
10.1452506093812	4.44089209850063e-16\\
10.163716026191	1.83186799063151e-15\\
10.1821814430008	8.88178419700125e-16\\
10.2006468598106	1.83186799063151e-15\\
10.2191122766204	2.22044604925031e-16\\
10.237686234429	3.71924713249427e-15\\
10.2562601922376	7.21644966006352e-16\\
10.2748341500462	3.33066907387547e-15\\
10.2934081078548	9.43689570931383e-16\\
10.3120918830375	5.55111512312578e-15\\
10.3307756582203	5.55111512312578e-16\\
10.3494594334031	4.88498130835069e-15\\
10.3681432085859	1.55431223447522e-15\\
10.3869381001644	7.43849426498855e-15\\
10.4057329917429	3.33066907387547e-16\\
10.4245278833213	6.32827124036339e-15\\
10.4433227748998	2.27595720048157e-15\\
10.4622301050096	9.38138455808257e-15\\
10.4811374351193	5.55111512312578e-17\\
10.5000447652291	7.7715611723761e-15\\
10.5189520953388	2.94209101525666e-15\\
10.5379732097484	1.12132525487141e-14\\
10.556994324158	1.11022302462516e-16\\
10.5760154385676	9.32587340685131e-15\\
10.5950365529771	3.60822483003176e-15\\
10.6141728216781	1.31006316905768e-14\\
10.6333090903791	3.33066907387547e-16\\
10.6524453590801	1.08246744900953e-14\\
10.6715816277811	4.27435864480685e-15\\
10.6908344455163	1.50435219836709e-14\\
10.7100872632516	5.55111512312578e-16\\
10.7293400809869	1.23789867245705e-14\\
10.7485928987221	4.9960036108132e-15\\
10.7679636856929	1.69309011255336e-14\\
10.7873344726636	7.7715611723761e-16\\
10.8067052596344	1.39332989590457e-14\\
10.8260760466051	5.71764857681956e-15\\
10.8455662488972	1.88737914186277e-14\\
10.8650564511894	9.99200722162641e-16\\
10.8845466534815	1.54876111935209e-14\\
10.9040368557736	6.38378239159465e-15\\
10.9236479463274	2.08721928629529e-14\\
10.9432590368811	1.22124532708767e-15\\
10.9628701274349	1.70419234279962e-14\\
10.9824812179887	7.105427357601e-15\\
11.0022146969848	2.28705943072782e-14\\
11.0219481759809	1.4432899320127e-15\\
11.041681654977	1.85962356624714e-14\\
11.0614151339731	7.71605002114484e-15\\
11.0812725294807	2.48689957516035e-14\\
11.1011299249883	1.72084568816899e-15\\
11.1209873204959	2.00950367457153e-14\\
11.1408447160036	8.38218383591993e-15\\
11.1608275848708	2.692290834716e-14\\
11.180810453738	1.99840144432528e-15\\
11.2007933226052	2.16493489801906e-14\\
11.2207761914724	9.10382880192628e-15\\
11.2408861199495	2.88657986402541e-14\\
11.2609960484265	2.16493489801906e-15\\
11.2811059769036	2.3259172365897e-14\\
11.3012159053807	9.82547376793264e-15\\
11.3214545099205	3.08642000845794e-14\\
11.3416931144603	2.44249065417534e-15\\
11.3619317190001	2.48134846003722e-14\\
11.3821703235399	1.0547118733939e-14\\
11.4025392513608	3.29736238313671e-14\\
11.4229081791817	2.72004641033163e-15\\
11.4432771070026	2.64233079860787e-14\\
11.4636460348235	1.12687636999453e-14\\
11.4841469650118	3.50275364269237e-14\\
11.5046478952002	2.94209101525666e-15\\
11.5251488253886	2.80886425230165e-14\\
11.545649755577	1.19904086659517e-14\\
11.5662843996785	3.7025937871249e-14\\
11.5869190437801	3.1641356201817e-15\\
11.6075536878816	2.96984659087229e-14\\
11.6281883319832	1.2712053631958e-14\\
11.6489584349037	3.91353616180368e-14\\
11.6697285378243	3.44169137633799e-15\\
11.6904986407448	3.13082892944294e-14\\
11.7112687436654	1.34336985979644e-14\\
11.7321760845019	4.12447853648246e-14\\
11.7530834253385	3.66373598126302e-15\\
11.7739907661751	3.29736238313671e-14\\
11.7948981070116	1.4210854715202e-14\\
11.8159445000986	4.32986979603811e-14\\
11.8369908931856	3.88578058618805e-15\\
11.8580372862726	3.46389583683049e-14\\
11.8790836793596	1.49880108324396e-14\\
11.900270975065	4.53526105559376e-14\\
11.9214582707703	4.10782519111308e-15\\
11.9426455664757	3.63598040564739e-14\\
11.9638328621811	1.5709655798446e-14\\
11.9851629478153	4.75175454539567e-14\\
12.0064930334495	4.38538094726937e-15\\
12.0278231190837	3.80251385934116e-14\\
12.0491532047179	1.64868119156836e-14\\
12.0706280057176	4.96269692007445e-14\\
12.0921028067173	4.6074255521944e-15\\
12.113577607717	3.97459842815806e-14\\
12.1350524087168	1.72084568816899e-14\\
12.1566738894695	5.17919040987636e-14\\
12.1782953702222	4.88498130835069e-15\\
12.1999168509749	4.14113188185183e-14\\
12.2215383317276	1.79856129989275e-14\\
12.2433084968313	5.40123501480139e-14\\
12.265078661935	5.16253706450698e-15\\
12.2868488270387	4.31321645066873e-14\\
12.3086189921423	1.87627691161651e-14\\
12.3305398873956	5.61217738948017e-14\\
12.3524607826488	5.38458166943201e-15\\
12.3743816779021	4.48530101948563e-14\\
12.3963025731553	1.95399252334028e-14\\
12.4183762865519	5.83977310952832e-14\\
12.4404499999485	5.6621374255883e-15\\
12.462523713345	4.65738558830253e-14\\
12.4845974267416	2.03170813506404e-14\\
12.50682608981	6.05626659933023e-14\\
12.5290547528783	5.93969318174459e-15\\
12.5512834159467	4.83502127224256e-14\\
12.5735120790151	2.1094237467878e-14\\
12.5958978680728	6.28941343450151e-14\\
12.6182836571306	6.27276008913213e-15\\
12.6406694461884	5.00155472593633e-14\\
12.6630552352461	2.18713935851156e-14\\
12.6856003725544	6.50590692430342e-14\\
12.7081455098627	6.49480469405717e-15\\
12.730690647171	5.18474152499948e-14\\
12.7532357844793	2.26485497023532e-14\\
12.7759425395622	6.73350264435157e-14\\
12.7986492946451	6.82787160144471e-15\\
12.821356049728	5.36237720893951e-14\\
12.844062804811	2.34812169708221e-14\\
12.8668397124378	6.85007606193722e-14\\
12.8896166200647	7.49400541621981e-15\\
12.9123935276916	5.38458166943201e-14\\
12.9351704353184	2.36477504245158e-14\\
12.9579981520666	6.90003609804535e-14\\
12.9808258688147	7.54951656745106e-15\\
13.0036535855629	5.42899059041702e-14\\
13.026481302311	2.38697950294409e-14\\
13.049363098101	6.94999613415348e-14\\
13.0722448938909	7.66053886991358e-15\\
13.0951266896808	5.46229728115577e-14\\
13.1180084854707	2.40363284831346e-14\\
13.1409474667556	6.99440505513849e-14\\
13.1638864480404	7.71605002114484e-15\\
13.1868254293253	5.50115508701765e-14\\
13.2097644106101	2.42028619368284e-14\\
13.2327635996633	7.03881397612349e-14\\
13.2557627887165	7.82707232360735e-15\\
13.2787619777697	5.53446177775641e-14\\
13.3017611668229	2.43693953905222e-14\\
13.3248235100991	7.0832228971085e-14\\
13.3478858533754	7.88258347483861e-15\\
13.3709481966517	5.56776846849516e-14\\
13.3940105399279	2.4535928844216e-14\\
13.4171389165768	7.12208070297038e-14\\
13.4402672932257	7.93809462606987e-15\\
13.4633956698746	5.60107515923391e-14\\
13.4865240465235	2.47024622979097e-14\\
13.5097212774941	7.16093850883226e-14\\
13.5329185084647	8.04911692853238e-15\\
13.5561157394353	5.62883073484954e-14\\
13.5793129704059	2.48134846003722e-14\\
13.6025818233678	7.19979631469414e-14\\
13.6258506763298	8.1601392309949e-15\\
13.6491195292917	5.65658631046517e-14\\
13.6723883822536	2.4980018054066e-14\\
13.6957315789086	7.23310300543289e-14\\
13.7190747755636	8.1601392309949e-15\\
13.7424179722186	5.6843418860808e-14\\
13.7657611688735	2.51465515077598e-14\\
13.789181390828	7.27196081129478e-14\\
13.8126016127825	8.27116153345742e-15\\
13.8360218347369	5.7065463465733e-14\\
13.8594420566914	2.52020626589911e-14\\
13.8829419497165	7.30526750203353e-14\\
13.9064418427417	8.32667268468867e-15\\
13.9299417357668	5.73985303731206e-14\\
13.9534416287919	2.53685961126848e-14\\
13.9770238098647	7.33857419277228e-14\\
14.0006059909374	8.38218383591993e-15\\
14.0241881720102	5.76760861292769e-14\\
14.0477703530829	2.54796184151473e-14\\
14.0714374141426	7.37743199863417e-14\\
14.0951044752024	8.43769498715119e-15\\
14.1187715362621	5.78426195829707e-14\\
14.1424385973218	2.56461518688411e-14\\
14.1661931077477	7.40518757424979e-14\\
14.1899476181736	8.49320613838245e-15\\
14.2137021285995	5.81201753391269e-14\\
14.2374566390255	2.57016630200724e-14\\
14.2613011526304	7.43849426498855e-14\\
14.2851456662352	8.54871728961371e-15\\
14.3089901798401	5.83977310952832e-14\\
14.332834693445	2.58681964737661e-14\\
14.3567717508919	7.46624984060418e-14\\
14.3807088083388	8.60422844084496e-15\\
14.4046458657857	5.86197757002083e-14\\
14.4285829232326	2.60347299274599e-14\\
14.4526150559176	7.48845430109668e-14\\
14.4766471886026	8.60422844084496e-15\\
14.5006793212876	5.88973314563646e-14\\
14.5247114539725	2.60902410786912e-14\\
14.548841186964	7.52176099183544e-14\\
14.5729709199554	8.65973959207622e-15\\
14.5971006529469	5.90638649100583e-14\\
14.6212303859383	2.62012633811537e-14\\
14.6454602425121	7.54951656745106e-14\\
14.6696900990858	8.71525074330748e-15\\
14.6939199556595	5.92859095149834e-14\\
14.7181498122333	2.63122856836162e-14\\
14.7424823150752	7.57727214306669e-14\\
14.7668148179171	8.77076189453874e-15\\
14.7911473207591	5.95079541199084e-14\\
14.815479823601	2.64233079860787e-14\\
14.8399174997295	7.60502771868232e-14\\
14.864355175858	8.82627304576999e-15\\
14.8887928519865	5.97299987248334e-14\\
14.913230528115	2.65343302885412e-14\\
14.9377759105528	7.62723217917483e-14\\
14.9623212929905	8.88178419700125e-15\\
14.9868666754283	5.99520433297585e-14\\
15.0114120578661	2.6700863742235e-14\\
15.0360676892572	7.64943663966733e-14\\
15.0607233206484	8.82627304576999e-15\\
15.0853789520395	6.01740879346835e-14\\
15.1100345834307	2.68118860446975e-14\\
15.1348030187572	7.67164110015983e-14\\
15.1595714540837	8.82627304576999e-15\\
15.1843398894102	6.04516436908398e-14\\
15.2091083247367	2.69784194983913e-14\\
15.2339921326276	7.69384556065233e-14\\
15.2588759405185	8.93729534823251e-15\\
15.2837597484095	6.06181771445335e-14\\
15.3086435563004	2.70339306496226e-14\\
15.333645323629	7.72160113626796e-14\\
15.3586470909575	8.93729534823251e-15\\
15.383648858286	6.08402217494586e-14\\
15.4086506256146	2.71449529520851e-14\\
15.4337729580942	7.74380559676047e-14\\
15.4588952905739	9.04831765069503e-15\\
15.4840176230536	6.10067552031524e-14\\
15.5091399555333	2.72004641033163e-14\\
15.5343854824194	7.7715611723761e-14\\
15.5596310093056	9.04831765069503e-15\\
15.5848765361918	6.11732886568461e-14\\
15.610122063078	2.72559752545476e-14\\
15.6354934388609	7.79931674799172e-14\\
15.6608648146438	9.15933995315754e-15\\
15.6862361904268	6.13398221105399e-14\\
15.7116075662097	2.73669975570101e-14\\
15.7371074715309	7.82152120848423e-14\\
15.762607376852	9.2148511043888e-15\\
15.7881072821731	6.15063555642337e-14\\
15.8136071874943	2.74225087082414e-14\\
15.8392383339326	7.84927678409986e-14\\
15.8648694803709	9.2148511043888e-15\\
15.8905006268092	6.16728890179274e-14\\
15.9161317732476	2.75335310107039e-14\\
15.9418969045212	7.87148124459236e-14\\
15.9676620357948	9.32587340685131e-15\\
15.9934271670685	6.18394224716212e-14\\
16.0191922983421	2.75890421619351e-14\\
16.0450941933734	7.89368570508486e-14\\
16.0709960884047	9.38138455808257e-15\\
16.096897983436	6.2005955925315e-14\\
16.1227998784672	2.76445533131664e-14\\
16.1488413541292	7.92144128070049e-14\\
16.1748828297911	9.38138455808257e-15\\
16.2009243054531	6.21724893790088e-14\\
16.226965781115	2.77555756156289e-14\\
16.2531496945447	7.93809462606987e-14\\
16.2793336079744	9.43689570931383e-15\\
16.3055175214041	6.23945339839338e-14\\
16.3317014348337	2.78665979180914e-14\\
16.3580306848324	7.9658502016855e-14\\
16.384359934831	9.49240686054509e-15\\
16.4106891848297	6.25610674376276e-14\\
16.4370184348283	2.79776202205539e-14\\
16.4634959660973	7.988054662178e-14\\
16.4899734973663	9.54791801177635e-15\\
16.5164510286353	6.27831120425526e-14\\
16.5429285599043	2.80331313717852e-14\\
16.5695573662806	8.0102591226705e-14\\
16.5961861726568	9.6034291630076e-15\\
16.6228149790331	6.28941343450151e-14\\
16.6494437854093	2.80331313717852e-14\\
16.6762269105277	8.03801469828613e-14\\
16.7030100356461	9.65894031423886e-15\\
16.7297931607645	6.30606677987089e-14\\
16.7565762858829	2.81441536742477e-14\\
16.7835168283287	8.06021915877864e-14\\
16.8104573707745	9.65894031423886e-15\\
16.8373979132204	6.32272012524027e-14\\
16.8643384556662	2.82551759767102e-14\\
16.8914395705478	8.08242361927114e-14\\
16.9185406854293	9.76996261670138e-15\\
16.9456418003109	6.33937347060964e-14\\
16.9727429151925	2.83106871279415e-14\\
17.0000078170973	8.11017919488677e-14\\
17.027272719002	9.76996261670138e-15\\
17.0545376209068	6.35602681597902e-14\\
17.0818025228116	2.8421709430404e-14\\
17.1092344893586	8.12683254025615e-14\\
17.1366664559057	9.82547376793264e-15\\
17.1640984224528	6.38378239159465e-14\\
17.1915303889999	2.85882428840978e-14\\
17.219132763799	8.14903700074865e-14\\
17.2467351385981	9.82547376793264e-15\\
17.2743375133972	6.3948846218409e-14\\
17.3019398881964	2.8643754035329e-14\\
17.3297160854172	8.17679257636428e-14\\
17.3574922826381	9.88098491916389e-15\\
17.385268479859	6.41153796721028e-14\\
17.4130446770799	2.87547763377916e-14\\
17.4409981837359	8.19344592173366e-14\\
17.4689516903919	9.88098491916389e-15\\
17.4969051970478	6.43374242770278e-14\\
17.5248587037038	2.88102874890228e-14\\
17.5529930820312	8.22120149734928e-14\\
17.5811274603586	9.99200722162641e-15\\
17.6092618386859	6.45594688819529e-14\\
17.6373962170133	2.89213097914853e-14\\
17.6657151116334	8.24340595784179e-14\\
17.6940340062535	9.99200722162641e-15\\
17.7223529008736	6.47260023356466e-14\\
17.7506717954937	2.89768209427166e-14\\
17.7791789342782	8.27116153345742e-14\\
17.8076860730626	1.00475183728577e-14\\
17.836193211847	6.48925357893404e-14\\
17.8647003506315	2.90878432451791e-14\\
17.8933995511133	8.29891710907305e-14\\
17.922098751595	1.01030295240889e-14\\
17.9507979520768	6.50590692430342e-14\\
17.9794971525585	2.91988655476416e-14\\
18.0083923245633	8.31557045444242e-14\\
18.0372874965681	1.01030295240889e-14\\
18.0661826685729	6.52811138479592e-14\\
18.0950778405776	2.92543766988729e-14\\
18.124172992061	8.34332603005805e-14\\
18.1532681435443	1.01585406753202e-14\\
18.1823632950276	6.5447647301653e-14\\
18.2114584465109	2.93653990013354e-14\\
18.2407576866174	8.37663272079681e-14\\
18.270056926724	1.02695629777827e-14\\
18.2993561668305	6.5669691906578e-14\\
18.328655406937	2.94209101525666e-14\\
18.3581629537927	8.39883718128931e-14\\
18.3876705006484	1.02695629777827e-14\\
18.4171780475041	6.5891736511503e-14\\
18.4466855943597	2.95874436062604e-14\\
18.4764057786275	8.41549052665869e-14\\
18.5061259628954	1.0325074129014e-14\\
18.5358461471632	6.61692922676593e-14\\
18.565566331431	2.96984659087229e-14\\
18.5955036012303	8.44879721739744e-14\\
18.6254408710296	1.03805852802452e-14\\
18.6553781408289	6.63358257213531e-14\\
18.6853154106283	2.98094882111855e-14\\
18.7154743397902	8.47100167788994e-14\\
18.7456332689521	1.03805852802452e-14\\
18.7757921981141	6.66133814775094e-14\\
18.805951127276	2.9920510513648e-14\\
18.8363364194138	8.5043083686287e-14\\
18.8667217115517	1.03805852802452e-14\\
18.8971070036895	6.67799149312032e-14\\
18.9274922958273	3.00315328161105e-14\\
18.9581087922171	8.53206394424433e-14\\
18.9887252886069	1.04360964314765e-14\\
19.0193417849967	6.70574706873595e-14\\
19.0499582813864	3.0142555118573e-14\\
19.0808109690034	8.55981951985996e-14\\
19.1116636566204	1.04916075827077e-14\\
19.1425163442373	6.72795152922845e-14\\
19.1733690318543	3.03090885722668e-14\\
19.2044630478062	8.58757509547559e-14\\
19.235557063758	1.0547118733939e-14\\
19.2666510797099	6.75570710484408e-14\\
19.2977450956618	3.04756220259605e-14\\
19.3290857385735	8.62088178621434e-14\\
19.3604263814852	1.06026298851702e-14\\
19.3917670243968	6.78346268045971e-14\\
19.4231076673085	3.05866443284231e-14\\
19.4547004046881	8.64863736182997e-14\\
19.4862931420676	1.06026298851702e-14\\
19.5178858794472	6.81121825607534e-14\\
19.5494786168267	3.07531777821168e-14\\
19.581329093847	8.68194405256872e-14\\
19.6131795708672	1.06581410364015e-14\\
19.6450300478875	6.83897383169096e-14\\
19.6768805249077	3.08642000845794e-14\\
19.7089945750868	8.71525074330748e-14\\
19.7411086252659	1.06581410364015e-14\\
19.7732226754451	6.86672940730659e-14\\
19.8053367256242	3.10307335382731e-14\\
19.8377203807137	8.74300631892311e-14\\
19.8701040358032	1.07136521876328e-14\\
19.9024876908927	6.90003609804535e-14\\
19.9348713459822	3.11972669919669e-14\\
19.967530847752	8.78186412478499e-14\\
20.0001903495219	1.08246744900953e-14\\
20.0328498512918	6.92779167366098e-14\\
20.0655093530616	3.13638004456607e-14\\
20.0984511655557	8.82072193064687e-14\\
20.1313929780498	1.08246744900953e-14\\
20.1643347905439	6.96109836439973e-14\\
20.197276603038	3.14748227481232e-14\\
20.2305074253896	8.85957973650875e-14\\
20.2637382477412	1.08801856413265e-14\\
20.2969690700928	6.98885394001536e-14\\
20.3301998924444	3.1641356201817e-14\\
20.3637266734178	8.90398865749376e-14\\
20.3972534543913	1.0991207943789e-14\\
20.4307802353647	7.02216063075412e-14\\
20.4643070163382	3.18078896555107e-14\\
20.4981369694764	8.94839757847876e-14\\
20.5319669226146	1.0991207943789e-14\\
20.5657968757529	7.05546732149287e-14\\
20.5996268288911	3.19744231092045e-14\\
20.6337674490278	8.99280649946377e-14\\
20.6679080691645	1.11022302462516e-14\\
20.7020486893012	7.08877401223162e-14\\
20.7361893094379	3.2085445411667e-14\\
20.7706483924226	9.0427665355719e-14\\
20.8051074754072	1.12132525487141e-14\\
20.8395665583918	7.12208070297038e-14\\
20.8740256413764	3.22519788653608e-14\\
20.908811298434	9.08717545655691e-14\\
20.9435969554916	1.13242748511766e-14\\
20.9783826125492	7.16648962395539e-14\\
21.0131682696068	3.24740234702858e-14\\
21.0482889562591	9.13713549266504e-14\\
21.0834096429114	1.13242748511766e-14\\
21.1185303295637	7.20534742981727e-14\\
21.153651016216	3.26960680752109e-14\\
21.189115551059	9.18709552877317e-14\\
21.224580085902	1.13797860024079e-14\\
21.260044620745	7.24975635080227e-14\\
21.295509155588	3.29181126801359e-14\\
21.3313267450985	9.2370555648813e-14\\
21.367144334609	1.14352971536391e-14\\
21.4029619241194	7.29416527178728e-14\\
21.4387795136299	3.31401572850609e-14\\
21.474959783243	9.29256671611256e-14\\
21.5111400528561	1.15463194561016e-14\\
21.5473203224691	7.34412530789541e-14\\
21.5835005920822	3.3362201889986e-14\\
21.6200536143196	9.35918009759007e-14\\
21.656606636557	1.16573417585641e-14\\
21.6931596587944	7.39408534400354e-14\\
21.7297126810318	3.36397576461422e-14\\
21.7666490124925	9.41469124882133e-14\\
21.8035853439532	1.16573417585641e-14\\
21.8405216754139	7.4495964952348e-14\\
21.8774580068746	3.39173134022985e-14\\
21.9147887220248	9.48130463029884e-14\\
21.9521194371751	1.17683640610267e-14\\
21.9894501523253	7.50510764646606e-14\\
22.0267808674756	3.41948691584548e-14\\
22.0645176030464	9.54791801177635e-14\\
22.1022543386172	1.18238752122579e-14\\
22.139991074188	7.56616991282044e-14\\
22.1777278097588	3.44724249146111e-14\\
22.2158828101726	9.62008250837698e-14\\
22.2540378105863	1.19348975147204e-14\\
22.2921928110001	7.62723217917483e-14\\
22.3303478114138	3.48054918219987e-14\\
22.3689339797914	9.70334923522387e-14\\
22.407520148169	1.21014309684142e-14\\
22.4461063165466	7.68829444552921e-14\\
22.4846924849242	3.50830475781549e-14\\
22.5237234418978	9.78661596207075e-14\\
22.5627543988714	1.22124532708767e-14\\
22.601785355845	7.76045894212984e-14\\
22.6408163128186	3.54161144855425e-14\\
22.6803064595616	9.88098491916389e-14\\
22.7197966063045	1.23789867245705e-14\\
22.7592867530475	7.83262343873048e-14\\
22.7987768997904	3.574918139293e-14\\
22.838741493306	9.97535387625703e-14\\
22.8787060868217	1.2490009027033e-14\\
22.9186706803373	7.91033905045424e-14\\
22.9586352738529	3.61377594515488e-14\\
22.9990905077973	1.00808250635964e-13\\
23.0395457417417	1.26565424807268e-14\\
23.0800009756861	7.99360577730113e-14\\
23.1204562096305	3.65263375101677e-14\\
23.1614193085936	1.01862962509358e-13\\
23.2023824075567	1.27675647831893e-14\\
23.2433455065199	8.07687250414801e-14\\
23.284308605483	3.69704267200177e-14\\
23.3257979315477	1.03028696685215e-13\\
23.3672872576124	1.29340982368831e-14\\
23.4087765836772	8.17679257636428e-14\\
23.4502659097419	3.74145159298678e-14\\
23.4922104776085	1.03361763592602e-13\\
23.5341550454751	1.34336985979644e-14\\
23.5760996133418	8.14903700074865e-14\\
23.6180441812084	3.73590047786365e-14\\
23.6604170261043	1.03084207836446e-13\\
23.7027898710001	1.34336985979644e-14\\
23.745162715896	8.12128142513302e-14\\
23.7875355607918	3.7247982476174e-14\\
23.8303479690752	1.02751140929058e-13\\
23.8731603773585	1.34336985979644e-14\\
23.9159727856418	8.09352584951739e-14\\
23.9587851939251	3.71369601737115e-14\\
24.0020482633235	1.02473585172902e-13\\
24.0453113327218	1.34336985979644e-14\\
24.0885744021202	8.07132138902489e-14\\
24.1318374715185	3.7025937871249e-14\\
24.1755626771472	1.02140518265514e-13\\
24.219287882776	1.33781874467331e-14\\
24.2630130884048	8.04356581340926e-14\\
24.3067382940335	3.69704267200177e-14\\
24.3509374990427	1.01807451358127e-13\\
24.3951367040518	1.33781874467331e-14\\
24.4393359090609	8.02691246803988e-14\\
24.48353511407	3.68594044175552e-14\\
24.5282205890312	1.01529895601971e-13\\
24.5729060639923	1.34336985979644e-14\\
24.6175915389535	7.99915689242425e-14\\
24.6622770139146	3.68038932663239e-14\\
24.7074614574793	1.01141317543352e-13\\
24.7526459010441	1.33226762955019e-14\\
24.7978303446088	7.98250354705488e-14\\
24.8430147881735	3.66928709638614e-14\\
24.888711348921	1.00919272938427e-13\\
24.9344079096686	1.33781874467331e-14\\
24.9801044704162	7.94919685631612e-14\\
25.0258010311637	3.65818486613989e-14\\
25.0720233306192	1.0064171718227e-13\\
25.1182456300747	1.33226762955019e-14\\
25.1644679295302	7.93254351094674e-14\\
25.2106902289857	3.65263375101677e-14\\
25.2574523862228	1.00308650274883e-13\\
25.3042145434599	1.33226762955019e-14\\
25.350976700697	7.91033905045424e-14\\
25.3977388579342	3.64153152077051e-14\\
25.4450555156315	1.00086605669958e-13\\
25.4923721733288	1.33781874467331e-14\\
25.5396888310261	7.88258347483861e-14\\
25.5870054887234	3.63042929052426e-14\\
25.6348918408033	9.98090499138016e-14\\
25.6827781928832	1.33781874467331e-14\\
25.7306645449631	7.86037901434611e-14\\
25.7785508970431	3.62487817540114e-14\\
25.8270227182624	9.95870053088765e-14\\
25.8754945394818	1.34336985979644e-14\\
25.9239663607012	7.83817455385361e-14\\
25.9724381819206	3.61377594515488e-14\\
26.0215118591534	9.93094495527203e-14\\
26.0705855363863	1.34336985979644e-14\\
26.1196592136191	7.81041897823798e-14\\
26.168732890852	3.60267371490863e-14\\
26.2184254566214	9.9031893796564e-14\\
26.2681180223908	1.34336985979644e-14\\
26.3178105881602	7.7937656328686e-14\\
26.3675031539296	3.59712259978551e-14\\
26.4178323219975	9.87543380404077e-14\\
26.4681614900653	1.33781874467331e-14\\
26.5184906581332	7.77711228749922e-14\\
26.5688198262011	3.59712259978551e-14\\
26.6198040298383	9.84767822842514e-14\\
26.6707882334756	1.33781874467331e-14\\
26.7217724371128	7.75490782700672e-14\\
26.7727566407501	3.59157148466238e-14\\
26.8244150733526	9.81992265280951e-14\\
26.8760735059551	1.33781874467331e-14\\
26.9277319385576	7.73825448163734e-14\\
26.9793903711602	3.58046925441613e-14\\
27.0317430299247	9.79216707719388e-14\\
27.0840956886892	1.33781874467331e-14\\
27.1364483474537	7.71605002114484e-14\\
27.1888010062182	3.574918139293e-14\\
27.2418687387887	9.76996261670138e-14\\
27.2949364713593	1.33781874467331e-14\\
27.3480042039299	7.69939667577546e-14\\
27.4010719365005	3.574918139293e-14\\
27.4548764910367	9.73665592596262e-14\\
27.508681045573	1.33226762955019e-14\\
27.5624856001092	7.68274333040608e-14\\
27.6162901546455	3.56936702416988e-14\\
27.6708542335196	9.70890035034699e-14\\
27.7254183123937	1.33226762955019e-14\\
27.7799823912678	7.66053886991358e-14\\
27.834546470142	3.56381590904675e-14\\
27.8898937876417	9.68669588985449e-14\\
27.9452411051415	1.33226762955019e-14\\
28.0005884226412	7.6438855245442e-14\\
28.055935740141	3.5527136788005e-14\\
28.1120910845417	9.66449142936199e-14\\
28.1682464289425	1.33781874467331e-14\\
28.2244017733432	7.6216810640517e-14\\
28.2805571177439	3.5527136788005e-14\\
28.3375464180691	9.64228696886948e-14\\
28.3945357183943	1.33781874467331e-14\\
28.4515250187195	7.5994766035592e-14\\
28.5085143190447	3.54161144855425e-14\\
28.5663647170539	9.62008250837698e-14\\
28.6242151150631	1.33781874467331e-14\\
28.6820655130723	7.58282325818982e-14\\
28.7399159110815	3.53606033343112e-14\\
28.7983043617203	9.37028232783632e-14\\
28.856692812359	1.45439216225896e-14\\
28.9150812629978	7.20534742981727e-14\\
28.9734697136365	3.37507799486048e-14\\
29.0321280026001	8.8484775062625e-14\\
29.0907862915636	1.37112543541207e-14\\
29.1494445805272	6.80566714095221e-14\\
29.2081028694908	3.19189119579733e-14\\
29.2670352867619	8.34887714518118e-14\\
29.325967704033	1.29340982368831e-14\\
29.3849001213041	6.42264019745653e-14\\
29.4438325385752	3.00870439673417e-14\\
29.5030416355931	7.88258347483861e-14\\
29.562250732611	1.2267964422108e-14\\
29.621459829629	6.0507154842071e-14\\
29.6806689266469	2.8421709430404e-14\\
29.7401573016333	7.4273920347423e-14\\
29.7996456766196	1.16018306073329e-14\\
29.859134051606	5.71209746169643e-14\\
29.9186224265924	2.68673971959288e-14\\
29.9783927156743	6.99995617026161e-14\\
30.0381630047562	1.08801856413265e-14\\
30.0979332938381	5.38458166943201e-14\\
30.15770358292	2.53130849614536e-14\\
30.2177584582065	6.60027588139656e-14\\
30.2778133334929	1.03805852802452e-14\\
30.3378682087793	5.06816810741384e-14\\
30.3979230840658	2.38142838782096e-14\\
30.4582652560913	6.21169782277775e-14\\
30.5186074281169	9.76996261670138e-15\\
30.5789496001424	4.7795101210113e-14\\
30.639291772168	2.24265050974282e-14\\
30.6999239910786	5.85087533977457e-14\\
30.7605562099893	9.15933995315754e-15\\
30.8211884289	4.49640324973188e-14\\
30.8818206478106	2.11497486191092e-14\\
30.9427457014925	5.50670620214078e-14\\
31.0036707551745	8.65973959207622e-15\\
31.0645958088564	4.23550083894497e-14\\
31.1255208625383	1.99285032920216e-14\\
31.186741581882	5.17919040987636e-14\\
31.2479623012256	8.1601392309949e-15\\
31.3091830205692	3.98570065840431e-14\\
31.3704037399129	1.87627691161651e-14\\
31.4319229948219	4.86832796298131e-14\\
31.493442249731	7.66053886991358e-15\\
31.55496150464	3.7470027081099e-14\\
31.6164807595491	1.765254609154e-14\\
31.6783014623118	4.57411886145564e-14\\
31.7401221650745	7.21644966006352e-15\\
31.8019428678372	3.51940698806175e-14\\
31.8637635705999	1.66533453693773e-14\\
31.9258886754761	4.29101199017623e-14\\
31.9880137803524	6.77236045021345e-15\\
32.0501388852286	3.30846461338297e-14\\
32.1122639901049	1.56541446472147e-14\\
32.1746964962864	4.03566069451244e-14\\
32.2371290024679	6.38378239159465e-15\\
32.2995615086494	3.10862446895044e-14\\
32.361994014831	1.47104550762833e-14\\
32.4247369657647	3.78030939884866e-14\\
32.4874799166984	5.93969318174459e-15\\
32.5502228676321	2.92543766988729e-14\\
32.6129658185659	1.38777878078145e-14\\
32.6760223019509	3.54161144855425e-14\\
32.739078785336	5.55111512312578e-15\\
32.8021352687211	2.74225087082414e-14\\
32.8651917521062	1.29896093881143e-14\\
32.9285649023539	3.33066907387547e-14\\
32.9919380526016	5.27355936696949e-15\\
33.0553112028493	2.57016630200724e-14\\
33.118684353097	1.21569421196455e-14\\
33.1823773519202	3.12527781431982e-14\\
33.2460703507434	4.94049245958195e-15\\
33.3097633495666	2.40918396343659e-14\\
33.3734563483898	1.14352971536391e-14\\
33.4374724249187	2.91988655476416e-14\\
33.5014885014475	4.6074255521944e-15\\
33.5655045779763	2.25930385511219e-14\\
33.6295206545052	1.0769163338864e-14\\
33.6938630873744	2.73669975570101e-14\\
33.7582055202435	4.27435864480685e-15\\
33.8225479531127	2.12052597703405e-14\\
33.8868903859819	1.00475183728577e-14\\
33.951562503147	2.55906407176099e-14\\
34.016234620312	4.05231403988182e-15\\
34.0809067374771	1.97619698383278e-14\\
34.1455788546421	9.38138455808257e-15\\
34.2105840369751	2.39808173319034e-14\\
34.2755892193081	3.83026943495679e-15\\
34.340594401641	1.84852133600089e-14\\
34.405599583974	8.77076189453874e-15\\
34.4709412619942	2.24820162486594e-14\\
34.5362829400144	3.5527136788005e-15\\
34.6016246180346	1.72639680329212e-14\\
34.6669662960548	8.21565038222616e-15\\
34.7326479550893	2.09832151654155e-14\\
34.7983296141237	3.38618022510673e-15\\
34.8640112731582	1.6153745008296e-14\\
34.9296929321926	7.71605002114484e-15\\
34.9957181104378	1.9595436384634e-14\\
35.061743288683	3.1641356201817e-15\\
35.1277684669281	1.50990331349021e-14\\
35.1937936451733	7.16093850883226e-15\\
35.2601659376236	1.83186799063151e-14\\
35.3265382300739	2.94209101525666e-15\\
35.3929105225242	1.40443212615082e-14\\
35.4592828149746	6.7168492989822e-15\\
35.526005872278	1.71529457304587e-14\\
35.5927289295814	2.77555756156289e-15\\
35.6594519868847	1.31561428418081e-14\\
35.7261750441881	6.21724893790088e-15\\
35.7932525744811	1.5931700403371e-14\\
35.8603301047741	2.55351295663786e-15\\
35.927407635067	1.23234755733392e-14\\
35.99448516536	5.82867087928207e-15\\
36.0619209371048	1.48769885299771e-14\\
36.1293567088496	2.38697950294409e-15\\
36.1967924805944	1.14908083048704e-14\\
36.2642282523392	5.44009282066327e-15\\
36.3320260925956	1.38222766565832e-14\\
36.3998239328521	2.16493489801906e-15\\
36.4676217731086	1.07136521876328e-14\\
36.5354196133651	5.10702591327572e-15\\
36.6035834111162	1.28785870856518e-14\\
36.6717472088673	2.1094237467878e-15\\
36.7399110066184	9.99200722162641e-15\\
36.8080748043695	4.77395900588817e-15\\
36.8766085113362	1.20459198171829e-14\\
36.9451422183029	1.94289029309402e-15\\
37.0136759252696	9.27036225562006e-15\\
37.0822096322362	4.44089209850063e-15\\
37.1511172647722	1.11577413974828e-14\\
37.2200248973081	1.77635683940025e-15\\
37.2889325298441	8.65973959207622e-15\\
37.35784016238	4.16333634234434e-15\\
37.4271258025917	1.0325074129014e-14\\
37.4964114428034	1.60982338570648e-15\\
37.5656970830151	8.10462807976364e-15\\
37.6349827232268	3.88578058618805e-15\\
37.7046505191923	9.6034291630076e-15\\
37.7743183151579	1.49880108324396e-15\\
37.8439861111234	7.54951656745106e-15\\
37.913653907089	3.60822483003176e-15\\
37.9837080759575	8.93729534823251e-15\\
38.0537622448261	1.38777878078145e-15\\
38.1238164136947	6.99440505513849e-15\\
38.1938705825633	3.38618022510673e-15\\
38.2643154110918	8.32667268468867e-15\\
38.3347602396202	1.27675647831893e-15\\
38.4052050681487	6.55031584528842e-15\\
38.4756498966772	3.1641356201817e-15\\
38.5464897434044	7.71605002114484e-15\\
38.6173295901316	1.22124532708767e-15\\
38.6881694368588	6.0507154842071e-15\\
38.759009283586	2.88657986402541e-15\\
38.8302485799245	7.16093850883226e-15\\
38.9014878762629	1.16573417585641e-15\\
38.9727271726013	5.60662627435704e-15\\
39.0439664689397	2.72004641033163e-15\\
39.1156097204774	6.60582699651968e-15\\
39.1872529720151	9.99200722162641e-16\\
39.2588962235527	5.21804821573824e-15\\
39.3305394750904	2.55351295663786e-15\\
39.4025912659757	6.10622663543836e-15\\
39.474643056861	9.99200722162641e-16\\
39.5466948477463	4.82947015711943e-15\\
39.6187466386316	2.33146835171283e-15\\
39.6912116291116	5.71764857681956e-15\\
39.7636766195915	8.88178419700125e-16\\
39.8361416100714	4.44089209850063e-15\\
39.9086066005514	2.1094237467878e-15\\
39.9814895308678	5.27355936696949e-15\\
40.0543724611842	8.32667268468867e-16\\
40.1272553915006	4.16333634234434e-15\\
40.200138321817	1.99840144432528e-15\\
40.2734440149134	4.88498130835069e-15\\
40.3467497080097	7.7715611723761e-16\\
40.420055401106	3.83026943495679e-15\\
40.4933610942023	1.88737914186277e-15\\
40.567094455962	4.55191440096314e-15\\
40.6408278177218	7.21644966006352e-16\\
40.7145611794816	3.5527136788005e-15\\
40.7882945412414	1.72084568816899e-15\\
40.8624605638291	4.16333634234434e-15\\
40.9366265864169	6.66133814775094e-16\\
41.0107926090047	3.33066907387547e-15\\
41.0849586315925	1.60982338570648e-15\\
41.1595623935287	3.83026943495679e-15\\
41.2341661554649	6.10622663543836e-16\\
41.3087699174011	3.05311331771918e-15\\
41.3833736793372	1.49880108324396e-15\\
41.4584203505013	3.49720252756924e-15\\
41.5334670216654	4.9960036108132e-16\\
41.6085136928295	2.88657986402541e-15\\
41.6835603639935	1.4432899320127e-15\\
41.7590552049343	3.21964677141295e-15\\
41.8345500458751	4.44089209850063e-16\\
41.910044886816	2.66453525910038e-15\\
41.9855397277568	1.33226762955019e-15\\
42.0614880926214	2.94209101525666e-15\\
42.137436457486	3.88578058618805e-16\\
42.2133848223507	2.4980018054066e-15\\
42.2893331872153	1.27675647831893e-15\\
42.3657405267235	2.66453525910038e-15\\
42.4421478662318	3.33066907387547e-16\\
42.51855520574	2.33146835171283e-15\\
42.5949625452482	1.16573417585641e-15\\
42.6718344094257	2.44249065417534e-15\\
42.7487062736032	3.33066907387547e-16\\
42.8255781377807	2.1094237467878e-15\\
42.9024500019582	1.11022302462516e-15\\
42.9797920408555	2.27595720048157e-15\\
43.0571340797527	2.77555756156289e-16\\
43.1344761186499	1.94289029309402e-15\\
43.2118181575471	9.99200722162641e-16\\
43.2896361244367	2.1094237467878e-15\\
43.3674540913263	2.77555756156289e-16\\
43.4452720582158	1.77635683940025e-15\\
43.5230900251054	8.88178419700125e-16\\
43.6013897790167	1.94289029309402e-15\\
43.679689532928	2.22044604925031e-16\\
43.7579892868393	1.66533453693773e-15\\
43.8362890407505	8.32667268468867e-16\\
43.9150765502395	1.77635683940025e-15\\
43.9938640597285	2.22044604925031e-16\\
44.0726515692175	1.49880108324396e-15\\
44.1514390787065	7.7715611723761e-16\\
44.2307204224475	1.60982338570648e-15\\
44.3100017661886	2.22044604925031e-16\\
44.3892831099296	1.38777878078145e-15\\
44.4685644536707	7.21644966006352e-16\\
44.5483458244143	1.4432899320127e-15\\
44.6281271951579	1.66533453693773e-16\\
44.7079085659016	1.27675647831893e-15\\
44.7876899366452	7.21644966006352e-16\\
44.867977644264	1.27675647831893e-15\\
44.9482653518828	1.11022302462516e-16\\
45.0285530595016	1.22124532708767e-15\\
45.1088407671204	6.66133814775094e-16\\
45.1896412407794	1.16573417585641e-15\\
45.2704417144383	5.55111512312578e-17\\
45.3512421880972	1.16573417585641e-15\\
45.4320426617561	6.66133814775094e-16\\
45.5133624533588	1.0547118733939e-15\\
45.5946822449614	5.55111512312578e-17\\
45.676002036564	1.11022302462516e-15\\
45.7573218281666	6.10622663543836e-16\\
45.8391676159519	9.43689570931383e-16\\
45.9210134037371	0\\
46.0028591915224	9.99200722162641e-16\\
46.0847049793077	5.55111512312578e-16\\
46.1670835705048	9.43689570931383e-16\\
46.2494621617019	5.55111512312578e-17\\
46.3318407528991	9.43689570931383e-16\\
46.4142193440962	4.9960036108132e-16\\
46.4971376778898	8.32667268468867e-16\\
46.5800560116834	5.55111512312578e-17\\
46.662974345477	7.7715611723761e-16\\
46.7458926792706	4.44089209850063e-16\\
46.8293578314074	7.7715611723761e-16\\
46.9128229835443	1.11022302462516e-16\\
46.9962881356811	7.21644966006352e-16\\
47.079753287818	3.88578058618805e-16\\
47.1637724728227	7.21644966006352e-16\\
47.2477916578275	5.55111512312578e-17\\
47.3318108428323	6.66133814775094e-16\\
47.4158300278371	3.33066907387547e-16\\
47.5004106040352	6.10622663543836e-16\\
47.5849911802333	5.55111512312578e-17\\
47.6695717564315	6.10622663543836e-16\\
47.7541523326296	3.88578058618805e-16\\
47.8393018049434	5.55111512312578e-16\\
47.9244512772573	5.55111512312578e-17\\
48.0096007495711	5.55111512312578e-16\\
48.094750221885	3.33066907387547e-16\\
48.1804762463428	4.9960036108132e-16\\
48.2662022708007	0\\
48.3519282952586	5.55111512312578e-16\\
48.4376543197165	3.33066907387547e-16\\
48.5239647071256	4.44089209850063e-16\\
48.6102750945348	0\\
48.6965854819439	5.55111512312578e-16\\
48.7828958693531	2.77555756156289e-16\\
48.8697985896823	4.44089209850063e-16\\
48.9567013100115	0\\
49.0436040303407	4.44089209850063e-16\\
49.1305067506699	2.77555756156289e-16\\
49.2180099370951	3.33066907387547e-16\\
49.3055131235204	0\\
49.3930163099456	4.44089209850063e-16\\
49.4805194963708	2.77555756156289e-16\\
49.5686314502397	2.77555756156289e-16\\
49.6567434041087	5.55111512312578e-17\\
49.7448553579776	3.88578058618805e-16\\
49.8329673118465	2.22044604925031e-16\\
49.9216965075321	3.33066907387547e-16\\
50.0104257032176	0\\
50.0991548989032	3.33066907387547e-16\\
50.1878840945888	1.66533453693773e-16\\
50.2772391829867	2.77555756156289e-16\\
50.3665942713846	0\\
50.4559493597825	2.77555756156289e-16\\
50.5453044481804	1.66533453693773e-16\\
50.6352942631106	2.77555756156289e-16\\
50.7252840780408	5.55111512312578e-17\\
50.8152738929709	2.22044604925031e-16\\
50.9052637079011	1.11022302462516e-16\\
50.9958972708063	2.22044604925031e-16\\
51.0865308337115	5.55111512312578e-17\\
51.1771643966166	2.22044604925031e-16\\
51.2677979595218	1.11022302462516e-16\\
51.3590844844193	2.22044604925031e-16\\
51.4503710093168	5.55111512312578e-17\\
51.5416575342142	2.22044604925031e-16\\
51.6329440591117	1.11022302462516e-16\\
51.7248929579717	2.22044604925031e-16\\
51.8168418568316	0\\
51.9087907556916	2.22044604925031e-16\\
52.0007396545516	1.11022302462516e-16\\
52.0933605443554	1.66533453693773e-16\\
52.1859814341592	0\\
52.278602323963	2.22044604925031e-16\\
52.3712232137668	1.11022302462516e-16\\
52.4645259207004	1.11022302462516e-16\\
52.557828627634	5.55111512312578e-17\\
52.6511313345676	2.22044604925031e-16\\
52.7444340415012	1.66533453693773e-16\\
52.8384286082082	5.55111512312578e-17\\
52.9324231749152	5.55111512312578e-17\\
53.0264177416223	2.22044604925031e-16\\
53.1204123083293	1.66533453693773e-16\\
53.2151089998621	5.55111512312578e-17\\
53.3098056913948	5.55111512312578e-17\\
53.4045023829275	2.22044604925031e-16\\
53.4991990744603	1.66533453693773e-16\\
53.5946083849179	5.55111512312578e-17\\
53.6900176953756	1.11022302462516e-16\\
53.7854270058332	2.22044604925031e-16\\
53.8808363162909	1.66533453693773e-16\\
53.9769689760186	0\\
54.0731016357463	1.11022302462516e-16\\
54.169234295474	2.22044604925031e-16\\
54.2653669552017	2.22044604925031e-16\\
54.3622339372453	5.55111512312578e-17\\
54.4591009192888	1.11022302462516e-16\\
54.5559679013324	2.22044604925031e-16\\
54.652834883376	2.22044604925031e-16\\
54.7504474111939	5.55111512312578e-17\\
54.8480599390117	1.66533453693773e-16\\
54.9456724668296	2.22044604925031e-16\\
55.0432849946475	2.22044604925031e-16\\
55.1416545496904	1.11022302462516e-16\\
55.2400241047332	2.22044604925031e-16\\
55.338393659776	2.77555756156289e-16\\
55.4367632148189	2.22044604925031e-16\\
55.5359015449639	1.11022302462516e-16\\
55.635039875109	1.66533453693773e-16\\
55.734178205254	2.77555756156289e-16\\
55.8333165353991	2.77555756156289e-16\\
55.9332356621868	1.66533453693773e-16\\
56.0331547889744	2.22044604925031e-16\\
56.1330739157621	3.33066907387547e-16\\
56.2329930425498	2.77555756156289e-16\\
56.3337052708769	2.22044604925031e-16\\
56.434417499204	2.77555756156289e-16\\
56.5351297275311	3.33066907387547e-16\\
56.6358419558582	3.33066907387547e-16\\
56.7373598818851	2.77555756156289e-16\\
56.8388778079119	3.33066907387547e-16\\
56.9403957339388	3.88578058618805e-16\\
57.0419136599657	3.33066907387547e-16\\
57.1442501810399	2.77555756156289e-16\\
57.246586702114	3.33066907387547e-16\\
57.3489232231882	3.88578058618805e-16\\
57.4512597442624	3.33066907387547e-16\\
57.5544280683576	3.33066907387547e-16\\
57.6575963924529	3.33066907387547e-16\\
57.7607647165482	3.88578058618805e-16\\
57.8639330406435	3.88578058618805e-16\\
57.9679466963614	3.33066907387547e-16\\
58.0719603520793	3.88578058618805e-16\\
58.1759740077972	4.44089209850063e-16\\
58.2799876635152	3.88578058618805e-16\\
58.3848605103868	3.88578058618805e-16\\
58.4897333572585	3.88578058618805e-16\\
58.5946062041302	3.88578058618805e-16\\
58.6994790510019	3.88578058618805e-16\\
58.8052252907196	3.88578058618805e-16\\
58.9109715304374	3.88578058618805e-16\\
59.0167177701551	4.44089209850063e-16\\
59.1224640098728	3.88578058618805e-16\\
59.2290981971279	3.88578058618805e-16\\
59.3357323843829	3.88578058618805e-16\\
59.442366571638	4.44089209850063e-16\\
59.549000758893	4.44089209850063e-16\\
59.6617505691698	3.33066907387547e-16\\
59.7745003794465	3.33066907387547e-16\\
59.8872501897232	3.33066907387547e-16\\
60	3.88578058618805e-16\\
};
\addlegendentry{Suscettibili}

\addplot [color=mycolor2, line width=2.0pt]
  table[row sep=crcr]{%
0	0\\
0.00199053585276749	3.31129901855854e-16\\
0.00398107170553497	2.93660817098909e-16\\
0.00597160755830246	1.2587870978594e-15\\
0.00796214341106994	1.19414959013795e-15\\
0.0143083073603051	8.44438739446745e-14\\
0.0206544713095403	9.72925760138812e-15\\
0.0270006352587754	6.60457135487867e-14\\
0.0333467992080106	2.91462275321674e-14\\
0.0396935645593046	8.34270667956702e-14\\
0.0460403299105986	9.47470726007926e-15\\
0.0523870952618926	6.54886843538816e-14\\
0.0587338606131866	2.89331547002197e-14\\
0.0650961133419323	8.33694550027297e-14\\
0.0714583660706779	9.47161728388768e-15\\
0.0778206187994235	6.5439569995468e-14\\
0.0841828715281692	2.89113351314985e-14\\
0.0905606411731996	8.33080891597671e-14\\
0.0969384108182301	9.46757863079517e-15\\
0.103316180463261	6.5388341442818e-14\\
0.109693950108291	2.88887837263108e-14\\
0.116087318034554	8.32461270056095e-14\\
0.122480685960817	9.46335024232248e-15\\
0.12887405388708	6.53370586800595e-14\\
0.135267421813343	2.88660154806886e-14\\
0.141676469438281	8.31837853806916e-14\\
0.14808551706322	9.45901343363253e-15\\
0.154494564688158	6.52850169757802e-14\\
0.160903612313097	2.88432472350664e-14\\
0.167328422015208	8.31212269153392e-14\\
0.173753231717319	9.45489346537709e-15\\
0.18017804141943	6.52331921119353e-14\\
0.186602851121541	2.88204789894442e-14\\
0.193043505744106	8.3057909508466e-14\\
0.199484160366671	9.45033981625265e-15\\
0.205924814989236	6.51809335672215e-14\\
0.212365469611801	2.87970602225185e-14\\
0.218822053394839	8.29948089420274e-14\\
0.225278637177877	9.44600300756271e-15\\
0.231735220960914	6.51275908203353e-14\\
0.238191804743952	2.87742919768963e-14\\
0.244664401671674	8.29310578542852e-14\\
0.251136998599397	9.44253356061076e-15\\
0.25760959552712	6.5074248073449e-14\\
0.264082192454842	2.87508732099706e-14\\
0.270570887963067	8.28664394048051e-14\\
0.277059583471292	9.43776307105182e-15\\
0.283548278979517	6.50217726883007e-14\\
0.290036974487742	2.87274544430449e-14\\
0.296541854059766	8.2801387274456e-14\\
0.303046733631789	9.43299258149288e-15\\
0.309551613203813	6.49679962605454e-14\\
0.316056492775836	2.87036019952502e-14\\
0.322577643612098	8.27359014632378e-14\\
0.32909879444836	9.42735473019596e-15\\
0.335619945284621	6.49133524710521e-14\\
0.342141096120883	2.86810505900625e-14\\
0.348678605036902	8.26691146094127e-14\\
0.35521611395292	9.42301792150602e-15\\
0.361753622868939	6.48595760432968e-14\\
0.368291131784958	2.86576318231369e-14\\
0.374845086704082	8.26031951173256e-14\\
0.381399041623206	9.41781375107809e-15\\
0.387952996542331	6.48057996155416e-14\\
0.394506951461455	2.86333456944732e-14\\
0.401077441531393	8.25364082635005e-14\\
0.407647931601331	9.41347694238814e-15\\
0.414218421671268	6.47502884643103e-14\\
0.420788911741206	2.86081922040715e-14\\
0.427376026319146	8.24704887714134e-14\\
0.433963140897085	9.4091401336982e-15\\
0.440550255475025	6.4694777313079e-14\\
0.447137370052964	2.85830387136699e-14\\
0.453741199367022	8.24045692793263e-14\\
0.46034502868108	9.40567068674625e-15\\
0.466948857995137	6.46375314383718e-14\\
0.473552687309195	2.85587525850062e-14\\
0.480173322720616	8.23351803402872e-14\\
0.486793958132037	9.40046651631832e-15\\
0.493414593543458	6.45802855636646e-14\\
0.500035228954879	2.85327317328665e-14\\
0.506672761762064	8.22675261247241e-14\\
0.513310294569248	9.39699706936636e-15\\
0.519947827376433	6.45230396889573e-14\\
0.526585360183617	2.85084456042028e-14\\
0.533239883483442	8.2198137185685e-14\\
0.539894406783268	9.39005817546246e-15\\
0.546548930083094	6.44657938142501e-14\\
0.553203453382919	2.84841594755392e-14\\
0.559875060934079	8.212701352317e-14\\
0.566546668485238	9.38485400503453e-15\\
0.573218276036397	6.44120173864948e-14\\
0.579889883587557	2.84598733468755e-14\\
0.586578669168294	8.20576245841309e-14\\
0.593267454749031	9.3796498346066e-15\\
0.599956240329768	6.43530367883116e-14\\
0.606645025910505	2.84355872182118e-14\\
0.613351084619436	8.19847661981399e-14\\
0.620057143328367	9.37271094070269e-15\\
0.626763202037298	6.42975256370804e-14\\
0.633469260746229	2.84113010895481e-14\\
0.640192689007476	8.19153772591008e-14\\
0.646916117268722	9.36924149375074e-15\\
0.653639545529968	6.42368103154212e-14\\
0.660362973791214	2.83835455139325e-14\\
0.667103867904104	8.18442535965858e-14\\
0.673844762016994	9.3640373233228e-15\\
0.680585656129883	6.4177829717238e-14\\
0.687326550242773	2.83592593852688e-14\\
0.694085008124096	8.17679257636428e-14\\
0.700843466005419	9.3570984294189e-15\\
0.707601923886742	6.41223185660067e-14\\
0.714360381768066	2.83315038096532e-14\\
0.721136501534817	8.16985368246037e-14\\
0.727912621301569	9.35015953551499e-15\\
0.73468874106832	6.40598685208715e-14\\
0.741464860835071	2.83072176809895e-14\\
0.748258742408364	8.16256784386127e-14\\
0.755052623981657	9.35015953551499e-15\\
0.761846505554949	6.39939490287844e-14\\
0.768640387128242	2.82794621053739e-14\\
0.775452130474897	8.15528200526217e-14\\
0.782263873821551	9.34322064161108e-15\\
0.789075617168206	6.39384378775532e-14\\
0.79588736051486	2.82517065297583e-14\\
0.80271706685068	8.14764922196787e-14\\
0.809546773186499	9.33975119465913e-15\\
0.816376479522319	6.38794572793699e-14\\
0.823206185858138	2.82239509541427e-14\\
0.830053957332632	8.14036338336876e-14\\
0.836901728807126	9.33628174770718e-15\\
0.84374950028162	6.38135377872828e-14\\
0.850597271756113	2.8196195378527e-14\\
0.857463211448599	8.13273060007447e-14\\
0.864329151141086	9.32934285380327e-15\\
0.871195090833572	6.37510877421477e-14\\
0.878061030526058	2.81719092498633e-14\\
0.884945241937476	8.12544476147536e-14\\
0.891829453348894	9.32240395989936e-15\\
0.898713664760312	6.36921071439644e-14\\
0.90559787617173	2.81476231211997e-14\\
0.912500464439821	8.11746503348587e-14\\
0.919403052707912	9.3119956190435e-15\\
0.926305640976003	6.36296570988293e-14\\
0.933208229244093	2.8123336992536e-14\\
0.940129300503941	8.10913836080118e-14\\
0.947050371763788	9.30505672513959e-15\\
0.953971443023636	6.35706765006461e-14\\
0.960892514283483	2.80955814169204e-14\\
0.967832174888143	8.10115863281169e-14\\
0.974771835492802	9.29117893733178e-15\\
0.981711496097461	6.35116959024629e-14\\
0.988651156702121	2.80678258413047e-14\\
0.995609514344775	8.092831960127e-14\\
1.00256787198743	9.29117893733178e-15\\
1.00952622963008	6.34492458573277e-14\\
1.01648458727274	2.8047009159593e-14\\
1.0234617510246	8.0851991768327e-14\\
1.03043891477646	9.27730114952396e-15\\
1.03741607852832	6.33867958121925e-14\\
1.04439324228018	2.80192535839774e-14\\
1.05138932194451	8.07687250414801e-14\\
1.05838540160883	9.27036225562006e-15\\
1.06538148127316	6.33312846609613e-14\\
1.07237756093749	2.79914980083618e-14\\
1.07939266704796	8.06923972085372e-14\\
1.08640777315843	9.26342336171615e-15\\
1.0934228792689	6.32549568280183e-14\\
1.10043798537936	2.79568035388422e-14\\
1.10747222983544	8.06160693755942e-14\\
1.11450647429151	9.27036225562006e-15\\
1.12154071874758	6.31855678889792e-14\\
1.12857496320366	2.79290479632266e-14\\
1.13562845894337	8.05328026487473e-14\\
1.14268195468308	9.25648446781224e-15\\
1.14973545042278	6.31231178438441e-14\\
1.15678894616249	2.78943534937071e-14\\
1.16386180727142	8.04495359219004e-14\\
1.17093466838034	9.25648446781224e-15\\
1.17800752948927	6.30467900109011e-14\\
1.18508039059819	2.78665979180914e-14\\
1.19217273186328	8.03732080889574e-14\\
1.19926507312837	9.24954557390834e-15\\
1.20635741439346	6.29843399657659e-14\\
1.21344975565855	2.78388423424758e-14\\
1.22056169301535	8.02899413621105e-14\\
1.22767363037216	9.24260668000443e-15\\
1.23478556772896	6.29149510267268e-14\\
1.24189750508577	2.78110867668602e-14\\
1.2490291561515	8.01997357413597e-14\\
1.25616080721723	9.23566778610052e-15\\
1.26329245828296	6.28455620876878e-14\\
1.27042410934869	2.77833311912445e-14\\
1.27757559222201	8.0109530120609e-14\\
1.28472707509533	9.22872889219661e-15\\
1.29187855796865	6.27900509364565e-14\\
1.29903004084196	2.77555756156289e-14\\
1.30620147471517	8.00262633937621e-14\\
1.31337290858837	9.2148511043888e-15\\
1.32054434246157	6.27137231035135e-14\\
1.32771577633477	2.77278200400133e-14\\
1.33490728210935	7.99360577730113e-14\\
1.34209878788392	9.20791221048489e-15\\
1.3492902936585	6.26512730583784e-14\\
1.35648179943308	2.77000644643977e-14\\
1.36369349872651	7.98527910461644e-14\\
1.37090519801994	9.20097331658098e-15\\
1.37811689731337	6.25749452254354e-14\\
1.3853285966068	2.76653699948781e-14\\
1.39256061231053	7.97695243193175e-14\\
1.39979262801426	9.20097331658098e-15\\
1.40702464371799	6.25055562863963e-14\\
1.41425665942172	2.76376144192625e-14\\
1.42150911582444	7.96793186985667e-14\\
1.42876157222716	9.18709552877317e-15\\
1.43601402862988	6.24292284534533e-14\\
1.44326648503259	2.7602919949743e-14\\
1.45053950730217	7.95891130778159e-14\\
1.45781252957175	9.18015663486926e-15\\
1.46508555184133	6.23667784083182e-14\\
1.47235857411091	2.75751643741273e-14\\
1.47965228878223	7.94919685631612e-14\\
1.48694600345356	9.15933995315754e-15\\
1.49423971812488	6.22973894692791e-14\\
1.5015334327962	2.75612865863195e-14\\
1.50884796768823	7.93948240485065e-14\\
1.51616250258027	9.14546216534973e-15\\
1.5234770374723	6.22418783180478e-14\\
1.53079157236433	2.75335310107039e-14\\
1.53812705668821	7.9283801746044e-14\\
1.54546254101209	9.13158437754191e-15\\
1.55279802533598	6.2158611591201e-14\\
1.56013350965986	2.74918976472804e-14\\
1.56749007329995	7.92144128070049e-14\\
1.57484663694003	9.13158437754191e-15\\
1.58220320058012	6.20892226521619e-14\\
1.5895597642202	2.74641420716648e-14\\
1.59693753927215	7.91033905045424e-14\\
1.60431531432409	9.1177065897341e-15\\
1.61169308937604	6.20198337131228e-14\\
1.61907086442798	2.74363864960492e-14\\
1.6264699837717	7.90201237776955e-14\\
1.63386910311542	9.1177065897341e-15\\
1.64126822245913	6.19365669862759e-14\\
1.64866734180285	2.74086309204336e-14\\
1.65608793943564	7.89229792630408e-14\\
1.66350853706843	9.10382880192628e-15\\
1.67092913470122	6.18671780472368e-14\\
1.67834973233401	2.73669975570101e-14\\
1.68579194412116	7.88397125361939e-14\\
1.69323415590832	9.10382880192628e-15\\
1.70067636769548	6.178391132039e-14\\
1.70811857948264	2.73392419813945e-14\\
1.71558254216844	7.87425680215392e-14\\
1.72304650485425	9.10382880192628e-15\\
1.73051046754005	6.17006445935431e-14\\
1.73797443022586	2.72837308301632e-14\\
1.74546028230335	7.86593012946923e-14\\
1.75294613438084	9.10382880192628e-15\\
1.76043198645833	6.16035000788884e-14\\
1.76791783853582	2.72559752545476e-14\\
1.77542572012108	7.85621567800376e-14\\
1.78293360170633	9.08995101411847e-15\\
1.79044148329159	6.15341111398493e-14\\
1.79794936487685	2.72143418911241e-14\\
1.80547941694215	7.84511344775751e-14\\
1.81300946900745	9.07607322631065e-15\\
1.82053952107275	6.14647222008102e-14\\
1.82806957313806	2.72004641033163e-14\\
1.8356219382731	7.83401121751126e-14\\
1.84317430340815	9.04831765069503e-15\\
1.85072666854319	6.13953332617712e-14\\
1.85827903367824	2.71727085277007e-14\\
1.86585385599785	7.82429676604579e-14\\
1.87342867831746	9.04831765069503e-15\\
1.88100350063707	6.13120665349243e-14\\
1.88857832295669	2.71310751642773e-14\\
1.89617574793629	7.81458231458032e-14\\
1.90377317291589	9.04831765069503e-15\\
1.9113705978955	6.12287998080774e-14\\
1.9189680228751	2.70894418008538e-14\\
1.92658819822989	7.80348008433407e-14\\
1.93420837358469	9.0205620750794e-15\\
1.94182854893948	6.11594108690383e-14\\
1.94944872429427	2.70616862252382e-14\\
1.95709179801755	7.7937656328686e-14\\
1.96473487174083	9.0205620750794e-15\\
1.97237794546411	6.10761441421914e-14\\
1.98002101918739	2.70339306496226e-14\\
1.98768714199827	7.78405118140313e-14\\
1.99535326480915	9.00668428727158e-15\\
2.00301938762003	6.09928774153445e-14\\
2.01068551043091	2.69922972861991e-14\\
2.0183748340895	7.7715611723761e-14\\
2.02606415774808	9.00668428727158e-15\\
2.03375348140667	6.09096106884977e-14\\
2.04144280506525	2.69645417105835e-14\\
2.0491554826466	7.76184672091063e-14\\
2.05686816022795	8.99280649946377e-15\\
2.06458083780929	6.08124661738429e-14\\
2.07229351539064	2.692290834716e-14\\
2.08002970237887	7.75074449066437e-14\\
2.08776588936711	8.97892871165595e-15\\
2.09550207635534	6.07430772348039e-14\\
2.10323826334358	2.68951527715444e-14\\
2.11099811595816	7.7410300391989e-14\\
2.11875796857274	8.96505092384814e-15\\
2.12651782118732	6.06459327201492e-14\\
2.1342776738019	2.68396416203132e-14\\
2.14206135075448	7.73131558773343e-14\\
2.14984502770706	8.96505092384814e-15\\
2.15762870465964	6.05626659933023e-14\\
2.16541238161222	2.68118860446975e-14\\
2.17322004315878	7.7188255787064e-14\\
2.18102770470534	8.95117313604032e-15\\
2.1888353662519	6.04793992664554e-14\\
2.19664302779846	2.67841304690819e-14\\
2.2044748353169	7.70772334846015e-14\\
2.21230664283533	8.93729534823251e-15\\
2.22013845035377	6.04100103274163e-14\\
2.2279702578722	2.67424971056585e-14\\
2.23582637510447	7.6966211182139e-14\\
2.24368249233675	8.93729534823251e-15\\
2.25153860956902	6.02989880249538e-14\\
2.25939472680129	2.6700863742235e-14\\
2.26727531955162	7.68551888796765e-14\\
2.27515591230195	8.93729534823251e-15\\
2.28303650505228	6.02018435102991e-14\\
2.29091709780261	2.66453525910038e-14\\
2.29882233260671	7.67580443650218e-14\\
2.30672756741082	8.9234175604247e-15\\
2.31463280221492	6.01046989956444e-14\\
2.32253803701902	2.66175970153881e-14\\
2.33046808291417	7.66331442747514e-14\\
2.33839812880931	8.9234175604247e-15\\
2.34632817470446	6.00214322687975e-14\\
2.3542582205996	2.65759636519647e-14\\
2.36221324911394	7.65221219722889e-14\\
2.37016827762829	8.90953977261688e-15\\
2.37812330614264	5.99520433297585e-14\\
2.38607833465698	2.65343302885412e-14\\
2.39405851771369	7.63833440942108e-14\\
2.40203870077039	8.88178419700125e-15\\
2.4100188838271	5.98410210272959e-14\\
2.4179990668838	2.65065747129256e-14\\
2.42600457974877	7.62445662161326e-14\\
2.43401009261373	8.85402862138562e-15\\
2.4420156054787	5.97577543004491e-14\\
2.45002111834367	2.647881913731e-14\\
2.45805213720175	7.61335439136701e-14\\
2.46608315605982	8.85402862138562e-15\\
2.4741141749179	5.97022431492178e-14\\
2.48214519377598	2.64510635616944e-14\\
2.4902018978345	7.60225216112076e-14\\
2.49825860189303	8.82627304576999e-15\\
2.50631530595155	5.95912208467553e-14\\
2.51437201001007	2.64233079860787e-14\\
2.52245457948208	7.58559881575138e-14\\
2.53053714895408	8.79851747015437e-15\\
2.53861971842609	5.95079541199084e-14\\
2.54670228789809	2.63955524104631e-14\\
2.55481090545363	7.57449658550513e-14\\
2.56291952300918	8.77076189453874e-15\\
2.57102814056472	5.94246873930615e-14\\
2.57913675812026	2.63400412592318e-14\\
2.58727160888194	7.56339435525888e-14\\
2.59540645964362	8.79851747015437e-15\\
2.6035413104053	5.9313665090599e-14\\
2.61167616116698	2.62845301080006e-14\\
2.61983743200028	7.55229212501263e-14\\
2.62799870283359	8.77076189453874e-15\\
2.63615997366689	5.92026427881365e-14\\
2.6443212445002	2.62290189567693e-14\\
2.65250912437636	7.54118989476638e-14\\
2.66069700425252	8.77076189453874e-15\\
2.66888488412869	5.9091620485674e-14\\
2.67707276400485	2.61735078055381e-14\\
2.6852874441124	7.52731210695856e-14\\
2.69350212421995	8.77076189453874e-15\\
2.7017168043275	5.89805981832114e-14\\
2.70993148443506	2.61457522299224e-14\\
2.71817315820996	7.51343431915075e-14\\
2.72641483198486	8.77076189453874e-15\\
2.73465650575977	5.88973314563646e-14\\
2.74289817953467	2.61179966543068e-14\\
2.75116704265512	7.49955653134293e-14\\
2.75943590577556	8.74300631892311e-15\\
2.767704768896	5.88140647295177e-14\\
2.77597363201644	2.60624855030755e-14\\
2.78426988288509	7.49122985865824e-14\\
2.79256613375373	8.71525074330748e-15\\
2.80086238462237	5.86752868514395e-14\\
2.80915863549101	2.60069743518443e-14\\
2.81748247408502	7.4718009557273e-14\\
2.82580631267902	8.71525074330748e-15\\
2.83413015127303	5.8564264548977e-14\\
2.84245398986704	2.59792187762287e-14\\
2.85080561880438	7.46069872548105e-14\\
2.85915724774172	8.71525074330748e-15\\
2.86750887667906	5.84809978221301e-14\\
2.8758605056164	2.5951463200613e-14\\
2.88424013037944	7.44404538011167e-14\\
2.89261975514247	8.65973959207622e-15\\
2.90099937990551	5.83977310952832e-14\\
2.90937900466855	2.59237076249974e-14\\
2.91778683316272	7.43016759230386e-14\\
2.9261946616569	8.63198401646059e-15\\
2.93460249015107	5.82867087928207e-14\\
2.94301031864525	2.58681964737661e-14\\
2.95144656047954	7.41628980449605e-14\\
2.95988280231384	8.63198401646059e-15\\
2.96831904414814	5.81756864903582e-14\\
2.97675528598243	2.58404408981505e-14\\
2.98522015378253	7.40241201668823e-14\\
2.99368502158263	8.63198401646059e-15\\
3.00214988938273	5.80924197635113e-14\\
3.01061475718283	2.58126853225349e-14\\
3.01910846625213	7.38298311375729e-14\\
3.02760217532142	8.57647286522933e-15\\
3.03609588439071	5.79813974610488e-14\\
3.04458959346	2.57571741713036e-14\\
3.05311236220025	7.37188088351104e-14\\
3.0616351309405	8.57647286522933e-15\\
3.07015789968076	5.78703751585863e-14\\
3.07868066842101	2.57016630200724e-14\\
3.0872327162858	7.35800309570322e-14\\
3.0957847641506	8.54871728961371e-15\\
3.10433681201539	5.77593528561238e-14\\
3.11288885988019	2.56739074444567e-14\\
3.12147041092284	7.34134975033385e-14\\
3.13005196196549	8.52096171399808e-15\\
3.13863351300815	5.76483305536613e-14\\
3.1472150640508	2.56183962932255e-14\\
3.15582634421903	7.32747196252603e-14\\
3.16443762438727	8.52096171399808e-15\\
3.1730489045555	5.75373082511987e-14\\
3.18166018472374	2.55906407176099e-14\\
3.19030142291735	7.31081861715666e-14\\
3.19894266111096	8.46545056276682e-15\\
3.20758389930457	5.74262859487362e-14\\
3.21622513749818	2.55351295663786e-14\\
3.22489656499851	7.29416527178728e-14\\
3.23356799249884	8.46545056276682e-15\\
3.24223941999916	5.73430192218893e-14\\
3.25091084749949	2.5507373990763e-14\\
3.25961269965928	7.28028748397946e-14\\
3.26831455181906	8.46545056276682e-15\\
3.27701640397885	5.71764857681956e-14\\
3.28571825613863	2.54241072639161e-14\\
3.29445076992401	7.26363413861009e-14\\
3.30318328370939	8.43769498715119e-15\\
3.31191579749477	5.70932190413487e-14\\
3.32064831128015	2.53963516883005e-14\\
3.32941172759282	7.24975635080227e-14\\
3.33817514390549	8.40993941153556e-15\\
3.34693856021816	5.69821967388862e-14\\
3.35570197653083	2.53685961126848e-14\\
3.36449653919211	7.23032744787133e-14\\
3.37329110185339	8.38218383591993e-15\\
3.38208566451467	5.68711744364236e-14\\
3.39088022717596	2.53130849614536e-14\\
3.39970618282859	7.21367410250195e-14\\
3.40853213848122	8.3544282603043e-15\\
3.41735809413385	5.67601521339611e-14\\
3.42618404978648	2.52575738102223e-14\\
3.43504164855909	7.19979631469414e-14\\
3.44389924733171	8.3544282603043e-15\\
3.45275684610433	5.65936186802674e-14\\
3.46161444487695	2.52020626589911e-14\\
3.47050394016004	7.18314296932476e-14\\
3.47939343544313	8.3544282603043e-15\\
3.48828293072622	5.64825963778048e-14\\
3.49717242600932	2.51187959321442e-14\\
3.50609407425931	7.17204073907851e-14\\
3.5150157225093	8.3544282603043e-15\\
3.52393737075929	5.63160629241111e-14\\
3.53285901900928	2.50355292052973e-14\\
3.54181308032811	7.15538739370913e-14\\
3.55076714164694	8.3544282603043e-15\\
3.55972120296577	5.61772850460329e-14\\
3.56867526428461	2.4980018054066e-14\\
3.57766200230417	7.13873404833976e-14\\
3.58664874032374	8.3544282603043e-15\\
3.5956354783433	5.60385071679548e-14\\
3.60462221636287	2.49245069028348e-14\\
3.61364189770245	7.12208070297038e-14\\
3.62266157904204	8.32667268468867e-15\\
3.63168126038162	5.58997292898766e-14\\
3.64070094172121	2.48689957516035e-14\\
3.64975383695641	7.10265180003944e-14\\
3.65880673219161	8.29891710907305e-15\\
3.66785962742681	5.57609514117985e-14\\
3.67691252266202	2.48134846003722e-14\\
3.6859989062531	7.08599845467006e-14\\
3.69508528984418	8.29891710907305e-15\\
3.70417167343526	5.56221735337203e-14\\
3.71325805702634	2.4757973449141e-14\\
3.72237820615531	7.06934510930068e-14\\
3.73149835528429	8.29891710907305e-15\\
3.74061850441326	5.54833956556422e-14\\
3.74973865354223	2.47024622979097e-14\\
3.75889285078234	7.05269176393131e-14\\
3.76804704802246	8.27116153345742e-15\\
3.77720124526257	5.53446177775641e-14\\
3.78635544250268	2.46191955710628e-14\\
3.79554397276389	7.03603841856193e-14\\
3.80473250302509	8.27116153345742e-15\\
3.8139210332863	5.51780843238703e-14\\
3.8231095635475	2.4535928844216e-14\\
3.83233271658159	7.01660951563099e-14\\
3.84155586961568	8.24340595784179e-15\\
3.85077902264978	5.50393064457921e-14\\
3.86000217568387	2.45081732686003e-14\\
3.86926024490616	6.99718061270005e-14\\
3.87851831412845	8.24340595784179e-15\\
3.88777638335074	5.49282841433296e-14\\
3.89703445257303	2.44249065417534e-14\\
3.90632773542166	6.97775170976911e-14\\
3.91562101827029	8.18789480661053e-15\\
3.92491430111892	5.47617506896358e-14\\
3.93420758396755	2.43693953905222e-14\\
3.94353638235509	6.95832280683817e-14\\
3.95286518074264	8.18789480661053e-15\\
3.96219397913019	5.46229728115577e-14\\
3.97152277751774	2.43416398149066e-14\\
3.9808873974673	6.93889390390723e-14\\
3.99025201741687	8.1601392309949e-15\\
3.99961663736644	5.44841949334796e-14\\
4.008981257316	2.42583730880597e-14\\
4.01838200905393	6.92224055853785e-14\\
4.02778276079185	8.1601392309949e-15\\
4.03718351252977	5.43176614797858e-14\\
4.0465842642677	2.42028619368284e-14\\
4.05602146260214	6.90281165560691e-14\\
4.06545866093658	8.13238365537927e-15\\
4.07489585927102	5.4151128026092e-14\\
4.08433305760547	2.41195952099815e-14\\
4.09380702196714	6.88060719511441e-14\\
4.10328098632881	8.10462807976364e-15\\
4.11275495069048	5.40123501480139e-14\\
4.12222891505216	2.40640840587503e-14\\
4.13173996944755	6.86117829218347e-14\\
4.14125102384295	8.07687250414801e-15\\
4.15076207823835	5.38735722699357e-14\\
4.16027313263375	2.4008572907519e-14\\
4.16982160537755	6.83897383169096e-14\\
4.17937007812135	8.04911692853238e-15\\
4.18891855086514	5.37070388162419e-14\\
4.19846702360894	2.39530617562878e-14\\
4.20805324837695	6.81954492876002e-14\\
4.21763947314496	8.04911692853238e-15\\
4.22722569791297	5.35405053625482e-14\\
4.23681192268098	2.38697950294409e-14\\
4.24643623800907	6.79734046826752e-14\\
4.25606055333716	7.99360577730113e-15\\
4.26568486866525	5.340172748447e-14\\
4.27530918399335	2.38142838782096e-14\\
4.28497193298002	6.77513600777502e-14\\
4.29463468196669	7.99360577730113e-15\\
4.30429743095337	5.32074384551606e-14\\
4.31396017994004	2.37587727269783e-14\\
4.32366171143296	6.75293154728251e-14\\
4.33336324292588	7.9658502016855e-15\\
4.3430647744188	5.30409050014669e-14\\
4.35276630591172	2.36477504245158e-14\\
4.3625069732434	6.73072708679001e-14\\
4.37224764057509	7.93809462606987e-15\\
4.38198830790678	5.28743715477731e-14\\
4.39172897523846	2.35644836976689e-14\\
4.40150913802251	6.71129818385907e-14\\
4.41128930080656	7.93809462606987e-15\\
4.42106946359061	5.26800825184637e-14\\
4.43084962637466	2.34812169708221e-14\\
4.44066964916844	6.69186928092813e-14\\
4.45048967196222	7.93809462606987e-15\\
4.460309694756	5.24857934891543e-14\\
4.47012971754978	2.33979502439752e-14\\
4.47998997038358	6.67244037799719e-14\\
4.48985022321737	7.93809462606987e-15\\
4.49971047605117	5.22915044598449e-14\\
4.50957072888496	2.33146835171283e-14\\
4.51947158785354	6.64746035994312e-14\\
4.52937244682212	7.93809462606987e-15\\
4.5392733057907	5.21249710061511e-14\\
4.54917416475928	2.3259172365897e-14\\
4.55911601157961	6.62803145701218e-14\\
4.56905785839994	7.88258347483861e-15\\
4.57899970522028	5.19584375524573e-14\\
4.58894155204061	2.31481500634345e-14\\
4.59892477418835	6.60582699651968e-14\\
4.6089079963361	7.88258347483861e-15\\
4.61889121848384	5.17363929475323e-14\\
4.62887444063159	2.3037127760972e-14\\
4.6388994324499	6.57807142090405e-14\\
4.64892442426821	7.88258347483861e-15\\
4.65894941608653	5.15698594938385e-14\\
4.66897440790484	2.29816166097407e-14\\
4.67904156948105	6.55586696041155e-14\\
4.68910873105727	7.82707232360735e-15\\
4.69917589263348	5.13478148889135e-14\\
4.7092430542097	2.29261054585095e-14\\
4.71935279189367	6.52256026967279e-14\\
4.72946252957763	7.7715611723761e-15\\
4.7395722672616	5.11812814352197e-14\\
4.74968200494557	2.28705943072782e-14\\
4.75983473174927	6.50035580918029e-14\\
4.76998745855297	7.7715611723761e-15\\
4.78014018535667	5.10147479815259e-14\\
4.79029291216037	2.27595720048157e-14\\
4.8004890483408	6.47260023356466e-14\\
4.81068518452122	7.71605002114484e-15\\
4.82088132070165	5.07927033766009e-14\\
4.83107745688208	2.27040608535845e-14\\
4.84131742876219	6.45039577307216e-14\\
4.85155740064229	7.71605002114484e-15\\
4.8617973725224	5.06261699229071e-14\\
4.87203734440251	2.25930385511219e-14\\
4.88232158523569	6.42819131257966e-14\\
4.89260582606888	7.66053886991358e-15\\
4.90289006690206	5.04041253179821e-14\\
4.91317430773525	2.24820162486594e-14\\
4.92350325885451	6.40043573696403e-14\\
4.93383220997378	7.66053886991358e-15\\
4.94416116109305	5.01820807130571e-14\\
4.95449011221231	2.24265050974282e-14\\
4.96486422170276	6.3726801613484e-14\\
4.97523833119322	7.66053886991358e-15\\
4.98561244068367	4.9960036108132e-14\\
4.99598655017412	2.23154827949656e-14\\
5.00640627395839	6.3504757008559e-14\\
5.01682599774266	7.60502771868232e-15\\
5.02724572152694	4.9737991503207e-14\\
5.03766544531121	2.22044604925031e-14\\
5.04813124666974	6.31716901011714e-14\\
5.05859704802827	7.60502771868232e-15\\
5.0690628493868	4.9515946898282e-14\\
5.07952865074533	2.20934381900406e-14\\
5.09004100143523	6.29496454962464e-14\\
5.10055335212514	7.60502771868232e-15\\
5.11106570281504	4.9293902293357e-14\\
5.12157805350494	2.19824158875781e-14\\
5.13213743298625	6.26720897400901e-14\\
5.14269681246755	7.54951656745106e-15\\
5.15325619194885	4.90163465372007e-14\\
5.16381557143015	2.18713935851156e-14\\
5.17442246764722	6.23390228327025e-14\\
5.1850293638643	7.54951656745106e-15\\
5.19563626008137	4.88498130835069e-14\\
5.20624315629844	2.17603712826531e-14\\
5.21689806557204	6.20614670765463e-14\\
5.22755297484563	7.49400541621981e-15\\
5.23820788411923	4.85722573273506e-14\\
5.24886279339282	2.16493489801906e-14\\
5.25956622085953	6.18394224716212e-14\\
5.27026964832624	7.49400541621981e-15\\
5.28097307579295	4.82947015711943e-14\\
5.29167650325966	2.1538326677728e-14\\
5.30242896340108	6.15618667154649e-14\\
5.31318142354251	7.54951656745106e-15\\
5.32393388368394	4.8017145815038e-14\\
5.33468634382537	2.14273043752655e-14\\
5.34548835981846	6.12287998080774e-14\\
5.35629037581155	7.49400541621981e-15\\
5.36709239180464	4.78506123613442e-14\\
5.37789440779773	2.1316282072803e-14\\
5.38874651272417	6.08957329006898e-14\\
5.39959861765061	7.43849426498855e-15\\
5.41045072257705	4.7573056605188e-14\\
5.42130282750349	2.12607709215717e-14\\
5.43220556367259	6.05626659933023e-14\\
5.44310829984169	7.38298311375729e-15\\
5.45401103601079	4.73510120002629e-14\\
5.46491377217989	2.11497486191092e-14\\
5.47586769235055	6.0285110237146e-14\\
5.48682161252121	7.43849426498855e-15\\
5.49777553269188	4.70179450928754e-14\\
5.50872945286254	2.09832151654155e-14\\
5.51973511980852	5.98965321785272e-14\\
5.5307407867545	7.32747196252603e-15\\
5.54174645370048	4.67959004879503e-14\\
5.55275212064646	2.09277040141842e-14\\
5.5638101076295	5.95634652711396e-14\\
5.57486809461254	7.27196081129478e-15\\
5.58592608159558	4.66293670342566e-14\\
5.59698406857862	2.08166817117217e-14\\
5.60809495979115	5.91748872125208e-14\\
5.61920585100368	7.21644966006352e-15\\
5.63031674221621	4.63518112781003e-14\\
5.64142763342874	2.07611705604904e-14\\
5.65259202401047	5.88973314563646e-14\\
5.66375641459219	7.21644966006352e-15\\
5.67492080517392	4.6074255521944e-14\\
5.68608519575564	2.05946371067967e-14\\
5.69730369280843	5.84532422465145e-14\\
5.70852218986121	7.16093850883226e-15\\
5.719740686914	4.57411886145564e-14\\
5.73095918396678	2.04836148043341e-14\\
5.74223240540217	5.81756864903582e-14\\
5.75350562683755	7.16093850883226e-15\\
5.76477884827294	4.54636328584002e-14\\
5.77605206970832	2.03725925018716e-14\\
5.78738064638032	5.77315972805081e-14\\
5.79870922305231	7.04991620636974e-15\\
5.81003779972431	4.51860771022439e-14\\
5.8213663763963	2.02615701994091e-14\\
5.83275095107969	5.73430192218893e-14\\
5.84413552576308	7.04991620636974e-15\\
5.85552010044646	4.49085213460876e-14\\
5.86690467512985	2.01505478969466e-14\\
5.87834590369998	5.70099523145018e-14\\
5.88978713227011	6.99440505513849e-15\\
5.90122836084025	4.46309655899313e-14\\
5.91266958941038	2.00395255944841e-14\\
5.92416814082324	5.6621374255883e-14\\
5.9356666922361	6.93889390390723e-15\\
5.94716524364896	4.4353409833775e-14\\
5.95866379506182	1.99285032920216e-14\\
5.97022035244649	5.61772850460329e-14\\
5.98177690983116	6.88338275267597e-15\\
5.99333346721583	4.40758540776187e-14\\
6.00489002460049	1.97619698383278e-14\\
6.01650528373054	5.57331958361829e-14\\
6.02812054286059	6.82787160144471e-15\\
6.03973580199063	4.37427871702312e-14\\
6.05135106112068	1.96509475358653e-14\\
6.06302573298457	5.53446177775641e-14\\
6.07470040484846	6.82787160144471e-15\\
6.08637507671235	4.34097202628436e-14\\
6.09804974857625	1.95399252334028e-14\\
6.10978455915361	5.49560397189452e-14\\
6.12151936973097	6.77236045021345e-15\\
6.13325418030833	4.30766533554561e-14\\
6.14498899088569	1.9373391779709e-14\\
6.15678468114828	5.45674616603264e-14\\
6.16858037141086	6.77236045021345e-15\\
6.18037606167345	4.27435864480685e-14\\
6.19217175193603	1.92068583260152e-14\\
6.20402907883544	5.41233724504764e-14\\
6.21588640573484	6.66133814775094e-15\\
6.22774373263424	4.2410519540681e-14\\
6.23960105953364	1.90958360235527e-14\\
6.25152079648947	5.36237720893951e-14\\
6.2634405334453	6.66133814775094e-15\\
6.27536027040113	4.20774526332934e-14\\
6.28728000735696	1.89848137210902e-14\\
6.2992629439367	5.32351940307763e-14\\
6.31124588051644	6.55031584528842e-15\\
6.32322881709618	4.17998968771371e-14\\
6.33521175367592	1.88182802673964e-14\\
6.34725869756642	5.27355936696949e-14\\
6.35930564145691	6.49480469405717e-15\\
6.3713525853474	4.13558076672871e-14\\
6.3833995292379	1.86517468137026e-14\\
6.39551130573695	5.22915044598449e-14\\
6.40762308223601	6.49480469405717e-15\\
6.41973485873506	4.10227407598995e-14\\
6.43184663523411	1.85407245112401e-14\\
6.44402408779825	5.17919040987636e-14\\
6.45620154036239	6.43929354282591e-15\\
6.46837899292652	4.06341627012807e-14\\
6.48055644549066	1.83741910575463e-14\\
6.49280043746748	5.12923037376822e-14\\
6.5050444294443	6.38378239159465e-15\\
6.51728842142112	4.03010957938932e-14\\
6.52953241339793	1.82076576038526e-14\\
6.5418438273366	5.07371922253697e-14\\
6.55415524127526	6.27276008913213e-15\\
6.56646665521392	3.99125177352744e-14\\
6.57877806915258	1.80411241501588e-14\\
6.59115780864942	5.02375918642883e-14\\
6.60353754814627	6.21724893790088e-15\\
6.61591728764311	3.95239396766556e-14\\
6.62829702713995	1.79301018476963e-14\\
6.64074601624377	4.96824803519758e-14\\
6.65319500534758	6.21724893790088e-15\\
6.66564399445139	3.90798504668055e-14\\
6.67809298355521	1.77080572427712e-14\\
6.69061216875419	4.91828799908944e-14\\
6.70313135395318	6.10622663543836e-15\\
6.71565053915216	3.86912724081867e-14\\
6.72816972435114	1.74860126378462e-14\\
6.74076007448507	4.86277684785819e-14\\
6.75335042461899	6.0507154842071e-15\\
6.76594077475292	3.82471831983366e-14\\
6.77853112488684	1.73194791841524e-14\\
6.79119363328311	4.81281681175005e-14\\
6.80385614167938	6.0507154842071e-15\\
6.81651865007565	3.78586051397178e-14\\
6.82918115847192	1.71529457304587e-14\\
6.84191684190172	4.74620343027254e-14\\
6.85465252533151	5.93969318174459e-15\\
6.86738820876131	3.74145159298678e-14\\
6.88012389219111	1.69864122767649e-14\\
6.8929337927896	4.69069227904129e-14\\
6.90574369338808	5.88418203051333e-15\\
6.91855359398656	3.69704267200177e-14\\
6.93136349458505	1.68198788230711e-14\\
6.94424868139473	4.6296300126869e-14\\
6.95713386820441	5.77315972805081e-15\\
6.97001905501408	3.65263375101677e-14\\
6.98290424182376	1.66533453693773e-14\\
6.99586581007284	4.56301663120939e-14\\
7.00882737832193	5.71764857681956e-15\\
7.02178894657101	3.60822483003176e-14\\
7.03475051482009	1.64313007644523e-14\\
7.04778958943027	4.50750547997814e-14\\
7.06082866404044	5.6621374255883e-15\\
7.07386773865062	3.5527136788005e-14\\
7.08690681326079	1.62092561595273e-14\\
7.10002454751536	4.4353409833775e-14\\
7.11314228176992	5.55111512312578e-15\\
7.12626001602448	3.50830475781549e-14\\
7.13937775027905	1.59872115546023e-14\\
7.15257532859946	4.37427871702312e-14\\
7.16577290691987	5.49560397189452e-15\\
7.17897048524028	3.45279360658424e-14\\
7.19216806356068	1.57651669496772e-14\\
7.20544670103548	4.30211422042248e-14\\
7.21872533851028	5.44009282066327e-15\\
7.23200397598508	3.39728245535298e-14\\
7.24528261345988	1.55431223447522e-14\\
7.2586435601875	4.23550083894497e-14\\
7.27200450691511	5.32907051820075e-15\\
7.28536545364272	3.34732241924485e-14\\
7.29872640037034	1.53210777398272e-14\\
7.31217093962371	4.16333634234434e-14\\
7.32561547887708	5.27355936696949e-15\\
7.33906001813046	3.29181126801359e-14\\
7.35250455738383	1.50990331349021e-14\\
7.36603400873681	4.09672296086683e-14\\
7.37956346008979	5.21804821573824e-15\\
7.39309291144277	3.23074900165921e-14\\
7.40662236279575	1.48214773787458e-14\\
7.42023808324719	4.01900734914307e-14\\
7.43385380369863	5.16253706450698e-15\\
7.44746952415006	3.18078896555107e-14\\
7.4610852446015	1.45994327738208e-14\\
7.47478862987629	3.93574062229618e-14\\
7.48849201515108	5.05151476204446e-15\\
7.50219540042587	3.11972669919669e-14\\
7.51589878570066	1.43773881688958e-14\\
7.52969127264357	3.85247389544929e-14\\
7.54348375958647	4.94049245958195e-15\\
7.55727624652937	3.06421554796543e-14\\
7.57106873347227	1.41553435639707e-14\\
7.58495180146776	3.77475828372553e-14\\
7.59883486946326	4.82947015711943e-15\\
7.61271793745875	2.99760216648792e-14\\
7.62660100545424	1.38777878078145e-14\\
7.64057617834747	3.69149155687865e-14\\
7.6545513512407	4.77395900588817e-15\\
7.66852652413393	2.93098878501041e-14\\
7.68250169702716	1.35447209004269e-14\\
7.69657054527904	3.61377594515488e-14\\
7.71063939353093	4.71844785465692e-15\\
7.72470824178282	2.85882428840978e-14\\
7.7387770900347	1.32671651442706e-14\\
7.75294123299068	3.51940698806175e-14\\
7.76710537594666	4.6074255521944e-15\\
7.78126951890264	2.79221090693227e-14\\
7.79543366185862	1.29340982368831e-14\\
7.80969476969516	3.43058914609173e-14\\
7.8239558775317	4.55191440096314e-15\\
7.83821698536823	2.72004641033163e-14\\
7.85247809320477	1.26565424807268e-14\\
7.86683788984999	3.33066907387547e-14\\
7.8811976864952	4.44089209850063e-15\\
7.89555748314041	2.647881913731e-14\\
7.90991727978563	1.23789867245705e-14\\
7.9243775449477	3.23630011678233e-14\\
7.93883781010978	4.32986979603811e-15\\
7.95329807527185	2.57016630200724e-14\\
7.96775834043393	1.20459198171829e-14\\
7.98232091364819	3.14193115968919e-14\\
7.99688348686244	4.27435864480685e-15\\
8.0114460600767	2.49245069028348e-14\\
8.02600863329096	1.16573417585641e-14\\
8.04067541501396	3.0364599723498e-14\\
8.05534219673697	4.16333634234434e-15\\
8.07000897845997	2.41473507855972e-14\\
8.08467576018298	1.13242748511766e-14\\
8.0994487168151	2.93098878501041e-14\\
8.11422167344722	3.99680288865056e-15\\
8.12899463007934	2.33146835171283e-14\\
8.14376758671146	1.0991207943789e-14\\
8.1586487531091	2.8199664825479e-14\\
8.17352991950675	3.88578058618805e-15\\
8.1884110859044	2.24820162486594e-14\\
8.20329225230204	1.06026298851702e-14\\
8.21828373575118	2.70894418008538e-14\\
8.23327521920032	3.77475828372553e-15\\
8.24826670264945	2.1538326677728e-14\\
8.26325818609859	1.02140518265514e-14\\
8.27836217017066	2.59237076249974e-14\\
8.29346615424273	3.66373598126302e-15\\
8.3085701383148	2.06501482580279e-14\\
8.32367412238687	9.88098491916389e-15\\
8.33889287092472	2.47024622979097e-14\\
8.35411161946257	3.49720252756924e-15\\
8.36933036800042	1.97064586870965e-14\\
8.38454911653827	9.49240686054509e-15\\
8.39988497972518	2.34257058195908e-14\\
8.41522084291208	3.38618022510673e-15\\
8.43055670609898	1.87072579649339e-14\\
8.44589256928588	9.04831765069503e-15\\
8.46134798619136	2.20934381900406e-14\\
8.47680340309684	3.21964677141295e-15\\
8.49225882000231	1.77080572427712e-14\\
8.50771423690779	8.60422844084496e-15\\
8.52329174240573	2.07611705604904e-14\\
8.53886924790367	3.1641356201817e-15\\
8.55444675340161	1.65978342181461e-14\\
8.57002425889955	8.10462807976364e-15\\
8.58572649044944	1.94289029309402e-14\\
8.60142872199933	3.05311331771918e-15\\
8.61713095354922	1.54321000422897e-14\\
8.63283318509911	7.54951656745106e-15\\
8.64866288641169	1.79856129989275e-14\\
8.66449258772426	2.88657986402541e-15\\
8.68032228903684	1.42663658664333e-14\\
8.69615199034942	7.04991620636974e-15\\
8.71211202044306	1.65423230669148e-14\\
8.7280720505367	2.77555756156289e-15\\
8.74403208063034	1.30451205393456e-14\\
8.75999211072399	6.55031584528842e-15\\
8.77608545081172	1.49324996812084e-14\\
8.79217879089946	2.66453525910038e-15\\
8.80827213098719	1.18238752122579e-14\\
8.82436547107493	5.99520433297585e-15\\
8.84059523179885	1.33226762955019e-14\\
8.85682499252278	2.4980018054066e-15\\
8.87305475324671	1.0547118733939e-14\\
8.88928451397064	5.44009282066327e-15\\
8.9056539454559	1.15463194561016e-14\\
8.92202337694115	2.33146835171283e-15\\
8.93839280842641	9.2148511043888e-15\\
8.95476223991167	4.82947015711943e-15\\
8.97127474100433	9.82547376793264e-15\\
8.987787242097	2.1094237467878e-15\\
9.00429974318966	7.7715611723761e-15\\
9.02081224428233	4.27435864480685e-15\\
9.0374713729445	7.93809462606987e-15\\
9.05413050160667	1.88737914186277e-15\\
9.07078963026884	6.32827124036339e-15\\
9.08744875893101	3.66373598126302e-15\\
9.10425824450065	5.93969318174459e-15\\
9.12106773007028	1.72084568816899e-15\\
9.13787721563991	4.82947015711943e-15\\
9.15468670120955	2.99760216648792e-15\\
9.17165045698708	3.94129173741931e-15\\
9.18861421276461	1.49880108324396e-15\\
9.20557796854214	3.21964677141295e-15\\
9.22254172431967	2.27595720048157e-15\\
9.23966386191763	1.77635683940025e-15\\
9.25678599951559	1.27675647831893e-15\\
9.27390813711355	1.49880108324396e-15\\
9.29103027471151	1.55431223447522e-15\\
9.30828551504259	4.9960036108132e-16\\
9.32554075537366	9.99200722162641e-16\\
9.34279599570473	2.77555756156289e-16\\
9.36005123603581	7.7715611723761e-16\\
9.37740098280311	2.77555756156289e-15\\
9.39475072957042	7.7715611723761e-16\\
9.41210047633773	2.05391259555654e-15\\
9.42945022310504	0\\
9.44689582322271	5.05151476204446e-15\\
9.46434142334038	4.9960036108132e-16\\
9.48178702345805	3.88578058618805e-15\\
9.49923262357573	8.32667268468867e-16\\
9.51677513419308	7.43849426498855e-15\\
9.53431764481043	2.22044604925031e-16\\
9.55186015542778	5.6621374255883e-15\\
9.56940266604513	1.60982338570648e-15\\
9.58704316415657	9.71445146547012e-15\\
9.60468366226802	0\\
9.62232416037946	7.54951656745106e-15\\
9.6399646584909	2.44249065417534e-15\\
9.65770423902654	1.21014309684142e-14\\
9.67544381956217	3.33066907387547e-16\\
9.6931834000978	9.32587340685131e-15\\
9.71092298063344	3.21964677141295e-15\\
9.72876275712183	1.4432899320127e-14\\
9.74660253361022	6.10622663543836e-16\\
9.76444231009861	1.12132525487141e-14\\
9.782282086587	4.05231403988182e-15\\
9.80022319117994	1.68198788230711e-14\\
9.81816429577288	9.43689570931383e-16\\
9.83610540036582	1.29896093881143e-14\\
9.85404650495876	4.82947015711943e-15\\
9.87209008914481	1.92068583260152e-14\\
9.89013367333085	1.16573417585641e-15\\
9.9081772575169	1.48769885299771e-14\\
9.92622084170295	5.6621374255883e-15\\
9.94436807654274	2.15938378289593e-14\\
9.96251531138254	1.49880108324396e-15\\
9.98066254622234	1.67088565206086e-14\\
9.99880978106213	6.49480469405717e-15\\
10.0170618577606	2.39808173319034e-14\\
10.0353139344591	1.72084568816899e-15\\
10.0535660111575	1.86517468137026e-14\\
10.071818087856	7.32747196252603e-15\\
10.0901762182373	2.63677968348475e-14\\
10.1085343486186	2.05391259555654e-15\\
10.1268924789999	2.05391259555654e-14\\
10.1452506093812	8.21565038222616e-15\\
10.163716026191	2.88102874890228e-14\\
10.1821814430008	2.33146835171283e-15\\
10.2006468598106	2.24265050974282e-14\\
10.2191122766204	9.04831765069503e-15\\
10.237686234429	3.11972669919669e-14\\
10.2562601922376	2.60902410786912e-15\\
10.2748341500462	2.43693953905222e-14\\
10.2934081078548	9.93649607039515e-15\\
10.3120918830375	3.36952687973735e-14\\
10.3307756582203	2.88657986402541e-15\\
10.3494594334031	2.6256774532385e-14\\
10.3681432085859	1.0769163338864e-14\\
10.3869381001644	3.61377594515488e-14\\
10.4057329917429	3.1641356201817e-15\\
10.4245278833213	2.82551759767102e-14\\
10.4433227748998	1.16573417585641e-14\\
10.4622301050096	3.85247389544929e-14\\
10.4811374351193	3.44169137633799e-15\\
10.5000447652291	3.01980662698043e-14\\
10.5189520953388	1.26010313294955e-14\\
10.5379732097484	4.10227407598995e-14\\
10.556994324158	3.66373598126302e-15\\
10.5760154385676	3.21964677141295e-14\\
10.5950365529771	1.34336985979644e-14\\
10.6141728216781	4.35207425653061e-14\\
10.6333090903791	3.94129173741931e-15\\
10.6524453590801	3.41948691584548e-14\\
10.6715816277811	1.43773881688958e-14\\
10.6908344455163	4.60187443707127e-14\\
10.7100872632516	4.21884749357559e-15\\
10.7293400809869	3.61932706027801e-14\\
10.7485928987221	1.53210777398272e-14\\
10.7679636856929	4.85167461761193e-14\\
10.7873344726636	4.49640324973188e-15\\
10.8067052596344	3.81916720471054e-14\\
10.8260760466051	1.62092561595273e-14\\
10.8455662488972	5.09592368302947e-14\\
10.8650564511894	4.77395900588817e-15\\
10.8845466534815	4.02455846426619e-14\\
10.9040368557736	1.71529457304587e-14\\
10.9236479463274	5.35127497869325e-14\\
10.9432590368811	5.05151476204446e-15\\
10.9628701274349	4.22439860869872e-14\\
10.9824812179887	1.80411241501588e-14\\
11.0022146969848	5.60662627435704e-14\\
11.0219481759809	5.38458166943201e-15\\
11.041681654977	4.4353409833775e-14\\
11.0614151339731	1.89848137210902e-14\\
11.0812725294807	5.86197757002083e-14\\
11.1011299249883	5.6621374255883e-15\\
11.1209873204959	4.63518112781003e-14\\
11.1408447160036	1.98729921407903e-14\\
11.1608275848708	6.11732886568461e-14\\
11.180810453738	5.93969318174459e-15\\
11.2007933226052	4.84057238736568e-14\\
11.2207761914724	2.08166817117217e-14\\
11.2408861199495	6.38378239159465e-14\\
11.2609960484265	6.27276008913213e-15\\
11.2811059769036	5.04041253179821e-14\\
11.3012159053807	2.17048601314218e-14\\
11.3214545099205	6.64468480238156e-14\\
11.3416931144603	6.55031584528842e-15\\
11.3619317190001	5.25135490647699e-14\\
11.3821703235399	2.26485497023532e-14\\
11.4025392513608	6.9111383282916e-14\\
11.4229081791817	6.93889390390723e-15\\
11.4432771070026	5.45674616603264e-14\\
11.4636460348235	2.35922392732846e-14\\
11.4841469650118	7.17204073907851e-14\\
11.5046478952002	7.21644966006352e-15\\
11.5251488253886	5.67323965583455e-14\\
11.545649755577	2.4535928844216e-14\\
11.5662843996785	7.44404538011167e-14\\
11.5869190437801	7.54951656745106e-15\\
11.6075536878816	5.8786309153902e-14\\
11.6281883319832	2.54241072639161e-14\\
11.6489584349037	7.71049890602171e-14\\
11.6697285378243	7.88258347483861e-15\\
11.6904986407448	6.08402217494586e-14\\
11.7112687436654	2.63677968348475e-14\\
11.7321760845019	7.98250354705488e-14\\
11.7530834253385	8.21565038222616e-15\\
11.7739907661751	6.29496454962464e-14\\
11.7948981070116	2.73114864057789e-14\\
11.8159445000986	8.24895707296491e-14\\
11.8369908931856	8.54871728961371e-15\\
11.8580372862726	6.51145803942654e-14\\
11.8790836793596	2.83106871279415e-14\\
11.900270975065	8.52096171399808e-14\\
11.9214582707703	8.82627304576999e-15\\
11.9426455664757	6.72795152922845e-14\\
11.9638328621811	2.92543766988729e-14\\
11.9851629478153	8.79851747015437e-14\\
12.0064930334495	9.15933995315754e-15\\
12.0278231190837	6.94444501903035e-14\\
12.0491532047179	3.02535774210355e-14\\
12.0706280057176	9.07052211118753e-14\\
12.0921028067173	9.49240686054509e-15\\
12.113577607717	7.16093850883226e-14\\
12.1350524087168	3.11972669919669e-14\\
12.1566738894695	9.34807786734382e-14\\
12.1782953702222	9.82547376793264e-15\\
12.1999168509749	7.37743199863417e-14\\
12.2215383317276	3.21964677141295e-14\\
12.2433084968313	9.62563362350011e-14\\
12.265078661935	1.01585406753202e-14\\
12.2868488270387	7.5994766035592e-14\\
12.3086189921423	3.31956684362922e-14\\
12.3305398873956	9.9031893796564e-14\\
12.3524607826488	1.04916075827077e-14\\
12.3743816779021	7.82152120848423e-14\\
12.3963025731553	3.41393580072236e-14\\
12.4183762865519	1.01918473660589e-13\\
12.4404499999485	1.08801856413265e-14\\
12.462523713345	8.03801469828613e-14\\
12.4845974267416	3.51385587293862e-14\\
12.50682608981	1.04749542373384e-13\\
12.5290547528783	1.12687636999453e-14\\
12.5512834159467	8.26005930321116e-14\\
12.5735120790151	3.61932706027801e-14\\
12.5958978680728	1.07580611086178e-13\\
12.6182836571306	1.15463194561016e-14\\
12.6406694461884	8.48765502325932e-14\\
12.6630552352461	3.71369601737115e-14\\
12.6856003725544	1.10467190950203e-13\\
12.7081455098627	1.19348975147204e-14\\
12.730690647171	8.70969962818435e-14\\
12.7532357844793	3.81916720471054e-14\\
12.7759425395622	1.13353770814228e-13\\
12.7986492946451	1.22957199977236e-14\\
12.821356049728	8.93729534823251e-14\\
12.844062804811	3.9190872769268e-14\\
12.8668397124378	1.14380727112007e-13\\
12.8896166200647	1.33504318711175e-14\\
12.9123935276916	8.91509088774001e-14\\
12.9351704353184	3.92463839204993e-14\\
12.9579981520666	1.14325215960775e-13\\
12.9808258688147	1.33504318711175e-14\\
13.0036535855629	8.91231533017844e-14\\
13.026481302311	3.92463839204993e-14\\
13.049363098101	1.1429746038516e-13\\
13.0722448938909	1.33781874467331e-14\\
13.0951266896808	8.91231533017844e-14\\
13.1180084854707	3.92741394961149e-14\\
13.1409474667556	1.14214193658313e-13\\
13.1638864480404	1.33781874467331e-14\\
13.1868254293253	8.91231533017844e-14\\
13.2097644106101	3.93018950717305e-14\\
13.2327635996633	1.14186438082697e-13\\
13.2557627887165	1.33781874467331e-14\\
13.2787619777697	8.91231533017844e-14\\
13.3017611668229	3.93018950717305e-14\\
13.3248235100991	1.14158682507082e-13\\
13.3478858533754	1.34059430223488e-14\\
13.3709481966517	8.90676421505532e-14\\
13.3940105399279	3.93018950717305e-14\\
13.4171389165768	1.14130926931466e-13\\
13.4402672932257	1.346145417358e-14\\
13.4633956698746	8.90398865749376e-14\\
13.4865240465235	3.93296506473462e-14\\
13.5097212774941	1.14130926931466e-13\\
13.5329185084647	1.346145417358e-14\\
13.5561157394353	8.90121309993219e-14\\
13.5793129704059	3.93296506473462e-14\\
13.6025818233678	1.14075415780235e-13\\
13.6258506763298	1.346145417358e-14\\
13.6491195292917	8.90121309993219e-14\\
13.6723883822536	3.93574062229618e-14\\
13.6957315789086	1.13992149053388e-13\\
13.7190747755636	1.346145417358e-14\\
13.7424179722186	8.90121309993219e-14\\
13.7657611688735	3.93851617985774e-14\\
13.789181390828	1.13992149053388e-13\\
13.8126016127825	1.34892097491957e-14\\
13.8360218347369	8.90121309993219e-14\\
13.8594420566914	3.94129173741931e-14\\
13.8829419497165	1.13936637902157e-13\\
13.9064418427417	1.34892097491957e-14\\
13.9299417357668	8.89843754237063e-14\\
13.9534416287919	3.94129173741931e-14\\
13.9770238098647	1.13936637902157e-13\\
14.0006059909374	1.34892097491957e-14\\
14.0241881720102	8.89843754237063e-14\\
14.0477703530829	3.94129173741931e-14\\
14.0714374141426	1.13908882326541e-13\\
14.0951044752024	1.35169653248113e-14\\
14.1187715362621	8.89566198480907e-14\\
14.1424385973218	3.94406729498087e-14\\
14.1661931077477	1.1385337117531e-13\\
14.1899476181736	1.35169653248113e-14\\
14.2137021285995	8.89566198480907e-14\\
14.2374566390255	3.94406729498087e-14\\
14.2613011526304	1.13825615599694e-13\\
14.2851456662352	1.35447209004269e-14\\
14.3089901798401	8.89566198480907e-14\\
14.332834693445	3.94684285254243e-14\\
14.3567717508919	1.13797860024079e-13\\
14.3807088083388	1.35447209004269e-14\\
14.4046458657857	8.89566198480907e-14\\
14.4285829232326	3.94961841010399e-14\\
14.4526150559176	1.13742348872847e-13\\
14.4766471886026	1.35447209004269e-14\\
14.5006793212876	8.89566198480907e-14\\
14.5247114539725	3.94961841010399e-14\\
14.548841186964	1.13742348872847e-13\\
14.5729709199554	1.35724764760425e-14\\
14.5971006529469	8.89011086968594e-14\\
14.6212303859383	3.95239396766556e-14\\
14.6454602425121	1.13714593297232e-13\\
14.6696900990858	1.35724764760425e-14\\
14.6939199556595	8.8928864272475e-14\\
14.7181498122333	3.94961841010399e-14\\
14.7424823150752	1.13686837721616e-13\\
14.7668148179171	1.35724764760425e-14\\
14.7911473207591	8.89011086968594e-14\\
14.815479823601	3.95239396766556e-14\\
14.8399174997295	1.13631326570385e-13\\
14.864355175858	1.36002320516582e-14\\
14.8887928519865	8.89011086968594e-14\\
14.913230528115	3.95516952522712e-14\\
14.9377759105528	1.13631326570385e-13\\
14.9623212929905	1.36279876272738e-14\\
14.9868666754283	8.88733531212438e-14\\
15.0114120578661	3.95239396766556e-14\\
15.0360676892572	1.13603570994769e-13\\
15.0607233206484	1.36279876272738e-14\\
15.0853789520395	8.88733531212438e-14\\
15.1100345834307	3.95516952522712e-14\\
15.1348030187572	1.13575815419154e-13\\
15.1595714540837	1.36557432028894e-14\\
15.1843398894102	8.88455975456282e-14\\
15.2091083247367	3.95516952522712e-14\\
15.2339921326276	1.13548059843538e-13\\
15.2588759405185	1.36557432028894e-14\\
15.2837597484095	8.88178419700125e-14\\
15.3086435563004	3.95516952522712e-14\\
15.333645323629	1.13548059843538e-13\\
15.3586470909575	1.36834987785051e-14\\
15.383648858286	8.88178419700125e-14\\
15.4086506256146	3.95794508278868e-14\\
15.4337729580942	1.13492548692307e-13\\
15.4588952905739	1.36834987785051e-14\\
15.4840176230536	8.88178419700125e-14\\
15.5091399555333	3.95794508278868e-14\\
15.5343854824194	1.13492548692307e-13\\
15.5596310093056	1.36834987785051e-14\\
15.5848765361918	8.88178419700125e-14\\
15.610122063078	3.96072064035025e-14\\
15.6354934388609	1.13464793116691e-13\\
15.6608648146438	1.36834987785051e-14\\
15.6862361904268	8.88178419700125e-14\\
15.7116075662097	3.96349619791181e-14\\
15.7371074715309	1.13381526389844e-13\\
15.762607376852	1.36557432028894e-14\\
15.7881072821731	8.88178419700125e-14\\
15.8136071874943	3.96627175547337e-14\\
15.8392383339326	1.13381526389844e-13\\
15.8648694803709	1.36834987785051e-14\\
15.8905006268092	8.88178419700125e-14\\
15.9161317732476	3.96627175547337e-14\\
15.9418969045212	1.13353770814228e-13\\
15.9676620357948	1.37112543541207e-14\\
15.9934271670685	8.87900863943969e-14\\
16.0191922983421	3.96627175547337e-14\\
16.0450941933734	1.13381526389844e-13\\
16.0709960884047	1.37390099297363e-14\\
16.096897983436	8.87623308187813e-14\\
16.1227998784672	3.96627175547337e-14\\
16.1488413541292	1.13326015238613e-13\\
16.1748828297911	1.37390099297363e-14\\
16.2009243054531	8.87623308187813e-14\\
16.226965781115	3.96627175547337e-14\\
16.2531496945447	1.13326015238613e-13\\
16.2793336079744	1.37667655053519e-14\\
16.3055175214041	8.87345752431656e-14\\
16.3317014348337	3.96627175547337e-14\\
16.3580306848324	1.13298259662997e-13\\
16.384359934831	1.37667655053519e-14\\
16.4106891848297	8.87345752431656e-14\\
16.4370184348283	3.96904731303493e-14\\
16.4634959660973	1.13242748511766e-13\\
16.4899734973663	1.37667655053519e-14\\
16.5164510286353	8.870681966755e-14\\
16.5429285599043	3.96904731303493e-14\\
16.5695573662806	1.13270504087382e-13\\
16.5961861726568	1.37945210809676e-14\\
16.6228149790331	8.86790640919344e-14\\
16.6494437854093	3.96904731303493e-14\\
16.6762269105277	1.13242748511766e-13\\
16.7030100356461	1.38222766565832e-14\\
16.7297931607645	8.86790640919344e-14\\
16.7565762858829	3.96904731303493e-14\\
16.7835168283287	1.1321499293615e-13\\
16.8104573707745	1.38500322321988e-14\\
16.8373979132204	8.86790640919344e-14\\
16.8643384556662	3.9718228705965e-14\\
16.8914395705478	1.13187237360535e-13\\
16.9185406854293	1.38500322321988e-14\\
16.9456418003109	8.86513085163187e-14\\
16.9727429151925	3.9718228705965e-14\\
17.0000078170973	1.13159481784919e-13\\
17.027272719002	1.38500322321988e-14\\
17.0545376209068	8.86235529407031e-14\\
17.0818025228116	3.9718228705965e-14\\
17.1092344893586	1.13131726209303e-13\\
17.1366664559057	1.38777878078145e-14\\
17.1640984224528	8.86513085163187e-14\\
17.1915303889999	3.97459842815806e-14\\
17.219132763799	1.13103970633688e-13\\
17.2467351385981	1.38777878078145e-14\\
17.2743375133972	8.86235529407031e-14\\
17.3019398881964	3.97459842815806e-14\\
17.3297160854172	1.13076215058072e-13\\
17.3574922826381	1.39055433834301e-14\\
17.385268479859	8.86235529407031e-14\\
17.4130446770799	3.97459842815806e-14\\
17.4409981837359	1.13076215058072e-13\\
17.4689516903919	1.39332989590457e-14\\
17.4969051970478	8.85957973650875e-14\\
17.5248587037038	3.97459842815806e-14\\
17.5529930820312	1.13048459482457e-13\\
17.5811274603586	1.39332989590457e-14\\
17.6092618386859	8.85680417894719e-14\\
17.6373962170133	3.97459842815806e-14\\
17.6657151116334	1.13020703906841e-13\\
17.6940340062535	1.39610545346613e-14\\
17.7223529008736	8.85402862138562e-14\\
17.7506717954937	3.97459842815806e-14\\
17.7791789342782	1.12992948331225e-13\\
17.8076860730626	1.39610545346613e-14\\
17.836193211847	8.85402862138562e-14\\
17.8647003506315	3.97737398571962e-14\\
17.8933995511133	1.1296519275561e-13\\
17.922098751595	1.3988810110277e-14\\
17.9507979520768	8.85125306382406e-14\\
17.9794971525585	3.97737398571962e-14\\
18.0083923245633	1.1296519275561e-13\\
18.0372874965681	1.40165656858926e-14\\
18.0661826685729	8.85125306382406e-14\\
18.0950778405776	3.97737398571962e-14\\
18.124172992061	1.12937437179994e-13\\
18.1532681435443	1.40443212615082e-14\\
18.1823632950276	8.84570194870093e-14\\
18.2114584465109	3.98014954328119e-14\\
18.2407576866174	1.12881926028763e-13\\
18.270056926724	1.40443212615082e-14\\
18.2993561668305	8.8484775062625e-14\\
18.328655406937	3.98014954328119e-14\\
18.3581629537927	1.12854170453147e-13\\
18.3876705006484	1.40443212615082e-14\\
18.4171780475041	8.84570194870093e-14\\
18.4466855943597	3.98292510084275e-14\\
18.4764057786275	1.12826414877532e-13\\
18.5061259628954	1.40443212615082e-14\\
18.5358461471632	8.84570194870093e-14\\
18.565566331431	3.98570065840431e-14\\
18.5955036012303	1.12798659301916e-13\\
18.6254408710296	1.40720768371239e-14\\
18.6553781408289	8.84570194870093e-14\\
18.6853154106283	3.98570065840431e-14\\
18.7154743397902	1.12743148150685e-13\\
18.7456332689521	1.40720768371239e-14\\
18.7757921981141	8.84570194870093e-14\\
18.805951127276	3.98847621596587e-14\\
18.8363364194138	1.12715392575069e-13\\
18.8667217115517	1.40998324127395e-14\\
18.8971070036895	8.84015083357781e-14\\
18.9274922958273	3.98570065840431e-14\\
18.9581087922171	1.12687636999453e-13\\
18.9887252886069	1.41275879883551e-14\\
19.0193417849967	8.84015083357781e-14\\
19.0499582813864	3.98847621596587e-14\\
19.0808109690034	1.12659881423838e-13\\
19.1116636566204	1.41275879883551e-14\\
19.1425163442373	8.84015083357781e-14\\
19.1733690318543	3.99125177352744e-14\\
19.2044630478062	1.12632125848222e-13\\
19.235557063758	1.41553435639707e-14\\
19.2666510797099	8.83737527601625e-14\\
19.2977450956618	3.994027331089e-14\\
19.3290857385735	1.12576614696991e-13\\
19.3604263814852	1.41275879883551e-14\\
19.3917670243968	8.83737527601625e-14\\
19.4231076673085	3.994027331089e-14\\
19.4547004046881	1.12548859121375e-13\\
19.4862931420676	1.41275879883551e-14\\
19.5178858794472	8.83737527601625e-14\\
19.5494786168267	3.99680288865056e-14\\
19.581329093847	1.1252110354576e-13\\
19.6131795708672	1.41830991395864e-14\\
19.6450300478875	8.83737527601625e-14\\
19.6768805249077	3.99680288865056e-14\\
19.7089945750868	1.12465592394528e-13\\
19.7411086252659	1.41830991395864e-14\\
19.7732226754451	8.83182416089312e-14\\
19.8053367256242	3.99957844621213e-14\\
19.8377203807137	1.12437836818913e-13\\
19.8701040358032	1.4210854715202e-14\\
19.9024876908927	8.83182416089312e-14\\
19.9348713459822	4.00235400377369e-14\\
19.967530847752	1.12382325667681e-13\\
20.0001903495219	1.4210854715202e-14\\
20.0328498512918	8.83182416089312e-14\\
20.0655093530616	4.00235400377369e-14\\
20.0984511655557	1.12382325667681e-13\\
20.1313929780498	1.42663658664333e-14\\
20.1643347905439	8.82904860333156e-14\\
20.197276603038	4.00512956133525e-14\\
20.2305074253896	1.1232681451645e-13\\
20.2637382477412	1.42386102908176e-14\\
20.2969690700928	8.82904860333156e-14\\
20.3301998924444	4.00790511889682e-14\\
20.3637266734178	1.12271303365219e-13\\
20.3972534543913	1.42386102908176e-14\\
20.4307802353647	8.82627304576999e-14\\
20.4643070163382	4.01068067645838e-14\\
20.4981369694764	1.12215792213988e-13\\
20.5319669226146	1.42941214420489e-14\\
20.5657968757529	8.82627304576999e-14\\
20.5996268288911	4.01068067645838e-14\\
20.6337674490278	1.12188036638372e-13\\
20.6679080691645	1.42941214420489e-14\\
20.7020486893012	8.82349748820843e-14\\
20.7361893094379	4.01206845523916e-14\\
20.7706483924226	1.12146403274949e-13\\
20.8051074754072	1.43218770176645e-14\\
20.8395665583918	8.82210970942765e-14\\
20.8740256413764	4.01345623401994e-14\\
20.908811298434	1.12104769911525e-13\\
20.9435969554916	1.43357548054723e-14\\
20.9783826125492	8.81933415186609e-14\\
21.0131682696068	4.01484401280072e-14\\
21.0482889562591	1.12063136548102e-13\\
21.0834096429114	1.4363510381088e-14\\
21.1185303295637	8.81794637308531e-14\\
21.153651016216	4.01761957036229e-14\\
21.189115551059	1.1200762539687e-13\\
21.224580085902	1.43912659567036e-14\\
21.260044620745	8.81517081552374e-14\\
21.295509155588	4.01900734914307e-14\\
21.3313267450985	1.11952114245639e-13\\
21.367144334609	1.44051437445114e-14\\
21.4029619241194	8.81239525796218e-14\\
21.4387795136299	4.02039512792385e-14\\
21.474959783243	1.11910480882216e-13\\
21.5111400528561	1.4432899320127e-14\\
21.5473203224691	8.8110074791814e-14\\
21.5835005920822	4.02317068548541e-14\\
21.6200536143196	1.11854969730985e-13\\
21.656606636557	1.44606548957427e-14\\
21.6931596587944	8.80823192161984e-14\\
21.7297126810318	4.02594624304697e-14\\
21.7666490124925	1.11799458579753e-13\\
21.8035853439532	1.45022882591661e-14\\
21.8405216754139	8.80545636405827e-14\\
21.8774580068746	4.02872180060854e-14\\
21.9147887220248	1.11730069640714e-13\\
21.9521194371751	1.45022882591661e-14\\
21.9894501523253	8.80406858527749e-14\\
22.0267808674756	4.0314973581701e-14\\
22.0645176030464	1.11674558489483e-13\\
22.1022543386172	1.45300438347817e-14\\
22.139991074188	8.80129302771593e-14\\
22.1777278097588	4.03427291573166e-14\\
22.2158828101726	1.11605169550444e-13\\
22.2540378105863	1.45577994103974e-14\\
22.2921928110001	8.79990524893515e-14\\
22.3303478114138	4.03704847329323e-14\\
22.3689339797914	1.11535780611405e-13\\
22.407520148169	1.4585554986013e-14\\
22.4461063165466	8.79851747015437e-14\\
22.4846924849242	4.03982403085479e-14\\
22.5237234418978	1.11466391672366e-13\\
22.5627543988714	1.46133105616286e-14\\
22.601785355845	8.79296635503124e-14\\
22.6408163128186	4.04259958841635e-14\\
22.6803064595616	1.11397002733327e-13\\
22.7197966063045	1.46688217128599e-14\\
22.7592867530475	8.79019079746968e-14\\
22.7987768997904	4.04398736719713e-14\\
22.838741493306	1.11327613794288e-13\\
22.8787060868217	1.46965772884755e-14\\
22.9186706803373	8.78741523990811e-14\\
22.9586352738529	4.04815070353948e-14\\
22.9990905077973	1.11230469279633e-13\\
23.0395457417417	1.4738210651899e-14\\
23.0800009756861	8.78325190356577e-14\\
23.1204562096305	4.05092626110104e-14\\
23.1614193085936	1.11161080340594e-13\\
23.2023824075567	1.47937218031302e-14\\
23.2433455065199	8.77770078844264e-14\\
23.284308605483	4.05231403988182e-14\\
23.3257979315477	1.11091691401555e-13\\
23.3672872576124	1.48492329543615e-14\\
23.4087765836772	8.7735374521003e-14\\
23.4502659097419	4.05508959744338e-14\\
23.4922104776085	1.10064735103776e-13\\
23.5341550454751	1.52933221642115e-14\\
23.5760996133418	8.62920845889903e-14\\
23.6180441812084	3.99957844621213e-14\\
23.6604170261043	1.08218989325337e-13\\
23.7027898710001	1.50990331349021e-14\\
23.745162715896	8.48349168691698e-14\\
23.7875355607918	3.93712840107696e-14\\
23.8303479690752	1.06303854607859e-13\\
23.8731603773585	1.49047441055927e-14\\
23.9159727856418	8.33361157859258e-14\\
23.9587851939251	3.87190279838023e-14\\
24.0020482633235	1.04333208739149e-13\\
24.0453113327218	1.47243328640911e-14\\
24.0885744021202	8.17818035514506e-14\\
24.1318374715185	3.80528941690272e-14\\
24.1755626771472	1.022931739314e-13\\
24.219287882776	1.45161660469739e-14\\
24.2630130884048	8.01719801657441e-14\\
24.3067382940335	3.73728825664443e-14\\
24.3509374990427	1.00183750184613e-13\\
24.3951367040518	1.43218770176645e-14\\
24.4393359090609	7.85205234166142e-14\\
24.48353511407	3.6651237600438e-14\\
24.5282205890312	9.80188152865935e-14\\
24.5729060639923	1.41275879883551e-14\\
24.6175915389535	7.67719221528296e-14\\
24.6622770139146	3.5901837058816e-14\\
24.7074614574793	9.57706136617276e-14\\
24.7526459010441	1.39194211712379e-14\\
24.7978303446088	7.49816875256215e-14\\
24.8430147881735	3.51246809415784e-14\\
24.888711348921	9.34113897343991e-14\\
24.9344079096686	1.36834987785051e-14\\
24.9801044704162	7.31081861715666e-14\\
25.0258010311637	3.4333647036533e-14\\
25.0720233306192	9.09411435046081e-14\\
25.1182456300747	1.34336985979644e-14\\
25.1644679295302	7.11930514540882e-14\\
25.2106902289857	3.35009797680641e-14\\
25.2574523862228	8.83737527601625e-14\\
25.3042145434599	1.31838984174237e-14\\
25.350976700697	6.91530166463394e-14\\
25.3977388579342	3.26266791361718e-14\\
25.4450555156315	8.57092175010621e-14\\
25.4923721733288	1.29340982368831e-14\\
25.5396888310261	6.70435928995516e-14\\
25.5870054887234	3.1710745140856e-14\\
25.6348918408033	8.29197821516914e-14\\
25.6827781928832	1.26565424807268e-14\\
25.7306645449631	6.48092690624935e-14\\
25.7785508970431	3.07531777821168e-14\\
25.8270227182624	7.99776911364347e-14\\
25.8754945394818	1.23789867245705e-14\\
25.9239663607012	6.24916784985885e-14\\
25.9724381819206	2.97539770599542e-14\\
26.0215118591534	7.69107000309077e-14\\
26.0705855363863	1.20875531806064e-14\\
26.1196592136191	6.00491878444132e-14\\
26.168732890852	2.86853873987525e-14\\
26.2184254566214	7.36910532594948e-14\\
26.2681180223908	1.17822418488345e-14\\
26.3178105881602	5.74956748877753e-14\\
26.3675031539296	2.75751643741273e-14\\
26.4178323219975	7.02909952465802e-14\\
26.4681614900653	1.14491749414469e-14\\
26.5184906581332	5.47895062652515e-14\\
26.5688198262011	2.64094301982709e-14\\
26.6198040298383	6.67382815677797e-14\\
26.6707882334756	1.11161080340594e-14\\
26.7217724371128	5.19306819768417e-14\\
26.7727566407501	2.51743070833754e-14\\
26.8244150733526	6.2977401071862e-14\\
26.8760735059551	1.07552855510562e-14\\
26.9277319385576	4.89330798103538e-14\\
26.9793903711602	2.38697950294409e-14\\
27.0317430299247	5.89944759710193e-14\\
27.0840956886892	1.03667074924374e-14\\
27.1364483474537	4.57550664023643e-14\\
27.1888010062182	2.24820162486594e-14\\
27.2418687387887	5.48033840530593e-14\\
27.2949364713593	9.97812943381859e-15\\
27.3480042039299	4.23966417528732e-14\\
27.4010719365005	2.10248485288389e-14\\
27.4548764910367	5.03486141667508e-14\\
27.508681045573	9.53404022396853e-15\\
27.5624856001092	3.88300502862649e-14\\
27.6162901546455	1.94705362943637e-14\\
27.6708542335196	4.56301663120939e-14\\
27.7254183123937	9.08301212021456e-15\\
27.7799823912678	3.50483531086354e-14\\
27.834546470142	1.78190795452338e-14\\
27.8898937876417	4.0613346019569e-14\\
27.9452411051415	8.60422844084496e-15\\
28.0005884226412	3.10307335382731e-14\\
28.055935740141	1.60635393875452e-14\\
28.1120910845417	3.52773366074643e-14\\
28.1682464289425	8.08381139805192e-15\\
28.2244017733432	2.67563748934663e-14\\
28.2805571177439	1.41900380334903e-14\\
28.3375464180691	2.95943825001643e-14\\
28.3945357183943	7.54951656745106e-15\\
28.4515250187195	2.21836438107914e-14\\
28.5085143190447	1.21846976952611e-14\\
28.5663647170539	2.35436670159572e-14\\
28.6242151150631	6.96664947952286e-15\\
28.6820655130723	1.73125402902485e-14\\
28.7399159110815	1.00544572667616e-14\\
28.7983043617203	1.66533453693773e-14\\
28.856692812359	6.42541575501809e-15\\
28.9150812629978	1.1442236047543e-14\\
28.9734697136365	7.41767758327683e-15\\
29.0321280026001	9.32587340685131e-15\\
29.0907862915636	5.32213162429684e-15\\
29.1494445805272	5.78703751585863e-15\\
29.2081028694908	4.77395900588817e-15\\
29.2670352867619	2.36616282123236e-15\\
29.325967704033	4.26741975090295e-15\\
29.3849001213041	4.09394740330526e-16\\
29.4438325385752	2.26207941267376e-15\\
29.5030416355931	4.23966417528732e-15\\
29.562250732611	3.27515792264421e-15\\
29.621459829629	4.69069227904129e-15\\
29.6806689266469	1.2490009027033e-16\\
29.7401573016333	1.05124242644195e-14\\
29.7996456766196	2.32452945780892e-15\\
29.859134051606	9.54791801177635e-15\\
29.9186224265924	2.39391839684799e-15\\
29.9783927156743	1.64659952339719e-14\\
30.0381630047562	1.42247325030098e-15\\
30.0979332938381	1.41414657761629e-14\\
30.15770358292	4.54497550705923e-15\\
30.2177584582065	2.21142548717523e-14\\
30.2778133334929	5.62050406216485e-16\\
30.3378682087793	1.84990911478167e-14\\
30.3979230840658	6.58501031480796e-15\\
30.4582652560913	2.74710809655687e-14\\
30.5186074281169	2.56739074444567e-16\\
30.5789496001424	2.26416108084493e-14\\
30.639291772168	8.52096171399808e-15\\
30.6999239910786	3.25503513032288e-14\\
30.7605562099893	1.03389519168218e-15\\
30.8211884289	2.6555146970253e-14\\
30.8818206478106	1.03528297046296e-14\\
30.9427457014925	3.73590047786365e-14\\
31.0036707551745	1.76941794549634e-15\\
31.0645958088564	3.02605163149394e-14\\
31.1255208625383	1.20875531806064e-14\\
31.186741581882	4.19109191795997e-14\\
31.2479623012256	2.47024622979097e-15\\
31.3091830205692	3.37715966303165e-14\\
31.3704037399129	1.37390099297363e-14\\
31.4319229948219	4.622691118783e-14\\
31.493442249731	3.13638004456607e-15\\
31.55496150464	3.7095326810288e-14\\
31.6164807595491	1.53002610581154e-14\\
31.6783014623118	5.03000419094235e-14\\
31.7401221650745	3.76781938982163e-15\\
31.8019428678372	4.02455846426619e-14\\
31.8637635705999	1.67782454596477e-14\\
31.9258886754761	5.4151128026092e-14\\
31.9880137803524	4.35762537165374e-15\\
32.0501388852286	4.32154312335342e-14\\
32.1122639901049	1.8172963134333e-14\\
32.1746964962864	5.78009862195472e-14\\
32.2371290024679	4.91967577787022e-15\\
32.2995615086494	4.60256832646166e-14\\
32.361994014831	1.94982918699793e-14\\
32.4247369657647	6.12357387019813e-14\\
32.4874799166984	5.46090950237499e-15\\
32.5502228676321	4.86763407359092e-14\\
32.6129658185659	2.07472927726826e-14\\
32.6760223019509	6.44900799429138e-14\\
32.739078785336	5.96050986345631e-15\\
32.8021352687211	5.11882203291236e-14\\
32.8651917521062	2.19269047363468e-14\\
32.9285649023539	6.75570710484408e-14\\
32.9919380526016	6.44623243672982e-15\\
33.0553112028493	5.3554383150356e-14\\
33.118684353097	2.30440666548759e-14\\
33.1823773519202	7.04505898063701e-14\\
33.2460703507434	6.89726054048379e-15\\
33.3097633495666	5.57817680935102e-14\\
33.3734563483898	2.40987785282698e-14\\
33.4374724249187	7.31845140045095e-14\\
33.5014885014475	7.32747196252603e-15\\
33.5655045779763	5.78842529463941e-14\\
33.6295206545052	2.50841014626246e-14\\
33.6938630873744	7.5765782536763e-14\\
33.7582055202435	7.73686670285656e-15\\
33.8225479531127	5.98618377090077e-14\\
33.8868903859819	2.60208521396521e-14\\
33.951562503147	7.81805176153227e-14\\
34.016234620312	8.11850586757146e-15\\
34.0809067374771	6.17422779569665e-14\\
34.1455788546421	2.69020916654483e-14\\
34.2105840369751	8.04634137097082e-14\\
34.2755892193081	8.47932835057463e-15\\
34.340594401641	6.3504757008559e-14\\
34.405599583974	2.77347589339172e-14\\
34.4709412619942	8.26144708199195e-14\\
34.5362829400144	8.81933415186609e-15\\
34.6016246180346	6.5149274863785e-14\\
34.6669662960548	2.85223233920107e-14\\
34.7326479550893	8.46371583929084e-14\\
34.7983296141237	9.14546216534973e-15\\
34.8640112731582	6.67105259921641e-14\\
34.9296929321926	2.92613155927768e-14\\
34.9957181104378	8.6534945875627e-14\\
35.061743288683	9.44730405016969e-15\\
35.1277684669281	6.81781020528405e-14\\
35.1937936451733	2.99586744301195e-14\\
35.2601659376236	8.83182416089312e-14\\
35.3265382300739	9.73179870022989e-15\\
35.3929105225242	6.95554724927661e-14\\
35.4592828149746	3.06143999040387e-14\\
35.526005872278	8.99974539336768e-14\\
35.5927289295814	1.00024155624823e-14\\
35.6594519868847	7.08530456527967e-14\\
35.7261750441881	3.12319614614864e-14\\
35.7932525744811	9.15725828498637e-14\\
35.8603301047741	1.02556851899749e-14\\
35.927407635067	7.20673520859805e-14\\
35.99448516536	3.18078896555107e-14\\
36.0619209371048	9.30505672513959e-14\\
36.1293567088496	1.04985464766116e-14\\
36.1967924805944	7.32088001331732e-14\\
36.2642282523392	3.23525928269675e-14\\
36.3320260925956	9.44383460321774e-14\\
36.3998239328521	1.07205910815367e-14\\
36.4676217731086	7.42773897943749e-14\\
36.5354196133651	3.28660709758566e-14\\
36.6035834111162	9.573938863916e-14\\
36.6717472088673	1.09322273456058e-14\\
36.7399110066184	7.52800599634895e-14\\
36.8080748043695	3.33413852082742e-14\\
36.8766085113362	9.69571645192957e-14\\
36.9451422183029	1.11334552688191e-14\\
37.0136759252696	7.62237495344209e-14\\
37.0822096322362	3.37924133120282e-14\\
37.1511172647722	9.80986125664884e-14\\
37.2200248973081	1.13208054042246e-14\\
37.2889325298441	7.71015196132652e-14\\
37.35784016238	3.42122163932146e-14\\
37.4271258025917	9.91637327807382e-14\\
37.4964114428034	1.14942777518223e-14\\
37.5656970830151	7.79272479878301e-14\\
37.6349827232268	3.46077333457373e-14\\
37.7046505191923	1.00162933502901e-13\\
37.7743183151579	1.16608112055161e-14\\
37.8439861111234	7.87009346581158e-14\\
37.913653907089	3.49824336165483e-14\\
37.9837080759575	1.01096214732976e-13\\
38.0537622448261	1.1816936318354e-14\\
38.1238164136947	7.94191101771702e-14\\
38.1938705825633	3.53328477586956e-14\\
38.2643154110918	1.01973984811821e-13\\
38.3347602396202	1.19626530903361e-14\\
38.4052050681487	8.00956523328011e-14\\
38.4756498966772	3.56555063252273e-14\\
38.5464897434044	1.02785835398578e-13\\
38.6173295901316	1.20979615214623e-14\\
38.6881694368588	8.07270916780567e-14\\
38.759009283586	3.59608176569992e-14\\
38.8302485799245	1.03549113728008e-13\\
38.9014878762629	1.22263310586845e-14\\
38.9727271726013	8.13134282129369e-14\\
39.0439664689397	3.62453123070594e-14\\
39.1156097204774	1.04256880906206e-13\\
39.1872529720151	1.23512311489549e-14\\
39.2588962235527	8.18650702782975e-14\\
39.3305394750904	3.65124597223598e-14\\
39.4025912659757	1.04923014720981e-13\\
39.474643056861	1.24622534514174e-14\\
39.5466948477463	8.23750789802347e-14\\
39.6187466386316	3.67622599029005e-14\\
39.6912116291116	1.05540576278429e-13\\
39.7636766195915	1.25698063069279e-14\\
39.8361416100714	8.28538626596043e-14\\
39.9086066005514	3.69947128486814e-14\\
39.9814895308678	1.06113035025501e-13\\
40.0543724611842	1.26704202685346e-14\\
40.1272553915006	8.32979518694543e-14\\
40.200138321817	3.72098185597025e-14\\
40.2734440149134	1.06650799303054e-13\\
40.3467497080097	1.27640953362373e-14\\
40.420055401106	8.37108160567368e-14\\
40.4933610942023	3.74110464829158e-14\\
40.567094455962	1.07150399664135e-13\\
40.6408278177218	1.28508315100362e-14\\
40.7145611794816	8.40959246684037e-14\\
40.7882945412414	3.76053355122252e-14\\
40.8624605638291	1.07608366661793e-13\\
40.9366265864169	1.29306287899311e-14\\
41.0107926090047	8.44567471514068e-14\\
41.0849586315925	3.77822773067749e-14\\
41.1595623935287	1.08038578083836e-13\\
41.2341661554649	1.30069566228741e-14\\
41.3087699174011	8.47898140587944e-14\\
41.3833736793372	3.79453413135167e-14\\
41.4584203505013	1.0843756448331e-13\\
41.5334670216654	1.30832844558171e-14\\
41.6085136928295	8.50968601140423e-14\\
41.6835603639935	3.80997317028786e-14\\
41.7590552049343	1.08805325860217e-13\\
41.8345500458751	1.31492039479042e-14\\
41.910044886816	8.53848242110544e-14\\
41.9855397277568	3.82437137513847e-14\\
42.0614880926214	1.09150535831937e-13\\
42.137436457486	1.32133887165153e-14\\
42.2133848223507	8.56519716263549e-14\\
42.2893331872153	3.83772874590349e-14\\
42.3657405267235	1.09466255504564e-13\\
42.4421478662318	1.32741040381745e-14\\
42.51855520574	8.58983023599436e-14\\
42.5949625452482	3.85021875493052e-14\\
42.6718344094257	1.09759423772005e-13\\
42.7487062736032	1.33296151894058e-14\\
42.8255781377807	8.61255511352965e-14\\
42.9024500019582	3.86166792987197e-14\\
42.9797920408555	1.10030040634257e-13\\
43.0571340797527	1.33816568936851e-14\\
43.1344761186499	8.63389221228417e-14\\
43.2118181575471	3.87259668777062e-14\\
43.2896361244367	1.10281575538274e-13\\
43.3674540913263	1.34302291510124e-14\\
43.4452720582158	8.65332111521511e-14\\
43.5230900251054	3.88265808393129e-14\\
43.6013897790167	1.10512293760578e-13\\
43.679689532928	1.34753319613878e-14\\
43.7579892868393	8.67153571171286e-14\\
43.8362890407505	3.89202559070156e-14\\
43.9150765502395	1.10723930024648e-13\\
43.9938640597285	1.35187000482873e-14\\
44.0726515692175	8.68818905708224e-14\\
44.1514390787065	3.90087268042905e-14\\
44.2307204224475	1.10919953777433e-13\\
44.3100017661886	1.35585986882347e-14\\
44.3892831099296	8.70380156836603e-14\\
44.4685644536707	3.90885240841854e-14\\
44.5483458244143	1.11100365018935e-13\\
44.6281271951579	1.35967626047062e-14\\
44.7079085659016	8.71785282852144e-14\\
44.7876899366452	3.91648519171284e-14\\
44.867977644264	1.11266898472628e-13\\
44.9482653518828	1.36331917977017e-14\\
45.0285530595016	8.73103672693887e-14\\
45.1088407671204	3.92325061326915e-14\\
45.1896412407794	1.11419554138514e-13\\
45.2704417144383	1.36661515437453e-14\\
45.3512421880972	8.74300631892311e-14\\
45.4320426617561	3.92984256247786e-14\\
45.5133624533588	1.11560066740068e-13\\
45.5946822449614	1.36991112897888e-14\\
45.676002036564	8.75393507682176e-14\\
45.7573218281666	3.93591409464378e-14\\
45.8391676159519	1.11686701553815e-13\\
45.9210134037371	1.37303363123564e-14\\
46.0028591915224	8.76399647298243e-14\\
46.0847049793077	3.9414652097669e-14\\
46.1670835705048	1.11802928026705e-13\\
46.2494621617019	1.37563571644961e-14\\
46.3318407528991	8.7733639797527e-14\\
46.4142193440962	3.94684285254243e-14\\
46.4971376778898	1.1190874615874e-13\\
46.5800560116834	1.37823780166357e-14\\
46.662974345477	8.78186412478499e-14\\
46.7458926792706	3.95152660592757e-14\\
46.8293578314074	1.12005890673395e-13\\
46.9128229835443	1.38083988687754e-14\\
46.9962881356811	8.78967038042688e-14\\
47.079753287818	3.95603688696511e-14\\
47.1637724728227	1.12094361570669e-13\\
47.2477916578275	1.38326849974391e-14\\
47.3318108428323	8.79660927433079e-14\\
47.4158300278371	3.96020022330745e-14\\
47.5004106040352	1.1217589357404e-13\\
47.5849911802333	1.38569711261027e-14\\
47.6695717564315	8.8032012235395e-14\\
47.7541523326296	3.9641900873022e-14\\
47.8393018049434	1.12247017236555e-13\\
47.9244512772573	1.38777878078145e-14\\
48.0096007495711	8.80892581101023e-14\\
48.094750221885	3.96765953425415e-14\\
48.1804762463428	1.12312936728642e-13\\
48.2662022708007	1.38986044895262e-14\\
48.3519282952586	8.81430345378575e-14\\
48.4376543197165	3.97112898120611e-14\\
48.5239647071256	1.1237018260335e-13\\
48.6102750945348	1.39176864477619e-14\\
48.6965854819439	8.81898720717089e-14\\
48.7828958693531	3.97425148346287e-14\\
48.8697985896823	1.12423959031105e-13\\
48.9567013100115	1.39367684059977e-14\\
49.0436040303407	8.82332401586083e-14\\
49.1305067506699	3.97720051337203e-14\\
49.2180099370951	1.12470796564956e-13\\
49.3055131235204	1.39549830024954e-14\\
49.3930163099456	8.82722714368178e-14\\
49.4805194963708	3.97980259858599e-14\\
49.5686314502397	1.1251242992838e-13\\
49.6567434041087	1.39723302372552e-14\\
49.7448553579776	8.83069659063374e-14\\
49.8329673118465	3.98231794762616e-14\\
49.9216965075321	1.12549726483113e-13\\
50.0104257032176	1.3988810110277e-14\\
50.0991548989032	8.83381909289049e-14\\
50.1878840945888	3.98474656049252e-14\\
50.2772391829867	1.12580951505681e-13\\
50.3665942713846	1.40044226215608e-14\\
50.4559493597825	8.83659465045206e-14\\
50.5453044481804	3.98700170101129e-14\\
50.6352942631106	1.12609574443034e-13\\
50.7252840780408	1.40191677711066e-14\\
50.8152738929709	8.83919673566602e-14\\
50.9052637079011	3.98908336918247e-14\\
50.9958972708063	1.1263299320996e-13\\
51.0865308337115	1.40347802823904e-14\\
51.1771643966166	8.84136514001099e-14\\
51.2677979595218	3.99107830117984e-14\\
51.3590844844193	1.1265467725341e-13\\
51.4503710093168	1.40486580701982e-14\\
51.5416575342142	8.84327333583457e-14\\
51.6329440591117	3.99289976082962e-14\\
51.7248929579717	1.1267202448817e-13\\
51.8168418568316	1.4062535858006e-14\\
51.9087907556916	8.84500805931054e-14\\
52.0007396545516	3.99472122047939e-14\\
52.0933605443554	1.12685034914239e-13\\
52.1859814341592	1.40746789223378e-14\\
52.278602323963	8.84656931043892e-14\\
52.3712232137668	3.99636920778157e-14\\
52.4645259207004	1.12697177978571e-13\\
52.557828627634	1.40885567101456e-14\\
52.6511313345676	8.84778361687211e-14\\
52.7444340415012	3.99793045890995e-14\\
52.8384286082082	1.12704984234213e-13\\
52.9324231749152	1.41006997744775e-14\\
53.0264177416223	8.84882445095769e-14\\
53.1204123083293	3.99940497386453e-14\\
53.2151089998621	1.12711923128117e-13\\
53.3098056913948	1.41137102005473e-14\\
53.4045023829275	8.84977854886948e-14\\
53.4991990744603	4.00087948881911e-14\\
53.5946083849179	1.12714525213331e-13\\
53.6900176953756	1.41249859031412e-14\\
53.7854270058332	8.85055917443367e-14\\
53.8808363162909	4.00226726759989e-14\\
53.9769689760186	1.12717127298545e-13\\
54.0731016357463	1.4137128967473e-14\\
54.169234295474	8.85099285530266e-14\\
54.2653669552017	4.00356831020687e-14\\
54.3622339372453	1.12716259936807e-13\\
54.4591009192888	1.41484046700668e-14\\
54.5559679013324	8.85151327234546e-14\\
54.652834883376	4.00478261664006e-14\\
54.7504474111939	1.12715392575069e-13\\
54.8480599390117	1.41605477343987e-14\\
54.9456724668296	8.85177348086685e-14\\
55.0432849946475	4.00599692307324e-14\\
55.1416545496904	1.12711055766379e-13\\
55.2400241047332	1.41718234369925e-14\\
55.338393659776	8.85194695321445e-14\\
55.4367632148189	4.00712449333263e-14\\
55.5359015449639	1.12705851595951e-13\\
55.635039875109	1.41826654587174e-14\\
55.734178205254	8.85207705747515e-14\\
55.8333165353991	4.00825206359201e-14\\
55.9332356621868	1.12699346382916e-13\\
56.0331547889744	1.41939411613112e-14\\
56.1330739157621	8.85203368938825e-14\\
56.2329930425498	4.0093362657645e-14\\
56.3337052708769	1.12691106446405e-13\\
56.434417499204	1.42052168639051e-14\\
56.5351297275311	8.85190358512755e-14\\
56.6358419558582	4.01037709985008e-14\\
56.7373598818851	1.12681999148156e-13\\
56.8388778079119	1.42164925664989e-14\\
56.9403957339388	8.85168674469305e-14\\
57.0419136599657	4.01141793393567e-14\\
57.1442501810399	1.1267202448817e-13\\
57.246586702114	1.42277682690928e-14\\
57.3489232231882	8.85138316808476e-14\\
57.4512597442624	4.01241539993435e-14\\
57.5544280683576	1.12660315104707e-13\\
57.6575963924529	1.42390439716866e-14\\
57.7607647165482	8.85099285530266e-14\\
57.8639330406435	4.01341286593304e-14\\
57.9679466963614	1.12647738359506e-13\\
58.0719603520793	1.42503196742805e-14\\
58.1759740077972	8.85055917443367e-14\\
58.2799876635152	4.01436696384483e-14\\
58.3848605103868	1.12634294252567e-13\\
58.4897333572585	1.42615953768743e-14\\
58.5946062041302	8.85003875739088e-14\\
58.6994790510019	4.01527769366972e-14\\
58.8052252907196	1.12620416464759e-13\\
58.9109715304374	1.42728710794682e-14\\
59.0167177701551	8.84943160417428e-14\\
59.1224640098728	4.0161884234946e-14\\
59.2290981971279	1.12605237634344e-13\\
59.3357323843829	1.4284580462931e-14\\
59.442366571638	8.84878108287079e-14\\
59.549000758893	4.01709915331949e-14\\
59.6617505691698	1.56804690759627e-13\\
59.7745003794465	1.46097543785029e-13\\
59.8872501897232	4.75032339852799e-14\\
60	3.1936259192733e-15\\
};
\addlegendentry{Infetti}

\end{axis}
\end{tikzpicture}%}}
\caption[Errori assoluti relativi al grafo~\ref{fig::3nodi} tra modello esatto e cut-vertex] {Errore assoluto (in scala logaritmica) tra la soluzione del problema di Cauchy~\eqref{3nodi} e quello ottenuto utilizzando il Teorema~\ref{th_cut-vertex}.\\
Per ottenere i grafici abbiamo risolto numericamente, con una tolleranza di $1e-9$ i due problemi di Cauchy con condizioni iniziali di stati puri~\eqref{statipuri}. Abbiamo inoltre supposto l'indipendenza statica di tali condizioni iniziali.\\
Per la sperimentazione abbiamo usato come parametri $\tau = 0.3$ e $\gamma = 0.1$. }
\label{fig::errori3nodi}
\end{figure}

\begin{figure}[!htb]
	\centering
	\subfloat[][Nodo 1.]
	{\resizebox{0.45\textwidth}{!}{% This file was created by matlab2tikz.
%
\definecolor{mycolor1}{rgb}{0.00000,0.44700,0.74100}%
\definecolor{mycolor2}{rgb}{0.85000,0.32500,0.09800}%
%
\begin{tikzpicture}

\begin{axis}[%
width=6.028in,
height=4.754in,
at={(1.011in,0.642in)},
scale only axis,
xmin=0,
xmax=60,
xlabel style={font=\color{white!15!black}},
xlabel={T},
ymode=log,
ymin=1e-16,
ymax=1e-06,
yminorticks=true,
ylabel style={font=\color{white!15!black}},
ylabel={Errore assoluto},
axis background/.style={fill=white},
legend style={legend cell align=left, align=left, draw=white!15!black}
]
\addplot [color=mycolor1, line width=2.0pt]
  table[row sep=crcr]{%
0	0\\
0.000167459095433972	0\\
0.000334918190867944	0\\
0.000502377286301916	0\\
0.000669836381735888	0\\
0.00150713185890575	0\\
0.00234442733607561	0\\
0.00318172281324547	0\\
0.00401901829041533	0\\
0.00820549567626463	0\\
0.0123919730621139	0\\
0.0165784504479632	0\\
0.0207649278338125	0\\
0.041697314763059	0\\
0.0626297016923055	0\\
0.083562088621552	0\\
0.104494475550799	0\\
0.209156410197031	0\\
0.313818344843264	0\\
0.418480279489496	0\\
0.523142214135729	0\\
0.693500300195606	0\\
0.863858386255483	0\\
1.03421647231536	0\\
1.20457455837524	0\\
1.44051447786383	0\\
1.67645439735243	0\\
1.91239431684103	0\\
2.14833423632963	0\\
2.44722092525848	0\\
2.74610761418733	0\\
3.04499430311618	0\\
3.34388099204503	0\\
3.7078444355742	0\\
4.07180787910338	0\\
4.43577132263256	0\\
4.79973476616174	0\\
5.20292074647974	0\\
5.60610672679773	0\\
6.00929270711573	0\\
6.41247868743373	0\\
6.85758807700814	0\\
7.30269746658256	0\\
7.74780685615697	0\\
8.19291624573139	0\\
8.68220109096914	0\\
9.17148593620689	0\\
9.66077078144463	0\\
10.1500556266824	0\\
10.6416653734136	0\\
11.1332751201447	0\\
11.6248848668759	0\\
12.1164946136071	0\\
12.6079732666127	0\\
13.0994519196183	0\\
13.5909305726239	0\\
14.0824092256295	0\\
14.573895267778	0\\
15.0653813099266	0\\
15.5568673520752	0\\
16.0483533942238	0\\
16.5398390198718	0\\
17.0313246455198	0\\
17.5228102711679	0\\
18.0142958968159	0\\
18.5057815459406	0\\
18.9972671950653	0\\
19.48875284419	0\\
19.9802384933148	0\\
20.5017696853463	0\\
21.0233008773778	0\\
21.5448320694093	0\\
22.0663632614408	0\\
22.6746017709939	0\\
23.282840280547	0\\
23.8910787901001	0\\
24.4993172996532	0\\
25.0869400675273	0\\
25.6745628354014	0\\
26.2621856032755	0\\
26.8498083711495	0\\
27.5306427936611	0\\
28.2114772161728	0\\
28.8923116386844	0\\
29.573146061196	0\\
30.3571240725376	0\\
31.1411020838792	0\\
31.9250800952207	0\\
32.7090581065623	0\\
33.6393150845161	0\\
34.5695720624699	0\\
35.4998290404237	0\\
36.4300860183775	0\\
37.5675806203718	0\\
38.7050752223662	0\\
39.8425698243606	0\\
40.980064426355	0\\
42.4285780340909	0\\
43.8770916418269	0\\
45.3256052495628	0\\
46.7741188572988	0\\
48.2741188572988	0\\
49.7741188572988	0\\
51.2741188572988	0\\
52.7741188572988	0\\
54.2741188572988	0\\
55.7741188572988	0\\
57.2741188572988	0\\
58.7741188572988	0\\
59.0805891429741	0\\
59.3870594286494	0\\
59.6935297143247	0\\
60	0\\
};
\addlegendentry{Suscettibili}

\addplot [color=mycolor2, line width=2.0pt]
  table[row sep=crcr]{%
0	0\\
0.000167459095433972	0\\
0.000334918190867944	0\\
0.000502377286301916	0\\
0.000669836381735888	0\\
0.00150713185890575	1.11022302462516e-16\\
0.00234442733607561	1.11022302462516e-16\\
0.00318172281324547	0\\
0.00401901829041533	0\\
0.00820549567626463	0\\
0.0123919730621139	1.11022302462516e-16\\
0.0165784504479632	0\\
0.0207649278338125	1.11022302462516e-16\\
0.041697314763059	7.7715611723761e-15\\
0.0626297016923055	4.10782519111308e-14\\
0.083562088621552	5.25135490647699e-14\\
0.104494475550799	2.80886425230165e-14\\
0.209156410197031	4.80390172086231e-11\\
0.313818344843264	6.88210599619765e-11\\
0.418480279489496	2.86723977893644e-11\\
0.523142214135729	1.64657176782157e-12\\
0.693500300195606	5.35829491887796e-10\\
0.863858386255483	7.71713581926292e-10\\
1.03421647231536	3.19344661825482e-10\\
1.20457455837524	2.90805157732166e-11\\
1.44051447786383	2.57873844500267e-09\\
1.67645439735243	3.7373094619042e-09\\
1.91239431684103	1.52142931586496e-09\\
2.14833423632963	2.16112794326762e-10\\
2.44722092525848	7.67736607620861e-09\\
2.74610761418733	1.12278264463583e-08\\
3.04499430311618	4.47335857245434e-09\\
3.34388099204503	9.06798192090719e-10\\
3.7078444355742	1.82254300540308e-08\\
4.07180787910338	2.69290317822524e-08\\
4.43577132263256	1.04336603801158e-08\\
4.79973476616174	2.90191126683936e-09\\
5.20292074647974	2.50552135705462e-08\\
5.60610672679773	3.78949197576972e-08\\
6.00929270711573	1.37900147967329e-08\\
6.41247868743373	5.81769532548293e-09\\
6.85758807700814	3.38010213152273e-08\\
7.30269746658256	5.21488214832999e-08\\
7.74780685615697	1.79682339562781e-08\\
8.19291624573139	1.00885785836446e-08\\
8.68220109096914	4.39347014324909e-08\\
9.17148593620689	6.91607205949296e-08\\
9.66077078144463	2.24840371876311e-08\\
10.1500556266824	1.61344422822118e-08\\
10.6416653734136	2.97270874027866e-08\\
11.1332751201447	5.13094594589703e-08\\
11.6248848668759	1.22837813809618e-08\\
12.1164946136071	1.99422006130767e-08\\
12.6079732666127	1.80805697080899e-08\\
13.0994519196183	3.61536027226883e-08\\
13.5909305726239	4.37185215806934e-09\\
14.0824092256295	2.18967639298295e-08\\
14.573895267778	9.62376503488294e-09\\
15.0653813099266	2.47916549234173e-08\\
15.5568673520752	1.22431365046083e-09\\
16.0483533942238	2.24909054991063e-08\\
16.5398390198718	3.75051509293733e-09\\
17.0313246455198	1.59789604048388e-08\\
17.5228102711679	4.72748293289804e-09\\
18.0142958968159	2.22500514679425e-08\\
18.5057815459406	7.13090975157371e-10\\
18.9972671950653	9.90279999668608e-09\\
19.48875284419	7.52456003061397e-09\\
19.9802384933148	2.12988167103134e-08\\
20.5017696853463	2.19746107421592e-09\\
21.0233008773778	1.41736938852333e-08\\
21.5448320694093	5.42407337100759e-09\\
22.0663632614408	2.1540967024003e-08\\
22.6746017709939	2.09836452658152e-08\\
23.282840280547	4.04126624525825e-08\\
23.8910787901001	6.50092291110127e-09\\
24.4993172996532	2.52062478806048e-08\\
25.0869400675273	5.15288625657639e-09\\
25.6745628354014	1.71336633492114e-08\\
26.2621856032755	3.89340017170614e-09\\
26.8498083711495	2.43503433250414e-08\\
27.5306427936611	2.4158517852868e-08\\
28.2114772161728	4.60238832739046e-08\\
28.8923116386844	7.47492450575704e-09\\
29.573146061196	2.76710140287295e-08\\
30.3571240725376	4.81070906305225e-08\\
31.1411020838792	8.37468156489329e-08\\
31.9250800952207	1.94369943820605e-08\\
32.7090581065623	3.6043385769069e-08\\
33.6393150845161	9.62769395249086e-08\\
34.5695720624699	1.62277031971747e-07\\
35.4998290404237	4.43348593774651e-08\\
36.4300860183775	6.01118896499664e-08\\
37.5675806203718	2.04702823289327e-07\\
38.7050752223662	3.37110524235157e-07\\
39.8425698243606	9.45182538764078e-08\\
40.980064426355	1.22201772603803e-07\\
42.4285780340909	4.90217640885041e-07\\
43.8770916418269	7.99717012497231e-07\\
45.3256052495628	2.17947354899661e-07\\
46.7741188572988	3.10377581509355e-07\\
48.2741188572988	1.35916788487206e-07\\
49.7741188572988	3.73057690932453e-07\\
51.2741188572988	9.51955837942553e-10\\
52.7741188572988	3.38807476163959e-07\\
54.2741188572988	7.04903680833965e-08\\
55.7741188572988	7.97955118660675e-08\\
57.2741188572988	1.07069940925505e-07\\
58.7741188572988	2.78551773038975e-07\\
59.0805891429741	2.70117696943468e-07\\
59.3870594286494	2.61887474482456e-07\\
59.6935297143247	2.53957459244673e-07\\
60	2.46312058303864e-07\\
};
\addlegendentry{Infetti}

\end{axis}
\end{tikzpicture}%}}
 \quad 
\subfloat[][Nodo 2.]
{\resizebox{0.45\textwidth}{!}{ % This file was created by matlab2tikz.
%
\definecolor{mycolor1}{rgb}{0.00000,0.44700,0.74100}%
\definecolor{mycolor2}{rgb}{0.85000,0.32500,0.09800}%
%
\begin{tikzpicture}

\begin{axis}[%
width=0.39\columnwidth,
height=1.7in,
at={(1.011in,0.642in)},
scale only axis,
xmin=0,
xmax=60,
xlabel style={font=\color{white!15!black}},
xlabel={T},
ymode=log,
ymin=1e-12,
ymax=1,
yminorticks=true,
axis background/.style={fill=white},
legend style={at={(axis cs:25,1e-12)},anchor=south west,legend cell align=left, align=left, draw=none,fill=none}
]
\addplot [color=mycolor1, line width=2.0pt]
  table[row sep=crcr]{%
0	0\\
0.00199053585276749	4.72822891950386e-11\\
0.00398107170553497	3.77944675555852e-10\\
0.00597160755830246	1.27449939490987e-09\\
0.00796214341106994	3.01851699191502e-09\\
0.0143083073603051	1.74709008371465e-08\\
0.0206544713095403	5.24132313106662e-08\\
0.0270006352587754	1.16779558134716e-07\\
0.0333467992080106	2.19409007073956e-07\\
0.0396935645593046	3.69062722493929e-07\\
0.0460403299105986	5.74385673579592e-07\\
0.0523870952618926	8.43936890593255e-07\\
0.0587338606131866	1.18618281097493e-06\\
0.0650961133419323	1.61063723613797e-06\\
0.0714583660706779	2.12490150108824e-06\\
0.0778206187994235	2.73722807864196e-06\\
0.0841828715281692	3.45577822358223e-06\\
0.0905606411731996	4.29080110375679e-06\\
0.0969384108182301	5.24871821316886e-06\\
0.103316180463261	6.33748045264504e-06\\
0.109693950108291	7.56494936282603e-06\\
0.116087318034554	8.94244453675253e-06\\
0.122480685960817	1.04748798530618e-05\\
0.12887405388708	1.21699099100425e-05\\
0.135267421813343	1.40351019444207e-05\\
0.141676469438281	1.60831708501119e-05\\
0.14808551706322	1.83172041459123e-05\\
0.154494564688158	2.07445654245575e-05\\
0.160903612313097	2.33725326126422e-05\\
0.167328422015208	2.62155347408033e-05\\
0.173753231717319	2.92745205150435e-05\\
0.18017804141943	3.25565675532991e-05\\
0.186602851121541	3.60686695501933e-05\\
0.193043505744106	3.98272816455325e-05\\
0.199484160366671	4.38309080333932e-05\\
0.205924814989236	4.80863452054781e-05\\
0.212365469611801	5.26003077412351e-05\\
0.218822053394839	5.73915826932447e-05\\
0.225278637177877	6.24559301659611e-05\\
0.231735220960914	6.77998703031424e-05\\
0.238191804743952	7.34298434037495e-05\\
0.244664401671674	7.93672669157086e-05\\
0.251136998599397	8.56048643187446e-05\\
0.25760959552712	9.21488843632634e-05\\
0.264082192454842	9.90054977327848e-05\\
0.270570887963067	0.000106199045963651\\
0.277059583471292	0.000113718927941764\\
0.283548278979517	0.000121571126253994\\
0.290036974487742	0.000129761546579177\\
0.296541854059766	0.00013831774036821\\
0.303046733631789	0.000147225501619941\\
0.309551613203813	0.000156490551564303\\
0.316056492775836	0.000166118535729232\\
0.322577643612098	0.00017614049728254\\
0.32909879444836	0.000186538349080001\\
0.335619945284621	0.000197317555455823\\
0.342141096120883	0.000208483506871193\\
0.348678605036902	0.000220071010902378\\
0.35521611395292	0.000232057831597299\\
0.361753622868939	0.00024444918111266\\
0.368291131784958	0.000257250199383385\\
0.374845086704082	0.000270499727724416\\
0.381399041623206	0.00028417111749901\\
0.387952996542331	0.000298269333238732\\
0.394506951461455	0.000312799268693098\\
0.401077441531393	0.000327804063007919\\
0.407647931601331	0.000343252396433291\\
0.414218421671268	0.000359148990142888\\
0.420788911741206	0.000375498496302806\\
0.427376026319146	0.000392348607299642\\
0.433963140897085	0.000409663082203471\\
0.440550255475025	0.000427446403023835\\
0.447137370052964	0.000445702984656293\\
0.453741199367022	0.000464485323898645\\
0.46034502868108	0.000483752012792604\\
0.466948857995137	0.000503507298574957\\
0.473552687309195	0.000523755362975775\\
0.480173322720616	0.000544553753699883\\
0.486793958132037	0.000565855654234726\\
0.493414593543458	0.000587665081802968\\
0.500035228954879	0.0006099859890486\\
0.506672761762064	0.000632881211263481\\
0.513310294569248	0.000656298291458857\\
0.519947827376433	0.000680241021072225\\
0.526585360183617	0.000704713128104495\\
0.533239883483442	0.000729782972698212\\
0.539894406783268	0.00075539222494958\\
0.546548930083094	0.000781544454499739\\
0.553203453382919	0.000808243169273415\\
0.559875060934079	0.00083556248237171\\
0.566546668485238	0.000863437967349778\\
0.573218276036397	0.00089187297632487\\
0.579889883587557	0.000920870801021101\\
0.586578669168294	0.000950511527605946\\
0.593267454749031	0.000980724417813827\\
0.599956240329768	0.0010115126101703\\
0.606645025910505	0.00104287918426604\\
0.613351084619436	0.00107491041909491\\
0.620057143328367	0.00110752904983069\\
0.626763202037298	0.00114073800517311\\
0.633469260746229	0.0011745401566261\\
0.640192689007476	0.00120902819054414\\
0.646916117268722	0.00124411810593295\\
0.653639545529968	0.00127981262549703\\
0.660362973791214	0.00131611441636481\\
0.667103867904104	0.00135312277430977\\
0.673844762016994	0.00139074676442197\\
0.680585656129883	0.001428988907758\\
0.687326550242773	0.00146785167071783\\
0.694085008124096	0.00150744116418633\\
0.700843466005419	0.00154765931812961\\
0.707601923886742	0.00158850845596481\\
0.714360381768066	0.00162999084734761\\
0.721136501534817	0.00167221961052\\
0.727912621301569	0.00171508935215747\\
0.73468874106832	0.00175860220161017\\
0.741464860835071	0.00180276023606274\\
0.748258742408364	0.00184768377820332\\
0.755052623981657	0.00189325991848066\\
0.761846505554949	0.00193949059609655\\
0.768640387128242	0.00198637769928112\\
0.775452130474897	0.00203404893636971\\
0.782263873821551	0.00208238370474401\\
0.789075617168206	0.00213138375701927\\
0.79588736051486	0.00218105079621356\\
0.80271706685068	0.00223152009995087\\
0.809546773186499	0.00228266319349213\\
0.816376479522319	0.00233448164627303\\
0.823206185858138	0.00238697697976142\\
0.830053957332632	0.00244029221711128\\
0.836901728807126	0.00249429083962849\\
0.84374950028162	0.00254897423714595\\
0.850597271756113	0.00260434375292817\\
0.857463211448599	0.00266055032720391\\
0.864329151141086	0.00271744922975792\\
0.871195090833572	0.00277504167494114\\
0.878061030526058	0.00283332883117482\\
0.884945241937476	0.00289246971758517\\
0.891829453348894	0.0029523112344777\\
0.898713664760312	0.00301285442415544\\
0.90559787617173	0.00307410028411981\\
0.912500464439821	0.00313621607958747\\
0.919403052707912	0.00319904017797368\\
0.926305640976003	0.00326257345298819\\
0.933208229244093	0.00332681673491797\\
0.940129300503941	0.00339194569256307\\
0.947050371763788	0.00345779000677982\\
0.953971443023636	0.00352435038619958\\
0.960892514283483	0.00359162749709674\\
0.967832174888143	0.00365980555720646\\
0.974771835492802	0.00372870541963466\\
0.981711496097461	0.0037983276310869\\
0.988651156702121	0.00386867269738322\\
0.995609514344775	0.00393993353556632\\
1.00256787198743	0.00401192202483291\\
1.00952622963008	0.0040846385531268\\
1.01648458727274	0.0041580834689684\\
1.0234617510246	0.00423245853554799\\
1.03043891477646	0.00430756651521624\\
1.03741607852832	0.0043834076409297\\
1.04439324228018	0.00445998210696563\\
1.05138932194451	0.00453750065781788\\
1.05838540160883	0.00461575680741744\\
1.06538148127316	0.00469475063720914\\
1.07237756093749	0.00477448219072463\\
1.07939266704796	0.00485517132368085\\
1.08640777315843	0.00493660217506775\\
1.0934228792689	0.00501877467788914\\
1.10043798537936	0.00510168872863437\\
1.10747222983544	0.00518557342701387\\
1.11450647429151	0.00527020340776818\\
1.12154071874758	0.00535557845882462\\
1.12857496320366	0.00544169833256103\\
1.13562845894337	0.00552880149777069\\
1.14268195468308	0.00561665296391944\\
1.14973545042278	0.00570525237680275\\
1.15678894616249	0.0057945993480305\\
1.16386180727142	0.0058849418365523\\
1.17093466838034	0.00597603510942912\\
1.17800752948927	0.0060678786732552\\
1.18508039059819	0.0061604720018259\\
1.19217273186328	0.00625407265530609\\
1.19926507312837	0.00634842605080843\\
1.20635741439346	0.00644353155930089\\
1.21344975565855	0.00653938851960778\\
1.22056169301535	0.00663626420506114\\
1.22767363037216	0.00673389407429581\\
1.23478556772896	0.00683227736589764\\
1.24189750508577	0.00693141328707625\\
1.2490291561515	0.00703157893921569\\
1.25616080721723	0.00713249971089791\\
1.26329245828296	0.00723417471125032\\
1.27042410934869	0.00733660301930039\\
1.27757559222201	0.00744007165860083\\
1.28472707509533	0.00754429585668492\\
1.29187855796865	0.00764927459634479\\
1.29903004084196	0.00775500683119379\\
1.30620147471517	0.0078617896054396\\
1.31337290858837	0.00796932789098526\\
1.32054434246157	0.00807762054701244\\
1.32771577633477	0.00818666640486077\\
1.33490728210935	0.00829677263207496\\
1.34209878788392	0.0084076338460749\\
1.3492902936585	0.00851924878526078\\
1.35648179943308	0.00863161616140662\\
1.36369349872651	0.0087450533491944\\
1.37090519801994	0.00885924453094378\\
1.37811689731337	0.00897418832781793\\
1.3853285966068	0.00908988333474992\\
1.39256061231053	0.00920665722113512\\
1.39979262801426	0.00932418364988075\\
1.40702464371799	0.00944246112768299\\
1.41425665942172	0.00956148813608371\\
1.42150911582444	0.00968160272461094\\
1.42876157222716	0.00980246795425554\\
1.43601402862988	0.00992408222022656\\
1.44326648503259	0.0100464438935206\\
1.45053950730217	0.0101699014770988\\
1.45781252957175	0.0102941073601241\\
1.46508555184133	0.0104190598290589\\
1.47235857411091	0.010544757147233\\
1.47965228878223	0.0106715583478393\\
1.48694600345356	0.0107991050751501\\
1.49423971812488	0.0109273955094278\\
1.5015334327962	0.0110564278090949\\
1.50884796768823	0.0111865716087174\\
1.51616250258027	0.0113174577399483\\
1.5234770374723	0.011449084279919\\
1.53079157236433	0.0115814492846504\\
1.53812705668821	0.0117149330591144\\
1.54546254101209	0.0118491555563934\\
1.55279802533598	0.0119841147536394\\
1.56013350965986	0.0121198086074044\\
1.56749007329995	0.0122566281444155\\
1.57484663694003	0.0123941823906396\\
1.58220320058012	0.0125324692258055\\
1.5895597642202	0.0126714865102174\\
1.59693753927215	0.0128116360685345\\
1.60431531432409	0.0129525159265449\\
1.61169308937604	0.0130941238693821\\
1.61907086442798	0.0132364576635791\\
1.6264699837717	0.0133799299776235\\
1.63386910311542	0.0135241277948049\\
1.64126822245913	0.0136690488080816\\
1.64866734180285	0.0138146906930345\\
1.65608793943564	0.0139614770089892\\
1.66350853706843	0.014108983653109\\
1.67092913470122	0.0142572082288314\\
1.67834973233401	0.0144061483231787\\
1.68579194412116	0.0145562384446148\\
1.69323415590832	0.0147070433489237\\
1.70067636769548	0.0148585605532955\\
1.70811857948264	0.0150107875587788\\
1.71558254216844	0.015164169857086\\
1.72304650485425	0.0153182610313443\\
1.73051046754005	0.0154730585148726\\
1.73797443022586	0.0156285597259995\\
1.74546028230335	0.0157852211876832\\
1.75294613438084	0.0159425852651474\\
1.76043198645833	0.0161006493105996\\
1.76791783853582	0.0162594106619999\\
1.77542572012108	0.0164193369160391\\
1.78293360170633	0.0165799591811354\\
1.79044148329159	0.0167412747307153\\
1.79794936487685	0.016903280825097\\
1.80547941694215	0.0170664561559686\\
1.81300946900745	0.0172303205569645\\
1.82053952107275	0.0173948712251832\\
1.82806957313806	0.017560105345605\\
1.8356219382731	0.0177265127388435\\
1.84317430340815	0.017893601932743\\
1.85072666854319	0.0180613700512589\\
1.85827903367824	0.01822981420645\\
1.86585385599785	0.0183994353756234\\
1.87342867831746	0.0185697307559302\\
1.88100350063707	0.0187406974004765\\
1.88857832295669	0.0189123323515537\\
1.89617574793629	0.0190851477635212\\
1.90377317291589	0.019258629485032\\
1.9113705978955	0.0194327745009877\\
1.9189680228751	0.0196075797861758\\
1.92658819822989	0.0197835687095032\\
1.93420837358469	0.0199602157372151\\
1.94182854893948	0.0201375177882362\\
1.94944872429427	0.0203154717724616\\
1.95709179801755	0.0204946122608529\\
1.96473487174083	0.020674402352843\\
1.97237794546411	0.0208548389037447\\
1.98002101918739	0.0210359187607561\\
1.98768714199827	0.0212181877380107\\
1.99535326480915	0.0214010975300117\\
2.00301938762003	0.0215846449315334\\
2.01068551043091	0.0217688267294297\\
2.0183748340895	0.0219541999784573\\
2.02606415774808	0.022140204973401\\
2.03375348140667	0.022326838450677\\
2.04144280506525	0.0225140971398298\\
2.0491554826466	0.0227025493363423\\
2.05686816022795	0.0228916239376867\\
2.06458083780929	0.0230813176244444\\
2.07229351539064	0.0232716270709995\\
2.08002970237887	0.0234631318374631\\
2.08776588936711	0.0236552494036716\\
2.09550207635534	0.0238479763964767\\
2.10323826334358	0.024041309437584\\
2.11099811595816	0.0242358393298212\\
2.11875796857274	0.024430972160337\\
2.12651782118732	0.0246267045046499\\
2.1342776738019	0.0248230329338582\\
2.14206135075448	0.0250205595120668\\
2.14984502770706	0.0252186789180009\\
2.15762870465964	0.0254173876787173\\
2.16541238161222	0.0256166823171345\\
2.17322004315878	0.025817176149547\\
2.18102770470534	0.0260182524579193\\
2.1888353662519	0.0262199077229672\\
2.19664302779846	0.0264221384221988\\
2.2044748353169	0.0266255691007227\\
2.21230664283533	0.026829571669721\\
2.22013845035377	0.0270341425659003\\
2.2279702578722	0.0272392782235011\\
2.23582637510447	0.0274456144217996\\
2.24368249233675	0.0276525116993515\\
2.25153860956902	0.0278599664508489\\
2.25939472680129	0.0280679750695172\\
2.26727531955162	0.0282771845626372\\
2.27515591230195	0.0284869441052793\\
2.28303650505228	0.0286972500527406\\
2.29091709780261	0.0289080987592083\\
2.29882233260671	0.02912014841476\\
2.30672756741082	0.0293327368787051\\
2.31463280221492	0.0295458604693274\\
2.32253803701902	0.029759515504454\\
2.33046808291417	0.0299743713551774\\
2.33839812880931	0.0301897545693027\\
2.34632817470446	0.0304056614303421\\
2.3542582205996	0.0306220882220092\\
2.36221324911394	0.030839715492141\\
2.37016827762829	0.0310578584844277\\
2.37812330614264	0.0312765134496492\\
2.38607833465698	0.0314956766396557\\
2.39405851771369	0.0317160397132256\\
2.40203870077039	0.0319369066787887\\
2.4100188838271	0.0321582737563975\\
2.4179990668838	0.0323801371680136\\
2.42600457974877	0.0326031996953267\\
2.43401009261373	0.0328267541023739\\
2.4420156054787	0.0330507965812531\\
2.45002111834367	0.0332753233260746\\
2.45805213720175	0.0335010481825012\\
2.46608315605982	0.0337272527315514\\
2.4741141749179	0.0339539331393135\\
2.48214519377598	0.0341810855747832\\
2.4902018978345	0.0344094349443757\\
2.49825860189303	0.0346382516518453\\
2.50631530595155	0.034867531839474\\
2.51437201001007	0.0350972716530383\\
2.52245457948208	0.035328206996682\\
2.53053714895408	0.035559597163108\\
2.53861971842609	0.0357914382726259\\
2.54670228789809	0.0360237264499312\\
2.55481090545363	0.0362572085709146\\
2.56291952300918	0.0364911328461427\\
2.57102814056472	0.0367254953762767\\
2.57913675812026	0.0369602922667346\\
2.58727160888194	0.0371962813361075\\
2.59540645964362	0.037432699744439\\
2.6035413104053	0.0376695435750327\\
2.61167616116698	0.0379068089164685\\
2.61983743200028	0.0381452644768974\\
2.62799870283359	0.0383841364212313\\
2.63615997366689	0.0386234208174385\\
2.6443212445002	0.0388631137393566\\
2.65250912437636	0.0391039947394317\\
2.66069700425252	0.0393452790356302\\
2.66888488412869	0.0395869626824469\\
2.67707276400485	0.0398290417410239\\
2.6852874441124	0.0400723065622287\\
2.69350212421995	0.0403159614660707\\
2.7017168043275	0.0405600024954432\\
2.70993148443506	0.0408044257005779\\
2.71817315820996	0.0410500321826525\\
2.72641483198486	0.0412960154143174\\
2.73465650575977	0.0415423714294271\\
2.74289817953467	0.0417890962692561\\
2.75116704265512	0.042037001733466\\
2.75943590577556	0.0422852705015698\\
2.767704768896	0.0425338986001389\\
2.77597363201644	0.0427828820639315\\
2.78426988288509	0.0430330433509364\\
2.79256613375373	0.0432835543901742\\
2.80086238462237	0.0435344112028787\\
2.80915863549101	0.0437856098190686\\
2.81748247408502	0.0440379832771504\\
2.82580631267902	0.0442906928366789\\
2.83413015127303	0.0445437345152252\\
2.84245398986704	0.0447971043399036\\
2.85080561880438	0.0450516458801872\\
2.85915724774172	0.0453065097781193\\
2.86750887667906	0.0455616920499756\\
2.8758605056164	0.0458171887216598\\
2.88424013037944	0.0460738538480507\\
2.89261975514247	0.0463308275017519\\
2.90099937990551	0.0465881056995557\\
2.90937900466855	0.0468456844685743\\
2.91778683316272	0.0471044282868916\\
2.9261946616569	0.0473634667220666\\
2.93460249015107	0.0476227957932966\\
2.94301031864525	0.0478824115305733\\
2.95144656047954	0.0481431887481795\\
2.95988280231384	0.0484042465986474\\
2.96831904414814	0.0486655811051785\\
2.97675528598243	0.0489271883024884\\
2.98522015378253	0.0491899532906303\\
2.99368502158263	0.0494529848602571\\
3.00214988938273	0.0497162790406122\\
3.01061475718283	0.0499798318726972\\
3.01910846625213	0.050244538678524\\
3.02760217532142	0.0505094979529791\\
3.03609588439071	0.0507747057333187\\
3.04458959346	0.0510401580690164\\
3.05311236220025	0.051306760450144\\
3.0616351309405	0.0515736011318623\\
3.07015789968076	0.0518406761611697\\
3.07868066842101	0.0521079815977473\\
3.0872327162858	0.0523764329793056\\
3.0957847641506	0.052645108444444\\
3.10433681201539	0.0529140040514635\\
3.11288885988019	0.053183115871995\\
3.12147041092284	0.0534533694784542\\
3.13005196196549	0.0537238329086784\\
3.13863351300815	0.0539945022340304\\
3.1472150640508	0.0542653735395814\\
3.15582634421903	0.0545373823316502\\
3.16443762438727	0.0548095866502615\\
3.1730489045555	0.0550819825819125\\
3.18166018472374	0.0553545662270442\\
3.19030142291735	0.0556282829550422\\
3.19894266111096	0.055902180881045\\
3.20758389930457	0.0561762561082388\\
3.21622513749818	0.0564505047542398\\
3.22489656499851	0.0567258819601445\\
3.23356799249884	0.0570014260101452\\
3.24223941999916	0.0572771330256309\\
3.25091084749949	0.0575529991429909\\
3.25961269965928	0.0578299892338237\\
3.26831455181906	0.0581071317954002\\
3.27701640397885	0.0583844229688856\\
3.28571825613863	0.0586618589108707\\
3.29445076992401	0.0589404141010602\\
3.30318328370939	0.0592191073742609\\
3.31191579749477	0.059497934893471\\
3.32064831128015	0.0597768928372094\\
3.32941172759282	0.0600569652416231\\
3.33817514390549	0.0603371613327834\\
3.34693856021816	0.0606174772969342\\
3.35570197653083	0.0608979093363369\\
3.36449653919211	0.0611794509570566\\
3.37329110185339	0.0614611018653646\\
3.38208566451467	0.0617428582721995\\
3.39088022717596	0.0620247164050177\\
3.39970618282859	0.0623076791467464\\
3.40853213848122	0.062590736779627\\
3.41735809413385	0.062873885540737\\
3.42618404978648	0.063157121684085\\
3.43504164855909	0.0634414573946284\\
3.44389924733171	0.0637258736074388\\
3.45275684610433	0.0640103665877024\\
3.46161444487695	0.0642949326175589\\
3.47050394016004	0.064580593098826\\
3.47939343544313	0.0648663197065297\\
3.48828293072622	0.0651521087352619\\
3.49717242600932	0.0654379564970489\\
3.50609407425931	0.0657248935165722\\
3.5150157225093	0.0660118823050906\\
3.52393737075929	0.0662989191879634\\
3.53285901900928	0.0665860005084256\\
3.54181308032811	0.0668741658354056\\
3.55076714164694	0.067162368597643\\
3.55972120296577	0.0674506051526089\\
3.56867526428461	0.0677388718760176\\
3.57766200230417	0.068028217295612\\
3.58664874032374	0.0683175858449525\\
3.5956354783433	0.0686069739154745\\
3.60462221636287	0.0688963779168483\\
3.61364189770245	0.0691868552296099\\
3.62266157904204	0.0694773414000153\\
3.63168126038162	0.069767832854653\\
3.64070094172121	0.0700583260387891\\
3.64975383695641	0.0703498871084022\\
3.65880673219161	0.0706414428019956\\
3.66785962742681	0.0709329895825671\\
3.67691252266202	0.0712245239321833\\
3.6859989062531	0.071517120699339\\
3.69508528984418	0.0718096979002033\\
3.70417167343526	0.0721022520354508\\
3.71325805702634	0.0723947796251209\\
3.72237820615531	0.0726883640867527\\
3.73149835528429	0.0729819148394279\\
3.74061850441326	0.0732754284232117\\
3.74973865354223	0.073568901397528\\
3.75889285078234	0.0738634257083307\\
3.76804704802246	0.0741579022199765\\
3.77720124526257	0.0744523275130012\\
3.78635544250268	0.0747466981876866\\
3.79554397276389	0.0750421145769516\\
3.80473250302509	0.0753374691340412\\
3.8139210332863	0.0756327584811015\\
3.8231095635475	0.0759279792603705\\
3.83233271658159	0.0762242401274958\\
3.84155586961568	0.0765204251911524\\
3.85077902264978	0.0768165311163119\\
3.86000217568387	0.0771125545882374\\
3.86926024490616	0.0774096124801101\\
3.87851831412845	0.0777065806628467\\
3.88777638335074	0.0780034558457889\\
3.89703445257303	0.0783002347585971\\
3.90632773542166	0.0785980423952645\\
3.91562101827029	0.0788957464873065\\
3.92491430111892	0.0791933437894064\\
3.93420758396755	0.0794908310768904\\
3.94353638235509	0.0797893413797795\\
3.95286518074264	0.0800877343770892\\
3.96219397913019	0.0803860068698677\\
3.97152277751774	0.0806841556801027\\
3.9808873974673	0.0809833217738621\\
3.99025201741687	0.0812823568797251\\
3.99961663736644	0.0815812578462993\\
4.008981257316	0.0818800215432015\\
4.01838200905393	0.0821797967725187\\
4.02778276079185	0.082479427413884\\
4.03718351252977	0.0827789103647844\\
4.0465842642677	0.0830782425438189\\
4.05602146260214	0.0833785804980648\\
4.06545866093658	0.0836787603507655\\
4.07489585927102	0.0839787790491616\\
4.08433305760547	0.0842786335618616\\
4.09380702196714	0.0845794880885836\\
4.10328098632881	0.0848801710906291\\
4.11275495069048	0.0851806795659049\\
4.12222891505216	0.085481010533928\\
4.13173996944755	0.0857823357495154\\
4.14125102384295	0.0860834761113509\\
4.15076207823835	0.0863844286692049\\
4.16027313263375	0.0866851904943849\\
4.16982160537755	0.086986940787816\\
4.17937007812135	0.0872884929958888\\
4.18891855086514	0.0875898442212866\\
4.19846702360894	0.0878909915884243\\
4.20805324837695	0.0881931216655511\\
4.21763947314496	0.08849504052699\\
4.22722569791297	0.0887967453291275\\
4.23681192268098	0.0890982332502686\\
4.24643623800907	0.0894006981285459\\
4.25606055333716	0.0897029387656904\\
4.26568486866525	0.0900049523726081\\
4.27530918399335	0.0903067361823002\\
4.28497193298002	0.0906094911922792\\
4.29463468196669	0.0909120090436588\\
4.30429743095337	0.0912142870030477\\
4.31396017994004	0.0915163223589672\\
4.32366171143296	0.091819323191933\\
4.33336324292588	0.0921220740599926\\
4.3430647744188	0.0924245722862378\\
4.35276630591172	0.0927268152159156\\
4.3625069732434	0.0930300178935597\\
4.37224764057509	0.0933329579145983\\
4.38198830790678	0.0936356326593158\\
4.39172897523846	0.0939380395302817\\
4.40150913802251	0.0942414004699182\\
4.41128930080656	0.094544486178985\\
4.42106946359061	0.0948472940956995\\
4.43084962637466	0.0951498216806629\\
4.44066964916844	0.0954532976626822\\
4.45048967196222	0.0957564859605302\\
4.460309694756	0.0960593840714664\\
4.47012971754978	0.0963619895149031\\
4.47998997038358	0.0966655377078278\\
4.48985022321737	0.0969687858863402\\
4.49971047605117	0.0972717316073002\\
4.50957072888496	0.0975743724499618\\
4.51947158785354	0.0978779504363317\\
4.52937244682212	0.0981812162043582\\
4.5392733057907	0.0984841673711275\\
4.54917416475928	0.0987868015761948\\
4.55911601157961	0.0990903673463532\\
4.56905785839994	0.0993936088232852\\
4.57899970522028	0.0996965236849784\\
4.58894155204061	0.0999991096319043\\
4.59892477418835	0.100302621595911\\
4.6089079963361	0.100605797323055\\
4.61889121848384	0.100908634553204\\
4.62887444063159	0.101211131048485\\
4.6388994324499	0.101514548076368\\
4.64892442426821	0.101817617057663\\
4.65894941608653	0.102120335794509\\
4.66897440790484	0.102422702111501\\
4.67904156948105	0.102725983504914\\
4.68910873105727	0.103028905178483\\
4.69917589263348	0.103331464997155\\
4.7092430542097	0.103633660848353\\
4.71935279189367	0.103936766362274\\
4.72946252957763	0.104239500621839\\
4.7395722672616	0.104541861555415\\
4.74968200494557	0.104843847113775\\
4.75983473174927	0.105146736974377\\
4.76998745855297	0.10544924418666\\
4.78014018535667	0.105751366743227\\
4.79029291216037	0.106053102658884\\
4.8004890483408	0.106355737585379\\
4.81068518452122	0.106657978612488\\
4.82088132070165	0.106959823797305\\
4.83107745688208	0.107261271219271\\
4.84131742876219	0.107563612392969\\
4.85155740064229	0.107865548561245\\
4.8617973725224	0.108167077846138\\
4.87203734440251	0.108468198391994\\
4.88232158523569	0.10877020748636\\
4.89260582606888	0.109071800616119\\
4.90289006690206	0.1093729759688\\
4.91317430773525	0.109673731754088\\
4.92350325885451	0.109975370972341\\
4.93383220997378	0.11027658341506\\
4.94416116109305	0.110577367335898\\
4.95449011221231	0.110877721010486\\
4.96486422170276	0.111178953049086\\
4.97523833119322	0.111479747651538\\
4.98561244068367	0.111780103137771\\
4.99598655017412	0.112080017849776\\
5.00640627395839	0.11238080593271\\
5.01682599774266	0.112681146070814\\
5.02724572152694	0.112981036650658\\
5.03766544531121	0.113280476080778\\
5.04813124666974	0.113580783947812\\
5.05859704802827	0.113880633514673\\
5.0690628493868	0.114180023235034\\
5.07952865074533	0.114478951584321\\
5.09004100143523	0.114778743525255\\
5.10055335212514	0.115078066965098\\
5.11106570281504	0.115376920425098\\
5.12157805350494	0.115675302448089\\
5.13213743298625	0.115974543281164\\
5.14269681246755	0.116273305568303\\
5.15325619194885	0.116571587898388\\
5.16381557143015	0.116869388881911\\
5.17442246764722	0.117168043976782\\
5.1850293638643	0.117466210638152\\
5.19563626008137	0.117763887522808\\
5.20624315629844	0.118061073309005\\
5.21689806557204	0.118359108583948\\
5.22755297484563	0.118656645696081\\
5.23820788411923	0.118953683370464\\
5.24886279339282	0.119250220353367\\
5.25956622085953	0.119547602287705\\
5.27026964832624	0.11984447648895\\
5.28097307579295	0.120140841750775\\
5.29167650325966	0.120436696887879\\
5.30242896340108	0.120733392535379\\
5.31318142354251	0.121029571039711\\
5.32393388368394	0.121325231263123\\
5.33468634382537	0.121620372088872\\
5.34548835981846	0.121916349058311\\
5.35629037581155	0.122211799635575\\
5.36709239180464	0.122506722751666\\
5.37789440779773	0.12280111735841\\
5.38874651272417	0.123096343845362\\
5.39959861765061	0.123391034852727\\
5.41045072257705	0.123685189380531\\
5.42130282750349	0.123978806449323\\
5.43220556367259	0.12427325121553\\
5.44310829984169	0.124567151576779\\
5.45401103601079	0.12486050660236\\
5.46491377217989	0.125153315381852\\
5.47586769235055	0.125446947785318\\
5.48682161252121	0.125740027021278\\
5.49777553269188	0.126032552228139\\
5.50872945286254	0.126324522564563\\
5.51973511980852	0.126617312544237\\
5.5307407867545	0.126909540757115\\
5.54174645370048	0.127201206410816\\
5.55275212064646	0.127492308732985\\
5.5638101076295	0.127784226817847\\
5.57486809461254	0.128075574700096\\
5.58592608159558	0.128366351656712\\
5.59698406857862	0.128656556984393\\
5.60809495979115	0.128947574301135\\
5.61920585100368	0.129238013142897\\
5.63031674221621	0.129527872856208\\
5.64142763342874	0.129817152807002\\
5.65259202401047	0.130107241075199\\
5.66375641459219	0.130396742759971\\
5.67492080517392	0.130685657277142\\
5.68608519575564	0.130973984061895\\
5.69730369280843	0.131263115614968\\
5.70852218986121	0.131551652640078\\
5.719740686914	0.131839594622345\\
5.73095918396678	0.132126941065984\\
5.74223240540217	0.132415088815489\\
5.75350562683755	0.13270263425636\\
5.76477884827294	0.132989576943052\\
5.77605206970832	0.133275916448796\\
5.78738064638032	0.133563053933409\\
5.79870922305231	0.133849581492088\\
5.81003779972431	0.134135498748768\\
5.8213663763963	0.134420805345751\\
5.83275095107969	0.13470690669753\\
5.84413552576308	0.134992390669558\\
5.85552010044646	0.135277256954904\\
5.86690467512985	0.135561505264965\\
5.87834590369998	0.135846545231766\\
5.88978713227011	0.136130960528154\\
5.90122836084025	0.136414750916238\\
5.91266958941038	0.136697916176163\\
5.92416814082324	0.136981870112948\\
5.9356666922361	0.137265192251348\\
5.94716524364896	0.137547882422444\\
5.95866379506182	0.137829940475028\\
5.97022035244649	0.138112784362489\\
5.98177690983116	0.138394989485443\\
5.99333346721583	0.13867655574404\\
6.00489002460049	0.13895748305565\\
6.01650528373054	0.139239193453284\\
6.02812054286059	0.139520258281903\\
6.03973580199063	0.139800677510346\\
6.05135106112068	0.140080451124601\\
6.06302573298457	0.14036100522412\\
6.07470040484846	0.140640907111053\\
6.08637507671235	0.140920156822714\\
6.09804974857625	0.14119875441327\\
6.10978455915361	0.141478130022271\\
6.12151936973097	0.141756846935224\\
6.13325418030833	0.142034905257763\\
6.14498899088569	0.142312305112036\\
6.15678468114828	0.142590480643355\\
6.16858037141086	0.142867991154243\\
6.18037606167345	0.143144836818644\\
6.19217175193603	0.143421017826514\\
6.20402907883544	0.143697972310342\\
6.21588640573484	0.143974255607497\\
6.22774373263424	0.144249867959939\\
6.23960105953364	0.144524809625431\\
6.25152079648947	0.144800522708212\\
6.2634405334453	0.145075558595376\\
6.27536027040113	0.145349917596521\\
6.28728000735696	0.145623600036806\\
6.2992629439367	0.145898051961261\\
6.31124588051644	0.146171820837384\\
6.32322881709618	0.146444907042187\\
6.33521175367592	0.146717310967884\\
6.34725869756642	0.146990482605441\\
6.35930564145691	0.14726296549679\\
6.3713525853474	0.147534760086198\\
6.3833995292379	0.147805866832688\\
6.39551130573695	0.148077739657752\\
6.40762308223601	0.148348918192143\\
6.41973485873506	0.14861940294725\\
6.43184663523411	0.148889194448761\\
6.44402408779825	0.149159750536618\\
6.45620154036239	0.149429606941696\\
6.46837899292652	0.149698764241976\\
6.48055644549066	0.149967223029592\\
6.49280043746748	0.150236445079887\\
6.5050444294443	0.150504962206137\\
6.51728842142112	0.150772775052616\\
6.52953241339793	0.151039884277382\\
6.5418438273366	0.151307755582729\\
6.55415524127526	0.151574916872113\\
6.56646665521392	0.15184136885582\\
6.57877806915258	0.152107112257522\\
6.59115780864942	0.15237361672786\\
6.60353754814627	0.152639406237761\\
6.61591728764311	0.15290448156342\\
6.62829702713995	0.153168843493833\\
6.64074601624377	0.153433965626122\\
6.65319500534758	0.153698367999443\\
6.66564399445139	0.153962051455445\\
6.67809298355521	0.154225016848337\\
6.69061216875419	0.154488741751422\\
6.70313135395318	0.154751742241267\\
6.71565053915216	0.155014019224523\\
6.72816972435114	0.1552755736201\\
6.74076007448507	0.155537886994016\\
6.75335042461899	0.155799471442857\\
6.76594077475292	0.156060327937938\\
6.77853112488684	0.156320457462435\\
6.79119363328311	0.15658134562384\\
6.80385614167938	0.15684150048865\\
6.81651865007565	0.157100923092571\\
6.82918115847192	0.157359614482688\\
6.84191684190172	0.157619064322996\\
6.85465252533151	0.157877776633396\\
6.86738820876131	0.158135752513812\\
6.88012389219111	0.158392993074973\\
6.8929337927896	0.158650992079323\\
6.90574369338808	0.158908249456791\\
6.91855359398656	0.159164766370931\\
6.93136349458505	0.159420543995906\\
6.94424868139473	0.159677080252429\\
6.95713386820441	0.159932870919228\\
6.97001905501408	0.160187917223082\\
6.98290424182376	0.160442220400994\\
6.99586581007284	0.160697282563787\\
7.00882737832193	0.160951595306014\\
7.02178894657101	0.161205159917301\\
7.03475051482009	0.161457977697088\\
7.04778958943027	0.161711555032782\\
7.06082866404044	0.161964379246426\\
7.07386773865062	0.162216451690265\\
7.08690681326079	0.162467773725785\\
7.10002454751536	0.162719856062713\\
7.11314228176992	0.162971181703073\\
7.12626001602448	0.163221752061438\\
7.13937775027905	0.163471568561113\\
7.15257532859946	0.163722146318282\\
7.16577290691987	0.163971963928966\\
7.17897048524028	0.164221022869494\\
7.19216806356068	0.164469324624682\\
7.20544670103548	0.164718388776699\\
7.21872533851028	0.164966689454488\\
7.23200397598508	0.165214228195728\\
7.24528261345988	0.165461006546173\\
7.2586435601875	0.165708548678353\\
7.27200450691511	0.165955324127851\\
7.28536545364272	0.166201334493312\\
7.29872640037034	0.166446581381023\\
7.31217093962371	0.166692593626299\\
7.32561547887708	0.166937836096971\\
7.33906001813046	0.16718231045243\\
7.35250455738383	0.1674260183591\\
7.36603400873681	0.167670493427434\\
7.37956346008979	0.167914195742814\\
7.39309291144277	0.168157127025081\\
7.40662236279575	0.168399289000574\\
7.42023808324719	0.168642220168735\\
7.43385380369863	0.168884375716417\\
7.44746952415006	0.16912575742336\\
7.4610852446015	0.169366367075535\\
7.47478862987629	0.169607748179242\\
7.48849201515108	0.16984835090281\\
7.50219540042587	0.170088177085488\\
7.51589878570066	0.170327228572329\\
7.52969127264357	0.170567054016262\\
7.54348375958647	0.170806098425127\\
7.55727624652937	0.171044363697302\\
7.57106873347227	0.171281851736529\\
7.58495180146776	0.171520116485248\\
7.59883486946326	0.171757597645376\\
7.61271793745875	0.171994297174192\\
7.62660100545424	0.172230217033746\\
7.64057617834747	0.172466916609918\\
7.6545513512407	0.172702830141951\\
7.66852652413393	0.172937959645832\\
7.68250169702716	0.173172307141678\\
7.69657054527904	0.173407437626159\\
7.71063939353093	0.173641779705682\\
7.72470824178282	0.173875335454434\\
7.7387770900347	0.174108106950441\\
7.75294123299068	0.174341664982352\\
7.76710537594666	0.174574432339706\\
7.78126951890264	0.174806411154537\\
7.79543366185862	0.175037603562286\\
7.80969476969516	0.175269586331732\\
7.8239558775317	0.175500776244676\\
7.83821698536823	0.175731175490652\\
7.85247809320477	0.175960786262163\\
7.86683788984999	0.176191191517026\\
7.8811976864952	0.176420801817294\\
7.89555748314041	0.176649619409735\\
7.90991727978563	0.176877646543582\\
7.9243775449477	0.177106472579548\\
7.93883781010978	0.177334501642639\\
7.95329807527185	0.17756173603682\\
7.96775834043393	0.177788178067767\\
7.98232091364819	0.178015423747929\\
7.99688348686244	0.178241870512767\\
8.0114460600767	0.178467520723102\\
8.02600863329096	0.178692376741024\\
8.04067541501396	0.178918041466078\\
8.05534219673697	0.179142905405244\\
8.07000897845997	0.179366970975869\\
8.08467576018298	0.179590240596189\\
8.0994487168151	0.179814324333988\\
8.11422167344722	0.180037605482886\\
8.12899463007934	0.180260086516511\\
8.14376758671146	0.180481769908917\\
8.1586487531091	0.18070427318134\\
8.17352991950675	0.18092597212504\\
8.1884110859044	0.181146869269683\\
8.20329225230204	0.181366967144914\\
8.21828373575118	0.181587891034576\\
8.23327521920032	0.181808008914186\\
8.24826670264945	0.182027323369372\\
8.26325818609859	0.182245836985151\\
8.27836217017066	0.182465183136808\\
8.29346615424273	0.182683721650662\\
8.3085701383148	0.182901455168379\\
8.32367412238687	0.18311838633023\\
8.33889287092472	0.183336156950508\\
8.35411161946257	0.183553118354035\\
8.36933036800042	0.183769273238278\\
8.38454911653827	0.183984624298886\\
8.39988497972518	0.184200822182099\\
8.41522084291208	0.184416209313164\\
8.43055670609898	0.184630788445229\\
8.44589256928588	0.184844562329183\\
8.46134798619136	0.18505919082845\\
8.47680340309684	0.185273007078531\\
8.49225882000231	0.185486013888177\\
8.50771423690779	0.185698214063394\\
8.52329174240573	0.185911277120216\\
8.53886924790367	0.186123526463592\\
8.55444675340161	0.186334964957817\\
8.57002425889955	0.186545595463986\\
8.58572649044944	0.186757097629518\\
8.60142872199933	0.186967784644075\\
8.61713095354922	0.187177659427595\\
8.63283318509911	0.187386724896238\\
8.64866288641169	0.187596671305168\\
8.66449258772426	0.187805801146176\\
8.68032228903684	0.188014117395158\\
8.69615199034942	0.188221623023402\\
8.71211202044306	0.188430019435727\\
8.7280720505367	0.188637597877354\\
8.74403208063034	0.18884436138022\\
8.75999211072399	0.189050312971071\\
8.77608545081172	0.18925716577757\\
8.79217879089946	0.189463199218008\\
8.80827213098719	0.189668416380482\\
8.82436547107493	0.189872820347469\\
8.84059523179885	0.190078136573866\\
8.85682499252278	0.190282632039219\\
8.87305475324671	0.190486309888028\\
8.88928451397064	0.190689173258617\\
8.9056539454559	0.190892960598921\\
8.92202337694115	0.191095925776049\\
8.93839280842641	0.191298071991159\\
8.95476223991167	0.191499402438727\\
8.97127474100433	0.191701669269612\\
8.987787242097	0.191903112520115\\
9.00429974318966	0.192103735448377\\
9.02081224428233	0.192303541305361\\
9.0374713729445	0.192504296706915\\
9.05413050160667	0.19270422708727\\
9.07078963026884	0.192903335762107\\
9.08744875893101	0.193101626039261\\
9.10425824450065	0.193300879828939\\
9.12106773007028	0.19349930712386\\
9.13787721563991	0.19369691129799\\
9.15468670120955	0.19389369571655\\
9.17165045698708	0.194091458476221\\
9.18861421276461	0.194288393225356\\
9.20557796854214	0.194484503396772\\
9.22254172431967	0.194679792413879\\
9.23966386191763	0.194876075526569\\
9.25678599951559	0.195071529060457\\
9.27390813711355	0.195266156507882\\
9.29103027471151	0.195459961351266\\
9.30828551504259	0.195654444080832\\
9.32554075537366	0.195848098438658\\
9.34279599570473	0.196040927959249\\
9.36005123603581	0.196232936166636\\
9.37740098280311	0.196425171473248\\
9.39475072957042	0.196616583559085\\
9.41210047633773	0.196807175975063\\
9.42945022310504	0.196996952261201\\
9.44689582322271	0.197186957676958\\
9.46434142334038	0.197376145065668\\
9.48178702345805	0.197564517993458\\
9.49923262357573	0.197752080015215\\
9.51677513419308	0.197939869850471\\
9.53431764481043	0.198126846920481\\
9.55186015542778	0.198313014805261\\
9.56940266604513	0.198498377073229\\
9.58704316415657	0.198683965936352\\
9.60468366226802	0.198868747359486\\
9.62232416037946	0.199052724935447\\
9.6399646584909	0.199235902244863\\
9.65770423902654	0.199419305001111\\
9.67544381956217	0.1996019057019\\
9.6931834000978	0.199783707951773\\
9.71092298063344	0.199964715342368\\
9.72876275712183	0.200145947101637\\
9.74660253361022	0.200326382245257\\
9.76444231009861	0.200506024388207\\
9.782282086587	0.200684877132176\\
9.80022319117994	0.200863953230386\\
9.81816429577288	0.201042238204058\\
9.83610540036582	0.201219735677426\\
9.85404650495876	0.201396449261143\\
9.87209008914481	0.201573385248956\\
9.89013367333085	0.201749535650774\\
9.9081772575169	0.201924904098975\\
9.92622084170295	0.202099494211946\\
9.94436807654274	0.202274305838781\\
9.96251531138254	0.202448337461884\\
9.98066254622234	0.202621592720603\\
9.99880978106213	0.202794075240001\\
10.0170618577606	0.202966778441669\\
10.0353139344591	0.203138707261958\\
10.0535660111575	0.203309865346034\\
10.071818087856	0.203480256324562\\
10.0901762182373	0.203650867210174\\
10.1085343486186	0.203820709373224\\
10.1268924789999	0.203989786463667\\
10.1452506093812	0.204158102116658\\
10.163716026191	0.204326636954222\\
10.1821814430008	0.204494408760842\\
10.2006468598106	0.204661421190503\\
10.2191122766204	0.204827677881774\\
10.237686234429	0.204994153089324\\
10.2562601922376	0.205159870986995\\
10.2748341500462	0.205324835231888\\
10.2934081078548	0.205489049465121\\
10.3120918830375	0.2056534815937\\
10.3307756582203	0.205817162159675\\
10.3494594334031	0.205980094822201\\
10.3681432085859	0.206142283224269\\
10.3869381001644	0.206304688949007\\
10.4057329917429	0.206466348881364\\
10.4245278833213	0.206627266681695\\
10.4433227748998	0.20678744599389\\
10.4622301050096	0.206947842101777\\
10.4811374351193	0.2071074982074\\
10.5000447652291	0.207266417971397\\
10.5189520953388	0.207424605037645\\
10.5379732097484	0.207583008415988\\
10.556994324158	0.207740677599128\\
10.5760154385676	0.207897616247031\\
10.5950365529771	0.208053828002769\\
10.6141728216781	0.208210255628469\\
10.6333090903791	0.208365954880099\\
10.6524453590801	0.208520929416043\\
10.6715816277811	0.208675182877667\\
10.6908344455163	0.208829651806387\\
10.7100872632516	0.208983398193222\\
10.7293400809869	0.20913642569436\\
10.7485928987221	0.209288737948566\\
10.7679636856929	0.209441265305299\\
10.7873344726636	0.209593075960561\\
10.8067052596344	0.209744173567802\\
10.8260760466051	0.209894561762427\\
10.8455662488972	0.210045164729992\\
10.8650564511894	0.210195056842211\\
10.8845466534815	0.210344241748926\\
10.9040368557736	0.210492723081755\\
10.9236479463274	0.210641418893853\\
10.9432590368811	0.210789409699786\\
10.9628701274349	0.210936699145119\\
10.9824812179887	0.211083290857013\\
11.0022146969848	0.211230096786728\\
11.0219481759809	0.211376203560084\\
11.041681654977	0.211521614817704\\
11.0614151339731	0.211666334181513\\
11.0812725294807	0.211811267532111\\
11.1011299249883	0.211955507574487\\
11.1209873204959	0.212099057943543\\
11.1408447160036	0.212241922255421\\
11.1608275848708	0.21238500035306\\
11.180810453738	0.212527390986721\\
11.2007933226052	0.212669097784835\\
11.2207761914724	0.212810124357061\\
11.2408861199495	0.212951364541884\\
11.2609960484265	0.213091923100673\\
11.2811059769036	0.213231803654955\\
11.3012159053807	0.213371009807176\\
11.3214545099205	0.213510429425633\\
11.3416931144603	0.213649173247433\\
11.3619317190001	0.213787244886904\\
11.3821703235399	0.213924647938692\\
11.4025392513608	0.214062264334287\\
11.4229081791817	0.214199210752146\\
11.4432771070026	0.214335490798699\\
11.4636460348235	0.214471108060593\\
11.4841469650118	0.214606938568658\\
11.5046478952002	0.214742104905363\\
11.5251488253886	0.214876610668791\\
11.545649755577	0.21501045943708\\
11.5662843996785	0.215144521375957\\
11.5869190437801	0.215277924935476\\
11.6075536878816	0.215410673704872\\
11.6281883319832	0.215542771253211\\
11.6489584349037	0.215675081917684\\
11.6697285378243	0.215806739978726\\
11.6904986407448	0.215937749016096\\
11.7112687436654	0.216068112589427\\
11.7321760845019	0.216198689243817\\
11.7530834253385	0.216328619052971\\
11.7739907661751	0.216457905586604\\
11.7948981070116	0.216586552394349\\
11.8159445000986	0.216715412267367\\
11.8369908931856	0.216843631033657\\
11.8580372862726	0.216971212252762\\
11.8790836793596	0.217098159463726\\
11.900270975065	0.217225319741426\\
11.9214582707703	0.217351844629633\\
11.9426455664757	0.217477737677474\\
11.9638328621811	0.217603002413111\\
11.9851629478153	0.217728480232836\\
12.0064930334495	0.217853328357591\\
12.0278231190837	0.217977550325558\\
12.0491532047179	0.21810114965401\\
12.0706280057176	0.218224962100029\\
12.0921028067173	0.218348150521335\\
12.113577607717	0.218470718444954\\
12.1350524087168	0.21859266937675\\
12.1566738894695	0.21871483347367\\
12.1782953702222	0.218836379190625\\
12.1999168509749	0.218957310043031\\
12.2215383317276	0.21907762952512\\
12.2433084968313	0.219198162233886\\
12.265078661935	0.219318082180763\\
12.2868488270387	0.219437392869048\\
12.3086189921423	0.219556097781011\\
12.3305398873956	0.219675015993617\\
12.3524607826488	0.219793327034386\\
12.3743816779021	0.219911034394275\\
12.3963025731553	0.220028141543105\\
12.4183762865519	0.220145462077464\\
12.4404499999485	0.220262181000556\\
12.462523713345	0.220378301791132\\
12.4845974267416	0.22049382790621\\
12.50682608981	0.220609567502595\\
12.5290547528783	0.220724711017897\\
12.5512834159467	0.220839261918221\\
12.5735120790151	0.22095322364797\\
12.5958978680728	0.221067398965263\\
12.6182836571306	0.221180983700052\\
12.6406694461884	0.221293981305704\\
12.6630552352461	0.221406395213712\\
12.6856003725544	0.221519022824873\\
12.7081455098627	0.221631065319612\\
12.730690647171	0.221742526138289\\
12.7532357844793	0.221853408699306\\
12.7759425395622	0.22196450508767\\
12.7986492946451	0.22207502179246\\
12.821356049728	0.222184962240593\\
12.844062804811	0.222294329837253\\
12.8668397124378	0.222403463218907\\
12.8896166200647	0.222512027001944\\
12.9123935276916	0.222620024557962\\
12.9351704353184	0.222727459237182\\
12.9579981520666	0.222834572155148\\
12.9808258688147	0.222941126356387\\
13.0036535855629	0.223047125149693\\
13.026481302311	0.223152571822303\\
13.049363098101	0.223257717495498\\
13.0722448938909	0.223362314980323\\
13.0951266896808	0.223466367524803\\
13.1180084854707	0.223569878355935\\
13.1409474667556	0.223673107353295\\
13.1638864480404	0.22377579835873\\
13.1868254293253	0.223877954561789\\
13.2097644106101	0.223979579131142\\
13.2327635996633	0.224080939868586\\
13.2557627887165	0.224181772495727\\
13.2787619777697	0.224282080145434\\
13.3017611668229	0.224381865930205\\
13.3248235100991	0.224481404811465\\
13.3478858533754	0.224580425165157\\
13.3709481966517	0.22467893006905\\
13.3940105399279	0.224776922581172\\
13.4171389165768	0.224874684130571\\
13.4402672932257	0.224971936450289\\
13.4633956698746	0.225068682565243\\
13.4865240465235	0.225164925480444\\
13.5097212774941	0.225260952468708\\
13.5329185084647	0.225356479253785\\
13.5561157394353	0.225451508809359\\
13.5793129704059	0.225546044089591\\
13.6025818233678	0.225640377634347\\
13.6258506763298	0.22573421974337\\
13.6491195292917	0.225827573340898\\
13.6723883822536	0.225920441331744\\
13.6957315789086	0.22601312100107\\
13.7190747755636	0.226105317754916\\
13.7424179722186	0.226197034469383\\
13.7657611688735	0.226288274001667\\
13.789181390828	0.226379337905002\\
13.8126016127825	0.226469927177099\\
13.8360218347369	0.226560044647093\\
13.8594420566914	0.226649693125733\\
13.8829419497165	0.226739177993885\\
13.9064418427417	0.226828196288587\\
13.9299417357668	0.226916750794031\\
13.9534416287919	0.227004844275693\\
13.9770238098647	0.227092785545725\\
14.0006059909374	0.227180268083312\\
14.0241881720102	0.227267294628801\\
14.0477703530829	0.227353867904324\\
14.0714374141426	0.227440299787162\\
14.0951044752024	0.227526280570482\\
14.1187715362621	0.227611812952232\\
14.1424385973218	0.227696899612301\\
14.1661931077477	0.227781855151077\\
14.1899476181736	0.227866367024164\\
14.2137021285995	0.227950437887866\\
14.2374566390255	0.22803407038105\\
14.2613011526304	0.228117581522965\\
14.2851456662352	0.228200656241363\\
14.3089901798401	0.228283297152554\\
14.332834693445	0.228365506855126\\
14.3567717508919	0.228447604502724\\
14.3807088083388	0.228529272784698\\
14.4046458657857	0.228610514278373\\
14.4285829232326	0.228691331543827\\
14.4526150559176	0.228772045608464\\
14.4766471886026	0.228852337187865\\
14.5006793212876	0.228932208821682\\
14.5247114539725	0.229011663032368\\
14.548841186964	0.229091022480604\\
14.5729709199554	0.229169966153722\\
14.5971006529469	0.229248496554209\\
14.6212303859383	0.229326616167982\\
14.6454602425121	0.229404649071323\\
14.6696900990858	0.229482272745222\\
14.6939199556595	0.229559489656528\\
14.7181498122333	0.229636302255156\\
14.7424823150752	0.229713035826428\\
14.7668148179171	0.229789366555369\\
14.7911473207591	0.229865296874014\\
14.815479823601	0.229940829197892\\
14.8399174997295	0.23001628983728\\
14.864355175858	0.230091353868257\\
14.8887928519865	0.230166023689123\\
14.913230528115	0.230240301681868\\
14.9377759105528	0.230314515009086\\
14.9623212929905	0.230388337814707\\
14.9868666754283	0.230461772463656\\
15.0114120578661	0.230534821305063\\
15.0360676892572	0.230607812197238\\
15.0607233206484	0.230680418511818\\
15.0853789520395	0.23075264258187\\
15.1100345834307	0.230824486724437\\
15.1348030187572	0.230896279348735\\
15.1595714540837	0.23096769320143\\
15.1843398894102	0.231038730584553\\
15.2091083247367	0.23110939378416\\
15.2339921326276	0.231180011624425\\
15.2588759405185	0.231250256365852\\
15.2837597484095	0.231320130279915\\
15.3086435563004	0.231389635622694\\
15.333645323629	0.231459101513229\\
15.3586470909575	0.231528199849058\\
15.383648858286	0.231596932872024\\
15.4086506256146	0.231665302808361\\
15.4337729580942	0.231733638956438\\
15.4588952905739	0.231801612968955\\
15.4840176230536	0.231869227058953\\
15.5091399555333	0.231936483424213\\
15.5343854824194	0.232003711441876\\
15.5596310093056	0.232070582621885\\
15.5848765361918	0.232137099149357\\
15.610122063078	0.232203263194312\\
15.6354934388609	0.232269404118984\\
15.6608648146438	0.232335193387272\\
15.6862361904268	0.23240063315655\\
15.7116075662097	0.232465725569479\\
15.7371074715309	0.232530799883443\\
15.762607376852	0.23259552760847\\
15.7881072821731	0.232659910875442\\
15.8136071874943	0.232723951800527\\
15.8392383339326	0.232787979458859\\
15.8648694803709	0.232851665485232\\
15.8905006268092	0.232915011984834\\
15.9161317732476	0.23297802104804\\
15.9418969045212	0.233041021495189\\
15.9676620357948	0.233103685160719\\
15.9934271670685	0.2331660141242\\
16.0191922983421	0.233228010450959\\
16.0450941933734	0.233290002641259\\
16.0709960884047	0.233351662796489\\
16.096897983436	0.233412992971743\\
16.1227998784672	0.233473995207491\\
16.1488413541292	0.23353499762474\\
16.1748828297911	0.233595672652464\\
16.2009243054531	0.233656022321823\\
16.226965781115	0.233716048649519\\
16.2531496945447	0.233776079323814\\
16.2793336079744	0.233835787155997\\
16.3055175214041	0.23389517415375\\
16.3317014348337	0.233954242310755\\
16.3580306848324	0.234013318831342\\
16.384359934831	0.23407207696236\\
16.4106891848297	0.234130518688586\\
16.4370184348283	0.234188645980661\\
16.4634959660973	0.234246785516471\\
16.4899734973663	0.234304611022442\\
16.5164510286353	0.234362124461141\\
16.5429285599043	0.234419327781271\\
16.5695573662806	0.2344765470972\\
16.5961861726568	0.234533456652536\\
16.6228149790331	0.234590058388342\\
16.6494437854093	0.234646354231902\\
16.6762269105277	0.234702669697511\\
16.7030100356461	0.234758679584524\\
16.7297931607645	0.234814385812529\\
16.7565762858829	0.234869790287665\\
16.7835168283287	0.234925217896495\\
16.8104573707745	0.234980344023046\\
16.8373979132204	0.235035170566452\\
16.8643384556662	0.235089699412391\\
16.8914395705478	0.23514425479237\\
16.9185406854293	0.235198512702855\\
16.9456418003109	0.235252475023179\\
16.9727429151925	0.23530614361909\\
17.0000078170973	0.235359842044515\\
17.027272719002	0.235413246932378\\
17.0545376209068	0.235466360142153\\
17.0818025228116	0.235519183520223\\
17.1092344893586	0.235572039924409\\
17.1366664559057	0.235624606643829\\
17.1640984224528	0.235676885519078\\
17.1915303889999	0.235728878377321\\
17.219132763799	0.235780907364147\\
17.2467351385981	0.235832650441663\\
17.2743375133972	0.235884109432023\\
17.3019398881964	0.235935286144018\\
17.3297160854172	0.235986502001164\\
17.3574922826381	0.236037435649054\\
17.385268479859	0.236088088891654\\
17.4130446770799	0.236138463520004\\
17.4409981837359	0.23618888022618\\
17.4689516903919	0.236239018349879\\
17.4969051970478	0.236288879677466\\
17.5248587037038	0.236338465982114\\
17.5529930820312	0.23638809721591\\
17.5811274603586	0.236437453422057\\
17.6092618386859	0.236486536369801\\
17.6373962170133	0.236535347815386\\
17.6657151116334	0.236584206971239\\
17.6940340062535	0.236632794583782\\
17.7223529008736	0.236681112405643\\
17.7506717954937	0.236729162176662\\
17.7791789342782	0.236777262367544\\
17.8076860730626	0.236825094431277\\
17.836193211847	0.236872660103991\\
17.8647003506315	0.236919961109131\\
17.8933995511133	0.236967315180238\\
17.922098751595	0.237014404473018\\
17.9507979520768	0.237061230707859\\
17.9794971525585	0.237107795592569\\
18.0083923245633	0.237154416127258\\
18.0372874965681	0.237200775166461\\
18.0661826685729	0.237246874415396\\
18.0950778405776	0.237292715566614\\
18.124172992061	0.237338614896132\\
18.1532681435443	0.23738425594908\\
18.1823632950276	0.237429640415366\\
18.2114584465109	0.237474769972589\\
18.2407576866174	0.237519960181559\\
18.270056926724	0.237564895269746\\
18.2993561668305	0.237609576912622\\
18.328655406937	0.237654006773233\\
18.3581629537927	0.237698499712147\\
18.3876705006484	0.237742740624055\\
18.4171780475041	0.23778673117051\\
18.4466855943597	0.237830473000487\\
18.4764057786275	0.237874280290036\\
18.5061259628954	0.23791783858598\\
18.5358461471632	0.237961149535864\\
18.565566331431	0.238004214775089\\
18.5955036012303	0.238047347812018\\
18.6254408710296	0.238090234829371\\
18.6553781408289	0.238132877461475\\
18.6853154106283	0.238175277330164\\
18.7154743397902	0.238217747298377\\
18.7456332689521	0.23825997416253\\
18.7757921981141	0.238301959544095\\
18.805951127276	0.238343705052073\\
18.8363364194138	0.238385522926185\\
18.8667217115517	0.238427100554364\\
18.8971070036895	0.2384684395454\\
18.9274922958273	0.238509541495999\\
18.9581087922171	0.2385507180488\\
18.9887252886069	0.238591657157694\\
19.0193417849967	0.238632360419307\\
19.0499582813864	0.238672829417871\\
19.0808109690034	0.238713375228799\\
19.1116636566204	0.238753686342272\\
19.1425163442373	0.238793764343142\\
19.1733690318543	0.23883361080398\\
19.2044630478062	0.238873536261421\\
19.235557063758	0.238913229713057\\
19.2666510797099	0.238952692732346\\
19.2977450956618	0.238991926880693\\
19.3290857385735	0.239031242191532\\
19.3604263814852	0.239070328134825\\
19.3917670243968	0.239109186272839\\
19.4231076673085	0.239147818155696\\
19.4547004046881	0.239186533350356\\
19.4862931420676	0.23922502176258\\
19.5178858794472	0.239263284943986\\
19.5494786168267	0.239301324434168\\
19.581329093847	0.239339449372329\\
19.6131795708672	0.239377350060552\\
19.6450300478875	0.239415028040311\\
19.6768805249077	0.239452484841047\\
19.7089945750868	0.239490029217531\\
19.7411086252659	0.239527351825338\\
19.7732226754451	0.239564454195724\\
19.8053367256242	0.239601337848071\\
19.8377203807137	0.239638311198223\\
19.8701040358032	0.239675065209716\\
19.9024876908927	0.239711601404332\\
19.9348713459822	0.239747921291972\\
19.967530847752	0.239784332998199\\
20.0001903495219	0.239820527744961\\
20.0328498512918	0.239856507045086\\
20.0655093530616	0.239892272399354\\
20.0984511655557	0.239928131696121\\
20.1313929780498	0.239963776362924\\
20.1643347905439	0.239999207903454\\
20.197276603038	0.240034427809651\\
20.2305074253896	0.240069743788768\\
20.2637382477412	0.240104847417663\\
20.2969690700928	0.240139740191689\\
20.3301998924444	0.240174423594327\\
20.3637266734178	0.240209205211383\\
20.3972534543913	0.240243776708397\\
20.4307802353647	0.240278139572909\\
20.4643070163382	0.24031229528037\\
20.4981369694764	0.24034655135969\\
20.5319669226146	0.2403805995004\\
20.5657968757529	0.240414441182076\\
20.5996268288911	0.240448077872553\\
20.6337674490278	0.240481817112415\\
20.6679080691645	0.240515350546298\\
20.7020486893012	0.240548679646599\\
20.7361893094379	0.240581805873709\\
20.7706483924226	0.240615036854255\\
20.8051074754072	0.240648064112603\\
20.8395665583918	0.240680889114437\\
20.8740256413764	0.240713513313265\\
20.908811298434	0.240746244497009\\
20.9435969554916	0.240778773993856\\
20.9783826125492	0.240811103262835\\
21.0131682696068	0.240843233751128\\
21.0482889562591	0.240875473495508\\
21.0834096429114	0.240907513539863\\
21.1185303295637	0.240939355337202\\
21.153651016216	0.240971000328313\\
21.189115551059	0.2410027568879\\
21.224580085902	0.241034315685589\\
21.260044620745	0.241065678168748\\
21.295509155588	0.241096845772563\\
21.3313267450985	0.241128127304843\\
21.367144334609	0.241159212964331\\
21.4029619241194	0.241190104193199\\
21.4387795136299	0.241220802421561\\
21.474959783243	0.241251616995312\\
21.5111400528561	0.241282237536615\\
21.5473203224691	0.241312665482635\\
21.5835005920822	0.241342902258493\\
21.6200536143196	0.241373257858744\\
21.656606636557	0.24140342121729\\
21.6931596587944	0.241433393766989\\
21.7297126810318	0.241463176928607\\
21.7666490124925	0.241493081466404\\
21.8035853439532	0.241522795503044\\
21.8405216754139	0.241552320467707\\
21.8774580068746	0.241581657777207\\
21.9147887220248	0.241611119093699\\
21.9521194371751	0.241640391599139\\
21.9894501523253	0.241669476719137\\
22.0267808674756	0.241698375867183\\
22.0645176030464	0.241727401743761\\
22.1022543386172	0.241756240448114\\
22.139991074188	0.241784893402993\\
22.1777278097588	0.241813362018681\\
22.2158828101726	0.241841960185232\\
22.2540378105863	0.24187037276609\\
22.2921928110001	0.2418986011816\\
22.3303478114138	0.241926646839729\\
22.3689339797914	0.24195482498367\\
22.407520148169	0.241982819074886\\
22.4461063165466	0.242010630532051\\
22.4846924849242	0.242038260761261\\
22.5237234418978	0.242066026538062\\
22.5627543988714	0.242093609740677\\
22.601785355845	0.242121011786339\\
22.6408163128186	0.242148234079881\\
22.6803064595616	0.242175595123069\\
22.7197966063045	0.242202775014497\\
22.7592867530475	0.242229775170813\\
22.7987768997904	0.242256596995896\\
22.838741493306	0.242283560931158\\
22.8787060868217	0.242310345079238\\
22.9186706803373	0.242336950856743\\
22.9586352738529	0.242363379667558\\
22.9990905077973	0.242389954125405\\
23.0395457417417	0.242416350100605\\
23.0800009756861	0.242442569010604\\
23.1204562096305	0.242468612259826\\
23.1614193085936	0.242494804891128\\
23.2023824075567	0.242520820282212\\
23.2433455065199	0.242546659851846\\
23.284308605483	0.24257232500588\\
23.3257979315477	0.242598143498982\\
23.3672872576124	0.242623785929572\\
23.4087765836772	0.242649253718475\\
23.4502659097419	0.242674548273491\\
23.4922104776085	0.242699945700744\\
23.5341550454751	0.242725168925455\\
23.5760996133418	0.242750219362415\\
23.6180441812084	0.242775098413163\\
23.6604170261043	0.242800058887369\\
23.7027898710001	0.242824847293326\\
23.745162715896	0.242849465036218\\
23.7875355607918	0.242873913508049\\
23.8303479690752	0.242898445092295\\
23.8731603773585	0.242922806692286\\
23.9159727856418	0.242946999704099\\
23.9587851939251	0.242971025510541\\
24.0020482633235	0.242995135764214\\
24.0453113327218	0.243019078072952\\
24.0885744021202	0.243042853823894\\
24.1318374715185	0.243066464391098\\
24.1755626771472	0.24309016071431\\
24.219287882776	0.243113691088035\\
24.2630130884048	0.243137056890789\\
24.3067382940335	0.243160259488063\\
24.3509374990427	0.243183549121827\\
24.3951367040518	0.243206674758281\\
24.4393359090609	0.243229637767881\\
24.48353511407	0.243252439507825\\
24.5282205890312	0.243275329538343\\
24.5729060639923	0.243298057481343\\
24.6175915389535	0.243320624699449\\
24.6622770139146	0.243343032542118\\
24.7074614574793	0.243365529905179\\
24.7526459010441	0.243387867048717\\
24.7978303446088	0.243410045328001\\
24.8430147881735	0.243432066085\\
24.888711348921	0.243454177569946\\
24.9344079096686	0.243476130662335\\
24.9801044704162	0.243497926710363\\
25.0258010311637	0.243519567048925\\
25.0720233306192	0.243541299302412\\
25.1182456300747	0.243562874949997\\
25.1644679295302	0.243584295333036\\
25.2106902289857	0.243605561779709\\
25.2574523862228	0.243626921309475\\
25.3042145434599	0.243648125980263\\
25.350976700697	0.24366917712692\\
25.3977388579342	0.243690076071149\\
25.4450555156315	0.243711069249631\\
25.4923721733288	0.243731909276772\\
25.5396888310261	0.243752597481386\\
25.5870054887234	0.243773135178978\\
25.6348918408033	0.243793768246555\\
25.6827781928832	0.243814249831735\\
25.7306645449631	0.243834581257632\\
25.7785508970431	0.24385476383394\\
25.8270227182624	0.243875042902261\\
25.8754945394818	0.243895172118909\\
25.9239663607012	0.243915152801608\\
25.9724381819206	0.243934986254672\\
26.0215118591534	0.243954917309664\\
26.0705855363863	0.243974700105903\\
26.1196592136191	0.243994335956122\\
26.168732890852	0.244013826159584\\
26.2184254566214	0.244033415064397\\
26.2681180223908	0.244052857266007\\
26.3178105881602	0.24407215407253\\
26.3675031539296	0.244091306778491\\
26.4178323219975	0.244110559276242\\
26.4681614900653	0.24412966658937\\
26.5184906581332	0.244148630021707\\
26.5688198262011	0.244167450863416\\
26.6198040298383	0.244186372579914\\
26.6707882334756	0.244205150593752\\
26.7217724371128	0.244223786204811\\
26.7727566407501	0.244242280699268\\
26.8244150733526	0.244260877145511\\
26.8760735059551	0.244279331334747\\
26.9277319385576	0.244297644563259\\
26.9793903711602	0.244315818113598\\
27.0317430299247	0.244334094688228\\
27.0840956886892	0.244352230415464\\
27.1364483474537	0.244370226588355\\
27.1888010062182	0.244388084486189\\
27.2418687387887	0.244406046477788\\
27.2949364713593	0.244423868995793\\
27.3480042039299	0.244441553330404\\
27.4010719365005	0.244459100758005\\
27.4548764910367	0.244476753347292\\
27.508681045573	0.244494267801166\\
27.5624856001092	0.244511645407373\\
27.6162901546455	0.244528887439772\\
27.6708542335196	0.244546235701674\\
27.7254183123937	0.244563447130887\\
27.7799823912678	0.24458052301311\\
27.834546470142	0.244597464620063\\
27.8898937876417	0.244614513525687\\
27.9452411051415	0.244631426866017\\
28.0005884226412	0.24464820592512\\
28.055935740141	0.244664851972973\\
28.1120910845417	0.244681606391484\\
28.1682464289425	0.244698226476858\\
28.2244017733432	0.244714713511948\\
28.2805571177439	0.244731068765404\\
28.3375464180691	0.244747533465778\\
28.3945357183943	0.244763865029981\\
28.4515250187195	0.244780064740091\\
28.5085143190447	0.244796133863848\\
28.5663647170539	0.244812313516533\\
28.6242151150631	0.244828361194838\\
28.6820655130723	0.244844278180508\\
28.7399159110815	0.244860065740807\\
28.7983043617203	0.244875870175113\\
28.856692812359	0.244891545315214\\
28.9150812629978	0.244907092420241\\
28.9734697136365	0.244922512735026\\
29.0321280026001	0.244937877884327\\
29.0907862915636	0.244953117544486\\
29.1494445805272	0.244968232935197\\
29.2081028694908	0.244983225262285\\
29.2670352867619	0.244998164927854\\
29.325967704033	0.245012982775399\\
29.3849001213041	0.24502767998665\\
29.4438325385752	0.24504225772989\\
29.5030416355931	0.245056784767988\\
29.562250732611	0.245071193537697\\
29.621459829629	0.245085485184093\\
29.6806689266469	0.245099660839215\\
29.7401573016333	0.245113787673863\\
29.7996456766196	0.245127799672561\\
29.859134051606	0.245141697944991\\
29.9186224265924	0.245155483588203\\
29.9783927156743	0.245169222224244\\
30.0381630047562	0.245182849343747\\
30.0979332938381	0.245196366022216\\
30.15770358292	0.245209773322923\\
30.2177584582065	0.245223135360513\\
30.2778133334929	0.245236389091974\\
30.3378682087793	0.245249535559809\\
30.3979230840658	0.245262575794679\\
30.4582652560913	0.245275572444263\\
30.5186074281169	0.245288463893003\\
30.5789496001424	0.245301251151529\\
30.639291772168	0.245313935219023\\
30.6999239910786	0.245326577315448\\
30.7605562099893	0.245339117214919\\
30.8211884289	0.245351555897292\\
30.8818206478106	0.245363894331344\\
30.9427457014925	0.245376192346818\\
31.0036707551745	0.245388391071439\\
31.0645958088564	0.245400491455346\\
31.1255208625383	0.245412494437955\\
31.186741581882	0.245424458496\\
31.2479623012256	0.245436326074964\\
31.3091830205692	0.245448098096291\\
31.3704037399129	0.24545977547104\\
31.4319229948219	0.245471415358051\\
31.493442249731	0.245482961486774\\
31.55496150464	0.245494414750943\\
31.6164807595491	0.245505776034235\\
31.6783014623118	0.245517101212098\\
31.7401221650745	0.245528335264707\\
31.8019428678372	0.245539479059039\\
31.8637635705999	0.245550533452324\\
31.9258886754761	0.245561553069827\\
31.9880137803524	0.245572484110461\\
32.0501388852286	0.245583327415358\\
32.1122639901049	0.245594083816211\\
32.1746964962864	0.245604806720644\\
32.2371290024679	0.245615443514935\\
32.2995615086494	0.245625995015246\\
32.361994014831	0.245636462028615\\
32.4247369657647	0.245646896776285\\
32.4874799166984	0.245657247801763\\
32.5502228676321	0.245667515897098\\
32.6129658185659	0.245677701845508\\
32.6760223019509	0.245687856712014\\
32.739078785336	0.245697930168289\\
32.8021352687211	0.245707922983083\\
32.8651917521062	0.245717835916611\\
32.9285649023539	0.245727718907147\\
32.9919380526016	0.2457375227261\\
33.0553112028493	0.245747248119714\\
33.118684353097	0.245756895825976\\
33.1823773519202	0.245766514684937\\
33.2460703507434	0.245776056540225\\
33.3097633495666	0.245785522116351\\
33.3734563483898	0.24579491212982\\
33.4374724249187	0.245804274349984\\
33.5014885014475	0.245813561666141\\
33.5655045779763	0.245822774781807\\
33.6295206545052	0.24583191439273\\
33.6938630873744	0.245841027224403\\
33.7582055202435	0.245850067185891\\
33.8225479531127	0.245859034960423\\
33.8868903859819	0.245867931223686\\
33.951562503147	0.245876801683179\\
34.016234620312	0.245885601242774\\
34.0809067374771	0.245894330566102\\
34.1455788546421	0.245902990309458\\
34.2105840369751	0.245911625187766\\
34.2755892193081	0.245920191075158\\
34.340594401641	0.245928688616319\\
34.405599583974	0.245937118448804\\
34.4709412619942	0.245945524319042\\
34.5362829400144	0.24595386304819\\
34.6016246180346	0.245962135262619\\
34.6669662960548	0.245970341581774\\
34.7326479550893	0.245978524807465\\
34.7983296141237	0.245986642684802\\
34.8640112731582	0.245994695822449\\
34.9296929321926	0.246002684822351\\
34.9957181104378	0.246010651564478\\
35.061743288683	0.246018554695891\\
35.1277684669281	0.246026394808135\\
35.1937936451733	0.24603417248624\\
35.2601659376236	0.246041928710772\\
35.3265382300739	0.246049623009052\\
35.3929105225242	0.246057255956075\\
35.4592828149746	0.246064828120519\\
35.526005872278	0.24607237960504\\
35.5927289295814	0.246079870796444\\
35.6594519868847	0.246087302253728\\
35.7261750441881	0.246094674529757\\
35.7932525744811	0.246102026870213\\
35.8603301047741	0.246109320501144\\
35.927407635067	0.246116555966081\\
35.99448516536	0.246123733802601\\
36.0619209371048	0.246130892419889\\
36.1293567088496	0.24613799386342\\
36.1967924805944	0.24614503866177\\
36.2642282523392	0.246152027337735\\
36.3320260925956	0.24615899748356\\
36.3998239328521	0.24616591194523\\
36.4676217731086	0.246172771236859\\
36.5354196133651	0.246179575866956\\
36.6035834111162	0.246186362630012\\
36.6717472088673	0.246193095153946\\
36.7399110066184	0.246199773938888\\
36.8080748043695	0.246206399479529\\
36.8766085113362	0.246213007791183\\
36.9451422183029	0.24621956326572\\
37.0136759252696	0.24622606638974\\
37.0822096322362	0.246232517644577\\
37.1511172647722	0.246238952284446\\
37.2200248973081	0.246245335447655\\
37.2889325298441	0.24625166760772\\
37.35784016238	0.246257949233055\\
37.4271258025917	0.2462642148343\\
37.4964114428034	0.246270430279224\\
37.5656970830151	0.246276596028688\\
37.6349827232268	0.246282712538602\\
37.7046505191923	0.246288813592952\\
37.7743183151579	0.246294865772576\\
37.8439861111234	0.2463008695261\\
37.913653907089	0.246306825297335\\
37.9837080759575	0.246312766160174\\
38.0537622448261	0.246318659392471\\
38.1238164136947	0.246324505431015\\
38.1938705825633	0.246330304707901\\
38.2643154110918	0.246336089602948\\
38.3347602396202	0.246341828075504\\
38.4052050681487	0.246347520550901\\
38.4756498966772	0.246353167449901\\
38.5464897434044	0.246358800473843\\
38.6173295901316	0.246364388248446\\
38.6881694368588	0.246369931187955\\
38.759009283586	0.246375429702156\\
38.8302485799245	0.24638091482903\\
38.9014878762629	0.246386355846006\\
38.9727271726013	0.246391753156591\\
39.0439664689397	0.246397107159954\\
39.1156097204774	0.246402448245377\\
39.1872529720151	0.246407746327777\\
39.2588962235527	0.246413001800264\\
39.3305394750904	0.246418215051728\\
39.4025912659757	0.246423415837202\\
39.474643056861	0.24642857469506\\
39.5466948477463	0.246433692008345\\
39.6187466386316	0.246438768155995\\
39.6912116291116	0.246443832272535\\
39.7636766195915	0.246448855506464\\
39.8361416100714	0.246453838231076\\
39.9086066005514	0.246458780815672\\
39.9814895308678	0.246463711787781\\
40.0543724611842	0.246468602892904\\
40.1272553915006	0.246473454494896\\
40.200138321817	0.246478266953722\\
40.2734440149134	0.24648306820311\\
40.3467497080097	0.246487830572742\\
40.420055401106	0.246492554417332\\
40.4933610942023	0.246497240087808\\
40.567094455962	0.246501914936806\\
40.6408278177218	0.246506551865839\\
40.7145611794816	0.246511151220763\\
40.7882945412414	0.246515713343754\\
40.8624605638291	0.246520265018814\\
40.9366265864169	0.246524779707173\\
41.0107926090047	0.246529257746107\\
41.0849586315925	0.246533699469316\\
41.1595623935287	0.246538131104161\\
41.2341661554649	0.246542526659929\\
41.3087699174011	0.246546886465582\\
41.3833736793372	0.246551210846611\\
41.4584203505013	0.246555525485577\\
41.5334670216654	0.246559804928298\\
41.6085136928295	0.246564049495682\\
41.6835603639935	0.246568259505262\\
41.7590552049343	0.246572460106165\\
41.8345500458751	0.246576626369684\\
41.910044886816	0.246580758608926\\
41.9855397277568	0.246584857133707\\
42.0614880926214	0.246588946570823\\
42.137436457486	0.246593002506237\\
42.2133848223507	0.246597025245494\\
42.2893331872153	0.246601015090933\\
42.3657405267235	0.246604996157859\\
42.4421478662318	0.246608944536349\\
42.51855520574	0.24661286052462\\
42.5949625452482	0.246616744417753\\
42.6718344094257	0.246620619830146\\
42.7487062736032	0.246624463345688\\
42.8255781377807	0.246628275255485\\
42.9024500019582	0.246632055847586\\
42.9797920408555	0.246635828245693\\
43.0571340797527	0.246639569517559\\
43.1344761186499	0.246643279947397\\
43.2118181575471	0.246646959816429\\
43.2896361244367	0.246650631767653\\
43.3674540913263	0.246654273342954\\
43.4452720582158	0.246657884819856\\
43.5230900251054	0.246661466472964\\
43.6013897790167	0.246665040474293\\
43.679689532928	0.246668584830385\\
43.7579892868393	0.24667209981227\\
43.8362890407505	0.246675585688132\\
43.9150765502395	0.246679064168538\\
43.9938640597285	0.246682513715387\\
44.0726515692175	0.246685934593413\\
44.1514390787065	0.246689327064565\\
44.2307204224475	0.246692712387155\\
44.3100017661886	0.24669606946948\\
44.3892831099296	0.246699398570157\\
44.4685644536707	0.246702699945087\\
44.5483458244143	0.246705994409341\\
44.6281271951579	0.246709261308814\\
44.7079085659016	0.246712500896191\\
44.7876899366452	0.246715713421503\\
44.867977644264	0.246718919265366\\
44.9482653518828	0.246722098202697\\
45.0285530595016	0.246725250480419\\
45.1088407671204	0.246728376342865\\
45.1896412407794	0.246731495744737\\
45.2704417144383	0.246734588881634\\
45.3512421880972	0.246737655994885\\
45.4320426617561	0.246740697323296\\
45.5133624533588	0.246743732403989\\
45.5946822449614	0.246746741845103\\
45.676002036564	0.246749725882531\\
45.7573218281666	0.246752684749713\\
45.8391676159519	0.246755637574343\\
45.9210134037371	0.246758565369133\\
46.0028591915224	0.246761468364699\\
46.0847049793077	0.246764346789267\\
46.1670835705048	0.246767219369035\\
46.2494621617019	0.246770067513534\\
46.3318407528991	0.246772891448257\\
46.4142193440962	0.246775691396362\\
46.4971376778898	0.246778485690278\\
46.5800560116834	0.246781256128801\\
46.662974345477	0.246784002932449\\
46.7458926792706	0.246786726319452\\
46.8293578314074	0.246789444236051\\
46.9128229835443	0.246792138862894\\
46.9962881356811	0.246794810415663\\
47.079753287818	0.2467974591078\\
47.1637724728227	0.246800102506707\\
47.2477916578275	0.246802723167696\\
47.3318108428323	0.246805321301749\\
47.4158300278371	0.246807897117647\\
47.5004106040352	0.246810467811181\\
47.5849911802333	0.246813016305257\\
47.6695717564315	0.246815542806285\\
47.7541523326296	0.246818047518516\\
47.8393018049434	0.24682054727315\\
47.9244512772573	0.246823025353818\\
48.0096007495711	0.246825481962478\\
48.094750221885	0.246827917298979\\
48.1804762463428	0.246830347836805\\
48.2662022708007	0.246832757213579\\
48.3519282952586	0.246835145626929\\
48.4376543197165	0.246837513272416\\
48.5239647071256	0.246839876272523\\
48.6102750945348	0.246842218612291\\
48.6965854819439	0.246844540485137\\
48.7828958693531	0.246846842082447\\
48.8697985896823	0.246849139182265\\
48.9567013100115	0.246851416110623\\
49.0436040303407	0.246853673056834\\
49.1305067506699	0.24685591020823\\
49.2180099370951	0.246858143004815\\
49.3055131235204	0.246860356107333\\
49.3930163099456	0.246862549701108\\
49.4805194963708	0.246864723969519\\
49.5686314502397	0.246866894020809\\
49.6567434041087	0.24686904484428\\
49.7448553579776	0.24687117662137\\
49.8329673118465	0.246873289531616\\
49.9216965075321	0.24687539835763\\
50.0104257032176	0.246877488411258\\
50.0991548989032	0.246879559870149\\
50.1878840945888	0.246881612910103\\
50.2772391829867	0.246883661994071\\
50.3665942713846	0.246885692750582\\
50.4559493597825	0.246887705353598\\
50.5453044481804	0.246889699975276\\
50.6352942631106	0.246891690764784\\
50.7252840780408	0.246893663661562\\
50.8152738929709	0.246895618835987\\
50.9052637079011	0.246897556456664\\
50.9958972708063	0.246899490364713\\
51.0865308337115	0.246901406804855\\
51.1771643966166	0.246903305943972\\
51.2677979595218	0.246905187947206\\
51.3590844844193	0.24690706635324\\
51.4503710093168	0.246908927706567\\
51.5416575342142	0.246910772170664\\
51.6329440591117	0.246912599907297\\
51.7248929579717	0.246914424158196\\
51.8168418568316	0.246916231762239\\
51.9087907556916	0.246918022879583\\
52.0007396545516	0.2469197976687\\
52.0933605443554	0.246921569079767\\
52.1859814341592	0.246923324240745\\
52.278602323963	0.246925063308548\\
52.3712232137668	0.246926786438435\\
52.4645259207004	0.246928506294289\\
52.557828627634	0.246930210287986\\
52.6511313345676	0.246931898573277\\
52.7444340415012	0.246933571302288\\
52.8384286082082	0.246935240857775\\
52.9324231749152	0.246936894930447\\
53.0264177416223	0.24693853367097\\
53.1204123083293	0.246940157228408\\
53.2151089998621	0.246941777709453\\
53.3098056913948	0.246943383078672\\
53.4045023829275	0.246944973483716\\
53.4991990744603	0.246946549070663\\
53.5946083849179	0.246948121675099\\
53.6900176953756	0.24694967953057\\
53.7854270058332	0.246951222781783\\
53.8808363162909	0.246952751571904\\
53.9769689760186	0.246954277470276\\
54.0731016357463	0.246955788974634\\
54.169234295474	0.24695728622681\\
54.2653669552017	0.246958769367131\\
54.3622339372453	0.246960249703454\\
54.4591009192888	0.246961715993015\\
54.5559679013324	0.246963168374841\\
54.652834883376	0.246964606986483\\
54.7504474111939	0.246966042878988\\
54.8480599390117	0.246967465064489\\
54.9456724668296	0.246968873679273\\
55.0432849946475	0.246970268858182\\
55.1416545496904	0.246971661400031\\
55.2400241047332	0.246973040567334\\
55.338393659776	0.246974406493707\\
55.4367632148189	0.246975759311341\\
55.5359015449639	0.246977109571321\\
55.635039875109	0.246978446782107\\
55.734178205254	0.246979771074703\\
55.8333165353991	0.246981082578707\\
55.9332356621868	0.24698239160188\\
56.0331547889744	0.24698368789429\\
56.1330739157621	0.246984971584382\\
56.2329930425498	0.24698624279922\\
56.3337052708769	0.246987511607581\\
56.434417499204	0.24698876799686\\
56.5351297275311	0.246990012093004\\
56.6358419558582	0.246991244020592\\
56.7373598818851	0.24699247361367\\
56.8388778079119	0.246993691092772\\
56.9403957339388	0.246994896581394\\
57.0419136599657	0.246996090201689\\
57.1442501810399	0.246997281557152\\
57.246586702114	0.24699846109733\\
57.3489232231882	0.246999628943324\\
57.4512597442624	0.247000785214905\\
57.5544280683576	0.247001939289132\\
57.6575963924529	0.247003081840507\\
57.7607647165482	0.247004212987784\\
57.8639330406435	0.247005332848413\\
57.9679466963614	0.247006450577041\\
58.0719603520793	0.247007557069147\\
58.1759740077972	0.247008652441187\\
58.2799876635152	0.247009736808338\\
58.3848605103868	0.247010819106798\\
58.4897333572585	0.247011890449109\\
58.5946062041302	0.247012950949476\\
58.6994790510019	0.247014000720854\\
58.8052252907196	0.247015048484886\\
58.9109715304374	0.247016085567325\\
59.0167177701551	0.247017112080178\\
59.1224640098728	0.24701812813422\\
59.2290981971279	0.247019142240364\\
59.3357323843829	0.247020145933795\\
59.442366571638	0.247021139324365\\
59.549000758893	0.247022122520715\\
59.6617505691698	0.247023151135876\\
59.7745003794465	0.247024168599609\\
59.8872501897232	0.247025175035998\\
60	0.247026170567671\\
};
\addlegendentry{Suscettibili}

\addplot [color=mycolor2, line width=2.0pt]
  table[row sep=crcr]{%
0	0\\
0.00199053585276749	4.72799014565942e-11\\
0.00398107170553497	3.77907153270227e-10\\
0.00597160755830246	1.27430926661481e-09\\
0.00796214341106994	3.01791596313966e-09\\
0.0143083073603051	1.74646430052203e-08\\
0.0206544713095403	5.23861599360104e-08\\
0.0270006352587754	1.16700628138497e-07\\
0.0333467992080106	2.19225715102078e-07\\
0.0396935645593046	3.68695550402587e-07\\
0.0460403299105986	5.73722624170811e-07\\
0.0523870952618926	8.42827954771613e-07\\
0.0587338606131866	1.18443458776291e-06\\
0.0650961133419323	1.60800519713108e-06\\
0.0714583660706779	2.12108815650022e-06\\
0.0778206187994235	2.73187638252792e-06\\
0.0841828715281692	3.44846630640286e-06\\
0.0905606411731996	4.28103059224583e-06\\
0.0969384108182301	5.23591955222183e-06\\
0.103316180463261	6.32100358598925e-06\\
0.109693950108291	7.54405864864666e-06\\
0.116087318034554	8.91629993741289e-06\\
0.122480685960817	1.04425552208895e-05\\
0.12887405388708	1.21303782473561e-05\\
0.135267421813343	1.39872305500097e-05\\
0.141676469438281	1.60256913929319e-05\\
0.14808551706322	1.8248751383898e-05\\
0.154494564688158	2.06636535250299e-05\\
0.160903612313097	2.32775505015215e-05\\
0.167328422015208	2.61047003348377e-05\\
0.173753231717319	2.91459489877091e-05\\
0.18017804141943	3.24082343345494e-05\\
0.186602851121541	3.58984058682413e-05\\
0.193043505744106	3.96327073343763e-05\\
0.199484160366671	4.36095401823641e-05\\
0.205924814989236	4.7835542575772e-05\\
0.212365469611801	5.2317266522682e-05\\
0.218822053394839	5.70732413931949e-05\\
0.225278637177877	6.20991312548641e-05\\
0.231735220960914	6.74012799010693e-05\\
0.238191804743952	7.29859472576311e-05\\
0.244664401671674	7.887424008933e-05\\
0.251136998599397	8.50588000421426e-05\\
0.25760959552712	9.15456820815702e-05\\
0.264082192454842	9.8340859266785e-05\\
0.270570887963067	0.000105468301927217\\
0.277059583471292	0.000112917349571542\\
0.283548278979517	0.00012069377419878\\
0.290036974487742	0.00012880326715882\\
0.296541854059766	0.000137272939831456\\
0.303046733631789	0.000146088558359708\\
0.309551613203813	0.000155255617166533\\
0.316056492775836	0.000164779531326223\\
0.322577643612098	0.000174690825899354\\
0.32909879444836	0.000184971427298022\\
0.335619945284621	0.000195626557350753\\
0.342141096120883	0.000206661360560453\\
0.348678605036902	0.00021811003843171\\
0.35521611395292	0.000229950420609526\\
0.361753622868939	0.000242187461589993\\
0.368291131784958	0.000254826040397241\\
0.374845086704082	0.000267904293621793\\
0.381399041623206	0.000281395701691875\\
0.387952996542331	0.000295304956862558\\
0.394506951461455	0.000309636677522804\\
0.401077441531393	0.000324433187993828\\
0.407647931601331	0.000339663373282534\\
0.414218421671268	0.000355331668181094\\
0.420788911741206	0.00037144243558071\\
0.427376026319146	0.000388042430669644\\
0.433963140897085	0.000405095705995007\\
0.440550255475025	0.000422606443584092\\
0.447137370052964	0.000440578755657922\\
0.453741199367022	0.000459064066375706\\
0.46034502868108	0.000478021363771022\\
0.466948857995137	0.000497454582028711\\
0.473552687309195	0.000517367587298923\\
0.480173322720616	0.00053781670539374\\
0.486793958132037	0.000558755632787428\\
0.493414593543458	0.000580188061170245\\
0.500035228954879	0.000602117615193881\\
0.506672761762064	0.000624605744479367\\
0.513310294569248	0.000647600637211143\\
0.519947827376433	0.000671105747337031\\
0.526585360183617	0.000695124462996105\\
0.533239883483442	0.000719723579498482\\
0.539894406783268	0.000744845560336771\\
0.546548930083094	0.00077049362616971\\
0.553203453382919	0.000796670933762927\\
0.559875060934079	0.000823449836183099\\
0.566546668485238	0.000850766865292951\\
0.573218276036397	0.00087862501323463\\
0.579889883587557	0.000907027209745642\\
0.586578669168294	0.000936051569126101\\
0.593267454749031	0.00096562849368248\\
0.599956240329768	0.000995760751447777\\
0.606645025910505	0.00102645104966895\\
0.613351084619436	0.00105778346613869\\
0.620057143328367	0.00108968207709115\\
0.626763202037298	0.0011221494306749\\
0.633469260746229	0.00115518801617506\\
0.640192689007476	0.00118888807109871\\
0.646916117268722	0.00122316715500856\\
0.653639545529968	0.00125802760046648\\
0.660362973791214	0.00129347168295788\\
0.667103867904104	0.00132959598295379\\
0.673844762016994	0.00136631136505116\\
0.680585656129883	0.00140361995111346\\
0.687326550242773	0.00144152380691931\\
0.694085008124096	0.00148012604185116\\
0.700843466005419	0.00151933064468227\\
0.707601923886742	0.00155913953109446\\
0.714360381768066	0.00159955456163341\\
0.721136501534817	0.00164068554658001\\
0.727912621301569	0.00168242943159652\\
0.73468874106832	0.00172478793017403\\
0.741464860835071	0.00176776270244125\\
0.748258742408364	0.00181147043547589\\
0.755052623981657	0.00185580086067638\\
0.761846505554949	0.00190075549371921\\
0.768640387128242	0.00194633579826292\\
0.775452130474897	0.00199266549789481\\
0.782263873821551	0.00203962695508664\\
0.789075617168206	0.00208722149169571\\
0.79588736051486	0.00213545037908192\\
0.80271706685068	0.00218444453855399\\
0.809546773186499	0.00223407880772383\\
0.816376479522319	0.0022843543184525\\
0.823206185858138	0.00233527215390333\\
0.830053957332632	0.00238697058622594\\
0.836901728807126	0.002439316780116\\
0.84374950028162	0.00249231168145403\\
0.850597271756113	0.00254595618896228\\
0.857463211448599	0.00260039607343523\\
0.864329151141086	0.00265549068340767\\
0.871195090833572	0.00271124078342397\\
0.878061030526058	0.00276764709153568\\
0.884945241937476	0.00282486301471888\\
0.891829453348894	0.00288273995226035\\
0.898713664760312	0.00294127849123155\\
0.90559787617173	0.00300047917342799\\
0.912500464439821	0.00306050318391779\\
0.919403052707912	0.00312119383516751\\
0.926305640976003	0.00318255154062305\\
0.933208229244093	0.00324457667000333\\
0.940129300503941	0.00330743831824606\\
0.947050371763788	0.00337097158363905\\
0.953971443023636	0.00343517670993526\\
0.960892514283483	0.00350005389834762\\
0.967832174888143	0.00356578027301552\\
0.974771835492802	0.00363218260356449\\
0.981711496097461	0.00369926096759168\\
0.988651156702121	0.00376701540177599\\
0.995609514344775	0.00383563118267774\\
1.00256787198743	0.00390492663289638\\
1.00952622963008	0.00397490166742256\\
1.01648458727274	0.00404555616193833\\
1.0234617510246	0.00411708366446378\\
1.03043891477646	0.004189293936007\\
1.03741607852832	0.00426218673319945\\
1.04439324228018	0.00433576177415446\\
1.05138932194451	0.00441022098532712\\
1.05838540160883	0.00448536546328207\\
1.06538148127316	0.00456119481021056\\
1.07237756093749	0.00463770859058654\\
1.07939266704796	0.00471511721042944\\
1.08640777315843	0.0047932130046752\\
1.0934228792689	0.00487199542449218\\
1.10043798537936	0.00495146388489276\\
1.10747222983544	0.00503183737457141\\
1.11450647429151	0.00511289936772885\\
1.12154071874758	0.00519464916826401\\
1.12857496320366	0.0052770860449991\\
1.13562845894337	0.00536043766436001\\
1.14268195468308	0.00544447854947083\\
1.14973545042278	0.0055292078602614\\
1.15678894616249	0.00561462472308405\\
1.16386180727142	0.00570096557234967\\
1.17093466838034	0.00578799589442269\\
1.17800752948927	0.00587571470854692\\
1.18508039059819	0.00596412100191235\\
1.19217273186328	0.00605346005682872\\
1.19926507312837	0.00614348824705696\\
1.20635741439346	0.0062342044551762\\
1.21344975565855	0.00632560753239808\\
1.22056169301535	0.00641795168889853\\
1.22767363037216	0.00651098410979861\\
1.23478556772896	0.00660470354466636\\
1.24189750508577	0.00669910871250329\\
1.2490291561515	0.00679446283328328\\
1.25616080721723	0.00689050382536449\\
1.26329245828296	0.00698723030854659\\
1.27042410934869	0.00708464087349153\\
1.27757559222201	0.00718300781139913\\
1.28472707509533	0.00728205971610207\\
1.29187855796865	0.00738179508111225\\
1.29903004084196	0.00748221237182511\\
1.30620147471517	0.00758359301557748\\
1.31337290858837	0.00768565622130957\\
1.32054434246157	0.00778840035929312\\
1.32771577633477	0.00789182377314834\\
1.33490728210935	0.00799621709425868\\
1.34209878788392	0.00810129008317423\\
1.3492902936585	0.00820704099009206\\
1.35648179943308	0.00831346803988425\\
1.36369349872651	0.0084208711172481\\
1.37090519801994	0.00852895048876612\\
1.37811689731337	0.00863770428854782\\
1.3853285966068	0.00874713062576249\\
1.39256061231053	0.00885753869045064\\
1.39979262801426	0.00896861920694997\\
1.40702464371799	0.00908037019638369\\
1.41425665942172	0.00919278965609394\\
1.42150911582444	0.00930619613065348\\
1.42876157222716	0.00942027075635332\\
1.43601402862988	0.00953501144462254\\
1.44326648503259	0.00965041608416978\\
1.45053950730217	0.00976681261120038\\
1.45781252957175	0.00988387254081841\\
1.46508555184133	0.0100015936778128\\
1.47235857411091	0.0101199738054342\\
1.47965228878223	0.0102393502910859\\
1.48694600345356	0.0103593849934641\\
1.49423971812488	0.0104800756135585\\
1.5015334327962	0.010601419832232\\
1.50884796768823	0.0107237644814479\\
1.51616250258027	0.0108467617340742\\
1.5234770374723	0.0109704091907344\\
1.53079157236433	0.0110947044326963\\
1.53812705668821	0.0112200037871081\\
1.54546254101209	0.0113459497139142\\
1.55279802533598	0.0114725397169169\\
1.56013350965986	0.0115997712810649\\
1.56749007329995	0.0117280102419046\\
1.57484663694003	0.0118568893365293\\
1.58220320058012	0.0119864059747353\\
1.5895597642202	0.0121165575487741\\
1.59693753927215	0.0122477194383679\\
1.60431531432409	0.0123795146255574\\
1.61169308937604	0.0125119404292633\\
1.61907086442798	0.0126449941517678\\
1.6264699837717	0.0127790607237087\\
1.63386910311542	0.0129137533695293\\
1.64126822245913	0.0130490693199659\\
1.64866734180285	0.0131850057904467\\
1.65608793943564	0.0133219572693538\\
1.66350853706843	0.0134595272206364\\
1.67092913470122	0.0135977127898017\\
1.67834973233401	0.0137365111080824\\
1.68579194412116	0.0138763262382593\\
1.69323415590832	0.0140167518705993\\
1.70067636769548	0.0141577850690549\\
1.70811857948264	0.0142994228835387\\
1.71558254216844	0.0144420789445054\\
1.72304650485425	0.0145853371785578\\
1.73051046754005	0.0147291945707121\\
1.73797443022586	0.0148736480932099\\
1.74546028230335	0.0150191209496496\\
1.75294613438084	0.0151651873006857\\
1.76043198645833	0.0153118440554348\\
1.76791783853582	0.0154590881110582\\
1.77542572012108	0.0156073522440112\\
1.78293360170633	0.0157562008534436\\
1.79044148329159	0.0159056307751525\\
1.79794936487685	0.0160556388342143\\
1.80547941694215	0.0162066673589973\\
1.81300946900745	0.016358271011936\\
1.82053952107275	0.0165104465582266\\
1.82806957313806	0.0166631907534093\\
1.8356219382731	0.0168169554678544\\
1.84317430340815	0.0169712856405282\\
1.85072666854319	0.0171261779695923\\
1.85827903367824	0.0172816291437258\\
1.86585385599785	0.0174381005588641\\
1.87342867831746	0.0175951274502292\\
1.88100350063707	0.0177527064514709\\
1.88857832295669	0.0179108341879357\\
1.89617574793629	0.0180699815571103\\
1.90377317291589	0.0182296741174673\\
1.9113705978955	0.0183899084410382\\
1.9189680228751	0.018550681092323\\
1.92658819822989	0.0187124724615114\\
1.93420837358469	0.0188747984432962\\
1.94182854893948	0.0190376555505466\\
1.94944872429427	0.019201040289769\\
1.95709179801755	0.0193654424880635\\
1.96473487174083	0.0195303684358045\\
1.97237794546411	0.0196958145893096\\
1.98002101918739	0.0198617773995098\\
1.98768714199827	0.020028756123144\\
1.99535326480915	0.0201962474568953\\
2.00301938762003	0.0203642478039628\\
2.01068551043091	0.0205327535622976\\
2.0183748340895	0.0207022733701218\\
2.02606415774808	0.0208722943816663\\
2.03375348140667	0.0210428129493908\\
2.04144280506525	0.0212138254216537\\
2.0491554826466	0.021385849771759\\
2.05686816022795	0.0215583636608184\\
2.06458083780929	0.0217313633933021\\
2.07229351539064	0.0219048452703136\\
2.08002970237887	0.0220793365749429\\
2.08776588936711	0.0222543055045831\\
2.09550207635534	0.0224297483180251\\
2.10323826334358	0.0226056612718203\\
2.11099811595816	0.0227825808907797\\
2.11875796857274	0.0229599659801992\\
2.12651782118732	0.0231378127558294\\
2.1342776738019	0.023316117431939\\
2.14206135075448	0.0234954257431382\\
2.14984502770706	0.0236751871377065\\
2.15762870465964	0.0238553977915527\\
2.16541238161222	0.0240360538793355\\
2.17322004315878	0.024217710287562\\
2.18102770470534	0.0243998071683618\\
2.1888353662519	0.0245823406600865\\
2.19664302779846	0.0247653069008661\\
2.2044748353169	0.0249492698576359\\
2.21230664283533	0.0251336604608934\\
2.22013845035377	0.025318474813969\\
2.2279702578722	0.0255037090207653\\
2.23582637510447	0.0256899360824713\\
2.24368249233675	0.0258765777581578\\
2.25153860956902	0.02606363011831\\
2.25939472680129	0.026251089235052\\
2.26727531955162	0.0264395370862444\\
2.27515591230195	0.0266283863205045\\
2.28303650505228	0.02681763297836\\
2.29091709780261	0.0270072731023098\\
2.29882233260671	0.0271978975534342\\
2.30672756741082	0.0273889099662508\\
2.31463280221492	0.0275803063539594\\
2.32253803701902	0.0277720827323878\\
2.33046808291417	0.0279648387906853\\
2.33839812880931	0.0281579692072307\\
2.34632817470446	0.0283514699702926\\
2.3542582205996	0.0285453370714956\\
2.36221324911394	0.028740178970004\\
2.37016827762829	0.0289353814496741\\
2.37812330614264	0.0291309404760557\\
2.38607833465698	0.0293268520189747\\
2.39405851771369	0.0295237331935735\\
2.40203870077039	0.0297209610068358\\
2.4100188838271	0.0299185314037686\\
2.4179990668838	0.0301164403345388\\
2.42600457974877	0.0303153135249889\\
2.43401009261373	0.0305145192536338\\
2.4420156054787	0.0307140534480064\\
2.45002111834367	0.0309139120408262\\
2.45805213720175	0.0311147292592627\\
2.46608315605982	0.0313158647655521\\
2.4741141749179	0.0315173144718403\\
2.48214519377598	0.031719074296428\\
2.4902018978345	0.0319217869073294\\
2.49825860189303	0.0321248034138389\\
2.50631530595155	0.0323281197150762\\
2.51437201001007	0.0325317317169392\\
2.52245457948208	0.0327362904140657\\
2.53053714895408	0.0329411384807706\\
2.53861971842609	0.033146271805123\\
2.54670228789809	0.0333516862829111\\
2.55481090545363	0.0335580411530441\\
2.56291952300918	0.0337646707405329\\
2.57102814056472	0.0339715709249319\\
2.57913675812026	0.0341787375938589\\
2.58727160888194	0.0343868381446574\\
2.59540645964362	0.0345951986417231\\
2.6035413104053	0.0348038149586101\\
2.61167616116698	0.0350126829774342\\
2.61983743200028	0.0352224781460633\\
2.62799870283359	0.035432518378779\\
2.63615997366689	0.0356427995452644\\
2.6443212445002	0.0358533175244151\\
2.65250912437636	0.0360647557131178\\
2.66069700425252	0.0362764239804051\\
2.66888488412869	0.0364883181940825\\
2.67707276400485	0.0367004342319819\\
2.6852874441124	0.0369134633370326\\
2.69350212421995	0.0371267074395732\\
2.7017168043275	0.0373401624075429\\
2.70993148443506	0.0375538241196178\\
2.71817315820996	0.0377683915590409\\
2.72641483198486	0.037983158825964\\
2.73465650575977	0.0381981217912802\\
2.74289817953467	0.038413276336612\\
2.75116704265512	0.038629329075731\\
2.75943590577556	0.0388455663910873\\
2.767704768896	0.0390619841583687\\
2.77597363201644	0.0392785782648274\\
2.78426988288509	0.0394960628544471\\
2.79256613375373	0.0397137166951148\\
2.80086238462237	0.0399315356693751\\
2.80915863549101	0.0401495156719584\\
2.81748247408502	0.0403683782432132\\
2.82580631267902	0.0405873946739431\\
2.83413015127303	0.0408065608553297\\
2.84245398986704	0.0410258726915265\\
2.85080561880438	0.0412460590092757\\
2.85915724774172	0.041466383735183\\
2.86750887667906	0.0416868427716482\\
2.8758605056164	0.0419074320340475\\
2.88424013037944	0.042128887527914\\
2.89261975514247	0.0423504659259175\\
2.90099937990551	0.0425721631436085\\
2.90937900466855	0.0427939751102333\\
2.91778683316272	0.0430166448874225\\
2.9261946616569	0.0432394220191022\\
2.93460249015107	0.0434623024359441\\
2.94301031864525	0.0436852820828124\\
2.95144656047954	0.0439091109328847\\
2.95988280231384	0.0441330315493985\\
2.96831904414814	0.0443570398798326\\
2.97675528598243	0.0445811318866012\\
2.98522015378253	0.0448060643402735\\
2.99368502158263	0.0450310729406097\\
3.00214988938273	0.0452561536540891\\
3.01061475718283	0.0454813024623177\\
3.01910846625213	0.0457072828068489\\
3.02760217532142	0.0459333236529075\\
3.03609588439071	0.0461594209881028\\
3.04458959346	0.0463855708156022\\
3.05311236220025	0.0466125431291308\\
3.0616351309405	0.0468395602805861\\
3.07015789968076	0.0470666182804821\\
3.07868066842101	0.0472937131553884\\
3.0872327162858	0.0475216212752649\\
3.0957847641506	0.0477495585578475\\
3.10433681201539	0.0479775210381874\\
3.11288885988019	0.0482055047680521\\
3.12147041092284	0.0484342924079806\\
3.13005196196549	0.0486630935303701\\
3.13863351300815	0.0488919041966748\\
3.1472150640508	0.0491207204854145\\
3.15582634421903	0.049350331186709\\
3.16443762438727	0.0495799396910159\\
3.1730489045555	0.0498095420884455\\
3.18166018472374	0.0500391344863356\\
3.19030142291735	0.0502695116679625\\
3.19894266111096	0.0504998709806079\\
3.20758389930457	0.0507302085446074\\
3.21622513749818	0.0509605204980501\\
3.22489656499851	0.0511916074631078\\
3.23356799249884	0.0514226609011219\\
3.24223941999916	0.0516536769642236\\
3.25091084749949	0.0518846518228718\\
3.25961269965928	0.052116391824943\\
3.26831455181906	0.0523480826621914\\
3.27701640397885	0.0525797205201907\\
3.28571825613863	0.0528113016032468\\
3.29445076992401	0.0530436378029415\\
3.30318328370939	0.0532759092257646\\
3.31191579749477	0.0535081120929681\\
3.32064831128015	0.053740242644532\\
3.32941172759282	0.0539731181920293\\
3.33817514390549	0.0542059133826568\\
3.34693856021816	0.0544386244747517\\
3.35570197653083	0.0546712477459167\\
3.36449653919211	0.0549046057742641\\
3.37329110185339	0.0551378679039036\\
3.38208566451467	0.0553710304317452\\
3.39088022717596	0.0556040896744612\\
3.39970618282859	0.0558378733169533\\
3.40853213848122	0.0560715455631223\\
3.41735809413385	0.0563051027499448\\
3.42618404978648	0.0565385412345527\\
3.43504164855909	0.0567726936630706\\
3.44389924733171	0.0570067192469562\\
3.45275684610433	0.0572406143653906\\
3.46161444487695	0.0574743754176178\\
3.47050394016004	0.0577088398560237\\
3.47939343544313	0.0579431620567546\\
3.48828293072622	0.0581773384424692\\
3.49717242600932	0.0584113654564082\\
3.50609407425931	0.0586460851943784\\
3.5150157225093	0.0588806473626616\\
3.52393737075929	0.0591150484287761\\
3.53285901900928	0.0593492848812542\\
3.54181308032811	0.0595842033081444\\
3.55076714164694	0.0598189489000523\\
3.55972120296577	0.0600535181707326\\
3.56867526428461	0.0602879076552939\\
3.57766200230417	0.0605229682763009\\
3.58664874032374	0.0607578408684799\\
3.5956354783433	0.0609925219938261\\
3.60462221636287	0.0612270082355607\\
3.61364189770245	0.0614621546755037\\
3.62266157904204	0.061697097969839\\
3.63168126038162	0.0619318347299487\\
3.64070094172121	0.0621663615889171\\
3.64975383695641	0.0624015376351474\\
3.65880673219161	0.0626364955014402\\
3.66785962742681	0.0628712318498231\\
3.67691252266202	0.0631057433644016\\
3.6859989062531	0.0633408929819267\\
3.69508528984418	0.0635758094729067\\
3.70417167343526	0.0638104895513025\\
3.71325805702634	0.064044929953411\\
3.72237820615531	0.064279997272616\\
3.73149835528429	0.064514816610814\\
3.74061850441326	0.0647493847357475\\
3.74973865354223	0.0649836984373682\\
3.75889285078234	0.0652186278394462\\
3.76804704802246	0.065453294503356\\
3.77720124526257	0.0656876952516368\\
3.78635544250268	0.0659218269294518\\
3.79554397276389	0.0661565629834867\\
3.80473250302509	0.0663910216446047\\
3.8139210332863	0.0666251997912691\\
3.8231095635475	0.0668590943248902\\
3.83233271658159	0.0670935818680765\\
3.84155586961568	0.0673277774707864\\
3.85077902264978	0.067561678068635\\
3.86000217568387	0.0677952806203304\\
3.86926024490616	0.0680294647439086\\
3.87851831412845	0.0682633424906697\\
3.88777638335074	0.0684969108550357\\
3.89703445257303	0.0687301668544401\\
3.90632773542166	0.0689639929277076\\
3.91562101827029	0.0691974983039333\\
3.92491430111892	0.0694306800372338\\
3.93420758396755	0.0696635352050762\\
3.94353638235509	0.0698969489011061\\
3.95286518074264	0.0701300277006061\\
3.96219397913019	0.070362768718381\\
3.97152277751774	0.0705951690928563\\
3.9808873974673	0.0708281163934308\\
3.99025201741687	0.071060714723041\\
3.99961663736644	0.0712929612583941\\
4.008981257316	0.0715248531998056\\
4.01838200905393	0.0717572804124348\\
4.02778276079185	0.071989344708432\\
4.03718351252977	0.0722210433277955\\
4.0465842642677	0.0724523735341545\\
4.05602146260214	0.0726842273144917\\
4.06545866093658	0.0729157043658008\\
4.07489585927102	0.0731468019921486\\
4.08433305760547	0.073377517521486\\
4.09380702196714	0.073608744886918\\
4.10328098632881	0.0738395818484242\\
4.11275495069048	0.0740700257750078\\
4.12222891505216	0.0743000740597667\\
4.13173996944755	0.0745306224009146\\
4.14125102384295	0.0747607668044622\\
4.15076207823835	0.074990504705586\\
4.16027313263375	0.0752198335633647\\
4.16982160537755	0.0754496506504891\\
4.17937007812135	0.0756790504110751\\
4.18891855086514	0.075908030347525\\
4.19846702360894	0.0761365879863025\\
4.20805324837695	0.0763656220060105\\
4.21763947314496	0.0765942254590292\\
4.22722569791297	0.0768223959156816\\
4.23681192268098	0.0770501309705143\\
4.24643623800907	0.0772783305244313\\
4.25606055333716	0.0775060864240955\\
4.26568486866525	0.077733396308507\\
4.27530918399335	0.0779602578410281\\
4.28497193298002	0.0781875719500867\\
4.29463468196669	0.0784144294730715\\
4.30429743095337	0.078640828118907\\
4.31396017994004	0.0788667656205423\\
4.32366171143296	0.0790931437638513\\
4.33336324292588	0.0793190525482698\\
4.3430647744188	0.0795444897533583\\
4.35276630591172	0.0797694531829451\\
4.3625069732434	0.0799948452774314\\
4.37224764057509	0.0802197554029102\\
4.38198830790678	0.0804441814101964\\
4.39172897523846	0.080668121174463\\
4.40150913802251	0.080892477626092\\
4.41128930080656	0.0811163396645292\\
4.42106946359061	0.0813397052125153\\
4.43084962637466	0.0815625722171993\\
4.44066964916844	0.0817858438993123\\
4.45048967196222	0.0820086088925984\\
4.460309694756	0.0822308651929091\\
4.47012971754978	0.0824526108201015\\
4.47998997038358	0.0826747490945324\\
4.48985022321737	0.082896368576192\\
4.49971047605117	0.0831174673344742\\
4.50957072888496	0.0833380434630345\\
4.51947158785354	0.0835590002011333\\
4.52937244682212	0.0837794262172988\\
4.5392733057907	0.0839993196550023\\
4.54917416475928	0.0842186786820054\\
4.55911601157961	0.0844384062620897\\
4.56905785839994	0.0846575913687126\\
4.57899970522028	0.0848762322200282\\
4.58894155204061	0.0850943270584299\\
4.59892477418835	0.0853127783769141\\
4.6089079963361	0.0855306756501643\\
4.61889121848384	0.0857480171720618\\
4.62887444063159	0.0859648012603389\\
4.6388994324499	0.0861819297625396\\
4.64892442426821	0.0863984928302733\\
4.65894941608653	0.08661448883338\\
4.66897440790484	0.086829916165757\\
4.67904156948105	0.0870456758269428\\
4.68910873105727	0.0872608588496626\\
4.69917589263348	0.0874754636801547\\
4.7092430542097	0.0876894887886827\\
4.71935279189367	0.0879038341314388\\
4.72946252957763	0.0881175918192988\\
4.7395722672616	0.0883307603754495\\
4.74968200494557	0.0885433383469416\\
4.75983473174927	0.0887562244557465\\
4.76998745855297	0.0889685120824857\\
4.78014018535667	0.0891801998281437\\
4.79029291216037	0.0893912863172294\\
4.8004890483408	0.0896026688549015\\
4.81068518452122	0.0898134422750685\\
4.82088132070165	0.0900236052565984\\
4.83107745688208	0.0902331565020178\\
4.84131742876219	0.0904429916890767\\
4.85155740064229	0.0906522073171748\\
4.8617973725224	0.0908608021434059\\
4.87203734440251	0.0910687749484308\\
4.88232158523569	0.0912770195855395\\
4.89260582606888	0.0914846344180471\\
4.90289006690206	0.091691618281766\\
4.91317430773525	0.0918979700358085\\
4.92350325885451	0.0921045815309449\\
4.93383220997378	0.0923105531729054\\
4.94416116109305	0.0925158838768549\\
4.95449011221231	0.0927205725809713\\
4.96486422170276	0.0929255089246485\\
4.97523833119322	0.0931297955658618\\
4.98561244068367	0.0933334314991031\\
4.99598655017412	0.0935364157419368\\
5.00640627395839	0.0937396355313924\\
5.01682599774266	0.0939421959700463\\
5.02724572152694	0.0941440961319591\\
5.03766544531121	0.0943453351141175\\
5.04813124666974	0.0945467975458887\\
5.05859704802827	0.0947475911808598\\
5.0690628493868	0.0949477151730837\\
5.07952865074533	0.0951471686991835\\
5.09004100143523	0.0953468335928924\\
5.10055335212514	0.0955458204470955\\
5.11106570281504	0.0957441284962728\\
5.12157805350494	0.095941756997218\\
5.13213743298625	0.0961395847809123\\
5.14269681246755	0.0963367254874756\\
5.15325619194885	0.0965331784316918\\
5.16381557143015	0.0967289429506507\\
5.17442246764722	0.096924894676164\\
5.1850293638643	0.0971201504932415\\
5.19563626008137	0.0973147097971112\\
5.20624315629844	0.0975085720051089\\
5.21689806557204	0.0977026093461527\\
5.22755297484563	0.0978959421547\\
5.23820788411923	0.0980885699067569\\
5.24886279339282	0.0982804921000182\\
5.25956622085953	0.098472577360354\\
5.27026964832624	0.0986639496720412\\
5.28097307579295	0.0988546085921423\\
5.29167650325966	0.0990445536991423\\
5.30242896340108	0.0992346498204097\\
5.31318142354251	0.0994240247860623\\
5.32393388368394	0.0996126782339932\\
5.33468634382537	0.0998006098234629\\
5.34548835981846	0.0999886803718969\\
5.35629037581155	0.100176021767848\\
5.36709239180464	0.10036263373008\\
5.37789440779773	0.10054851599848\\
5.38874651272417	0.100734525184612\\
5.39959861765061	0.100919797431937\\
5.41045072257705	0.101104332540315\\
5.42130282750349	0.10128813033027\\
5.43220556367259	0.101472042995029\\
5.44310829984169	0.101655211145557\\
5.45401103601079	0.101837634662983\\
5.46491377217989	0.102019313448783\\
5.47586769235055	0.102201095081008\\
5.48682161252121	0.102382124835348\\
5.49777553269188	0.102562402673864\\
5.50872945286254	0.102741928578882\\
5.51973511980852	0.102921545304132\\
5.5307407867545	0.103100403000091\\
5.54174645370048	0.103278501709697\\
5.55275212064646	0.103455841495863\\
5.5638101076295	0.103633260080617\\
5.57486809461254	0.103809912697085\\
5.58592608159558	0.103985799469173\\
5.59698406857862	0.104160920540296\\
5.60809495979115	0.104336108395108\\
5.61920585100368	0.104510523554911\\
5.63031674221621	0.104684166224707\\
5.64142763342874	0.104857036628586\\
5.65259202401047	0.105029961802848\\
5.66375641459219	0.105202107768235\\
5.67492080517392	0.105373474810388\\
5.68608519575564	0.105544063233954\\
5.69730369280843	0.105714694426086\\
5.70852218986121	0.105884540108346\\
5.719740686914	0.10605360064686\\
5.73095918396678	0.106221876426433\\
5.74223240540217	0.106390182960249\\
5.75350562683755	0.106557697896043\\
5.76477884827294	0.106724421680372\\
5.77605206970832	0.106890354778023\\
5.78738064638032	0.107056306629403\\
5.79870922305231	0.107221461006798\\
5.81003779972431	0.10738581843732\\
5.8213663763963	0.107549379465748\\
5.83275095107969	0.10771294723955\\
5.84413552576308	0.107875711875847\\
5.85552010044646	0.108037673981745\\
5.86690467512985	0.108198834181947\\
5.87834590369998	0.108359989121832\\
5.88978713227011	0.108520335472866\\
5.90122836084025	0.108679873921891\\
5.91266958941038	0.108838605172988\\
5.92416814082324	0.108997319152954\\
5.9356666922361	0.109155219304474\\
5.94716524364896	0.109312306393927\\
5.95866379506182	0.109468581204504\\
5.97022035244649	0.109624826735938\\
5.98177690983116	0.109780253410084\\
5.99333346721583	0.109934862072972\\
6.00489002460049	0.11008865358674\\
6.01650528373054	0.110242403788291\\
6.02812054286059	0.110395330314413\\
6.03973580199063	0.110547434090178\\
6.05135106112068	0.110698716056689\\
6.06302573298457	0.110849944679801\\
6.07470040484846	0.111000345019439\\
6.08637507671235	0.111149918079341\\
6.09804974857625	0.111298664878911\\
6.10978455915361	0.11144734629402\\
6.12151936973097	0.111595195026979\\
6.13325418030833	0.111742212159869\\
6.14498899088569	0.111888398790027\\
6.15678468114828	0.112034507976428\\
6.16858037141086	0.112179780290256\\
6.18037606167345	0.11232421689195\\
6.19217175193603	0.112467818956461\\
6.20402907883544	0.112611331503843\\
6.21588640573484	0.112754003195663\\
6.22774373263424	0.11289583527023\\
6.23960105953364	0.113036828980131\\
6.25152079648947	0.113177721082344\\
6.2634405334453	0.113317768552748\\
6.27536027040113	0.113456972706968\\
6.28728000735696	0.113595334874595\\
6.2992629439367	0.113733583313825\\
6.31124588051644	0.113870983550845\\
6.32322881709618	0.114007536978198\\
6.33521175367592	0.114143245001968\\
6.34725869756642	0.114278827159582\\
6.35930564145691	0.114413557749071\\
6.3713525853474	0.114547438239677\\
6.3833995292379	0.114680470113558\\
6.39551130573695	0.114813363950918\\
6.40762308223601	0.114945403057344\\
6.41973485873506	0.115076588978576\\
6.43184663523411	0.115206923272675\\
6.44402408779825	0.115337107323575\\
6.45620154036239	0.115466433683047\\
6.46837899292652	0.115594903972584\\
6.48055644549066	0.115722519825815\\
6.49280043746748	0.115849973200922\\
6.5050444294443	0.115976566125093\\
6.51728842142112	0.116102300295106\\
6.52953241339793	0.116227177419433\\
6.5418438273366	0.116351879783791\\
6.55415524127526	0.116475719137429\\
6.56646665521392	0.116598697251971\\
6.57877806915258	0.116720815910252\\
6.59115780864942	0.116842747487622\\
6.60353754814627	0.116963813692321\\
6.61591728764311	0.117084016370752\\
6.62829702713995	0.117203357379684\\
6.64074601624377	0.117322498930346\\
6.65319500534758	0.117440772943044\\
6.66564399445139	0.117558181338284\\
6.67809298355521	0.117674726046685\\
6.69061216875419	0.117791058870494\\
6.70313135395318	0.117906522186295\\
6.71565053915216	0.118021117988085\\
6.72816972435114	0.118134848279586\\
6.74076007448507	0.118248354197748\\
6.75335042461899	0.118360988831623\\
6.76594077475292	0.118472754248175\\
6.77853112488684	0.118583652523639\\
6.79119363328311	0.118694313880691\\
6.80385614167938	0.118804102369039\\
6.81651865007565	0.118913020128268\\
6.82918115847192	0.119021069306589\\
6.84191684190172	0.119128868942817\\
6.85465252533151	0.119235794315909\\
6.86738820876131	0.11934184763787\\
6.88012389219111	0.11944703112856\\
6.8929337927896	0.119551952378213\\
6.90574369338808	0.119655998158864\\
6.91855359398656	0.119759170754128\\
6.93136349458505	0.119861472455254\\
6.94424868139473	0.119963499139526\\
6.95713386820441	0.120064649335691\\
6.97001905501408	0.120164925398408\\
6.98290424182376	0.120264329689508\\
6.99586581007284	0.12036344609049\\
7.00882737832193	0.120461685169373\\
7.02178894657101	0.120559049351312\\
7.03475051482009	0.12065554106815\\
7.04778958943027	0.120751731936051\\
7.06082866404044	0.120847044830643\\
7.07386773865062	0.120941482247329\\
7.08690681326079	0.121035046687408\\
7.10002454751536	0.12112829720854\\
7.11314228176992	0.121220669286007\\
7.12626001602448	0.121312165485077\\
7.13937775027905	0.121402788376268\\
7.15257532859946	0.121493084171546\\
7.16577290691987	0.121582501231879\\
7.17897048524028	0.121671042191635\\
7.19216806356068	0.121758709690138\\
7.20544670103548	0.121846036788999\\
7.21872533851028	0.121932485038905\\
7.23200397598508	0.122018057142756\\
7.24528261345988	0.122102755807922\\
7.2586435601875	0.122187100654185\\
7.27200450691511	0.122270566712486\\
7.28536545364272	0.122353156753707\\
7.29872640037034	0.122434873552698\\
7.31217093962371	0.122516222967715\\
7.32561547887708	0.122596693828556\\
7.33906001813046	0.122676288973871\\
7.35250455738383	0.122755011245429\\
7.36603400873681	0.122833352423576\\
7.37956346008979	0.122910815451846\\
7.39309291144277	0.122987403236261\\
7.40662236279575	0.123063118685291\\
7.42023808324719	0.123138439174359\\
7.43385380369863	0.123212882086431\\
7.44746952415006	0.123286450394171\\
7.4610852446015	0.12335914707237\\
7.47478862987629	0.123431434754548\\
7.48849201515108	0.123502845599082\\
7.50219540042587	0.123573382644737\\
7.51589878570066	0.123643048931895\\
7.52969127264357	0.123712292009468\\
7.54348375958647	0.123780659152999\\
7.55727624652937	0.123848153466806\\
7.57106873347227	0.123914778056323\\
7.58495180146776	0.123980965030741\\
7.59883486946326	0.124046277136473\\
7.61271793745875	0.124110717543149\\
7.62660100545424	0.124174289420711\\
7.64057617834747	0.124237409073127\\
7.6545513512407	0.124299655081475\\
7.66852652413393	0.12436103068047\\
7.68250169702716	0.124421539104288\\
7.69657054527904	0.124481580475333\\
7.71063939353093	0.124540749584086\\
7.72470824178282	0.124599049729652\\
7.7387770900347	0.124656484210273\\
7.75294123299068	0.124713436578869\\
7.76710537594666	0.124769518221771\\
7.78126951890264	0.124824732501988\\
7.79543366185862	0.124879082781138\\
7.80969476969516	0.124932935640448\\
7.8239558775317	0.124985919463183\\
7.83821698536823	0.125038037675761\\
7.85247809320477	0.125089293702704\\
7.86683788984999	0.125140036738563\\
7.8811976864952	0.125189912576989\\
7.89555748314041	0.125238924707449\\
7.90991727978563	0.125287076616883\\
7.9243775449477	0.125334699681278\\
7.93883781010978	0.125381457534626\\
7.95329807527185	0.12542735372947\\
7.96775834043393	0.125472391814772\\
7.98232091364819	0.125516884904003\\
7.99688348686244	0.125560514913411\\
8.0114460600767	0.125603285458114\\
8.02600863329096	0.125645200149129\\
8.04067541501396	0.12568655337028\\
8.05534219673697	0.125727045785359\\
8.07000897845997	0.125766681071611\\
8.08467576018298	0.125805462901722\\
8.0994487168151	0.125843666449917\\
8.11422167344722	0.125881011605745\\
8.12899463007934	0.125917502108207\\
8.14376758671146	0.125953141691191\\
8.1586487531091	0.125988185816402\\
8.17352991950675	0.12602237410043\\
8.1884110859044	0.126055710343661\\
8.20329225230204	0.126088198340846\\
8.21828373575118	0.126120073317123\\
8.23327521920032	0.126151095138097\\
8.24826670264945	0.126181267665443\\
8.26325818609859	0.126210594754425\\
8.27836217017066	0.126239290846181\\
8.29346615424273	0.126267136600323\\
8.3085701383148	0.126294135939945\\
8.32367412238687	0.126320292780645\\
8.33889287092472	0.126345800206479\\
8.35411161946257	0.126370460241782\\
8.36933036800042	0.126394276870678\\
8.38454911653827	0.126417254069308\\
8.39988497972518	0.126439562966739\\
8.41522084291208	0.126461027547307\\
8.43055670609898	0.126481651855962\\
8.44589256928588	0.126501439929137\\
8.46134798619136	0.126520540310286\\
8.47680340309684	0.126538799572291\\
8.49225882000231	0.126556221820759\\
8.50771423690779	0.126572811152186\\
8.52329174240573	0.126588692863962\\
8.53886924790367	0.126603736775638\\
8.55444675340161	0.126617947053313\\
8.57002425889955	0.126631327853439\\
8.58572649044944	0.12664398053257\\
8.60142872199933	0.126655798848901\\
8.61713095354922	0.126666787029102\\
8.63283318509911	0.126676949289459\\
8.64866288641169	0.126686362311132\\
8.66449258772426	0.126694944522558\\
8.68032228903684	0.1267027002114\\
8.69615199034942	0.126709633653778\\
8.71211202044306	0.126715796081476\\
8.7280720505367	0.126721131363897\\
8.74403208063034	0.1267256438497\\
8.75999211072399	0.126729337875262\\
8.77608545081172	0.126732238403682\\
8.79217879089946	0.126734315561242\\
8.80827213098719	0.126735573757648\\
8.82436547107493	0.126736017389814\\
8.84059523179885	0.126735644284107\\
8.85682499252278	0.126734451688465\\
8.87305475324671	0.126732444073847\\
8.88928451397064	0.126729625897725\\
8.9056539454559	0.126725966928436\\
8.92202337694115	0.126721492453545\\
8.93839280842641	0.126716207005464\\
8.95476223991167	0.126710115102504\\
8.97127474100433	0.126703157474403\\
8.987787242097	0.126695388425298\\
9.00429974318966	0.126686812549337\\
9.02081224428233	0.126677434425965\\
9.0374713729445	0.126667164701446\\
9.05413050160667	0.126656087737326\\
9.07078963026884	0.126644208190115\\
9.08744875893101	0.126631530700738\\
9.10425824450065	0.126617934715813\\
9.12106773007028	0.126603535765886\\
9.13787721563991	0.126588338570713\\
9.15468670120955	0.126572347833214\\
9.17165045698708	0.126555410605838\\
9.18861421276461	0.126537674777906\\
9.20557796854214	0.126519145132973\\
9.22254172431967	0.126499826436892\\
9.23966386191763	0.126479532066573\\
9.25678599951559	0.126458443546325\\
9.27390813711355	0.126436565724171\\
9.29103027471151	0.126413903429821\\
9.30828551504259	0.126390276202043\\
9.32554075537366	0.12636586205147\\
9.34279599570473	0.126340665866334\\
9.36005123603581	0.126314692515882\\
9.37740098280311	0.126287798247599\\
9.39475072957042	0.126260128093145\\
9.41210047633773	0.126231686945398\\
9.42945022310504	0.126202479677812\\
9.44689582322271	0.126172343469834\\
9.46434142334038	0.126141442477032\\
9.48178702345805	0.126109781595472\\
9.49923262357573	0.126077365701475\\
9.51677513419308	0.126044013327767\\
9.53431764481043	0.126009907359022\\
9.55186015542778	0.12597505269287\\
9.56940266604513	0.125939454206831\\
9.58704316415657	0.125902911721387\\
9.60468366226802	0.125865626914278\\
9.62232416037946	0.125827604683421\\
9.6399646584909	0.125788849905893\\
9.65770423902654	0.125749143628938\\
9.67544381956217	0.125708706382942\\
9.6931834000978	0.12566754306484\\
9.71092298063344	0.125625658549797\\
9.72876275712183	0.125582815050598\\
9.74660253361022	0.125539252010186\\
9.76444231009861	0.125494974322888\\
9.782282086587	0.125449986860844\\
9.80022319117994	0.125404032942427\\
9.81816429577288	0.12535737098176\\
9.83610540036582	0.125310005869078\\
9.85404650495876	0.12526194247218\\
9.87209008914481	0.125212905154899\\
9.89013367333085	0.125163171361561\\
9.9081772575169	0.125112745976919\\
9.92622084170295	0.12506163386285\\
9.94436807654274	0.125009540369498\\
9.96251531138254	0.124956762029789\\
9.98066254622234	0.124903303721511\\
9.99880978106213	0.124849170299287\\
10.0170618577606	0.1247940480398\\
10.0353139344591	0.124738252623566\\
10.0535660111575	0.124681788919934\\
10.071818087856	0.124624661774923\\
10.0901762182373	0.124566538331879\\
10.1085343486186	0.124507753477981\\
10.1268924789999	0.12444831207285\\
10.1452506093812	0.12438821895247\\
10.163716026191	0.12432712206769\\
10.1821814430008	0.124265375570515\\
10.2006468598106	0.124202984309944\\
10.2191122766204	0.124139953110543\\
10.237686234429	0.124075910669857\\
10.2562601922376	0.12401123046455\\
10.2748341500462	0.123945917331866\\
10.2934081078548	0.123879976083878\\
10.3120918830375	0.123813016105534\\
10.3307756582203	0.123745430256565\\
10.3494594334031	0.123677223361054\\
10.3681432085859	0.12360840021778\\
10.3869381001644	0.123538550838486\\
10.4057329917429	0.123468087525637\\
10.4245278833213	0.123397015089038\\
10.4433227748998	0.123325338312917\\
10.4622301050096	0.123252627775452\\
10.4811374351193	0.123179315281709\\
10.5000447652291	0.123105405626016\\
10.5189520953388	0.123030903576818\\
10.5379732097484	0.122955360217961\\
10.556994324158	0.122879226917574\\
10.5760154385676	0.122802508453238\\
10.5950365529771	0.122725209576571\\
10.6141728216781	0.12264686181524\\
10.6333090903791	0.122567936161957\\
10.6524453590801	0.122488437376343\\
10.6715816277811	0.122408370191989\\
10.6908344455163	0.122327246518407\\
10.7100872632516	0.122245557034399\\
10.7293400809869	0.122163306480847\\
10.7485928987221	0.122080499572102\\
10.7679636856929	0.121996628536346\\
10.7873344726636	0.121912203801029\\
10.8067052596344	0.121827230087614\\
10.8260760466051	0.121741712090196\\
10.8455662488972	0.121655122292366\\
10.8650564511894	0.12156799093302\\
10.8845466534815	0.121480322713\\
10.9040368557736	0.121392122305611\\
10.9236479463274	0.121302842384276\\
10.9432590368811	0.121213033064302\\
10.9628701274349	0.121122699024956\\
10.9824812179887	0.121031844917855\\
11.0022146969848	0.120939903541333\\
11.0219481759809	0.120847444951811\\
11.041681654977	0.120754473806049\\
11.0614151339731	0.120660994732856\\
11.0812725294807	0.120566420589867\\
11.1011299249883	0.120471341440397\\
11.1209873204959	0.120375761917597\\
11.1408447160036	0.120279686626668\\
11.1608275848708	0.120182508416317\\
11.180810453738	0.120084837425086\\
11.2007933226052	0.119986678261455\\
11.2207761914724	0.119888035506012\\
11.2408861199495	0.119788281929811\\
11.2609960484265	0.119688047815347\\
11.2811059769036	0.119587337745805\\
11.3012159053807	0.119486156276107\\
11.3214545099205	0.119383856029352\\
11.3416931144603	0.119281087502073\\
11.3619317190001	0.119177855251743\\
11.3821703235399	0.119074163806759\\
11.4025392513608	0.118969345571442\\
11.4229081791817	0.11886407132713\\
11.4432771070026	0.11875834560454\\
11.4636460348235	0.118652172905245\\
11.4841469650118	0.118544865340961\\
11.5046478952002	0.118437114051408\\
11.5251488253886	0.118328923539811\\
11.545649755577	0.118220298280156\\
11.5662843996785	0.118110530017851\\
11.5869190437801	0.118000330324929\\
11.6075536878816	0.117889703676337\\
11.6281883319832	0.117778654517562\\
11.6489584349037	0.117666454152181\\
11.6697285378243	0.117553834660674\\
11.6904986407448	0.117440800488736\\
11.7112687436654	0.117327356052733\\
11.7321760845019	0.117212752136768\\
11.7530834253385	0.117097741407969\\
11.7739907661751	0.116982328281882\\
11.7948981070116	0.116866517144851\\
11.8159445000986	0.116749538181652\\
11.8369908931856	0.116632164726241\\
11.8580372862726	0.116514401163746\\
11.8790836793596	0.116396251849545\\
11.900270975065	0.116276926287909\\
11.9214582707703	0.116157218561059\\
11.9426455664757	0.116037133023264\\
11.9638328621811	0.115916673998407\\
11.9851629478153	0.11579503022725\\
12.0064930334495	0.115673016623554\\
12.0278231190837	0.11555063750981\\
12.0491532047179	0.115427897178293\\
12.0706280057176	0.115303963520275\\
12.0921028067173	0.115179672367203\\
12.113577607717	0.115055028009304\\
12.1350524087168	0.114930034706354\\
12.1566738894695	0.114803839413847\\
12.1782953702222	0.114677298968126\\
12.1999168509749	0.114550417626347\\
12.2215383317276	0.114423199615255\\
12.2433084968313	0.11429477086431\\
12.265078661935	0.114166009306016\\
12.2868488270387	0.11403691916356\\
12.3086189921423	0.113907504630001\\
12.3305398873956	0.11377687051673\\
12.3524607826488	0.113645915945388\\
12.3743816779021	0.113514645104702\\
12.3963025731553	0.113383062153165\\
12.4183762865519	0.113250250690704\\
12.4404499999485	0.113117131122083\\
12.462523713345	0.112983707601585\\
12.4845974267416	0.11284998425244\\
12.50682608981	0.112715023365853\\
12.5290547528783	0.11257976672771\\
12.5512834159467	0.112444218457004\\
12.5735120790151	0.112308382641764\\
12.5958978680728	0.112171300164402\\
12.6182836571306	0.112033934292429\\
12.6406694461884	0.111896289109126\\
12.6630552352461	0.111758368666695\\
12.6856003725544	0.111619192337344\\
12.7081455098627	0.111479744972745\\
12.730690647171	0.111340030619826\\
12.7532357844793	0.111200053294401\\
12.7759425395622	0.111058810753926\\
12.7986492946451	0.110917309540273\\
12.821356049728	0.110775553663154\\
12.844062804811	0.11063354710152\\
12.8668397124378	0.110490853933582\\
12.8896166200647	0.110347916456999\\
12.9123935276916	0.110204738595244\\
12.9351704353184	0.110061324241582\\
12.9579981520666	0.109917356564597\\
12.9808258688147	0.10977315907932\\
13.0036535855629	0.109628735614788\\
13.026481302311	0.109484089969591\\
13.049363098101	0.109338882472741\\
13.0722448938909	0.109193459295863\\
13.0951266896808	0.109047824176267\\
13.1180084854707	0.108901980821604\\
13.1409474667556	0.108755567658425\\
13.1638864480404	0.108608952592447\\
13.1868254293253	0.108462139272263\\
13.2097644106101	0.108315131317153\\
13.2327635996633	0.108167545717375\\
13.2557627887165	0.108019771659548\\
13.2787619777697	0.107871812705933\\
13.3017611668229	0.107723672390238\\
13.3248235100991	0.107574946704455\\
13.3478858533754	0.107426045690266\\
13.3709481966517	0.107276972825693\\
13.3940105399279	0.107127731561111\\
13.4171389165768	0.106977897297471\\
13.4402672932257	0.106827900535125\\
13.4633956698746	0.106677744670868\\
13.4865240465235	0.106527433073655\\
13.5097212774941	0.10637652092276\\
13.5329185084647	0.106225458817697\\
13.5561157394353	0.106074250076186\\
13.5793129704059	0.10592289798866\\
13.6025818233678	0.105770937869389\\
13.6258506763298	0.10561884006894\\
13.6491195292917	0.10546660782823\\
13.6723883822536	0.105314244361144\\
13.6957315789086	0.105161265444933\\
13.7190747755636	0.105008160862167\\
13.7424179722186	0.104854933778633\\
13.7657611688735	0.104701587333854\\
13.789181390828	0.104547618072932\\
13.8126016127825	0.104393534914242\\
13.8360218347369	0.104239340949942\\
13.8594420566914	0.104085039246678\\
13.8829419497165	0.103930107409728\\
13.9064418427417	0.103775073208415\\
13.9299417357668	0.103619939663963\\
13.9534416287919	0.103464709771633\\
13.9770238098647	0.103308842455927\\
14.0006059909374	0.103152884084866\\
14.0241881720102	0.102996837610088\\
14.0477703530829	0.102840705958035\\
14.0714374141426	0.102683929619854\\
14.0951044752024	0.102527073320805\\
14.1187715362621	0.102370139944738\\
14.1424385973218	0.102213132350591\\
14.1661931077477	0.102055472839859\\
14.1899476181736	0.101897744258495\\
14.2137021285995	0.101739949423466\\
14.2374566390255	0.101582091127718\\
14.2613011526304	0.101423573690887\\
14.2851456662352	0.101264997877862\\
14.3089901798401	0.101106366440865\\
14.332834693445	0.100947682107726\\
14.3567717508919	0.100788331418192\\
14.3807088083388	0.100628932859554\\
14.4046458657857	0.100469489120533\\
14.4285829232326	0.100310002866156\\
14.4526150559176	0.100149843042271\\
14.4766471886026	0.0999896456769466\\
14.5006793212876	0.0998294133969684\\
14.5247114539725	0.099669148805582\\
14.548841186964	0.0995082034317119\\
14.5729709199554	0.0993472306731795\\
14.5971006529469	0.0991862330953931\\
14.6212303859383	0.0990252132411013\\
14.6454602425121	0.0988635053771544\\
14.6696900990858	0.0987017801212291\\
14.6939199556595	0.0985400399792913\\
14.7181498122333	0.0983782874341818\\
14.7424823150752	0.0982158396458137\\
14.7668148179171	0.0980533843009507\\
14.7911473207591	0.0978909238470747\\
14.815479823601	0.0977284607091861\\
14.8399174997295	0.0975652950711028\\
14.864355175858	0.0974021315613388\\
14.8887928519865	0.0972389725701251\\
14.913230528115	0.097075820465546\\
14.9377759105528	0.096911958583667\\
14.9623212929905	0.0967481083716551\\
14.9868666754283	0.0965842721628312\\
15.0114120578661	0.096420452269097\\
15.0360676892572	0.0962559152904963\\
15.0607233206484	0.096091399385147\\
15.0853789520395	0.0959269068313716\\
15.1100345834307	0.0957624398857704\\
15.1348030187572	0.0955972485127509\\
15.1595714540837	0.0954320874841764\\
15.1843398894102	0.095266959024227\\
15.2091083247367	0.0951018653355425\\
15.2339921326276	0.0949360398475923\\
15.2588759405185	0.0947702538490412\\
15.2837597484095	0.094604509510394\\
15.3086435563004	0.0944388089814219\\
15.333645323629	0.0942723692333229\\
15.3586470909575	0.0941059779991469\\
15.383648858286	0.0939396373967842\\
15.4086506256146	0.0937733495231486\\
15.4337729580942	0.0936063149713969\\
15.4588952905739	0.0934393378421717\\
15.4840176230536	0.0932724202016808\\
15.5091399555333	0.0931055640956751\\
15.5343854824194	0.0929379537954345\\
15.5596310093056	0.0927704097153551\\
15.5848765361918	0.09260293387088\\
15.610122063078	0.0924355282572785\\
15.6354934388609	0.0922673608771594\\
15.6608648146438	0.0920992684098741\\
15.6862361904268	0.0919312528201533\\
15.7116075662097	0.091763316053112\\
15.7371074715309	0.0915946098969827\\
15.762607376852	0.0914259872450034\\
15.7881072821731	0.0912574500126745\\
15.8136071874943	0.0910890000959007\\
15.8392383339326	0.0909197730977262\\
15.8648694803709	0.0907506380982557\\
15.8905006268092	0.0905815969645413\\
15.9161317732476	0.0904126515439982\\
15.9418969045212	0.0902429212850747\\
15.9676620357948	0.0900732914278222\\
15.9934271670685	0.0899037637907328\\
16.0191922983421	0.0897343401734619\\
16.0450941933734	0.0895641238872421\\
16.0709960884047	0.0893940163180064\\
16.096897983436	0.0892240192369751\\
16.1227998784672	0.0890541343960715\\
16.1488413541292	0.0888834489761315\\
16.1748828297911	0.0887128805047894\\
16.2009243054531	0.0885424307064505\\
16.226965781115	0.0883721012865226\\
16.2531496945447	0.0882009632961133\\
16.2793336079744	0.08802995040637\\
16.3055175214041	0.0878590642952269\\
16.3317014348337	0.0876883066222704\\
16.3580306848324	0.0875167323083448\\
16.384359934831	0.0873452911722127\\
16.4106891848297	0.0871739848458432\\
16.4370184348283	0.0870028149427356\\
16.4634959660973	0.0868308202339229\\
16.4899734973663	0.086658966708272\\
16.5164510286353	0.086487255952433\\
16.5429285599043	0.0863156895349995\\
16.5695573662806	0.0861432900451035\\
16.5961861726568	0.085971039675635\\
16.6228149790331	0.0857989399685565\\
16.6494437854093	0.0856269924479494\\
16.6762269105277	0.0854542034946192\\
16.7030100356461	0.0852815715356342\\
16.7297931607645	0.0851090980680382\\
16.7565762858829	0.084936784571484\\
16.7835168283287	0.0847636211679219\\
16.8104573707745	0.0845906225720485\\
16.8373979132204	0.0844177902370185\\
16.8643384556662	0.0842451255986459\\
16.8914395705478	0.0840716024682275\\
16.9185406854293	0.0838982519018103\\
16.9456418003109	0.0837250753091951\\
16.9727429151925	0.0835520740827487\\
17.0000078170973	0.0833782056638821\\
17.027272719002	0.083204517512554\\
17.0545376209068	0.0830310109948427\\
17.0818025228116	0.0828576874600886\\
17.1092344893586	0.0826834879078287\\
17.1366664559057	0.0825094762770043\\
17.1640984224528	0.0823356538908868\\
17.1915303889999	0.0821620220556664\\
17.219132763799	0.0819875052484986\\
17.2467351385981	0.0818131839703016\\
17.2743375133972	0.0816390595017868\\
17.3019398881964	0.0814651331067211\\
17.3297160854172	0.0812903126427573\\
17.3574922826381	0.0811156952725074\\
17.385268479859	0.0809412822341068\\
17.4130446770799	0.0807670747493732\\
17.4409981837359	0.0805919639563596\\
17.4689516903919	0.0804170637828452\\
17.4969051970478	0.0802423754247719\\
17.5248587037038	0.0800679000615181\\
17.5529930820312	0.0798925120087509\\
17.5811274603586	0.0797173420651996\\
17.6092618386859	0.0795423913848478\\
17.6373962170133	0.0793676611054188\\
17.6657151116334	0.0791920085855254\\
17.6940340062535	0.0790165816316893\\
17.7223529008736	0.078841381356213\\
17.7506717954937	0.0786664088554795\\
17.7791789342782	0.0784905044047365\\
17.8076860730626	0.0783148329482585\\
17.836193211847	0.078139395556433\\
17.8647003506315	0.0779641932839363\\
17.8933995511133	0.0777880491708078\\
17.922098751595	0.0776121454541979\\
17.9507979520768	0.0774364831630624\\
17.9794971525585	0.0772610633108741\\
18.0083923245633	0.077084691548348\\
18.0372874965681	0.0769085675618259\\
18.0661826685729	0.0767326923391811\\
18.0950778405776	0.0765570668527544\\
18.124172992061	0.0763804791905465\\
18.1532681435443	0.076204146665309\\
18.1823632950276	0.0760280702232291\\
18.2114584465109	0.0758522507954886\\
18.2407576866174	0.0756754587312259\\
18.270056926724	0.0754989291492152\\
18.2993561668305	0.0753226629545037\\
18.328655406937	0.0751466610371313\\
18.3581629537927	0.0749696758001069\\
18.3876705006484	0.0747929603782361\\
18.4171780475041	0.0746165156356621\\
18.4466855943597	0.0744403424213742\\
18.4764057786275	0.0742631749825373\\
18.5061259628954	0.0740862846833441\\
18.5358461471632	0.0739096723464537\\
18.565566331431	0.0737333387799842\\
18.5955036012303	0.0735559998543414\\
18.6254408710296	0.0733789453878468\\
18.6553781408289	0.0732021761620417\\
18.6853154106283	0.0730256929436782\\
18.7154743397902	0.0728481929754999\\
18.7456332689521	0.0726709847843891\\
18.7757921981141	0.0724940691107337\\
18.805951127276	0.0723174466801882\\
18.8363364194138	0.0721397958541476\\
18.8667217115517	0.0719624441251488\\
18.8971070036895	0.0717853921921262\\
18.9274922958273	0.0716086407398322\\
18.9581087922171	0.0714308489697532\\
18.9887252886069	0.0712533636229342\\
19.0193417849967	0.0710761853568354\\
19.0499582813864	0.070899314814539\\
19.0808109690034	0.070721391735077\\
19.1116636566204	0.0705437824148616\\
19.1425163442373	0.0703664874697373\\
19.1733690318543	0.0701895075013552\\
19.2044630478062	0.0700114624820458\\
19.235557063758	0.0698337385713122\\
19.2666510797099	0.0696563363432905\\
19.2977450956618	0.0694792563582919\\
19.3290857385735	0.069301098477687\\
19.3604263814852	0.0691232690731457\\
19.3917670243968	0.068945768676656\\
19.4231076673085	0.0687685978064206\\
19.4547004046881	0.0685903358535522\\
19.4862931420676	0.068412409766177\\
19.5178858794472	0.0682348200339933\\
19.5494786168267	0.0680575671331635\\
19.581329093847	0.0678792096031596\\
19.6131795708672	0.067701195353904\\
19.6450300478875	0.0675235248328193\\
19.6768805249077	0.0673461984738623\\
19.7089945750868	0.0671677535539212\\
19.7411086252659	0.0669896593611029\\
19.7732226754451	0.0668119162997199\\
19.8053367256242	0.0666345247609424\\
19.8377203807137	0.0664560003247396\\
19.8701040358032	0.0662778340973777\\
19.9024876908927	0.0661000264399595\\
19.9348713459822	0.0659225777006355\\
19.967530847752	0.0657439812954851\\
20.0001903495219	0.0655657506207189\\
20.0328498512918	0.0653878859942021\\
20.0655093530616	0.0652103877206988\\
20.0984511655557	0.065031726554549\\
20.1313929780498	0.0648534386858611\\
20.1643347905439	0.0646755243882149\\
20.197276603038	0.064497983922558\\
20.2305074253896	0.0643192648516917\\
20.2637382477412	0.0641409266959416\\
20.2969690700928	0.0639629696843719\\
20.3301998924444	0.0637853940335589\\
20.3637266734178	0.0636066235455568\\
20.3972534543913	0.0634282416461069\\
20.4307802353647	0.063250248519624\\
20.4643070163382	0.0630726443377964\\
20.4981369694764	0.0628938285350986\\
20.5319669226146	0.0627154090564207\\
20.5657968757529	0.0625373860404023\\
20.5996268288911	0.0623597596134813\\
20.6337674490278	0.0621809041936557\\
20.6679080691645	0.0620024529014961\\
20.7020486893012	0.0618244058294304\\
20.7361893094379	0.0616467630577561\\
20.7706483924226	0.0614678732830639\\
20.8051074754072	0.0612893955141439\\
20.8395665583918	0.0611113297968334\\
20.8740256413764	0.0609336761646196\\
20.908811298434	0.0607547568650921\\
20.9435969554916	0.0605762575307485\\
20.9783826125492	0.0603981781598015\\
21.0131682696068	0.0602205187386181\\
21.0482889562591	0.0600415742441783\\
21.0834096429114	0.0598630577633652\\
21.1185303295637	0.0596849692460355\\
21.153651016216	0.0595073086301659\\
21.189115551059	0.0593283427637125\\
21.224580085902	0.0591498130561052\\
21.260044620745	0.0589717194079953\\
21.295509155588	0.0587940617082085\\
21.3313267450985	0.0586150777549803\\
21.367144334609	0.0584365382104445\\
21.4029619241194	0.0582584429253009\\
21.4387795136299	0.0580807917387126\\
21.474959783243	0.0579017923908548\\
21.5111400528561	0.0577232458157584\\
21.5473203224691	0.0575451518128998\\
21.5835005920822	0.0573675101704648\\
21.6200536143196	0.0571884974965569\\
21.656606636557	0.0570099460832803\\
21.6931596587944	0.0568318556777765\\
21.7297126810318	0.0566542260162046\\
21.7666490124925	0.0564752013971937\\
21.8035853439532	0.0562966466604919\\
21.8405216754139	0.0561185615002277\\
21.8774580068746	0.0559409455992846\\
21.9147887220248	0.055761909691064\\
21.9521194371751	0.0555833524320857\\
21.9894501523253	0.0554052734618816\\
22.0267808674756	0.0552276724091787\\
22.0645176030464	0.0550486250653853\\
22.1022543386172	0.0548700652953691\\
22.139991074188	0.0546919926827311\\
22.1777278097588	0.0545144068003545\\
22.2158828101726	0.0543353470048934\\
22.2540378105863	0.0541567838787685\\
22.2921928110001	0.0539787169480517\\
22.3303478114138	0.053801145728358\\
22.3689339797914	0.0536220715175954\\
22.407520148169	0.0534435032566541\\
22.4461063165466	0.0532654404129338\\
22.4846924849242	0.0530878824432992\\
22.5237234418978	0.0529087908081517\\
22.5627543988714	0.0527302146044738\\
22.601785355845	0.0525521532389863\\
22.6408163128186	0.0523746061082799\\
22.6803064595616	0.0521954928967804\\
22.7197966063045	0.0520169048169918\\
22.7592867530475	0.0518388412131155\\
22.7987768997904	0.0516613014193804\\
22.838741493306	0.0514821612051736\\
22.8787060868217	0.0513035560609653\\
22.9186706803373	0.0511254852662532\\
22.9586352738529	0.0509479480908819\\
22.9990905077973	0.0507687740431485\\
23.0395457417417	0.0505901452624598\\
23.0800009756861	0.0504120609618847\\
23.1204562096305	0.0502345203446581\\
23.1614193085936	0.0500553040656134\\
23.2023824075567	0.0498766435332384\\
23.2433455065199	0.049698537891606\\
23.284308605483	0.0495209862753301\\
23.3257979315477	0.0493417176166841\\
23.3672872576124	0.0491630154933449\\
23.4087765836772	0.0489848789777087\\
23.4502659097419	0.0488073071329986\\
23.4922104776085	0.0486283599141691\\
23.5341550454751	0.0484499878724707\\
23.5760996133418	0.0482721900120856\\
23.6180441812084	0.0480949653285214\\
23.6604170261043	0.0479165120368637\\
23.7027898710001	0.0477386415991527\\
23.745162715896	0.0475613529529321\\
23.7875355607918	0.0473846450271927\\
23.8303479690752	0.0472066926738086\\
23.8731603773585	0.047029330928493\\
23.9159727856418	0.0468525586611247\\
23.9587851939251	0.0466763747331527\\
24.0020482633235	0.0464989327130902\\
24.0453113327218	0.0463220891218264\\
24.0885744021202	0.046145842760178\\
24.1318374715185	0.0459701924210078\\
24.1755626771472	0.0457932701519468\\
24.219287882776	0.0456169542034241\\
24.2630130884048	0.0454412433057864\\
24.3067382940335	0.0452661361818627\\
24.3509374990427	0.0450897431406072\\
24.3951367040518	0.044913964387068\\
24.4393359090609	0.0447387985794996\\
24.48353511407	0.0445642443692184\\
24.5282205890312	0.0443883900958768\\
24.5729060639923	0.0442131581572014\\
24.6175915389535	0.0440385471378212\\
24.6622770139146	0.0438645556158822\\
24.7074614574793	0.0436892497178451\\
24.7526459010441	0.0435145742854157\\
24.7978303446088	0.0433405278283277\\
24.8430147881735	0.043167108849825\\
24.888711348921	0.0429923610057931\\
24.9344079096686	0.0428182518469894\\
24.9801044704162	0.0426447798065403\\
25.0258010311637	0.0424719433114309\\
25.0720233306192	0.0422977632762365\\
25.1182456300747	0.0421242302394017\\
25.1644679295302	0.0419513425555929\\
25.2106902289857	0.0417790985738205\\
25.2574523862228	0.0416054961828285\\
25.3042145434599	0.0414325492018071\\
25.350976700697	0.0412602559050308\\
25.3977388579342	0.0410886145615999\\
25.4450555156315	0.0409155997355191\\
25.4923721733288	0.0407432488347422\\
25.5396888310261	0.0405715600510062\\
25.5870054887234	0.0404005315714802\\
25.6348918408033	0.0402281143217706\\
25.6827781928832	0.040056369621791\\
25.7306645449631	0.0398852955784482\\
25.7785508970431	0.0397148902948302\\
25.8270227182624	0.0395430807288319\\
25.8754945394818	0.0393719524514307\\
25.9239663607012	0.0392015034825562\\
25.9724381819206	0.0390317318388412\\
26.0215118591534	0.0388605401658318\\
26.0705855363863	0.0386900386403722\\
26.1196592136191	0.0385202251932053\\
26.168732890852	0.0383510977520465\\
26.2184254566214	0.0381805342889207\\
26.2681180223908	0.0380106699583413\\
26.3178105881602	0.0378415025993998\\
26.3675031539296	0.0376730300484663\\
26.4178323219975	0.0375031052263578\\
26.4681614900653	0.0373338886540305\\
26.5184906581332	0.0371653780762555\\
26.5688198262011	0.0369975712355442\\
26.6198040298383	0.0368282956058806\\
26.6707882334756	0.036659737481924\\
26.7217724371128	0.0364918945113028\\
26.7727566407501	0.0363247643399407\\
26.8244150733526	0.036156148581803\\
26.8760735059551	0.035988259730713\\
26.9277319385576	0.0358210953341759\\
26.9793903711602	0.035654652938593\\
27.0317430299247	0.0354867078657148\\
27.0840956886892	0.0353194992536624\\
27.1364483474537	0.0351530245466669\\
27.1888010062182	0.034987281188482\\
27.2418687387887	0.0348200177570451\\
27.2949364713593	0.0346535004999906\\
27.3480042039299	0.0344877267549392\\
27.4010719365005	0.0343226938596892\\
27.4548764910367	0.0341561231763317\\
27.508681045573	0.0339903085483765\\
27.5624856001092	0.0338252472032973\\
27.6162901546455	0.0336609363694357\\
27.6708542335196	0.0334950696998045\\
27.7254183123937	0.0333299691421036\\
27.7799823912678	0.0331656318099129\\
27.834546470142	0.0330020548184091\\
27.8898937876417	0.0328369035961203\\
27.9452411051415	0.0326725287262292\\
28.0005884226412	0.0325089272044538\\
28.055935740141	0.0323460960288775\\
28.1120910845417	0.0321816718649181\\
28.1682464289425	0.032018034486613\\
28.2244017733432	0.0318551807676141\\
28.2805571177439	0.0316931075847459\\
28.3375464180691	0.0315294222775948\\
28.3945357183943	0.0313665343914435\\
28.4515250187195	0.0312044406734173\\
28.5085143190447	0.031043137874663\\
28.5663647170539	0.0308802034208815\\
28.6242151150631	0.0307180772353181\\
28.6820655130723	0.0305567559338313\\
28.7399159110815	0.0303962361372024\\
28.7983043617203	0.0302350326739916\\
28.856692812359	0.0300746387933468\\
28.9150812629978	0.0299150510375888\\
28.9734697136365	0.0297562659547827\\
29.0321280026001	0.0295975518277931\\
29.0907862915636	0.0294396408443124\\
29.1494445805272	0.0292825295231788\\
29.2081028694908	0.0291262143896886\\
29.2670352867619	0.0289699670270657\\
29.325967704033	0.0288145163008429\\
29.3849001213041	0.0286598587082521\\
29.4438325385752	0.0285059907536797\\
29.5030416355931	0.0283521920991758\\
29.562250732611	0.0281991834637736\\
29.621459829629	0.028046961325216\\
29.6806689266469	0.02789552216908\\
29.7401573016333	0.0277441536977868\\
29.7996456766196	0.0275935685254955\\
29.859134051606	0.0274437631124992\\
29.9186224265924	0.0272947339275931\\
29.9783927156743	0.0271457766983808\\
30.0381630047562	0.0269975959521333\\
30.0979332938381	0.0268501881336601\\
30.15770358292	0.0267035496969334\\
30.2177584582065	0.0265569843839944\\
30.2778133334929	0.0264111886485924\\
30.3378682087793	0.026266158921948\\
30.3979230840658	0.0261218916451043\\
30.4582652560913	0.0259776985574093\\
30.5186074281169	0.0258342680587843\\
30.5789496001424	0.0256915965686863\\
30.639291772168	0.0255496805170624\\
30.6999239910786	0.0254078396204508\\
30.7605562099893	0.0252667542475337\\
30.8211884289	0.0251264208077628\\
30.8818206478106	0.0249868357217595\\
30.9427457014925	0.0248473266673377\\
31.0036707551745	0.0247085660001671\\
31.0645958088564	0.0245705501213961\\
31.1255208625383	0.0244332754440076\\
31.186741581882	0.0242960775790519\\
31.2479623012256	0.0241596208995728\\
31.3091830205692	0.0240239018000617\\
31.3704037399129	0.0238889166874669\\
31.4319229948219	0.023754009087135\\
31.493442249731	0.0236198354105834\\
31.55496150464	0.0234863920472393\\
31.6164807595491	0.0233536753995407\\
31.6783014623118	0.0232210368807295\\
31.7401221650745	0.023089124969352\\
31.8019428678372	0.0229579360513123\\
31.8637635705999	0.0228274665259835\\
31.9258886754761	0.0226970756691558\\
31.9880137803524	0.0225674040538096\\
32.0501388852286	0.0224384480638233\\
32.1122639901049	0.0223102040968688\\
32.1746964962864	0.0221820392599632\\
32.2371290024679	0.0220545862538988\\
32.2995615086494	0.0219278414619483\\
32.361994014831	0.0218018012814017\\
32.4247369657647	0.021675840620965\\
32.4874799166984	0.0215505843407403\\
32.5502228676321	0.0214260288247359\\
32.6129658185659	0.0213021704711422\\
32.6760223019509	0.0211783919610997\\
32.739078785336	0.021055310345155\\
32.8021352687211	0.0209329220093171\\
32.8651917521062	0.0208112233539246\\
32.9285649023539	0.0206896047988112\\
32.9919380526016	0.020568675620567\\
33.0553112028493	0.020448432208391\\
33.118684353097	0.020328870965986\\
33.1823773519202	0.0202093900175161\\
33.2460703507434	0.0200905909017726\\
33.3097633495666	0.019972470012258\\
33.3734563483898	0.0198550237572258\\
33.4374724249187	0.019737657931103\\
33.5014885014475	0.0196209663706438\\
33.5655045779763	0.01950494547474\\
33.6295206545052	0.0193895916573387\\
33.6938630873744	0.0192743183457879\\
33.7582055202435	0.0191597117137831\\
33.8225479531127	0.0190457681666661\\
33.8868903859819	0.0189324841251778\\
33.951562503147	0.0188192806132103\\
34.016234620312	0.0187067361793108\\
34.0809067374771	0.0185948472363075\\
34.1455788546421	0.0184836102127891\\
34.2105840369751	0.0183724536872479\\
34.2755892193081	0.0182619486265368\\
34.340594401641	0.0181520914519885\\
34.405599583974	0.0180428786010561\\
34.4709412619942	0.0179337461712255\\
34.5362829400144	0.0178252575846139\\
34.6016246180346	0.0177174092720492\\
34.6669662960548	0.017610197680838\\
34.7326479550893	0.0175030663838613\\
34.7983296141237	0.0173965713034671\\
34.8640112731582	0.0172907088809318\\
34.9296929321926	0.0171854755743747\\
34.9957181104378	0.0170803223924814\\
35.061743288683	0.0169757977986692\\
35.1277684669281	0.016871898245577\\
35.1937936451733	0.0167686202030715\\
35.2601659376236	0.0166654220705872\\
35.3265382300739	0.0165628448989082\\
35.3929105225242	0.0164608851529106\\
35.4592828149746	0.0163595393151213\\
35.526005872278	0.0162582731333844\\
35.5927289295814	0.0161576202893301\\
35.6594519868847	0.0160575772609238\\
35.7261750441881	0.0159581405441912\\
35.7932525744811	0.0158587831911654\\
35.8603301047741	0.0157600315596776\\
35.927407635067	0.0156618821416113\\
35.99448516536	0.0155643314472565\\
36.0619209371048	0.0154668597854728\\
36.1293567088496	0.0153699862388025\\
36.1967924805944	0.0152737073138517\\
36.2642282523392	0.0151780195358726\\
36.3320260925956	0.0150824104267738\\
36.3998239328521	0.0149873918386573\\
36.4676217731086	0.0148929602936495\\
36.5354196133651	0.0147991123325845\\
36.6035834111162	0.0147053426429657\\
36.6717472088673	0.0146121558950159\\
36.7399110066184	0.0145195486271146\\
36.8080748043695	0.0144275173962834\\
36.8766085113362	0.0143355640081888\\
36.9451422183029	0.0142441859996846\\
37.0136759252696	0.0141533799260608\\
37.0822096322362	0.0140631423611348\\
37.1511172647722	0.0139729821799234\\
37.2200248973081	0.0138833898357823\\
37.2889325298441	0.0137943619015037\\
37.35784016238	0.0137058949683093\\
37.4271258025917	0.0136175049314991\\
37.4964114428034	0.0135296752110342\\
37.5656970830151	0.0134424023977201\\
37.6349827232268	0.0133556831007969\\
37.7046505191923	0.0132690401877024\\
37.7743183151579	0.0131829500941098\\
37.8439861111234	0.0130974094293318\\
37.913653907089	0.0130124148212259\\
37.9837080759575	0.0129274960585926\\
38.0537622448261	0.0128431226445094\\
38.1238164136947	0.0127592912072995\\
38.1938705825633	0.0126759983940121\\
38.2643154110918	0.0125927808646999\\
38.3347602396202	0.0125101012407914\\
38.4052050681487	0.0124279561701304\\
38.4756498966772	0.0123463423194987\\
38.5464897434044	0.0122648031688553\\
38.6173295901316	0.0121837945101126\\
38.6881694368588	0.0121033130111568\\
38.759009283586	0.0120233553590277\\
38.8302485799245	0.0119434718023117\\
38.9014878762629	0.0118641113554115\\
38.9727271726013	0.0117852707067666\\
39.0439664689397	0.0117069465641918\\
39.1156097204774	0.0116286958934515\\
39.1872529720151	0.0115509609835659\\
39.2588962235527	0.0114737385440176\\
39.3305394750904	0.011397025303899\\
39.4025912659757	0.0113203848913126\\
39.474643056861	0.0112442529254152\\
39.5466948477463	0.0111686261371907\\
39.6187466386316	0.0110935012775204\\
39.6912116291116	0.0110184485860439\\
39.7636766195915	0.010943897063426\\
39.8361416100714	0.0108698434625989\\
39.9086066005514	0.0107962845566894\\
39.9814895308678	0.0107227971434021\\
40.0543724611842	0.0106498036589808\\
40.1272553915006	0.0105773008787475\\
40.200138321817	0.010505285598454\\
40.2734440149134	0.0104333411192263\\
40.3467497080097	0.0103618833681577\\
40.420055401106	0.01029090914339\\
40.4933610942023	0.0102204152636144\\
40.567094455962	0.0101499914800734\\
40.6408278177218	0.0100800472646205\\
40.7145611794816	0.0100105794386604\\
40.7882945412414	0.00994158484405892\\
40.8624605638291	0.00987265962723768\\
40.9366265864169	0.00980420686040076\\
41.0107926090047	0.00973622338859316\\
41.0849586315925	0.00966870607709654\\
41.1595623935287	0.00960125741407168\\
41.2341661554649	0.00953427412614134\\
41.3087699174011	0.00946775308229343\\
41.3833736793372	0.00940169117149249\\
41.4584203505013	0.00933569716724575\\
41.5334670216654	0.00927016150764158\\
41.6085136928295	0.00920508108584593\\
41.6835603639935	0.00914045281479475\\
41.7590552049343	0.0090758916992015\\
41.8345500458751	0.00901178194337419\\
41.910044886816	0.00894812046482127\\
41.9855397277568	0.00888490420076056\\
42.0614880926214	0.00882175433205198\\
42.137436457486	0.00875904888484294\\
42.2133848223507	0.00869678480117041\\
42.2893331872153	0.0086349590428341\\
42.3657405267235	0.00857319891096421\\
42.4421478662318	0.00851187630992766\\
42.51855520574	0.00845098820650426\\
42.5949625452482	0.00839053158735535\\
42.6718344094257	0.00833013981752427\\
42.7487062736032	0.00827017873640652\\
42.8255781377807	0.00821064533576611\\
42.9024500019582	0.00815153662738666\\
42.9797920408555	0.00809249198501387\\
43.0571340797527	0.00803387123866841\\
43.1344761186499	0.00797567140535823\\
43.2118181575471	0.00791788952225342\\
43.2896361244367	0.00786017091575464\\
43.3674540913263	0.00780286946291183\\
43.4452720582158	0.0077459822062248\\
43.5230900251054	0.00768950620850999\\
43.6013897790167	0.00763309269256469\\
43.679689532928	0.00757708963905569\\
43.7579892868393	0.00752149411619894\\
43.8362890407505	0.00746630321273593\\
43.9150765502395	0.00741117399058732\\
43.9938640597285	0.00735644859161239\\
44.0726515692175	0.00730212410996016\\
44.1514390787065	0.00724819766053689\\
44.2307204224475	0.007194332088482\\
44.3100017661886	0.00714086375303648\\
44.3892831099296	0.00708778977450331\\
44.4685644536707	0.00703510729411148\\
44.5483458244143	0.00698248488342614\\
44.6281271951579	0.00693025317619566\\
44.7079085659016	0.00687840931909708\\
44.7876899366452	0.00682695047977256\\
44.867977644264	0.00677555089927095\\
44.9482653518828	0.00672453554314274\\
45.0285530595016	0.00667390158467966\\
45.1088407671204	0.00662364621794871\\
45.1896412407794	0.00657344929701862\\
45.2704417144383	0.00652363017609793\\
45.3512421880972	0.00647418605526305\\
45.4320426617561	0.00642511415504081\\
45.5133624533588	0.00637609988574335\\
45.5946822449614	0.00632745704741867\\
45.676002036564	0.00627918286702162\\
45.7573218281666	0.00623127459161383\\
45.8391676159519	0.00618342313065949\\
45.9210134037371	0.00613593678754919\\
46.0028591915224	0.00608881281613532\\
46.0847049793077	0.0060420484901263\\
46.1670835705048	0.00599534016136842\\
46.2494621617019	0.00594899069375993\\
46.3318407528991	0.00590299736802183\\
46.4142193440962	0.00585735748464703\\
46.4971376778898	0.00581177278129763\\
46.5800560116834	0.00576654073926873\\
46.662974345477	0.00572165866616558\\
46.7458926792706	0.00567712388939528\\
46.8293578314074	0.00563264347523773\\
46.9128229835443	0.00558850957985416\\
46.9962881356811	0.00554471953778898\\
47.079753287818	0.00550127070347099\\
47.1637724728227	0.00545787541537282\\
47.2477916578275	0.00541482056115603\\
47.3318108428323	0.00537210350240151\\
47.4158300278371	0.00532972162066845\\
47.5004106040352	0.0052873924694722\\
47.5849911802333	0.00524539772530554\\
47.6695717564315	0.00520373477689168\\
47.7541523326296	0.00516240103303589\\
47.8393018049434	0.00512111920562268\\
47.9244512772573	0.00508016581679478\\
48.0096007495711	0.00503953828251076\\
48.094750221885	0.00499923403894719\\
48.1804762463428	0.00495898089937591\\
48.2662022708007	0.0049190502886966\\
48.3519282952586	0.00487943965017936\\
48.4376543197165	0.00484014644750752\\
48.5239647071256	0.00480090353854529\\
48.6102750945348	0.00476197730784605\\
48.6965854819439	0.00472336522607865\\
48.7828958693531	0.00468506478448338\\
48.8697985896823	0.00464681382875276\\
48.9567013100115	0.00460887376000927\\
49.0436040303407	0.00457124207641052\\
49.1305067506699	0.00453391629673136\\
49.2180099370951	0.00449663919798372\\
49.3055131235204	0.00445966725454878\\
49.3930163099456	0.00442299799219265\\
49.4805194963708	0.00438662895711495\\
49.5686314502397	0.00435030780107222\\
49.6567434041087	0.00431428612851808\\
49.7448553579776	0.0042785614928997\\
49.8329673118465	0.00424313146773298\\
49.9216965075321	0.00420774852298609\\
50.0104257032176	0.00417265944998919\\
50.0991548989032	0.00413786182985721\\
50.1878840945888	0.0041033532633663\\
50.2772391829867	0.00406889098278228\\
50.3665942713846	0.00403471702249687\\
50.4559493597825	0.00400082899120122\\
50.5453044481804	0.00396722451692335\\
50.6352942631106	0.00393366553785412\\
50.7252840780408	0.00390038938810359\\
50.8152738929709	0.00386739370378116\\
50.9052637079011	0.00383467614019992\\
50.9958972708063	0.00380200328543076\\
51.0865308337115	0.00376960782955578\\
51.1771643966166	0.00373748743599759\\
51.2677979595218	0.00370563978738604\\
51.3590844844193	0.0036738360659089\\
51.4503710093168	0.00364230437351957\\
51.5416575342142	0.00361104240090515\\
51.6329440591117	0.00358004785802467\\
51.7248929579717	0.00354909646563834\\
51.8168418568316	0.00351841179319263\\
51.9087907556916	0.00348799155864927\\
52.0007396545516	0.00345783349932362\\
52.0933605443554	0.0034277178186244\\
52.1859814341592	0.0033978636094565\\
52.278602323963	0.00336826861707734\\
52.3712232137668	0.00333893060619868\\
52.4645259207004	0.00330963420758715\\
52.557828627634	0.00328059409290131\\
52.6511313345676	0.00325180803469617\\
52.7444340415012	0.00322327382516092\\
52.8384286082082	0.00319478046682219\\
52.9324231749152	0.00316653826566277\\
53.0264177416223	0.00313854502154902\\
53.1204123083293	0.00311079855415405\\
53.2151089998621	0.00308309218249259\\
53.3098056913948	0.00305563190216036\\
53.4045023829275	0.00302841554036563\\
53.4991990744603	0.00300144094415808\\
53.5946083849179	0.00297450569401672\\
53.6900176953756	0.00294781153025624\\
53.7854270058332	0.00292135630750035\\
53.8808363162909	0.00289513789994566\\
53.9769689760186	0.00286895809468203\\
54.0731016357463	0.00284301443177693\\
54.169234295474	0.00281730479326392\\
54.2653669552017	0.00279182708027738\\
54.3622339372453	0.00276638723206448\\
54.4591009192888	0.00274117864312661\\
54.5559679013324	0.00271619922277577\\
54.652834883376	0.00269144689896537\\
54.7504474111939	0.00266673170873248\\
54.8480599390117	0.00264224295563833\\
54.9456724668296	0.0026179785760125\\
55.0432849946475	0.00259393652459559\\
55.1416545496904	0.00256993088209179\\
55.2400241047332	0.00254614691542417\\
55.338393659776	0.00252258258772126\\
55.4367632148189	0.00249923588050044\\
55.5359015449639	0.002475924864132\\
55.635039875109	0.00245283082297484\\
55.734178205254	0.00242995174683785\\
55.8333165353991	0.00240728564398628\\
55.9332356621868	0.00238465452095134\\
56.0331547889744	0.0023622357330183\\
56.1330739157621	0.00234002729664748\\
56.2329930425498	0.00231802724685029\\
56.3337052708769	0.0022960614726449\\
56.434417499204	0.00227430345385329\\
56.5351297275311	0.00225275123356579\\
56.6358419558582	0.00223140287357194\\
56.7373598818851	0.00221008809200737\\
56.8388778079119	0.00218897654649643\\
56.9403957339388	0.00216806630672968\\
57.0419136599657	0.00214735546130253\\
57.1442501810399	0.00212667750410978\\
57.246586702114	0.00210619832383737\\
57.3489232231882	0.00208591601677796\\
57.4512597442624	0.002065828698195\\
57.5544280683576	0.00204577358478259\\
57.6575963924529	0.00202591284925541\\
57.7607647165482	0.0020062446145583\\
57.8639330406435	0.00198676702232323\\
57.9679466963614	0.00196732095945083\\
58.0719603520793	0.00194806493541925\\
58.1759740077972	0.00192899709978763\\
58.2799876635152	0.0019101156202674\\
58.3848605103868	0.00189126500169325\\
58.4897333572585	0.0018726001428065\\
58.5946062041302	0.00185411921956467\\
58.6994790510019	0.00183582042560368\\
58.8052252907196	0.00181755183159132\\
58.9109715304374	0.00179946477787778\\
59.0167177701551	0.00178155746646795\\
59.1224640098728	0.00176382811687761\\
59.2290981971279	0.00174612831388384\\
59.3357323843829	0.00172860589127558\\
59.442366571638	0.00171125907686606\\
59.549000758893	0.00169408611600837\\
59.6617505691698	0.00167611543178471\\
59.7745003794465	0.00165833514013953\\
59.8872501897232	0.00164074322954053\\
60	0.00162333770996781\\
};
\addlegendentry{Infetti}

\end{axis}
\end{tikzpicture}%}}
\\
\subfloat[][Nodo 3.]
{\resizebox{0.45\textwidth}{!}
{% This file was created by matlab2tikz.
%
\definecolor{mycolor1}{rgb}{0.00000,0.44700,0.74100}%
\definecolor{mycolor2}{rgb}{0.85000,0.32500,0.09800}%
%
\begin{tikzpicture}

\begin{axis}[%
width=6.028in,
height=4.754in,
at={(1.011in,0.642in)},
scale only axis,
xmin=0,
xmax=60,
xlabel style={font=\color{white!15!black}},
xlabel={T},
ymode=log,
ymin=1e-14,
ymax=1,
yminorticks=true,
ylabel style={font=\color{white!15!black}},
ylabel={Errore assoluto},
axis background/.style={fill=white},
legend style={legend cell align=left, align=left, draw=white!15!black}
]
\addplot [color=mycolor1, line width=2.0pt]
  table[row sep=crcr]{%
0	0\\
0.000167459095433972	2.12052597703405e-14\\
0.000334918190867944	1.69086966650411e-13\\
0.000502377286301916	5.70432590052405e-13\\
0.000669836381735888	1.35214062169098e-12\\
0.00150713185890575	1.5397128017014e-11\\
0.00234442733607561	5.79385428522983e-11\\
0.00318172281324547	1.4478251930683e-10\\
0.00401901829041533	2.9171698390229e-10\\
0.00820549567626463	2.47902642858122e-09\\
0.0123919730621139	8.52610382295893e-09\\
0.0165784504479632	2.03856569402916e-08\\
0.0207649278338125	3.99989424915148e-08\\
0.041697314763059	3.21537886871504e-07\\
0.0626297016923055	1.08153758882068e-06\\
0.083562088621552	2.55004560456573e-06\\
0.104494475550799	4.95034395353589e-06\\
0.209156410197031	3.83987437618805e-05\\
0.313818344843264	0.000124862428860895\\
0.418480279489496	0.000285240565031319\\
0.523142214135729	0.000537367787855403\\
0.693500300195606	0.00118205304885921\\
0.863858386255483	0.00215522208286456\\
1.03421647231536	0.00348804548033899\\
1.20457455837524	0.00520126518475572\\
1.44051447786383	0.00822112462124558\\
1.67645439735243	0.0119745520784267\\
1.91239431684103	0.0164250184957161\\
2.14833423632963	0.0215302694261397\\
2.44722092525848	0.0288557037071069\\
2.74610761418733	0.0369847521605157\\
3.04499430311618	0.0457583992833683\\
3.34388099204503	0.0550469141097905\\
3.7078444355742	0.0668810070121546\\
4.07180787910338	0.0790377120619541\\
4.43577132263256	0.0913051151964988\\
4.79973476616174	0.103549433392614\\
5.20292074647974	0.116962974080918\\
5.60610672679773	0.13005053960648\\
6.00929270711573	0.142704147196024\\
6.41247868743373	0.154884160034565\\
6.85758807700814	0.167761367178154\\
7.30269746658256	0.17998914068122\\
7.74780685615697	0.191555833110132\\
8.19291624573139	0.202489469229862\\
8.68220109096914	0.213819157484482\\
9.17148593620689	0.224441835824522\\
9.66077078144463	0.234390945749897\\
10.1500556266824	0.243715193874306\\
10.6416653734136	0.252500654395663\\
11.1332751201447	0.260731679313261\\
11.6248848668759	0.268442859839898\\
12.1164946136071	0.275669328996255\\
12.6079732666127	0.282441473240408\\
13.0994519196183	0.288788960021593\\
13.5909305726239	0.294738740861025\\
14.0824092256295	0.300315492163557\\
14.573895267778	0.305542147288272\\
15.0653813099266	0.310441045156035\\
15.5568673520752	0.315032999219405\\
16.0483533942238	0.319336572985861\\
16.5398390198718	0.323369171181932\\
17.0313246455198	0.327148581754091\\
17.5228102711679	0.330691093258076\\
18.0142958968159	0.334011238329952\\
18.5057815459406	0.337122696971682\\
18.9972671950653	0.340039393502076\\
19.48875284419	0.342774066317036\\
19.9802384933148	0.345338147377638\\
20.5017696853463	0.347884385240239\\
21.0233008773778	0.350263443967116\\
21.5448320694093	0.352487048844214\\
22.0663632614408	0.35456568974827\\
22.6746017709939	0.356819753758529\\
23.282840280547	0.358905140929333\\
23.8910787901001	0.36083563295391\\
24.4993172996532	0.362623327212993\\
25.0869400675273	0.364225261036058\\
25.6745628354014	0.365714105497706\\
26.2621856032755	0.367098643338712\\
26.8498083711495	0.368386739429506\\
27.5306427936611	0.369768017264732\\
28.2114772161728	0.371040154913207\\
28.8923116386844	0.372212697290781\\
29.573146061196	0.373294074342671\\
30.3571240725376	0.374436246776342\\
31.1411020838792	0.375478557712352\\
31.9250800952207	0.376430649133277\\
32.7090581065623	0.377300946688868\\
33.6393150845161	0.378237869436599\\
34.5695720624699	0.379081806165915\\
35.4998290404237	0.379842842411353\\
36.4300860183775	0.380529681280664\\
37.5675806203718	0.381279849115068\\
38.7050752223662	0.381943160517947\\
39.8425698243606	0.382530414794223\\
40.980064426355	0.383050788260447\\
42.4285780340909	0.383629107951331\\
43.8770916418269	0.384126207215411\\
45.3256052495628	0.384554074914151\\
46.7741188572988	0.384922672039908\\
48.2741188572988	0.38525089357549\\
49.7741188572988	0.385532686648326\\
51.2741188572988	0.385774814845653\\
52.7741188572988	0.385982984824378\\
54.2741188572988	0.386162049756273\\
55.7741188572988	0.386316169483394\\
57.2741188572988	0.386448878602247\\
58.7741188572988	0.386563192897077\\
59.0805891429741	0.386584525872112\\
59.3870594286494	0.386605220006483\\
59.6935297143247	0.386625294696691\\
60	0.386644768725527\\
};
\addlegendentry{Suscettibili}

\addplot [color=mycolor2, line width=2.0pt]
  table[row sep=crcr]{%
0	0\\
0.000167459095433972	2.11305836386166e-14\\
0.000334918190867944	1.69034043845751e-13\\
0.000502377286301916	5.70454069226378e-13\\
0.000669836381735888	1.35210251526356e-12\\
0.00150713185890575	1.5396460919911e-11\\
0.00234442733607561	5.79350034404249e-11\\
0.00318172281324547	1.44770929082533e-10\\
0.00401901829041533	2.91687586575017e-10\\
0.00820549567626463	2.47851845158776e-09\\
0.0123919730621139	8.52345888571967e-09\\
0.0165784504479632	2.03771961559774e-08\\
0.0207649278338125	3.99781532033026e-08\\
0.041697314763059	3.21203867945745e-07\\
0.0626297016923055	1.07983198354832e-06\\
0.083562088621552	2.54468075446988e-06\\
0.104494475550799	4.93733350289598e-06\\
0.209156410197031	3.82234795484517e-05\\
0.313818344843264	0.000123908484158344\\
0.418480279489496	0.000282210280901781\\
0.523142214135729	0.000530151035230474\\
0.693500300195606	0.00116113649772319\\
0.863858386255483	0.00210710153464147\\
1.03421647231536	0.00339334341915044\\
1.20457455837524	0.00503505889036449\\
1.44051447786383	0.00790471353596028\\
1.67645439735243	0.0114319782292086\\
1.91239431684103	0.0155642324448038\\
2.14833423632963	0.0202471070108581\\
2.44722092525848	0.0268724448374547\\
2.74610761418733	0.0340932691019478\\
3.04499430311618	0.0417327633043924\\
3.34388099204503	0.0496540015442231\\
3.7078444355742	0.0595053131287592\\
4.07180787910338	0.0693197153483811\\
4.43577132263256	0.0788864017993465\\
4.79973476616174	0.0880875731814273\\
5.20292074647974	0.097754849921034\\
5.60610672679773	0.106719698403456\\
6.00929270711573	0.114901116007684\\
6.41247868743373	0.122294421525055\\
6.85758807700814	0.129563959996664\\
7.30269746658256	0.135881667101538\\
7.74780685615697	0.141275951305753\\
8.19291624573139	0.145815893481157\\
8.68220109096914	0.149907175530158\\
9.17148593620689	0.153113462947913\\
9.66077078144463	0.155509265235361\\
10.1500556266824	0.157180751719896\\
10.6416653734136	0.158210681904858\\
11.1332751201447	0.158650577111093\\
11.6248848668759	0.158562868528608\\
12.1164946136071	0.158006440445983\\
12.6079732666127	0.157034731305336\\
13.0994519196183	0.155695528010961\\
13.5909305726239	0.154033174392667\\
14.0824092256295	0.15208662700242\\
14.573895267778	0.14989114352621\\
15.0653813099266	0.147481318109886\\
15.5568673520752	0.144888041550578\\
16.0483533942238	0.142137814806487\\
16.5398390198718	0.139254726395755\\
17.0313246455198	0.136262973387668\\
17.5228102711679	0.133183840708282\\
18.0142958968159	0.130035463019832\\
18.5057815459406	0.126834259424789\\
18.9972671950653	0.123596614317112\\
19.48875284419	0.120336806698526\\
19.9802384933148	0.117066925173526\\
20.5017696853463	0.113598181626494\\
21.0233008773778	0.110142880328805\\
21.5448320694093	0.106711835190426\\
22.0663632614408	0.103313946191057\\
22.6746017709939	0.0994032432302367\\
23.282840280547	0.0955597467854827\\
23.8910787901001	0.0917929377005569\\
24.4993172996532	0.0881099420897194\\
25.0869400675273	0.08463679328364\\
25.6745628354014	0.0812520643824299\\
26.2621856032755	0.0779592566990827\\
26.8498083711495	0.0747607454489119\\
27.5306427936611	0.0711748696445441\\
28.2114772161728	0.0677195428273623\\
28.8923116386844	0.0643954017922297\\
29.573146061196	0.0612019677841101\\
30.3571240725376	0.0576850572306711\\
31.1411020838792	0.0543373246672689\\
31.9250800952207	0.0511552576158859\\
32.7090581065623	0.0481345196357281\\
33.6393150845161	0.0447527706594896\\
34.5695720624699	0.0415826612674732\\
35.4998290404237	0.0386149265660073\\
36.4300860183775	0.0358400345276016\\
37.5675806203718	0.0326950467710911\\
38.7050752223662	0.0298061091003939\\
39.8425698243606	0.0271556619959117\\
40.980064426355	0.0247270526372696\\
42.4285780340909	0.0219306050678202\\
43.8770916418269	0.0194355729297507\\
45.3256052495628	0.0172118495994503\\
46.7741188572988	0.015232899989236\\
48.2741188572988	0.0134156864054742\\
49.7741188572988	0.0118084207470409\\
51.2741188572988	0.0103877034986972\\
52.7741188572988	0.009133389508066\\
54.2741188572988	0.00802737181389069\\
55.7741188572988	0.00705221285297851\\
57.2741188572988	0.00619272619391241\\
58.7741188572988	0.00543589936071481\\
59.0805891429741	0.00529283991048947\\
59.3870594286494	0.00515346931442191\\
59.6935297143247	0.00501769516199231\\
60	0.00488542737426637\\
};
\addlegendentry{Infetti}

\end{axis}
\end{tikzpicture}%}}
\caption[Errori assoluti relativi al grafo~\ref{fig::3nodi} tra modello esatto e chiuso alle coppie] {Errore assoluto (in scala logaritmica) tra la soluzione del problema di Cauchy~\eqref{3nodi} e quello utilizzando la chiusura alle coppie.\\
Per ottenere i grafici abbiamo risolto numericamente, con una tolleranza di $1e-9$ i due problemi di Cauchy con condizioni iniziali di stati puri~\eqref{statipuri}. Abbiamo inoltre supposto l'indipendenza statica di tali condizioni iniziali.\\
Per la sperimentazione abbiamo usato come parametri $\tau = 0.3$ e $\gamma = 0.1$. }
\label{fig::errori3nodiPair}
\end{figure}
%%% Local Variables:
%%% mode: latex
%%% TeX-master: "main"
%%% End:

