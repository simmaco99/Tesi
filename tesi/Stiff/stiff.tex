\documentclass[11pt]{article}
%Formattazione e codifica caratteri
\usepackage[italian]{babel}
\usepackage[utf8]{inputenc}
\usepackage{microtype}
\usepackage{libertine}
% Tavola dei contenuti
\usepackage{tocloft}
\renewcommand{\cftdot}{}
\setlength{\cftbeforechapskip}{5pt}
% Figure
\usepackage{graphicx}
\usepackage{caption}
\usepackage{subfig}
% Matematica
\usepackage{amsmath,amsfonts,amssymb,amsthm}
\usepackage{color}
% Tabelle
\usepackage{booktabs}
%Listati
\usepackage{listings}
\lstloadlanguages{MATLAB}
% Testatine
\author{S. Di Lillo}
\usepackage{fancyhdr}
\pagestyle{fancy}
\setlength{\headheight}{15pt}
\fancyhf{}
\fancyhead[LE,RO]{S. Di Lillo}
\fancyhead[RE,LO]{\leftmark}
\fancyfoot[CE,CO]{}
\fancyfoot[LE,RO]{\thepage}
\renewcommand{\headrulewidth}{1.5pt}
% Titoli dei capitoli che "rubano" meno spazio
\usepackage{titlesec, blindtext}
\usepackage[dvipsnames]{xcolor}
\definecolor{gray75}{gray}{0.75}
\newcommand{\hsp}{\hspace{15pt}}
\titleformat{\chapter}[hang]{\LARGE\bfseries}{\thechapter\hsp\textcolor{gray75}{|}\hsp}{0pt}{\Large\bfseries}
\titleformat{\section}[hang]{\Large\bfseries}{\thesection\hsp\textcolor{gray75}{|}\hsp}{0pt}{\large\bfseries}

%Grafi
\usepackage{tikz-network}
\SetVertexStyle[MinSize=1\DefaultUnit,FillColor=white]
% Hyperref e metadati
\usepackage[linktocpage,
			colorlinks,
			citecolor=ForestGreen,
			bookmarks,
			pdftitle={Il Modello Epidemiologico SIR sulle Reti Complesse},
			pdfauthor={S. Di Lillo},
			pdfsubject={Tesi di Laurea Triennale},
			pdfkeywords={SIR,Epidemiologia,Reti Complesse},
			]{hyperref}


% Per importare le figure in MATLAB
\usepackage{pgfplots}
 \pgfplotsset{compat=newest}
  \usetikzlibrary{plotmarks}
  \usetikzlibrary{arrows.meta}
  \usepgfplotslibrary{patchplots}
  \usepackage{grffile}
\usepackage{tikz}
\pgfplotsset{plot coordinates/math parser=false}
  \newlength\figureheight
 \newlength\figurewidth

%Per le osservazioni
\theoremstyle{plain}
\newtheorem{oss}{Osservazione} 
\newtheorem{thm}{Teorema}[section] 
\newtheorem{prop}[thm]{Proposizione} 
%COMANDI PERSONALIZZATI 
\newcommand{\R}{\mathbb{R}}
\newcommand{\ro}{\mathcal R_0}
\newcommand{\di}{\,\,\mathrm{d} }
\newcommand{\tonde}[1]{\left( #1 \right)}
\newcommand{\ses}{\Leftrightarrow} 
\newcommand{\abs}[1]{\left\vert#1\right\vert}
\newcommand{\angol}[1]{\langle #1 \rangle}
\newcommand{\spa}{\mathbin{\textcolor{white}{-}}} %per indentare le equazioni in o-o-o serve pacchetto color



\begin{document}
Consideriamo il sistema 
\begin{equation}
\begin{cases}
y'= F(y)\\
y(0 )=y_0 
\end{cases}	
\end{equation}
Sia $J$ la matrice Jacobiana di $F$ definiamo l'\textit{indice di stiff} del sistema come il rapporto 
$$ \frac{\lambda_{\max}}{\lambda_{\min}}$$ 
ovvero il rapporto tra l'autovalore di modulo massimo e quello di modulo minimo.\\
Diremo che un sistema \`e stiff se l'indice \`e un numero molto maggiore di $1$.\\ \\
Andiamo a studiare come i parametri ($\tau$,$\gamma$ e la dimensione del network) influiscono sul fatto che il modello chiuso alle coppie sia pi\`u o meno stiff. Per fare questo abbiamo calcolato l'indice di stiff per alcune matrici di Erdos-Renyi vedi Figura~\ref{Erdos}.\\
Per tali matrici inoltre abbiamo osservato anche come aumentasse il numero di intervalli temporali usando le funzione di MATLAB ode45 e ode 15s. Come si evince dai grafici~\ref{Erdos_Lenght} la funzione ode15s richiede sempre un numero minore di intervalli temporale. Possiamo dunque conclduere che il modello in esame risulta stiff.\\ \\
Andiamo ora a indagare sul tempo di calcolo, in particolare abbiamo confrontato ode15s (a cui abbiamo passato anche il Jacobiano) con ode45 ottenendo che in tutti i casi la funzione ode45 impiega meno tempo~\ref{Erdos_tempo}.\\ \\ 
In~\ref{stiff_soluzione} si osserva come varia l'indice di stiff lungo le soluzioni del sistema di ode.

\begin{figure}[h]
\centering
\subfloat[][$N=10$]
{\resizebox{0.45\textwidth}{!}{% This file was created by matlab2tikz.
%
%The latest updates can be retrieved from
%  http://www.mathworks.com/matlabcentral/fileexchange/22022-matlab2tikz-matlab2tikz
%where you can also make suggestions and rate matlab2tikz.
%
\definecolor{mycolor1}{rgb}{0.00000,0.44700,0.74100}%
%
\begin{tikzpicture}

\begin{axis}[%
width=1.66in,
height=1.258in,
at={(1.011in,4.137in)},
scale only axis,
xmin=0,
xmax=1,
xlabel style={font=\color{white!15!black}},
xlabel={$\tau$},
ymin=0,
ymax=8e+15,
axis background/.style={fill=white},
title style={font=\bfseries},
title={$\gamma\text{= 0.10}$}
]
\addplot [color=mycolor1, line width=2.0pt, forget plot]
  table[row sep=crcr]{%
0.1	7.3365674687604e+15\\
0.109090909090909	41.0302259215271\\
0.118181818181818	22.7111987976705\\
0.127272727272727	16.6298455111258\\
0.136363636363636	13.5926820366811\\
0.145454545454545	13.608544866257\\
0.154545454545455	14.066425613187\\
0.163636363636364	14.5141006240047\\
0.172727272727273	14.9512916150482\\
0.181818181818182	15.3777905720538\\
0.190909090909091	15.7934798217266\\
0.2	16.1983359773607\\
0.209090909090909	16.592426747234\\
0.218181818181818	16.9759042481577\\
0.227272727272727	17.3489963237616\\
0.236363636363636	17.7119965560128\\
0.245454545454545	18.0652534029568\\
0.254545454545455	18.4091588562244\\
0.263636363636364	18.769636814415\\
0.272727272727273	19.2086883382031\\
0.281818181818182	19.6488401909359\\
0.290909090909091	20.0898144550306\\
0.3	20.5711090943093\\
0.309090909090909	21.1029954540012\\
0.318181818181818	21.6372211349414\\
0.327272727272727	22.1734957744219\\
0.336363636363636	22.7115739776902\\
0.345454545454545	23.25124714956\\
0.354545454545454	23.792336999749\\
0.363636363636364	24.3346903455475\\
0.372727272727273	24.8781749263841\\
0.381818181818182	25.4226760126145\\
0.390909090909091	30.7095472570638\\
0.4	35.0899699751397\\
0.409090909090909	38.7893180185742\\
0.418181818181818	42.216405229894\\
0.427272727272727	45.5029624708996\\
0.436363636363636	48.7102583811904\\
0.445454545454545	51.8721239208753\\
0.454545454545455	55.009236802961\\
0.463636363636364	58.1351093263299\\
0.472727272727273	61.2589933993713\\
0.481818181818182	64.3874407959634\\
0.490909090909091	67.5252067239969\\
0.5	70.6758044707869\\
0.509090909090909	73.8418620412831\\
0.518181818181818	77.0253603296308\\
0.527272727272727	80.2277972633023\\
0.536363636363636	83.4503039739782\\
0.545454545454546	86.693728902678\\
0.554545454545455	89.958699893782\\
0.563636363636364	93.2456708262942\\
0.572727272727273	96.5549571598863\\
0.581818181818182	99.8867633903624\\
0.590909090909091	103.241204505226\\
0.6	106.618322925949\\
0.609090909090909	110.018102011326\\
0.618181818181818	113.440476910042\\
0.627272727272727	116.885343348366\\
0.636363636363636	120.352564793864\\
0.645454545454545	123.841978330749\\
0.654545454545455	127.353399504821\\
0.663636363636364	130.886626338279\\
0.672727272727273	134.441442671158\\
0.681818181818182	138.017620953146\\
0.690909090909091	141.614924584271\\
0.7	145.233109883054\\
0.709090909090909	148.871927745903\\
0.718181818181818	152.531125049044\\
0.727272727272727	156.210445835097\\
0.736363636363636	159.909632318673\\
0.745454545454545	163.628425739433\\
0.754545454545455	167.366567086007\\
0.763636363636364	171.123797710364\\
0.772727272727273	174.899859848926\\
0.781818181818182	178.694497064095\\
0.790909090909091	182.507454617661\\
0.8	186.338479785823\\
0.809090909090909	190.187322124071\\
0.818181818181818	194.05373368886\\
0.827272727272727	197.937469222045\\
0.836363636363636	201.838286303247\\
0.845454545454545	205.75594547431\\
0.854545454545454	209.690210339887\\
0.863636363636364	213.640847647051\\
0.872727272727273	217.607627346966\\
0.881818181818182	221.590322640887\\
0.890909090909091	225.588710012644\\
0.9	229.602569249391\\
0.909090909090909	233.631683452074\\
0.918181818181818	237.675839037231\\
0.927272727272727	241.734825731147\\
0.936363636363636	245.808436557374\\
0.945454545454545	249.896467818632\\
0.954545454545455	253.998719073867\\
0.963636363636364	258.114993111095\\
0.972727272727273	262.245095916696\\
0.981818181818182	266.388836641739\\
0.990909090909091	270.546027565653\\
1	274.716484057871\\
};
\end{axis}

\begin{axis}[%
width=1.66in,
height=1.258in,
at={(3.195in,4.137in)},
scale only axis,
xmin=0,
xmax=1,
xlabel style={font=\color{white!15!black}},
xlabel={$\tau$},
ymin=0,
ymax=8e+15,
axis background/.style={fill=white},
title style={font=\bfseries},
title={$\gamma\text{= 0.20}$}
]
\addplot [color=mycolor1, line width=2.0pt, forget plot]
  table[row sep=crcr]{%
0.1	35.3522161216637\\
0.109090909090909	267.632680474365\\
0.118181818181818	56.0309873958523\\
0.127272727272727	54.0792278259292\\
0.136363636363636	55.6218181271794\\
0.145454545454545	24.4243398567402\\
0.154545454545455	16.9189095173904\\
0.163636363636364	15.5819091531722\\
0.172727272727273	17.9138266271542\\
0.181818181818182	30.7398401586874\\
0.190909090909091	68.4602733292083\\
0.2	7.3365674687604e+15\\
0.209090909090909	77.7307298421096\\
0.218181818181818	41.0302259215271\\
0.227272727272727	28.808888286343\\
0.236363636363636	22.7111987976705\\
0.245454545454545	19.0603105284539\\
0.254545454545455	16.6298455111259\\
0.263636363636364	14.8946465108861\\
0.272727272727273	13.5926820366811\\
0.281818181818182	13.3758706349325\\
0.290909090909091	13.608544866257\\
0.3	13.8387419849742\\
0.309090909090909	14.066425613187\\
0.318181818181818	14.2915574613881\\
0.327272727272727	14.5141006240047\\
0.336363636363636	14.7340216175104\\
0.345454545454545	14.9512916150482\\
0.354545454545454	15.1658871645019\\
0.363636363636364	15.3777905720538\\
0.372727272727273	15.5869900670232\\
0.381818181818182	15.7934798217266\\
0.390909090909091	15.9972598733038\\
0.4	16.1983359773607\\
0.409090909090909	16.3967194124086\\
0.418181818181818	16.592426747234\\
0.427272727272727	16.7854795791001\\
0.436363636363636	16.9759042481577\\
0.445454545454545	17.1637315320218\\
0.454545454545455	17.3489963237617\\
0.463636363636364	17.5317372962767\\
0.472727272727273	17.7119965560128\\
0.481818181818182	17.889819289077\\
0.490909090909091	18.0652534029568\\
0.5	18.2383491671684\\
0.509090909090909	18.4091588562244\\
0.518181818181818	18.5777363982915\\
0.527272727272727	18.769636814415\\
0.536363636363636	18.9890076200486\\
0.545454545454546	19.2086883382031\\
0.554545454545455	19.4286441420975\\
0.563636363636364	19.6488401909359\\
0.572727272727273	19.8692417749193\\
0.581818181818182	20.0898144550306\\
0.590909090909091	20.3105241953656\\
0.6	20.5711090943093\\
0.609090909090909	20.8367396895067\\
0.618181818181818	21.1029954540012\\
0.627272727272727	21.3698351900781\\
0.636363636363636	21.6372211349415\\
0.645454545454545	21.9051186210093\\
0.654545454545455	22.1734957744219\\
0.663636363636364	22.4423232470204\\
0.672727272727273	22.7115739776902\\
0.681818181818182	22.9812229795028\\
0.690909090909091	23.25124714956\\
0.7	23.5216250988471\\
0.709090909090909	23.792336999749\\
0.718181818181818	24.0633644491808\\
0.727272727272727	24.3346903455475\\
0.736363636363636	24.6062987779666\\
0.745454545454545	24.8781749263841\\
0.754545454545455	25.1503049713809\\
0.763636363636364	25.4226760126145\\
0.772727272727273	27.6006291136912\\
0.781818181818182	30.7095472570638\\
0.790909090909091	33.040372545547\\
0.8	35.0899699751401\\
0.809090909090909	36.9874519382409\\
0.818181818181818	38.7893180185739\\
0.827272727272727	40.5260994405942\\
0.836363636363636	42.216405229894\\
0.845454545454545	43.8725073064134\\
0.854545454545454	45.5029624708996\\
0.863636363636364	47.1139901525664\\
0.872727272727273	48.7102583811902\\
0.881818181818182	50.2953612726599\\
0.890909090909091	51.8721239208753\\
0.9	53.4428051268115\\
0.909090909090909	55.009236802961\\
0.918181818181818	56.5729225891852\\
0.927272727272727	58.1351093263302\\
0.936363636363636	59.6968399562547\\
0.945454545454545	61.2589933993717\\
0.954545454545455	62.8223151052411\\
0.963636363636364	64.3874407959634\\
0.972727272727273	65.9549151574209\\
0.981818181818182	67.5252067239973\\
0.990909090909091	69.0987198562339\\
1	70.6758044707869\\
};
\end{axis}

\begin{axis}[%
width=1.66in,
height=1.258in,
at={(5.379in,4.137in)},
scale only axis,
xmin=0,
xmax=1,
xlabel style={font=\color{white!15!black}},
xlabel={$\tau$},
ymin=0,
ymax=3e+16,
axis background/.style={fill=white},
title style={font=\bfseries},
title={$\gamma\text{= 0.30}$}
]
\addplot [color=mycolor1, line width=2.0pt, forget plot]
  table[row sep=crcr]{%
0.1	7.56782937784212\\
0.109090909090909	8.00446978986753\\
0.118181818181818	9.64834579265795\\
0.127272727272727	12.7303020465381\\
0.136363636363636	17.629650421866\\
0.145454545454545	26.8267967166534\\
0.154545454545455	50.6977562917487\\
0.163636363636364	267.632680474359\\
0.172727272727273	91.4191828901314\\
0.181818181818182	40.8181813194745\\
0.190909090909091	54.0792278259287\\
0.2	118.175688966578\\
0.209090909090909	37.9151755905754\\
0.218181818181818	24.4243398567402\\
0.227272727272727	18.711849646602\\
0.236363636363636	15.5176650717098\\
0.245454545454545	15.5819091531722\\
0.254545454545455	17.0069670447415\\
0.263636363636364	18.9991288548541\\
0.272727272727273	30.7398401586872\\
0.281818181818182	49.9113218271788\\
0.290909090909091	105.276890737475\\
0.3	2.55145336870524e+16\\
0.309090909090909	114.422653110314\\
0.318181818181818	59.3801553843773\\
0.327272727272727	41.0302259215275\\
0.336363636363636	31.8619828198425\\
0.345454545454545	26.3682027173634\\
0.354545454545454	22.7111987976706\\
0.363636363636364	20.1027443626097\\
0.372727272727273	18.148607724631\\
0.381818181818182	16.6298455111259\\
0.390909090909091	15.4152171188977\\
0.4	14.421327748566\\
0.409090909090909	13.5926820366811\\
0.418181818181818	13.2977674805116\\
0.427272727272727	13.453701921383\\
0.436363636363636	13.6085448662571\\
0.445454545454545	13.7622870126652\\
0.454545454545455	13.9149176942526\\
0.463636363636364	14.066425613187\\
0.472727272727273	14.2167993762249\\
0.481818181818182	14.3660278858247\\
0.490909090909091	14.5141006240047\\
0.5	14.6610078566527\\
0.509090909090909	14.8067407786778\\
0.518181818181818	14.9512916150482\\
0.527272727272727	15.0946536888238\\
0.536363636363636	15.2368214644016\\
0.545454545454546	15.3777905720538\\
0.554545454545455	15.5175578182636\\
0.563636363636364	15.6561211851898\\
0.572727272727273	15.7934798217266\\
0.581818181818182	15.9296340279814\\
0.590909090909091	16.064585234518\\
0.6	16.1983359773607\\
0.609090909090909	16.330889869498\\
0.618181818181818	16.4622515694382\\
0.627272727272727	16.592426747234\\
0.636363636363636	16.7214220482984\\
0.645454545454545	16.8492450552721\\
0.654545454545455	16.9759042481577\\
0.663636363636364	17.1014089629107\\
0.672727272727273	17.2257693486646\\
0.681818181818182	17.3489963237616\\
0.690909090909091	17.4711015307577\\
0.7	17.5920972905798\\
0.709090909090909	17.7119965560128\\
0.718181818181818	17.8308128647047\\
0.727272727272727	17.9485602918796\\
0.736363636363636	18.0652534029568\\
0.745454545454545	18.1809072062737\\
0.754545454545455	18.2955371061119\\
0.763636363636364	18.4091588562243\\
0.772727272727273	18.5217885140571\\
0.781818181818182	18.6334423958513\\
0.790909090909091	18.769636814415\\
0.8	18.9158478668389\\
0.809090909090909	19.0622018093133\\
0.818181818181818	19.2086883382032\\
0.827272727272727	19.3552971277039\\
0.836363636363636	19.5020178492901\\
0.845454545454545	19.6488401909359\\
0.854545454545454	19.7957538759451\\
0.863636363636364	19.9427486812467\\
0.872727272727273	20.0898144550306\\
0.881818181818182	20.236941133616\\
0.890909090909091	20.3943914074111\\
0.9	20.5711090943093\\
0.909090909090909	20.7481245540058\\
0.918181818181818	20.9254242568896\\
0.927272727272727	21.1029954540013\\
0.936363636363636	21.2808261234672\\
0.945454545454545	21.4589049211198\\
0.954545454545455	21.6372211349415\\
0.963636363636364	21.8157646430068\\
0.972727272727273	21.9945258746269\\
0.981818181818182	22.1734957744219\\
0.990909090909091	22.3526657690815\\
1	22.5320277365866\\
};
\end{axis}

\begin{axis}[%
width=1.66in,
height=1.258in,
at={(1.011in,2.39in)},
scale only axis,
xmin=0,
xmax=1,
xlabel style={font=\color{white!15!black}},
xlabel={$\tau$},
ymin=0,
ymax=8e+15,
axis background/.style={fill=white},
title style={font=\bfseries},
title={$\gamma\text{= 0.40}$}
]
\addplot [color=mycolor1, line width=2.0pt, forget plot]
  table[row sep=crcr]{%
0.1	8.48309260426806\\
0.109090909090909	7.600734360567\\
0.118181818181818	7.08062002017599\\
0.127272727272727	7.36624608358306\\
0.136363636363636	7.67250011817417\\
0.145454545454545	8.00446978986754\\
0.154545454545455	9.03974057328292\\
0.163636363636364	11.0367934752471\\
0.172727272727273	13.7297449035219\\
0.181818181818182	17.6296504218661\\
0.190909090909091	23.8409629569753\\
0.2	35.3522161216637\\
0.209090909090909	64.1355661195779\\
0.218181818181818	267.632680474365\\
0.227272727272727	135.475602448838\\
0.236363636363636	56.0309873958526\\
0.245454545454545	36.0563580543919\\
0.254545454545455	54.0792278259285\\
0.263636363636364	323.045524438301\\
0.272727272727273	55.6218181271794\\
0.281818181818182	33.0479157852466\\
0.290909090909091	24.4243398567402\\
0.3	19.8120373234866\\
0.309090909090909	16.9189095173904\\
0.318181818181818	14.9252324024303\\
0.327272727272727	15.5819091531722\\
0.336363636363636	16.6075151831311\\
0.345454545454545	17.9138266271542\\
0.354545454545454	20.761571352297\\
0.363636363636364	30.7398401586874\\
0.372727272727273	43.6568559300297\\
0.381818181818182	68.4602733292083\\
0.390909090909091	142.000126667352\\
0.4	7.3365674687604e+15\\
0.409090909090909	151.105071353213\\
0.418181818181818	77.7307298421096\\
0.427272727272727	53.2629497962122\\
0.436363636363636	41.0302259215271\\
0.445454545454545	33.6947555466789\\
0.454545454545455	28.808888286343\\
0.463636363636364	25.3227997248793\\
0.472727272727273	22.7111987976705\\
0.481818181818182	20.6821090701226\\
0.490909090909091	19.0603105284539\\
0.5	17.7343176815733\\
0.509090909090909	16.6298455111258\\
0.518181818181818	15.695509293862\\
0.527272727272727	14.8946465108861\\
0.536363636363636	14.2004068313216\\
0.545454545454546	13.5926820366811\\
0.554545454545455	13.2586142068425\\
0.563636363636364	13.3758706349325\\
0.572727272727273	13.4925153171285\\
0.581818181818182	13.608544866257\\
0.590909090909091	13.7239552436016\\
0.6	13.8387419849741\\
0.609090909090909	13.952900379851\\
0.618181818181818	14.066425613187\\
0.627272727272727	14.1793128775184\\
0.636363636363636	14.2915574613881\\
0.645454545454545	14.4031548188719\\
0.654545454545455	14.5141006240047\\
0.663636363636364	14.6243908131213\\
0.672727272727273	14.7340216175104\\
0.681818181818182	14.8429895882934\\
0.690909090909091	14.9512916150482\\
0.7	15.0589249393939\\
0.709090909090909	15.1658871645019\\
0.718181818181818	15.2721762613103\\
0.727272727272727	15.3777905720538\\
0.736363636363636	15.4827288116059\\
0.745454545454545	15.5869900670232\\
0.754545454545455	15.6905737956061\\
0.763636363636364	15.7934798217266\\
0.772727272727273	15.8957083326204\\
0.781818181818182	15.9972598733038\\
0.790909090909091	16.0981353407406\\
0.8	16.1983359773607\\
0.809090909090909	16.2978633640118\\
0.818181818181818	16.3967194124086\\
0.827272727272727	16.4949063571337\\
0.836363636363636	16.592426747234\\
0.845454545454545	16.6892834374482\\
0.854545454545454	16.7854795791001\\
0.863636363636364	16.8810186106821\\
0.872727272727273	16.9759042481577\\
0.881818181818182	17.0701404750043\\
0.890909090909091	17.1637315320218\\
0.9	17.2566819069274\\
0.909090909090909	17.3489963237616\\
0.918181818181818	17.4406797321255\\
0.927272727272727	17.5317372962767\\
0.936363636363636	17.6221743841049\\
0.945454545454545	17.7119965560128\\
0.954545454545455	17.8012095537291\\
0.963636363636364	17.889819289077\\
0.972727272727273	17.9778318327267\\
0.981818181818182	18.0652534029568\\
0.990909090909091	18.152090354453\\
1	18.2383491671684\\
};
\end{axis}

\begin{axis}[%
width=1.66in,
height=1.258in,
at={(3.195in,2.39in)},
scale only axis,
xmin=0,
xmax=1,
xlabel style={font=\color{white!15!black}},
xlabel={$\tau$},
ymin=0,
ymax=1e+16,
axis background/.style={fill=white},
title style={font=\bfseries},
title={$\gamma\text{= 0.50}$}
]
\addplot [color=mycolor1, line width=2.0pt, forget plot]
  table[row sep=crcr]{%
0.1	10.4853533652214\\
0.109090909090909	9.72328892594724\\
0.118181818181818	8.99192607398537\\
0.127272727272727	8.31698779804945\\
0.136363636363636	7.600734360567\\
0.145454545454545	7.02555005458302\\
0.154545454545455	7.24978210563321\\
0.163636363636364	7.48600778096689\\
0.172727272727273	7.7366548788151\\
0.181818181818182	8.00446978986751\\
0.190909090909091	8.69672195973106\\
0.2	10.1734598013445\\
0.209090909090909	12.0091872125148\\
0.218181818181818	14.3901060658669\\
0.227272727272727	17.6296504218661\\
0.236363636363636	22.3190416998629\\
0.245454545454545	29.7384853165322\\
0.254545454545455	43.28852876088\\
0.263636363636364	76.0037221693622\\
0.272727272727273	267.632680474362\\
0.281818181818182	191.845284886781\\
0.290909090909091	72.8107918605882\\
0.3	45.7293038895197\\
0.309090909090909	33.72904809492\\
0.318181818181818	54.0792278259284\\
0.327272727272727	2501.44010198424\\
0.336363636363636	80.4242584391619\\
0.345454545454545	43.2417993445376\\
0.354545454545454	30.764471183248\\
0.363636363636364	24.4243398567402\\
0.372727272727273	20.5564630339872\\
0.381818181818182	17.9374679944581\\
0.390909090909091	16.0399218418092\\
0.4	14.9130926355802\\
0.409090909090909	15.5819091531722\\
0.418181818181818	16.3828351467239\\
0.427272727272727	17.3511692621054\\
0.436363636363636	18.5403627276737\\
0.445454545454545	23.1785095376974\\
0.454545454545455	30.7398401586874\\
0.463636363636364	40.4997514926405\\
0.472727272727273	56.1184279937194\\
0.481818181818182	86.8899464060532\\
0.490909090909091	178.695323845714\\
0.5	9.80855275619962e+15\\
0.509090909090909	187.781830617663\\
0.518181818181818	96.0781817464704\\
0.527272727272727	65.4972948422929\\
0.536363636363636	50.2044294212574\\
0.545454545454546	41.0302259215272\\
0.554545454545455	34.9169062477219\\
0.563636363636364	30.5532523901936\\
0.572727272727273	27.283242671309\\
0.581818181818182	24.7421980261045\\
0.590909090909091	22.7111987976705\\
0.6	21.0508836266554\\
0.609090909090909	19.668331492877\\
0.618181818181818	18.4992178180279\\
0.627272727272727	17.4976109158859\\
0.636363636363636	16.6298455111258\\
0.645454545454545	15.8706910690051\\
0.654545454545455	15.2008709591945\\
0.663636363636364	14.6054077304527\\
0.672727272727273	14.0724905138239\\
0.681818181818182	13.5926820366811\\
0.690909090909091	13.2350897476081\\
0.7	13.3290413115149\\
0.709090909090909	13.4226020893923\\
0.718181818181818	13.5157705590452\\
0.727272727272727	13.608544866257\\
0.736363636363636	13.7009229115961\\
0.745454545454545	13.7929024225031\\
0.754545454545455	13.8844810131343\\
0.763636363636364	13.9756562340126\\
0.772727272727273	14.0664256131869\\
0.781818181818182	14.1567866903148\\
0.790909090909091	14.2467370448391\\
0.8	14.3362743192346\\
0.809090909090909	14.4253962381336\\
0.818181818181818	14.5141006240047\\
0.827272727272727	14.6023854099457\\
0.836363636363636	14.6902486500574\\
0.845454545454545	14.7776885277885\\
0.854545454545454	14.8647033625753\\
0.863636363636364	14.9512916150482\\
0.872727272727273	15.0374518910303\\
0.881818181818182	15.1231829445175\\
0.890909090909091	15.2084836797967\\
0.9	15.2933531528361\\
0.909090909090909	15.3777905720538\\
0.918181818181818	15.4617952985599\\
0.927272727272727	15.5453668459469\\
0.936363636363636	15.6285048796933\\
0.945454545454545	15.7112092162311\\
0.954545454545455	15.7934798217266\\
0.963636363636364	15.8753168106061\\
0.972727272727273	15.956720443861\\
0.981818181818182	16.0376911271572\\
0.990909090909091	16.1182294087696\\
1	16.1983359773607\\
};
\end{axis}

\begin{axis}[%
width=1.66in,
height=1.258in,
at={(5.379in,2.39in)},
scale only axis,
xmin=0,
xmax=1,
xlabel style={font=\color{white!15!black}},
xlabel={$\tau$},
ymin=0,
ymax=1e+16,
axis background/.style={fill=white},
title style={font=\bfseries},
title={$\gamma\text{= 0.60}$}
]
\addplot [color=mycolor1, line width=2.0pt, forget plot]
  table[row sep=crcr]{%
0.1	11.9750343529278\\
0.109090909090909	11.2738432231596\\
0.118181818181818	10.6142808645779\\
0.127272727272727	9.97536866022952\\
0.136363636363636	9.35103239649722\\
0.145454545454545	8.762139207943\\
0.154545454545455	8.20540082327047\\
0.163636363636364	7.600734360567\\
0.172727272727273	6.98916378607578\\
0.181818181818182	7.17382608728201\\
0.190909090909091	7.36624608358307\\
0.2	7.56782937784212\\
0.209090909090909	7.7800160771218\\
0.218181818181818	8.00446978986752\\
0.227272727272727	8.47621555533363\\
0.236363636363636	9.64834579265796\\
0.245454545454545	11.0367934752472\\
0.254545454545455	12.7303020465381\\
0.263636363636364	14.859084149718\\
0.272727272727273	17.6296504218661\\
0.281818181818182	21.3962817899737\\
0.290909090909091	26.8267967166534\\
0.3	35.3522161216636\\
0.309090909090909	50.6977562917487\\
0.318181818181818	86.5629691683263\\
0.327272727272727	267.632680474361\\
0.336363636363636	266.535999377285\\
0.345454545454545	91.4191828901316\\
0.354545454545454	56.0309873958524\\
0.363636363636364	40.8181813194744\\
0.372727272727273	32.3492671431073\\
0.381818181818182	54.0792278259295\\
0.390909090909091	347.564010416952\\
0.4	118.175688966578\\
0.409090909090909	55.6218181271793\\
0.418181818181818	37.9151755905752\\
0.427272727272727	29.4379328947403\\
0.436363636363636	24.4243398567402\\
0.445454545454545	21.0938139340165\\
0.454545454545455	18.711849646602\\
0.463636363636364	16.9189095173904\\
0.472727272727273	15.5176650717097\\
0.481818181818182	15.0165080981167\\
0.490909090909091	15.5819091531722\\
0.5	16.2388540087236\\
0.509090909090909	17.0069670447415\\
0.518181818181818	17.9138266271542\\
0.527272727272727	18.9991288548541\\
0.536363636363636	24.4527535232056\\
0.545454545454546	30.7398401586874\\
0.554545454545455	38.5908030071869\\
0.563636363636364	49.9113218271786\\
0.572727272727273	68.4602733292087\\
0.581818181818182	105.276890737476\\
0.590909090909091	215.378526614275\\
0.6	9.5793512251322e+15\\
0.609090909090909	224.455134474464\\
0.618181818181818	114.422653110313\\
0.627272727272727	77.730729842109\\
0.636363636363636	59.3801553843777\\
0.645454545454545	48.3693826028891\\
0.654545454545455	41.0302259215273\\
0.663636363636364	35.7899980298746\\
0.672727272727273	31.8619828198426\\
0.681818181818182	28.8088882863429\\
0.690909090909091	26.3682027173632\\
0.7	24.3727942173616\\
0.709090909090909	22.7111987976705\\
0.718181818181818	21.3062289479375\\
0.727272727272727	20.1027443626097\\
0.736363636363636	19.0603105284539\\
0.745454545454545	18.1486077246309\\
0.754545454545455	17.3444586196997\\
0.763636363636364	16.6298455111259\\
0.772727272727273	15.9905529516981\\
0.781818181818182	15.4152171188977\\
0.790909090909091	14.8946465108861\\
0.8	14.421327748566\\
0.809090909090909	13.9890602201851\\
0.818181818181818	13.5926820366811\\
0.827272727272727	13.2278617604711\\
0.836363636363636	13.2977674805116\\
0.845454545454545	13.3758706349326\\
0.854545454545454	13.453701921383\\
0.863636363636364	13.5312603700554\\
0.872727272727273	13.6085448662571\\
0.881818181818182	13.6855541836401\\
0.890909090909091	13.7622870126653\\
0.9	13.8387419849741\\
0.909090909090909	13.9149176942526\\
0.918181818181818	13.9908127140796\\
0.927272727272727	14.0664256131869\\
0.936363636363636	14.1417549684934\\
0.945454545454545	14.2167993762249\\
0.954545454545455	14.2915574613881\\
0.963636363636364	14.3660278858246\\
0.972727272727273	14.4402093550443\\
0.981818181818182	14.5141006240047\\
0.990909090909091	14.5877005019811\\
1	14.6610078566527\\
};
\end{axis}

\begin{axis}[%
width=1.66in,
height=1.258in,
at={(1.011in,0.642in)},
scale only axis,
xmin=0,
xmax=1,
xlabel style={font=\color{white!15!black}},
xlabel={$\tau$},
ymin=0,
ymax=1.5e+16,
axis background/.style={fill=white},
title style={font=\bfseries},
title={$\gamma\text{= 0.70}$}
]
\addplot [color=mycolor1, line width=2.0pt, forget plot]
  table[row sep=crcr]{%
0.1	13.2323607675455\\
0.109090909090909	12.5167965061183\\
0.118181818181818	11.8713186564987\\
0.127272727272727	11.2738432231595\\
0.136363636363636	10.7068454330508\\
0.145454545454545	10.1566164694365\\
0.154545454545455	9.61602764320207\\
0.163636363636364	9.09279248399625\\
0.172727272727273	8.60195266002345\\
0.181818181818182	8.12475882847633\\
0.190909090909091	7.600734360567\\
0.2	6.96332359683452\\
0.209090909090909	7.12033926031056\\
0.218181818181818	7.28273817173466\\
0.227272727272727	7.45143634336579\\
0.236363636363636	7.62730475778069\\
0.245454545454545	7.8112872566993\\
0.254545454545455	8.00446978986752\\
0.263636363636364	8.32240713333559\\
0.272727272727273	9.29457420647555\\
0.281818181818182	10.4102559540299\\
0.290909090909091	11.7188678129719\\
0.3	13.2869218042631\\
0.309090909090909	15.209394915667\\
0.318181818181818	17.6296504218661\\
0.327272727272727	20.7770177490051\\
0.336363636363636	25.0444278245394\\
0.345454545454545	31.1675728401512\\
0.354545454545454	40.7044685655523\\
0.363636363636364	57.6318384427595\\
0.372727272727273	96.0188637081097\\
0.381818181818182	267.632680474366\\
0.390909090909091	370.217186057308\\
0.4	112.175341185418\\
0.409090909090909	67.0282836700114\\
0.418181818181818	48.2432801979169\\
0.427272727272727	37.9436614758919\\
0.436363636363636	31.4362041056644\\
0.445454545454545	54.0792278259293\\
0.454545454545455	211.240427771794\\
0.463636363636364	183.228033295831\\
0.472727272727273	71.1010769491534\\
0.481818181818182	46.1025837465565\\
0.490909090909091	34.9443873214844\\
0.5	28.5707382024534\\
0.509090909090909	24.4243398567402\\
0.518181818181818	21.4999939685063\\
0.527272727272727	19.3206859497825\\
0.536363636363636	17.6302680995388\\
0.545454545454546	16.2785608601256\\
0.554545454545455	15.1715215498238\\
0.563636363636364	15.0922590369421\\
0.572727272727273	15.5819091531722\\
0.581818181818182	16.1387281829156\\
0.590909090909091	16.7747370481612\\
0.6	17.5059809290356\\
0.609090909090909	18.3542132454062\\
0.618181818181818	19.349604999788\\
0.627272727272727	25.3285646595182\\
0.636363636363636	30.7398401586874\\
0.645454545454545	37.3101001479193\\
0.654545454545455	46.1663166086571\\
0.663636363636364	59.2108886166065\\
0.672727272727273	80.7533829955726\\
0.681818181818182	123.64391874591\\
0.690909090909091	252.055528587911\\
0.7	1.4421929425013e+16\\
0.709090909090909	261.126206792904\\
0.718181818181818	132.764765718243\\
0.727272727272727	89.9627235544246\\
0.736363636363636	68.5557766378287\\
0.745454545454545	55.7098214805564\\
0.754545454545455	47.1460577747451\\
0.763636363636364	41.0302259215274\\
0.772727272727273	36.4448791638708\\
0.781818181818182	32.8801174137996\\
0.790909090909091	30.0298650842953\\
0.8	27.6992592197799\\
0.809090909090909	25.7583358148264\\
0.818181818181818	24.1170860789942\\
0.827272727272727	22.7111987976704\\
0.836363636363636	21.4935027769649\\
0.845454545454545	20.4286162111528\\
0.854545454545454	19.4894829462735\\
0.863636363636364	18.6550622012807\\
0.872727272727273	17.9087470538765\\
0.881818181818182	17.2372568205004\\
0.890909090909091	16.6298455111258\\
0.9	16.0777258959271\\
0.909090909090909	15.5736436398728\\
0.918181818181818	15.1115577907496\\
0.927272727272727	14.6863978809404\\
0.936363636363636	14.29387704373\\
0.945454545454545	13.9303466408966\\
0.954545454545455	13.5926820366811\\
0.963636363636364	13.278192008584\\
0.972727272727273	13.2754024739136\\
0.981818181818182	13.3424310939054\\
0.990909090909091	13.409260243311\\
1	13.4758893650729\\
};
\end{axis}

\begin{axis}[%
width=1.66in,
height=1.258in,
at={(3.195in,0.642in)},
scale only axis,
xmin=0,
xmax=1,
xlabel style={font=\color{white!15!black}},
xlabel={$\tau$},
ymin=0,
ymax=4.01193073237732e+16,
axis background/.style={fill=white},
title style={font=\bfseries},
title={$\gamma\text{= 0.80}$}
]
\addplot [color=mycolor1, line width=2.0pt, forget plot]
  table[row sep=crcr]{%
0.1	14.3860619841261\\
0.109090909090909	13.6251537005912\\
0.118181818181818	12.954154782507\\
0.127272727272727	12.3498150448764\\
0.136363636363636	11.7944053569472\\
0.145454545454545	11.2738432231596\\
0.154545454545455	10.7765633209917\\
0.163636363636364	10.2931656658611\\
0.172727272727273	9.81755990566879\\
0.181818181818182	9.35103239649721\\
0.190909090909091	8.90487303350517\\
0.2	8.48309260426806\\
0.209090909090909	8.06353933994066\\
0.218181818181818	7.600734360567\\
0.227272727272727	6.95976762990694\\
0.236363636363636	7.080620020176\\
0.245454545454545	7.22114692173905\\
0.254545454545455	7.36624608358306\\
0.263636363636364	7.51649848251567\\
0.272727272727273	7.67250011817417\\
0.281818181818182	7.83490776786238\\
0.290909090909091	8.00446978986754\\
0.3	8.20896118289829\\
0.309090909090909	9.03974057328292\\
0.318181818181818	9.97220835820308\\
0.327272727272727	11.0367934752471\\
0.336363636363636	12.2721596606153\\
0.345454545454545	13.7297449035219\\
0.354545454545454	15.4810373860691\\
0.363636363636364	17.6296504218661\\
0.372727272727273	20.3326518781715\\
0.381818181818182	23.8409629569753\\
0.390909090909091	28.5824854468088\\
0.4	35.3522161216637\\
0.409090909090909	45.8141024355379\\
0.418181818181818	64.1355661195779\\
0.427272727272727	104.535907007721\\
0.436363636363636	267.632680474365\\
0.445454545454545	523.835152531488\\
0.454545454545455	135.475602448838\\
0.463636363636364	78.7960048495231\\
0.472727272727273	56.0309873958523\\
0.481818181818182	43.746445348725\\
0.490909090909091	36.0563580543919\\
0.5	30.7873016931501\\
0.509090909090909	54.0792278259292\\
0.518181818181818	161.75942751365\\
0.527272727272727	323.045524438301\\
0.536363636363636	91.1285003349132\\
0.545454545454546	55.621818127179\\
0.554545454545455	41.0510792011057\\
0.563636363636364	33.0479157852466\\
0.572727272727273	27.9594259287669\\
0.581818181818182	24.4243398567402\\
0.590909090909091	21.8178299640439\\
0.6	19.8120373234866\\
0.609090909090909	18.217992505144\\
0.618181818181818	16.9189095173904\\
0.627272727272727	15.8386125277598\\
0.636363636363636	14.9252324024302\\
0.645454545454545	15.1501260179412\\
0.654545454545455	15.5819091531722\\
0.663636363636364	16.0650749879016\\
0.672727272727273	16.6075151831311\\
0.681818181818182	17.2193728739242\\
0.690909090909091	17.9138266271542\\
0.7	18.7082683946499\\
0.709090909090909	20.7615713522977\\
0.718181818181818	25.9801799460832\\
0.727272727272727	30.7398401586874\\
0.736363636363636	36.3905046091966\\
0.745454545454545	43.6568559300297\\
0.754545454545455	53.639690887343\\
0.763636363636364	68.4602733292083\\
0.772727272727273	93.022079726183\\
0.781818181818182	142.000126667352\\
0.790909090909091	288.728918167813\\
0.8	4.01193073237732e+16\\
0.809090909090909	297.79576338375\\
0.818181818181818	151.10507135321\\
0.827272727272727	102.193306666648\\
0.836363636363636	77.7307298421096\\
0.845454545454545	63.0504619629739\\
0.854545454545454	53.2629497962122\\
0.863636363636364	46.2722785112695\\
0.872727272727273	41.0302259215275\\
0.881818181818182	36.9542651903866\\
0.890909090909091	33.6947555466789\\
0.9	31.0291126087828\\
0.909090909090909	28.808888286343\\
0.918181818181818	26.9312714564352\\
0.927272727272727	25.3227997248792\\
0.936363636363636	23.9295842861593\\
0.945454545454545	22.7111987976705\\
0.954545454545455	21.636723862963\\
0.963636363636364	20.6821090701226\\
0.972727272727273	19.8283673562184\\
0.981818181818182	19.0603105284539\\
0.990909090909091	18.365645668711\\
1	17.7343176815733\\
};
\end{axis}

\begin{axis}[%
width=1.66in,
height=1.258in,
at={(5.379in,0.642in)},
scale only axis,
xmin=0,
xmax=1,
xlabel style={font=\color{white!15!black}},
xlabel={$\tau$},
ymin=0,
ymax=1.5e+16,
axis background/.style={fill=white},
title style={font=\bfseries},
title={$\gamma\text{= 0.90}$}
]
\addplot [color=mycolor1, line width=2.0pt, forget plot]
  table[row sep=crcr]{%
0.1	15.4854868133908\\
0.109090909090909	14.6648419973116\\
0.118181818181818	13.9505876949156\\
0.127272727272727	13.3173702270275\\
0.136363636363636	12.7462690801059\\
0.145454545454545	12.2227050380885\\
0.154545454545455	11.7350746576189\\
0.163636363636364	11.2738432231596\\
0.172727272727273	10.8309796365543\\
0.181818181818182	10.399775134906\\
0.190909090909091	9.9753686602295\\
0.2	9.55670341160063\\
0.209090909090909	9.14945844673654\\
0.218181818181818	8.76213920794303\\
0.227272727272727	8.39088709772725\\
0.236363636363636	8.01537651904851\\
0.245454545454545	7.600734360567\\
0.254545454545455	7.05414007785838\\
0.263636363636364	7.04995065588831\\
0.272727272727273	7.17382608728202\\
0.281818181818182	7.30115557904235\\
0.290909090909091	7.43235426356436\\
0.3	7.56782937784214\\
0.309090909090909	7.70801236730637\\
0.318181818181818	7.85337996622645\\
0.327272727272727	8.00446978986752\\
0.336363636363636	8.16189348254265\\
0.345454545454545	8.84728391061263\\
0.354545454545454	9.64834579265795\\
0.363636363636364	10.5450923202226\\
0.372727272727273	11.562064645902\\
0.381818181818182	12.7303020465381\\
0.390909090909091	14.0905028248878\\
0.4	15.6978432165043\\
0.409090909090909	17.6296504218661\\
0.418181818181818	19.9982521418897\\
0.427272727272727	22.9737185097714\\
0.436363636363636	26.8267967166533\\
0.445454545454545	32.0166816809168\\
0.454545454545455	39.3898440053864\\
0.463636363636364	50.6977562917488\\
0.472727272727273	70.2480554368067\\
0.481818181818182	112.247385853008\\
0.490909090909091	267.63268047436\\
0.5	774.914291129257\\
0.509090909090909	161.819342528126\\
0.518181818181818	91.4191828901319\\
0.527272727272727	64.2098124881314\\
0.536363636363636	49.7709323874801\\
0.545454545454546	40.8181813194744\\
0.554545454545455	34.7221906327225\\
0.563636363636364	30.3024058860552\\
0.572727272727273	54.0792278259293\\
0.581818181818182	136.136980505928\\
0.590909090909091	844.979753400409\\
0.6	118.175688966577\\
0.609090909090909	66.8739028963443\\
0.618181818181818	47.8888656461592\\
0.627272727272727	37.9151755905754\\
0.636363636363636	31.731671974042\\
0.645454545454545	27.5052682520098\\
0.654545454545455	24.4243398567402\\
0.663636363636364	22.0733274683369\\
0.672727272727273	20.2169682686178\\
0.681818181818182	18.7118496466021\\
0.690909090909091	17.4654399752599\\
0.7	16.4152861760891\\
0.709090909090909	15.5176650717097\\
0.718181818181818	14.8458170312097\\
0.727272727272727	15.1957692071583\\
0.736363636363636	15.5819091531722\\
0.745454545454545	16.0086242498268\\
0.754545454545455	16.481372035351\\
0.763636363636364	17.0069670447415\\
0.772727272727273	17.5939851218388\\
0.781818181818182	18.2533408971027\\
0.790909090909091	18.9991288548541\\
0.8	22.2480629844049\\
0.809090909090909	26.4875200611698\\
0.818181818181818	30.7398401586875\\
0.827272727272727	35.6977632095712\\
0.836363636363636	41.8560436730615\\
0.845454545454545	49.9113218271788\\
0.854545454545454	61.0637362130993\\
0.863636363636364	77.6829990024965\\
0.872727272727273	105.276890737477\\
0.881818181818182	160.349828395116\\
0.890909090909091	325.400022133949\\
0.9	1.28792026998949e+16\\
0.909090909090909	334.464246512316\\
0.918181818181818	169.443989811873\\
0.927272727272727	114.422653110316\\
0.936363636363636	86.904862658691\\
0.945454545454545	70.3908292322721\\
0.954545454545455	59.3801553843775\\
0.963636363636364	51.515210656887\\
0.972727272727273	45.6169593394838\\
0.981818181818182	41.0302259215275\\
0.990909090909091	37.3617941386523\\
1	34.3613560367572\\
};
\end{axis}
\end{tikzpicture}%}}
 \hfill 
\subfloat[][$N=20$]
{\resizebox{0.45\textwidth}{!}{ % This file was created by matlab2tikz.
%
%The latest updates can be retrieved from
%  http://www.mathworks.com/matlabcentral/fileexchange/22022-matlab2tikz-matlab2tikz
%where you can also make suggestions and rate matlab2tikz.
%
\definecolor{mycolor1}{rgb}{0.00000,0.44700,0.74100}%
%
\begin{tikzpicture}

\begin{axis}[%
width=1.66in,
height=1.258in,
at={(1.011in,4.137in)},
scale only axis,
xmin=0,
xmax=1,
xlabel style={font=\color{white!15!black}},
xlabel={$\tau$},
ymin=0,
ymax=2230.4394018384,
axis background/.style={fill=white},
title style={font=\bfseries},
title={$\gamma\text{= 0.10}$}
]
\addplot [color=mycolor1, line width=2.0pt, forget plot]
  table[row sep=crcr]{%
0.1	81.5492272704193\\
0.109090909090909	1141.2690179434\\
0.118181818181818	221.590565230114\\
0.127272727272727	163.31804130074\\
0.136363636363636	142.605042206801\\
0.145454545454545	132.854518428382\\
0.154545454545455	127.950072093798\\
0.163636363636364	125.662655743679\\
0.172727272727273	124.963497436937\\
0.181818181818182	125.303989416883\\
0.190909090909091	126.362134339363\\
0.2	127.935756816658\\
0.209090909090909	129.891414898911\\
0.218181818181818	132.13755162171\\
0.227272727272727	134.609350285713\\
0.236363636363636	137.259711933289\\
0.245454545454545	140.053642128249\\
0.254545454545455	142.96463308031\\
0.263636363636364	145.972261695333\\
0.272727272727273	149.060553883394\\
0.281818181818182	152.216845726188\\
0.290909090909091	155.430974882345\\
0.3	158.694696313421\\
0.309090909090909	162.001253368093\\
0.318181818181818	165.345058358341\\
0.327272727272727	177.648246628758\\
0.336363636363636	190.764361568014\\
0.345454545454545	204.263923159998\\
0.354545454545454	218.150530486038\\
0.363636363636364	232.427041626953\\
0.372727272727273	247.095751613099\\
0.381818181818182	262.158521369948\\
0.390909090909091	277.616872861058\\
0.4	293.472060407381\\
0.409090909090909	309.72512489631\\
0.418181818181818	326.376935495993\\
0.427272727272727	343.4282221095\\
0.436363636363636	360.879600875132\\
0.445454545454545	378.731594382279\\
0.454545454545455	396.984647828612\\
0.463636363636364	415.639142029217\\
0.472727272727273	434.695403962626\\
0.481818181818182	454.153715373706\\
0.490909090909091	474.014319832473\\
0.5	494.277428557009\\
0.509090909090909	514.943225240974\\
0.518181818181818	536.011870075295\\
0.527272727272727	557.48350311252\\
0.536363636363636	579.358247093509\\
0.545454545454546	601.636209831619\\
0.554545454545455	624.317486230772\\
0.563636363636364	647.402159999978\\
0.572727272727273	670.890305114628\\
0.581818181818182	694.781987065775\\
0.590909090909091	719.077263931615\\
0.6	743.776187298889\\
0.609090909090909	768.878803057321\\
0.618181818181818	794.385152087209\\
0.627272727272727	820.295270853992\\
0.636363636363636	846.609191926465\\
0.645454545454545	873.326944426735\\
0.654545454545455	900.44855442374\\
0.663636363636364	927.974045277008\\
0.672727272727273	955.90343793766\\
0.681818181818182	984.236751213863\\
0.690909090909091	1012.97400200234\\
0.7	1042.11520549473\\
0.709090909090909	1071.66037535782\\
0.718181818181818	1101.60952389387\\
0.727272727272727	1131.96266218276\\
0.736363636363636	1162.71980020714\\
0.745454545454545	1193.88094696491\\
0.754545454545455	1225.4461105679\\
0.763636363636364	1257.41529833134\\
0.772727272727273	1289.78851685217\\
0.781818181818182	1322.56577207958\\
0.790909090909091	1355.7470693789\\
0.8	1389.33241358756\\
0.809090909090909	1423.32180906589\\
0.818181818181818	1457.71525974247\\
0.827272727272727	1492.51276915576\\
0.836363636363636	1527.71434048944\\
0.845454545454545	1563.31997660739\\
0.854545454545454	1599.32968008187\\
0.863636363636364	1635.74345322168\\
0.872727272727273	1672.56129809561\\
0.881818181818182	1709.78321655438\\
0.890909090909091	1747.40921025171\\
0.9	1785.43928066083\\
0.909090909090909	1823.87342909183\\
0.918181818181818	1862.71165670654\\
0.927272727272727	1901.95396453076\\
0.936363636363636	1941.60035346677\\
0.945454545454545	1981.65082430607\\
0.954545454545455	2022.10537773785\\
0.963636363636364	2062.96401435719\\
0.972727272727273	2104.2267346769\\
0.981818181818182	2145.89353913106\\
0.990909090909091	2187.96442808425\\
1	2230.4394018384\\
};
\end{axis}

\begin{axis}[%
width=1.66in,
height=1.258in,
at={(3.195in,4.137in)},
scale only axis,
xmin=0,
xmax=1,
xlabel style={font=\color{white!15!black}},
xlabel={$\tau$},
ymin=0,
ymax=1141.2690179434,
axis background/.style={fill=white},
title style={font=\bfseries},
title={$\gamma\text{= 0.20}$}
]
\addplot [color=mycolor1, line width=2.0pt, forget plot]
  table[row sep=crcr]{%
0.1	32.2583504285888\\
0.109090909090909	72.6105428448077\\
0.118181818181818	62.2722962127997\\
0.127272727272727	862.189788758504\\
0.136363636363636	158.581414729448\\
0.145454545454545	887.251505614209\\
0.154545454545455	406.482994217252\\
0.163636363636364	189.269278078905\\
0.172727272727273	130.981074427736\\
0.181818181818182	104.332212462666\\
0.190909090909091	89.9370650702523\\
0.2	81.5492272704193\\
0.209090909090909	198.161549898646\\
0.218181818181818	1141.2690179434\\
0.227272727272727	317.903475050265\\
0.236363636363636	221.590565230117\\
0.245454545454545	183.604384313977\\
0.254545454545455	163.318041300737\\
0.263636363636364	150.860008965259\\
0.272727272727273	142.605042206801\\
0.281818181818182	136.894289308621\\
0.290909090909091	132.854518428382\\
0.3	129.978669155324\\
0.309090909090909	127.950072093798\\
0.318181818181818	126.559548382575\\
0.327272727272727	125.662655743679\\
0.336363636363636	125.156026597524\\
0.345454545454545	124.963497436937\\
0.354545454545454	125.027578406982\\
0.363636363636364	125.303989416883\\
0.372727272727273	125.758036347981\\
0.381818181818182	126.362134339363\\
0.390909090909091	127.094070408882\\
0.4	127.935756816658\\
0.409090909090909	128.872318728913\\
0.418181818181818	129.891414898911\\
0.427272727272727	130.98272410198\\
0.436363636363636	132.13755162171\\
0.445454545454545	133.348524089692\\
0.454545454545455	134.609350285713\\
0.463636363636364	135.91463181449\\
0.472727272727273	137.25971193329\\
0.481818181818182	138.640553866615\\
0.490909090909091	140.053642128249\\
0.5	141.495901950813\\
0.509090909090909	142.96463308031\\
0.518181818181818	144.457455050643\\
0.527272727272727	145.972261695333\\
0.536363636363636	147.507183139223\\
0.545454545454546	149.060553883395\\
0.554545454545455	150.630885881448\\
0.563636363636364	152.216845726189\\
0.572727272727273	153.817235238073\\
0.581818181818182	155.430974882346\\
0.590909090909091	157.057089548801\\
0.6	158.69469631342\\
0.609090909090909	160.342993869162\\
0.618181818181818	162.001253368093\\
0.627272727272727	163.668810461215\\
0.636363636363636	165.345058358342\\
0.645454545454545	171.232594659112\\
0.654545454545455	177.648246628758\\
0.663636363636364	184.158633372193\\
0.672727272727273	190.764361568012\\
0.681818181818182	197.465966713445\\
0.690909090909091	204.263923159998\\
0.7	211.158652480194\\
0.709090909090909	218.150530486042\\
0.718181818181818	225.239893150053\\
0.727272727272727	232.427041626953\\
0.736363636363636	239.712246533549\\
0.745454545454545	247.095751613099\\
0.754545454545455	254.577776886023\\
0.763636363636364	262.158521369948\\
0.772727272727273	269.83816543655\\
0.781818181818182	277.616872861058\\
0.790909090909091	285.49479261029\\
0.8	293.472060407374\\
0.809090909090909	301.548800105437\\
0.818181818181818	309.725124896321\\
0.827272727272727	318.001138377498\\
0.836363636363636	326.376935495993\\
0.845454545454545	334.852603385331\\
0.854545454545454	343.4282221095\\
0.863636363636364	352.103865325569\\
0.872727272727273	360.879600875131\\
0.881818181818182	369.755491313256\\
0.890909090909091	378.731594382279\\
0.9	387.807963437346\\
0.909090909090909	396.984647828612\\
0.918181818181818	406.261693245889\\
0.927272727272727	415.639142029227\\
0.936363636363636	425.11703344951\\
0.945454545454545	434.695403962599\\
0.954545454545455	444.374287439235\\
0.963636363636364	454.153715373706\\
0.972727272727273	464.033717073664\\
0.981818181818182	474.014319832487\\
0.990909090909091	484.095549086218\\
1	494.277428557009\\
};
\end{axis}

\begin{axis}[%
width=1.66in,
height=1.258in,
at={(5.379in,4.137in)},
scale only axis,
xmin=0,
xmax=1,
xlabel style={font=\color{white!15!black}},
xlabel={$\tau$},
ymin=0,
ymax=1141.26901794347,
axis background/.style={fill=white},
title style={font=\bfseries},
title={$\gamma\text{= 0.30}$}
]
\addplot [color=mycolor1, line width=2.0pt, forget plot]
  table[row sep=crcr]{%
0.1	26.1126698856499\\
0.109090909090909	43.243461015937\\
0.118181818181818	23.2834620928403\\
0.127272727272727	23.2179466826717\\
0.136363636363636	23.6632357505637\\
0.145454545454545	28.4817137623818\\
0.154545454545455	37.9276977907739\\
0.163636363636364	72.6105428448061\\
0.172727272727273	79.0763417304965\\
0.181818181818182	54.4809372692474\\
0.190909090909091	862.189788758753\\
0.2	112.289954205124\\
0.209090909090909	232.943087444306\\
0.218181818181818	887.25150561417\\
0.227272727272727	728.393817489906\\
0.236363636363636	288.537977540804\\
0.245454545454545	189.269278078906\\
0.254545454545455	145.015767742149\\
0.263636363636364	120.051428120815\\
0.272727272727273	104.332212462666\\
0.281818181818182	93.8645243157317\\
0.290909090909091	86.6532416513453\\
0.3	81.5492272704197\\
0.309090909090909	94.1850975647141\\
0.318181818181818	497.854660517712\\
0.327272727272727	1141.26901794347\\
0.336363636363636	395.075815963063\\
0.345454545454545	272.64311750294\\
0.354545454545454	221.590565230117\\
0.363636363636364	193.42920611457\\
0.372727272727273	175.586266019134\\
0.381818181818182	163.318041300738\\
0.390909090909091	154.426066511053\\
0.4	147.747019740769\\
0.409090909090909	142.605042206801\\
0.418181818181818	138.579111394622\\
0.427272727272727	135.392334656355\\
0.436363636363636	132.854518428383\\
0.445454545454545	130.830281478717\\
0.454545454545455	129.22034972555\\
0.463636363636364	127.950072093797\\
0.472727272727273	126.962093412292\\
0.481818181818182	126.211522913339\\
0.490909090909091	125.662655743681\\
0.5	125.286691555516\\
0.509090909090909	125.060111055525\\
0.518181818181818	124.963497436937\\
0.527272727272727	124.980665263128\\
0.536363636363636	125.098006059376\\
0.545454545454546	125.303989416882\\
0.554545454545455	125.58877753882\\
0.563636363636364	125.943923792974\\
0.572727272727273	126.362134339363\\
0.581818181818182	126.837077723477\\
0.590909090909091	127.363231375731\\
0.6	127.935756816658\\
0.609090909090909	128.550397413524\\
0.618181818181818	129.203394017482\\
0.627272727272727	129.891414898913\\
0.636363636363636	130.611497206559\\
0.645454545454545	131.360997782186\\
0.654545454545455	132.13755162171\\
0.663636363636364	132.939036625193\\
0.672727272727273	133.763543549149\\
0.681818181818182	134.609350285713\\
0.690909090909091	135.474899758718\\
0.7	136.358780857494\\
0.709090909090909	137.259711933289\\
0.718181818181818	138.176526466492\\
0.727272727272727	139.108160579973\\
0.736363636363636	140.053642128249\\
0.745454545454545	141.012081136439\\
0.754545454545455	141.982661399288\\
0.763636363636364	142.964633080311\\
0.772727272727273	143.957306175783\\
0.781818181818182	144.960044728766\\
0.790909090909091	145.972261695333\\
0.8	146.993414379473\\
0.809090909090909	148.023000365069\\
0.818181818181818	149.060553883394\\
0.827272727272727	150.105642563149\\
0.836363636363636	151.157864517173\\
0.845454545454545	152.216845726188\\
0.854545454545454	153.282237685081\\
0.863636363636364	154.353715281749\\
0.872727272727273	155.430974882345\\
0.881818181818182	156.513732600015\\
0.890909090909091	157.601722727138\\
0.9	158.69469631342\\
0.909090909090909	159.792419874404\\
0.918181818181818	160.894674216725\\
0.927272727272727	162.001253368092\\
0.936363636363636	163.111963601326\\
0.945454545454545	164.226622542996\\
0.954545454545455	165.345058358342\\
0.963636363636364	169.114979585584\\
0.972727272727273	173.360659957782\\
0.981818181818182	177.648246628759\\
0.990909090909091	181.977939351291\\
1	186.349921471459\\
};
\end{axis}

\begin{axis}[%
width=1.66in,
height=1.258in,
at={(1.011in,2.39in)},
scale only axis,
xmin=0,
xmax=1,
xlabel style={font=\color{white!15!black}},
xlabel={$\tau$},
ymin=0,
ymax=1500,
axis background/.style={fill=white},
title style={font=\bfseries},
title={$\gamma\text{= 0.40}$}
]
\addplot [color=mycolor1, line width=2.0pt, forget plot]
  table[row sep=crcr]{%
0.1	23.148004772981\\
0.109090909090909	23.1407034662428\\
0.118181818181818	23.1632078109845\\
0.127272727272727	23.2011172099712\\
0.136363636363636	33.2470441775758\\
0.145454545454545	43.2434610159371\\
0.154545454545455	23.2867519898134\\
0.163636363636364	23.262640262836\\
0.172727272727273	23.1844880486668\\
0.181818181818182	23.6632357505636\\
0.190909090909091	27.0159900219206\\
0.2	32.2583504285888\\
0.209090909090909	42.0611051754928\\
0.218181818181818	72.6105428448077\\
0.227272727272727	99.8303979743027\\
0.236363636363636	62.2722962128002\\
0.245454545454545	60.1937833790973\\
0.254545454545455	862.189788758596\\
0.263636363636364	109.625102166665\\
0.272727272727273	158.581414729448\\
0.281818181818182	292.828309634637\\
0.290909090909091	887.251505614209\\
0.3	1260.85288067823\\
0.309090909090909	406.482994217252\\
0.318181818181818	253.652803656004\\
0.327272727272727	189.269278078905\\
0.336363636363636	153.612439216195\\
0.345454545454545	130.981074427736\\
0.354545454545454	115.464982108761\\
0.363636363636364	104.332212462666\\
0.372727272727273	96.1174252175632\\
0.381818181818182	89.9370650702523\\
0.390909090909091	85.213025958552\\
0.4	81.5492272704193\\
0.409090909090909	87.3345672342251\\
0.418181818181818	198.161549898646\\
0.427272727272727	930.189748783358\\
0.436363636363636	1141.2690179434\\
0.445454545454545	458.725185193125\\
0.454545454545455	317.903475050277\\
0.463636363636364	256.335351841131\\
0.472727272727273	221.590565230114\\
0.481818181818182	199.217348361864\\
0.490909090909091	183.604384313978\\
0.5	172.108699583369\\
0.509090909090909	163.31804130074\\
0.518181818181818	156.407084284478\\
0.527272727272727	150.860008965259\\
0.536363636363636	146.337106231861\\
0.545454545454546	142.605042206801\\
0.554545454545455	139.497911921315\\
0.563636363636364	136.894289308622\\
0.572727272727273	134.703088181008\\
0.581818181818182	132.854518428384\\
0.590909090909091	131.294112586711\\
0.6	129.978669155323\\
0.609090909090909	128.87342974643\\
0.618181818181818	127.950072093798\\
0.627272727272727	127.185255504329\\
0.636363636363636	126.559548382575\\
0.645454545454545	126.056625062359\\
0.654545454545455	125.662655743679\\
0.663636363636364	125.365837069054\\
0.672727272727273	125.156026597525\\
0.681818181818182	125.024455044622\\
0.690909090909091	124.963497436937\\
0.7	124.966489402975\\
0.709090909090909	125.027578406981\\
0.718181818181818	125.141602299296\\
0.727272727272727	125.303989416883\\
0.736363636363636	125.510675830928\\
0.745454545454545	125.758036347981\\
0.754545454545455	126.042826626328\\
0.763636363636364	126.362134339363\\
0.772727272727273	126.713337751903\\
0.781818181818182	127.094070408882\\
0.790909090909091	127.502190893929\\
0.8	127.935756816658\\
0.809090909090909	128.393002345613\\
0.818181818181818	128.872318728914\\
0.827272727272727	129.372237344149\\
0.836363636363636	129.891414898911\\
0.845454545454545	130.428620467563\\
0.854545454545454	130.98272410198\\
0.863636363636364	131.552686796457\\
0.872727272727273	132.137551621711\\
0.881818181818182	132.736435871579\\
0.890909090909091	133.348524089692\\
0.9	133.973061863026\\
0.909090909090909	134.609350285713\\
0.918181818181818	135.256741010207\\
0.927272727272727	135.914631814489\\
0.936363636363636	136.582462623754\\
0.945454545454545	137.259711933289\\
0.954545454545455	137.945893586331\\
0.963636363636364	138.640553866615\\
0.972727272727273	139.343268870543\\
0.981818181818182	140.053642128249\\
0.990909090909091	140.771302446643\\
1	141.495901950813\\
};
\end{axis}

\begin{axis}[%
width=1.66in,
height=1.258in,
at={(3.195in,2.39in)},
scale only axis,
xmin=0,
xmax=1,
xlabel style={font=\color{white!15!black}},
xlabel={$\tau$},
ymin=0,
ymax=2314.18957431426,
axis background/.style={fill=white},
title style={font=\bfseries},
title={$\gamma\text{= 0.50}$}
]
\addplot [color=mycolor1, line width=2.0pt, forget plot]
  table[row sep=crcr]{%
0.1	23.3671485372853\\
0.109090909090909	23.2436839428583\\
0.118181818181818	23.1739625776333\\
0.127272727272727	23.1433655147415\\
0.136363636363636	23.1407034662428\\
0.145454545454545	23.1570491632072\\
0.154545454545455	23.1849370798156\\
0.163636363636364	23.2176924327743\\
0.172727272727273	101.958295305732\\
0.181818181818182	43.2434610159373\\
0.190909090909091	23.416742770291\\
0.2	23.2776060639124\\
0.209090909090909	23.2391073324549\\
0.218181818181818	23.1602674827227\\
0.227272727272727	23.6632357505636\\
0.236363636363636	26.2349380819961\\
0.245454545454545	29.8344656958662\\
0.254545454545455	35.3416952697005\\
0.263636363636364	45.2493775268986\\
0.272727272727273	72.6105428448069\\
0.281818181818182	129.367530307007\\
0.290909090909091	70.3670960415603\\
0.3	57.0644275891226\\
0.309090909090909	67.4779226352127\\
0.318181818181818	862.189788758527\\
0.327272727272727	124.38201588223\\
0.336363636363636	128.858603419669\\
0.345454545454545	197.987775982403\\
0.354545454545454	342.502289346321\\
0.363636363636364	887.251505614194\\
0.372727272727273	2314.18957431426\\
0.381818181818182	549.863157494924\\
0.390909090909091	325.486412262815\\
0.4	236.919510031585\\
0.409090909090909	189.269278078904\\
0.418181818181818	159.41380718352\\
0.427272727272727	138.954470640889\\
0.436363636363636	124.115241058399\\
0.445454545454545	112.94665234062\\
0.454545454545455	104.332212462664\\
0.463636363636364	97.5748601837626\\
0.472727272727273	92.2070786251809\\
0.481818181818182	87.8980987583776\\
0.490909090909091	84.4057440160981\\
0.5	81.5492272704189\\
0.509090909090909	83.8420419651731\\
0.518181818181818	100.906689474885\\
0.527272727272727	338.753926577712\\
0.536363636363636	1658.82473134443\\
0.545454545454546	1141.26901794336\\
0.554545454545455	512.244909440299\\
0.563636363636364	358.456781571704\\
0.572727272727273	288.311517719794\\
0.581818181818182	247.926034199969\\
0.590909090909091	221.590565230114\\
0.6	203.031976281947\\
0.609090909090909	189.245814606581\\
0.618181818181818	178.609175504501\\
0.627272727272727	170.166635298186\\
0.636363636363636	163.318041300739\\
0.645454545454545	157.666787471942\\
0.654545454545455	152.939855833386\\
0.663636363636364	148.942878941014\\
0.672727272727273	145.533526406073\\
0.681818181818182	142.6050422068\\
0.690909090909091	140.075679548371\\
0.7	137.881703494966\\
0.709090909090909	135.97262826307\\
0.718181818181818	134.307897099735\\
0.727272727272727	132.854518428382\\
0.736363636363636	131.585350926482\\
0.745454545454545	130.477838271329\\
0.754545454545455	129.513061370244\\
0.763636363636364	128.675018568892\\
0.772727272727273	127.950072093797\\
0.781818181818182	127.326517412255\\
0.790909090909091	126.794244649764\\
0.8	126.344469768103\\
0.809090909090909	125.96951918447\\
0.818181818181818	125.662655743681\\
0.827272727272727	125.417936990435\\
0.836363636363636	125.230098891866\\
0.845454545454545	125.094459777793\\
0.854545454545454	125.006840465829\\
0.863636363636364	124.963497436936\\
0.872727272727273	124.961066606137\\
0.881818181818182	124.996515750742\\
0.890909090909091	125.067104056258\\
0.9	125.17034754803\\
0.909090909090909	125.303989416884\\
0.918181818181818	125.465974435471\\
0.927272727272727	125.654426811045\\
0.936363636363636	125.867630938707\\
0.945454545454545	126.104014613953\\
0.954545454545455	126.362134339363\\
0.963636363636364	126.640662421916\\
0.972727272727273	126.938375607351\\
0.981818181818182	127.254145038866\\
0.990909090909091	127.586927360937\\
1	127.935756816657\\
};
\end{axis}

\begin{axis}[%
width=1.66in,
height=1.258in,
at={(5.379in,2.39in)},
scale only axis,
xmin=0,
xmax=1,
xlabel style={font=\color{white!15!black}},
xlabel={$\tau$},
ymin=0,
ymax=6000,
axis background/.style={fill=white},
title style={font=\bfseries},
title={$\gamma\text{= 0.60}$}
]
\addplot [color=mycolor1, line width=2.0pt, forget plot]
  table[row sep=crcr]{%
0.1	23.8046514758995\\
0.109090909090909	23.564230393814\\
0.118181818181818	23.3942463582088\\
0.127272727272727	23.2780164328327\\
0.136363636363636	23.203116661083\\
0.145454545454545	23.1600262543622\\
0.154545454545455	23.1412460228378\\
0.163636363636364	23.1407034662427\\
0.172727272727273	23.1533352197182\\
0.181818181818182	23.1747787045098\\
0.190909090909091	23.2011172099712\\
0.2	26.11266988565\\
0.209090909090909	794.121053068869\\
0.218181818181818	43.2434610159369\\
0.227272727272727	25.1395812443188\\
0.236363636363636	23.2834620928403\\
0.245454545454545	23.262640262836\\
0.254545454545455	23.2179466826717\\
0.263636363636364	23.1422374419793\\
0.272727272727273	23.6632357505637\\
0.281818181818182	25.7492842499839\\
0.290909090909091	28.4817137623818\\
0.3	32.2583504285888\\
0.309090909090909	37.927697790774\\
0.318181818181818	47.8011708040342\\
0.327272727272727	72.6105428448066\\
0.336363636363636	181.817404098692\\
0.345454545454545	79.0763417304962\\
0.354545454545454	62.2722962127997\\
0.363636363636364	54.4809372692475\\
0.372727272727273	75.0903705652518\\
0.381818181818182	862.189788758813\\
0.390909090909091	138.296993419136\\
0.4	112.289954205123\\
0.409090909090909	158.581414729449\\
0.418181818181818	232.943087444305\\
0.427272727272727	384.452308838984\\
0.436363636363636	887.251505614257\\
0.445454545454545	5389.23029216818\\
0.454545454545455	728.393817489947\\
0.463636363636364	406.482994217242\\
0.472727272727273	288.537977540805\\
0.481818181818182	227.092398946401\\
0.490909090909091	189.269278078903\\
0.5	163.591853193281\\
0.509090909090909	145.01576774215\\
0.518181818181818	130.981074427736\\
0.527272727272727	120.051428120816\\
0.536363636363636	111.356181086364\\
0.545454545454546	104.332212462665\\
0.554545454545455	98.5940006871756\\
0.563636363636364	93.864524315731\\
0.572727272727273	89.9370650702521\\
0.581818181818182	86.6532416513448\\
0.590909090909091	83.889663214202\\
0.6	81.5492272704192\\
0.609090909090909	81.7197348459351\\
0.618181818181818	94.1850975647131\\
0.627272727272727	198.161549898639\\
0.636363636363636	497.854660517696\\
0.645454545454545	3166.15450201931\\
0.654545454545455	1141.26901794352\\
0.663636363636364	557.920870502237\\
0.672727272727273	395.075815963065\\
0.681818181818182	317.903475050268\\
0.690909090909091	272.643117502933\\
0.7	242.794740687625\\
0.709090909090909	221.590565230114\\
0.718181818181818	205.735316324098\\
0.727272727272727	193.429206114574\\
0.736363636363636	183.604384313977\\
0.745454545454545	175.586266019133\\
0.754545454545455	168.927190410684\\
0.763636363636364	163.318041300737\\
0.772727272727273	158.53825628591\\
0.781818181818182	154.426066511053\\
0.790909090909091	150.860008965261\\
0.8	147.747019740771\\
0.809090909090909	145.014524652298\\
0.818181818181818	142.605042206801\\
0.827272727272727	140.472413208607\\
0.836363636363636	138.579111394624\\
0.845454545454545	136.894289308621\\
0.854545454545454	135.392334656354\\
0.863636363636364	134.051787716631\\
0.872727272727273	132.854518428382\\
0.881818181818182	131.785093095268\\
0.890909090909091	130.830281478717\\
0.9	129.978669155323\\
0.909090909090909	129.22034972555\\
0.918181818181818	128.54667824902\\
0.927272727272727	127.950072093798\\
0.936363636363636	127.423848842828\\
0.945454545454545	126.962093412292\\
0.954545454545455	126.559548382575\\
0.963636363636364	126.211522913338\\
0.972727272727273	125.913816642116\\
0.981818181818182	125.662655743679\\
0.990909090909091	125.454638921031\\
1	125.286691555516\\
};
\end{axis}

\begin{axis}[%
width=1.66in,
height=1.258in,
at={(1.011in,0.642in)},
scale only axis,
xmin=0,
xmax=1,
xlabel style={font=\color{white!15!black}},
xlabel={$\tau$},
ymin=0,
ymax=156121.198056444,
axis background/.style={fill=white},
title style={font=\bfseries},
title={$\gamma\text{= 0.70}$}
]
\addplot [color=mycolor1, line width=2.0pt, forget plot]
  table[row sep=crcr]{%
0.1	24.3836445154385\\
0.109090909090909	24.0319594631874\\
0.118181818181818	23.7652049207876\\
0.127272727272727	23.5642303938138\\
0.136363636363636	23.4149035598869\\
0.145454545454545	23.3065280410104\\
0.154545454545455	23.2308208105395\\
0.163636363636364	23.1812302260183\\
0.172727272727273	23.1524680273634\\
0.181818181818182	23.1401793278556\\
0.190909090909091	23.1407034662428\\
0.2	23.150895409941\\
0.209090909090909	23.1679870100202\\
0.218181818181818	23.1894706172285\\
0.227272727272727	23.2129685222368\\
0.236363636363636	29.6050883951335\\
0.245454545454545	296.73865219066\\
0.254545454545455	43.2434610159374\\
0.263636363636364	26.5811610911853\\
0.272727272727273	23.2858548210341\\
0.281818181818182	23.2741267202697\\
0.290909090909091	23.2467865589003\\
0.3	23.1998383163097\\
0.309090909090909	23.1283833856749\\
0.318181818181818	23.6632357505637\\
0.327272727272727	25.4180034996511\\
0.336363636363636	27.616959785665\\
0.345454545454545	30.4733165808369\\
0.354545454545454	34.3769118375903\\
0.363636363636364	40.1398321953614\\
0.372727272727273	49.8983124105711\\
0.381818181818182	72.6105428448072\\
0.390909090909091	345.591342976819\\
0.4	88.7456427381642\\
0.409090909090909	67.6159406167013\\
0.418181818181818	58.3588439608478\\
0.427272727272727	56.4943876345258\\
0.436363636363636	82.9167005105371\\
0.445454545454545	862.189788758578\\
0.454545454545455	151.510086695384\\
0.463636363636364	101.866473086858\\
0.472727272727273	136.636237389507\\
0.481818181818182	185.410705105187\\
0.490909090909091	264.364192410044\\
0.5	420.375153679922\\
0.509090909090909	887.251505614279\\
0.518181818181818	156121.198056444\\
0.527272727272727	957.110214197803\\
0.536363636363636	498.7251347072\\
0.545454545454546	344.80285068065\\
0.554545454545455	267.372698790854\\
0.563636363636364	220.626530535162\\
0.572727272727273	189.269278078902\\
0.581818181818182	166.744143744992\\
0.590909090909091	149.775980115771\\
0.6	136.549087386415\\
0.609090909090909	125.976721220796\\
0.618181818181818	117.368021930649\\
0.627272727272727	110.260934016838\\
0.636363636363636	104.332212462666\\
0.645454545454545	99.3464173522055\\
0.654545454545455	95.1259755047274\\
0.663636363636364	91.5331355355665\\
0.672727272727273	88.4587831564861\\
0.681818181818182	85.8152409029808\\
0.690909090909091	83.5313972212079\\
0.7	81.5492272704194\\
0.709090909090909	80.2922786840395\\
0.718181818181818	90.0938937890593\\
0.727272727272727	104.234911266303\\
0.736363636363636	290.826545298676\\
0.745454545454545	690.019491567218\\
0.754545454545455	8161.00976477523\\
0.763636363636364	1141.26901794325\\
0.772727272727273	597.379560723712\\
0.781818181818182	428.345054538044\\
0.790909090909091	345.409589987989\\
0.8	295.920130520819\\
0.809090909090909	262.93829357025\\
0.818181818181818	239.337190457451\\
0.827272727272727	221.590565230114\\
0.836363636363636	207.751224078865\\
0.845454545454545	196.654691966896\\
0.854545454545454	187.560999827269\\
0.863636363636364	179.976889414305\\
0.872727272727273	173.560577003706\\
0.881818181818182	168.067534363202\\
0.890909090909091	163.318041300739\\
0.9	159.176940884633\\
0.909090909090909	155.540553588092\\
0.918181818181818	152.327957840859\\
0.927272727272727	149.4750245581\\
0.936363636363636	146.930240163108\\
0.945454545454545	144.651721343035\\
0.954545454545455	142.605042206801\\
0.963636363636364	140.761626646211\\
0.972727272727273	139.097541179404\\
0.981818181818182	137.592576291117\\
0.990909090909091	136.229538741743\\
1	134.993700278165\\
};
\end{axis}

\begin{axis}[%
width=1.66in,
height=1.258in,
at={(3.195in,0.642in)},
scale only axis,
xmin=0,
xmax=1,
xlabel style={font=\color{white!15!black}},
xlabel={$\tau$},
ymin=0,
ymax=60043.9585267769,
axis background/.style={fill=white},
title style={font=\bfseries},
title={$\gamma\text{= 0.80}$}
]
\addplot [color=mycolor1, line width=2.0pt, forget plot]
  table[row sep=crcr]{%
0.1	25.0568400486565\\
0.109090909090909	24.5992870200858\\
0.118181818181818	24.2403178776577\\
0.127272727272727	23.9581618617134\\
0.136363636363636	23.7368159376971\\
0.145454545454545	23.5642303938138\\
0.154545454545455	23.4311370989818\\
0.163636363636364	23.3302696346326\\
0.172727272727273	23.2558302447259\\
0.181818181818182	23.2031166610831\\
0.190909090909091	23.1682550344861\\
0.2	23.148004772981\\
0.209090909090909	23.139612967919\\
0.218181818181818	23.1407034662428\\
0.227272727272727	23.1491902799246\\
0.236363636363636	23.1632078109845\\
0.245454545454545	23.1810513387068\\
0.254545454545455	23.2011172099712\\
0.263636363636364	23.9045668106725\\
0.272727272727273	33.2470441775758\\
0.281818181818182	155.544554149688\\
0.290909090909091	43.2434610159371\\
0.3	27.8066200533839\\
0.309090909090909	23.2867519898134\\
0.318181818181818	23.2801622886875\\
0.327272727272727	23.262640262836\\
0.336363636363636	23.2317189260471\\
0.345454545454545	23.1844880486668\\
0.354545454545454	23.1174378824567\\
0.363636363636364	23.6632357505636\\
0.372727272727273	25.1775551478246\\
0.381818181818182	27.0159900219206\\
0.390909090909091	29.3058783351856\\
0.4	32.2583504285888\\
0.409090909090909	36.25177640159\\
0.418181818181818	42.0611051754928\\
0.427272727272727	51.6568468219745\\
0.436363636363636	72.6105428448077\\
0.445454545454545	328.198614141492\\
0.454545454545455	99.8303979743022\\
0.463636363636364	73.1862290432155\\
0.472727272727273	62.2722962127997\\
0.481818181818182	56.0306754004445\\
0.490909090909091	60.1937833790978\\
0.5	91.7571201672855\\
0.509090909090909	862.189788758504\\
0.518181818181818	164.116940320295\\
0.527272727272727	109.625102166665\\
0.536363636363636	122.418742612324\\
0.545454545454546	158.581414729451\\
0.554545454545455	210.07776871046\\
0.563636363636364	292.828309634639\\
0.572727272727273	451.492659848809\\
0.581818181818182	887.251505614244\\
0.590909090909091	7711.49337373424\\
0.6	1260.85288067826\\
0.609090909090909	604.865025725313\\
0.618181818181818	406.482994217252\\
0.627272727272727	310.445698543487\\
0.636363636363636	253.652803656014\\
0.645454545454545	216.048977601814\\
0.654545454545455	189.269278078905\\
0.663636363636364	169.207089117894\\
0.672727272727273	153.612439216194\\
0.681818181818182	141.150769308971\\
0.690909090909091	130.981074427736\\
0.7	122.546981737403\\
0.709090909090909	115.46498210876\\
0.718181818181818	109.46092124096\\
0.727272727272727	104.332212462666\\
0.736363636363636	99.9245847088928\\
0.745454545454545	96.1174252175632\\
0.754545454545455	92.8143587222431\\
0.763636363636364	89.9370650702523\\
0.772727272727273	87.4211092586394\\
0.781818181818182	85.213025958552\\
0.790909090909091	83.2681947616617\\
0.8	81.5492272704195\\
0.809090909090909	80.0247003890663\\
0.818181818181818	87.3345672342241\\
0.827272727272727	98.230135978894\\
0.836363636363636	198.161549898646\\
0.845454545454545	388.896591469046\\
0.854545454545454	930.189748783358\\
0.863636363636364	60043.9585267769\\
0.872727272727273	1141.26901794334\\
0.881818181818182	631.819150456914\\
0.890909090909091	458.725185193125\\
0.9	371.068321252215\\
0.909090909090909	317.903475050277\\
0.918181818181818	282.117892802799\\
0.927272727272727	256.335351841127\\
0.936363636363636	236.849223407905\\
0.945454545454545	221.590565230117\\
0.954545454545455	209.312328877192\\
0.963636363636364	199.217348361864\\
0.972727272727273	190.771783627745\\
0.981818181818182	183.604384313977\\
0.990909090909091	177.448768959493\\
1	172.108699583369\\
};
\end{axis}

\begin{axis}[%
width=1.66in,
height=1.258in,
at={(5.379in,0.642in)},
scale only axis,
xmin=0,
xmax=1,
xlabel style={font=\color{white!15!black}},
xlabel={$\tau$},
ymin=0,
ymax=10000,
axis background/.style={fill=white},
title style={font=\bfseries},
title={$\gamma\text{= 0.90}$}
]
\addplot [color=mycolor1, line width=2.0pt, forget plot]
  table[row sep=crcr]{%
0.1	25.7953708425615\\
0.109090909090909	25.23619915782\\
0.118181818181818	24.7889065010385\\
0.127272727272727	24.4290293493337\\
0.136363636363636	24.1386191092679\\
0.145454545454545	23.9041950462503\\
0.154545454545455	23.7154219938995\\
0.163636363636364	23.5642303938139\\
0.172727272727273	23.4442152374319\\
0.181818181818182	23.3502159245344\\
0.190909090909091	23.2780164328326\\
0.2	23.2241272674333\\
0.209090909090909	23.1856240834652\\
0.218181818181818	23.1600262543622\\
0.227272727272727	23.1452040097343\\
0.236363636363636	23.1393062522669\\
0.245454545454545	23.1407034662428\\
0.254545454545455	23.1479416524801\\
0.263636363636364	23.1597041710288\\
0.272727272727273	23.1747787045098\\
0.281818181818182	23.1920254198051\\
0.290909090909091	23.2103335415205\\
0.3	26.1126698856501\\
0.309090909090909	48.195799023496\\
0.318181818181818	115.815824379665\\
0.327272727272727	43.2434610159375\\
0.336363636363636	28.8619765813014\\
0.345454545454545	23.2869522780821\\
0.354545454545454	23.2834620928403\\
0.363636363636364	23.2719266479093\\
0.372727272727273	23.2507128925659\\
0.381818181818182	23.2179466826716\\
0.390909090909091	23.1714385414335\\
0.4	23.1085860188534\\
0.409090909090909	23.6632357505636\\
0.418181818181818	24.9950748244024\\
0.427272727272727	26.5738907395154\\
0.436363636363636	28.4817137623819\\
0.445454545454545	30.8455658846248\\
0.454545454545455	33.8721794286567\\
0.463636363636364	37.9276977907739\\
0.472727272727273	43.7500195747959\\
0.481818181818182	53.1551715224586\\
0.490909090909091	72.610542844807\\
0.5	185.198730106\\
0.509090909090909	113.004411873849\\
0.518181818181818	79.0763417304961\\
0.527272727272727	66.2622957854725\\
0.536363636363636	59.1373127368005\\
0.545454545454546	54.4809372692475\\
0.554545454545455	63.5686045457378\\
0.563636363636364	102.561634193828\\
0.572727272727273	862.189788758544\\
0.581818181818182	176.187124863976\\
0.590909090909091	117.120105357263\\
0.6	112.289954205123\\
0.609090909090909	141.17877041696\\
0.618181818181818	178.924793211323\\
0.627272727272727	232.943087444307\\
0.636363636363636	318.76214172166\\
0.645454545454545	478.713146457374\\
0.654545454545455	887.251505614169\\
0.663636363636364	4227.17869205608\\
0.672727272727273	1683.96321921014\\
0.681818181818182	728.393817489942\\
0.690909090909091	474.477229180459\\
0.7	356.679051998687\\
0.709090909090909	288.53797754081\\
0.718181818181818	244.03561336842\\
0.727272727272727	212.637827376115\\
0.736363636363636	189.269278078904\\
0.745454545454545	171.184511063899\\
0.754545454545455	156.769894273067\\
0.763636363636364	145.015767742148\\
0.772727272727273	135.258545568351\\
0.781818181818182	127.044229662191\\
0.790909090909091	120.051428120816\\
0.8	114.045646496949\\
0.809090909090909	108.851011114467\\
0.818181818181818	104.332212462665\\
0.827272727272727	100.382699185796\\
0.836363636363636	96.9168215094497\\
0.845454545454545	93.8645243157307\\
0.854545454545454	91.1677043474067\\
0.863636363636364	88.7776575433696\\
0.872727272727273	86.6532416513455\\
0.881818181818182	84.7595108627809\\
0.890909090909091	83.0666667638881\\
0.9	81.5492272704195\\
0.909090909090909	80.1853517870482\\
0.918181818181818	85.345251421235\\
0.927272727272727	94.1850975647128\\
0.936363636363636	135.168020043746\\
0.945454545454545	267.485951243023\\
0.954545454545455	497.854660517673\\
0.963636363636364	1240.68019986921\\
0.972727272727273	8216.2281142374\\
0.981818181818182	1141.26901794336\\
0.990909090909091	662.145115802177\\
1	486.589230789707\\
};
\end{axis}
\end{tikzpicture}%}}  \\
\subfloat[][$N=30$]
{\resizebox{0.45\textwidth}{!}{% This file was created by matlab2tikz.
%
%The latest updates can be retrieved from
%  http://www.mathworks.com/matlabcentral/fileexchange/22022-matlab2tikz-matlab2tikz
%where you can also make suggestions and rate matlab2tikz.
%
\definecolor{mycolor1}{rgb}{0.00000,0.44700,0.74100}%
%
\begin{tikzpicture}

\begin{axis}[%
width=1.66in,
height=1.258in,
at={(1.011in,4.137in)},
scale only axis,
xmin=0,
xmax=1,
xlabel style={font=\color{white!15!black}},
xlabel={$\tau$},
ymin=0,
ymax=1500000,
axis background/.style={fill=white},
title style={font=\bfseries},
title={$\gamma\text{= 0.10}$}
]
\addplot [color=mycolor1, line width=2.0pt, forget plot]
  table[row sep=crcr]{%
0.1	476.21817390513\\
0.109090909090909	259.625254240438\\
0.118181818181818	204.71885493333\\
0.127272727272727	180.720352693281\\
0.136363636363636	168.141502912989\\
0.145454545454545	168.272103286178\\
0.154545454545455	189.420694351708\\
0.163636363636364	210.914932463867\\
0.172727272727273	232.80625564236\\
0.181818181818182	255.117909513451\\
0.190909090909091	277.858788664657\\
0.2	301.030274119056\\
0.209090909090909	324.629900562307\\
0.218181818181818	348.653465223835\\
0.227272727272727	373.096325021531\\
0.236363636363636	397.954256512516\\
0.245454545454545	423.224080641504\\
0.254545454545455	448.904169746645\\
0.263636363636364	474.994911644857\\
0.272727272727273	501.499184487124\\
0.281818181818182	528.422887028681\\
0.290909090909091	555.775567617488\\
0.3	583.571199705397\\
0.309090909090909	611.829161566368\\
0.318181818181818	640.575493611051\\
0.327272727272727	669.844529305668\\
0.336363636363636	699.68102678754\\
0.345454545454545	730.142969430085\\
0.354545454545454	761.305255636664\\
0.363636363636364	793.264558860474\\
0.372727272727273	826.145698563942\\
0.381818181818182	860.109895178567\\
0.390909090909091	895.365227731029\\
0.4	932.179354932365\\
0.409090909090909	970.893903127533\\
0.418181818181818	1011.93861654158\\
0.427272727272727	1055.84128146418\\
0.436363636363636	1103.22706590918\\
0.445454545454545	1154.80007118955\\
0.454545454545455	1211.30363698247\\
0.463636363636364	1273.46589901497\\
0.472727272727273	1341.94872260807\\
0.481818181818182	1417.32105086742\\
0.490909090909091	1500.0664733524\\
0.5	1590.61786426945\\
0.509090909090909	1689.40283222049\\
0.518181818181818	1796.88625655212\\
0.527272727272727	1913.60397099829\\
0.536363636363636	2040.18795134867\\
0.545454545454546	2177.38614391773\\
0.554545454545455	2326.08040960931\\
0.563636363636364	2487.30546149472\\
0.572727272727273	2662.2710280766\\
0.581818181818182	2852.38909985652\\
0.590909090909091	3059.30806125492\\
0.6	3284.95574649093\\
0.609090909090909	3531.59397187521\\
0.618181818181818	3801.88791548511\\
0.627272727272727	4098.99491970506\\
0.636363636363636	4426.67903433019\\
0.645454545454545	4789.4601486494\\
0.654545454545455	5192.81028447907\\
0.663636363636364	5643.41519350666\\
0.672727272727273	6149.52789994758\\
0.681818181818182	6721.45406888078\\
0.690909090909091	7372.23019901468\\
0.7	8118.59021822855\\
0.709090909090909	8982.37436406352\\
0.718181818181818	9992.63584065215\\
0.727272727272727	11188.884614381\\
0.736363636363636	12626.2551578509\\
0.745454545454545	14384.0754044719\\
0.754545454545455	16580.7702369606\\
0.763636363636364	19401.330321873\\
0.772727272727273	23151.7266073646\\
0.781818181818182	28377.1353780645\\
0.790909090909091	36152.4817856725\\
0.8	48932.3468415752\\
0.809090909090909	73810.2388637532\\
0.818181818181818	143292.837472571\\
0.827272727272727	1405907.45642716\\
0.836363636363636	189770.318318841\\
0.845454545454545	91088.8595459665\\
0.854545454545454	60869.028601483\\
0.863636363636364	46224.1463875154\\
0.872727272727273	37586.2848661259\\
0.881818181818182	31892.4494096094\\
0.890909090909091	27859.5535830685\\
0.9	24855.5358177361\\
0.909090909090909	22532.935368635\\
0.918181818181818	20684.9144034277\\
0.927272727272727	19180.6542103089\\
0.936363636363636	17933.370001577\\
0.945454545454545	16883.221617272\\
0.954545454545455	15987.6096772422\\
0.963636363636364	15215.3836578236\\
0.972727272727273	14543.2378729569\\
0.981818181818182	13953.3882402362\\
0.990909090909091	13432.0285043434\\
1	12968.276976988\\
};
\end{axis}

\begin{axis}[%
width=1.66in,
height=1.258in,
at={(3.195in,4.137in)},
scale only axis,
xmin=0,
xmax=1,
xlabel style={font=\color{white!15!black}},
xlabel={$\tau$},
ymin=0,
ymax=1590.61786426945,
axis background/.style={fill=white},
title style={font=\bfseries},
title={$\gamma\text{= 0.20}$}
]
\addplot [color=mycolor1, line width=2.0pt, forget plot]
  table[row sep=crcr]{%
0.1	89.9601100726513\\
0.109090909090909	163.903090660577\\
0.118181818181818	96.6066960899387\\
0.127272727272727	76.4516757440059\\
0.136363636363636	62.4966435930657\\
0.145454545454545	271.688063187902\\
0.154545454545455	169.270717062847\\
0.163636363636364	94.6310333933158\\
0.172727272727273	184.582086741625\\
0.181818181818182	666.736700012295\\
0.190909090909091	1449.35896004313\\
0.2	476.21817390513\\
0.209090909090909	322.408184214304\\
0.218181818181818	259.625254240438\\
0.227272727272727	225.717444386772\\
0.236363636363636	204.718854933333\\
0.245454545454545	190.640811982545\\
0.254545454545455	180.720352693283\\
0.263636363636364	173.501389083556\\
0.272727272727273	168.141502912989\\
0.281818181818182	164.118345865112\\
0.290909090909091	168.272103286178\\
0.3	178.808157273238\\
0.309090909090909	189.420694351708\\
0.318181818181818	200.120492194913\\
0.327272727272727	210.914932463867\\
0.336363636363636	221.809055664534\\
0.345454545454545	232.80625564236\\
0.354545454545454	243.908750411783\\
0.363636363636364	255.117909513451\\
0.372727272727273	266.43448670684\\
0.381818181818182	277.858788664657\\
0.390909090909091	289.390799481727\\
0.4	301.030274119056\\
0.409090909090909	312.776809666627\\
0.418181818181818	324.629900562307\\
0.427272727272727	336.588982086145\\
0.436363636363636	348.653465223835\\
0.445454545454545	360.822765152972\\
0.454545454545455	373.096325021525\\
0.463636363636364	385.473636276934\\
0.472727272727273	397.954256512512\\
0.481818181818182	410.537825589245\\
0.490909090909091	423.224080641503\\
0.5	436.012870469526\\
0.509090909090909	448.904169746643\\
0.518181818181818	461.898093419107\\
0.527272727272727	474.994911644857\\
0.536363636363636	488.195065601079\\
0.545454545454546	501.499184487125\\
0.554545454545455	514.908104056291\\
0.563636363636364	528.422887028686\\
0.572727272727273	542.044845764741\\
0.581818181818182	555.775567617503\\
0.590909090909091	569.616943430032\\
0.6	583.571199705396\\
0.609090909090909	597.640935048434\\
0.618181818181818	611.829161566368\\
0.627272727272727	626.139352016644\\
0.636363636363636	640.575493611055\\
0.645454545454545	655.142149523019\\
0.654545454545455	669.844529305668\\
0.663636363636364	684.688569608634\\
0.672727272727273	699.681026787545\\
0.681818181818182	714.829583225389\\
0.690909090909091	730.142969430085\\
0.7	745.631104228882\\
0.709090909090909	761.30525563666\\
0.718181818181818	777.178225207753\\
0.727272727272727	793.264558860474\\
0.736363636363636	809.580787238339\\
0.745454545454545	826.145698563942\\
0.754545454545455	842.980646558429\\
0.763636363636364	860.109895178567\\
0.772727272727273	877.561000493868\\
0.781818181818182	895.365227731029\\
0.790909090909091	913.55799807522\\
0.8	932.179354932413\\
0.809090909090909	951.274432757992\\
0.818181818181818	970.893903127551\\
0.827272727272727	991.09436266831\\
0.836363636363636	1011.93861654158\\
0.845454545454545	1033.49580096766\\
0.854545454545454	1055.84128146418\\
0.863636363636364	1079.056263749\\
0.872727272727273	1103.22706590918\\
0.881818181818182	1128.44402713674\\
0.890909090909091	1154.80007118955\\
0.9	1182.38899820091\\
0.909090909090909	1211.30363698247\\
0.918181818181818	1241.63403698383\\
0.927272727272727	1273.46589901493\\
0.936363636363636	1306.87942629749\\
0.945454545454545	1341.94872260802\\
0.954545454545455	1378.74178456756\\
0.963636363636364	1417.32105086742\\
0.972727272727273	1457.74440301213\\
0.981818181818182	1500.06647335244\\
0.990909090909091	1544.34010960193\\
1	1590.61786426945\\
};
\end{axis}

\begin{axis}[%
width=1.66in,
height=1.258in,
at={(5.379in,4.137in)},
scale only axis,
xmin=0,
xmax=1,
xlabel style={font=\color{white!15!black}},
xlabel={$\tau$},
ymin=0,
ymax=15000,
axis background/.style={fill=white},
title style={font=\bfseries},
title={$\gamma\text{= 0.30}$}
]
\addplot [color=mycolor1, line width=2.0pt, forget plot]
  table[row sep=crcr]{%
0.1	36.3436883287856\\
0.109090909090909	35.2888838004028\\
0.118181818181818	607.037560320699\\
0.127272727272727	85.781427207222\\
0.136363636363636	47.8086165169407\\
0.145454545454545	64.3401071888458\\
0.154545454545455	1528.62398934469\\
0.163636363636364	163.903090660576\\
0.172727272727273	108.45590704848\\
0.181818181818182	88.2369042366601\\
0.190909090909091	76.4516757440061\\
0.2	67.1551880597727\\
0.209090909090909	91.498242886589\\
0.218181818181818	271.688063187902\\
0.227272727272727	397.80816574851\\
0.236363636363636	120.227572956511\\
0.245454545454545	94.6310333933147\\
0.254545454545455	139.254891654079\\
0.263636363636364	260.217352202585\\
0.272727272727273	666.736700012288\\
0.281818181818182	12954.8562050106\\
0.290909090909091	819.060843225456\\
0.3	476.21817390514\\
0.309090909090909	356.991887587863\\
0.318181818181818	296.321558762113\\
0.327272727272727	259.625254240432\\
0.336363636363636	235.120784671414\\
0.345454545454545	217.683077347583\\
0.354545454545454	204.718854933333\\
0.363636363636364	194.772299784982\\
0.372727272727273	186.961543145473\\
0.381818181818182	180.72035269328\\
0.390909090909091	175.668020576859\\
0.4	171.538831160212\\
0.409090909090909	168.141502912988\\
0.418181818181818	165.334707584154\\
0.427272727272727	163.011681337144\\
0.436363636363636	168.272103286176\\
0.445454545454545	175.288298680575\\
0.454545454545455	182.336554525864\\
0.463636363636364	189.420694351707\\
0.472727272727273	196.543685216188\\
0.481818181818182	203.707833899611\\
0.490909090909091	210.914932463867\\
0.5	218.166367909416\\
0.509090909090909	225.463205924435\\
0.518181818181818	232.80625564236\\
0.527272727272727	240.196120279299\\
0.536363636363636	247.63323713837\\
0.545454545454546	255.117909513449\\
0.554545454545455	262.650332356591\\
0.563636363636364	270.230613097868\\
0.572727272727273	277.858788664655\\
0.581818181818182	285.534839497139\\
0.590909090909091	293.258701173324\\
0.6	301.030274119051\\
0.609090909090909	308.849431775595\\
0.618181818181818	316.716027518607\\
0.627272727272727	324.62990056231\\
0.636363636363636	332.590881036402\\
0.645454545454545	340.598794386859\\
0.654545454545455	348.65346522383\\
0.663636363636364	356.754720717623\\
0.672727272727273	364.902393625893\\
0.681818181818182	373.096325021532\\
0.690909090909091	381.336366779476\\
0.7	389.622383871685\\
0.709090909090909	397.954256512509\\
0.718181818181818	406.331882190873\\
0.727272727272727	414.75517762121\\
0.736363636363636	423.224080641512\\
0.745454545454545	431.738552083748\\
0.754545454545455	440.298577640166\\
0.763636363636364	448.90416974665\\
0.772727272727273	457.555369503979\\
0.781818181818182	466.252248656097\\
0.790909090909091	474.99491164484\\
0.8	483.783497759886\\
0.809090909090909	492.618183403108\\
0.818181818181818	501.499184487129\\
0.827272727272727	510.426758987921\\
0.836363636363636	519.401209673214\\
0.845454545454545	528.422887028676\\
0.854545454545454	537.492192406262\\
0.863636363636364	546.609581420027\\
0.872727272727273	555.775567617497\\
0.881818181818182	564.990726456465\\
0.890909090909091	574.255699619961\\
0.9	583.571199705385\\
0.909090909090909	592.938015326581\\
0.918181818181818	602.357016672092\\
0.927272727272727	611.829161566372\\
0.936363636363636	621.355502086054\\
0.945454545454545	630.937191787878\\
0.954545454545455	640.575493611037\\
0.963636363636364	650.271788522621\\
0.972727272727273	660.027584981814\\
0.981818181818182	669.844529305689\\
0.990909090909091	679.724417027765\\
1	689.66920534868\\
};
\end{axis}

\begin{axis}[%
width=1.66in,
height=1.258in,
at={(1.011in,2.39in)},
scale only axis,
xmin=0,
xmax=1,
xlabel style={font=\color{white!15!black}},
xlabel={$\tau$},
ymin=0,
ymax=4000,
axis background/.style={fill=white},
title style={font=\bfseries},
title={$\gamma\text{= 0.40}$}
]
\addplot [color=mycolor1, line width=2.0pt, forget plot]
  table[row sep=crcr]{%
0.1	29.196008123414\\
0.109090909090909	69.0549782579689\\
0.118181818181818	103.225370593654\\
0.127272727272727	37.7324670715463\\
0.136363636363636	35.9498497653336\\
0.145454545454545	35.2888838004027\\
0.154545454545455	134.595580499949\\
0.163636363636364	181.013060184365\\
0.172727272727273	67.1703096921117\\
0.181818181818182	47.8086165169404\\
0.190909090909091	57.4591193608043\\
0.2	89.9601100726513\\
0.209090909090909	530.38393345852\\
0.218181818181818	163.903090660577\\
0.227272727272727	116.63641599291\\
0.236363636363636	96.6066960899391\\
0.245454545454545	84.8357576381935\\
0.254545454545455	76.451675744006\\
0.263636363636364	69.4231619022666\\
0.272727272727273	62.4966435930657\\
0.281818181818182	114.847624815336\\
0.290909090909091	271.688063187902\\
0.3	1184.70206205697\\
0.309090909090909	169.270717062847\\
0.318181818181818	109.652864872628\\
0.327272727272727	94.6310333933158\\
0.336363636363636	123.915302463195\\
0.345454545454545	184.582086741625\\
0.354545454545454	316.396304043223\\
0.363636363636364	666.736700012295\\
0.372727272727273	3745.77877747646\\
0.381818181818182	1449.35896004313\\
0.390909090909091	684.389694875907\\
0.4	476.21817390513\\
0.409090909090909	378.877586672044\\
0.418181818181818	322.408184214304\\
0.427272727272727	285.546431653427\\
0.436363636363636	259.625254240438\\
0.445454545454545	240.444177490185\\
0.454545454545455	225.717444386769\\
0.463636363636364	214.092978728289\\
0.472727272727273	204.71885493333\\
0.481818181818182	197.031021874975\\
0.490909090909091	190.640811982544\\
0.5	185.271454305358\\
0.509090909090909	180.720352693281\\
0.518181818181818	176.835691233729\\
0.527272727272727	173.501389083556\\
0.536363636363636	170.627118838238\\
0.545454545454546	168.141502912988\\
0.554545454545455	165.987365009691\\
0.563636363636364	164.118345865111\\
0.572727272727273	163.028071367436\\
0.581818181818182	168.272103286178\\
0.590909090909091	173.531415806194\\
0.6	178.808157273239\\
0.609090909090909	184.104096543512\\
0.618181818181818	189.420694351708\\
0.627272727272727	194.759159293542\\
0.636363636363636	200.120492194911\\
0.645454545454545	205.505521609365\\
0.654545454545455	210.914932463867\\
0.663636363636364	216.34928935958\\
0.672727272727273	221.809055664535\\
0.681818181818182	227.294609264923\\
0.690909090909091	232.80625564236\\
0.7	238.344238795453\\
0.709090909090909	243.908750411782\\
0.718181818181818	249.499937610915\\
0.727272727272727	255.117909513451\\
0.736363636363636	260.762742840306\\
0.745454545454545	266.43448670684\\
0.754545454545455	272.133166745329\\
0.763636363636364	277.858788664657\\
0.772727272727273	283.611341336496\\
0.781818181818182	289.390799481727\\
0.790909090909091	295.197126017958\\
0.8	301.030274119048\\
0.809090909090909	306.890189029111\\
0.818181818181818	312.776809666624\\
0.827272727272727	318.690070048903\\
0.836363636363636	324.629900562307\\
0.845454545454545	330.596229100072\\
0.854545454545454	336.588982086145\\
0.863636363636364	342.608085401107\\
0.872727272727273	348.65346522383\\
0.881818181818182	354.725048800725\\
0.890909090909091	360.822765152972\\
0.9	366.946545730603\\
0.909090909090909	373.096325021525\\
0.918181818181818	379.272041122312\\
0.927272727272727	385.473636276931\\
0.936363636363636	391.701057389016\\
0.945454545454545	397.954256512516\\
0.954545454545455	404.233191325232\\
0.963636363636364	410.537825589245\\
0.972727272727273	416.868129602017\\
0.981818181818182	423.224080641504\\
0.990909090909091	429.605663408542\\
1	436.012870469526\\
};
\end{axis}

\begin{axis}[%
width=1.66in,
height=1.258in,
at={(3.195in,2.39in)},
scale only axis,
xmin=0,
xmax=1,
xlabel style={font=\color{white!15!black}},
xlabel={$\tau$},
ymin=0,
ymax=10000,
axis background/.style={fill=white},
title style={font=\bfseries},
title={$\gamma\text{= 0.50}$}
]
\addplot [color=mycolor1, line width=2.0pt, forget plot]
  table[row sep=crcr]{%
0.1	30.2366410632987\\
0.109090909090909	29.8226561281053\\
0.118181818181818	29.4542339090008\\
0.127272727272727	30.7576187200539\\
0.136363636363636	69.0549782579701\\
0.145454545454545	195.644145629718\\
0.154545454545455	46.257914031276\\
0.163636363636364	36.777418674334\\
0.172727272727273	35.7712418669421\\
0.181818181818182	35.2888838004027\\
0.190909090909091	89.2904836646293\\
0.2	348.159169631992\\
0.209090909090909	124.382799692882\\
0.218181818181818	60.8565736503367\\
0.227272727272727	47.8086165169405\\
0.236363636363636	54.2008109519577\\
0.245454545454545	71.8663408902098\\
0.254545454545455	222.10212233971\\
0.263636363636364	333.577294387099\\
0.272727272727273	163.903090660577\\
0.281818181818182	122.675287214482\\
0.290909090909091	103.143107740024\\
0.3	91.2903953136264\\
0.309090909090909	82.9729001318502\\
0.318181818181818	76.4516757440063\\
0.327272727272727	70.7856322238239\\
0.336363636363636	65.3209577930806\\
0.345454545454545	75.8515875772261\\
0.354545454545454	132.237486507607\\
0.363636363636364	271.688063187898\\
0.372727272727273	9774.691044635\\
0.381818181818182	255.662567889763\\
0.390909090909091	133.2897454076\\
0.4	105.118231513925\\
0.409090909090909	94.6310333933162\\
0.418181818181818	116.378631584039\\
0.427272727272727	154.667047154946\\
0.436363636363636	225.193508656054\\
0.445454545454545	359.103462028563\\
0.454545454545455	666.736700012305\\
0.463636363636364	2050.66539754539\\
0.472727272727273	2998.94884150344\\
0.481818181818182	983.140809279017\\
0.490909090909091	625.69641018426\\
0.5	476.218173905133\\
0.509090909090909	393.973786757658\\
0.518181818181818	341.890252558938\\
0.527272727272727	305.946492327071\\
0.536363636363636	279.661901927056\\
0.545454545454546	259.62525424044\\
0.554545454545455	243.868256630128\\
0.563636363636364	231.174563285021\\
0.572727272727273	220.751367060894\\
0.581818181818182	212.059408718659\\
0.590909090909091	204.71885493333\\
0.6	198.45425757423\\
0.609090909090909	193.060876504139\\
0.618181818181818	188.383272482083\\
0.627272727272727	184.301248478333\\
0.636363636363636	180.72035269328\\
0.645454545454545	177.565302963402\\
0.654545454545455	174.775333850904\\
0.663636363636364	172.300840036895\\
0.672727272727273	170.100912686945\\
0.681818181818182	168.141502912987\\
0.690909090909091	166.394033332331\\
0.7	164.834334897254\\
0.709090909090909	163.441823244168\\
0.718181818181818	164.075773395437\\
0.727272727272727	168.272103286177\\
0.736363636363636	172.478221755289\\
0.745454545454545	176.695248418507\\
0.754545454545455	180.924141205721\\
0.763636363636364	185.16572142746\\
0.772727272727273	189.420694351708\\
0.781818181818182	193.689666211816\\
0.790909090909091	197.973158354433\\
0.8	202.271619078518\\
0.809090909090909	206.585433597225\\
0.818181818181818	210.914932463868\\
0.827272727272727	215.26039873344\\
0.836363636363636	219.622074077345\\
0.845454545454545	224.000164026858\\
0.854545454545454	228.394842487742\\
0.863636363636364	232.806255642359\\
0.872727272727273	237.234525334719\\
0.881818181818182	241.679752017139\\
0.890909090909091	246.142017323886\\
0.9	250.621386326083\\
0.909090909090909	255.117909513451\\
0.918181818181818	259.631624540974\\
0.927272727272727	264.162557772832\\
0.936363636363636	268.710725650831\\
0.945454545454545	273.276135910491\\
0.954545454545455	277.858788664654\\
0.963636363636364	282.458677371398\\
0.972727272727273	287.075789700929\\
0.981818181818182	291.710108313862\\
0.990909090909091	296.361611561729\\
1	301.030274119052\\
};
\end{axis}

\begin{axis}[%
width=1.66in,
height=1.258in,
at={(5.379in,2.39in)},
scale only axis,
xmin=0,
xmax=1,
xlabel style={font=\color{white!15!black}},
xlabel={$\tau$},
ymin=0,
ymax=15000,
axis background/.style={fill=white},
title style={font=\bfseries},
title={$\gamma\text{= 0.60}$}
]
\addplot [color=mycolor1, line width=2.0pt, forget plot]
  table[row sep=crcr]{%
0.1	31.1584016494526\\
0.109090909090909	30.7082763718817\\
0.118181818181818	30.310867325685\\
0.127272727272727	29.9550211912685\\
0.136363636363636	29.6335525912321\\
0.145454545454545	29.3387265440546\\
0.154545454545455	32.708101455046\\
0.163636363636364	69.0549782579696\\
0.172727272727273	580.88375631014\\
0.181818181818182	60.3752883281344\\
0.190909090909091	37.7324670715463\\
0.2	36.3436883287855\\
0.209090909090909	35.6706265067356\\
0.218181818181818	35.2888838004027\\
0.227272727272727	71.8364615991854\\
0.236363636363636	607.0375603207\\
0.245454545454545	181.013060184371\\
0.254545454545455	85.7814272072222\\
0.263636363636364	57.6506314807259\\
0.272727272727273	47.8086165169405\\
0.281818181818182	52.2928207680682\\
0.290909090909091	64.3401071888466\\
0.3	89.9601100726514\\
0.309090909090909	1528.62398934478\\
0.318181818181818	274.861353902272\\
0.327272727272727	163.903090660578\\
0.336363636363636	127.330150531068\\
0.345454545454545	108.45590704848\\
0.354545454545454	96.6066960899388\\
0.363636363636364	88.2369042366602\\
0.372727272727273	81.7925663064394\\
0.381818181818182	76.4516757440064\\
0.390909090909091	71.7001186294854\\
0.4	67.1551880597729\\
0.409090909090909	62.4966435930659\\
0.418181818181818	91.4982428865889\\
0.427272727272727	145.942017272339\\
0.436363636363636	271.688063187906\\
0.445454545454545	1400.1136621649\\
0.454545454545455	397.808165748552\\
0.463636363636364	169.270717062846\\
0.472727272727273	120.22757295651\\
0.481818181818182	102.600904964111\\
0.490909090909091	94.631033393316\\
0.5	111.92026492712\\
0.509090909090909	139.254891654081\\
0.518181818181818	184.582086741623\\
0.527272727272727	260.217352202581\\
0.536363636363636	392.572726668614\\
0.545454545454546	666.736700012263\\
0.554545454545455	1559.43350607549\\
0.563636363636364	12954.8562050044\\
0.572727272727273	1449.35896004336\\
0.581818181818182	819.060843225414\\
0.590909090909091	592.82513158229\\
0.6	476.218173905116\\
0.609090909090909	405.015814250089\\
0.618181818181818	356.991887587869\\
0.627272727272727	322.408184214309\\
0.636363636363636	296.321558762114\\
0.645454545454545	275.95450418966\\
0.654545454545455	259.625254240438\\
0.663636363636364	246.255310192566\\
0.672727272727273	235.120784671411\\
0.681818181818182	225.717444386773\\
0.690909090909091	217.683077347586\\
0.7	210.750645253223\\
0.709090909090909	204.718854933333\\
0.718181818181818	199.433029077769\\
0.727272727272727	194.772299784982\\
0.736363636363636	190.640811982544\\
0.745454545454545	186.961543145474\\
0.754545454545455	183.671873351953\\
0.763636363636364	180.720352693278\\
0.772727272727273	178.064304232286\\
0.781818181818182	175.668020576857\\
0.790909090909091	173.501389083554\\
0.8	171.538831160212\\
0.809090909090909	169.758474861379\\
0.818181818181818	168.141502912989\\
0.827272727272727	166.67163416317\\
0.836363636363636	165.334707584153\\
0.845454545454545	164.11834586511\\
0.854545454545454	163.011681337144\\
0.863636363636364	164.774535001989\\
0.872727272727273	168.272103286178\\
0.881818181818182	171.776472583182\\
0.890909090909091	175.288298680575\\
0.9	178.808157273237\\
0.909090909090909	182.336554525863\\
0.918181818181818	185.87393601726\\
0.927272727272727	189.420694351707\\
0.936363636363636	192.977175665944\\
0.945454545454545	196.543685216187\\
0.954545454545455	200.120492194911\\
0.963636363636364	203.707833899609\\
0.972727272727273	207.305919353907\\
0.981818181818182	210.914932463869\\
0.990909090909091	214.535034778167\\
1	218.166367909416\\
};
\end{axis}

\begin{axis}[%
width=1.66in,
height=1.258in,
at={(1.011in,0.642in)},
scale only axis,
xmin=0,
xmax=1,
xlabel style={font=\color{white!15!black}},
xlabel={$\tau$},
ymin=0,
ymax=10000,
axis background/.style={fill=white},
title style={font=\bfseries},
title={$\gamma\text{= 0.70}$}
]
\addplot [color=mycolor1, line width=2.0pt, forget plot]
  table[row sep=crcr]{%
0.1	32.0186855303424\\
0.109090909090909	31.5217511004587\\
0.118181818181818	31.0902831975368\\
0.127272727272727	30.7082763718816\\
0.136363636363636	30.3648918009368\\
0.145454545454545	30.0529115328861\\
0.154545454545455	29.7675163912883\\
0.163636363636364	29.504606592809\\
0.172727272727273	29.257224818291\\
0.181818181818182	34.3458411534703\\
0.190909090909091	69.054978257969\\
0.2	968.259900493875\\
0.209090909090909	78.5828470489368\\
0.218181818181818	42.2120620178608\\
0.227272727272727	37.0064370679647\\
0.236363636363636	36.1019302177055\\
0.245454545454545	35.6062893440572\\
0.254545454545455	35.2888838004027\\
0.263636363636364	62.4231473212583\\
0.272727272727273	204.300235180039\\
0.281818181818182	197.197229508214\\
0.290909090909091	168.402917125306\\
0.3	73.3988680621876\\
0.309090909090909	55.7077052665493\\
0.318181818181818	47.8086165169406\\
0.327272727272727	51.0381503175452\\
0.336363636363636	60.1470859890362\\
0.345454545454545	75.9460953472687\\
0.354545454545454	115.560644822273\\
0.363636363636364	1068.67634815305\\
0.372727272727273	246.476993101404\\
0.381818181818182	163.903090660575\\
0.390909090909091	131.032941428317\\
0.4	112.882165622694\\
0.409090909090909	101.122318976541\\
0.418181818181818	92.7091659602921\\
0.427272727272727	86.243697842153\\
0.436363636363636	80.9762213348017\\
0.445454545454545	76.4516757440065\\
0.454545454545455	72.3581287224075\\
0.463636363636364	68.4524265385092\\
0.472727272727273	64.524504645382\\
0.481818181818182	69.5811347847601\\
0.490909090909091	104.15182419815\\
0.5	157.097808060024\\
0.509090909090909	271.688063187918\\
0.518181818181818	873.951533021603\\
0.527272727272727	647.148519655315\\
0.536363636363636	221.872827728339\\
0.545454545454546	140.943779811388\\
0.554545454545455	113.622152213528\\
0.563636363636364	100.999627142066\\
0.572727272727273	94.6310333933164\\
0.581818181818182	108.977805779947\\
0.590909090909091	130.020428111025\\
0.6	162.301764056611\\
0.609090909090909	212.291166642969\\
0.618181818181818	290.336564893501\\
0.627272727272727	419.488100690107\\
0.636363636363636	666.736700012319\\
0.645454545454545	1325.61858443972\\
0.654545454545455	8618.93442422379\\
0.663636363636364	2280.19340733484\\
0.672727272727273	1079.42457918869\\
0.681818181818182	735.258710091374\\
0.690909090909091	571.806516231386\\
0.7	476.21817390512\\
0.709090909090909	413.443593636427\\
0.718181818181818	369.040783770991\\
0.727272727272727	335.96869782349\\
0.736363636363636	310.383344800965\\
0.745454545454545	290.007271616505\\
0.754545454545455	273.404718907254\\
0.763636363636364	259.625254240435\\
0.772727272727273	248.014398131901\\
0.781818181818182	238.106549594669\\
0.790909090909091	229.56129366028\\
0.8	222.123876271027\\
0.809090909090909	215.599780521498\\
0.818181818181818	209.837861485357\\
0.827272727272727	204.718854933329\\
0.836363636363636	200.147361595618\\
0.845454545454545	196.046138462479\\
0.854545454545454	192.351957208647\\
0.863636363636364	189.012549282565\\
0.872727272727273	185.984318576371\\
0.881818181818182	183.230605428633\\
0.890909090909091	180.720352693281\\
0.9	178.427069109667\\
0.909090909090909	176.328015313871\\
0.918181818181818	174.403558536731\\
0.927272727272727	172.636656492167\\
0.936363636363636	171.012441197609\\
0.945454545454545	169.517880812534\\
0.954545454545455	168.141502912987\\
0.963636363636364	166.87316653439\\
0.972727272727273	165.703873218728\\
0.981818181818182	164.625609477354\\
0.990909090909091	163.631214725005\\
1	162.71426999419\\
};
\end{axis}

\begin{axis}[%
width=1.66in,
height=1.258in,
at={(3.195in,0.642in)},
scale only axis,
xmin=0,
xmax=1,
xlabel style={font=\color{white!15!black}},
xlabel={$\tau$},
ymin=0,
ymax=4180.44520158369,
axis background/.style={fill=white},
title style={font=\bfseries},
title={$\gamma\text{= 0.80}$}
]
\addplot [color=mycolor1, line width=2.0pt, forget plot]
  table[row sep=crcr]{%
0.1	32.8542134439819\\
0.109090909090909	32.2986964824883\\
0.118181818181818	31.8233666977295\\
0.127272727272727	31.4084789649182\\
0.136363636363636	31.0400917805542\\
0.145454545454545	30.7082763718816\\
0.154545454545455	30.4059775477801\\
0.163636363636364	30.1282546979753\\
0.172727272727273	29.8716728234091\\
0.181818181818182	29.6335525912322\\
0.190909090909091	29.410628795612\\
0.2	29.196008123414\\
0.209090909090909	35.7444349151707\\
0.218181818181818	69.0549782579689\\
0.227272727272727	283.671557723617\\
0.236363636363636	103.225370593653\\
0.245454545454545	50.6055336020594\\
0.254545454545455	37.7324670715464\\
0.263636363636364	36.5997549851561\\
0.272727272727273	35.9498497653336\\
0.281818181818182	35.5615882875213\\
0.290909090909091	35.2888838004027\\
0.3	56.4212240448706\\
0.309090909090909	134.595580499949\\
0.318181818181818	871.246126625653\\
0.327272727272727	181.013060184365\\
0.336363636363636	104.642406731023\\
0.345454545454545	67.1703096921117\\
0.354545454545454	54.403902487494\\
0.363636363636364	47.8086165169404\\
0.372727272727273	50.1497976508172\\
0.381818181818182	57.4591193608043\\
0.390909090909091	68.7642216595517\\
0.4	89.9601100726513\\
0.409090909090909	380.059444215442\\
0.418181818181818	530.38393345852\\
0.427272727272727	229.705231257816\\
0.436363636363636	163.903090660577\\
0.445454545454545	134.050837934444\\
0.454545454545455	116.63641599291\\
0.463636363636364	105.029876891723\\
0.472727272727273	96.6066960899387\\
0.481818181818182	90.1057035182821\\
0.490909090909091	84.8357576381935\\
0.5	80.3773971183594\\
0.509090909090909	76.4516757440059\\
0.518181818181818	72.8549625674166\\
0.527272727272727	69.4231619022666\\
0.536363636363636	66.0122325741425\\
0.545454545454546	62.4966435930659\\
0.554545454545455	81.509959572153\\
0.563636363636364	114.847624815336\\
0.572727272727273	166.386083214706\\
0.581818181818182	271.688063187906\\
0.590909090909091	683.162076372756\\
0.6	1184.70206205739\\
0.609090909090909	295.403810957942\\
0.618181818181818	169.270717062847\\
0.627272727272727	127.727609464329\\
0.636363636363636	109.652864872627\\
0.645454545454545	99.8912311893521\\
0.654545454545455	94.6310333933158\\
0.663636363636364	106.891223953159\\
0.672727272727273	123.915302463196\\
0.681818181818182	148.514737465029\\
0.690909090909091	184.582086741625\\
0.7	237.465593019135\\
0.709090909090909	316.39630404321\\
0.718181818181818	441.596790972169\\
0.727272727272727	666.736700012295\\
0.736363636363636	1188.87921410766\\
0.745454545454545	3745.77877747646\\
0.754545454545455	4180.44520158369\\
0.763636363636364	1449.35896004313\\
0.772727272727273	913.383465124697\\
0.781818181818182	684.389694875907\\
0.790909090909091	557.20862215993\\
0.8	476.218173905134\\
0.809090909090909	420.087312076085\\
0.818181818181818	378.877586672054\\
0.827272727272727	347.332022592303\\
0.836363636363636	322.408184214304\\
0.845454545454545	302.222586755491\\
0.854545454545454	285.546431653427\\
0.863636363636364	271.543612648879\\
0.872727272727273	259.625254240438\\
0.881818181818182	249.364478126227\\
0.890909090909091	240.444177490185\\
0.9	232.623786728747\\
0.909090909090909	225.717444386769\\
0.918181818181818	219.579236610958\\
0.927272727272727	214.092978728286\\
0.936363636363636	209.164985181791\\
0.945454545454545	204.718854933333\\
0.954545454545455	200.69164545553\\
0.963636363636364	197.031021874975\\
0.972727272727273	193.693102834116\\
0.981818181818182	190.640811982545\\
0.990909090909091	187.842601681889\\
1	185.271454305358\\
};
\end{axis}

\begin{axis}[%
width=1.66in,
height=1.258in,
at={(5.379in,0.642in)},
scale only axis,
xmin=0,
xmax=1,
xlabel style={font=\color{white!15!black}},
xlabel={$\tau$},
ymin=0,
ymax=15000,
axis background/.style={fill=white},
title style={font=\bfseries},
title={$\gamma\text{= 0.90}$}
]
\addplot [color=mycolor1, line width=2.0pt, forget plot]
  table[row sep=crcr]{%
0.1	33.6845649675984\\
0.109090909090909	33.0617918009202\\
0.118181818181818	32.5342462505993\\
0.127272727272727	32.0788743707736\\
0.136363636363636	31.6791421527393\\
0.145454545454545	31.3229993978382\\
0.154545454545455	31.0015679860766\\
0.163636363636364	30.7082763718816\\
0.172727272727273	30.4382775180628\\
0.181818181818182	30.188041069615\\
0.190909090909091	29.9550211912686\\
0.2	29.7372771140588\\
0.209090909090909	29.5328697678665\\
0.218181818181818	29.3387265440546\\
0.227272727272727	29.6320104019907\\
0.236363636363636	36.9547872779762\\
0.245454545454545	69.0549782579691\\
0.254545454545455	179.652222229154\\
0.263636363636364	138.851569667476\\
0.272727272727273	60.3752883281341\\
0.281818181818182	40.2912860723892\\
0.290909090909091	37.1472273226347\\
0.3	36.3436883287855\\
0.309090909090909	35.8461517461103\\
0.318181818181818	35.5286640182098\\
0.327272727272727	35.2888838004027\\
0.336363636363636	52.1680203687614\\
0.345454545454545	105.419099050198\\
0.354545454545454	607.037560320696\\
0.363636363636364	149.935788333348\\
0.372727272727273	221.831058864714\\
0.381818181818182	85.7814272072222\\
0.390909090909091	63.3957486798933\\
0.4	53.4684917321304\\
0.409090909090909	47.8086165169403\\
0.418181818181818	49.4875927323509\\
0.427272727272727	55.5844892414267\\
0.436363636363636	64.3401071888466\\
0.445454545454545	78.5168297890304\\
0.454545454545455	108.081183782405\\
0.463636363636364	1528.62398934455\\
0.472727272727273	395.388513694146\\
0.481818181818182	218.616766554177\\
0.490909090909091	163.903090660577\\
0.5	136.558835669991\\
0.509090909090909	119.865631205533\\
0.518181818181818	108.455907048479\\
0.527272727272727	100.056478356052\\
0.536363636363636	93.5297344142452\\
0.545454545454546	88.2369042366602\\
0.554545454545455	83.786305990533\\
0.563636363636364	79.9191114559637\\
0.572727272727273	76.4516757440061\\
0.581818181818182	73.2436722412636\\
0.590909090909091	70.1791315206479\\
0.6	67.155188059773\\
0.609090909090909	64.0786606429091\\
0.618181818181818	66.1503053984256\\
0.627272727272727	91.4982428865886\\
0.636363636363636	124.103823665029\\
0.645454545454545	174.253724086657\\
0.654545454545455	271.688063187896\\
0.663636363636364	584.456030703187\\
0.672727272727273	3188.57895386642\\
0.681818181818182	397.808165748532\\
0.690909090909091	206.956823768032\\
0.7	145.938918932655\\
0.709090909090909	120.227572956511\\
0.718181818181818	107.007181529904\\
0.727272727272727	99.0784334906581\\
0.736363636363636	94.6310333933145\\
0.745454545454545	105.334675647313\\
0.754545454545455	119.593666743346\\
0.763636363636364	139.25489165408\\
0.772727272727273	166.841355361487\\
0.781818181818182	205.599882725048\\
0.790909090909091	260.217352202589\\
0.8	339.123969772177\\
0.809090909090909	460.078637339896\\
0.818181818181818	666.736700012318\\
0.827272727272727	1099.14810486413\\
0.836363636363636	2576.19124441218\\
0.845454545454545	12954.8562050122\\
0.854545454545454	2016.94687006252\\
0.863636363636364	1142.74300168084\\
0.872727272727273	819.060843225443\\
0.881818181818182	650.225314027261\\
0.890909090909091	546.479014965914\\
0.9	476.218173905142\\
0.909090909090909	425.459273911203\\
0.918181818181818	387.060291449262\\
0.927272727272727	356.991887587869\\
0.936363636363636	332.807741782318\\
0.945454545454545	312.936293350421\\
0.954545454545455	296.321558762112\\
0.963636363636364	282.227533547859\\
0.972727272727273	270.125386893267\\
0.981818181818182	259.625254240438\\
0.990909090909091	250.433326117658\\
1	242.323923009075\\
};
\end{axis}
\end{tikzpicture}%}}
 \hfill 
\subfloat[][$N=40$]
{\resizebox{0.45\textwidth}{!}{ % This file was created by matlab2tikz.
%
%The latest updates can be retrieved from
%  http://www.mathworks.com/matlabcentral/fileexchange/22022-matlab2tikz-matlab2tikz
%where you can also make suggestions and rate matlab2tikz.
%
\definecolor{mycolor1}{rgb}{0.00000,0.44700,0.74100}%
%
\begin{tikzpicture}

\begin{axis}[%
width=1.66in,
height=1.258in,
at={(1.011in,4.137in)},
scale only axis,
xmin=0,
xmax=1,
xlabel style={font=\color{white!15!black}},
xlabel={$\tau$},
ymin=0,
ymax=41160.9588703624,
axis background/.style={fill=white},
title style={font=\bfseries},
title={$\gamma\text{= 0.10}$}
]
\addplot [color=mycolor1, line width=2.0pt, forget plot]
  table[row sep=crcr]{%
0.1	3455.81810222158\\
0.109090909090909	915.795541933859\\
0.118181818181818	630.185955946254\\
0.127272727272727	605.922485473726\\
0.136363636363636	1116.02013501372\\
0.145454545454545	2586.3027821879\\
0.154545454545455	41160.9588703624\\
0.163636363636364	4095.42358269983\\
0.172727272727273	2221.80577046475\\
0.181818181818182	1631.30865358265\\
0.190909090909091	1740.60637564776\\
0.2	3194.56889956491\\
0.209090909090909	9228.02383230087\\
0.218181818181818	19372.1617111604\\
0.227272727272727	5664.08885520872\\
0.236363636363636	3648.49435663144\\
0.245454545454545	2854.84999172631\\
0.254545454545455	2439.78892411579\\
0.263636363636364	2190.49306465024\\
0.272727272727273	2028.26194997301\\
0.281818181818182	1917.28827914251\\
0.290909090909091	1838.97032046234\\
0.3	1782.70385546659\\
0.309090909090909	1742.01140171305\\
0.318181818181818	1712.7183106641\\
0.327272727272727	1692.01595079939\\
0.336363636363636	1677.946841037\\
0.345454545454545	1669.1057850792\\
0.354545454545454	1664.45836703182\\
0.363636363636364	1663.22644326485\\
0.372727272727273	1664.81350857666\\
0.381818181818182	1668.75465840061\\
0.390909090909091	1674.68220114513\\
0.4	1682.30150343649\\
0.409090909090909	1691.37368949409\\
0.418181818181818	1701.70303135704\\
0.427272727272727	1713.12761211948\\
0.436363636363636	1725.51231313796\\
0.445454545454545	1738.74347775084\\
0.454545454545455	1752.72480208376\\
0.463636363636364	1767.37413598996\\
0.472727272727273	1782.62096733675\\
0.481818181818182	1798.40442517152\\
0.490909090909091	1814.67168101861\\
0.5	1831.37665862571\\
0.509090909090909	1848.47898484657\\
0.518181818181818	1865.94313062617\\
0.527272727272727	1883.73770303717\\
0.536363636363636	1901.83485823099\\
0.545454545454546	1920.20981184187\\
0.554545454545455	1938.84042845789\\
0.563636363636364	1957.70687562789\\
0.572727272727273	1976.79133085326\\
0.581818181818182	1996.0777323195\\
0.590909090909091	2015.55156591996\\
0.6	2035.19968254606\\
0.609090909090909	2055.01014072925\\
0.618181818181818	2074.97207061964\\
0.627272727272727	2095.07555599058\\
0.636363636363636	2115.31153154161\\
0.645454545454545	2135.67169322613\\
0.654545454545455	2156.14841971558\\
0.663636363636364	2176.73470341395\\
0.672727272727273	2197.4240896951\\
0.681818181818182	2218.21062323529\\
0.690909090909091	2239.08880049452\\
0.7	2260.05352753661\\
0.709090909090909	2281.10008250041\\
0.718181818181818	2302.22408213246\\
0.727272727272727	2323.42145187819\\
0.736363636363636	2344.68839909551\\
0.745454545454545	2366.0213890168\\
0.754545454545455	2387.4171231346\\
0.763636363636364	2408.87251973143\\
0.772727272727273	2430.3846963078\\
0.781818181818182	2451.95095369683\\
0.790909090909091	2473.56876167654\\
0.8	2495.23574592164\\
0.809090909090909	2516.94967614717\\
0.818181818181818	2538.70845532131\\
0.827272727272727	2560.51010983495\\
0.836363636363636	2582.35278053004\\
0.845454545454545	2604.23471450359\\
0.854545454545454	2626.15425760302\\
0.863636363636364	2648.10984755472\\
0.872727272727273	2670.10000765685\\
0.881818181818182	2692.12334098612\\
0.890909090909091	2714.17852506997\\
0.9	2736.26430698168\\
0.909090909090909	2758.37949881781\\
0.918181818181818	2780.52297352527\\
0.927272727272727	2802.69366104623\\
0.936363636363636	2824.89054475313\\
0.945454545454545	2847.11265814684\\
0.954545454545455	2869.35908180123\\
0.963636363636364	2891.62894052369\\
0.972727272727273	2913.92140072272\\
0.981818181818182	2936.23566796081\\
0.990909090909091	2958.5709846763\\
1	2980.92662806708\\
};
\end{axis}

\begin{axis}[%
width=1.66in,
height=1.258in,
at={(3.195in,4.137in)},
scale only axis,
xmin=0,
xmax=1,
xlabel style={font=\color{white!15!black}},
xlabel={$\tau$},
ymin=0,
ymax=42661.6032019987,
axis background/.style={fill=white},
title style={font=\bfseries},
title={$\gamma\text{= 0.20}$}
]
\addplot [color=mycolor1, line width=2.0pt, forget plot]
  table[row sep=crcr]{%
0.1	738.689641885675\\
0.109090909090909	269.263395197009\\
0.118181818181818	299.638439252794\\
0.127272727272727	1061.09127268402\\
0.136363636363636	1695.78257150439\\
0.145454545454545	611.188536496972\\
0.154545454545455	1004.97829809339\\
0.163636363636364	3997.9574644868\\
0.172727272727273	3120.35050389857\\
0.181818181818182	1306.65986517501\\
0.190909090909091	3465.33444396115\\
0.2	3455.81810222158\\
0.209090909090909	1352.47111858437\\
0.218181818181818	915.795541933859\\
0.227272727272727	730.459027232894\\
0.236363636363636	630.18595594628\\
0.245454545454545	568.882423680266\\
0.254545454545455	605.922485473728\\
0.263636363636364	810.66700301237\\
0.272727272727273	1116.02013501372\\
0.281818181818182	1617.29012116618\\
0.290909090909091	2586.3027821879\\
0.3	5232.77864007187\\
0.309090909090909	41160.9588703624\\
0.318181818181818	8377.9580343219\\
0.327272727272727	4095.42358269983\\
0.336363636363636	2828.66452844351\\
0.345454545454545	2221.80577046475\\
0.354545454545454	1865.66706874404\\
0.363636363636364	1631.30865358265\\
0.372727272727273	1465.20824192253\\
0.381818181818182	1740.60637564776\\
0.390909090909091	2294.67247947785\\
0.4	3194.56889956491\\
0.409090909090909	4891.83670151323\\
0.418181818181818	9228.02383230087\\
0.427272727272727	42661.6032019987\\
0.436363636363636	19372.1617111604\\
0.445454545454545	8478.33707683159\\
0.454545454545455	5664.08885520906\\
0.463636363636364	4379.62309389511\\
0.472727272727273	3648.49435663111\\
0.481818181818182	3179.37776272835\\
0.490909090909091	2854.8499917267\\
0.5	2618.48161206717\\
0.509090909090909	2439.78892411582\\
0.518181818181818	2300.86258449189\\
0.527272727272727	2190.49306465024\\
0.536363636363636	2101.30806445078\\
0.545454545454546	2028.26194997289\\
0.554545454545455	1967.78652246826\\
0.563636363636364	1917.28827914233\\
0.572727272727273	1874.83773257578\\
0.581818181818182	1838.97032046235\\
0.590909090909091	1808.55479644058\\
0.6	1782.7038554666\\
0.609090909090909	1760.71199438326\\
0.618181818181818	1742.01140171305\\
0.627272727272727	1726.1400623671\\
0.636363636363636	1712.71831066411\\
0.645454545454545	1701.43133566451\\
0.654545454545455	1692.01595079939\\
0.663636363636364	1684.25046502024\\
0.672727272727273	1677.94684103713\\
0.681818181818182	1672.94456144704\\
0.690909090909091	1669.1057850792\\
0.7	1666.31148847673\\
0.709090909090909	1664.45836703178\\
0.718181818181818	1663.4563272926\\
0.727272727272727	1663.22644326485\\
0.736363636363636	1663.69927980023\\
0.745454545454545	1664.81350857666\\
0.754545454545455	1666.51475892507\\
0.763636363636364	1668.75465840061\\
0.772727272727273	1671.49002760514\\
0.781818181818182	1674.68220114513\\
0.790909090909091	1678.29645230336\\
0.8	1682.30150343647\\
0.809090909090909	1686.66910758153\\
0.818181818181818	1691.37368949413\\
0.827272727272727	1696.39203651051\\
0.836363636363636	1701.70303135704\\
0.845454545454545	1707.28742042154\\
0.854545454545454	1713.12761211948\\
0.863636363636364	1719.20750089546\\
0.872727272727273	1725.51231313797\\
0.881818181818182	1732.02847189266\\
0.890909090909091	1738.74347775084\\
0.9	1745.64580370325\\
0.909090909090909	1752.72480208376\\
0.918181818181818	1759.97062201406\\
0.927272727272727	1767.37413598999\\
0.936363636363636	1774.92687445097\\
0.945454545454545	1782.62096733682\\
0.954545454545455	1790.44909177669\\
0.963636363636364	1798.40442517152\\
0.972727272727273	1806.48060303218\\
0.981818181818182	1814.67168101863\\
0.990909090909091	1822.97210070181\\
1	1831.37665862571\\
};
\end{axis}

\begin{axis}[%
width=1.66in,
height=1.258in,
at={(5.379in,4.137in)},
scale only axis,
xmin=0,
xmax=1,
xlabel style={font=\color{white!15!black}},
xlabel={$\tau$},
ymin=0,
ymax=400000,
axis background/.style={fill=white},
title style={font=\bfseries},
title={$\gamma\text{= 0.30}$}
]
\addplot [color=mycolor1, line width=2.0pt, forget plot]
  table[row sep=crcr]{%
0.1	135.945908034256\\
0.109090909090909	356.347724729277\\
0.118181818181818	293.198862547446\\
0.127272727272727	165.838452005682\\
0.136363636363636	229.824356133225\\
0.145454545454545	16709.915753003\\
0.154545454545455	427.805590779589\\
0.163636363636364	269.263395197009\\
0.172727272727273	224.197137324583\\
0.181818181818182	415.975869516105\\
0.190909090909091	1061.09127268398\\
0.2	6863.28600077641\\
0.209090909090909	1019.22870689446\\
0.218181818181818	611.188536496951\\
0.227272727272727	772.769539745067\\
0.236363636363636	1382.289836503\\
0.245454545454545	3997.95746448719\\
0.254545454545455	6837.97662439582\\
0.263636363636364	2079.59515946655\\
0.272727272727273	1306.65986517505\\
0.281818181818182	1935.4123813455\\
0.290909090909091	12601.0379666873\\
0.3	3455.81810222154\\
0.309090909090909	1662.49953546297\\
0.318181818181818	1154.05502208646\\
0.327272727272727	915.79554193389\\
0.336363636363636	778.852077080795\\
0.345454545454545	690.839425425633\\
0.354545454545454	630.185955946256\\
0.363636363636364	586.375125818334\\
0.372727272727273	553.661159052407\\
0.381818181818182	605.922485473731\\
0.390909090909091	733.947695847201\\
0.4	898.191144643877\\
0.409090909090909	1116.0201350137\\
0.418181818181818	1418.05744078267\\
0.427272727272727	1863.81658292578\\
0.436363636363636	2586.30278218796\\
0.445454545454545	3956.14546561072\\
0.454545454545455	7540.690834504\\
0.463636363636364	41160.9588704587\\
0.472727272727273	13557.4820795668\\
0.481818181818182	6139.65031372435\\
0.490909090909091	4095.42358269967\\
0.5	3138.58481604602\\
0.509090909090909	2583.8729285286\\
0.518181818181818	2221.80577046478\\
0.527272727272727	1966.83642924381\\
0.536363636363636	1777.51161984264\\
0.545454545454546	1631.30865358264\\
0.554545454545455	1514.93958167377\\
0.563636363636364	1477.40458792106\\
0.572727272727273	1740.60637564791\\
0.581818181818182	2082.62661050101\\
0.590909090909091	2543.28376410454\\
0.6	3194.56889956455\\
0.609090909090909	4181.38057797729\\
0.618181818181818	5845.14382174\\
0.627272727272727	9228.02383230496\\
0.636363636363636	19765.5633814681\\
0.645454545454545	382038.981809421\\
0.654545454545455	19372.1617111591\\
0.663636363636364	10336.5254800739\\
0.672727272727273	7230.98778718578\\
0.681818181818182	5664.08885520736\\
0.690909090909091	4721.94197381077\\
0.7	4094.76147202234\\
0.709090909090909	3648.49435663123\\
0.718181818181818	3315.68640356779\\
0.727272727272727	3058.69280994078\\
0.736363636363636	2854.84999172679\\
0.745454545454545	2689.69795126113\\
0.754545454545455	2553.57698773875\\
0.763636363636364	2439.78892411598\\
0.772727272727273	2343.54344697269\\
0.781818181818182	2261.32442595525\\
0.790909090909091	2190.49306465013\\
0.8	2129.03066343997\\
0.809090909090909	2075.36686069199\\
0.818181818181818	2028.26194997291\\
0.827272727272727	1986.72439924141\\
0.836363636363636	1949.95186998882\\
0.845454545454545	1917.28827914238\\
0.854545454545454	1888.19203409294\\
0.863636363636364	1862.21219053805\\
0.872727272727273	1838.97032046213\\
0.881818181818182	1818.14655677118\\
0.890909090909091	1799.46873438264\\
0.9	1782.70385546663\\
0.909090909090909	1767.65131906542\\
0.918181818181818	1754.13750424649\\
0.927272727272727	1742.01140171306\\
0.936363636363636	1731.14106489488\\
0.945454545454545	1721.41070692403\\
0.954545454545455	1712.71831066404\\
0.963636363636364	1704.97364925804\\
0.972727272727273	1698.09663740499\\
0.981818181818182	1692.01595079947\\
0.990909090909091	1686.66786431876\\
1	1681.99526966709\\
};
\end{axis}

\begin{axis}[%
width=1.66in,
height=1.258in,
at={(1.011in,2.39in)},
scale only axis,
xmin=0,
xmax=1,
xlabel style={font=\color{white!15!black}},
xlabel={$\tau$},
ymin=0,
ymax=80000,
axis background/.style={fill=white},
title style={font=\bfseries},
title={$\gamma\text{= 0.40}$}
]
\addplot [color=mycolor1, line width=2.0pt, forget plot]
  table[row sep=crcr]{%
0.1	53.4930108932153\\
0.109090909090909	248.500940657273\\
0.118181818181818	193.615184817563\\
0.127272727272727	206.289649757684\\
0.136363636363636	182.40474821274\\
0.145454545454545	356.347724729264\\
0.154545454545455	417.014563672632\\
0.163636363636364	202.562052186748\\
0.172727272727273	154.667465248204\\
0.181818181818182	229.824356133228\\
0.190909090909091	1330.72551585582\\
0.2	738.689641885675\\
0.209090909090909	364.424334775883\\
0.218181818181818	269.263395197009\\
0.227272727272727	226.706102494041\\
0.236363636363636	299.638439252791\\
0.245454545454545	501.502642479916\\
0.254545454545455	1061.09127268398\\
0.263636363636364	10100.8938528511\\
0.272727272727273	1695.78257150439\\
0.281818181818182	862.323443894832\\
0.290909090909091	611.188536496972\\
0.3	687.30105879447\\
0.309090909090909	1004.97829809339\\
0.318181818181818	1675.34870883379\\
0.327272727272727	3997.9574644868\\
0.336363636363636	18705.3661100918\\
0.345454545454545	3120.35050389857\\
0.354545454545454	1797.62517044859\\
0.363636363636364	1306.65986517501\\
0.372727272727273	1566.26260299368\\
0.381818181818182	3465.33444396115\\
0.390909090909091	53380.1312135808\\
0.4	3455.81810222158\\
0.409090909090909	1893.14671581189\\
0.418181818181818	1352.47111858437\\
0.427272727272727	1079.54320313623\\
0.436363636363636	915.795541933859\\
0.445454545454545	807.241501994017\\
0.454545454545455	730.459027232905\\
0.463636363636364	673.640561415243\\
0.472727272727273	630.185955946254\\
0.481818181818182	596.11487092869\\
0.490909090909091	568.882423680251\\
0.5	546.7840236847\\
0.509090909090909	605.922485473726\\
0.518181818181818	699.044587581658\\
0.527272727272727	810.66700301237\\
0.536363636363636	946.728006354656\\
0.545454545454546	1116.02013501371\\
0.554545454545455	1332.14004893826\\
0.563636363636364	1617.29012116616\\
0.572727272727273	2010.36892793298\\
0.581818181818182	2586.30278218784\\
0.590909090909091	3510.33071008253\\
0.6	5232.77864007143\\
0.609090909090909	9570.2459159284\\
0.618181818181818	41160.9588703624\\
0.627272727272727	19947.2137685078\\
0.636363636363636	8377.95803432183\\
0.645454545454545	5437.93420228369\\
0.654545454545455	4095.42358269983\\
0.663636363636364	3326.66846013191\\
0.672727272727273	2828.66452844324\\
0.681818181818182	2479.80443497344\\
0.690909090909091	2221.80577046475\\
0.7	2023.24062541165\\
0.709090909090909	1865.66706874409\\
0.718181818181818	1737.55077062906\\
0.727272727272727	1631.30865358265\\
0.736363636363636	1541.75126695394\\
0.745454545454545	1465.20824192253\\
0.754545454545455	1537.30519080454\\
0.763636363636364	1740.60637564776\\
0.772727272727273	1987.93273529928\\
0.781818181818182	2294.67247947785\\
0.790909090909091	2684.31147737681\\
0.8	3194.5688995644\\
0.809090909090909	3890.10865910568\\
0.818181818181818	4891.8367015142\\
0.827272727272727	6455.04835430443\\
0.836363636363636	9228.02383230087\\
0.845454545454545	15481.3004470471\\
0.854545454545454	42661.6032019987\\
0.863636363636364	65576.030936271\\
0.872727272727273	19372.1617111675\\
0.881818181818182	11653.6590311155\\
0.890909090909091	8478.33707683159\\
0.9	6749.6386248687\\
0.909090909090909	5664.08885520906\\
0.918181818181818	4920.26025234102\\
0.927272727272727	4379.62309389586\\
0.936363636363636	3969.60075445514\\
0.945454545454545	3648.49435663144\\
0.954545454545455	3390.64015919515\\
0.963636363636364	3179.37776272835\\
0.972727272727273	3003.42035119879\\
0.981818181818182	2854.84999172631\\
0.990909090909091	2727.94811877149\\
1	2618.48161206717\\
};
\end{axis}

\begin{axis}[%
width=1.66in,
height=1.258in,
at={(3.195in,2.39in)},
scale only axis,
xmin=0,
xmax=1,
xlabel style={font=\color{white!15!black}},
xlabel={$\tau$},
ymin=0,
ymax=300000,
axis background/.style={fill=white},
title style={font=\bfseries},
title={$\gamma\text{= 0.50}$}
]
\addplot [color=mycolor1, line width=2.0pt, forget plot]
  table[row sep=crcr]{%
0.1	38.7398902459126\\
0.109090909090909	39.0114516851962\\
0.118181818181818	39.3016338302424\\
0.127272727272727	63.7084615202248\\
0.136363636363636	248.500940657275\\
0.145454545454545	101.228913529317\\
0.154545454545455	387.59545791168\\
0.163636363636364	154.519670244442\\
0.172727272727273	423.661181112592\\
0.181818181818182	356.347724729276\\
0.190909090909091	596.961559670541\\
0.2	244.635471085431\\
0.209090909090909	177.534770309039\\
0.218181818181818	149.279277014117\\
0.227272727272727	229.824356133225\\
0.236363636363636	764.279496739967\\
0.245454545454545	1605.07724119345\\
0.254545454545455	506.547065723854\\
0.263636363636364	337.198063016822\\
0.272727272727273	269.263395197009\\
0.281818181818182	233.171680575122\\
0.290909090909091	250.952678827142\\
0.3	362.61176290188\\
0.309090909090909	567.018205482573\\
0.318181818181818	1061.09127268399\\
0.327272727272727	3938.45805151034\\
0.336363636363636	3028.57446007546\\
0.345454545454545	1204.34656384421\\
0.354545454545454	792.540952747291\\
0.363636363636364	611.188536496973\\
0.372727272727273	642.892519940762\\
0.381818181818182	854.019365750064\\
0.390909090909091	1206.25939698303\\
0.4	1909.42939858931\\
0.409090909090909	3997.9574644872\\
0.418181818181818	241555.024541792\\
0.427272727272727	4593.65930339416\\
0.436363636363636	2390.74124873914\\
0.445454545454545	1666.36072383102\\
0.454545454545455	1306.65986517503\\
0.463636363636364	1400.2474425286\\
0.472727272727273	2363.80420905679\\
0.481818181818182	6215.16226103006\\
0.490909090909091	13223.2453556349\\
0.5	3455.81810222176\\
0.509090909090909	2071.31052572182\\
0.518181818181818	1519.7344905872\\
0.527272727272727	1224.15054492731\\
0.536363636363636	1040.51592418582\\
0.545454545454546	915.795541933875\\
0.554545454545455	825.891072663296\\
0.563636363636364	758.273774072782\\
0.572727272727273	705.784701579921\\
0.581818181818182	664.035580705226\\
0.590909090909091	630.185955946258\\
0.6	602.31536061742\\
0.609090909090909	579.078092015586\\
0.618181818181818	559.502576206938\\
0.627272727272727	542.869321856245\\
0.636363636363636	605.922485473732\\
0.645454545454545	679.093885770113\\
0.654545454545455	763.467571171865\\
0.663636363636364	861.745795894316\\
0.672727272727273	977.578718579956\\
0.681818181818182	1116.02013501374\\
0.690909090909091	1284.27420250334\\
0.7	1492.97569449644\\
0.709090909090909	1758.50860358471\\
0.718181818181818	2107.49913472469\\
0.727272727272727	2586.3027821879\\
0.736363636363636	3283.43423408891\\
0.745454545454545	4391.60279059186\\
0.754545454545455	6424.96576436807\\
0.763636363636364	11368.8005880215\\
0.772727272727273	41160.958870422\\
0.781818181818182	28026.9269366338\\
0.790909090909091	10838.9680312859\\
0.8	6862.88877625086\\
0.809090909090909	5094.85505291397\\
0.818181818181818	4095.42358269994\\
0.827272727272727	3452.95764141487\\
0.836363636363636	3005.2218694582\\
0.845454545454545	2675.34575067415\\
0.854545454545454	2422.20456154281\\
0.863636363636364	2221.80577046472\\
0.872727272727273	2059.20945357062\\
0.881818181818182	1924.62708842341\\
0.890909090909091	1811.37777218083\\
0.9	1714.7459764935\\
0.909090909090909	1631.30865358265\\
0.918181818181818	1558.52111299465\\
0.927272727272727	1494.45256077618\\
0.936363636363636	1437.61178175155\\
0.945454545454545	1575.00066623016\\
0.954545454545455	1740.60637564791\\
0.963636363636364	1934.29724234038\\
0.972727272727273	2163.57767396134\\
0.981818181818182	2438.88005006474\\
0.990909090909091	2775.147622133\\
1	3194.56889956465\\
};
\end{axis}

\begin{axis}[%
width=1.66in,
height=1.258in,
at={(5.379in,2.39in)},
scale only axis,
xmin=0,
xmax=1,
xlabel style={font=\color{white!15!black}},
xlabel={$\tau$},
ymin=0,
ymax=41160.958870455,
axis background/.style={fill=white},
title style={font=\bfseries},
title={$\gamma\text{= 0.60}$}
]
\addplot [color=mycolor1, line width=2.0pt, forget plot]
  table[row sep=crcr]{%
0.1	38.6857272989289\\
0.109090909090909	38.5511211771193\\
0.118181818181818	38.6997870148874\\
0.127272727272727	38.9168616608096\\
0.136363636363636	39.1565082946562\\
0.145454545454545	43.5382086819972\\
0.154545454545455	74.7644445399834\\
0.163636363636364	248.500940657277\\
0.172727272727273	109.527929438416\\
0.181818181818182	3007.23177328091\\
0.190909090909091	206.289649757682\\
0.2	135.945908034255\\
0.209090909090909	1482.36027866643\\
0.218181818181818	356.347724729278\\
0.227272727272727	883.40254059682\\
0.236363636363636	293.19886254745\\
0.245454545454545	202.562052186746\\
0.254545454545455	165.838452005682\\
0.263636363636364	146.11064666488\\
0.272727272727273	229.824356133222\\
0.281818181818182	583.114707156149\\
0.290909090909091	16709.9157529997\\
0.3	738.689641885696\\
0.309090909090909	427.80559077959\\
0.318181818181818	322.070135866955\\
0.327272727272727	269.263395197008\\
0.336363636363636	237.938054296909\\
0.345454545454545	224.197137324581\\
0.354545454545454	299.638439252793\\
0.363636363636364	415.97586951611\\
0.372727272727273	618.806171636297\\
0.381818181818182	1061.09127268398\\
0.390909090909091	2771.9335677916\\
0.4	6863.28600077585\\
0.409090909090909	1695.78257150441\\
0.418181818181818	1019.22870689448\\
0.427272727272727	753.095982608284\\
0.436363636363636	611.188536496968\\
0.445454545454545	615.690022417681\\
0.454545454545455	772.769539745098\\
0.463636363636364	1004.97829809345\\
0.472727272727273	1382.28983650296\\
0.481818181818182	2100.67760580352\\
0.490909090909091	3997.95746448685\\
0.5	23055.0633330213\\
0.509090909090909	6837.9766243936\\
0.518181818181818	3120.35050389837\\
0.527272727272727	2079.59515946668\\
0.536363636363636	1590.46886497409\\
0.545454545454546	1306.65986517509\\
0.554545454545455	1305.89191111392\\
0.563636363636364	1935.41238134547\\
0.572727272727273	3465.33444396093\\
0.581818181818182	12601.0379666869\\
0.590909090909091	8868.45247613791\\
0.6	3455.81810222123\\
0.609090909090909	2213.04037387957\\
0.618181818181818	1662.49953546292\\
0.627272727272727	1352.4711185844\\
0.636363636363636	1154.05502208646\\
0.645454545454545	1016.50574405273\\
0.654545454545455	915.795541933877\\
0.663636363636364	839.075646624316\\
0.672727272727273	778.852077080802\\
0.681818181818182	730.459027232895\\
0.690909090909091	690.839425425626\\
0.7	657.9063007722\\
0.709090909090909	630.185955946259\\
0.718181818181818	606.607814959433\\
0.727272727272727	586.375125818322\\
0.736363636363636	568.882423680253\\
0.745454545454545	553.661159052408\\
0.754545454545455	552.014346578815\\
0.763636363636364	605.922485473739\\
0.772727272727273	666.182841300237\\
0.781818181818182	733.947695847194\\
0.790909090909091	810.667003012364\\
0.8	898.191144643872\\
0.809090909090909	998.919448352099\\
0.818181818181818	1116.02013501368\\
0.827272727272727	1253.76525598045\\
0.836363636363636	1418.05744078272\\
0.845454545454545	1617.29012116624\\
0.854545454545454	1863.81658292587\\
0.863636363636364	2176.59778571462\\
0.872727272727273	2586.30278218777\\
0.881818181818182	3145.99599126731\\
0.890909090909091	3956.14546561055\\
0.9	5232.77864007197\\
0.909090909090909	7540.69083450444\\
0.918181818181818	12973.6300178498\\
0.927272727272727	41160.958870455\\
0.936363636363636	38569.0258826277\\
0.945454545454545	13557.4820795662\\
0.954545454545455	8377.95803432244\\
0.963636363636364	6139.65031372442\\
0.972727272727273	4891.43264537791\\
0.981818181818182	4095.42358270008\\
0.990909090909091	3543.60884509611\\
1	3138.58481604639\\
};
\end{axis}

\begin{axis}[%
width=1.66in,
height=1.258in,
at={(1.011in,0.642in)},
scale only axis,
xmin=0,
xmax=1,
xlabel style={font=\color{white!15!black}},
xlabel={$\tau$},
ymin=0,
ymax=150000,
axis background/.style={fill=white},
title style={font=\bfseries},
title={$\gamma\text{= 0.70}$}
]
\addplot [color=mycolor1, line width=2.0pt, forget plot]
  table[row sep=crcr]{%
0.1	40.3918288952178\\
0.109090909090909	39.3014661629741\\
0.118181818181818	38.6246294925637\\
0.127272727272727	38.5511211771194\\
0.136363636363636	38.6726044646247\\
0.145454545454545	38.851294083472\\
0.154545454545455	39.0526540840834\\
0.163636363636364	39.2603886765821\\
0.172727272727273	48.5232499190902\\
0.181818181818182	124.021947337951\\
0.190909090909091	248.500940657272\\
0.2	116.8817460311\\
0.209090909090909	476.600384716939\\
0.218181818181818	300.767254264553\\
0.227272727272727	165.25996601381\\
0.236363636363636	126.208189229764\\
0.245454545454545	2636.0047145103\\
0.254545454545455	356.347724729274\\
0.263636363636364	1412.01965210928\\
0.272727272727273	349.895571240247\\
0.281818181818182	229.956051238772\\
0.290909090909091	183.583248275718\\
0.3	159.073026081091\\
0.309090909090909	144.025520366678\\
0.318181818181818	229.824356133225\\
0.327272727272727	493.836331071717\\
0.336363636363636	2580.85068862065\\
0.345454545454545	1183.93003662741\\
0.354545454545454	553.480837315288\\
0.363636363636364	388.223736570591\\
0.372727272727273	312.446688378934\\
0.381818181818182	269.26339519701\\
0.390909090909091	241.595138754419\\
0.4	222.528214102799\\
0.409090909090909	263.698397307688\\
0.418181818181818	342.809182533897\\
0.427272727272727	461.772345434919\\
0.436363636363636	660.77023467069\\
0.445454545454545	1061.09127268398\\
0.454545454545455	2278.3735634857\\
0.463636363636364	127072.273863951\\
0.472727272727273	2460.79001271324\\
0.481818181818182	1309.93861832543\\
0.490909090909091	922.119163787783\\
0.5	727.740450045053\\
0.509090909090909	611.188536496973\\
0.518181818181818	597.317814101038\\
0.527272727272727	722.013451980319\\
0.536363636363636	893.151476251476\\
0.545454545454546	1142.22299231384\\
0.554545454545455	1537.49659946084\\
0.563636363636364	2259.85322336118\\
0.572727272727273	3997.95746448689\\
0.581818181818182	13924.2021966879\\
0.590909090909091	10671.4727200728\\
0.6	4039.01205918097\\
0.609090909090909	2558.67664540566\\
0.618181818181818	1907.17320650538\\
0.627272727272727	1541.01787549769\\
0.636363636363636	1306.65986517505\\
0.645454545454545	1245.03963786354\\
0.654545454545455	1707.61070042913\\
0.663636363636364	2604.7955238476\\
0.672727272727273	5083.737071126\\
0.681818181818182	43866.6336021223\\
0.690909090909091	7199.21848238542\\
0.7	3455.81810222176\\
0.709090909090909	2328.46343774998\\
0.718181818181818	1785.72686825067\\
0.727272727272727	1466.99089147668\\
0.736363636363636	1257.65292894942\\
0.745454545454545	1109.90129324466\\
0.754545454545455	1000.24536596851\\
0.763636363636364	915.795541933876\\
0.772727272727273	848.889254728788\\
0.781818181818182	794.685125031272\\
0.790909090909091	749.973882096925\\
0.8	712.543913502427\\
0.809090909090909	680.82139726734\\
0.818181818181818	653.655702083042\\
0.827272727272727	630.185955946255\\
0.836363636363636	609.755092785123\\
0.845454545454545	591.852774828019\\
0.854545454545454	576.076477627126\\
0.863636363636364	562.104339055148\\
0.872727272727273	549.675826663815\\
0.881818181818182	559.363818543074\\
0.890909090909091	605.922485473728\\
0.9	657.144466963297\\
0.909090909090909	713.742480128067\\
0.918181818181818	776.582808086404\\
0.927272727272727	846.729009760904\\
0.936363636363636	925.501476142126\\
0.945454545454545	1014.55996137943\\
0.954545454545455	1116.0201350137\\
0.963636363636364	1232.62172040648\\
0.972727272727273	1367.97697516513\\
0.981818181818182	1526.9481746401\\
0.990909090909091	1716.23963401095\\
1	1945.36140345903\\
};
\end{axis}

\begin{axis}[%
width=1.66in,
height=1.258in,
at={(3.195in,0.642in)},
scale only axis,
xmin=0,
xmax=1,
xlabel style={font=\color{white!15!black}},
xlabel={$\tau$},
ymin=0,
ymax=60000,
axis background/.style={fill=white},
title style={font=\bfseries},
title={$\gamma\text{= 0.80}$}
]
\addplot [color=mycolor1, line width=2.0pt, forget plot]
  table[row sep=crcr]{%
0.1	42.0298276624711\\
0.109090909090909	40.9773915357312\\
0.118181818181818	39.9646622188558\\
0.127272727272727	39.0703842626676\\
0.136363636363636	38.5916831294402\\
0.145454545454545	38.5511211771194\\
0.154545454545455	38.6531529265696\\
0.163636363636364	38.8036572648731\\
0.172727272727273	38.9756813872027\\
0.181818181818182	39.1565082946559\\
0.190909090909091	40.1534523776248\\
0.2	53.4930108932154\\
0.209090909090909	217.78117834927\\
0.218181818181818	248.500940657273\\
0.227272727272727	123.458709604692\\
0.236363636363636	193.615184817558\\
0.245454545454545	544.262434622803\\
0.254545454545455	206.289649757684\\
0.263636363636364	146.679125572079\\
0.272727272727273	182.40474821274\\
0.281818181818182	910.546699508901\\
0.290909090909091	356.347724729264\\
0.3	2717.76593390783\\
0.309090909090909	417.014563672632\\
0.318181818181818	260.032057289986\\
0.327272727272727	202.562052186748\\
0.336363636363636	172.792706207861\\
0.345454545454545	154.667465248204\\
0.354545454545454	142.54972985452\\
0.363636363636364	229.824356133228\\
0.372727272727273	440.651800877749\\
0.381818181818182	1330.72551585582\\
0.390909090909091	2382.36726970475\\
0.4	738.689641885675\\
0.409090909090909	472.87647192647\\
0.418181818181818	364.424334775883\\
0.427272727272727	305.787413936342\\
0.436363636363636	269.263395197009\\
0.445454545454545	244.488974299575\\
0.454545454545455	226.70610249404\\
0.463636363636364	240.458746144196\\
0.472727272727273	299.638439252794\\
0.481818181818182	381.353153830066\\
0.490909090909091	501.502642479916\\
0.5	695.462760076101\\
0.509090909090909	1061.09127268402\\
0.518181818181818	2005.81232232409\\
0.527272727272727	10100.8938528511\\
0.536363636363636	3814.90898179039\\
0.545454545454546	1695.78257150446\\
0.554545454545455	1126.45080599784\\
0.563636363636364	862.323443894856\\
0.572727272727273	710.070399871832\\
0.581818181818182	611.188536496965\\
0.590909090909091	584.077080417029\\
0.6	687.301058794487\\
0.609090909090909	822.007175073482\\
0.618181818181818	1004.97829809339\\
0.627272727272727	1267.50047117348\\
0.636363636363636	1675.34870883373\\
0.645454545454545	2394.39532986903\\
0.654545454545455	3997.9574644868\\
0.663636363636364	10708.6816225952\\
0.672727272727273	18705.3661100776\\
0.681818181818182	5230.6884147221\\
0.690909090909091	3120.35050389857\\
0.7	2262.504138631\\
0.709090909090909	1797.62517044873\\
0.718181818181818	1506.2407525052\\
0.727272727272727	1306.65986517501\\
0.736363636363636	1202.53850291307\\
0.745454545454545	1566.26260299368\\
0.754545454545455	2184.46632660904\\
0.763636363636364	3465.33444396115\\
0.772727272727273	7692.7332197332\\
0.781818181818182	53380.1312135808\\
0.790909090909091	6316.81720684923\\
0.8	3455.81810222187\\
0.809090909090909	2424.27697069219\\
0.818181818181818	1893.14671581186\\
0.827272727272727	1569.76994546697\\
0.836363636363636	1352.47111858437\\
0.845454545454545	1196.6192700422\\
0.854545454545454	1079.54320313623\\
0.863636363636364	988.504702684299\\
0.872727272727273	915.795541933853\\
0.881818181818182	856.47771865152\\
0.890909090909091	807.241501994017\\
0.9	765.785330353847\\
0.909090909090909	730.459027232905\\
0.918181818181818	700.048506191639\\
0.927272727272727	673.64056141524\\
0.936363636363636	650.535017845692\\
0.945454545454545	630.18595594628\\
0.954545454545455	612.161373027729\\
0.963636363636364	596.11487092869\\
0.972727272727273	581.7653868092\\
0.981818181818182	568.882423680266\\
0.990909090909091	557.275116823571\\
1	546.7840236847\\
};
\end{axis}

\begin{axis}[%
width=1.66in,
height=1.258in,
at={(5.379in,0.642in)},
scale only axis,
xmin=0,
xmax=1,
xlabel style={font=\color{white!15!black}},
xlabel={$\tau$},
ymin=0,
ymax=46203.2178509148,
axis background/.style={fill=white},
title style={font=\bfseries},
title={$\gamma\text{= 0.90}$}
]
\addplot [color=mycolor1, line width=2.0pt, forget plot]
  table[row sep=crcr]{%
0.1	43.3716109019866\\
0.109090909090909	42.3883320378516\\
0.118181818181818	41.4411808020236\\
0.127272727272727	40.5206763349849\\
0.136363636363636	39.6445980658665\\
0.145454545454545	38.9141867076892\\
0.154545454545455	38.5726382369246\\
0.163636363636364	38.5511211771193\\
0.172727272727273	38.6386493268086\\
0.181818181818182	38.7677541562228\\
0.190909090909091	38.9168616608095\\
0.2	39.0756558647374\\
0.209090909090909	39.2373743505298\\
0.218181818181818	43.5382086819976\\
0.227272727272727	58.5308154014742\\
0.236363636363636	342.317098392473\\
0.245454545454545	248.500940657271\\
0.254545454545455	129.384611509686\\
0.263636363636364	96.6543791273824\\
0.272727272727273	3007.23177328143\\
0.281818181818182	270.266370755896\\
0.290909090909091	172.292142571569\\
0.3	135.945908034255\\
0.309090909090909	274.274705933091\\
0.318181818181818	617.800054602862\\
0.327272727272727	356.347724729284\\
0.336363636363636	11297.9295443458\\
0.345454545454545	497.794897874938\\
0.354545454545454	293.198862547451\\
0.363636363636364	222.870029425116\\
0.372727272727273	187.277787469715\\
0.381818181818182	165.838452005681\\
0.390909090909091	151.572230682743\\
0.4	141.450416196783\\
0.409090909090909	229.824356133223\\
0.418181818181818	405.340492878257\\
0.427272727272727	949.151383912985\\
0.436363636363636	16709.9157530135\\
0.445454545454545	1038.28639754244\\
0.454545454545455	584.632400066849\\
0.463636363636364	427.805590779589\\
0.472727272727273	348.544329121032\\
0.481818181818182	300.906223583569\\
0.490909090909091	269.263395197008\\
0.5	246.835541514618\\
0.509090909090909	230.203019933564\\
0.518181818181818	224.197137324582\\
0.527272727272727	271.155118794462\\
0.536363636363636	332.481397910093\\
0.545454545454546	415.975869516092\\
0.554545454545455	536.296045823901\\
0.563636363636364	724.62267639573\\
0.572727272727273	1061.091272684\\
0.581818181818182	1832.97268205563\\
0.590909090909091	5426.51992343502\\
0.6	6863.28600077516\\
0.609090909090909	2232.24089276392\\
0.618181818181818	1378.1831820586\\
0.627272727272727	1019.22870689447\\
0.636363636363636	821.806254358066\\
0.645454545454545	697.051867008489\\
0.654545454545455	611.188536496961\\
0.663636363636364	574.081565835292\\
0.672727272727273	662.065364325098\\
0.681818181818182	772.76953974509\\
0.690909090909091	916.172737684487\\
0.7	1109.09709548969\\
0.709090909090909	1382.28983650289\\
0.718181818181818	1798.59304888169\\
0.727272727272727	2509.60929557404\\
0.736363636363636	3997.95746448744\\
0.745454545454545	9066.65333913475\\
0.754545454545455	46203.2178509148\\
0.763636363636364	6837.97662439594\\
0.772727272727273	3787.75156754355\\
0.781818181818182	2663.77160427671\\
0.790909090909091	2079.59515946669\\
0.8	1721.8689611979\\
0.809090909090909	1480.45040390673\\
0.818181818181818	1306.65986517508\\
0.827272727272727	1175.65664645974\\
0.836363636363636	1470.01110312245\\
0.845454545454545	1935.41238134542\\
0.854545454545454	2759.27360462166\\
0.863636363636364	4611.9405247964\\
0.872727272727273	12601.0379666847\\
0.881818181818182	19795.0957145226\\
0.890909090909091	5771.03622330154\\
0.9	3455.81810222145\\
0.909090909090909	2505.08461896554\\
0.918181818181818	1987.60590853782\\
0.927272727272727	1662.49953546302\\
0.936363636363636	1439.53526666516\\
0.945454545454545	1277.28408339146\\
0.954545454545455	1154.0550220865\\
0.963636363636364	1057.39142318339\\
0.972727272727273	979.629529916741\\
0.981818181818182	915.79554193387\\
0.990909090909091	862.520572050021\\
1	817.441595781969\\
};
\end{axis}
\end{tikzpicture}%}}  \\
\subfloat[][$N=50$]
{\resizebox{0.45\textwidth}{!}{% This file was created by matlab2tikz.
%
%The latest updates can be retrieved from
%  http://www.mathworks.com/matlabcentral/fileexchange/22022-matlab2tikz-matlab2tikz
%where you can also make suggestions and rate matlab2tikz.
%
\definecolor{mycolor1}{rgb}{0.00000,0.44700,0.74100}%
%
\begin{tikzpicture}

\begin{axis}[%
width=1.66in,
height=1.258in,
at={(1.011in,4.137in)},
scale only axis,
xmin=0,
xmax=1,
xlabel style={font=\color{white!15!black}},
xlabel={$\tau$},
ymin=0,
ymax=8000000,
axis background/.style={fill=white},
title style={font=\bfseries},
title={$\gamma\text{= 0.10}$}
]
\addplot [color=mycolor1, line width=2.0pt, forget plot]
  table[row sep=crcr]{%
0.1	530.165790127878\\
0.109090909090909	394.708922140893\\
0.118181818181818	410.917745109258\\
0.127272727272727	495.579540149313\\
0.136363636363636	633.405051391706\\
0.145454545454545	932.998412585887\\
0.154545454545455	1531.51279418181\\
0.163636363636364	3107.83265943492\\
0.172727272727273	16750.5570989155\\
0.181818181818182	6722.64814996448\\
0.190909090909091	3172.01234725697\\
0.2	2395.20260995081\\
0.209090909090909	2928.42241319619\\
0.218181818181818	3607.95952049775\\
0.227272727272727	4496.78705449126\\
0.236363636363636	5700.05696646778\\
0.245454545454545	7407.51457222059\\
0.254545454545455	9999.98493175041\\
0.263636363636364	14370.8390079757\\
0.272727272727273	23225.5497835165\\
0.281818181818182	50403.1617307652\\
0.290909090909091	1867590.75158774\\
0.3	54160.0226450587\\
0.309090909090909	29176.2209516553\\
0.318181818181818	20727.5322046158\\
0.327272727272727	16498.8181418958\\
0.336363636363636	13972.1950061581\\
0.345454545454545	12300.5720823305\\
0.354545454545454	11118.9820457133\\
0.363636363636364	10244.3046775642\\
0.372727272727273	9574.55400231551\\
0.381818181818182	9048.42444007249\\
0.390909090909091	8626.84667503275\\
0.4	8283.74412239831\\
0.409090909090909	8284.08298628241\\
0.418181818181818	8889.83925003667\\
0.427272727272727	9540.8978385282\\
0.436363636363636	10241.7758524859\\
0.445454545454545	10997.6125591063\\
0.454545454545455	11814.2798539226\\
0.463636363636364	12698.5171769888\\
0.472727272727273	13658.0974389313\\
0.481818181818182	14702.0326003103\\
0.490909090909091	15840.8304151325\\
0.5	17086.8178331458\\
0.509090909090909	18454.5521607694\\
0.518181818181818	19961.3490747269\\
0.527272727272727	21627.9681540785\\
0.536363636363636	23479.5136161875\\
0.545454545454546	25546.6334181481\\
0.554545454545455	27867.1387593594\\
0.563636363636364	30488.2266049365\\
0.572727272727273	33469.5845113177\\
0.581818181818182	36887.815332622\\
0.590909090909091	40842.8863177308\\
0.6	45467.772353691\\
0.609090909090909	50943.3050906982\\
0.618181818181818	57521.8308951862\\
0.627272727272727	65566.4430345306\\
0.636363636363636	75619.2236290272\\
0.645454545454545	88527.0391548831\\
0.654545454545455	105690.780360761\\
0.663636363636364	129607.007165655\\
0.672727272727273	165199.494385946\\
0.681818181818182	223716.149392484\\
0.690909090909091	337684.804691724\\
0.7	656400.954337947\\
0.709090909090909	6554978.75359518\\
0.718181818181818	865891.150580968\\
0.727272727272727	416295.541625525\\
0.736363636363636	278372.077207891\\
0.745454545454545	211492.233491336\\
0.754545454545455	172032.873693797\\
0.763636363636364	146017.599087453\\
0.772727272727273	127589.067259963\\
0.781818181818182	113861.04656507\\
0.790909090909091	103246.571218595\\
0.8	94800.7988766513\\
0.809090909090909	87926.0723520262\\
0.818181818181818	82225.8683137056\\
0.827272727272727	77426.752083651\\
0.836363636363636	73334.0556383089\\
0.845454545454545	69805.420516042\\
0.854545454545454	66734.3335008031\\
0.863636363636364	64039.5125687552\\
0.872727272727273	61657.8534565336\\
0.881818181818182	59539.6171062441\\
0.890909090909091	57645.0691744322\\
0.9	55942.0849332238\\
0.909090909090909	54404.4107917533\\
0.918181818181818	53010.3815974164\\
0.927272727272727	51741.9601453385\\
0.936363636363636	50584.0082649391\\
0.945454545454545	49523.7268589738\\
0.954545454545455	48550.2209063158\\
0.963636363636364	47654.1580619252\\
0.972727272727273	46827.4981688894\\
0.981818181818182	46063.2770754902\\
0.990909090909091	45355.4324488832\\
1	44698.6623647057\\
};
\end{axis}

\begin{axis}[%
width=1.66in,
height=1.258in,
at={(3.195in,4.137in)},
scale only axis,
xmin=0,
xmax=1,
xlabel style={font=\color{white!15!black}},
xlabel={$\tau$},
ymin=0,
ymax=2000000,
axis background/.style={fill=white},
title style={font=\bfseries},
title={$\gamma\text{= 0.20}$}
]
\addplot [color=mycolor1, line width=2.0pt, forget plot]
  table[row sep=crcr]{%
0.1	21262.1612221055\\
0.109090909090909	584.110922638368\\
0.118181818181818	848.763868177615\\
0.127272727272727	422.649547158862\\
0.136363636363636	807.381205392668\\
0.145454545454545	2045.47798187314\\
0.154545454545455	670.23215256767\\
0.163636363636364	1801.44011416152\\
0.172727272727273	4357.63932477888\\
0.181818181818182	1125.89839410891\\
0.190909090909091	696.168891945296\\
0.2	530.165790127877\\
0.209090909090909	444.741035389557\\
0.218181818181818	394.708922140893\\
0.227272727272727	374.022655229122\\
0.236363636363636	410.917745109262\\
0.245454545454545	450.908706988198\\
0.254545454545455	495.579540149327\\
0.263636363636364	549.974588743057\\
0.272727272727273	633.405051391706\\
0.281818181818182	759.038175790885\\
0.290909090909091	932.998412585887\\
0.3	1176.02857676286\\
0.309090909090909	1531.51279418181\\
0.318181818181818	2093.85818098966\\
0.327272727272727	3107.83265943492\\
0.336363636363636	5459.66291798636\\
0.345454545454545	16750.5570989155\\
0.354545454545454	19833.7974610077\\
0.363636363636364	6722.64814996448\\
0.372727272727273	4225.79563894212\\
0.381818181818182	3172.01234725697\\
0.390909090909091	2593.29192024717\\
0.4	2395.20260995081\\
0.409090909090909	2646.41981460514\\
0.418181818181818	2928.42241319619\\
0.427272727272727	3246.66235522857\\
0.436363636363636	3607.95952049775\\
0.445454545454545	4020.95929020282\\
0.454545454545455	4496.78705449106\\
0.463636363636364	5050.00613958939\\
0.472727272727273	5700.05696646697\\
0.481818181818182	6473.48523723054\\
0.490909090909091	7407.51457222038\\
0.5	8556.01582323707\\
0.509090909090909	9999.98493174825\\
0.518181818181818	11867.0742276715\\
0.527272727272727	14370.8390079757\\
0.536363636363636	17897.6133703591\\
0.545454545454546	23225.5497835144\\
0.554545454545455	32190.0694185863\\
0.563636363636364	50403.1617307696\\
0.572727272727273	107256.198059996\\
0.581818181818182	1867590.75164999\\
0.590909090909091	102212.056064721\\
0.6	54160.0226450488\\
0.609090909090909	37572.2131973138\\
0.618181818181818	29176.2209516553\\
0.627272727272727	24111.3870255424\\
0.636363636363636	20727.5322046233\\
0.645454545454545	18309.954194091\\
0.654545454545455	16498.8181418958\\
0.663636363636364	15093.2269791593\\
0.672727272727273	13972.1950061628\\
0.681818181818182	13058.5084970627\\
0.690909090909091	12300.5720823305\\
0.7	11662.5977920411\\
0.709090909090909	11118.9820457106\\
0.718181818181818	10650.9234816976\\
0.727272727272727	10244.3046775642\\
0.736363636363636	9888.31904416215\\
0.745454545454545	9574.55400231551\\
0.754545454545455	9296.36283190384\\
0.763636363636364	9048.42444007249\\
0.772727272727273	8826.42857545002\\
0.781818181818182	8626.84667503275\\
0.790909090909091	8446.76234495597\\
0.8	8283.74412239881\\
0.809090909090909	8135.74870634685\\
0.818181818181818	8284.08298628174\\
0.827272727272727	8581.55894649937\\
0.836363636363636	8889.83925003667\\
0.845454545454545	9209.43686581806\\
0.854545454545454	9540.8978385282\\
0.863636363636364	9884.80398387819\\
0.872727272727273	10241.7758524882\\
0.881818181818182	10612.4759942509\\
0.890909090909091	10997.6125591063\\
0.9	11397.9432755726\\
0.909090909090909	11814.2798539226\\
0.918181818181818	12247.4928679492\\
0.927272727272727	12698.5171769882\\
0.936363636363636	13168.3579593284\\
0.945454545454545	13658.0974389298\\
0.954545454545455	14168.9024001338\\
0.963636363636364	14702.0326003103\\
0.972727272727273	15258.8502079334\\
0.981818181818182	15840.8304151278\\
0.990909090909091	16449.5733985922\\
1	17086.8178331458\\
};
\end{axis}

\begin{axis}[%
width=1.66in,
height=1.258in,
at={(5.379in,4.137in)},
scale only axis,
xmin=0,
xmax=1,
xlabel style={font=\color{white!15!black}},
xlabel={$\tau$},
ymin=0,
ymax=2000000,
axis background/.style={fill=white},
title style={font=\bfseries},
title={$\gamma\text{= 0.30}$}
]
\addplot [color=mycolor1, line width=2.0pt, forget plot]
  table[row sep=crcr]{%
0.1	135.283243204846\\
0.109090909090909	338.465126704616\\
0.118181818181818	13932.1715680333\\
0.127272727272727	812.754453454927\\
0.136363636363636	384.970942072093\\
0.145454545454545	1436.86454535469\\
0.154545454545455	1459.60636518019\\
0.163636363636364	584.110922638372\\
0.172727272727273	2138.36332690941\\
0.181818181818182	588.220668206784\\
0.190909090909091	422.649547158867\\
0.2	497.790575559698\\
0.209090909090909	1739.3080937752\\
0.218181818181818	2045.47798187316\\
0.227272727272727	783.69587118097\\
0.236363636363636	861.828328000931\\
0.245454545454545	1801.44011416151\\
0.254545454545455	58520.9388732151\\
0.263636363636364	2163.81804774401\\
0.272727272727273	1125.89839410885\\
0.281818181818182	790.39730535525\\
0.290909090909091	626.27514024567\\
0.3	530.16579012789\\
0.309090909090909	467.989080147927\\
0.318181818181818	425.238652634793\\
0.327272727272727	394.70892214089\\
0.336363636363636	372.431194812671\\
0.345454545454545	386.031244179533\\
0.354545454545454	410.917745109261\\
0.363636363636364	437.164864658639\\
0.372727272727273	465.153943041727\\
0.381818181818182	495.579540149323\\
0.390909090909091	530.010501207906\\
0.4	573.220250927514\\
0.409090909090909	633.405051391729\\
0.418181818181818	712.367052848097\\
0.427272727272727	810.961949905996\\
0.436363636363636	932.998412585888\\
0.445454545454545	1085.30878101325\\
0.454545454545455	1278.87696335701\\
0.463636363636364	1531.5127941818\\
0.472727272727273	1873.54108591457\\
0.481818181818182	2360.74098819348\\
0.490909090909091	3107.83265943486\\
0.5	4393.97543336626\\
0.509090909090909	7121.53894372754\\
0.518181818181818	16750.5570989218\\
0.527272727272727	66248.5799782314\\
0.536363636363636	11843.369445724\\
0.545454545454546	6722.64814996611\\
0.554545454545455	4797.782291091\\
0.563636363636364	3790.39616985328\\
0.572727272727273	3172.01234725687\\
0.581818181818182	2754.66655260296\\
0.590909090909091	2454.70310353618\\
0.6	2395.20260995074\\
0.609090909090909	2559.49678514863\\
0.618181818181818	2736.75496437796\\
0.627272727272727	2928.42241319601\\
0.636363636363636	3136.16751255403\\
0.645454545454545	3361.92651272094\\
0.654545454545455	3607.95952049737\\
0.663636363636364	3876.92114957708\\
0.672727272727273	4171.95051633034\\
0.681818181818182	4496.78705449152\\
0.690909090909091	4855.92122279413\\
0.7	5254.79301738362\\
0.709090909090909	5700.05696646865\\
0.718181818181818	6199.9411178562\\
0.727272727272727	6764.74135793799\\
0.736363636363636	7407.51457221988\\
0.745454545454545	8145.07066798127\\
0.754545454545455	8999.42543112574\\
0.763636363636364	9999.98493175009\\
0.772727272727273	11186.9305909467\\
0.781818181818182	12616.6524903358\\
0.790909090909091	14370.8390079735\\
0.8	16572.4560097154\\
0.809090909090909	19415.588815931\\
0.818181818181818	23225.5497835128\\
0.827272727272727	28592.3472528857\\
0.836363636363636	36708.6516447295\\
0.845454545454545	50403.1617307015\\
0.854545454545454	78409.9587014904\\
0.863636363636364	167602.6369598\\
0.872727272727273	1867590.75154776\\
0.881818181818182	147797.544920841\\
0.890909090909091	78504.7711728488\\
0.9	54160.0226450545\\
0.909090909090909	41745.539278918\\
0.918181818181818	34221.4133707602\\
0.927272727272727	29176.2209516346\\
0.936363636363636	25560.1578547165\\
0.945454545454545	22842.8686718042\\
0.954545454545455	20727.532204627\\
0.963636363636364	19035.0668740732\\
0.972727272727273	17651.0182312699\\
0.981818181818182	16498.8181418974\\
0.990909090909091	15525.3066627206\\
1	14692.4218267387\\
};
\end{axis}

\begin{axis}[%
width=1.66in,
height=1.258in,
at={(1.011in,2.39in)},
scale only axis,
xmin=0,
xmax=1,
xlabel style={font=\color{white!15!black}},
xlabel={$\tau$},
ymin=0,
ymax=400000,
axis background/.style={fill=white},
title style={font=\bfseries},
title={$\gamma\text{= 0.40}$}
]
\addplot [color=mycolor1, line width=2.0pt, forget plot]
  table[row sep=crcr]{%
0.1	513.31399463908\\
0.109090909090909	192.854163389579\\
0.118181818181818	104.46553325621\\
0.127272727272727	161.456005185078\\
0.136363636363636	153.684765493945\\
0.145454545454545	338.465126704609\\
0.154545454545455	1485.84926602389\\
0.163636363636364	6077.50073129946\\
0.172727272727273	602.077152364732\\
0.181818181818182	384.970942072095\\
0.190909090909091	898.108873636228\\
0.2	21262.1612221055\\
0.209090909090909	1030.69230047731\\
0.218181818181818	584.110922638368\\
0.227272727272727	44449.4423744144\\
0.236363636363636	848.763868177617\\
0.245454545454545	524.618847503078\\
0.254545454545455	422.649547158861\\
0.263636363636364	409.958955103577\\
0.272727272727273	807.381205392668\\
0.281818181818182	3591.38311618179\\
0.290909090909091	2045.4779818731\\
0.3	908.229606531004\\
0.309090909090909	670.23215256767\\
0.318181818181818	997.902078527709\\
0.327272727272727	1801.44011416152\\
0.336363636363636	6857.97590697443\\
0.345454545454545	4357.63932477888\\
0.354545454545454	1743.7920046143\\
0.363636363636364	1125.89839410891\\
0.372727272727273	850.824826751665\\
0.381818181818182	696.168891945296\\
0.390909090909091	597.755297051747\\
0.4	530.165790127878\\
0.409090909090909	481.319153017239\\
0.418181818181818	444.741035389557\\
0.427272727272727	416.653673723833\\
0.436363636363636	394.708922140893\\
0.445454545454545	377.374599601306\\
0.454545454545455	374.022655229121\\
0.463636363636364	392.139662056208\\
0.472727272727273	410.917745109258\\
0.481818181818182	430.458239280886\\
0.490909090909091	450.908706988196\\
0.5	472.493425816869\\
0.509090909090909	495.579540149313\\
0.518181818181818	520.862845631451\\
0.527272727272727	549.974588743057\\
0.536363636363636	586.47066706544\\
0.545454545454546	633.40505139171\\
0.554545454545455	690.872169047353\\
0.563636363636364	759.038175790878\\
0.572727272727273	839.062634371975\\
0.581818181818182	932.998412585872\\
0.590909090909091	1043.8813882615\\
0.6	1176.02857676283\\
0.609090909090909	1335.59068558848\\
0.618181818181818	1531.51279418181\\
0.627272727272727	1777.23549639346\\
0.636363636363636	2093.85818098951\\
0.645454545454545	2516.44657738272\\
0.654545454545455	3107.83265943492\\
0.663636363636364	3992.83403134598\\
0.672727272727273	5459.66291798692\\
0.681818181818182	8357.28409366458\\
0.690909090909091	16750.5570989155\\
0.7	325206.503951443\\
0.709090909090909	19833.7974610022\\
0.718181818181818	9903.97115776414\\
0.727272727272727	6722.64814996448\\
0.736363636363636	5156.63888644338\\
0.745454545454545	4225.79563894212\\
0.754545454545455	3609.55585944645\\
0.763636363636364	3172.01234725697\\
0.772727272727273	2845.69043165899\\
0.781818181818182	2593.29192024717\\
0.790909090909091	2392.52188644569\\
0.8	2395.20260995077\\
0.809090909090909	2517.25996520689\\
0.818181818181818	2646.41981460518\\
0.827272727272727	2783.25990938097\\
0.836363636363636	2928.42241319619\\
0.845454545454545	3082.62312811526\\
0.854545454545454	3246.66235522857\\
0.863636363636364	3421.43773760018\\
0.872727272727273	3607.95952049815\\
0.881818181818182	3807.36877656266\\
0.890909090909091	4020.95929020282\\
0.9	4250.20398777534\\
0.909090909090909	4496.78705449106\\
0.918181818181818	4762.64321843362\\
0.927272727272727	5050.0061395891\\
0.936363636363636	5361.46846471443\\
0.945454545454545	5700.05696646778\\
0.954545454545455	6069.32738011591\\
0.963636363636364	6473.48523723054\\
0.972727272727273	6917.54140850811\\
0.981818181818182	7407.51457222059\\
0.990909090909091	7950.69799925866\\
1	8556.01582323707\\
};
\end{axis}

\begin{axis}[%
width=1.66in,
height=1.258in,
at={(3.195in,2.39in)},
scale only axis,
xmin=0,
xmax=1,
xlabel style={font=\color{white!15!black}},
xlabel={$\tau$},
ymin=0,
ymax=80000,
axis background/.style={fill=white},
title style={font=\bfseries},
title={$\gamma\text{= 0.50}$}
]
\addplot [color=mycolor1, line width=2.0pt, forget plot]
  table[row sep=crcr]{%
0.1	49.5241905993154\\
0.109090909090909	69.8054034886481\\
0.118181818181818	327.721414363849\\
0.127272727272727	314.815732437579\\
0.136363636363636	192.854163389582\\
0.145454545454545	95.9050849426643\\
0.154545454545455	204.384963244246\\
0.163636363636364	143.018385065431\\
0.172727272727273	177.461282764316\\
0.181818181818182	338.465126704616\\
0.190909090909091	938.295934770467\\
0.2	2719.06066757791\\
0.209090909090909	1187.49075753271\\
0.218181818181818	527.565375383633\\
0.227272727272727	384.97094207209\\
0.236363636363636	724.306798631899\\
0.245454545454545	2597.97906659935\\
0.254545454545455	2265.97348638716\\
0.263636363636364	883.768594711244\\
0.272727272727273	584.110922638361\\
0.281818181818182	3587.29892586543\\
0.290909090909091	1285.05301348165\\
0.3	662.024529755166\\
0.309090909090909	496.179379464832\\
0.318181818181818	422.649547158868\\
0.327272727272727	382.406775200729\\
0.336363636363636	592.798616365379\\
0.345454545454545	1206.87411666444\\
0.354545454545454	9018.21990646893\\
0.363636363636364	2045.47798187316\\
0.372727272727273	1010.6252660578\\
0.381818181818182	710.903097678421\\
0.390909090909091	774.903105161236\\
0.4	1099.5481547045\\
0.409090909090909	1801.4401141615\\
0.418181818181818	4447.32794577448\\
0.427272727272727	12011.5108824453\\
0.436363636363636	2697.54958023413\\
0.445454545454545	1565.70651266501\\
0.454545454545455	1125.89839410886\\
0.463636363636364	892.832607833526\\
0.472727272727273	749.049958514059\\
0.481818181818182	651.931128897218\\
0.490909090909091	582.277588398871\\
0.5	530.165790127885\\
0.509090909090909	489.950135812695\\
0.518181818181818	458.18317934477\\
0.527272727272727	432.642225613495\\
0.536363636363636	411.83146914832\\
0.545454545454546	394.708922140889\\
0.554545454545455	380.527864238557\\
0.563636363636364	368.740278230561\\
0.572727272727273	381.195247918796\\
0.581818181818182	395.839655313005\\
0.590909090909091	410.917745109256\\
0.6	426.482916647665\\
0.609090909090909	442.607199619103\\
0.618181818181818	459.390223671739\\
0.627272727272727	476.974939858831\\
0.636363636363636	495.579540149328\\
0.645454545454545	515.574622862109\\
0.654545454545455	537.689637764405\\
0.663636363636364	563.439469473495\\
0.672727272727273	594.994081505541\\
0.681818181818182	633.405051391726\\
0.690909090909091	678.543397101344\\
0.7	730.434094507555\\
0.709090909090909	789.529961827607\\
0.718181818181818	856.655164558445\\
0.727272727272727	932.998412585889\\
0.736363636363636	1020.16378580372\\
0.745454545454545	1120.27416444811\\
0.754545454545455	1236.13906419908\\
0.763636363636364	1371.51526790936\\
0.772727272727273	1531.51279418178\\
0.781818181818182	1723.24171861267\\
0.790909090909091	1956.87873227089\\
0.8	2247.50522076552\\
0.809090909090909	2618.45370989128\\
0.818181818181818	3107.83265943502\\
0.827272727272727	3782.41115268081\\
0.836363636363636	4770.78892832467\\
0.845454545454545	6356.51372423401\\
0.854545454545454	9312.02401014869\\
0.863636363636364	16750.5570989247\\
0.872727272727273	70705.3818381043\\
0.881818181818182	34031.2169490141\\
0.890909090909091	14087.9985851574\\
0.9	9029.99306080406\\
0.909090909090909	6722.64814996621\\
0.918181818181818	5402.74182005373\\
0.927272727272727	4548.74994149416\\
0.936363636363636	3951.43679192612\\
0.945454545454545	3510.54321929028\\
0.954545454545455	3172.01234725686\\
0.963636363636364	2904.12482442917\\
0.972727272727273	2687.04119509681\\
0.981818181818182	2507.71632632043\\
0.990909090909091	2357.22025318278\\
1	2395.20260995094\\
};
\end{axis}

\begin{axis}[%
width=1.66in,
height=1.258in,
at={(5.379in,2.39in)},
scale only axis,
xmin=0,
xmax=1,
xlabel style={font=\color{white!15!black}},
xlabel={$\tau$},
ymin=0,
ymax=304385.814443959,
axis background/.style={fill=white},
title style={font=\bfseries},
title={$\gamma\text{= 0.60}$}
]
\addplot [color=mycolor1, line width=2.0pt, forget plot]
  table[row sep=crcr]{%
0.1	48.155051529529\\
0.109090909090909	48.7301759668384\\
0.118181818181818	49.3883692891343\\
0.127272727272727	50.0661589802184\\
0.136363636363636	164.530451226118\\
0.145454545454545	2115.58020370694\\
0.154545454545455	252.666113271258\\
0.163636363636364	192.854163389579\\
0.172727272727273	103.600832100874\\
0.181818181818182	311.51041801655\\
0.190909090909091	161.456005185077\\
0.2	135.283243204845\\
0.209090909090909	195.754043448525\\
0.218181818181818	338.465126704611\\
0.227272727272727	744.724660340862\\
0.236363636363636	13932.1715680357\\
0.245454545454545	6077.50073129809\\
0.254545454545455	812.754453454935\\
0.263636363636364	489.345855014812\\
0.272727272727273	384.970942072091\\
0.281818181818182	638.461974715515\\
0.290909090909091	1436.86454535475\\
0.3	21262.1612220971\\
0.309090909090909	1459.60636518024\\
0.318181818181818	809.572824125265\\
0.327272727272727	584.110922638374\\
0.336363636363636	2018.86268327926\\
0.345454545454545	2138.36332690933\\
0.354545454545454	848.763868177603\\
0.363636363636364	588.220668206794\\
0.372727272727273	480.105167642186\\
0.381818181818182	422.649547158867\\
0.390909090909091	387.758313299394\\
0.4	497.790575559697\\
0.409090909090909	807.381205392684\\
0.418181818181818	1739.30809377525\\
0.427272727272727	304385.814443959\\
0.436363636363636	2045.47798187312\\
0.445454545454545	1096.20405554843\\
0.454545454545455	783.695871180958\\
0.463636363636364	670.232152567676\\
0.472727272727273	861.828328000876\\
0.481818181818182	1178.36655175561\\
0.490909090909091	1801.4401141614\\
0.5	3593.0809029362\\
0.509090909090909	58520.9388733512\\
0.518181818181818	4357.63932477857\\
0.527272727272727	2163.81804774385\\
0.536363636363636	1467.28477887815\\
0.545454545454546	1125.89839410888\\
0.554545454545455	923.719192029611\\
0.563636363636364	790.397305355277\\
0.572727272727273	696.168891945306\\
0.581818181818182	626.275140245695\\
0.590909090909091	572.564128855961\\
0.6	530.16579012788\\
0.609090909090909	495.991685043223\\
0.618181818181818	467.989080147922\\
0.627272727272727	444.741035389548\\
0.636363636363636	425.238652634798\\
0.645454545454545	408.744771572958\\
0.654545454545455	394.708922140892\\
0.663636363636364	382.712325404298\\
0.672727272727273	372.431194812672\\
0.681818181818182	374.022655229125\\
0.690909090909091	386.031244179536\\
0.7	398.321296574392\\
0.709090909090909	410.917745109257\\
0.718181818181818	423.85212147691\\
0.727272727272727	437.164864658649\\
0.736363636363636	450.908706988199\\
0.745454545454545	465.153943041736\\
0.754545454545455	479.997392228413\\
0.763636363636364	495.579540149325\\
0.772727272727273	512.121523149955\\
0.781818181818182	530.010501207916\\
0.790909090909091	549.974588743064\\
0.8	573.220250927531\\
0.809090909090909	600.918144747834\\
0.818181818181818	633.405051391724\\
0.827272727272727	670.557707041868\\
0.836363636363636	712.367052848093\\
0.845454545454545	759.038175790889\\
0.854545454545454	810.961949906039\\
0.863636363636364	868.701134911777\\
0.872727272727273	932.998412585898\\
0.881818181818182	1004.8020023304\\
0.890909090909091	1085.30878101324\\
0.9	1176.02857676292\\
0.909090909090909	1278.87696335695\\
0.918181818181818	1396.30883055338\\
0.927272727272727	1531.51279418174\\
0.936363636363636	1688.69963049054\\
0.945454545454545	1873.54108591454\\
0.954545454545455	2093.85818098988\\
0.963636363636364	2360.74098819358\\
0.972727272727273	2690.45155340592\\
0.981818181818182	3107.83265943467\\
0.990909090909091	3652.8233142159\\
1	4393.97543336643\\
};
\end{axis}

\begin{axis}[%
width=1.66in,
height=1.258in,
at={(1.011in,0.642in)},
scale only axis,
xmin=0,
xmax=1,
xlabel style={font=\color{white!15!black}},
xlabel={$\tau$},
ymin=0,
ymax=15000,
axis background/.style={fill=white},
title style={font=\bfseries},
title={$\gamma\text{= 0.70}$}
]
\addplot [color=mycolor1, line width=2.0pt, forget plot]
  table[row sep=crcr]{%
0.1	47.8288156274786\\
0.109090909090909	47.8613478849261\\
0.118181818181818	48.2286904221288\\
0.127272727272727	48.7301759668385\\
0.136363636363636	49.2918212482873\\
0.145454545454545	49.8739513604904\\
0.154545454545455	90.7856680669649\\
0.163636363636364	224.076211480228\\
0.172727272727273	1014.38356270909\\
0.181818181818182	221.319295268082\\
0.190909090909091	192.854163389581\\
0.2	110.638262727944\\
0.209090909090909	2622.26492960139\\
0.218181818181818	186.873100267495\\
0.227272727272727	147.199105238733\\
0.236363636363636	138.857053494522\\
0.245454545454545	210.289702348676\\
0.254545454545455	338.465126704617\\
0.263636363636364	645.675371338866\\
0.272727272727273	2452.75626074522\\
0.281818181818182	2444.00184463345\\
0.290909090909091	1512.05117018713\\
0.3	674.309201307221\\
0.309090909090909	466.067598384817\\
0.318181818181818	384.970942072094\\
0.327272727272727	587.288100731543\\
0.336363636363636	1074.76904748947\\
0.345454545454545	3884.78047442802\\
0.354545454545454	3009.55865043486\\
0.363636363636364	1175.63823412476\\
0.372727272727273	764.823883165707\\
0.381818181818182	584.110922638362\\
0.390909090909091	1519.18269632857\\
0.4	4513.56104241147\\
0.409090909090909	1111.00862643618\\
0.418181818181818	703.33710186178\\
0.427272727272727	548.936853935457\\
0.436363636363636	469.784110885055\\
0.445454545454545	422.649547158864\\
0.454545454545455	391.85713178005\\
0.463636363636364	444.269331702364\\
0.472727272727273	643.211503006105\\
0.481818181818182	1061.56138193702\\
0.490909090909091	2482.55924880066\\
0.5	13961.7295015708\\
0.509090909090909	2045.47798187319\\
0.518181818181818	1168.75800461102\\
0.527272727272727	849.138724063151\\
0.536363636363636	684.813442376958\\
0.545454545454546	742.236964404096\\
0.554545454545455	935.177000834633\\
0.563636363636364	1241.27099661694\\
0.572727272727273	1801.44011416151\\
0.581818181818182	3155.79007744355\\
0.590909090909091	11087.5047130863\\
0.6	7964.80416494664\\
0.609090909090909	3022.43470199197\\
0.618181818181818	1900.57179356467\\
0.627272727272727	1404.84934250797\\
0.636363636363636	1125.89839410889\\
0.645454545454545	947.381487137565\\
0.654545454545455	823.59356661541\\
0.663636363636364	732.920013295033\\
0.672727272727273	663.811429261545\\
0.681818181818182	609.535224073991\\
0.690909090909091	565.901385761759\\
0.7	530.165790127883\\
0.709090909090909	500.456225492308\\
0.718181818181818	475.45184437764\\
0.727272727272727	454.194432589193\\
0.736363636363636	435.972298787832\\
0.745454545454545	420.246136508523\\
0.754545454545455	406.600149682876\\
0.763636363636364	394.708922140891\\
0.772727272727273	384.314397795818\\
0.781818181818182	375.209527310075\\
0.790909090909091	368.957186503762\\
0.8	379.136107176358\\
0.809090909090909	389.512981617033\\
0.818181818181818	400.10137523553\\
0.827272727272727	410.917745109261\\
0.836363636363636	421.982245623813\\
0.845454545454545	433.319814387012\\
0.854545454545454	444.961676062739\\
0.863636363636364	456.947510685439\\
0.872727272727273	469.328768528909\\
0.881818181818182	482.174151012755\\
0.890909090909091	495.579540149331\\
0.9	509.68759280975\\
0.909090909090909	524.72822847494\\
0.918181818181818	541.097025557305\\
0.927272727272727	559.453843660979\\
0.936363636363636	580.6475609521\\
0.945454545454545	605.26780054422\\
0.954545454545455	633.405051391733\\
0.963636363636364	664.967295729242\\
0.972727272727273	699.938116243381\\
0.981818181818182	738.419848248123\\
0.990909090909091	780.619162772586\\
1	826.836499021323\\
};
\end{axis}

\begin{axis}[%
width=1.66in,
height=1.258in,
at={(3.195in,0.642in)},
scale only axis,
xmin=0,
xmax=1,
xlabel style={font=\color{white!15!black}},
xlabel={$\tau$},
ymin=0,
ymax=44449.4423744456,
axis background/.style={fill=white},
title style={font=\bfseries},
title={$\gamma\text{= 0.80}$}
]
\addplot [color=mycolor1, line width=2.0pt, forget plot]
  table[row sep=crcr]{%
0.1	48.7114533037037\\
0.109090909090909	48.0317911133312\\
0.118181818181818	47.7805714694073\\
0.127272727272727	47.9331445335712\\
0.136363636363636	48.2862201082278\\
0.145454545454545	48.7301759668386\\
0.154545454545455	49.2197752792995\\
0.163636363636364	49.7284500649055\\
0.172727272727273	59.2878520339608\\
0.181818181818182	164.530451226112\\
0.190909090909091	502.807721020997\\
0.2	513.31399463908\\
0.209090909090909	201.994552056676\\
0.218181818181818	192.85416338958\\
0.227272727272727	116.757593555356\\
0.236363636363636	104.465533256207\\
0.245454545454545	227.556864339503\\
0.254545454545455	161.456005185079\\
0.263636363636364	139.832896760649\\
0.272727272727273	153.684765493945\\
0.281818181818182	222.126687327924\\
0.290909090909091	338.46512670461\\
0.3	585.48487138316\\
0.309090909090909	1485.84926602389\\
0.318181818181818	4833.33577009737\\
0.327272727272727	6077.50073129946\\
0.336363636363636	1007.14832924515\\
0.345454545454545	602.077152364732\\
0.354545454545454	450.391891900038\\
0.363636363636364	384.970942072095\\
0.372727272727273	553.309031181561\\
0.381818181818182	898.108873636228\\
0.390909090909091	2002.85453758444\\
0.4	21262.1612221055\\
0.409090909090909	1871.09084730251\\
0.418181818181818	1030.69230047731\\
0.427272727272727	734.895743593878\\
0.436363636363636	584.110922638368\\
0.445454545454545	1273.62980388745\\
0.454545454545455	44449.4423744456\\
0.463636363636364	1501.1830688611\\
0.472727272727273	848.763868177615\\
0.481818181818182	631.167397372086\\
0.490909090909091	524.618847503078\\
0.5	462.600442027748\\
0.509090909090909	422.649547158862\\
0.518181818181818	395.095985988802\\
0.527272727272727	409.958955103577\\
0.536363636363636	553.892030933529\\
0.545454545454546	807.381205392676\\
0.554545454545455	1366.69393401541\\
0.563636363636364	3591.38311618098\\
0.572727272727273	7888.67747307426\\
0.581818181818182	2045.47798187296\\
0.590909090909091	1231.03466689471\\
0.6	908.229606530971\\
0.609090909090909	736.051501211089\\
0.618181818181818	670.23215256767\\
0.627272727272727	805.612865834417\\
0.636363636363636	997.902078527708\\
0.645454545454545	1292.63916013349\\
0.654545454545455	1801.44011416152\\
0.663636363636364	2890.0742008553\\
0.672727272727273	6857.97590697442\\
0.681818181818182	21788.0489339737\\
0.690909090909091	4357.63932477888\\
0.7	2467.62615846388\\
0.709090909090909	1743.79200461434\\
0.718181818181818	1361.71635090427\\
0.727272727272727	1125.89839410891\\
0.736363636363636	966.087459301586\\
0.745454545454545	850.824826751665\\
0.754545454545455	763.914825544843\\
0.763636363636364	696.168891945296\\
0.772727272727273	641.985516044948\\
0.781818181818182	597.755297051747\\
0.790909090909091	561.047616843672\\
0.8	530.165790127891\\
0.809090909090909	503.889597915047\\
0.818181818181818	481.319153017237\\
0.827272727272727	461.776427844623\\
0.836363636363636	444.741035389557\\
0.845454545454545	429.807119410403\\
0.854545454545454	416.653673723833\\
0.863636363636364	405.02364498774\\
0.872727272727273	394.70892214089\\
0.881818181818182	385.539357028149\\
0.890909090909091	377.374599601306\\
0.9	370.097934761012\\
0.909090909090909	374.022655229121\\
0.918181818181818	383.003564151192\\
0.927272727272727	392.139662056212\\
0.936363636363636	401.440580035485\\
0.945454545454545	410.917745109262\\
0.954545454545455	420.584814328902\\
0.963636363636364	430.458239280886\\
0.972727272727273	440.558015499225\\
0.981818181818182	450.908706988198\\
0.990909090909091	461.540905525113\\
1	472.493425816869\\
};
\end{axis}

\begin{axis}[%
width=1.66in,
height=1.258in,
at={(5.379in,0.642in)},
scale only axis,
xmin=0,
xmax=1,
xlabel style={font=\color{white!15!black}},
xlabel={$\tau$},
ymin=0,
ymax=60000,
axis background/.style={fill=white},
title style={font=\bfseries},
title={$\gamma\text{= 0.90}$}
]
\addplot [color=mycolor1, line width=2.0pt, forget plot]
  table[row sep=crcr]{%
0.1	49.9160334511765\\
0.109090909090909	49.0055266235201\\
0.118181818181818	48.2906955286791\\
0.127272727272727	47.8601152359264\\
0.136363636363636	47.797297720229\\
0.145454545454545	47.9998275641009\\
0.154545454545455	48.3321922805394\\
0.163636363636364	48.7301759668383\\
0.172727272727273	49.1640037204289\\
0.181818181818182	49.6149658447483\\
0.190909090909091	50.0661589802185\\
0.2	113.839945322316\\
0.209090909090909	186.24281016312\\
0.218181818181818	2115.58020370715\\
0.227272727272727	378.337348391102\\
0.236363636363636	188.699562361258\\
0.245454545454545	192.854163389582\\
0.254545454545455	122.001825972947\\
0.263636363636364	92.0139765307597\\
0.272727272727273	311.510418016554\\
0.281818181818182	179.554117529617\\
0.290909090909091	149.827379444312\\
0.3	135.283243204845\\
0.309090909090909	166.406992798097\\
0.318181818181818	231.95629972172\\
0.327272727272727	338.465126704619\\
0.336363636363636	545.029371343727\\
0.345454545454545	1126.14722099539\\
0.354545454545454	13932.1715680448\\
0.363636363636364	3648.12516152408\\
0.372727272727273	1796.12459918501\\
0.381818181818182	812.754453454903\\
0.390909090909091	557.662945442897\\
0.4	439.113220010015\\
0.409090909090909	384.970942072093\\
0.418181818181818	529.104384660045\\
0.427272727272727	793.487672582073\\
0.436363636363636	1436.8645453548\\
0.445454545454545	5318.48913347461\\
0.454545454545455	3697.22193669106\\
0.463636363636364	1459.60636518015\\
0.472727272727273	942.774282899628\\
0.481818181818182	713.471974998575\\
0.490909090909091	584.110922638366\\
0.5	1127.69806427038\\
0.509090909090909	7067.58223804446\\
0.518181818181818	2138.36332690922\\
0.527272727272727	1036.17238408458\\
0.536363636363636	729.702810348839\\
0.545454545454546	588.220668206786\\
0.554545454545455	508.109613709714\\
0.563636363636364	457.313966347428\\
0.572727272727273	422.64954715887\\
0.581818181818182	397.719345388583\\
0.590909090909091	386.101727255905\\
0.6	497.790575559692\\
0.609090909090909	674.448568844479\\
0.618181818181818	993.754879796934\\
0.627272727272727	1739.30809377512\\
0.636363636363636	5422.11212897349\\
0.645454545454545	5915.67967434101\\
0.654545454545455	2045.47798187324\\
0.663636363636364	1285.06570540263\\
0.672727272727273	961.819439672117\\
0.681818181818182	783.695871180973\\
0.690909090909091	671.402088699568\\
0.7	725.110605081873\\
0.709090909090909	861.828328000884\\
0.718181818181818	1052.1565988077\\
0.727272727272727	1335.37793010144\\
0.736363636363636	1801.44011416156\\
0.745454545454545	2711.51782560804\\
0.754545454545455	5275.46339821519\\
0.763636363636364	58520.9388733984\\
0.772727272727273	6717.21498374407\\
0.781818181818182	3240.97876101195\\
0.790909090909091	2163.81804774396\\
0.8	1639.76566444476\\
0.809090909090909	1330.13394702997\\
0.818181818181818	1125.89839410893\\
0.827272727272727	981.246143524955\\
0.836363636363636	873.562648364795\\
0.845454545454545	790.397305355272\\
0.854545454545454	724.329680312856\\
0.863636363636364	670.66198632243\\
0.872727272727273	626.275140245673\\
0.881818181818182	589.016936047062\\
0.890909090909091	557.354336672994\\
0.9	530.165790127884\\
0.909090909090909	506.611918010716\\
0.918181818181818	486.052111457666\\
0.927272727272727	467.989080147925\\
0.936363636363636	452.031000318343\\
0.945454545454545	437.865070379678\\
0.954545454545455	425.238652634794\\
0.963636363636364	413.945575427183\\
0.972727272727273	403.81601720852\\
0.981818181818182	394.708922140888\\
0.990909090909091	386.506234183997\\
1	379.108457045588\\
};
\end{axis}
\end{tikzpicture}%}}
 \hfill 
\subfloat[][$N=60$]
{\resizebox{0.45\textwidth}{!}{ % This file was created by matlab2tikz.
%
%The latest updates can be retrieved from
%  http://www.mathworks.com/matlabcentral/fileexchange/22022-matlab2tikz-matlab2tikz
%where you can also make suggestions and rate matlab2tikz.
%
\definecolor{mycolor1}{rgb}{0.00000,0.44700,0.74100}%
%
\begin{tikzpicture}

\begin{axis}[%
width=1.66in,
height=1.258in,
at={(1.011in,4.137in)},
scale only axis,
xmin=0,
xmax=1,
xlabel style={font=\color{white!15!black}},
xlabel={$\tau$},
ymin=0,
ymax=600000,
axis background/.style={fill=white},
title style={font=\bfseries},
title={$\gamma\text{= 0.10}$}
]
\addplot [color=mycolor1, line width=2.0pt, forget plot]
  table[row sep=crcr]{%
0.1	1533.30461839689\\
0.109090909090909	1145.1568388197\\
0.118181818181818	1185.31325396078\\
0.127272727272727	1865.77543561673\\
0.136363636363636	3710.61269416631\\
0.145454545454545	31054.6602876426\\
0.154545454545455	6206.23132664991\\
0.163636363636364	14171.3233508616\\
0.172727272727273	42148.2689239062\\
0.181818181818182	10237.5784187339\\
0.190909090909091	6382.55160553104\\
0.2	4896.60016149608\\
0.209090909090909	4118.94257343421\\
0.218181818181818	3647.46498443786\\
0.227272727272727	3336.14952421122\\
0.236363636363636	3119.15716269035\\
0.245454545454545	3120.96606272913\\
0.254545454545455	3589.19230931657\\
0.263636363636364	4129.93610042813\\
0.272727272727273	4758.40624272092\\
0.281818181818182	5494.38959248626\\
0.290909090909091	6364.11122604428\\
0.3	7403.08304848102\\
0.309090909090909	8660.61254401716\\
0.318181818181818	10207.2214478448\\
0.327272727272727	12147.4291733018\\
0.336363636363636	14643.0470694991\\
0.345454545454545	17958.6670425678\\
0.354545454545454	22558.672375298\\
0.363636363636364	29339.6954190683\\
0.372727272727273	40286.0629500757\\
0.381818181818182	60837.057375728\\
0.390909090909091	113202.15804625\\
0.4	518337.487995315\\
0.409090909090909	229967.639206113\\
0.418181818181818	99375.7009757209\\
0.427272727272727	65488.2349488751\\
0.436363636363636	49951.931091519\\
0.445454545454545	41059.8361628218\\
0.454545454545455	35316.6120000724\\
0.463636363636364	31312.870258541\\
0.472727272727273	28371.1222510094\\
0.481818181818182	26125.2741408781\\
0.490909090909091	24360.1576636471\\
0.5	22941.0214987642\\
0.509090909090909	21779.1727824939\\
0.518181818181818	20813.8476144732\\
0.527272727272727	20002.0199635767\\
0.536363636363636	19312.3694595981\\
0.545454545454546	18721.5541011098\\
0.554545454545455	18211.8218991922\\
0.563636363636364	17769.4320130715\\
0.572727272727273	17383.5824019156\\
0.581818181818182	17045.6640365033\\
0.590909090909091	16748.7312539662\\
0.6	16487.1185403236\\
0.609090909090909	16256.1585923442\\
0.618181818181818	16051.97174812\\
0.627272727272727	15871.3065626607\\
0.636363636363636	15711.4175993871\\
0.645454545454545	15569.9706814556\\
0.654545454545455	15444.9686643295\\
0.663636363636364	15334.6927248253\\
0.672727272727273	15237.6555099406\\
0.681818181818182	15152.5634413783\\
0.690909090909091	15078.2861543466\\
0.7	15013.8315434607\\
0.709090909090909	14958.3252510799\\
0.718181818181818	14910.993702135\\
0.727272727272727	14871.1499899957\\
0.736363636363636	14838.1820696606\\
0.745454545454545	14811.5428298928\\
0.754545454545455	14790.7417044203\\
0.763636363636364	14775.337550936\\
0.772727272727273	14764.9325797106\\
0.781818181818182	14759.1671558559\\
0.790909090909091	14757.7153317887\\
0.8	14760.280993109\\
0.809090909090909	14766.5945217546\\
0.818181818181818	14776.4098972651\\
0.827272727272727	14789.5021703761\\
0.836363636363636	14805.66525447\\
0.845454545454545	14824.709989053\\
0.854545454545454	14846.4624370163\\
0.863636363636364	14870.7623834076\\
0.872727272727273	14897.4620084188\\
0.881818181818182	14926.4247114046\\
0.890909090909091	14957.5240663685\\
0.9	14990.6428919377\\
0.909090909090909	15025.672421538\\
0.918181818181818	15062.5115614289\\
0.927272727272727	15101.0662257976\\
0.936363636363636	15141.2487398779\\
0.945454545454545	15182.9773030645\\
0.954545454545455	15226.1755049944\\
0.963636363636364	15270.7718888459\\
0.972727272727273	15316.6995562719\\
0.981818181818182	15363.8958096225\\
0.990909090909091	15412.3018273502\\
1	15461.8623690063\\
};
\end{axis}

\begin{axis}[%
width=1.66in,
height=1.258in,
at={(3.195in,4.137in)},
scale only axis,
xmin=0,
xmax=1,
xlabel style={font=\color{white!15!black}},
xlabel={$\tau$},
ymin=0,
ymax=800000,
axis background/.style={fill=white},
title style={font=\bfseries},
title={$\gamma\text{= 0.20}$}
]
\addplot [color=mycolor1, line width=2.0pt, forget plot]
  table[row sep=crcr]{%
0.1	1029.93407711286\\
0.109090909090909	550.084912579875\\
0.118181818181818	2059.02190648627\\
0.127272727272727	2533.17258247201\\
0.136363636363636	999.106522524335\\
0.145454545454545	893.160530720598\\
0.154545454545455	1645.30616111023\\
0.163636363636364	4645.80179609603\\
0.172727272727273	10573.518854012\\
0.181818181818182	3699.68884623954\\
0.190909090909091	2015.76098107928\\
0.2	1533.30461839689\\
0.209090909090909	1290.25033343667\\
0.218181818181818	1145.1568388197\\
0.227272727272727	1050.3845572743\\
0.236363636363636	1185.31325396072\\
0.245454545454545	1465.13400386835\\
0.254545454545455	1865.77543561679\\
0.263636363636364	2504.28572651655\\
0.272727272727273	3710.61269416631\\
0.281818181818182	6832.65537584696\\
0.290909090909091	31054.6602876426\\
0.3	14164.8439579936\\
0.309090909090909	6206.23132664993\\
0.318181818181818	7812.76825884172\\
0.327272727272727	14171.3233508616\\
0.336363636363636	48957.1000311053\\
0.345454545454545	42148.2689239062\\
0.354545454545454	15911.0738992764\\
0.363636363636364	10237.5784187339\\
0.372727272727273	7764.1113433789\\
0.381818181818182	6382.55160553104\\
0.390909090909091	5503.29891119291\\
0.4	4896.60016149608\\
0.409090909090909	4454.34066325858\\
0.418181818181818	4118.9425734342\\
0.427272727272727	3856.91935971776\\
0.436363636363636	3647.46498443786\\
0.445454545454545	3476.96842228204\\
0.454545454545455	3336.14952421092\\
0.463636363636364	3218.46246517459\\
0.472727272727273	3119.15716269045\\
0.481818181818182	3034.70259546703\\
0.490909090909091	3120.96606272898\\
0.5	3346.82344050292\\
0.509090909090909	3589.19230931671\\
0.518181818181818	3849.63929361883\\
0.527272727272727	4129.93610042813\\
0.536363636363636	4432.09411077819\\
0.545454545454546	4758.40624272065\\
0.554545454545455	5111.49792714824\\
0.563636363636364	5494.38959248632\\
0.572727272727273	5910.57380385724\\
0.581818181818182	6364.11122604357\\
0.590909090909091	6859.75099357767\\
0.6	7403.08304848238\\
0.609090909090909	8000.73280727382\\
0.618181818181818	8660.61254401716\\
0.627272727272727	9392.24975004391\\
0.636363636363636	10207.2214478455\\
0.645454545454545	11119.7366122567\\
0.654545454545455	12147.4291733018\\
0.663636363636364	13312.4561240666\\
0.672727272727273	14643.0470695008\\
0.681818181818182	16175.7376650553\\
0.690909090909091	17958.6670425678\\
0.7	20056.5815760944\\
0.709090909090909	22558.6723753142\\
0.718181818181818	25591.313899553\\
0.727272727272727	29339.6954190683\\
0.736363636363636	34086.5417792841\\
0.745454545454545	40286.0629500757\\
0.754545454545455	48717.209761453\\
0.763636363636364	60837.057375728\\
0.772727272727273	79724.023435851\\
0.781818181818182	113202.15804625\\
0.790909090909091	188720.183227803\\
0.8	518337.487988705\\
0.809090909090909	778943.687936148\\
0.818181818181818	229967.639205135\\
0.827272727272727	137506.596873794\\
0.836363636363636	99375.7009757209\\
0.845454545454545	78575.921626196\\
0.854545454545454	65488.2349488751\\
0.863636363636364	56500.5676843641\\
0.872727272727273	49951.9310915123\\
0.881818181818182	44971.7882376422\\
0.890909090909091	41059.8361628218\\
0.9	37908.0988009154\\
0.909090909090909	35316.6120000724\\
0.918181818181818	33149.8888076813\\
0.927272727272727	31312.8702585468\\
0.936363636363636	29736.8930076304\\
0.945454545454545	28371.1222509868\\
0.954545454545455	27177.1179387342\\
0.963636363636364	26125.2741408781\\
0.972727272727273	25192.4191579319\\
0.981818181818182	24360.1576636443\\
0.990909090909091	23613.700327361\\
1	22941.0214987642\\
};
\end{axis}

\begin{axis}[%
width=1.66in,
height=1.258in,
at={(5.379in,4.137in)},
scale only axis,
xmin=0,
xmax=1,
xlabel style={font=\color{white!15!black}},
xlabel={$\tau$},
ymin=0,
ymax=600000,
axis background/.style={fill=white},
title style={font=\bfseries},
title={$\gamma\text{= 0.30}$}
]
\addplot [color=mycolor1, line width=2.0pt, forget plot]
  table[row sep=crcr]{%
0.1	301.711667527467\\
0.109090909090909	646.797383475311\\
0.118181818181818	374.791224451057\\
0.127272727272727	1370.33630610087\\
0.136363636363636	467.271539242408\\
0.145454545454545	4154.04744918379\\
0.154545454545455	650.792572304619\\
0.163636363636364	550.084912579868\\
0.172727272727273	1153.22535006736\\
0.181818181818182	6531.97091891469\\
0.190909090909091	2533.17258247216\\
0.2	1216.18062263424\\
0.209090909090909	861.942788645783\\
0.218181818181818	893.160530720628\\
0.227272727272727	1310.55484995599\\
0.236363636363636	2147.44066697617\\
0.245454545454545	4645.80179609571\\
0.254545454545455	597384.305981872\\
0.263636363636364	48010.7136196177\\
0.272727272727273	3699.68884623946\\
0.281818181818182	2307.43341357594\\
0.290909090909091	1809.36939000569\\
0.3	1533.30461839689\\
0.309090909090909	1356.57598166175\\
0.318181818181818	1234.23507447014\\
0.327272727272727	1145.15683881972\\
0.336363636363636	1077.97186744305\\
0.345454545454545	1039.72457075339\\
0.354545454545454	1185.31325396079\\
0.363636363636364	1361.71346657074\\
0.372727272727273	1581.50853424685\\
0.381818181818182	1865.77543561686\\
0.390909090909091	2252.22635908825\\
0.4	2814.05397944577\\
0.409090909090909	3710.61269416601\\
0.418181818181818	5360.63279176357\\
0.427272727272727	9331.26788506383\\
0.436363636363636	31054.660287669\\
0.445454545454545	26537.83250742\\
0.454545454545455	9790.63574171691\\
0.463636363636364	6206.23132664927\\
0.472727272727273	6708.73351936131\\
0.481818181818182	9264.26739530053\\
0.490909090909091	14171.3233508582\\
0.5	27458.918868867\\
0.509090909090909	196211.341568975\\
0.518181818181818	42148.2689239076\\
0.527272727272727	19873.6873336523\\
0.536363636363636	13350.2428435317\\
0.545454545454546	10237.5784187349\\
0.554545454545455	8417.06326003326\\
0.563636363636364	7223.86925615312\\
0.572727272727273	6382.55160553035\\
0.581818181818182	5758.37511154443\\
0.590909090909091	5277.63204667142\\
0.6	4896.60016149592\\
0.609090909090909	4587.69951069897\\
0.618181818181818	4332.67055969777\\
0.627272727272727	4118.94257343404\\
0.636363636363636	3937.57747735791\\
0.645454545454545	3782.0455299605\\
0.654545454545455	3647.46498443805\\
0.663636363636364	3530.1127626735\\
0.672727272727273	3427.09975062137\\
0.681818181818182	3336.14952421118\\
0.690909090909091	3255.44400023408\\
0.7	3183.5135261789\\
0.709090909090909	3119.15716269038\\
0.718181818181818	3061.38390385431\\
0.727272727272727	3009.3686869962\\
0.736363636363636	3120.96606272915\\
0.745454545454545	3269.77507350845\\
0.754545454545455	3425.70497129541\\
0.763636363636364	3589.19230931691\\
0.772727272727273	3760.71015966631\\
0.781818181818182	3940.77199846878\\
0.790909090909091	4129.93610042774\\
0.8	4328.81052192693\\
0.809090909090909	4538.05876634106\\
0.818181818181818	4758.40624272075\\
0.827272727272727	4990.64765026928\\
0.836363636363636	5235.65544696736\\
0.845454545454545	5494.38959248555\\
0.854545454545454	5767.908794637\\
0.863636363636364	6057.38353706101\\
0.872727272727273	6364.11122604439\\
0.881818181818182	6689.53386971433\\
0.890909090909091	7035.25879760324\\
0.9	7403.0830484826\\
0.909090909090909	7795.022207108\\
0.918181818181818	8213.34466635375\\
0.927272727272727	8660.61254401855\\
0.936363636363636	9139.73081271703\\
0.945454545454545	9654.00663265092\\
0.954545454545455	10207.2214478456\\
0.963636363636364	10803.7191681466\\
0.972727272727273	11448.5147858415\\
0.981818181818182	12147.4291733001\\
0.990909090909091	12907.2577323108\\
1	13735.9832463372\\
};
\end{axis}

\begin{axis}[%
width=1.66in,
height=1.258in,
at={(1.011in,2.39in)},
scale only axis,
xmin=0,
xmax=1,
xlabel style={font=\color{white!15!black}},
xlabel={$\tau$},
ymin=0,
ymax=428821.655785178,
axis background/.style={fill=white},
title style={font=\bfseries},
title={$\gamma\text{= 0.40}$}
]
\addplot [color=mycolor1, line width=2.0pt, forget plot]
  table[row sep=crcr]{%
0.1	459.788413385911\\
0.109090909090909	2251.99828469127\\
0.118181818181818	1022.68209893432\\
0.127272727272727	371.236003069147\\
0.136363636363636	689.132546948465\\
0.145454545454545	646.797383475287\\
0.154545454545455	320.798941550423\\
0.163636363636364	1579.16847485181\\
0.172727272727273	791.189603543397\\
0.181818181818182	467.271539242421\\
0.190909090909091	4747.27616669797\\
0.2	1029.93407711286\\
0.209090909090909	563.69742490469\\
0.218181818181818	550.084912579875\\
0.227272727272727	926.676893059467\\
0.236363636363636	2059.02190648628\\
0.245454545454545	375006.37071822\\
0.254545454545455	2533.1725824723\\
0.263636363636364	1379.72798418902\\
0.272727272727273	999.106522524335\\
0.281818181818182	810.825091232125\\
0.290909090909091	893.160530720598\\
0.3	1181.92395856999\\
0.309090909090909	1645.30616111023\\
0.318181818181818	2505.75358702777\\
0.327272727272727	4645.80179609603\\
0.336363636363636	19175.9955783269\\
0.345454545454545	10573.5188540122\\
0.354545454545454	15471.9169841679\\
0.363636363636364	3699.68884623954\\
0.372727272727273	2507.00161110502\\
0.381818181818182	2015.76098107928\\
0.390909090909091	1726.75193970039\\
0.4	1533.30461839689\\
0.409090909090909	1394.48814671082\\
0.418181818181818	1290.25033343667\\
0.427272727272727	1209.40300809857\\
0.436363636363636	1145.1568388197\\
0.445454545454545	1093.13548373546\\
0.454545454545455	1050.3845572743\\
0.463636363636364	1073.72465363577\\
0.472727272727273	1185.31325396078\\
0.481818181818182	1314.14252217023\\
0.490909090909091	1465.13400386838\\
0.5	1645.43278917023\\
0.509090909090909	1865.77543561673\\
0.518181818181818	2142.92847419051\\
0.527272727272727	2504.28572651655\\
0.536363636363636	2997.24872506076\\
0.545454545454546	3710.61269416599\\
0.554545454545455	4831.37775424747\\
0.563636363636364	6832.6553758479\\
0.572727272727273	11366.1532787412\\
0.581818181818182	31054.6602876752\\
0.590909090909091	48280.5234393707\\
0.6	14164.843957984\\
0.609090909090909	8518.29484012672\\
0.618181818181818	6206.23132664991\\
0.627272727272727	6250.11578666532\\
0.636363636363636	7812.76825884168\\
0.645454545454545	10176.1584177991\\
0.654545454545455	14171.3233508616\\
0.663636363636364	22383.5665389396\\
0.672727272727273	48957.100031102\\
0.681818181818182	428821.655785178\\
0.690909090909091	42148.2689239062\\
0.7	22796.7488036715\\
0.709090909090909	15911.0738992793\\
0.718181818181818	12382.1103909743\\
0.727272727272727	10237.5784187339\\
0.736363636363636	8797.33508391343\\
0.745454545454545	7764.1113433789\\
0.754545454545455	6987.32923863333\\
0.763636363636364	6382.55160553104\\
0.772727272727273	5898.75769899909\\
0.781818181818182	5503.29891119291\\
0.790909090909091	5174.31101054207\\
0.8	4896.60016149497\\
0.809090909090909	4659.27882871598\\
0.818181818181818	4454.34066325913\\
0.827272727272727	4275.76626408067\\
0.836363636363636	4118.94257343421\\
0.845454545454545	3980.27465310443\\
0.854545454545454	3856.91935971776\\
0.863636363636364	3746.598481528\\
0.872727272727273	3647.46498443789\\
0.881818181818182	3558.00555152952\\
0.890909090909091	3476.96842228204\\
0.9	3403.30918596914\\
0.909090909090909	3336.14952421093\\
0.918181818181818	3274.74543127815\\
0.927272727272727	3218.46246517438\\
0.936363636363636	3166.75627893981\\
0.945454545454545	3119.15716269035\\
0.954545454545455	3075.25766419268\\
0.963636363636364	3034.70259546703\\
0.972727272727273	3013.7885146395\\
0.981818181818182	3120.96606272913\\
0.990909090909091	3231.92153896333\\
1	3346.82344050292\\
};
\end{axis}

\begin{axis}[%
width=1.66in,
height=1.258in,
at={(3.195in,2.39in)},
scale only axis,
xmin=0,
xmax=1,
xlabel style={font=\color{white!15!black}},
xlabel={$\tau$},
ymin=0,
ymax=150000,
axis background/.style={fill=white},
title style={font=\bfseries},
title={$\gamma\text{= 0.50}$}
]
\addplot [color=mycolor1, line width=2.0pt, forget plot]
  table[row sep=crcr]{%
0.1	72.2403387947242\\
0.109090909090909	59.8504271430352\\
0.118181818181818	710.751155383416\\
0.127272727272727	146.33140326616\\
0.136363636363636	2251.99828469143\\
0.145454545454545	400.505945256248\\
0.154545454545455	2574.54933331435\\
0.163636363636364	263.334453527429\\
0.172727272727273	1559.83669451314\\
0.181818181818182	646.797383475297\\
0.190909090909091	348.963884803103\\
0.2	574.934521566632\\
0.209090909090909	4172.53819083643\\
0.218181818181818	645.156992255097\\
0.227272727272727	467.27153924242\\
0.236363636363636	1931.88695264356\\
0.245454545454545	1795.59676893532\\
0.254545454545455	753.939929550376\\
0.263636363636364	525.059565610081\\
0.272727272727273	550.084912579877\\
0.281818181818182	823.782623547261\\
0.290909090909091	1412.20601595512\\
0.3	3561.41626089149\\
0.309090909090909	12047.8130064275\\
0.318181818181818	2533.17258247228\\
0.327272727272727	1507.23434558789\\
0.336363636363636	1115.81920537856\\
0.345454545454545	910.232189867271\\
0.354545454545454	784.144693364551\\
0.363636363636364	893.160530720613\\
0.372727272727273	1113.87904309503\\
0.381818181818182	1430.07663534549\\
0.390909090909091	1919.24562063619\\
0.4	2774.21335776032\\
0.409090909090909	4645.80179609614\\
0.418181818181818	12004.1704475258\\
0.427272727272727	26659.7828389582\\
0.436363636363636	10674.2031676276\\
0.445454545454545	8889.18792293965\\
0.454545454545455	3699.68884623929\\
0.463636363636364	2654.00779493117\\
0.472727272727273	2177.05973530536\\
0.481818181818182	1884.40091338577\\
0.490909090909091	1682.20476277183\\
0.5	1533.30461839691\\
0.509090909090909	1419.01329331322\\
0.518181818181818	1328.64161319516\\
0.527272727272727	1255.55526441246\\
0.536363636363636	1195.3922929145\\
0.545454545454546	1145.15683881969\\
0.554545454545455	1102.71885711383\\
0.563636363636364	1066.52039929588\\
0.572727272727273	1035.39461157222\\
0.581818181818182	1094.846950383\\
0.590909090909091	1185.31325396079\\
0.6	1286.78689774791\\
0.609090909090909	1401.68290934098\\
0.618181818181818	1533.23546375025\\
0.627272727272727	1685.86874185516\\
0.636363636363636	1865.77543561675\\
0.645454545454545	2081.84585929636\\
0.654545454545455	2347.2172022062\\
0.663636363636364	2681.98795219301\\
0.672727272727273	3118.29262744391\\
0.681818181818182	3710.61269416581\\
0.690909090909091	4559.07445816294\\
0.7	5870.04044418443\\
0.709090909090909	8148.93975924893\\
0.718181818181818	13053.6500165162\\
0.727272727272727	31054.6602876559\\
0.736363636363636	96448.5760238752\\
0.745454545454545	19602.9739697099\\
0.754545454545455	11149.894649772\\
0.763636363636364	7912.38049120573\\
0.772727272727273	6206.23132665096\\
0.781818181818182	5998.93443180678\\
0.790909090909091	7117.06480874663\\
0.8	8632.06187519535\\
0.809090909090909	10802.2234299523\\
0.818181818181818	14171.3233508598\\
0.827272727272727	20113.7795238887\\
0.836363636363636	33401.1257265439\\
0.845454545454545	89674.8318910304\\
0.854545454545454	148895.685561902\\
0.863636363636364	42148.2689238943\\
0.872727272727273	25041.7309218657\\
0.881818181818182	18054.1122840015\\
0.890909090909091	14257.5427227837\\
0.9	11873.3130139807\\
0.909090909090909	10237.578418734\\
0.918181818181818	9046.20884724358\\
0.927272727272727	8140.20729044465\\
0.936363636363636	7428.36180105511\\
0.945454545454545	6854.59975914944\\
0.954545454545455	6382.55160553173\\
0.963636363636364	5987.59469698775\\
0.972727272727273	5652.46443087161\\
0.981818181818182	5364.69635361593\\
0.990909090909091	5115.06600453023\\
1	4896.60016149548\\
};
\end{axis}

\begin{axis}[%
width=1.66in,
height=1.258in,
at={(5.379in,2.39in)},
scale only axis,
xmin=0,
xmax=1,
xlabel style={font=\color{white!15!black}},
xlabel={$\tau$},
ymin=0,
ymax=600000,
axis background/.style={fill=white},
title style={font=\bfseries},
title={$\gamma\text{= 0.60}$}
]
\addplot [color=mycolor1, line width=2.0pt, forget plot]
  table[row sep=crcr]{%
0.1	56.0782432653308\\
0.109090909090909	151.44141476292\\
0.118181818181818	82.2230459681077\\
0.127272727272727	61.6132534272488\\
0.136363636363636	90.2203368309283\\
0.145454545454545	419.145332852472\\
0.154545454545455	159.633725844276\\
0.163636363636364	2251.99828469118\\
0.172727272727273	583.638902641657\\
0.181818181818182	1266.22446464503\\
0.190909090909091	371.236003069156\\
0.2	301.711667527465\\
0.209090909090909	5592.01680417081\\
0.218181818181818	646.797383475309\\
0.227272727272727	372.675024275019\\
0.236363636363636	374.791224451077\\
0.245454545454545	1579.16847485178\\
0.254545454545455	1370.3363061009\\
0.263636363636364	578.570538039378\\
0.272727272727273	467.271539242407\\
0.281818181818182	1351.18254858155\\
0.290909090909091	4154.04744918392\\
0.3	1029.93407711286\\
0.309090909090909	650.792572304618\\
0.318181818181818	503.24648087448\\
0.327272727272727	550.084912579862\\
0.336363636363636	765.028892539898\\
0.345454545454545	1153.22535006736\\
0.354545454545454	2059.02190648628\\
0.363636363636364	6531.97091891597\\
0.372727272727273	7230.34474747873\\
0.381818181818182	2533.17258247219\\
0.390909090909091	1609.39358323268\\
0.4	1216.18062263423\\
0.409090909090909	999.106522524338\\
0.418181818181818	861.942788645797\\
0.427272727272727	767.764388647896\\
0.436363636363636	893.160530720613\\
0.445454545454545	1071.78075691816\\
0.454545454545455	1310.55484995601\\
0.463636363636364	1645.30616111015\\
0.472727272727273	2147.44066697612\\
0.481818181818182	2982.83605609507\\
0.490909090909091	4645.80179609619\\
0.5	9572.96854187973\\
0.509090909090909	597384.305984569\\
0.518181818181818	10573.5188540126\\
0.527272727272727	48010.7136196424\\
0.536363636363636	7004.54710260668\\
0.545454545454546	3699.68884623918\\
0.554545454545455	2767.60338560611\\
0.563636363636364	2307.43341357596\\
0.572727272727273	2015.76098107932\\
0.581818181818182	1809.36939000571\\
0.590909090909091	1654.33606134521\\
0.6	1533.30461839692\\
0.609090909090909	1436.1756498518\\
0.618181818181818	1356.57598166169\\
0.627272727272727	1290.25033343668\\
0.636363636363636	1234.23507447011\\
0.645454545454545	1186.39659266396\\
0.654545454545455	1145.15683881974\\
0.663636363636364	1109.32208790403\\
0.672727272727273	1077.97186744308\\
0.681818181818182	1050.38455727428\\
0.690909090909091	1039.72457075342\\
0.7	1109.24415284698\\
0.709090909090909	1185.31325396076\\
0.718181818181818	1269.01631349591\\
0.727272727272727	1361.71346657075\\
0.736363636363636	1465.13400386834\\
0.745454545454545	1581.50853424685\\
0.754545454545455	1713.75882181731\\
0.763636363636364	1865.77543561684\\
0.772727272727273	2042.83276270403\\
0.781818181818182	2252.22635908835\\
0.790909090909091	2504.28572651667\\
0.8	2814.0539794457\\
0.809090909090909	3204.22425201553\\
0.818181818181818	3710.61269416612\\
0.827272727272727	4393.19036867838\\
0.836363636363636	5360.63279176347\\
0.845454545454545	6832.65537584876\\
0.854545454545454	9331.26788506349\\
0.863636363636364	14475.212499171\\
0.872727272727273	31054.6602876506\\
0.881818181818182	293673.089823492\\
0.890909090909091	26537.8325074097\\
0.9	14164.8439579909\\
0.909090909090909	9790.63574171641\\
0.918181818181818	7558.03518360876\\
0.927272727272727	6206.23132665\\
0.936363636363636	5840.40250110542\\
0.945454545454545	6708.73351936176\\
0.954545454545455	7812.76825884008\\
0.963636363636364	9264.26739530143\\
0.972727272727273	11258.6485080036\\
0.981818181818182	14171.3233508632\\
0.990909090909091	18826.9978590474\\
1	27458.9188688581\\
};
\end{axis}

\begin{axis}[%
width=1.66in,
height=1.258in,
at={(1.011in,0.642in)},
scale only axis,
xmin=0,
xmax=1,
xlabel style={font=\color{white!15!black}},
xlabel={$\tau$},
ymin=0,
ymax=114480.083835993,
axis background/.style={fill=white},
title style={font=\bfseries},
title={$\gamma\text{= 0.70}$}
]
\addplot [color=mycolor1, line width=2.0pt, forget plot]
  table[row sep=crcr]{%
0.1	56.211057835717\\
0.109090909090909	56.1210051469373\\
0.118181818181818	56.0661747409837\\
0.127272727272727	151.441414762915\\
0.136363636363636	112.170991216638\\
0.145454545454545	63.7249992247335\\
0.154545454545455	61.2278535985026\\
0.163636363636364	2068.31183007679\\
0.172727272727273	30666.0480922874\\
0.181818181818182	169.552603609852\\
0.190909090909091	2251.99828469165\\
0.2	1115.15338426626\\
0.209090909090909	7319.01871633079\\
0.218181818181818	1362.92546741302\\
0.227272727272727	281.885015806055\\
0.236363636363636	465.796936733709\\
0.245454545454545	8240.58937678451\\
0.254545454545455	646.797383475294\\
0.263636363636364	392.900488501779\\
0.272727272727273	304.648095078629\\
0.281818181818182	717.703691454605\\
0.290909090909091	110580.715962955\\
0.3	957.416457124331\\
0.309090909090909	540.45542360763\\
0.318181818181818	467.271539242419\\
0.327272727272727	1100.37337110937\\
0.336363636363636	114480.083835993\\
0.345454545454545	1466.24595069656\\
0.354545454545454	812.971052459754\\
0.363636363636364	596.85339927603\\
0.372727272727273	489.235664290823\\
0.381818181818182	550.084912579873\\
0.390909090909091	727.034255218565\\
0.4	1013.79130078227\\
0.409090909090909	1556.04804007519\\
0.418181818181818	2961.75665146988\\
0.427272727272727	15167.1775680701\\
0.436363636363636	5651.24034005427\\
0.445454545454545	2533.17258247235\\
0.454545454545455	1693.06739708565\\
0.463636363636364	1303.34436969068\\
0.472727272727273	1078.91296993111\\
0.481818181818182	933.378011740933\\
0.490909090909091	831.624199582354\\
0.5	772.564600551573\\
0.509090909090909	893.16053072061\\
0.518181818181818	1043.1666099219\\
0.527272727272727	1234.53197994621\\
0.536363636363636	1486.70861341532\\
0.545454545454546	1833.64725827116\\
0.554545454545455	2340.42665403588\\
0.563636363636364	3149.61680157702\\
0.572727272727273	4645.80179609612\\
0.581818181818182	8349.45639528594\\
0.590909090909091	33001.210386901\\
0.6	18515.6726556878\\
0.609090909090909	7934.34015451244\\
0.618181818181818	34697.1494387236\\
0.627272727272727	6116.31214758237\\
0.636363636363636	3699.68884623915\\
0.645454545454545	2858.37540182735\\
0.654545454545455	2415.54661940919\\
0.663636363636364	2127.28302836778\\
0.672727272727273	1919.4111957124\\
0.681818181818182	1760.78622043994\\
0.690909090909091	1635.25520832267\\
0.7	1533.30461839693\\
0.709090909090909	1448.8574066515\\
0.718181818181818	1377.80757141881\\
0.727272727272727	1317.26328941638\\
0.736363636363636	1265.1210288023\\
0.745454545454545	1219.81022893841\\
0.754545454545455	1180.13258062675\\
0.763636363636364	1145.15683881971\\
0.772727272727273	1114.14774348031\\
0.781818181818182	1086.51668097322\\
0.790909090909091	1061.78663996589\\
0.8	1039.56681843488\\
0.809090909090909	1058.9734129423\\
0.818181818181818	1119.68740696516\\
0.827272727272727	1185.31325396081\\
0.836363636363636	1256.5432909665\\
0.845454545454545	1334.21804933849\\
0.854545454545454	1419.36816454069\\
0.863636363636364	1513.27064722962\\
0.872727272727273	1617.52527900593\\
0.881818181818182	1734.15962220832\\
0.890909090909091	1865.77543561683\\
0.9	2015.75630848261\\
0.909090909090909	2188.56820807384\\
0.918181818181818	2390.20553930883\\
0.927272727272727	2628.87358561776\\
0.936363636363636	2916.07129872818\\
0.945454545454545	3268.38483319434\\
0.954545454545455	3710.61269416614\\
0.963636363636364	4281.54962158475\\
0.972727272727273	5045.51107286123\\
0.981818181818182	6117.55754068077\\
0.990909090909091	7726.08266980034\\
1	10398.4259922006\\
};
\end{axis}

\begin{axis}[%
width=1.66in,
height=1.258in,
at={(3.195in,0.642in)},
scale only axis,
xmin=0,
xmax=1,
xlabel style={font=\color{white!15!black}},
xlabel={$\tau$},
ymin=0,
ymax=400000,
axis background/.style={fill=white},
title style={font=\bfseries},
title={$\gamma\text{= 0.80}$}
]
\addplot [color=mycolor1, line width=2.0pt, forget plot]
  table[row sep=crcr]{%
0.1	56.6284510925037\\
0.109090909090909	56.3087400105358\\
0.118181818181818	56.1652885306279\\
0.127272727272727	56.1088442311084\\
0.136363636363636	56.0553675589386\\
0.145454545454545	151.44141476292\\
0.154545454545455	287.275247298391\\
0.163636363636364	66.1504033555767\\
0.172727272727273	60.3949384959293\\
0.181818181818182	90.2203368309282\\
0.190909090909091	361.781183700167\\
0.2	459.788413385911\\
0.209090909090909	179.055303475889\\
0.218181818181818	2251.99828469134\\
0.227272727272727	4241.13567343283\\
0.236363636363636	1022.68209893425\\
0.245454545454545	679.007870460741\\
0.254545454545455	371.236003069156\\
0.263636363636364	250.141997438833\\
0.272727272727273	689.132546948465\\
0.281818181818182	3009.55351293252\\
0.290909090909091	646.797383475286\\
0.3	410.35195509902\\
0.309090909090909	320.798941550423\\
0.318181818181818	484.053991769554\\
0.327272727272727	1579.16847485181\\
0.336363636363636	2320.94436540045\\
0.345454545454545	791.189603543395\\
0.354545454545454	515.765522763205\\
0.363636363636364	467.271539242421\\
0.372727272727273	960.500562662253\\
0.381818181818182	4747.27616669794\\
0.390909090909091	2259.79895127794\\
0.4	1029.93407711286\\
0.409090909090909	710.484037955208\\
0.418181818181818	563.69742490469\\
0.427272727272727	479.476849598356\\
0.436363636363636	550.084912579875\\
0.445454545454545	700.450630042474\\
0.454545454545455	926.676893059444\\
0.463636363636364	1304.3018997742\\
0.472727272727273	2059.02190648627\\
0.481818181818182	4308.42329370667\\
0.490909090909091	375006.370710225\\
0.5	4866.65439547206\\
0.509090909090909	2533.17258247201\\
0.518181818181818	1762.85133075251\\
0.527272727272727	1379.72798418902\\
0.536363636363636	1150.91654353738\\
0.545454545454546	999.10652252434\\
0.554545454545455	891.243142301131\\
0.563636363636364	810.825091232135\\
0.572727272727273	786.324572889858\\
0.581818181818182	893.160530720616\\
0.590909090909091	1022.45313543341\\
0.6	1181.92395856995\\
0.609090909090909	1383.30063129306\\
0.618181818181818	1645.30616111023\\
0.627272727272727	1999.79768762683\\
0.636363636363636	2505.75358702772\\
0.645454545454545	3285.99472940133\\
0.654545454545455	4645.80179609603\\
0.663636363636364	7612.75847535983\\
0.672727272727273	19175.9955783243\\
0.681818181818182	43606.4061268883\\
0.690909090909091	10573.518854012\\
0.7	15168.9711447081\\
0.709090909090909	15471.9169841608\\
0.718181818181818	5601.3342208213\\
0.727272727272727	3699.68884623954\\
0.736363636363636	2932.74413405832\\
0.745454545454545	2507.00161110502\\
0.754545454545455	2223.44316057872\\
0.763636363636364	2015.76098107928\\
0.772727272727273	1855.22608264762\\
0.781818181818182	1726.75193970039\\
0.790909090909091	1621.37145632474\\
0.8	1533.30461839689\\
0.809090909090909	1458.6103519327\\
0.818181818181818	1394.48814671083\\
0.827272727272727	1338.88353144419\\
0.836363636363636	1290.25033343667\\
0.845454545454545	1247.40011865594\\
0.854545454545454	1209.40300809857\\
0.863636363636364	1175.52020389233\\
0.872727272727273	1145.15683881972\\
0.881818181818182	1117.82826697197\\
0.890909090909091	1093.13548373546\\
0.9	1070.74689039687\\
0.909090909090909	1050.3845572743\\
0.918181818181818	1031.81373334159\\
0.927272727272727	1073.72465363576\\
0.936363636363636	1127.60905597602\\
0.945454545454545	1185.31325396072\\
0.954545454545455	1247.3059603355\\
0.963636363636364	1314.14252217023\\
0.972727272727273	1386.48593592396\\
0.981818181818182	1465.13400386835\\
0.990909090909091	1551.05471873573\\
1	1645.43278917023\\
};
\end{axis}

\begin{axis}[%
width=1.66in,
height=1.258in,
at={(5.379in,0.642in)},
scale only axis,
xmin=0,
xmax=1,
xlabel style={font=\color{white!15!black}},
xlabel={$\tau$},
ymin=0,
ymax=600000,
axis background/.style={fill=white},
title style={font=\bfseries},
title={$\gamma\text{= 0.90}$}
]
\addplot [color=mycolor1, line width=2.0pt, forget plot]
  table[row sep=crcr]{%
0.1	57.3774572667494\\
0.109090909090909	56.7893750760966\\
0.118181818181818	56.4236540238003\\
0.127272727272727	56.2286718584842\\
0.136363636363636	56.1410011449532\\
0.145454545454545	56.0996999969733\\
0.154545454545455	56.0456826805623\\
0.163636363636364	151.44141476292\\
0.172727272727273	100.74546143785\\
0.181818181818182	68.9361431188591\\
0.190909090909091	61.6132534272487\\
0.2	64.0070271689236\\
0.209090909090909	555.874244111548\\
0.218181818181818	419.145332852485\\
0.227272727272727	205.000262532271\\
0.236363636363636	188.379528588787\\
0.245454545454545	2251.99828469133\\
0.254545454545455	3194.16161896286\\
0.263636363636364	568.711372095323\\
0.272727272727273	1266.22446464499\\
0.281818181818182	780.457746594441\\
0.290909090909091	294.495191117579\\
0.3	301.711667527465\\
0.309090909090909	1018.88502240777\\
0.318181818181818	2046.26629462878\\
0.327272727272727	646.797383475304\\
0.336363636363636	425.561573190894\\
0.345454545454545	335.511386542325\\
0.354545454545454	374.791224451072\\
0.363636363636364	824.853470284319\\
0.372727272727273	7991.81002803101\\
0.381818181818182	1370.33630610096\\
0.390909090909091	701.393825560486\\
0.4	498.469399700135\\
0.409090909090909	467.271539242409\\
0.418181818181818	871.279102375746\\
0.427272727272727	2649.26651063261\\
0.436363636363636	4154.04744918489\\
0.445454545454545	1335.01658012502\\
0.454545454545455	851.200409990882\\
0.463636363636364	650.792572304631\\
0.472727272727273	541.256092179678\\
0.481818181818182	472.289806594794\\
0.490909090909091	550.084912579872\\
0.5	680.809432365273\\
0.509090909090909	867.096662510141\\
0.518181818181818	1153.22535006735\\
0.527272727272727	1647.55849715561\\
0.536363636363636	2704.21631707382\\
0.545454545454546	6531.97091891352\\
0.554545454545455	22057.1510895115\\
0.563636363636364	4397.49782335734\\
0.572727272727273	2533.17258247214\\
0.581818181818182	1821.93630317791\\
0.590909090909091	1447.20257361326\\
0.6	1216.18062263421\\
0.609090909090909	1059.73868269454\\
0.618181818181818	946.959838380106\\
0.627272727272727	861.942788645801\\
0.636363636363636	795.673156966087\\
0.645454545454545	797.266569568627\\
0.654545454545455	893.160530720611\\
0.663636363636364	1006.7653915914\\
0.672727272727273	1143.35961052115\\
0.681818181818182	1310.55484995599\\
0.690909090909091	1519.7471615917\\
0.7	1788.80628919202\\
0.709090909090909	2147.44066697622\\
0.718181818181818	2648.96354823893\\
0.727272727272727	3399.5881510393\\
0.736363636363636	4645.80179609646\\
0.745454545454545	7120.52956419533\\
0.754545454545455	14414.7850918742\\
0.763636363636364	597384.305986542\\
0.772727272727273	15846.9280331315\\
0.781818181818182	7963.8523887358\\
0.790909090909091	48010.7136196219\\
0.8	10920.643808926\\
0.809090909090909	5265.9261140191\\
0.818181818181818	3699.68884623922\\
0.827272727272727	2994.87203357227\\
0.836363636363636	2585.5959664701\\
0.845454545454545	2307.43341357592\\
0.854545454545454	2100.99142304876\\
0.863636363636364	1939.68361579195\\
0.872727272727273	1809.36939000566\\
0.881818181818182	1701.57925390206\\
0.890909090909091	1610.81592157258\\
0.9	1533.3046183969\\
0.909090909090909	1466.3439494521\\
0.918181818181818	1407.93835916195\\
0.927272727272727	1356.57598166169\\
0.936363636363636	1311.08740647375\\
0.945454545454545	1270.55221380132\\
0.954545454545455	1234.23507447012\\
0.963636363636364	1201.54086509335\\
0.972727272727273	1171.98241024906\\
0.981818181818182	1145.15683881971\\
0.990909090909091	1120.7279544099\\
1	1098.41288979779\\
};
\end{axis}
\end{tikzpicture}%}} 

\caption{Nei grafici viene rappresentato come varia l'indice di stiff (nelle condizioni iniziali)  facendo variare $\tau$ per alcuni valori di $\gamma$ fissati.\\
Per realizzare questi grafici abbiamo utilizzato grafi di Erdos-Renyi con probabilit\`a $0.5$ e dimensione $N$ }
\label{Erdos}
\end{figure}

\begin{figure}[h]
\centering
\subfloat[][$N=10$]
{\resizebox{0.45\textwidth}{!}{% This file was created by matlab2tikz.
%
%The latest updates can be retrieved from
%  http://www.mathworks.com/matlabcentral/fileexchange/22022-matlab2tikz-matlab2tikz
%where you can also make suggestions and rate matlab2tikz.
%
\definecolor{mycolor1}{rgb}{0.00000,0.44700,0.74100}%
%
\begin{tikzpicture}

\begin{axis}[%
width=1.66in,
height=1.258in,
at={(1.011in,4.137in)},
scale only axis,
xmin=0,
xmax=1,
xlabel style={font=\color{white!15!black}},
xlabel={$\tau$},
ymin=345,
ymax=1500,
axis background/.style={fill=white},
title style={font=\bfseries},
title={$\gamma\text{= 0.10}$}
]
\addplot [color=mycolor1, line width=2.0pt, forget plot]
  table[row sep=crcr]{%
0.1	345\\
0.109090909090909	389\\
0.118181818181818	437\\
0.127272727272727	484\\
0.136363636363636	527\\
0.145454545454545	571\\
0.154545454545455	613\\
0.163636363636364	648\\
0.172727272727273	677\\
0.181818181818182	704\\
0.190909090909091	734\\
0.2	753\\
0.209090909090909	770\\
0.218181818181818	793\\
0.227272727272727	813\\
0.236363636363636	833\\
0.245454545454545	849\\
0.254545454545455	867\\
0.263636363636364	886\\
0.272727272727273	900\\
0.281818181818182	915\\
0.290909090909091	933\\
0.3	949\\
0.309090909090909	954\\
0.318181818181818	966\\
0.327272727272727	979\\
0.336363636363636	998\\
0.345454545454545	1008\\
0.354545454545454	1011\\
0.363636363636364	1025\\
0.372727272727273	1035\\
0.381818181818182	1040\\
0.390909090909091	1051\\
0.4	1061\\
0.409090909090909	1074\\
0.418181818181818	1082\\
0.427272727272727	1089\\
0.436363636363636	1094\\
0.445454545454545	1098\\
0.454545454545455	1112\\
0.463636363636364	1120\\
0.472727272727273	1127\\
0.481818181818182	1126\\
0.490909090909091	1134\\
0.5	1138\\
0.509090909090909	1143\\
0.518181818181818	1154\\
0.527272727272727	1162\\
0.536363636363636	1167\\
0.545454545454546	1172\\
0.554545454545455	1178\\
0.563636363636364	1179\\
0.572727272727273	1184\\
0.581818181818182	1190\\
0.590909090909091	1195\\
0.6	1197\\
0.609090909090909	1202\\
0.618181818181818	1204\\
0.627272727272727	1210\\
0.636363636363636	1212\\
0.645454545454545	1212\\
0.654545454545455	1218\\
0.663636363636364	1220\\
0.672727272727273	1223\\
0.681818181818182	1229\\
0.690909090909091	1226\\
0.7	1228\\
0.709090909090909	1236\\
0.718181818181818	1238\\
0.727272727272727	1241\\
0.736363636363636	1243\\
0.745454545454545	1245\\
0.754545454545455	1248\\
0.763636363636364	1250\\
0.772727272727273	1253\\
0.781818181818182	1256\\
0.790909090909091	1259\\
0.8	1262\\
0.809090909090909	1265\\
0.818181818181818	1268\\
0.827272727272727	1271\\
0.836363636363636	1274\\
0.845454545454545	1277\\
0.854545454545454	1276\\
0.863636363636364	1279\\
0.872727272727273	1282\\
0.881818181818182	1286\\
0.890909090909091	1289\\
0.9	1288\\
0.909090909090909	1291\\
0.918181818181818	1294\\
0.927272727272727	1294\\
0.936363636363636	1297\\
0.945454545454545	1300\\
0.954545454545455	1299\\
0.963636363636364	1303\\
0.972727272727273	1306\\
0.981818181818182	1305\\
0.990909090909091	1308\\
1	1312\\
};
\end{axis}

\begin{axis}[%
width=1.66in,
height=1.258in,
at={(3.195in,4.137in)},
scale only axis,
xmin=0,
xmax=1,
xlabel style={font=\color{white!15!black}},
xlabel={$\tau$},
ymin=342,
ymax=1500,
axis background/.style={fill=white},
title style={font=\bfseries},
title={$\gamma\text{= 0.20}$}
]
\addplot [color=mycolor1, line width=2.0pt, forget plot]
  table[row sep=crcr]{%
0.1	342\\
0.109090909090909	369\\
0.118181818181818	381\\
0.127272727272727	412\\
0.136363636363636	446\\
0.145454545454545	478\\
0.154545454545455	510\\
0.163636363636364	538\\
0.172727272727273	566\\
0.181818181818182	588\\
0.190909090909091	608\\
0.2	628\\
0.209090909090909	649\\
0.218181818181818	666\\
0.227272727272727	678\\
0.236363636363636	704\\
0.245454545454545	720\\
0.254545454545455	737\\
0.263636363636364	749\\
0.272727272727273	762\\
0.281818181818182	776\\
0.290909090909091	791\\
0.3	801\\
0.309090909090909	819\\
0.318181818181818	829\\
0.327272727272727	841\\
0.336363636363636	851\\
0.345454545454545	861\\
0.354545454545454	871\\
0.363636363636364	883\\
0.372727272727273	892\\
0.381818181818182	904\\
0.390909090909091	912\\
0.4	922\\
0.409090909090909	932\\
0.418181818181818	933\\
0.427272727272727	944\\
0.436363636363636	951\\
0.445454545454545	957\\
0.454545454545455	969\\
0.463636363636364	976\\
0.472727272727273	984\\
0.481818181818182	991\\
0.490909090909091	999\\
0.5	1014\\
0.509090909090909	1016\\
0.518181818181818	1020\\
0.527272727272727	1029\\
0.536363636363636	1037\\
0.545454545454546	1041\\
0.554545454545455	1050\\
0.563636363636364	1055\\
0.572727272727273	1071\\
0.581818181818182	1076\\
0.590909090909091	1074\\
0.6	1091\\
0.609090909090909	1089\\
0.618181818181818	1094\\
0.627272727272727	1100\\
0.636363636363636	1104\\
0.645454545454545	1110\\
0.654545454545455	1116\\
0.663636363636364	1122\\
0.672727272727273	1134\\
0.681818181818182	1134\\
0.690909090909091	1146\\
0.7	1153\\
0.709090909090909	1148\\
0.718181818181818	1154\\
0.727272727272727	1160\\
0.736363636363636	1166\\
0.745454545454545	1168\\
0.754545454545455	1174\\
0.763636363636364	1176\\
0.772727272727273	1183\\
0.781818181818182	1189\\
0.790909090909091	1192\\
0.8	1198\\
0.809090909090909	1207\\
0.818181818181818	1214\\
0.827272727272727	1216\\
0.836363636363636	1219\\
0.845454545454545	1226\\
0.854545454545454	1228\\
0.863636363636364	1231\\
0.872727272727273	1232\\
0.881818181818182	1234\\
0.890909090909091	1237\\
0.9	1246\\
0.909090909090909	1253\\
0.918181818181818	1256\\
0.927272727272727	1259\\
0.936363636363636	1260\\
0.945454545454545	1263\\
0.954545454545455	1266\\
0.963636363636364	1268\\
0.972727272727273	1270\\
0.981818181818182	1274\\
0.990909090909091	1276\\
1	1279\\
};
\end{axis}

\begin{axis}[%
width=1.66in,
height=1.258in,
at={(5.379in,4.137in)},
scale only axis,
xmin=0,
xmax=1,
xlabel style={font=\color{white!15!black}},
xlabel={$\tau$},
ymin=423,
ymax=1258,
axis background/.style={fill=white},
title style={font=\bfseries},
title={$\gamma\text{= 0.30}$}
]
\addplot [color=mycolor1, line width=2.0pt, forget plot]
  table[row sep=crcr]{%
0.1	435\\
0.109090909090909	431\\
0.118181818181818	427\\
0.127272727272727	423\\
0.136363636363636	440\\
0.145454545454545	458\\
0.154545454545455	473\\
0.163636363636364	504\\
0.172727272727273	516\\
0.181818181818182	532\\
0.190909090909091	549\\
0.2	565\\
0.209090909090909	583\\
0.218181818181818	601\\
0.227272727272727	617\\
0.236363636363636	633\\
0.245454545454545	647\\
0.254545454545455	651\\
0.263636363636364	674\\
0.272727272727273	688\\
0.281818181818182	703\\
0.290909090909091	717\\
0.3	727\\
0.309090909090909	741\\
0.318181818181818	754\\
0.327272727272727	768\\
0.336363636363636	780\\
0.345454545454545	785\\
0.354545454545454	801\\
0.363636363636364	806\\
0.372727272727273	822\\
0.381818181818182	836\\
0.390909090909091	843\\
0.4	855\\
0.409090909090909	863\\
0.418181818181818	873\\
0.427272727272727	884\\
0.436363636363636	895\\
0.445454545454545	902\\
0.454545454545455	912\\
0.463636363636364	922\\
0.472727272727273	931\\
0.481818181818182	939\\
0.490909090909091	947\\
0.5	959\\
0.509090909090909	965\\
0.518181818181818	974\\
0.527272727272727	982\\
0.536363636363636	989\\
0.545454545454546	994\\
0.554545454545455	1002\\
0.563636363636364	1011\\
0.572727272727273	1021\\
0.581818181818182	1023\\
0.590909090909091	1032\\
0.6	1037\\
0.609090909090909	1046\\
0.618181818181818	1053\\
0.627272727272727	1055\\
0.636363636363636	1060\\
0.645454545454545	1065\\
0.654545454545455	1075\\
0.663636363636364	1086\\
0.672727272727273	1085\\
0.681818181818182	1091\\
0.690909090909091	1096\\
0.7	1102\\
0.709090909090909	1107\\
0.718181818181818	1113\\
0.727272727272727	1118\\
0.736363636363636	1124\\
0.745454545454545	1137\\
0.754545454545455	1136\\
0.763636363636364	1142\\
0.772727272727273	1148\\
0.781818181818182	1150\\
0.790909090909091	1156\\
0.8	1162\\
0.809090909090909	1168\\
0.818181818181818	1170\\
0.827272727272727	1176\\
0.836363636363636	1182\\
0.845454545454545	1185\\
0.854545454545454	1198\\
0.863636363636364	1200\\
0.872727272727273	1207\\
0.881818181818182	1209\\
0.890909090909091	1215\\
0.9	1218\\
0.909090909090909	1217\\
0.918181818181818	1220\\
0.927272727272727	1226\\
0.936363636363636	1229\\
0.945454545454545	1235\\
0.954545454545455	1237\\
0.963636363636364	1239\\
0.972727272727273	1246\\
0.981818181818182	1249\\
0.990909090909091	1251\\
1	1258\\
};
\end{axis}

\begin{axis}[%
width=1.66in,
height=1.258in,
at={(1.011in,2.39in)},
scale only axis,
xmin=0,
xmax=1,
xlabel style={font=\color{white!15!black}},
xlabel={$\tau$},
ymin=506,
ymax=1251,
axis background/.style={fill=white},
title style={font=\bfseries},
title={$\gamma\text{= 0.40}$}
]
\addplot [color=mycolor1, line width=2.0pt, forget plot]
  table[row sep=crcr]{%
0.1	534\\
0.109090909090909	536\\
0.118181818181818	534\\
0.127272727272727	529\\
0.136363636363636	524\\
0.145454545454545	514\\
0.154545454545455	508\\
0.163636363636364	506\\
0.172727272727273	512\\
0.181818181818182	523\\
0.190909090909091	538\\
0.2	547\\
0.209090909090909	563\\
0.218181818181818	584\\
0.227272727272727	586\\
0.236363636363636	598\\
0.245454545454545	612\\
0.254545454545455	634\\
0.263636363636364	647\\
0.272727272727273	663\\
0.281818181818182	670\\
0.290909090909091	692\\
0.3	705\\
0.309090909090909	719\\
0.318181818181818	730\\
0.327272727272727	745\\
0.336363636363636	749\\
0.345454545454545	767\\
0.354545454545454	778\\
0.363636363636364	792\\
0.372727272727273	799\\
0.381818181818182	811\\
0.390909090909091	822\\
0.4	830\\
0.409090909090909	844\\
0.418181818181818	854\\
0.427272727272727	865\\
0.436363636363636	870\\
0.445454545454545	880\\
0.454545454545455	885\\
0.463636363636364	898\\
0.472727272727273	910\\
0.481818181818182	909\\
0.490909090909091	927\\
0.5	935\\
0.509090909090909	943\\
0.518181818181818	949\\
0.527272727272727	958\\
0.536363636363636	966\\
0.545454545454546	972\\
0.554545454545455	980\\
0.563636363636364	988\\
0.572727272727273	997\\
0.581818181818182	1005\\
0.590909090909091	1014\\
0.6	1017\\
0.609090909090909	1028\\
0.618181818181818	1034\\
0.627272727272727	1040\\
0.636363636363636	1045\\
0.645454545454545	1054\\
0.654545454545455	1060\\
0.663636363636364	1069\\
0.672727272727273	1075\\
0.681818181818182	1077\\
0.690909090909091	1083\\
0.7	1092\\
0.709090909090909	1099\\
0.718181818181818	1102\\
0.727272727272727	1107\\
0.736363636363636	1113\\
0.745454545454545	1118\\
0.754545454545455	1124\\
0.763636363636364	1131\\
0.772727272727273	1135\\
0.781818181818182	1140\\
0.790909090909091	1146\\
0.8	1152\\
0.809090909090909	1158\\
0.818181818181818	1159\\
0.827272727272727	1167\\
0.836363636363636	1165\\
0.845454545454545	1171\\
0.854545454545454	1173\\
0.863636363636364	1179\\
0.872727272727273	1185\\
0.881818181818182	1187\\
0.890909090909091	1193\\
0.9	1206\\
0.909090909090909	1201\\
0.918181818181818	1208\\
0.927272727272727	1210\\
0.936363636363636	1216\\
0.945454545454545	1218\\
0.954545454545455	1225\\
0.963636363636364	1226\\
0.972727272727273	1233\\
0.981818181818182	1235\\
0.990909090909091	1244\\
1	1251\\
};
\end{axis}

\begin{axis}[%
width=1.66in,
height=1.258in,
at={(3.195in,2.39in)},
scale only axis,
xmin=0,
xmax=1,
xlabel style={font=\color{white!15!black}},
xlabel={$\tau$},
ymin=575,
ymax=1242,
axis background/.style={fill=white},
title style={font=\bfseries},
title={$\gamma\text{= 0.50}$}
]
\addplot [color=mycolor1, line width=2.0pt, forget plot]
  table[row sep=crcr]{%
0.1	611\\
0.109090909090909	611\\
0.118181818181818	614\\
0.127272727272727	613\\
0.136363636363636	616\\
0.145454545454545	614\\
0.154545454545455	609\\
0.163636363636364	599\\
0.172727272727273	592\\
0.181818181818182	585\\
0.190909090909091	580\\
0.2	575\\
0.209090909090909	576\\
0.218181818181818	584\\
0.227272727272727	594\\
0.236363636363636	607\\
0.245454545454545	622\\
0.254545454545455	632\\
0.263636363636364	650\\
0.272727272727273	668\\
0.281818181818182	675\\
0.290909090909091	688\\
0.3	701\\
0.309090909090909	703\\
0.318181818181818	724\\
0.327272727272727	728\\
0.336363636363636	751\\
0.345454545454545	764\\
0.354545454545454	769\\
0.363636363636364	785\\
0.372727272727273	798\\
0.381818181818182	803\\
0.390909090909091	819\\
0.4	821\\
0.409090909090909	841\\
0.418181818181818	844\\
0.427272727272727	860\\
0.436363636363636	871\\
0.445454545454545	876\\
0.454545454545455	887\\
0.463636363636364	896\\
0.472727272727273	903\\
0.481818181818182	914\\
0.490909090909091	921\\
0.5	926\\
0.509090909090909	937\\
0.518181818181818	945\\
0.527272727272727	953\\
0.536363636363636	960\\
0.545454545454546	967\\
0.554545454545455	974\\
0.563636363636364	983\\
0.572727272727273	983\\
0.581818181818182	998\\
0.590909090909091	1003\\
0.6	1003\\
0.609090909090909	1018\\
0.618181818181818	1023\\
0.627272727272727	1030\\
0.636363636363636	1036\\
0.645454545454545	1043\\
0.654545454545455	1049\\
0.663636363636364	1057\\
0.672727272727273	1061\\
0.681818181818182	1070\\
0.690909090909091	1074\\
0.7	1079\\
0.709090909090909	1085\\
0.718181818181818	1094\\
0.727272727272727	1099\\
0.736363636363636	1105\\
0.745454545454545	1111\\
0.754545454545455	1114\\
0.763636363636364	1121\\
0.772727272727273	1127\\
0.781818181818182	1130\\
0.790909090909091	1136\\
0.8	1142\\
0.809090909090909	1147\\
0.818181818181818	1153\\
0.827272727272727	1159\\
0.836363636363636	1166\\
0.845454545454545	1165\\
0.854545454545454	1171\\
0.863636363636364	1177\\
0.872727272727273	1183\\
0.881818181818182	1184\\
0.890909090909091	1191\\
0.9	1195\\
0.909090909090909	1197\\
0.918181818181818	1203\\
0.927272727272727	1209\\
0.936363636363636	1212\\
0.945454545454545	1218\\
0.954545454545455	1221\\
0.963636363636364	1220\\
0.972727272727273	1227\\
0.981818181818182	1233\\
0.990909090909091	1236\\
1	1242\\
};
\end{axis}

\begin{axis}[%
width=1.66in,
height=1.258in,
at={(5.379in,2.39in)},
scale only axis,
xmin=0,
xmax=1,
xlabel style={font=\color{white!15!black}},
xlabel={$\tau$},
ymin=600,
ymax=1238,
axis background/.style={fill=white},
title style={font=\bfseries},
title={$\gamma\text{= 0.60}$}
]
\addplot [color=mycolor1, line width=2.0pt, forget plot]
  table[row sep=crcr]{%
0.1	681\\
0.109090909090909	679\\
0.118181818181818	679\\
0.127272727272727	679\\
0.136363636363636	680\\
0.145454545454545	680\\
0.154545454545455	680\\
0.163636363636364	682\\
0.172727272727273	676\\
0.181818181818182	674\\
0.190909090909091	665\\
0.2	659\\
0.209090909090909	652\\
0.218181818181818	646\\
0.227272727272727	641\\
0.236363636363636	638\\
0.245454545454545	643\\
0.254545454545455	646\\
0.263636363636364	657\\
0.272727272727273	670\\
0.281818181818182	682\\
0.290909090909091	695\\
0.3	703\\
0.309090909090909	716\\
0.318181818181818	733\\
0.327272727272727	750\\
0.336363636363636	756\\
0.345454545454545	767\\
0.354545454545454	771\\
0.363636363636364	790\\
0.372727272727273	798\\
0.381818181818182	810\\
0.390909090909091	812\\
0.4	828\\
0.409090909090909	842\\
0.418181818181818	853\\
0.427272727272727	863\\
0.436363636363636	872\\
0.445454545454545	878\\
0.454545454545455	887\\
0.463636363636364	898\\
0.472727272727273	904\\
0.481818181818182	907\\
0.490909090909091	920\\
0.5	927\\
0.509090909090909	930\\
0.518181818181818	944\\
0.527272727272727	950\\
0.536363636363636	958\\
0.545454545454546	966\\
0.554545454545455	974\\
0.563636363636364	979\\
0.572727272727273	986\\
0.581818181818182	994\\
0.590909090909091	1003\\
0.6	1006\\
0.609090909090909	1013\\
0.618181818181818	1021\\
0.627272727272727	1026\\
0.636363636363636	1035\\
0.645454545454545	1040\\
0.654545454545455	1046\\
0.663636363636364	1050\\
0.672727272727273	1059\\
0.681818181818182	1057\\
0.690909090909091	1062\\
0.7	1070\\
0.709090909090909	1082\\
0.718181818181818	1080\\
0.727272727272727	1085\\
0.736363636363636	1100\\
0.745454545454545	1105\\
0.754545454545455	1109\\
0.763636363636364	1116\\
0.772727272727273	1120\\
0.781818181818182	1124\\
0.790909090909091	1131\\
0.8	1136\\
0.809090909090909	1140\\
0.818181818181818	1146\\
0.827272727272727	1151\\
0.836363636363636	1156\\
0.845454545454545	1158\\
0.854545454545454	1164\\
0.863636363636364	1169\\
0.872727272727273	1175\\
0.881818181818182	1181\\
0.890909090909091	1183\\
0.9	1187\\
0.909090909090909	1193\\
0.918181818181818	1197\\
0.927272727272727	1204\\
0.936363636363636	1207\\
0.945454545454545	1209\\
0.954545454545455	1215\\
0.963636363636364	1221\\
0.972727272727273	1223\\
0.981818181818182	1230\\
0.990909090909091	1232\\
1	1238\\
};
\end{axis}

\begin{axis}[%
width=1.66in,
height=1.258in,
at={(1.011in,0.642in)},
scale only axis,
xmin=0,
xmax=1,
xlabel style={font=\color{white!15!black}},
xlabel={$\tau$},
ymin=600,
ymax=1229,
axis background/.style={fill=white},
title style={font=\bfseries},
title={$\gamma\text{= 0.70}$}
]
\addplot [color=mycolor1, line width=2.0pt, forget plot]
  table[row sep=crcr]{%
0.1	737\\
0.109090909090909	735\\
0.118181818181818	733\\
0.127272727272727	732\\
0.136363636363636	733\\
0.145454545454545	729\\
0.154545454545455	730\\
0.163636363636364	731\\
0.172727272727273	731\\
0.181818181818182	736\\
0.190909090909091	734\\
0.2	728\\
0.209090909090909	725\\
0.218181818181818	721\\
0.227272727272727	716\\
0.236363636363636	711\\
0.245454545454545	710\\
0.254545454545455	706\\
0.263636363636364	703\\
0.272727272727273	706\\
0.281818181818182	709\\
0.290909090909091	710\\
0.3	719\\
0.309090909090909	729\\
0.318181818181818	744\\
0.327272727272727	754\\
0.336363636363636	764\\
0.345454545454545	778\\
0.354545454545454	782\\
0.363636363636364	794\\
0.372727272727273	812\\
0.381818181818182	823\\
0.390909090909091	833\\
0.4	839\\
0.409090909090909	847\\
0.418181818181818	856\\
0.427272727272727	864\\
0.436363636363636	873\\
0.445454545454545	878\\
0.454545454545455	888\\
0.463636363636364	898\\
0.472727272727273	904\\
0.481818181818182	915\\
0.490909090909091	923\\
0.5	930\\
0.509090909090909	940\\
0.518181818181818	947\\
0.527272727272727	952\\
0.536363636363636	960\\
0.545454545454546	967\\
0.554545454545455	974\\
0.563636363636364	974\\
0.572727272727273	988\\
0.581818181818182	996\\
0.590909090909091	992\\
0.6	1007\\
0.609090909090909	1015\\
0.618181818181818	1021\\
0.627272727272727	1026\\
0.636363636363636	1035\\
0.645454545454545	1039\\
0.654545454545455	1045\\
0.663636363636364	1050\\
0.672727272727273	1058\\
0.681818181818182	1063\\
0.690909090909091	1067\\
0.7	1074\\
0.709090909090909	1079\\
0.718181818181818	1083\\
0.727272727272727	1088\\
0.736363636363636	1093\\
0.745454545454545	1103\\
0.754545454545455	1105\\
0.763636363636364	1111\\
0.772727272727273	1116\\
0.781818181818182	1121\\
0.790909090909091	1127\\
0.8	1124\\
0.809090909090909	1136\\
0.818181818181818	1135\\
0.827272727272727	1147\\
0.836363636363636	1141\\
0.845454545454545	1147\\
0.854545454545454	1158\\
0.863636363636364	1164\\
0.872727272727273	1170\\
0.881818181818182	1176\\
0.890909090909091	1177\\
0.9	1181\\
0.909090909090909	1187\\
0.918181818181818	1193\\
0.927272727272727	1195\\
0.936363636363636	1201\\
0.945454545454545	1201\\
0.954545454545455	1207\\
0.963636363636364	1213\\
0.972727272727273	1214\\
0.981818181818182	1220\\
0.990909090909091	1222\\
1	1229\\
};
\end{axis}

\begin{axis}[%
width=1.66in,
height=1.258in,
at={(3.195in,0.642in)},
scale only axis,
xmin=0,
xmax=1,
xlabel style={font=\color{white!15!black}},
xlabel={$\tau$},
ymin=764,
ymax=1223,
axis background/.style={fill=white},
title style={font=\bfseries},
title={$\gamma\text{= 0.80}$}
]
\addplot [color=mycolor1, line width=2.0pt, forget plot]
  table[row sep=crcr]{%
0.1	780\\
0.109090909090909	779\\
0.118181818181818	777\\
0.127272727272727	776\\
0.136363636363636	775\\
0.145454545454545	773\\
0.154545454545455	775\\
0.163636363636364	772\\
0.172727272727273	773\\
0.181818181818182	774\\
0.190909090909091	775\\
0.2	775\\
0.209090909090909	775\\
0.218181818181818	776\\
0.227272727272727	774\\
0.236363636363636	771\\
0.245454545454545	773\\
0.254545454545455	769\\
0.263636363636364	766\\
0.272727272727273	767\\
0.281818181818182	768\\
0.290909090909091	765\\
0.3	764\\
0.309090909090909	768\\
0.318181818181818	768\\
0.327272727272727	772\\
0.336363636363636	775\\
0.345454545454545	786\\
0.354545454545454	797\\
0.363636363636364	805\\
0.372727272727273	817\\
0.381818181818182	829\\
0.390909090909091	837\\
0.4	840\\
0.409090909090909	848\\
0.418181818181818	857\\
0.427272727272727	872\\
0.436363636363636	883\\
0.445454545454545	894\\
0.454545454545455	891\\
0.463636363636364	902\\
0.472727272727273	904\\
0.481818181818182	917\\
0.490909090909091	915\\
0.5	932\\
0.509090909090909	938\\
0.518181818181818	945\\
0.527272727272727	952\\
0.536363636363636	959\\
0.545454545454546	966\\
0.554545454545455	974\\
0.563636363636364	974\\
0.572727272727273	989\\
0.581818181818182	997\\
0.590909090909091	1000\\
0.6	1007\\
0.609090909090909	1007\\
0.618181818181818	1018\\
0.627272727272727	1026\\
0.636363636363636	1034\\
0.645454545454545	1030\\
0.654545454545455	1045\\
0.663636363636364	1050\\
0.672727272727273	1046\\
0.681818181818182	1054\\
0.690909090909091	1066\\
0.7	1070\\
0.709090909090909	1077\\
0.718181818181818	1082\\
0.727272727272727	1088\\
0.736363636363636	1093\\
0.745454545454545	1096\\
0.754545454545455	1105\\
0.763636363636364	1108\\
0.772727272727273	1113\\
0.781818181818182	1119\\
0.790909090909091	1124\\
0.8	1128\\
0.809090909090909	1133\\
0.818181818181818	1138\\
0.827272727272727	1139\\
0.836363636363636	1144\\
0.845454545454545	1150\\
0.854545454545454	1155\\
0.863636363636364	1159\\
0.872727272727273	1164\\
0.881818181818182	1166\\
0.890909090909091	1171\\
0.9	1178\\
0.909090909090909	1176\\
0.918181818181818	1177\\
0.927272727272727	1190\\
0.936363636363636	1195\\
0.945454545454545	1197\\
0.954545454545455	1203\\
0.963636363636364	1197\\
0.972727272727273	1209\\
0.981818181818182	1215\\
0.990909090909091	1217\\
1	1223\\
};
\end{axis}

\begin{axis}[%
width=1.66in,
height=1.258in,
at={(5.379in,0.642in)},
scale only axis,
xmin=0,
xmax=1,
xlabel style={font=\color{white!15!black}},
xlabel={$\tau$},
ymin=800,
ymax=1214,
axis background/.style={fill=white},
title style={font=\bfseries},
title={$\gamma\text{= 0.90}$}
]
\addplot [color=mycolor1, line width=2.0pt, forget plot]
  table[row sep=crcr]{%
0.1	815\\
0.109090909090909	814\\
0.118181818181818	813\\
0.127272727272727	812\\
0.136363636363636	810\\
0.145454545454545	809\\
0.154545454545455	808\\
0.163636363636364	806\\
0.172727272727273	808\\
0.181818181818182	806\\
0.190909090909091	807\\
0.2	808\\
0.209090909090909	809\\
0.218181818181818	809\\
0.227272727272727	813\\
0.236363636363636	816\\
0.245454545454545	817\\
0.254545454545455	819\\
0.263636363636364	818\\
0.272727272727273	821\\
0.281818181818182	819\\
0.290909090909091	821\\
0.3	819\\
0.309090909090909	821\\
0.318181818181818	819\\
0.327272727272727	818\\
0.336363636363636	823\\
0.345454545454545	823\\
0.354545454545454	824\\
0.363636363636364	825\\
0.372727272727273	829\\
0.381818181818182	833\\
0.390909090909091	841\\
0.4	853\\
0.409090909090909	862\\
0.418181818181818	870\\
0.427272727272727	878\\
0.436363636363636	887\\
0.445454545454545	895\\
0.454545454545455	896\\
0.463636363636364	905\\
0.472727272727273	911\\
0.481818181818182	923\\
0.490909090909091	935\\
0.5	942\\
0.509090909090909	942\\
0.518181818181818	948\\
0.527272727272727	954\\
0.536363636363636	961\\
0.545454545454546	967\\
0.554545454545455	966\\
0.563636363636364	980\\
0.572727272727273	987\\
0.581818181818182	994\\
0.590909090909091	990\\
0.6	1005\\
0.609090909090909	1013\\
0.618181818181818	1017\\
0.627272727272727	1026\\
0.636363636363636	1022\\
0.645454545454545	1037\\
0.654545454545455	1041\\
0.663636363636364	1049\\
0.672727272727273	1054\\
0.681818181818182	1060\\
0.690909090909091	1065\\
0.7	1069\\
0.709090909090909	1074\\
0.718181818181818	1082\\
0.727272727272727	1079\\
0.736363636363636	1090\\
0.745454545454545	1095\\
0.754545454545455	1092\\
0.763636363636364	1097\\
0.772727272727273	1108\\
0.781818181818182	1113\\
0.790909090909091	1117\\
0.8	1122\\
0.809090909090909	1127\\
0.818181818181818	1133\\
0.827272727272727	1138\\
0.836363636363636	1141\\
0.845454545454545	1146\\
0.854545454545454	1152\\
0.863636363636364	1153\\
0.872727272727273	1157\\
0.881818181818182	1164\\
0.890909090909091	1169\\
0.9	1169\\
0.909090909090909	1175\\
0.918181818181818	1180\\
0.927272727272727	1185\\
0.936363636363636	1189\\
0.945454545454545	1192\\
0.954545454545455	1195\\
0.963636363636364	1200\\
0.972727272727273	1204\\
0.981818181818182	1206\\
0.990909090909091	1212\\
1	1214\\
};
\end{axis}
\end{tikzpicture}%}}
 \hfill 
\subfloat[][$N=20$]
{\resizebox{0.45\textwidth}{!}{ % This file was created by matlab2tikz.
%
%The latest updates can be retrieved from
%  http://www.mathworks.com/matlabcentral/fileexchange/22022-matlab2tikz-matlab2tikz
%where you can also make suggestions and rate matlab2tikz.
%
\definecolor{mycolor1}{rgb}{0.00000,0.44700,0.74100}%
%
\begin{tikzpicture}

\begin{axis}[%
width=1.66in,
height=1.258in,
at={(1.011in,4.137in)},
scale only axis,
xmin=0,
xmax=1,
xlabel style={font=\color{white!15!black}},
xlabel={$\tau$},
ymin=777,
ymax=1463,
axis background/.style={fill=white},
title style={font=\bfseries},
title={$\gamma\text{= 0.10}$}
]
\addplot [color=mycolor1, line width=2.0pt, forget plot]
  table[row sep=crcr]{%
0.1	777\\
0.109090909090909	830\\
0.118181818181818	873\\
0.127272727272727	911\\
0.136363636363636	946\\
0.145454545454545	979\\
0.154545454545455	1007\\
0.163636363636364	1031\\
0.172727272727273	1057\\
0.181818181818182	1080\\
0.190909090909091	1099\\
0.2	1120\\
0.209090909090909	1137\\
0.218181818181818	1152\\
0.227272727272727	1167\\
0.236363636363636	1183\\
0.245454545454545	1195\\
0.254545454545455	1208\\
0.263636363636364	1217\\
0.272727272727273	1227\\
0.281818181818182	1237\\
0.290909090909091	1248\\
0.3	1248\\
0.309090909090909	1255\\
0.318181818181818	1262\\
0.327272727272727	1270\\
0.336363636363636	1278\\
0.345454545454545	1279\\
0.354545454545454	1284\\
0.363636363636364	1293\\
0.372727272727273	1298\\
0.381818181818182	1303\\
0.390909090909091	1308\\
0.4	1314\\
0.409090909090909	1316\\
0.418181818181818	1321\\
0.427272727272727	1327\\
0.436363636363636	1329\\
0.445454545454545	1336\\
0.454545454545455	1338\\
0.463636363636364	1344\\
0.472727272727273	1347\\
0.481818181818182	1353\\
0.490909090909091	1355\\
0.5	1358\\
0.509090909090909	1361\\
0.518181818181818	1363\\
0.527272727272727	1370\\
0.536363636363636	1373\\
0.545454545454546	1376\\
0.554545454545455	1379\\
0.563636363636364	1381\\
0.572727272727273	1384\\
0.581818181818182	1389\\
0.590909090909091	1388\\
0.6	1391\\
0.609090909090909	1397\\
0.618181818181818	1400\\
0.627272727272727	1403\\
0.636363636363636	1406\\
0.645454545454545	1405\\
0.654545454545455	1409\\
0.663636363636364	1412\\
0.672727272727273	1411\\
0.681818181818182	1415\\
0.690909090909091	1417\\
0.7	1415\\
0.709090909090909	1419\\
0.718181818181818	1424\\
0.727272727272727	1423\\
0.736363636363636	1427\\
0.745454545454545	1426\\
0.754545454545455	1429\\
0.763636363636364	1431\\
0.772727272727273	1430\\
0.781818181818182	1433\\
0.790909090909091	1432\\
0.8	1435\\
0.809090909090909	1434\\
0.818181818181818	1437\\
0.827272727272727	1437\\
0.836363636363636	1440\\
0.845454545454545	1440\\
0.854545454545454	1441\\
0.863636363636364	1445\\
0.872727272727273	1444\\
0.881818181818182	1448\\
0.890909090909091	1448\\
0.9	1452\\
0.909090909090909	1451\\
0.918181818181818	1451\\
0.927272727272727	1454\\
0.936363636363636	1454\\
0.945454545454545	1454\\
0.954545454545455	1457\\
0.963636363636364	1457\\
0.972727272727273	1460\\
0.981818181818182	1460\\
0.990909090909091	1459\\
1	1463\\
};
\end{axis}

\begin{axis}[%
width=1.66in,
height=1.258in,
at={(3.195in,4.137in)},
scale only axis,
xmin=0,
xmax=1,
xlabel style={font=\color{white!15!black}},
xlabel={$\tau$},
ymin=500,
ymax=1519,
axis background/.style={fill=white},
title style={font=\bfseries},
title={$\gamma\text{= 0.20}$}
]
\addplot [color=mycolor1, line width=2.0pt, forget plot]
  table[row sep=crcr]{%
0.1	617\\
0.109090909090909	670\\
0.118181818181818	712\\
0.127272727272727	747\\
0.136363636363636	783\\
0.145454545454545	811\\
0.154545454545455	841\\
0.163636363636364	868\\
0.172727272727273	892\\
0.181818181818182	914\\
0.190909090909091	938\\
0.2	958\\
0.209090909090909	979\\
0.218181818181818	996\\
0.227272727272727	1015\\
0.236363636363636	1034\\
0.245454545454545	1049\\
0.254545454545455	1065\\
0.263636363636364	1082\\
0.272727272727273	1095\\
0.281818181818182	1112\\
0.290909090909091	1126\\
0.3	1136\\
0.309090909090909	1151\\
0.318181818181818	1162\\
0.327272727272727	1173\\
0.336363636363636	1184\\
0.345454545454545	1196\\
0.354545454545454	1208\\
0.363636363636364	1220\\
0.372727272727273	1229\\
0.381818181818182	1240\\
0.390909090909091	1249\\
0.4	1257\\
0.409090909090909	1267\\
0.418181818181818	1276\\
0.427272727272727	1285\\
0.436363636363636	1290\\
0.445454545454545	1300\\
0.454545454545455	1305\\
0.463636363636364	1315\\
0.472727272727273	1321\\
0.481818181818182	1327\\
0.490909090909091	1336\\
0.5	1342\\
0.509090909090909	1348\\
0.518181818181818	1354\\
0.527272727272727	1360\\
0.536363636363636	1367\\
0.545454545454546	1369\\
0.554545454545455	1375\\
0.563636363636364	1382\\
0.572727272727273	1388\\
0.581818181818182	1390\\
0.590909090909091	1397\\
0.6	1392\\
0.609090909090909	1399\\
0.618181818181818	1401\\
0.627272727272727	1408\\
0.636363636363636	1410\\
0.645454545454545	1417\\
0.654545454545455	1420\\
0.663636363636364	1423\\
0.672727272727273	1429\\
0.681818181818182	1432\\
0.690909090909091	1431\\
0.7	1434\\
0.709090909090909	1437\\
0.718181818181818	1440\\
0.727272727272727	1443\\
0.736363636363636	1450\\
0.745454545454545	1453\\
0.754545454545455	1456\\
0.763636363636364	1459\\
0.772727272727273	1462\\
0.781818181818182	1465\\
0.790909090909091	1464\\
0.8	1467\\
0.809090909090909	1471\\
0.818181818181818	1474\\
0.827272727272727	1477\\
0.836363636363636	1480\\
0.845454545454545	1483\\
0.854545454545454	1486\\
0.863636363636364	1485\\
0.872727272727273	1489\\
0.881818181818182	1492\\
0.890909090909091	1495\\
0.9	1495\\
0.909090909090909	1498\\
0.918181818181818	1501\\
0.927272727272727	1500\\
0.936363636363636	1504\\
0.945454545454545	1507\\
0.954545454545455	1506\\
0.963636363636364	1509\\
0.972727272727273	1513\\
0.981818181818182	1512\\
0.990909090909091	1515\\
1	1519\\
};
\end{axis}

\begin{axis}[%
width=1.66in,
height=1.258in,
at={(5.379in,4.137in)},
scale only axis,
xmin=0,
xmax=1,
xlabel style={font=\color{white!15!black}},
xlabel={$\tau$},
ymin=500,
ymax=1553,
axis background/.style={fill=white},
title style={font=\bfseries},
title={$\gamma\text{= 0.30}$}
]
\addplot [color=mycolor1, line width=2.0pt, forget plot]
  table[row sep=crcr]{%
0.1	533\\
0.109090909090909	569\\
0.118181818181818	599\\
0.127272727272727	636\\
0.136363636363636	671\\
0.145454545454545	704\\
0.154545454545455	735\\
0.163636363636364	770\\
0.172727272727273	795\\
0.181818181818182	824\\
0.190909090909091	849\\
0.2	871\\
0.209090909090909	896\\
0.218181818181818	915\\
0.227272727272727	937\\
0.236363636363636	955\\
0.245454545454545	974\\
0.254545454545455	994\\
0.263636363636364	1009\\
0.272727272727273	1026\\
0.281818181818182	1044\\
0.290909090909091	1061\\
0.3	1075\\
0.309090909090909	1089\\
0.318181818181818	1104\\
0.327272727272727	1118\\
0.336363636363636	1133\\
0.345454545454545	1144\\
0.354545454545454	1160\\
0.363636363636364	1171\\
0.372727272727273	1182\\
0.381818181818182	1194\\
0.390909090909091	1202\\
0.4	1214\\
0.409090909090909	1226\\
0.418181818181818	1235\\
0.427272727272727	1243\\
0.436363636363636	1255\\
0.445454545454545	1264\\
0.454545454545455	1273\\
0.463636363636364	1282\\
0.472727272727273	1291\\
0.481818181818182	1300\\
0.490909090909091	1305\\
0.5	1314\\
0.509090909090909	1324\\
0.518181818181818	1329\\
0.527272727272727	1338\\
0.536363636363636	1344\\
0.545454545454546	1354\\
0.554545454545455	1360\\
0.563636363636364	1366\\
0.572727272727273	1370\\
0.581818181818182	1376\\
0.590909090909091	1382\\
0.6	1388\\
0.609090909090909	1394\\
0.618181818181818	1400\\
0.627272727272727	1406\\
0.636363636363636	1412\\
0.645454545454545	1418\\
0.654545454545455	1425\\
0.663636363636364	1427\\
0.672727272727273	1433\\
0.681818181818182	1440\\
0.690909090909091	1442\\
0.7	1449\\
0.709090909090909	1451\\
0.718181818181818	1458\\
0.727272727272727	1460\\
0.736363636363636	1467\\
0.745454545454545	1469\\
0.754545454545455	1472\\
0.763636363636364	1479\\
0.772727272727273	1481\\
0.781818181818182	1484\\
0.790909090909091	1491\\
0.8	1494\\
0.809090909090909	1497\\
0.818181818181818	1500\\
0.827272727272727	1507\\
0.836363636363636	1509\\
0.845454545454545	1512\\
0.854545454545454	1515\\
0.863636363636364	1518\\
0.872727272727273	1521\\
0.881818181818182	1524\\
0.890909090909091	1520\\
0.9	1523\\
0.909090909090909	1526\\
0.918181818181818	1529\\
0.927272727272727	1532\\
0.936363636363636	1535\\
0.945454545454545	1538\\
0.954545454545455	1541\\
0.963636363636364	1545\\
0.972727272727273	1548\\
0.981818181818182	1551\\
0.990909090909091	1550\\
1	1553\\
};
\end{axis}

\begin{axis}[%
width=1.66in,
height=1.258in,
at={(1.011in,2.39in)},
scale only axis,
xmin=0,
xmax=1,
xlabel style={font=\color{white!15!black}},
xlabel={$\tau$},
ymin=500,
ymax=1576,
axis background/.style={fill=white},
title style={font=\bfseries},
title={$\gamma\text{= 0.40}$}
]
\addplot [color=mycolor1, line width=2.0pt, forget plot]
  table[row sep=crcr]{%
0.1	513\\
0.109090909090909	536\\
0.118181818181818	566\\
0.127272727272727	595\\
0.136363636363636	625\\
0.145454545454545	655\\
0.154545454545455	681\\
0.163636363636364	715\\
0.172727272727273	742\\
0.181818181818182	768\\
0.190909090909091	796\\
0.2	825\\
0.209090909090909	848\\
0.218181818181818	873\\
0.227272727272727	898\\
0.236363636363636	918\\
0.245454545454545	935\\
0.254545454545455	958\\
0.263636363636364	973\\
0.272727272727273	992\\
0.281818181818182	1009\\
0.290909090909091	1025\\
0.3	1042\\
0.309090909090909	1059\\
0.318181818181818	1072\\
0.327272727272727	1086\\
0.336363636363636	1100\\
0.345454545454545	1114\\
0.354545454545454	1128\\
0.363636363636364	1139\\
0.372727272727273	1151\\
0.381818181818182	1166\\
0.390909090909091	1177\\
0.4	1189\\
0.409090909090909	1197\\
0.418181818181818	1209\\
0.427272727272727	1220\\
0.436363636363636	1228\\
0.445454545454545	1241\\
0.454545454545455	1249\\
0.463636363636364	1257\\
0.472727272727273	1270\\
0.481818181818182	1278\\
0.490909090909091	1287\\
0.5	1296\\
0.509090909090909	1305\\
0.518181818181818	1310\\
0.527272727272727	1318\\
0.536363636363636	1328\\
0.545454545454546	1333\\
0.554545454545455	1342\\
0.563636363636364	1347\\
0.572727272727273	1357\\
0.581818181818182	1362\\
0.590909090909091	1372\\
0.6	1377\\
0.609090909090909	1383\\
0.618181818181818	1389\\
0.627272727272727	1394\\
0.636363636363636	1404\\
0.645454545454545	1410\\
0.654545454545455	1416\\
0.663636363636364	1422\\
0.672727272727273	1428\\
0.681818181818182	1430\\
0.690909090909091	1436\\
0.7	1442\\
0.709090909090909	1448\\
0.718181818181818	1455\\
0.727272727272727	1461\\
0.736363636363636	1463\\
0.745454545454545	1470\\
0.754545454545455	1476\\
0.763636363636364	1477\\
0.772727272727273	1483\\
0.781818181818182	1486\\
0.790909090909091	1492\\
0.8	1495\\
0.809090909090909	1501\\
0.818181818181818	1504\\
0.827272727272727	1507\\
0.836363636363636	1513\\
0.845454545454545	1516\\
0.854545454545454	1523\\
0.863636363636364	1525\\
0.872727272727273	1528\\
0.881818181818182	1531\\
0.890909090909091	1537\\
0.9	1540\\
0.909090909090909	1543\\
0.918181818181818	1546\\
0.927272727272727	1549\\
0.936363636363636	1556\\
0.945454545454545	1559\\
0.954545454545455	1562\\
0.963636363636364	1565\\
0.972727272727273	1568\\
0.981818181818182	1570\\
0.990909090909091	1573\\
1	1576\\
};
\end{axis}

\begin{axis}[%
width=1.66in,
height=1.258in,
at={(3.195in,2.39in)},
scale only axis,
xmin=0,
xmax=1,
xlabel style={font=\color{white!15!black}},
xlabel={$\tau$},
ymin=500,
ymax=1582,
axis background/.style={fill=white},
title style={font=\bfseries},
title={$\gamma\text{= 0.50}$}
]
\addplot [color=mycolor1, line width=2.0pt, forget plot]
  table[row sep=crcr]{%
0.1	570\\
0.109090909090909	560\\
0.118181818181818	570\\
0.127272727272727	594\\
0.136363636363636	616\\
0.145454545454545	642\\
0.154545454545455	675\\
0.163636363636364	697\\
0.172727272727273	725\\
0.181818181818182	749\\
0.190909090909091	770\\
0.2	796\\
0.209090909090909	822\\
0.218181818181818	847\\
0.227272727272727	869\\
0.236363636363636	889\\
0.245454545454545	909\\
0.254545454545455	931\\
0.263636363636364	950\\
0.272727272727273	969\\
0.281818181818182	989\\
0.290909090909091	1007\\
0.3	1022\\
0.309090909090909	1039\\
0.318181818181818	1051\\
0.327272727272727	1068\\
0.336363636363636	1080\\
0.345454545454545	1093\\
0.354545454545454	1108\\
0.363636363636364	1122\\
0.372727272727273	1132\\
0.381818181818182	1147\\
0.390909090909091	1158\\
0.4	1168\\
0.409090909090909	1179\\
0.418181818181818	1190\\
0.427272727272727	1202\\
0.436363636363636	1213\\
0.445454545454545	1221\\
0.454545454545455	1232\\
0.463636363636364	1242\\
0.472727272727273	1250\\
0.481818181818182	1262\\
0.490909090909091	1270\\
0.5	1278\\
0.509090909090909	1287\\
0.518181818181818	1296\\
0.527272727272727	1305\\
0.536363636363636	1313\\
0.545454545454546	1322\\
0.554545454545455	1328\\
0.563636363636364	1337\\
0.572727272727273	1346\\
0.581818181818182	1351\\
0.590909090909091	1360\\
0.6	1365\\
0.609090909090909	1371\\
0.618181818181818	1380\\
0.627272727272727	1385\\
0.636363636363636	1391\\
0.645454545454545	1401\\
0.654545454545455	1406\\
0.663636363636364	1412\\
0.672727272727273	1417\\
0.681818181818182	1423\\
0.690909090909091	1429\\
0.7	1435\\
0.709090909090909	1441\\
0.718181818181818	1447\\
0.727272727272727	1453\\
0.736363636363636	1455\\
0.745454545454545	1461\\
0.754545454545455	1467\\
0.763636363636364	1473\\
0.772727272727273	1479\\
0.781818181818182	1481\\
0.790909090909091	1488\\
0.8	1494\\
0.809090909090909	1496\\
0.818181818181818	1503\\
0.827272727272727	1505\\
0.836363636363636	1511\\
0.845454545454545	1514\\
0.854545454545454	1520\\
0.863636363636364	1523\\
0.872727272727273	1529\\
0.881818181818182	1532\\
0.890909090909091	1538\\
0.9	1541\\
0.909090909090909	1544\\
0.918181818181818	1551\\
0.927272727272727	1553\\
0.936363636363636	1556\\
0.945454545454545	1563\\
0.954545454545455	1564\\
0.963636363636364	1567\\
0.972727272727273	1570\\
0.981818181818182	1576\\
0.990909090909091	1579\\
1	1582\\
};
\end{axis}

\begin{axis}[%
width=1.66in,
height=1.258in,
at={(5.379in,2.39in)},
scale only axis,
xmin=0,
xmax=1,
xlabel style={font=\color{white!15!black}},
xlabel={$\tau$},
ymin=500,
ymax=1577,
axis background/.style={fill=white},
title style={font=\bfseries},
title={$\gamma\text{= 0.60}$}
]
\addplot [color=mycolor1, line width=2.0pt, forget plot]
  table[row sep=crcr]{%
0.1	662\\
0.109090909090909	652\\
0.118181818181818	634\\
0.127272727272727	628\\
0.136363636363636	637\\
0.145454545454545	656\\
0.154545454545455	681\\
0.163636363636364	696\\
0.172727272727273	721\\
0.181818181818182	745\\
0.190909090909091	769\\
0.2	793\\
0.209090909090909	816\\
0.218181818181818	835\\
0.227272727272727	851\\
0.236363636363636	871\\
0.245454545454545	892\\
0.254545454545455	913\\
0.263636363636364	931\\
0.272727272727273	952\\
0.281818181818182	969\\
0.290909090909091	984\\
0.3	1003\\
0.309090909090909	1019\\
0.318181818181818	1035\\
0.327272727272727	1052\\
0.336363636363636	1065\\
0.345454545454545	1080\\
0.354545454545454	1093\\
0.363636363636364	1107\\
0.372727272727273	1120\\
0.381818181818182	1130\\
0.390909090909091	1144\\
0.4	1154\\
0.409090909090909	1164\\
0.418181818181818	1176\\
0.427272727272727	1186\\
0.436363636363636	1197\\
0.445454545454545	1209\\
0.454545454545455	1216\\
0.463636363636364	1228\\
0.472727272727273	1236\\
0.481818181818182	1247\\
0.490909090909091	1255\\
0.5	1263\\
0.509090909090909	1271\\
0.518181818181818	1284\\
0.527272727272727	1292\\
0.536363636363636	1296\\
0.545454545454546	1305\\
0.554545454545455	1313\\
0.563636363636364	1323\\
0.572727272727273	1332\\
0.581818181818182	1337\\
0.590909090909091	1346\\
0.6	1351\\
0.609090909090909	1360\\
0.618181818181818	1366\\
0.627272727272727	1375\\
0.636363636363636	1380\\
0.645454545454545	1385\\
0.654545454545455	1395\\
0.663636363636364	1401\\
0.672727272727273	1406\\
0.681818181818182	1412\\
0.690909090909091	1417\\
0.7	1423\\
0.709090909090909	1429\\
0.718181818181818	1434\\
0.727272727272727	1440\\
0.736363636363636	1446\\
0.745454545454545	1452\\
0.754545454545455	1458\\
0.763636363636364	1464\\
0.772727272727273	1470\\
0.781818181818182	1476\\
0.790909090909091	1478\\
0.8	1484\\
0.809090909090909	1490\\
0.818181818181818	1492\\
0.827272727272727	1498\\
0.836363636363636	1505\\
0.845454545454545	1507\\
0.854545454545454	1513\\
0.863636363636364	1519\\
0.872727272727273	1522\\
0.881818181818182	1528\\
0.890909090909091	1531\\
0.9	1537\\
0.909090909090909	1540\\
0.918181818181818	1542\\
0.927272727272727	1549\\
0.936363636363636	1551\\
0.945454545454545	1558\\
0.954545454545455	1560\\
0.963636363636364	1563\\
0.972727272727273	1569\\
0.981818181818182	1572\\
0.990909090909091	1575\\
1	1577\\
};
\end{axis}

\begin{axis}[%
width=1.66in,
height=1.258in,
at={(1.011in,0.642in)},
scale only axis,
xmin=0,
xmax=1,
xlabel style={font=\color{white!15!black}},
xlabel={$\tau$},
ymin=500,
ymax=1572,
axis background/.style={fill=white},
title style={font=\bfseries},
title={$\gamma\text{= 0.70}$}
]
\addplot [color=mycolor1, line width=2.0pt, forget plot]
  table[row sep=crcr]{%
0.1	730\\
0.109090909090909	726\\
0.118181818181818	717\\
0.127272727272727	711\\
0.136363636363636	701\\
0.145454545454545	695\\
0.154545454545455	700\\
0.163636363636364	715\\
0.172727272727273	732\\
0.181818181818182	759\\
0.190909090909091	771\\
0.2	794\\
0.209090909090909	816\\
0.218181818181818	834\\
0.227272727272727	856\\
0.236363636363636	869\\
0.245454545454545	891\\
0.254545454545455	907\\
0.263636363636364	920\\
0.272727272727273	937\\
0.281818181818182	952\\
0.290909090909091	970\\
0.3	987\\
0.309090909090909	1002\\
0.318181818181818	1016\\
0.327272727272727	1031\\
0.336363636363636	1046\\
0.345454545454545	1062\\
0.354545454545454	1075\\
0.363636363636364	1088\\
0.372727272727273	1102\\
0.381818181818182	1116\\
0.390909090909091	1130\\
0.4	1139\\
0.409090909090909	1153\\
0.418181818181818	1163\\
0.427272727272727	1173\\
0.436363636363636	1184\\
0.445454545454545	1195\\
0.454545454545455	1206\\
0.463636363636364	1213\\
0.472727272727273	1224\\
0.481818181818182	1231\\
0.490909090909091	1243\\
0.5	1251\\
0.509090909090909	1259\\
0.518181818181818	1271\\
0.527272727272727	1279\\
0.536363636363636	1287\\
0.545454545454546	1295\\
0.554545454545455	1304\\
0.563636363636364	1308\\
0.572727272727273	1317\\
0.581818181818182	1325\\
0.590909090909091	1330\\
0.6	1339\\
0.609090909090909	1348\\
0.618181818181818	1352\\
0.627272727272727	1361\\
0.636363636363636	1366\\
0.645454545454545	1371\\
0.654545454545455	1382\\
0.663636363636364	1387\\
0.672727272727273	1393\\
0.681818181818182	1398\\
0.690909090909091	1407\\
0.7	1413\\
0.709090909090909	1419\\
0.718181818181818	1425\\
0.727272727272727	1430\\
0.736363636363636	1436\\
0.745454545454545	1442\\
0.754545454545455	1447\\
0.763636363636364	1453\\
0.772727272727273	1459\\
0.781818181818182	1465\\
0.790909090909091	1467\\
0.8	1473\\
0.809090909090909	1479\\
0.818181818181818	1485\\
0.827272727272727	1487\\
0.836363636363636	1493\\
0.845454545454545	1499\\
0.854545454545454	1502\\
0.863636363636364	1508\\
0.872727272727273	1514\\
0.881818181818182	1516\\
0.890909090909091	1523\\
0.9	1525\\
0.909090909090909	1531\\
0.918181818181818	1533\\
0.927272727272727	1540\\
0.936363636363636	1542\\
0.945454545454545	1549\\
0.954545454545455	1551\\
0.963636363636364	1558\\
0.972727272727273	1560\\
0.981818181818182	1563\\
0.990909090909091	1569\\
1	1572\\
};
\end{axis}

\begin{axis}[%
width=1.66in,
height=1.258in,
at={(3.195in,0.642in)},
scale only axis,
xmin=0,
xmax=1,
xlabel style={font=\color{white!15!black}},
xlabel={$\tau$},
ymin=760,
ymax=1561,
axis background/.style={fill=white},
title style={font=\bfseries},
title={$\gamma\text{= 0.80}$}
]
\addplot [color=mycolor1, line width=2.0pt, forget plot]
  table[row sep=crcr]{%
0.1	777\\
0.109090909090909	778\\
0.118181818181818	777\\
0.127272727272727	778\\
0.136363636363636	772\\
0.145454545454545	768\\
0.154545454545455	765\\
0.163636363636364	760\\
0.172727272727273	763\\
0.181818181818182	776\\
0.190909090909091	784\\
0.2	805\\
0.209090909090909	826\\
0.218181818181818	835\\
0.227272727272727	856\\
0.236363636363636	873\\
0.245454545454545	890\\
0.254545454545455	905\\
0.263636363636364	923\\
0.272727272727273	934\\
0.281818181818182	951\\
0.290909090909091	967\\
0.3	978\\
0.309090909090909	988\\
0.318181818181818	1002\\
0.327272727272727	1020\\
0.336363636363636	1033\\
0.345454545454545	1045\\
0.354545454545454	1062\\
0.363636363636364	1073\\
0.372727272727273	1085\\
0.381818181818182	1097\\
0.390909090909091	1111\\
0.4	1124\\
0.409090909090909	1134\\
0.418181818181818	1148\\
0.427272727272727	1158\\
0.436363636363636	1170\\
0.445454545454545	1180\\
0.454545454545455	1192\\
0.463636363636364	1201\\
0.472727272727273	1211\\
0.481818181818182	1218\\
0.490909090909091	1230\\
0.5	1237\\
0.509090909090909	1249\\
0.518181818181818	1256\\
0.527272727272727	1264\\
0.536363636363636	1271\\
0.545454545454546	1279\\
0.554545454545455	1287\\
0.563636363636364	1296\\
0.572727272727273	1304\\
0.581818181818182	1313\\
0.590909090909091	1318\\
0.6	1326\\
0.609090909090909	1335\\
0.618181818181818	1340\\
0.627272727272727	1348\\
0.636363636363636	1353\\
0.645454545454545	1362\\
0.654545454545455	1367\\
0.663636363636364	1372\\
0.672727272727273	1381\\
0.681818181818182	1387\\
0.690909090909091	1392\\
0.7	1397\\
0.709090909090909	1402\\
0.718181818181818	1408\\
0.727272727272727	1413\\
0.736363636363636	1423\\
0.745454545454545	1430\\
0.754545454545455	1431\\
0.763636363636364	1437\\
0.772727272727273	1443\\
0.781818181818182	1448\\
0.790909090909091	1454\\
0.8	1460\\
0.809090909090909	1466\\
0.818181818181818	1472\\
0.827272727272727	1474\\
0.836363636363636	1480\\
0.845454545454545	1486\\
0.854545454545454	1488\\
0.863636363636364	1494\\
0.872727272727273	1500\\
0.881818181818182	1503\\
0.890909090909091	1509\\
0.9	1515\\
0.909090909090909	1518\\
0.918181818181818	1524\\
0.927272727272727	1526\\
0.936363636363636	1532\\
0.945454545454545	1535\\
0.954545454545455	1541\\
0.963636363636364	1543\\
0.972727272727273	1550\\
0.981818181818182	1552\\
0.990909090909091	1555\\
1	1561\\
};
\end{axis}

\begin{axis}[%
width=1.66in,
height=1.258in,
at={(5.379in,0.642in)},
scale only axis,
xmin=0,
xmax=1,
xlabel style={font=\color{white!15!black}},
xlabel={$\tau$},
ymin=800,
ymax=1600,
axis background/.style={fill=white},
title style={font=\bfseries},
title={$\gamma\text{= 0.90}$}
]
\addplot [color=mycolor1, line width=2.0pt, forget plot]
  table[row sep=crcr]{%
0.1	813\\
0.109090909090909	815\\
0.118181818181818	816\\
0.127272727272727	825\\
0.136363636363636	824\\
0.145454545454545	829\\
0.154545454545455	825\\
0.163636363636364	823\\
0.172727272727273	819\\
0.181818181818182	817\\
0.190909090909091	817\\
0.2	823\\
0.209090909090909	834\\
0.218181818181818	849\\
0.227272727272727	865\\
0.236363636363636	880\\
0.245454545454545	890\\
0.254545454545455	905\\
0.263636363636364	920\\
0.272727272727273	933\\
0.281818181818182	948\\
0.290909090909091	965\\
0.3	973\\
0.309090909090909	987\\
0.318181818181818	1004\\
0.327272727272727	1015\\
0.336363636363636	1023\\
0.345454545454545	1034\\
0.354545454545454	1047\\
0.363636363636364	1060\\
0.372727272727273	1072\\
0.381818181818182	1085\\
0.390909090909091	1096\\
0.4	1110\\
0.409090909090909	1121\\
0.418181818181818	1130\\
0.427272727272727	1144\\
0.436363636363636	1154\\
0.445454545454545	1164\\
0.454545454545455	1174\\
0.463636363636364	1185\\
0.472727272727273	1197\\
0.481818181818182	1203\\
0.490909090909091	1215\\
0.5	1221\\
0.509090909090909	1234\\
0.518181818181818	1240\\
0.527272727272727	1248\\
0.536363636363636	1259\\
0.545454545454546	1267\\
0.554545454545455	1275\\
0.563636363636364	1283\\
0.572727272727273	1291\\
0.581818181818182	1299\\
0.590909090909091	1303\\
0.6	1311\\
0.609090909090909	1319\\
0.618181818181818	1324\\
0.627272727272727	1333\\
0.636363636363636	1338\\
0.645454545454545	1346\\
0.654545454545455	1351\\
0.663636363636364	1360\\
0.672727272727273	1365\\
0.681818181818182	1370\\
0.690909090909091	1380\\
0.7	1385\\
0.709090909090909	1390\\
0.718181818181818	1395\\
0.727272727272727	1400\\
0.736363636363636	1406\\
0.745454545454545	1411\\
0.754545454545455	1417\\
0.763636363636364	1422\\
0.772727272727273	1428\\
0.781818181818182	1434\\
0.790909090909091	1439\\
0.8	1445\\
0.809090909090909	1450\\
0.818181818181818	1456\\
0.827272727272727	1458\\
0.836363636363636	1466\\
0.845454545454545	1471\\
0.854545454545454	1477\\
0.863636363636364	1479\\
0.872727272727273	1485\\
0.881818181818182	1491\\
0.890909090909091	1493\\
0.9	1499\\
0.909090909090909	1502\\
0.918181818181818	1508\\
0.927272727272727	1514\\
0.936363636363636	1516\\
0.945454545454545	1523\\
0.954545454545455	1525\\
0.963636363636364	1531\\
0.972727272727273	1534\\
0.981818181818182	1536\\
0.990909090909091	1543\\
1	1545\\
};
\end{axis}
\end{tikzpicture}%}}  \\
\subfloat[][$N=30$]
{\resizebox{0.45\textwidth}{!}{% This file was created by matlab2tikz.
%
%The latest updates can be retrieved from
%  http://www.mathworks.com/matlabcentral/fileexchange/22022-matlab2tikz-matlab2tikz
%where you can also make suggestions and rate matlab2tikz.
%
\definecolor{mycolor1}{rgb}{0.00000,0.44700,0.74100}%
%
\begin{tikzpicture}

\begin{axis}[%
width=1.66in,
height=1.258in,
at={(1.011in,4.137in)},
scale only axis,
xmin=0,
xmax=1,
xlabel style={font=\color{white!15!black}},
xlabel={$\tau$},
ymin=945,
ymax=1415,
axis background/.style={fill=white},
title style={font=\bfseries},
title={$\gamma\text{= 0.10}$}
]
\addplot [color=mycolor1, line width=2.0pt, forget plot]
  table[row sep=crcr]{%
0.1	945\\
0.109090909090909	988\\
0.118181818181818	1024\\
0.127272727272727	1054\\
0.136363636363636	1082\\
0.145454545454545	1103\\
0.154545454545455	1123\\
0.163636363636364	1140\\
0.172727272727273	1153\\
0.181818181818182	1169\\
0.190909090909091	1180\\
0.2	1190\\
0.209090909090909	1200\\
0.218181818181818	1211\\
0.227272727272727	1218\\
0.236363636363636	1226\\
0.245454545454545	1234\\
0.254545454545455	1243\\
0.263636363636364	1252\\
0.272727272727273	1257\\
0.281818181818182	1263\\
0.290909090909091	1269\\
0.3	1274\\
0.309090909090909	1281\\
0.318181818181818	1287\\
0.327272727272727	1293\\
0.336363636363636	1295\\
0.345454545454545	1301\\
0.354545454545454	1304\\
0.363636363636364	1311\\
0.372727272727273	1313\\
0.381818181818182	1315\\
0.390909090909091	1323\\
0.4	1325\\
0.409090909090909	1328\\
0.418181818181818	1331\\
0.427272727272727	1334\\
0.436363636363636	1337\\
0.445454545454545	1340\\
0.454545454545455	1343\\
0.463636363636364	1345\\
0.472727272727273	1348\\
0.481818181818182	1348\\
0.490909090909091	1351\\
0.5	1354\\
0.509090909090909	1357\\
0.518181818181818	1356\\
0.527272727272727	1359\\
0.536363636363636	1363\\
0.545454545454546	1363\\
0.554545454545455	1366\\
0.563636363636364	1365\\
0.572727272727273	1368\\
0.581818181818182	1372\\
0.590909090909091	1370\\
0.6	1374\\
0.609090909090909	1374\\
0.618181818181818	1377\\
0.627272727272727	1378\\
0.636363636363636	1381\\
0.645454545454545	1381\\
0.654545454545455	1384\\
0.663636363636364	1384\\
0.672727272727273	1383\\
0.681818181818182	1386\\
0.690909090909091	1386\\
0.7	1390\\
0.709090909090909	1389\\
0.718181818181818	1389\\
0.727272727272727	1393\\
0.736363636363636	1393\\
0.745454545454545	1396\\
0.754545454545455	1395\\
0.763636363636364	1388\\
0.772727272727273	1392\\
0.781818181818182	1398\\
0.790909090909091	1398\\
0.8	1400\\
0.809090909090909	1401\\
0.818181818181818	1400\\
0.827272727272727	1400\\
0.836363636363636	1404\\
0.845454545454545	1404\\
0.854545454545454	1404\\
0.863636363636364	1403\\
0.872727272727273	1407\\
0.881818181818182	1407\\
0.890909090909091	1406\\
0.9	1409\\
0.909090909090909	1409\\
0.918181818181818	1408\\
0.927272727272727	1408\\
0.936363636363636	1409\\
0.945454545454545	1413\\
0.954545454545455	1413\\
0.963636363636364	1412\\
0.972727272727273	1412\\
0.981818181818182	1415\\
0.990909090909091	1414\\
1	1415\\
};
\end{axis}

\begin{axis}[%
width=1.66in,
height=1.258in,
at={(3.195in,4.137in)},
scale only axis,
xmin=0,
xmax=1,
xlabel style={font=\color{white!15!black}},
xlabel={$\tau$},
ymin=800,
ymax=1600,
axis background/.style={fill=white},
title style={font=\bfseries},
title={$\gamma\text{= 0.20}$}
]
\addplot [color=mycolor1, line width=2.0pt, forget plot]
  table[row sep=crcr]{%
0.1	804\\
0.109090909090909	838\\
0.118181818181818	875\\
0.127272727272727	904\\
0.136363636363636	934\\
0.145454545454545	962\\
0.154545454545455	988\\
0.163636363636364	1016\\
0.172727272727273	1034\\
0.181818181818182	1055\\
0.190909090909091	1079\\
0.2	1099\\
0.209090909090909	1117\\
0.218181818181818	1132\\
0.227272727272727	1151\\
0.236363636363636	1167\\
0.245454545454545	1182\\
0.254545454545455	1198\\
0.263636363636364	1209\\
0.272727272727273	1225\\
0.281818181818182	1232\\
0.290909090909091	1245\\
0.3	1257\\
0.309090909090909	1266\\
0.318181818181818	1275\\
0.327272727272727	1284\\
0.336363636363636	1293\\
0.345454545454545	1301\\
0.354545454545454	1310\\
0.363636363636364	1316\\
0.372727272727273	1324\\
0.381818181818182	1330\\
0.390909090909091	1337\\
0.4	1342\\
0.409090909090909	1348\\
0.418181818181818	1354\\
0.427272727272727	1360\\
0.436363636363636	1366\\
0.445454545454545	1369\\
0.454545454545455	1375\\
0.463636363636364	1382\\
0.472727272727273	1384\\
0.481818181818182	1390\\
0.490909090909091	1393\\
0.5	1400\\
0.509090909090909	1403\\
0.518181818181818	1405\\
0.527272727272727	1408\\
0.536363636363636	1414\\
0.545454545454546	1417\\
0.554545454545455	1420\\
0.563636363636364	1423\\
0.572727272727273	1426\\
0.581818181818182	1429\\
0.590909090909091	1431\\
0.6	1435\\
0.609090909090909	1439\\
0.618181818181818	1441\\
0.627272727272727	1444\\
0.636363636363636	1447\\
0.645454545454545	1450\\
0.654545454545455	1449\\
0.663636363636364	1453\\
0.672727272727273	1456\\
0.681818181818182	1459\\
0.690909090909091	1458\\
0.7	1461\\
0.709090909090909	1464\\
0.718181818181818	1467\\
0.727272727272727	1467\\
0.736363636363636	1471\\
0.745454545454545	1474\\
0.754545454545455	1472\\
0.763636363636364	1476\\
0.772727272727273	1475\\
0.781818181818182	1479\\
0.790909090909091	1483\\
0.8	1482\\
0.809090909090909	1486\\
0.818181818181818	1485\\
0.827272727272727	1488\\
0.836363636363636	1488\\
0.845454545454545	1491\\
0.854545454545454	1491\\
0.863636363636364	1495\\
0.872727272727273	1494\\
0.881818181818182	1497\\
0.890909090909091	1497\\
0.9	1500\\
0.909090909090909	1500\\
0.918181818181818	1500\\
0.927272727272727	1502\\
0.936363636363636	1501\\
0.945454545454545	1505\\
0.954545454545455	1504\\
0.963636363636364	1509\\
0.972727272727273	1508\\
0.981818181818182	1508\\
0.990909090909091	1511\\
1	1511\\
};
\end{axis}

\begin{axis}[%
width=1.66in,
height=1.258in,
at={(5.379in,4.137in)},
scale only axis,
xmin=0,
xmax=1,
xlabel style={font=\color{white!15!black}},
xlabel={$\tau$},
ymin=500,
ymax=1584,
axis background/.style={fill=white},
title style={font=\bfseries},
title={$\gamma\text{= 0.30}$}
]
\addplot [color=mycolor1, line width=2.0pt, forget plot]
  table[row sep=crcr]{%
0.1	705\\
0.109090909090909	750\\
0.118181818181818	788\\
0.127272727272727	823\\
0.136363636363636	856\\
0.145454545454545	887\\
0.154545454545455	917\\
0.163636363636364	948\\
0.172727272727273	977\\
0.181818181818182	1001\\
0.190909090909091	1021\\
0.2	1044\\
0.209090909090909	1068\\
0.218181818181818	1087\\
0.227272727272727	1107\\
0.236363636363636	1125\\
0.245454545454545	1143\\
0.254545454545455	1158\\
0.263636363636364	1178\\
0.272727272727273	1190\\
0.281818181818182	1205\\
0.290909090909091	1216\\
0.3	1231\\
0.309090909090909	1242\\
0.318181818181818	1255\\
0.327272727272727	1266\\
0.336363636363636	1279\\
0.345454545454545	1287\\
0.354545454545454	1301\\
0.363636363636364	1309\\
0.372727272727273	1319\\
0.381818181818182	1332\\
0.390909090909091	1337\\
0.4	1346\\
0.409090909090909	1355\\
0.418181818181818	1364\\
0.427272727272727	1369\\
0.436363636363636	1379\\
0.445454545454545	1384\\
0.454545454545455	1390\\
0.463636363636364	1396\\
0.472727272727273	1406\\
0.481818181818182	1412\\
0.490909090909091	1418\\
0.5	1420\\
0.509090909090909	1425\\
0.518181818181818	1431\\
0.527272727272727	1438\\
0.536363636363636	1444\\
0.545454545454546	1447\\
0.554545454545455	1453\\
0.563636363636364	1454\\
0.572727272727273	1461\\
0.581818181818182	1464\\
0.590909090909091	1471\\
0.6	1472\\
0.609090909090909	1475\\
0.618181818181818	1482\\
0.627272727272727	1485\\
0.636363636363636	1488\\
0.645454545454545	1490\\
0.654545454545455	1493\\
0.663636363636364	1500\\
0.672727272727273	1503\\
0.681818181818182	1506\\
0.690909090909091	1509\\
0.7	1513\\
0.709090909090909	1515\\
0.718181818181818	1517\\
0.727272727272727	1520\\
0.736363636363636	1524\\
0.745454545454545	1527\\
0.754545454545455	1527\\
0.763636363636364	1530\\
0.772727272727273	1532\\
0.781818181818182	1535\\
0.790909090909091	1539\\
0.8	1542\\
0.809090909090909	1541\\
0.818181818181818	1544\\
0.827272727272727	1548\\
0.836363636363636	1551\\
0.845454545454545	1550\\
0.854545454545454	1554\\
0.863636363636364	1557\\
0.872727272727273	1556\\
0.881818181818182	1559\\
0.890909090909091	1562\\
0.9	1562\\
0.909090909090909	1565\\
0.918181818181818	1569\\
0.927272727272727	1568\\
0.936363636363636	1572\\
0.945454545454545	1571\\
0.954545454545455	1574\\
0.963636363636364	1574\\
0.972727272727273	1577\\
0.981818181818182	1581\\
0.990909090909091	1580\\
1	1584\\
};
\end{axis}

\begin{axis}[%
width=1.66in,
height=1.258in,
at={(1.011in,2.39in)},
scale only axis,
xmin=0,
xmax=1,
xlabel style={font=\color{white!15!black}},
xlabel={$\tau$},
ymin=500,
ymax=1631,
axis background/.style={fill=white},
title style={font=\bfseries},
title={$\gamma\text{= 0.40}$}
]
\addplot [color=mycolor1, line width=2.0pt, forget plot]
  table[row sep=crcr]{%
0.1	645\\
0.109090909090909	693\\
0.118181818181818	741\\
0.127272727272727	786\\
0.136363636363636	819\\
0.145454545454545	855\\
0.154545454545455	886\\
0.163636363636364	911\\
0.172727272727273	943\\
0.181818181818182	968\\
0.190909090909091	997\\
0.2	1019\\
0.209090909090909	1041\\
0.218181818181818	1060\\
0.227272727272727	1085\\
0.236363636363636	1105\\
0.245454545454545	1120\\
0.254545454545455	1135\\
0.263636363636364	1152\\
0.272727272727273	1169\\
0.281818181818182	1183\\
0.290909090909091	1201\\
0.3	1215\\
0.309090909090909	1231\\
0.318181818181818	1241\\
0.327272727272727	1257\\
0.336363636363636	1266\\
0.345454545454545	1278\\
0.354545454545454	1288\\
0.363636363636364	1300\\
0.372727272727273	1309\\
0.381818181818182	1321\\
0.390909090909091	1329\\
0.4	1338\\
0.409090909090909	1350\\
0.418181818181818	1360\\
0.427272727272727	1369\\
0.436363636363636	1373\\
0.445454545454545	1382\\
0.454545454545455	1391\\
0.463636363636364	1402\\
0.472727272727273	1407\\
0.481818181818182	1417\\
0.490909090909091	1423\\
0.5	1429\\
0.509090909090909	1435\\
0.518181818181818	1443\\
0.527272727272727	1450\\
0.536363636363636	1456\\
0.545454545454546	1462\\
0.554545454545455	1468\\
0.563636363636364	1474\\
0.572727272727273	1479\\
0.581818181818182	1482\\
0.590909090909091	1488\\
0.6	1494\\
0.609090909090909	1497\\
0.618181818181818	1503\\
0.627272727272727	1510\\
0.636363636363636	1513\\
0.645454545454545	1519\\
0.654545454545455	1522\\
0.663636363636364	1524\\
0.672727272727273	1531\\
0.681818181818182	1532\\
0.690909090909091	1535\\
0.7	1542\\
0.709090909090909	1544\\
0.718181818181818	1547\\
0.727272727272727	1550\\
0.736363636363636	1557\\
0.745454545454545	1559\\
0.754545454545455	1561\\
0.763636363636364	1564\\
0.772727272727273	1568\\
0.781818181818182	1571\\
0.790909090909091	1574\\
0.8	1576\\
0.809090909090909	1579\\
0.818181818181818	1582\\
0.827272727272727	1585\\
0.836363636363636	1589\\
0.845454545454545	1592\\
0.854545454545454	1595\\
0.863636363636364	1598\\
0.872727272727273	1597\\
0.881818181818182	1600\\
0.890909090909091	1603\\
0.9	1607\\
0.909090909090909	1610\\
0.918181818181818	1613\\
0.927272727272727	1613\\
0.936363636363636	1616\\
0.945454545454545	1619\\
0.954545454545455	1622\\
0.963636363636364	1621\\
0.972727272727273	1624\\
0.981818181818182	1628\\
0.990909090909091	1627\\
1	1631\\
};
\end{axis}

\begin{axis}[%
width=1.66in,
height=1.258in,
at={(3.195in,2.39in)},
scale only axis,
xmin=0,
xmax=1,
xlabel style={font=\color{white!15!black}},
xlabel={$\tau$},
ymin=500,
ymax=1660,
axis background/.style={fill=white},
title style={font=\bfseries},
title={$\gamma\text{= 0.50}$}
]
\addplot [color=mycolor1, line width=2.0pt, forget plot]
  table[row sep=crcr]{%
0.1	617\\
0.109090909090909	660\\
0.118181818181818	705\\
0.127272727272727	746\\
0.136363636363636	789\\
0.145454545454545	827\\
0.154545454545455	867\\
0.163636363636364	903\\
0.172727272727273	926\\
0.181818181818182	955\\
0.190909090909091	978\\
0.2	999\\
0.209090909090909	1023\\
0.218181818181818	1045\\
0.227272727272727	1065\\
0.236363636363636	1087\\
0.245454545454545	1107\\
0.254545454545455	1123\\
0.263636363636364	1141\\
0.272727272727273	1157\\
0.281818181818182	1176\\
0.290909090909091	1188\\
0.3	1202\\
0.309090909090909	1215\\
0.318181818181818	1228\\
0.327272727272727	1244\\
0.336363636363636	1253\\
0.345454545454545	1268\\
0.354545454545454	1279\\
0.363636363636364	1291\\
0.372727272727273	1302\\
0.381818181818182	1315\\
0.390909090909091	1326\\
0.4	1334\\
0.409090909090909	1343\\
0.418181818181818	1353\\
0.427272727272727	1361\\
0.436363636363636	1370\\
0.445454545454545	1377\\
0.454545454545455	1386\\
0.463636363636364	1395\\
0.472727272727273	1405\\
0.481818181818182	1414\\
0.490909090909091	1418\\
0.5	1428\\
0.509090909090909	1433\\
0.518181818181818	1444\\
0.527272727272727	1450\\
0.536363636363636	1455\\
0.545454545454546	1463\\
0.554545454545455	1470\\
0.563636363636364	1475\\
0.572727272727273	1481\\
0.581818181818182	1488\\
0.590909090909091	1494\\
0.6	1500\\
0.609090909090909	1506\\
0.618181818181818	1513\\
0.627272727272727	1515\\
0.636363636363636	1522\\
0.645454545454545	1527\\
0.654545454545455	1534\\
0.663636363636364	1536\\
0.672727272727273	1543\\
0.681818181818182	1545\\
0.690909090909091	1551\\
0.7	1558\\
0.709090909090909	1560\\
0.718181818181818	1562\\
0.727272727272727	1569\\
0.736363636363636	1572\\
0.745454545454545	1578\\
0.754545454545455	1581\\
0.763636363636364	1584\\
0.772727272727273	1587\\
0.781818181818182	1593\\
0.790909090909091	1597\\
0.8	1599\\
0.809090909090909	1602\\
0.818181818181818	1605\\
0.827272727272727	1612\\
0.836363636363636	1615\\
0.845454545454545	1616\\
0.854545454545454	1619\\
0.863636363636364	1622\\
0.872727272727273	1625\\
0.881818181818182	1628\\
0.890909090909091	1631\\
0.9	1634\\
0.909090909090909	1638\\
0.918181818181818	1641\\
0.927272727272727	1644\\
0.936363636363636	1642\\
0.945454545454545	1645\\
0.954545454545455	1648\\
0.963636363636364	1652\\
0.972727272727273	1655\\
0.981818181818182	1659\\
0.990909090909091	1658\\
1	1660\\
};
\end{axis}

\begin{axis}[%
width=1.66in,
height=1.258in,
at={(5.379in,2.39in)},
scale only axis,
xmin=0,
xmax=1,
xlabel style={font=\color{white!15!black}},
xlabel={$\tau$},
ymin=500,
ymax=1681,
axis background/.style={fill=white},
title style={font=\bfseries},
title={$\gamma\text{= 0.60}$}
]
\addplot [color=mycolor1, line width=2.0pt, forget plot]
  table[row sep=crcr]{%
0.1	620\\
0.109090909090909	651\\
0.118181818181818	690\\
0.127272727272727	732\\
0.136363636363636	770\\
0.145454545454545	805\\
0.154545454545455	840\\
0.163636363636364	875\\
0.172727272727273	909\\
0.181818181818182	939\\
0.190909090909091	970\\
0.2	990\\
0.209090909090909	1014\\
0.218181818181818	1037\\
0.227272727272727	1054\\
0.236363636363636	1076\\
0.245454545454545	1090\\
0.254545454545455	1111\\
0.263636363636364	1128\\
0.272727272727273	1144\\
0.281818181818182	1160\\
0.290909090909091	1177\\
0.3	1195\\
0.309090909090909	1205\\
0.318181818181818	1219\\
0.327272727272727	1233\\
0.336363636363636	1248\\
0.345454545454545	1259\\
0.354545454545454	1274\\
0.363636363636364	1283\\
0.372727272727273	1295\\
0.381818181818182	1304\\
0.390909090909091	1317\\
0.4	1323\\
0.409090909090909	1335\\
0.418181818181818	1343\\
0.427272727272727	1355\\
0.436363636363636	1363\\
0.445454545454545	1371\\
0.454545454545455	1384\\
0.463636363636364	1393\\
0.472727272727273	1402\\
0.481818181818182	1406\\
0.490909090909091	1415\\
0.5	1423\\
0.509090909090909	1431\\
0.518181818181818	1436\\
0.527272727272727	1447\\
0.536363636363636	1450\\
0.545454545454546	1459\\
0.554545454545455	1465\\
0.563636363636364	1471\\
0.572727272727273	1481\\
0.581818181818182	1486\\
0.590909090909091	1492\\
0.6	1497\\
0.609090909090909	1503\\
0.618181818181818	1509\\
0.627272727272727	1516\\
0.636363636363636	1522\\
0.645454545454545	1528\\
0.654545454545455	1533\\
0.663636363636364	1540\\
0.672727272727273	1542\\
0.681818181818182	1548\\
0.690909090909091	1554\\
0.7	1558\\
0.709090909090909	1564\\
0.718181818181818	1570\\
0.727272727272727	1572\\
0.736363636363636	1580\\
0.745454545454545	1582\\
0.754545454545455	1589\\
0.763636363636364	1591\\
0.772727272727273	1597\\
0.781818181818182	1601\\
0.790909090909091	1603\\
0.8	1610\\
0.809090909090909	1612\\
0.818181818181818	1615\\
0.827272727272727	1621\\
0.836363636363636	1624\\
0.845454545454545	1627\\
0.854545454545454	1629\\
0.863636363636364	1632\\
0.872727272727273	1639\\
0.881818181818182	1642\\
0.890909090909091	1645\\
0.9	1648\\
0.909090909090909	1651\\
0.918181818181818	1654\\
0.927272727272727	1657\\
0.936363636363636	1660\\
0.945454545454545	1663\\
0.954545454545455	1666\\
0.963636363636364	1669\\
0.972727272727273	1672\\
0.981818181818182	1675\\
0.990909090909091	1678\\
1	1681\\
};
\end{axis}

\begin{axis}[%
width=1.66in,
height=1.258in,
at={(1.011in,0.642in)},
scale only axis,
xmin=0,
xmax=1,
xlabel style={font=\color{white!15!black}},
xlabel={$\tau$},
ymin=500,
ymax=1688,
axis background/.style={fill=white},
title style={font=\bfseries},
title={$\gamma\text{= 0.70}$}
]
\addplot [color=mycolor1, line width=2.0pt, forget plot]
  table[row sep=crcr]{%
0.1	674\\
0.109090909090909	687\\
0.118181818181818	696\\
0.127272727272727	727\\
0.136363636363636	759\\
0.145454545454545	793\\
0.154545454545455	829\\
0.163636363636364	861\\
0.172727272727273	891\\
0.181818181818182	916\\
0.190909090909091	946\\
0.2	974\\
0.209090909090909	1002\\
0.218181818181818	1026\\
0.227272727272727	1048\\
0.236363636363636	1062\\
0.245454545454545	1082\\
0.254545454545455	1101\\
0.263636363636364	1116\\
0.272727272727273	1135\\
0.281818181818182	1146\\
0.290909090909091	1164\\
0.3	1180\\
0.309090909090909	1196\\
0.318181818181818	1208\\
0.327272727272727	1222\\
0.336363636363636	1236\\
0.345454545454545	1249\\
0.354545454545454	1260\\
0.363636363636364	1272\\
0.372727272727273	1284\\
0.381818181818182	1294\\
0.390909090909091	1306\\
0.4	1318\\
0.409090909090909	1329\\
0.418181818181818	1336\\
0.427272727272727	1348\\
0.436363636363636	1356\\
0.445454545454545	1362\\
0.454545454545455	1375\\
0.463636363636364	1382\\
0.472727272727273	1391\\
0.481818181818182	1399\\
0.490909090909091	1408\\
0.5	1412\\
0.509090909090909	1421\\
0.518181818181818	1430\\
0.527272727272727	1439\\
0.536363636363636	1445\\
0.545454545454546	1454\\
0.554545454545455	1459\\
0.563636363636364	1468\\
0.572727272727273	1474\\
0.581818181818182	1478\\
0.590909090909091	1487\\
0.6	1492\\
0.609090909090909	1498\\
0.618181818181818	1505\\
0.627272727272727	1509\\
0.636363636363636	1514\\
0.645454545454545	1520\\
0.654545454545455	1527\\
0.663636363636364	1533\\
0.672727272727273	1539\\
0.681818181818182	1544\\
0.690909090909091	1550\\
0.7	1556\\
0.709090909090909	1558\\
0.718181818181818	1564\\
0.727272727272727	1572\\
0.736363636363636	1574\\
0.745454545454545	1581\\
0.754545454545455	1587\\
0.763636363636364	1588\\
0.772727272727273	1595\\
0.781818181818182	1597\\
0.790909090909091	1604\\
0.8	1606\\
0.809090909090909	1610\\
0.818181818181818	1617\\
0.827272727272727	1619\\
0.836363636363636	1626\\
0.845454545454545	1628\\
0.854545454545454	1631\\
0.863636363636364	1638\\
0.872727272727273	1641\\
0.881818181818182	1644\\
0.890909090909091	1647\\
0.9	1648\\
0.909090909090909	1656\\
0.918181818181818	1659\\
0.927272727272727	1662\\
0.936363636363636	1665\\
0.945454545454545	1667\\
0.954545454545455	1670\\
0.963636363636364	1673\\
0.972727272727273	1676\\
0.981818181818182	1683\\
0.990909090909091	1686\\
1	1688\\
};
\end{axis}

\begin{axis}[%
width=1.66in,
height=1.258in,
at={(3.195in,0.642in)},
scale only axis,
xmin=0,
xmax=1,
xlabel style={font=\color{white!15!black}},
xlabel={$\tau$},
ymin=500,
ymax=1685,
axis background/.style={fill=white},
title style={font=\bfseries},
title={$\gamma\text{= 0.80}$}
]
\addplot [color=mycolor1, line width=2.0pt, forget plot]
  table[row sep=crcr]{%
0.1	750\\
0.109090909090909	738\\
0.118181818181818	736\\
0.127272727272727	748\\
0.136363636363636	763\\
0.145454545454545	791\\
0.154545454545455	823\\
0.163636363636364	853\\
0.172727272727273	881\\
0.181818181818182	905\\
0.190909090909091	933\\
0.2	956\\
0.209090909090909	978\\
0.218181818181818	1006\\
0.227272727272727	1028\\
0.236363636363636	1053\\
0.245454545454545	1075\\
0.254545454545455	1094\\
0.263636363636364	1114\\
0.272727272727273	1124\\
0.281818181818182	1142\\
0.290909090909091	1158\\
0.3	1173\\
0.309090909090909	1186\\
0.318181818181818	1195\\
0.327272727272727	1207\\
0.336363636363636	1222\\
0.345454545454545	1237\\
0.354545454545454	1248\\
0.363636363636364	1258\\
0.372727272727273	1272\\
0.381818181818182	1283\\
0.390909090909091	1294\\
0.4	1305\\
0.409090909090909	1316\\
0.418181818181818	1325\\
0.427272727272727	1332\\
0.436363636363636	1344\\
0.445454545454545	1353\\
0.454545454545455	1366\\
0.463636363636364	1372\\
0.472727272727273	1382\\
0.481818181818182	1389\\
0.490909090909091	1398\\
0.5	1406\\
0.509090909090909	1413\\
0.518181818181818	1419\\
0.527272727272727	1426\\
0.536363636363636	1435\\
0.545454545454546	1440\\
0.554545454545455	1449\\
0.563636363636364	1454\\
0.572727272727273	1463\\
0.581818181818182	1468\\
0.590909090909091	1478\\
0.6	1483\\
0.609090909090909	1490\\
0.618181818181818	1499\\
0.627272727272727	1504\\
0.636363636363636	1510\\
0.645454545454545	1516\\
0.654545454545455	1521\\
0.663636363636364	1525\\
0.672727272727273	1531\\
0.681818181818182	1537\\
0.690909090909091	1543\\
0.7	1550\\
0.709090909090909	1550\\
0.718181818181818	1556\\
0.727272727272727	1562\\
0.736363636363636	1568\\
0.745454545454545	1571\\
0.754545454545455	1577\\
0.763636363636364	1584\\
0.772727272727273	1586\\
0.781818181818182	1591\\
0.790909090909091	1594\\
0.8	1600\\
0.809090909090909	1602\\
0.818181818181818	1609\\
0.827272727272727	1613\\
0.836363636363636	1619\\
0.845454545454545	1622\\
0.854545454545454	1628\\
0.863636363636364	1631\\
0.872727272727273	1632\\
0.881818181818182	1639\\
0.890909090909091	1642\\
0.9	1644\\
0.909090909090909	1651\\
0.918181818181818	1654\\
0.927272727272727	1658\\
0.936363636363636	1661\\
0.945454545454545	1663\\
0.954545454545455	1670\\
0.963636363636364	1673\\
0.972727272727273	1675\\
0.981818181818182	1679\\
0.990909090909091	1682\\
1	1685\\
};
\end{axis}

\begin{axis}[%
width=1.66in,
height=1.258in,
at={(5.379in,0.642in)},
scale only axis,
xmin=0,
xmax=1,
xlabel style={font=\color{white!15!black}},
xlabel={$\tau$},
ymin=792,
ymax=1678,
axis background/.style={fill=white},
title style={font=\bfseries},
title={$\gamma\text{= 0.90}$}
]
\addplot [color=mycolor1, line width=2.0pt, forget plot]
  table[row sep=crcr]{%
0.1	812\\
0.109090909090909	810\\
0.118181818181818	798\\
0.127272727272727	792\\
0.136363636363636	799\\
0.145454545454545	807\\
0.154545454545455	824\\
0.163636363636364	846\\
0.172727272727273	873\\
0.181818181818182	896\\
0.190909090909091	922\\
0.2	945\\
0.209090909090909	970\\
0.218181818181818	992\\
0.227272727272727	1013\\
0.236363636363636	1032\\
0.245454545454545	1054\\
0.254545454545455	1077\\
0.263636363636364	1096\\
0.272727272727273	1115\\
0.281818181818182	1129\\
0.290909090909091	1150\\
0.3	1159\\
0.309090909090909	1175\\
0.318181818181818	1185\\
0.327272727272727	1203\\
0.336363636363636	1215\\
0.345454545454545	1225\\
0.354545454545454	1239\\
0.363636363636364	1244\\
0.372727272727273	1260\\
0.381818181818182	1271\\
0.390909090909091	1281\\
0.4	1290\\
0.409090909090909	1300\\
0.418181818181818	1311\\
0.427272727272727	1322\\
0.436363636363636	1330\\
0.445454545454545	1341\\
0.454545454545455	1349\\
0.463636363636364	1358\\
0.472727272727273	1368\\
0.481818181818182	1374\\
0.490909090909091	1383\\
0.5	1393\\
0.509090909090909	1401\\
0.518181818181818	1410\\
0.527272727272727	1418\\
0.536363636363636	1422\\
0.545454545454546	1431\\
0.554545454545455	1438\\
0.563636363636364	1445\\
0.572727272727273	1452\\
0.581818181818182	1458\\
0.590909090909091	1462\\
0.6	1471\\
0.609090909090909	1477\\
0.618181818181818	1482\\
0.627272727272727	1492\\
0.636363636363636	1497\\
0.645454545454545	1502\\
0.654545454545455	1508\\
0.663636363636364	1513\\
0.672727272727273	1519\\
0.681818181818182	1525\\
0.690909090909091	1532\\
0.7	1537\\
0.709090909090909	1543\\
0.718181818181818	1549\\
0.727272727272727	1555\\
0.736363636363636	1557\\
0.745454545454545	1561\\
0.754545454545455	1567\\
0.763636363636364	1573\\
0.772727272727273	1576\\
0.781818181818182	1582\\
0.790909090909091	1589\\
0.8	1589\\
0.809090909090909	1596\\
0.818181818181818	1598\\
0.827272727272727	1604\\
0.836363636363636	1606\\
0.845454545454545	1614\\
0.854545454545454	1616\\
0.863636363636364	1623\\
0.872727272727273	1625\\
0.881818181818182	1627\\
0.890909090909091	1634\\
0.9	1636\\
0.909090909090909	1639\\
0.918181818181818	1645\\
0.927272727272727	1648\\
0.936363636363636	1652\\
0.945454545454545	1659\\
0.954545454545455	1661\\
0.963636363636364	1664\\
0.972727272727273	1667\\
0.981818181818182	1668\\
0.990909090909091	1676\\
1	1678\\
};
\end{axis}
\end{tikzpicture}%}}
 \hfill 
\subfloat[][$N=40$]
{\resizebox{0.45\textwidth}{!}{ % This file was created by matlab2tikz.
%
%The latest updates can be retrieved from
%  http://www.mathworks.com/matlabcentral/fileexchange/22022-matlab2tikz-matlab2tikz
%where you can also make suggestions and rate matlab2tikz.
%
\definecolor{mycolor1}{rgb}{0.00000,0.44700,0.74100}%
%
\begin{tikzpicture}

\begin{axis}[%
width=1.66in,
height=1.258in,
at={(1.011in,4.137in)},
scale only axis,
xmin=0,
xmax=1,
xlabel style={font=\color{white!15!black}},
xlabel={$\tau$},
ymin=1000,
ymax=1427,
axis background/.style={fill=white},
title style={font=\bfseries},
title={$\gamma\text{= 0.10}$}
]
\addplot [color=mycolor1, line width=2.0pt, forget plot]
  table[row sep=crcr]{%
0.1	1063\\
0.109090909090909	1089\\
0.118181818181818	1123\\
0.127272727272727	1143\\
0.136363636363636	1164\\
0.145454545454545	1177\\
0.154545454545455	1192\\
0.163636363636364	1203\\
0.172727272727273	1217\\
0.181818181818182	1227\\
0.190909090909091	1230\\
0.2	1238\\
0.209090909090909	1250\\
0.218181818181818	1259\\
0.227272727272727	1265\\
0.236363636363636	1273\\
0.245454545454545	1279\\
0.254545454545455	1289\\
0.263636363636364	1296\\
0.272727272727273	1302\\
0.281818181818182	1312\\
0.290909090909091	1318\\
0.3	1314\\
0.309090909090909	1318\\
0.318181818181818	1325\\
0.327272727272727	1327\\
0.336363636363636	1330\\
0.345454545454545	1337\\
0.354545454545454	1349\\
0.363636363636364	1351\\
0.372727272727273	1353\\
0.381818181818182	1357\\
0.390909090909091	1359\\
0.4	1363\\
0.409090909090909	1366\\
0.418181818181818	1369\\
0.427272727272727	1362\\
0.436363636363636	1361\\
0.445454545454545	1364\\
0.454545454545455	1368\\
0.463636363636364	1366\\
0.472727272727273	1370\\
0.481818181818182	1373\\
0.490909090909091	1372\\
0.5	1376\\
0.509090909090909	1379\\
0.518181818181818	1379\\
0.527272727272727	1381\\
0.536363636363636	1381\\
0.545454545454546	1385\\
0.554545454545455	1385\\
0.563636363636364	1387\\
0.572727272727273	1386\\
0.581818181818182	1389\\
0.590909090909091	1391\\
0.6	1390\\
0.609090909090909	1393\\
0.618181818181818	1392\\
0.627272727272727	1396\\
0.636363636363636	1395\\
0.645454545454545	1392\\
0.654545454545455	1397\\
0.663636363636364	1397\\
0.672727272727273	1396\\
0.681818181818182	1402\\
0.690909090909091	1402\\
0.7	1401\\
0.709090909090909	1405\\
0.718181818181818	1405\\
0.727272727272727	1404\\
0.736363636363636	1408\\
0.745454545454545	1407\\
0.754545454545455	1399\\
0.763636363636364	1411\\
0.772727272727273	1411\\
0.781818181818182	1411\\
0.790909090909091	1410\\
0.8	1414\\
0.809090909090909	1413\\
0.818181818181818	1413\\
0.827272727272727	1414\\
0.836363636363636	1417\\
0.845454545454545	1419\\
0.854545454545454	1419\\
0.863636363636364	1418\\
0.872727272727273	1418\\
0.881818181818182	1422\\
0.890909090909091	1422\\
0.9	1421\\
0.909090909090909	1418\\
0.918181818181818	1418\\
0.927272727272727	1421\\
0.936363636363636	1421\\
0.945454545454545	1421\\
0.954545454545455	1424\\
0.963636363636364	1424\\
0.972727272727273	1423\\
0.981818181818182	1426\\
0.990909090909091	1427\\
1	1427\\
};
\end{axis}

\begin{axis}[%
width=1.66in,
height=1.258in,
at={(3.195in,4.137in)},
scale only axis,
xmin=0,
xmax=1,
xlabel style={font=\color{white!15!black}},
xlabel={$\tau$},
ymin=800,
ymax=1600,
axis background/.style={fill=white},
title style={font=\bfseries},
title={$\gamma\text{= 0.20}$}
]
\addplot [color=mycolor1, line width=2.0pt, forget plot]
  table[row sep=crcr]{%
0.1	910\\
0.109090909090909	952\\
0.118181818181818	988\\
0.127272727272727	1020\\
0.136363636363636	1051\\
0.145454545454545	1080\\
0.154545454545455	1104\\
0.163636363636364	1128\\
0.172727272727273	1146\\
0.181818181818182	1168\\
0.190909090909091	1191\\
0.2	1204\\
0.209090909090909	1222\\
0.218181818181818	1231\\
0.227272727272727	1253\\
0.236363636363636	1265\\
0.245454545454545	1275\\
0.254545454545455	1287\\
0.263636363636364	1299\\
0.272727272727273	1309\\
0.281818181818182	1321\\
0.290909090909091	1330\\
0.3	1339\\
0.309090909090909	1345\\
0.318181818181818	1352\\
0.327272727272727	1358\\
0.336363636363636	1367\\
0.345454545454545	1373\\
0.354545454545454	1379\\
0.363636363636364	1385\\
0.372727272727273	1391\\
0.381818181818182	1389\\
0.390909090909091	1395\\
0.4	1398\\
0.409090909090909	1404\\
0.418181818181818	1406\\
0.427272727272727	1413\\
0.436363636363636	1415\\
0.445454545454545	1422\\
0.454545454545455	1425\\
0.463636363636364	1427\\
0.472727272727273	1433\\
0.481818181818182	1437\\
0.490909090909091	1440\\
0.5	1442\\
0.509090909090909	1446\\
0.518181818181818	1450\\
0.527272727272727	1453\\
0.536363636363636	1456\\
0.545454545454546	1459\\
0.554545454545455	1461\\
0.563636363636364	1472\\
0.572727272727273	1475\\
0.581818181818182	1479\\
0.590909090909091	1469\\
0.6	1471\\
0.609090909090909	1474\\
0.618181818181818	1479\\
0.627272727272727	1478\\
0.636363636363636	1481\\
0.645454545454545	1485\\
0.654545454545455	1484\\
0.663636363636364	1487\\
0.672727272727273	1491\\
0.681818181818182	1490\\
0.690909090909091	1493\\
0.7	1492\\
0.709090909090909	1505\\
0.718181818181818	1509\\
0.727272727272727	1508\\
0.736363636363636	1511\\
0.745454545454545	1510\\
0.754545454545455	1514\\
0.763636363636364	1514\\
0.772727272727273	1517\\
0.781818181818182	1516\\
0.790909090909091	1520\\
0.8	1520\\
0.809090909090909	1519\\
0.818181818181818	1523\\
0.827272727272727	1522\\
0.836363636363636	1526\\
0.845454545454545	1519\\
0.854545454545454	1519\\
0.863636363636364	1518\\
0.872727272727273	1518\\
0.881818181818182	1521\\
0.890909090909091	1521\\
0.9	1521\\
0.909090909090909	1525\\
0.918181818181818	1515\\
0.927272727272727	1523\\
0.936363636363636	1527\\
0.945454545454545	1527\\
0.954545454545455	1526\\
0.963636363636364	1530\\
0.972727272727273	1530\\
0.981818181818182	1529\\
0.990909090909091	1533\\
1	1533\\
};
\end{axis}

\begin{axis}[%
width=1.66in,
height=1.258in,
at={(5.379in,4.137in)},
scale only axis,
xmin=0,
xmax=1,
xlabel style={font=\color{white!15!black}},
xlabel={$\tau$},
ymin=839,
ymax=1615,
axis background/.style={fill=white},
title style={font=\bfseries},
title={$\gamma\text{= 0.30}$}
]
\addplot [color=mycolor1, line width=2.0pt, forget plot]
  table[row sep=crcr]{%
0.1	839\\
0.109090909090909	881\\
0.118181818181818	919\\
0.127272727272727	958\\
0.136363636363636	989\\
0.145454545454545	1015\\
0.154545454545455	1052\\
0.163636363636364	1080\\
0.172727272727273	1102\\
0.181818181818182	1127\\
0.190909090909091	1151\\
0.2	1172\\
0.209090909090909	1193\\
0.218181818181818	1211\\
0.227272727272727	1228\\
0.236363636363636	1245\\
0.245454545454545	1261\\
0.254545454545455	1274\\
0.263636363636364	1290\\
0.272727272727273	1302\\
0.281818181818182	1315\\
0.290909090909091	1325\\
0.3	1338\\
0.309090909090909	1350\\
0.318181818181818	1352\\
0.327272727272727	1361\\
0.336363636363636	1380\\
0.345454545454545	1389\\
0.354545454545454	1395\\
0.363636363636364	1402\\
0.372727272727273	1412\\
0.381818181818182	1417\\
0.390909090909091	1426\\
0.4	1432\\
0.409090909090909	1439\\
0.418181818181818	1445\\
0.427272727272727	1455\\
0.436363636363636	1461\\
0.445454545454545	1463\\
0.454545454545455	1469\\
0.463636363636364	1475\\
0.472727272727273	1479\\
0.481818181818182	1482\\
0.490909090909091	1488\\
0.5	1495\\
0.509090909090909	1497\\
0.518181818181818	1500\\
0.527272727272727	1507\\
0.536363636363636	1509\\
0.545454545454546	1516\\
0.554545454545455	1518\\
0.563636363636364	1513\\
0.572727272727273	1516\\
0.581818181818182	1519\\
0.590909090909091	1526\\
0.6	1529\\
0.609090909090909	1531\\
0.618181818181818	1534\\
0.627272727272727	1537\\
0.636363636363636	1540\\
0.645454545454545	1543\\
0.654545454545455	1546\\
0.663636363636364	1549\\
0.672727272727273	1549\\
0.681818181818182	1552\\
0.690909090909091	1555\\
0.7	1557\\
0.709090909090909	1561\\
0.718181818181818	1564\\
0.727272727272727	1563\\
0.736363636363636	1567\\
0.745454545454545	1570\\
0.754545454545455	1573\\
0.763636363636364	1573\\
0.772727272727273	1578\\
0.781818181818182	1581\\
0.790909090909091	1580\\
0.8	1584\\
0.809090909090909	1587\\
0.818181818181818	1586\\
0.827272727272727	1589\\
0.836363636363636	1588\\
0.845454545454545	1600\\
0.854545454545454	1599\\
0.863636363636364	1603\\
0.872727272727273	1606\\
0.881818181818182	1606\\
0.890909090909091	1599\\
0.9	1598\\
0.909090909090909	1602\\
0.918181818181818	1601\\
0.927272727272727	1607\\
0.936363636363636	1606\\
0.945454545454545	1609\\
0.954545454545455	1609\\
0.963636363636364	1608\\
0.972727272727273	1612\\
0.981818181818182	1611\\
0.990909090909091	1615\\
1	1614\\
};
\end{axis}

\begin{axis}[%
width=1.66in,
height=1.258in,
at={(1.011in,2.39in)},
scale only axis,
xmin=0,
xmax=1,
xlabel style={font=\color{white!15!black}},
xlabel={$\tau$},
ymin=783,
ymax=1674,
axis background/.style={fill=white},
title style={font=\bfseries},
title={$\gamma\text{= 0.40}$}
]
\addplot [color=mycolor1, line width=2.0pt, forget plot]
  table[row sep=crcr]{%
0.1	783\\
0.109090909090909	840\\
0.118181818181818	887\\
0.127272727272727	923\\
0.136363636363636	963\\
0.145454545454545	994\\
0.154545454545455	1024\\
0.163636363636364	1054\\
0.172727272727273	1084\\
0.181818181818182	1107\\
0.190909090909091	1130\\
0.2	1142\\
0.209090909090909	1171\\
0.218181818181818	1190\\
0.227272727272727	1210\\
0.236363636363636	1226\\
0.245454545454545	1243\\
0.254545454545455	1262\\
0.263636363636364	1276\\
0.272727272727273	1291\\
0.281818181818182	1305\\
0.290909090909091	1321\\
0.3	1332\\
0.309090909090909	1343\\
0.318181818181818	1354\\
0.327272727272727	1368\\
0.336363636363636	1380\\
0.345454545454545	1390\\
0.354545454545454	1400\\
0.363636363636364	1408\\
0.372727272727273	1419\\
0.381818181818182	1428\\
0.390909090909091	1435\\
0.4	1445\\
0.409090909090909	1454\\
0.418181818181818	1460\\
0.427272727272727	1462\\
0.436363636363636	1468\\
0.445454545454545	1484\\
0.454545454545455	1490\\
0.463636363636364	1496\\
0.472727272727273	1503\\
0.481818181818182	1508\\
0.490909090909091	1512\\
0.5	1518\\
0.509090909090909	1524\\
0.518181818181818	1531\\
0.527272727272727	1533\\
0.536363636363636	1540\\
0.545454545454546	1547\\
0.554545454545455	1549\\
0.563636363636364	1556\\
0.572727272727273	1558\\
0.581818181818182	1564\\
0.590909090909091	1567\\
0.6	1574\\
0.609090909090909	1576\\
0.618181818181818	1579\\
0.627272727272727	1584\\
0.636363636363636	1587\\
0.645454545454545	1589\\
0.654545454545455	1592\\
0.663636363636364	1595\\
0.672727272727273	1598\\
0.681818181818182	1601\\
0.690909090909091	1604\\
0.7	1611\\
0.709090909090909	1614\\
0.718181818181818	1617\\
0.727272727272727	1616\\
0.736363636363636	1619\\
0.745454545454545	1622\\
0.754545454545455	1617\\
0.763636363636364	1620\\
0.772727272727273	1623\\
0.781818181818182	1626\\
0.790909090909091	1630\\
0.8	1633\\
0.809090909090909	1632\\
0.818181818181818	1635\\
0.827272727272727	1638\\
0.836363636363636	1642\\
0.845454545454545	1641\\
0.854545454545454	1644\\
0.863636363636364	1647\\
0.872727272727273	1647\\
0.881818181818182	1649\\
0.890909090909091	1653\\
0.9	1653\\
0.909090909090909	1656\\
0.918181818181818	1660\\
0.927272727272727	1658\\
0.936363636363636	1661\\
0.945454545454545	1665\\
0.954545454545455	1665\\
0.963636363636364	1668\\
0.972727272727273	1668\\
0.981818181818182	1671\\
0.990909090909091	1670\\
1	1674\\
};
\end{axis}

\begin{axis}[%
width=1.66in,
height=1.258in,
at={(3.195in,2.39in)},
scale only axis,
xmin=0,
xmax=1,
xlabel style={font=\color{white!15!black}},
xlabel={$\tau$},
ymin=763,
ymax=1713,
axis background/.style={fill=white},
title style={font=\bfseries},
title={$\gamma\text{= 0.50}$}
]
\addplot [color=mycolor1, line width=2.0pt, forget plot]
  table[row sep=crcr]{%
0.1	763\\
0.109090909090909	806\\
0.118181818181818	859\\
0.127272727272727	893\\
0.136363636363636	939\\
0.145454545454545	972\\
0.154545454545455	1007\\
0.163636363636364	1039\\
0.172727272727273	1054\\
0.181818181818182	1090\\
0.190909090909091	1114\\
0.2	1136\\
0.209090909090909	1161\\
0.218181818181818	1182\\
0.227272727272727	1200\\
0.236363636363636	1207\\
0.245454545454545	1228\\
0.254545454545455	1246\\
0.263636363636364	1267\\
0.272727272727273	1283\\
0.281818181818182	1297\\
0.290909090909091	1312\\
0.3	1325\\
0.309090909090909	1336\\
0.318181818181818	1352\\
0.327272727272727	1364\\
0.336363636363636	1375\\
0.345454545454545	1386\\
0.354545454545454	1398\\
0.363636363636364	1408\\
0.372727272727273	1420\\
0.381818181818182	1428\\
0.390909090909091	1435\\
0.4	1446\\
0.409090909090909	1455\\
0.418181818181818	1464\\
0.427272727272727	1471\\
0.436363636363636	1481\\
0.445454545454545	1490\\
0.454545454545455	1495\\
0.463636363636364	1507\\
0.472727272727273	1512\\
0.481818181818182	1516\\
0.490909090909091	1526\\
0.5	1532\\
0.509090909090909	1538\\
0.518181818181818	1544\\
0.527272727272727	1550\\
0.536363636363636	1549\\
0.545454545454546	1555\\
0.554545454545455	1568\\
0.563636363636364	1570\\
0.572727272727273	1577\\
0.581818181818182	1584\\
0.590909090909091	1590\\
0.6	1592\\
0.609090909090909	1596\\
0.618181818181818	1599\\
0.627272727272727	1605\\
0.636363636363636	1608\\
0.645454545454545	1614\\
0.654545454545455	1617\\
0.663636363636364	1620\\
0.672727272727273	1628\\
0.681818181818182	1630\\
0.690909090909091	1633\\
0.7	1636\\
0.709090909090909	1642\\
0.718181818181818	1645\\
0.727272727272727	1648\\
0.736363636363636	1651\\
0.745454545454545	1654\\
0.754545454545455	1657\\
0.763636363636364	1660\\
0.772727272727273	1663\\
0.781818181818182	1668\\
0.790909090909091	1667\\
0.8	1670\\
0.809090909090909	1673\\
0.818181818181818	1676\\
0.827272727272727	1679\\
0.836363636363636	1682\\
0.845454545454545	1685\\
0.854545454545454	1689\\
0.863636363636364	1692\\
0.872727272727273	1691\\
0.881818181818182	1694\\
0.890909090909091	1697\\
0.9	1701\\
0.909090909090909	1700\\
0.918181818181818	1703\\
0.927272727272727	1706\\
0.936363636363636	1701\\
0.945454545454545	1700\\
0.954545454545455	1704\\
0.963636363636364	1707\\
0.972727272727273	1706\\
0.981818181818182	1710\\
0.990909090909091	1713\\
1	1713\\
};
\end{axis}

\begin{axis}[%
width=1.66in,
height=1.258in,
at={(5.379in,2.39in)},
scale only axis,
xmin=0,
xmax=1,
xlabel style={font=\color{white!15!black}},
xlabel={$\tau$},
ymin=500,
ymax=2000,
axis background/.style={fill=white},
title style={font=\bfseries},
title={$\gamma\text{= 0.60}$}
]
\addplot [color=mycolor1, line width=2.0pt, forget plot]
  table[row sep=crcr]{%
0.1	730\\
0.109090909090909	786\\
0.118181818181818	837\\
0.127272727272727	877\\
0.136363636363636	923\\
0.145454545454545	958\\
0.154545454545455	992\\
0.163636363636364	1022\\
0.172727272727273	1049\\
0.181818181818182	1068\\
0.190909090909091	1103\\
0.2	1124\\
0.209090909090909	1139\\
0.218181818181818	1170\\
0.227272727272727	1187\\
0.236363636363636	1207\\
0.245454545454545	1226\\
0.254545454545455	1243\\
0.263636363636364	1253\\
0.272727272727273	1275\\
0.281818181818182	1281\\
0.290909090909091	1294\\
0.3	1308\\
0.309090909090909	1331\\
0.318181818181818	1333\\
0.327272727272727	1355\\
0.336363636363636	1367\\
0.345454545454545	1379\\
0.354545454545454	1389\\
0.363636363636364	1400\\
0.372727272727273	1412\\
0.381818181818182	1420\\
0.390909090909091	1433\\
0.4	1441\\
0.409090909090909	1454\\
0.418181818181818	1462\\
0.427272727272727	1472\\
0.436363636363636	1481\\
0.445454545454545	1489\\
0.454545454545455	1494\\
0.463636363636364	1502\\
0.472727272727273	1511\\
0.481818181818182	1519\\
0.490909090909091	1528\\
0.5	1534\\
0.509090909090909	1537\\
0.518181818181818	1547\\
0.527272727272727	1553\\
0.536363636363636	1559\\
0.545454545454546	1565\\
0.554545454545455	1573\\
0.563636363636364	1579\\
0.572727272727273	1585\\
0.581818181818182	1589\\
0.590909090909091	1595\\
0.6	1602\\
0.609090909090909	1604\\
0.618181818181818	1610\\
0.627272727272727	1616\\
0.636363636363636	1612\\
0.645454545454545	1619\\
0.654545454545455	1625\\
0.663636363636364	1634\\
0.672727272727273	1641\\
0.681818181818182	1643\\
0.690909090909091	1646\\
0.7	1653\\
0.709090909090909	1656\\
0.718181818181818	1663\\
0.727272727272727	1663\\
0.736363636363636	1666\\
0.745454545454545	1669\\
0.754545454545455	1675\\
0.763636363636364	1678\\
0.772727272727273	1681\\
0.781818181818182	1684\\
0.790909090909091	1686\\
0.8	1689\\
0.809090909090909	1693\\
0.818181818181818	1696\\
0.827272727272727	1699\\
0.836363636363636	1706\\
0.845454545454545	1709\\
0.854545454545454	1708\\
0.863636363636364	1711\\
0.872727272727273	1714\\
0.881818181818182	1717\\
0.890909090909091	1721\\
0.9	1724\\
0.909090909090909	1727\\
0.918181818181818	1730\\
0.927272727272727	1733\\
0.936363636363636	1738\\
0.945454545454545	1733\\
0.954545454545455	1737\\
0.963636363636364	1740\\
0.972727272727273	1743\\
0.981818181818182	1742\\
0.990909090909091	1746\\
1	1749\\
};
\end{axis}

\begin{axis}[%
width=1.66in,
height=1.258in,
at={(1.011in,0.642in)},
scale only axis,
xmin=0,
xmax=1,
xlabel style={font=\color{white!15!black}},
xlabel={$\tau$},
ymin=500,
ymax=2000,
axis background/.style={fill=white},
title style={font=\bfseries},
title={$\gamma\text{= 0.70}$}
]
\addplot [color=mycolor1, line width=2.0pt, forget plot]
  table[row sep=crcr]{%
0.1	720\\
0.109090909090909	766\\
0.118181818181818	812\\
0.127272727272727	858\\
0.136363636363636	904\\
0.145454545454545	936\\
0.154545454545455	972\\
0.163636363636364	1004\\
0.172727272727273	1027\\
0.181818181818182	1057\\
0.190909090909091	1089\\
0.2	1105\\
0.209090909090909	1136\\
0.218181818181818	1149\\
0.227272727272727	1180\\
0.236363636363636	1198\\
0.245454545454545	1206\\
0.254545454545455	1230\\
0.263636363636364	1245\\
0.272727272727273	1264\\
0.281818181818182	1279\\
0.290909090909091	1295\\
0.3	1310\\
0.309090909090909	1318\\
0.318181818181818	1333\\
0.327272727272727	1339\\
0.336363636363636	1350\\
0.345454545454545	1361\\
0.354545454545454	1375\\
0.363636363636364	1396\\
0.372727272727273	1394\\
0.381818181818182	1414\\
0.390909090909091	1426\\
0.4	1434\\
0.409090909090909	1446\\
0.418181818181818	1453\\
0.427272727272727	1455\\
0.436363636363636	1470\\
0.445454545454545	1479\\
0.454545454545455	1488\\
0.463636363636364	1497\\
0.472727272727273	1506\\
0.481818181818182	1515\\
0.490909090909091	1520\\
0.5	1530\\
0.509090909090909	1536\\
0.518181818181818	1545\\
0.527272727272727	1550\\
0.536363636363636	1556\\
0.545454545454546	1564\\
0.554545454545455	1570\\
0.563636363636364	1577\\
0.572727272727273	1583\\
0.581818181818182	1589\\
0.590909090909091	1593\\
0.6	1599\\
0.609090909090909	1606\\
0.618181818181818	1612\\
0.627272727272727	1618\\
0.636363636363636	1620\\
0.645454545454545	1629\\
0.654545454545455	1635\\
0.663636363636364	1637\\
0.672727272727273	1643\\
0.681818181818182	1648\\
0.690909090909091	1650\\
0.7	1657\\
0.709090909090909	1660\\
0.718181818181818	1666\\
0.727272727272727	1669\\
0.736363636363636	1672\\
0.745454545454545	1671\\
0.754545454545455	1674\\
0.763636363636364	1681\\
0.772727272727273	1690\\
0.781818181818182	1693\\
0.790909090909091	1696\\
0.8	1702\\
0.809090909090909	1705\\
0.818181818181818	1709\\
0.827272727272727	1712\\
0.836363636363636	1714\\
0.845454545454545	1717\\
0.854545454545454	1718\\
0.863636363636364	1721\\
0.872727272727273	1728\\
0.881818181818182	1731\\
0.890909090909091	1734\\
0.9	1737\\
0.909090909090909	1740\\
0.918181818181818	1743\\
0.927272727272727	1742\\
0.936363636363636	1745\\
0.945454545454545	1749\\
0.954545454545455	1752\\
0.963636363636364	1755\\
0.972727272727273	1759\\
0.981818181818182	1762\\
0.990909090909091	1765\\
1	1764\\
};
\end{axis}

\begin{axis}[%
width=1.66in,
height=1.258in,
at={(3.195in,0.642in)},
scale only axis,
xmin=0,
xmax=1,
xlabel style={font=\color{white!15!black}},
xlabel={$\tau$},
ymin=500,
ymax=2000,
axis background/.style={fill=white},
title style={font=\bfseries},
title={$\gamma\text{= 0.80}$}
]
\addplot [color=mycolor1, line width=2.0pt, forget plot]
  table[row sep=crcr]{%
0.1	739\\
0.109090909090909	758\\
0.118181818181818	804\\
0.127272727272727	834\\
0.136363636363636	880\\
0.145454545454545	921\\
0.154545454545455	958\\
0.163636363636364	984\\
0.172727272727273	1015\\
0.181818181818182	1046\\
0.190909090909091	1074\\
0.2	1092\\
0.209090909090909	1116\\
0.218181818181818	1142\\
0.227272727272727	1156\\
0.236363636363636	1187\\
0.245454545454545	1201\\
0.254545454545455	1220\\
0.263636363636364	1237\\
0.272727272727273	1253\\
0.281818181818182	1260\\
0.290909090909091	1284\\
0.3	1298\\
0.309090909090909	1312\\
0.318181818181818	1324\\
0.327272727272727	1338\\
0.336363636363636	1351\\
0.345454545454545	1364\\
0.354545454545454	1375\\
0.363636363636364	1384\\
0.372727272727273	1386\\
0.381818181818182	1407\\
0.390909090909091	1405\\
0.4	1417\\
0.409090909090909	1426\\
0.418181818181818	1446\\
0.427272727272727	1444\\
0.436363636363636	1461\\
0.445454545454545	1469\\
0.454545454545455	1478\\
0.463636363636364	1487\\
0.472727272727273	1494\\
0.481818181818182	1503\\
0.490909090909091	1512\\
0.5	1517\\
0.509090909090909	1526\\
0.518181818181818	1536\\
0.527272727272727	1541\\
0.536363636363636	1547\\
0.545454545454546	1556\\
0.554545454545455	1561\\
0.563636363636364	1567\\
0.572727272727273	1577\\
0.581818181818182	1583\\
0.590909090909091	1589\\
0.6	1595\\
0.609090909090909	1601\\
0.618181818181818	1605\\
0.627272727272727	1611\\
0.636363636363636	1617\\
0.645454545454545	1625\\
0.654545454545455	1627\\
0.663636363636364	1633\\
0.672727272727273	1640\\
0.681818181818182	1644\\
0.690909090909091	1646\\
0.7	1653\\
0.709090909090909	1656\\
0.718181818181818	1662\\
0.727272727272727	1668\\
0.736363636363636	1673\\
0.745454545454545	1679\\
0.754545454545455	1682\\
0.763636363636364	1684\\
0.772727272727273	1689\\
0.781818181818182	1691\\
0.790909090909091	1698\\
0.8	1701\\
0.809090909090909	1704\\
0.818181818181818	1707\\
0.827272727272727	1714\\
0.836363636363636	1716\\
0.845454545454545	1720\\
0.854545454545454	1715\\
0.863636363636364	1718\\
0.872727272727273	1725\\
0.881818181818182	1735\\
0.890909090909091	1738\\
0.9	1741\\
0.909090909090909	1743\\
0.918181818181818	1746\\
0.927272727272727	1749\\
0.936363636363636	1753\\
0.945454545454545	1756\\
0.954545454545455	1759\\
0.963636363636364	1762\\
0.972727272727273	1763\\
0.981818181818182	1766\\
0.990909090909091	1769\\
1	1772\\
};
\end{axis}

\begin{axis}[%
width=1.66in,
height=1.258in,
at={(5.379in,0.642in)},
scale only axis,
xmin=0,
xmax=1,
xlabel style={font=\color{white!15!black}},
xlabel={$\tau$},
ymin=793,
ymax=2000,
axis background/.style={fill=white},
title style={font=\bfseries},
title={$\gamma\text{= 0.90}$}
]
\addplot [color=mycolor1, line width=2.0pt, forget plot]
  table[row sep=crcr]{%
0.1	799\\
0.109090909090909	793\\
0.118181818181818	802\\
0.127272727272727	835\\
0.136363636363636	874\\
0.145454545454545	902\\
0.154545454545455	937\\
0.163636363636364	968\\
0.172727272727273	1003\\
0.181818181818182	1036\\
0.190909090909091	1052\\
0.2	1078\\
0.209090909090909	1107\\
0.218181818181818	1130\\
0.227272727272727	1142\\
0.236363636363636	1163\\
0.245454545454545	1187\\
0.254545454545455	1198\\
0.263636363636364	1225\\
0.272727272727273	1231\\
0.281818181818182	1255\\
0.290909090909091	1272\\
0.3	1284\\
0.309090909090909	1297\\
0.318181818181818	1301\\
0.327272727272727	1322\\
0.336363636363636	1329\\
0.345454545454545	1345\\
0.354545454545454	1360\\
0.363636363636364	1371\\
0.372727272727273	1385\\
0.381818181818182	1394\\
0.390909090909091	1406\\
0.4	1413\\
0.409090909090909	1423\\
0.418181818181818	1422\\
0.427272727272727	1443\\
0.436363636363636	1442\\
0.445454545454545	1450\\
0.454545454545455	1458\\
0.463636363636364	1476\\
0.472727272727273	1485\\
0.481818181818182	1493\\
0.490909090909091	1500\\
0.5	1509\\
0.509090909090909	1514\\
0.518181818181818	1524\\
0.527272727272727	1531\\
0.536363636363636	1536\\
0.545454545454546	1542\\
0.554545454545455	1551\\
0.563636363636364	1556\\
0.572727272727273	1562\\
0.581818181818182	1572\\
0.590909090909091	1578\\
0.6	1583\\
0.609090909090909	1589\\
0.618181818181818	1595\\
0.627272727272727	1600\\
0.636363636363636	1606\\
0.645454545454545	1613\\
0.654545454545455	1619\\
0.663636363636364	1625\\
0.672727272727273	1631\\
0.681818181818182	1633\\
0.690909090909091	1638\\
0.7	1644\\
0.709090909090909	1651\\
0.718181818181818	1655\\
0.727272727272727	1661\\
0.736363636363636	1663\\
0.745454545454545	1670\\
0.754545454545455	1676\\
0.763636363636364	1676\\
0.772727272727273	1683\\
0.781818181818182	1686\\
0.790909090909091	1689\\
0.8	1695\\
0.809090909090909	1698\\
0.818181818181818	1704\\
0.827272727272727	1709\\
0.836363636363636	1712\\
0.845454545454545	1715\\
0.854545454545454	1721\\
0.863636363636364	1724\\
0.872727272727273	1725\\
0.881818181818182	1728\\
0.890909090909091	1734\\
0.9	1738\\
0.909090909090909	1741\\
0.918181818181818	1744\\
0.927272727272727	1746\\
0.936363636363636	1749\\
0.945454545454545	1752\\
0.954545454545455	1748\\
0.963636363636364	1752\\
0.972727272727273	1755\\
0.981818181818182	1758\\
0.990909090909091	1767\\
1	1770\\
};
\end{axis}
\end{tikzpicture}%}}  \\
\subfloat[][$N=50$]
{\resizebox{0.45\textwidth}{!}{% This file was created by matlab2tikz.
%
%The latest updates can be retrieved from
%  http://www.mathworks.com/matlabcentral/fileexchange/22022-matlab2tikz-matlab2tikz
%where you can also make suggestions and rate matlab2tikz.
%
\definecolor{mycolor1}{rgb}{0.00000,0.44700,0.74100}%
%
\begin{tikzpicture}

\begin{axis}[%
width=1.66in,
height=1.258in,
at={(1.011in,4.137in)},
scale only axis,
xmin=0,
xmax=1,
xlabel style={font=\color{white!15!black}},
xlabel={$\tau$},
ymin=1100,
ymax=1425,
axis background/.style={fill=white},
title style={font=\bfseries},
title={$\gamma\text{= 0.10}$}
]
\addplot [color=mycolor1, line width=2.0pt, forget plot]
  table[row sep=crcr]{%
0.1	1130\\
0.109090909090909	1148\\
0.118181818181818	1168\\
0.127272727272727	1193\\
0.136363636363636	1205\\
0.145454545454545	1210\\
0.154545454545455	1225\\
0.163636363636364	1227\\
0.172727272727273	1253\\
0.181818181818182	1262\\
0.190909090909091	1271\\
0.2	1276\\
0.209090909090909	1281\\
0.218181818181818	1286\\
0.227272727272727	1292\\
0.236363636363636	1298\\
0.245454545454545	1304\\
0.254545454545455	1318\\
0.263636363636364	1321\\
0.272727272727273	1328\\
0.281818181818182	1330\\
0.290909090909091	1337\\
0.3	1340\\
0.309090909090909	1342\\
0.318181818181818	1337\\
0.327272727272727	1339\\
0.336363636363636	1345\\
0.345454545454545	1345\\
0.354545454545454	1346\\
0.363636363636364	1349\\
0.372727272727273	1351\\
0.381818181818182	1356\\
0.390909090909091	1357\\
0.4	1358\\
0.409090909090909	1362\\
0.418181818181818	1366\\
0.427272727272727	1365\\
0.436363636363636	1368\\
0.445454545454545	1371\\
0.454545454545455	1371\\
0.463636363636364	1374\\
0.472727272727273	1374\\
0.481818181818182	1369\\
0.490909090909091	1376\\
0.5	1379\\
0.509090909090909	1378\\
0.518181818181818	1381\\
0.527272727272727	1382\\
0.536363636363636	1377\\
0.545454545454546	1385\\
0.554545454545455	1384\\
0.563636363636364	1388\\
0.572727272727273	1379\\
0.581818181818182	1389\\
0.590909090909091	1390\\
0.6	1388\\
0.609090909090909	1393\\
0.618181818181818	1391\\
0.627272727272727	1392\\
0.636363636363636	1395\\
0.645454545454545	1395\\
0.654545454545455	1386\\
0.663636363636364	1385\\
0.672727272727273	1398\\
0.681818181818182	1401\\
0.690909090909091	1400\\
0.7	1404\\
0.709090909090909	1404\\
0.718181818181818	1403\\
0.727272727272727	1403\\
0.736363636363636	1403\\
0.745454545454545	1402\\
0.754545454545455	1406\\
0.763636363636364	1406\\
0.772727272727273	1405\\
0.781818181818182	1409\\
0.790909090909091	1409\\
0.8	1402\\
0.809090909090909	1413\\
0.818181818181818	1412\\
0.827272727272727	1416\\
0.836363636363636	1415\\
0.845454545454545	1415\\
0.854545454545454	1418\\
0.863636363636364	1419\\
0.872727272727273	1419\\
0.881818181818182	1422\\
0.890909090909091	1422\\
0.9	1422\\
0.909090909090909	1422\\
0.918181818181818	1421\\
0.927272727272727	1421\\
0.936363636363636	1425\\
0.945454545454545	1425\\
0.954545454545455	1416\\
0.963636363636364	1424\\
0.972727272727273	1424\\
0.981818181818182	1424\\
0.990909090909091	1421\\
1	1424\\
};
\end{axis}

\begin{axis}[%
width=1.66in,
height=1.258in,
at={(3.195in,4.137in)},
scale only axis,
xmin=0,
xmax=1,
xlabel style={font=\color{white!15!black}},
xlabel={$\tau$},
ymin=1000,
ymax=1600,
axis background/.style={fill=white},
title style={font=\bfseries},
title={$\gamma\text{= 0.20}$}
]
\addplot [color=mycolor1, line width=2.0pt, forget plot]
  table[row sep=crcr]{%
0.1	1014\\
0.109090909090909	1042\\
0.118181818181818	1077\\
0.127272727272727	1104\\
0.136363636363636	1134\\
0.145454545454545	1156\\
0.154545454545455	1180\\
0.163636363636364	1201\\
0.172727272727273	1223\\
0.181818181818182	1247\\
0.190909090909091	1262\\
0.2	1277\\
0.209090909090909	1282\\
0.218181818181818	1298\\
0.227272727272727	1306\\
0.236363636363636	1319\\
0.245454545454545	1328\\
0.254545454545455	1347\\
0.263636363636364	1356\\
0.272727272727273	1361\\
0.281818181818182	1370\\
0.290909090909091	1368\\
0.3	1372\\
0.309090909090909	1385\\
0.318181818181818	1381\\
0.327272727272727	1387\\
0.336363636363636	1394\\
0.345454545454545	1413\\
0.354545454545454	1416\\
0.363636363636364	1422\\
0.372727272727273	1425\\
0.381818181818182	1432\\
0.390909090909091	1434\\
0.4	1436\\
0.409090909090909	1434\\
0.418181818181818	1437\\
0.427272727272727	1440\\
0.436363636363636	1443\\
0.445454545454545	1450\\
0.454545454545455	1453\\
0.463636363636364	1456\\
0.472727272727273	1459\\
0.481818181818182	1461\\
0.490909090909091	1461\\
0.5	1472\\
0.509090909090909	1475\\
0.518181818181818	1478\\
0.527272727272727	1481\\
0.536363636363636	1485\\
0.545454545454546	1485\\
0.554545454545455	1488\\
0.563636363636364	1491\\
0.572727272727273	1490\\
0.581818181818182	1494\\
0.590909090909091	1497\\
0.6	1496\\
0.609090909090909	1500\\
0.618181818181818	1499\\
0.627272727272727	1502\\
0.636363636363636	1498\\
0.645454545454545	1497\\
0.654545454545455	1500\\
0.663636363636364	1499\\
0.672727272727273	1502\\
0.681818181818182	1500\\
0.690909090909091	1506\\
0.7	1506\\
0.709090909090909	1503\\
0.718181818181818	1506\\
0.727272727272727	1506\\
0.736363636363636	1509\\
0.745454545454545	1508\\
0.754545454545455	1513\\
0.763636363636364	1513\\
0.772727272727273	1511\\
0.781818181818182	1514\\
0.790909090909091	1516\\
0.8	1515\\
0.809090909090909	1520\\
0.818181818181818	1519\\
0.827272727272727	1518\\
0.836363636363636	1523\\
0.845454545454545	1522\\
0.854545454545454	1522\\
0.863636363636364	1525\\
0.872727272727273	1525\\
0.881818181818182	1524\\
0.890909090909091	1528\\
0.9	1525\\
0.909090909090909	1528\\
0.918181818181818	1527\\
0.927272727272727	1531\\
0.936363636363636	1530\\
0.945454545454545	1531\\
0.954545454545455	1534\\
0.963636363636364	1526\\
0.972727272727273	1533\\
0.981818181818182	1533\\
0.990909090909091	1536\\
1	1535\\
};
\end{axis}

\begin{axis}[%
width=1.66in,
height=1.258in,
at={(5.379in,4.137in)},
scale only axis,
xmin=0,
xmax=1,
xlabel style={font=\color{white!15!black}},
xlabel={$\tau$},
ymin=946,
ymax=1630,
axis background/.style={fill=white},
title style={font=\bfseries},
title={$\gamma\text{= 0.30}$}
]
\addplot [color=mycolor1, line width=2.0pt, forget plot]
  table[row sep=crcr]{%
0.1	946\\
0.109090909090909	988\\
0.118181818181818	1035\\
0.127272727272727	1068\\
0.136363636363636	1103\\
0.145454545454545	1127\\
0.154545454545455	1160\\
0.163636363636364	1171\\
0.172727272727273	1199\\
0.181818181818182	1221\\
0.190909090909091	1241\\
0.2	1258\\
0.209090909090909	1276\\
0.218181818181818	1290\\
0.227272727272727	1309\\
0.236363636363636	1320\\
0.245454545454545	1335\\
0.254545454545455	1348\\
0.263636363636364	1358\\
0.272727272727273	1377\\
0.281818181818182	1386\\
0.290909090909091	1399\\
0.3	1408\\
0.309090909090909	1409\\
0.318181818181818	1419\\
0.327272727272727	1428\\
0.336363636363636	1434\\
0.345454545454545	1444\\
0.354545454545454	1449\\
0.363636363636364	1455\\
0.372727272727273	1462\\
0.381818181818182	1474\\
0.390909090909091	1480\\
0.4	1486\\
0.409090909090909	1492\\
0.418181818181818	1498\\
0.427272727272727	1493\\
0.436363636363636	1499\\
0.445454545454545	1499\\
0.454545454545455	1502\\
0.463636363636364	1512\\
0.472727272727273	1511\\
0.481818181818182	1511\\
0.490909090909091	1514\\
0.5	1522\\
0.509090909090909	1532\\
0.518181818181818	1541\\
0.527272727272727	1543\\
0.536363636363636	1546\\
0.545454545454546	1549\\
0.554545454545455	1553\\
0.563636363636364	1556\\
0.572727272727273	1559\\
0.581818181818182	1562\\
0.590909090909091	1564\\
0.6	1563\\
0.609090909090909	1562\\
0.618181818181818	1565\\
0.627272727272727	1564\\
0.636363636363636	1568\\
0.645454545454545	1571\\
0.654545454545455	1574\\
0.663636363636364	1574\\
0.672727272727273	1577\\
0.681818181818182	1580\\
0.690909090909091	1579\\
0.7	1583\\
0.709090909090909	1586\\
0.718181818181818	1585\\
0.727272727272727	1589\\
0.736363636363636	1588\\
0.745454545454545	1592\\
0.754545454545455	1603\\
0.763636363636364	1602\\
0.772727272727273	1606\\
0.781818181818182	1605\\
0.790909090909091	1609\\
0.8	1609\\
0.809090909090909	1612\\
0.818181818181818	1612\\
0.827272727272727	1616\\
0.836363636363636	1615\\
0.845454545454545	1619\\
0.854545454545454	1618\\
0.863636363636364	1618\\
0.872727272727273	1621\\
0.881818181818182	1621\\
0.890909090909091	1624\\
0.9	1624\\
0.909090909090909	1627\\
0.918181818181818	1627\\
0.927272727272727	1626\\
0.936363636363636	1630\\
0.945454545454545	1622\\
0.954545454545455	1621\\
0.963636363636364	1625\\
0.972727272727273	1624\\
0.981818181818182	1623\\
0.990909090909091	1627\\
1	1626\\
};
\end{axis}

\begin{axis}[%
width=1.66in,
height=1.258in,
at={(1.011in,2.39in)},
scale only axis,
xmin=0,
xmax=1,
xlabel style={font=\color{white!15!black}},
xlabel={$\tau$},
ymin=917,
ymax=1704,
axis background/.style={fill=white},
title style={font=\bfseries},
title={$\gamma\text{= 0.40}$}
]
\addplot [color=mycolor1, line width=2.0pt, forget plot]
  table[row sep=crcr]{%
0.1	917\\
0.109090909090909	967\\
0.118181818181818	1000\\
0.127272727272727	1039\\
0.136363636363636	1071\\
0.145454545454545	1105\\
0.154545454545455	1133\\
0.163636363636364	1156\\
0.172727272727273	1184\\
0.181818181818182	1213\\
0.190909090909091	1236\\
0.2	1255\\
0.209090909090909	1270\\
0.218181818181818	1282\\
0.227272727272727	1304\\
0.236363636363636	1320\\
0.245454545454545	1334\\
0.254545454545455	1348\\
0.263636363636364	1362\\
0.272727272727273	1374\\
0.281818181818182	1389\\
0.290909090909091	1401\\
0.3	1408\\
0.309090909090909	1421\\
0.318181818181818	1430\\
0.327272727272727	1442\\
0.336363636363636	1452\\
0.345454545454545	1460\\
0.354545454545454	1469\\
0.363636363636364	1485\\
0.372727272727273	1490\\
0.381818181818182	1500\\
0.390909090909091	1505\\
0.4	1511\\
0.409090909090909	1514\\
0.418181818181818	1520\\
0.427272727272727	1526\\
0.436363636363636	1532\\
0.445454545454545	1539\\
0.454545454545455	1545\\
0.463636363636364	1547\\
0.472727272727273	1553\\
0.481818181818182	1560\\
0.490909090909091	1562\\
0.5	1569\\
0.509090909090909	1582\\
0.518181818181818	1584\\
0.527272727272727	1587\\
0.536363636363636	1593\\
0.545454545454546	1596\\
0.554545454545455	1598\\
0.563636363636364	1605\\
0.572727272727273	1601\\
0.581818181818182	1603\\
0.590909090909091	1599\\
0.6	1602\\
0.609090909090909	1613\\
0.618181818181818	1616\\
0.627272727272727	1611\\
0.636363636363636	1612\\
0.645454545454545	1615\\
0.654545454545455	1618\\
0.663636363636364	1621\\
0.672727272727273	1625\\
0.681818181818182	1636\\
0.690909090909091	1645\\
0.7	1648\\
0.709090909090909	1647\\
0.718181818181818	1650\\
0.727272727272727	1654\\
0.736363636363636	1657\\
0.745454545454545	1657\\
0.754545454545455	1660\\
0.763636363636364	1663\\
0.772727272727273	1666\\
0.781818181818182	1666\\
0.790909090909091	1669\\
0.8	1668\\
0.809090909090909	1668\\
0.818181818181818	1666\\
0.827272727272727	1665\\
0.836363636363636	1669\\
0.845454545454545	1672\\
0.854545454545454	1672\\
0.863636363636364	1675\\
0.872727272727273	1675\\
0.881818181818182	1678\\
0.890909090909091	1678\\
0.9	1681\\
0.909090909090909	1681\\
0.918181818181818	1684\\
0.927272727272727	1684\\
0.936363636363636	1687\\
0.945454545454545	1686\\
0.954545454545455	1690\\
0.963636363636364	1689\\
0.972727272727273	1693\\
0.981818181818182	1693\\
0.990909090909091	1692\\
1	1704\\
};
\end{axis}

\begin{axis}[%
width=1.66in,
height=1.258in,
at={(3.195in,2.39in)},
scale only axis,
xmin=0,
xmax=1,
xlabel style={font=\color{white!15!black}},
xlabel={$\tau$},
ymin=899,
ymax=1753,
axis background/.style={fill=white},
title style={font=\bfseries},
title={$\gamma\text{= 0.50}$}
]
\addplot [color=mycolor1, line width=2.0pt, forget plot]
  table[row sep=crcr]{%
0.1	899\\
0.109090909090909	941\\
0.118181818181818	983\\
0.127272727272727	1024\\
0.136363636363636	1063\\
0.145454545454545	1095\\
0.154545454545455	1117\\
0.163636363636364	1149\\
0.172727272727273	1173\\
0.181818181818182	1198\\
0.190909090909091	1228\\
0.2	1239\\
0.209090909090909	1270\\
0.218181818181818	1280\\
0.227272727272727	1307\\
0.236363636363636	1324\\
0.245454545454545	1340\\
0.254545454545455	1354\\
0.263636363636364	1363\\
0.272727272727273	1372\\
0.281818181818182	1398\\
0.290909090909091	1402\\
0.3	1412\\
0.309090909090909	1427\\
0.318181818181818	1434\\
0.327272727272727	1446\\
0.336363636363636	1458\\
0.345454545454545	1466\\
0.354545454545454	1475\\
0.363636363636364	1488\\
0.372727272727273	1496\\
0.381818181818182	1501\\
0.390909090909091	1511\\
0.4	1520\\
0.409090909090909	1525\\
0.418181818181818	1536\\
0.427272727272727	1541\\
0.436363636363636	1550\\
0.445454545454545	1555\\
0.454545454545455	1568\\
0.463636363636364	1574\\
0.472727272727273	1580\\
0.481818181818182	1586\\
0.490909090909091	1592\\
0.5	1598\\
0.509090909090909	1605\\
0.518181818181818	1600\\
0.527272727272727	1606\\
0.536363636363636	1613\\
0.545454545454546	1615\\
0.554545454545455	1622\\
0.563636363636364	1625\\
0.572727272727273	1631\\
0.581818181818182	1634\\
0.590909090909091	1637\\
0.6	1644\\
0.609090909090909	1646\\
0.618181818181818	1650\\
0.627272727272727	1658\\
0.636363636363636	1665\\
0.645454545454545	1668\\
0.654545454545455	1671\\
0.663636363636364	1674\\
0.672727272727273	1677\\
0.681818181818182	1679\\
0.690909090909091	1682\\
0.7	1685\\
0.709090909090909	1688\\
0.718181818181818	1684\\
0.727272727272727	1687\\
0.736363636363636	1683\\
0.745454545454545	1686\\
0.754545454545455	1690\\
0.763636363636364	1697\\
0.772727272727273	1700\\
0.781818181818182	1695\\
0.790909090909091	1699\\
0.8	1700\\
0.809090909090909	1699\\
0.818181818181818	1702\\
0.827272727272727	1706\\
0.836363636363636	1710\\
0.845454545454545	1717\\
0.854545454545454	1720\\
0.863636363636364	1729\\
0.872727272727273	1728\\
0.881818181818182	1732\\
0.890909090909091	1735\\
0.9	1734\\
0.909090909090909	1738\\
0.918181818181818	1737\\
0.927272727272727	1741\\
0.936363636363636	1745\\
0.945454545454545	1744\\
0.954545454545455	1747\\
0.963636363636364	1747\\
0.972727272727273	1750\\
0.981818181818182	1749\\
0.990909090909091	1753\\
1	1748\\
};
\end{axis}

\begin{axis}[%
width=1.66in,
height=1.258in,
at={(5.379in,2.39in)},
scale only axis,
xmin=0,
xmax=1,
xlabel style={font=\color{white!15!black}},
xlabel={$\tau$},
ymin=867,
ymax=2000,
axis background/.style={fill=white},
title style={font=\bfseries},
title={$\gamma\text{= 0.60}$}
]
\addplot [color=mycolor1, line width=2.0pt, forget plot]
  table[row sep=crcr]{%
0.1	867\\
0.109090909090909	929\\
0.118181818181818	972\\
0.127272727272727	1015\\
0.136363636363636	1044\\
0.145454545454545	1080\\
0.154545454545455	1119\\
0.163636363636364	1142\\
0.172727272727273	1167\\
0.181818181818182	1191\\
0.190909090909091	1210\\
0.2	1234\\
0.209090909090909	1255\\
0.218181818181818	1274\\
0.227272727272727	1297\\
0.236363636363636	1318\\
0.245454545454545	1325\\
0.254545454545455	1347\\
0.263636363636364	1359\\
0.272727272727273	1379\\
0.281818181818182	1393\\
0.290909090909091	1403\\
0.3	1418\\
0.309090909090909	1433\\
0.318181818181818	1439\\
0.327272727272727	1445\\
0.336363636363636	1468\\
0.345454545454545	1469\\
0.354545454545454	1479\\
0.363636363636364	1487\\
0.372727272727273	1500\\
0.381818181818182	1507\\
0.390909090909091	1516\\
0.4	1525\\
0.409090909090909	1534\\
0.418181818181818	1539\\
0.427272727272727	1548\\
0.436363636363636	1557\\
0.445454545454545	1562\\
0.454545454545455	1568\\
0.463636363636364	1578\\
0.472727272727273	1584\\
0.481818181818182	1590\\
0.490909090909091	1595\\
0.5	1602\\
0.509090909090909	1608\\
0.518181818181818	1613\\
0.527272727272727	1619\\
0.536363636363636	1625\\
0.545454545454546	1638\\
0.554545454545455	1644\\
0.563636363636364	1647\\
0.572727272727273	1653\\
0.581818181818182	1655\\
0.590909090909091	1662\\
0.6	1668\\
0.609090909090909	1671\\
0.618181818181818	1666\\
0.627272727272727	1673\\
0.636363636363636	1676\\
0.645454545454545	1679\\
0.654545454545455	1685\\
0.663636363636364	1689\\
0.672727272727273	1692\\
0.681818181818182	1694\\
0.690909090909091	1701\\
0.7	1704\\
0.709090909090909	1707\\
0.718181818181818	1710\\
0.727272727272727	1713\\
0.736363636363636	1716\\
0.745454545454545	1720\\
0.754545454545455	1729\\
0.763636363636364	1732\\
0.772727272727273	1735\\
0.781818181818182	1738\\
0.790909090909091	1741\\
0.8	1744\\
0.809090909090909	1747\\
0.818181818181818	1746\\
0.827272727272727	1749\\
0.836363636363636	1752\\
0.845454545454545	1755\\
0.854545454545454	1751\\
0.863636363636364	1750\\
0.872727272727273	1754\\
0.881818181818182	1757\\
0.890909090909091	1753\\
0.9	1753\\
0.909090909090909	1756\\
0.918181818181818	1767\\
0.927272727272727	1770\\
0.936363636363636	1762\\
0.945454545454545	1765\\
0.954545454545455	1762\\
0.963636363636364	1766\\
0.972727272727273	1770\\
0.981818181818182	1769\\
0.990909090909091	1772\\
1	1776\\
};
\end{axis}

\begin{axis}[%
width=1.66in,
height=1.258in,
at={(1.011in,0.642in)},
scale only axis,
xmin=0,
xmax=1,
xlabel style={font=\color{white!15!black}},
xlabel={$\tau$},
ymin=850,
ymax=2000,
axis background/.style={fill=white},
title style={font=\bfseries},
title={$\gamma\text{= 0.70}$}
]
\addplot [color=mycolor1, line width=2.0pt, forget plot]
  table[row sep=crcr]{%
0.1	850\\
0.109090909090909	909\\
0.118181818181818	946\\
0.127272727272727	1000\\
0.136363636363636	1022\\
0.145454545454545	1064\\
0.154545454545455	1096\\
0.163636363636364	1124\\
0.172727272727273	1155\\
0.181818181818182	1186\\
0.190909090909091	1207\\
0.2	1229\\
0.209090909090909	1248\\
0.218181818181818	1263\\
0.227272727272727	1283\\
0.236363636363636	1298\\
0.245454545454545	1316\\
0.254545454545455	1331\\
0.263636363636364	1356\\
0.272727272727273	1362\\
0.281818181818182	1375\\
0.290909090909091	1397\\
0.3	1405\\
0.309090909090909	1420\\
0.318181818181818	1437\\
0.327272727272727	1448\\
0.336363636363636	1454\\
0.345454545454545	1470\\
0.354545454545454	1482\\
0.363636363636364	1494\\
0.372727272727273	1489\\
0.381818181818182	1499\\
0.390909090909091	1523\\
0.4	1525\\
0.409090909090909	1533\\
0.418181818181818	1540\\
0.427272727272727	1549\\
0.436363636363636	1555\\
0.445454545454545	1562\\
0.454545454545455	1571\\
0.463636363636364	1577\\
0.472727272727273	1586\\
0.481818181818182	1591\\
0.490909090909091	1598\\
0.5	1607\\
0.509090909090909	1612\\
0.518181818181818	1618\\
0.527272727272727	1624\\
0.536363636363636	1630\\
0.545454545454546	1637\\
0.554545454545455	1643\\
0.563636363636364	1649\\
0.572727272727273	1650\\
0.581818181818182	1658\\
0.590909090909091	1664\\
0.6	1670\\
0.609090909090909	1672\\
0.618181818181818	1678\\
0.627272727272727	1687\\
0.636363636363636	1694\\
0.645454545454545	1696\\
0.654545454545455	1703\\
0.663636363636364	1705\\
0.672727272727273	1708\\
0.681818181818182	1715\\
0.690909090909091	1717\\
0.7	1720\\
0.709090909090909	1727\\
0.718181818181818	1723\\
0.727272727272727	1725\\
0.736363636363636	1729\\
0.745454545454545	1731\\
0.754545454545455	1734\\
0.763636363636364	1741\\
0.772727272727273	1745\\
0.781818181818182	1748\\
0.790909090909091	1751\\
0.8	1754\\
0.809090909090909	1757\\
0.818181818181818	1760\\
0.827272727272727	1763\\
0.836363636363636	1762\\
0.845454545454545	1765\\
0.854545454545454	1768\\
0.863636363636364	1772\\
0.872727272727273	1776\\
0.881818181818182	1784\\
0.890909090909091	1788\\
0.9	1787\\
0.909090909090909	1790\\
0.918181818181818	1793\\
0.927272727272727	1796\\
0.936363636363636	1799\\
0.945454545454545	1799\\
0.954545454545455	1802\\
0.963636363636364	1805\\
0.972727272727273	1804\\
0.981818181818182	1808\\
0.990909090909091	1811\\
1	1807\\
};
\end{axis}

\begin{axis}[%
width=1.66in,
height=1.258in,
at={(3.195in,0.642in)},
scale only axis,
xmin=0,
xmax=1,
xlabel style={font=\color{white!15!black}},
xlabel={$\tau$},
ymin=850,
ymax=2000,
axis background/.style={fill=white},
title style={font=\bfseries},
title={$\gamma\text{= 0.80}$}
]
\addplot [color=mycolor1, line width=2.0pt, forget plot]
  table[row sep=crcr]{%
0.1	850\\
0.109090909090909	886\\
0.118181818181818	928\\
0.127272727272727	974\\
0.136363636363636	1018\\
0.145454545454545	1055\\
0.154545454545455	1077\\
0.163636363636364	1104\\
0.172727272727273	1138\\
0.181818181818182	1165\\
0.190909090909091	1188\\
0.2	1213\\
0.209090909090909	1241\\
0.218181818181818	1260\\
0.227272727272727	1276\\
0.236363636363636	1290\\
0.245454545454545	1307\\
0.254545454545455	1322\\
0.263636363636364	1338\\
0.272727272727273	1355\\
0.281818181818182	1369\\
0.290909090909091	1382\\
0.3	1403\\
0.309090909090909	1406\\
0.318181818181818	1421\\
0.327272727272727	1430\\
0.336363636363636	1450\\
0.345454545454545	1455\\
0.354545454545454	1467\\
0.363636363636364	1484\\
0.372727272727273	1492\\
0.381818181818182	1504\\
0.390909090909091	1515\\
0.4	1524\\
0.409090909090909	1532\\
0.418181818181818	1535\\
0.427272727272727	1537\\
0.436363636363636	1547\\
0.445454545454545	1568\\
0.454545454545455	1569\\
0.463636363636364	1574\\
0.472727272727273	1582\\
0.481818181818182	1587\\
0.490909090909091	1596\\
0.5	1602\\
0.509090909090909	1610\\
0.518181818181818	1616\\
0.527272727272727	1621\\
0.536363636363636	1627\\
0.545454545454546	1633\\
0.554545454545455	1639\\
0.563636363636364	1645\\
0.572727272727273	1651\\
0.581818181818182	1657\\
0.590909090909091	1662\\
0.6	1668\\
0.609090909090909	1671\\
0.618181818181818	1677\\
0.627272727272727	1684\\
0.636363636363636	1687\\
0.645454545454545	1693\\
0.654545454545455	1698\\
0.663636363636364	1702\\
0.672727272727273	1709\\
0.681818181818182	1711\\
0.690909090909091	1713\\
0.7	1719\\
0.709090909090909	1722\\
0.718181818181818	1731\\
0.727272727272727	1738\\
0.736363636363636	1741\\
0.745454545454545	1743\\
0.754545454545455	1750\\
0.763636363636364	1753\\
0.772727272727273	1756\\
0.781818181818182	1759\\
0.790909090909091	1762\\
0.8	1765\\
0.809090909090909	1768\\
0.818181818181818	1764\\
0.827272727272727	1767\\
0.836363636363636	1774\\
0.845454545454545	1777\\
0.854545454545454	1780\\
0.863636363636364	1779\\
0.872727272727273	1782\\
0.881818181818182	1786\\
0.890909090909091	1789\\
0.9	1792\\
0.909090909090909	1795\\
0.918181818181818	1798\\
0.927272727272727	1801\\
0.936363636363636	1805\\
0.945454545454545	1804\\
0.954545454545455	1807\\
0.963636363636364	1810\\
0.972727272727273	1813\\
0.981818181818182	1817\\
0.990909090909091	1817\\
1	1820\\
};
\end{axis}

\begin{axis}[%
width=1.66in,
height=1.258in,
at={(5.379in,0.642in)},
scale only axis,
xmin=0,
xmax=1,
xlabel style={font=\color{white!15!black}},
xlabel={$\tau$},
ymin=840,
ymax=2000,
axis background/.style={fill=white},
title style={font=\bfseries},
title={$\gamma\text{= 0.90}$}
]
\addplot [color=mycolor1, line width=2.0pt, forget plot]
  table[row sep=crcr]{%
0.1	840\\
0.109090909090909	884\\
0.118181818181818	919\\
0.127272727272727	961\\
0.136363636363636	1003\\
0.145454545454545	1040\\
0.154545454545455	1071\\
0.163636363636364	1104\\
0.172727272727273	1119\\
0.181818181818182	1152\\
0.190909090909091	1184\\
0.2	1200\\
0.209090909090909	1221\\
0.218181818181818	1242\\
0.227272727272727	1267\\
0.236363636363636	1287\\
0.245454545454545	1302\\
0.254545454545455	1315\\
0.263636363636364	1331\\
0.272727272727273	1348\\
0.281818181818182	1356\\
0.290909090909091	1369\\
0.3	1384\\
0.309090909090909	1396\\
0.318181818181818	1411\\
0.327272727272727	1421\\
0.336363636363636	1439\\
0.345454545454545	1442\\
0.354545454545454	1462\\
0.363636363636364	1465\\
0.372727272727273	1476\\
0.381818181818182	1492\\
0.390909090909091	1497\\
0.4	1505\\
0.409090909090909	1524\\
0.418181818181818	1532\\
0.427272727272727	1540\\
0.436363636363636	1544\\
0.445454545454545	1557\\
0.454545454545455	1566\\
0.463636363636364	1574\\
0.472727272727273	1574\\
0.481818181818182	1576\\
0.490909090909091	1583\\
0.5	1604\\
0.509090909090909	1609\\
0.518181818181818	1611\\
0.527272727272727	1616\\
0.536363636363636	1621\\
0.545454545454546	1626\\
0.554545454545455	1636\\
0.563636363636364	1642\\
0.572727272727273	1646\\
0.581818181818182	1652\\
0.590909090909091	1658\\
0.6	1660\\
0.609090909090909	1666\\
0.618181818181818	1671\\
0.627272727272727	1679\\
0.636363636363636	1685\\
0.645454545454545	1687\\
0.654545454545455	1693\\
0.663636363636364	1699\\
0.672727272727273	1701\\
0.681818181818182	1708\\
0.690909090909091	1711\\
0.7	1718\\
0.709090909090909	1720\\
0.718181818181818	1727\\
0.727272727272727	1729\\
0.736363636363636	1731\\
0.745454545454545	1739\\
0.754545454545455	1741\\
0.763636363636364	1744\\
0.772727272727273	1750\\
0.781818181818182	1753\\
0.790909090909091	1755\\
0.8	1758\\
0.809090909090909	1767\\
0.818181818181818	1775\\
0.827272727272727	1777\\
0.836363636363636	1780\\
0.845454545454545	1783\\
0.854545454545454	1786\\
0.863636363636364	1789\\
0.872727272727273	1792\\
0.881818181818182	1795\\
0.890909090909091	1798\\
0.9	1801\\
0.909090909090909	1804\\
0.918181818181818	1807\\
0.927272727272727	1803\\
0.936363636363636	1807\\
0.945454545454545	1810\\
0.954545454545455	1813\\
0.963636363636364	1812\\
0.972727272727273	1815\\
0.981818181818182	1818\\
0.990909090909091	1822\\
1	1826\\
};
\end{axis}
\end{tikzpicture}%}}
 \hfill 
\subfloat[][$N=60$]
{\resizebox{0.45\textwidth}{!}{ % This file was created by matlab2tikz.
%
%The latest updates can be retrieved from
%  http://www.mathworks.com/matlabcentral/fileexchange/22022-matlab2tikz-matlab2tikz
%where you can also make suggestions and rate matlab2tikz.
%
\definecolor{mycolor1}{rgb}{0.00000,0.44700,0.74100}%
%
\begin{tikzpicture}

\begin{axis}[%
width=1.66in,
height=1.258in,
at={(1.011in,4.137in)},
scale only axis,
xmin=0,
xmax=1,
xlabel style={font=\color{white!15!black}},
xlabel={$\tau$},
ymin=1100,
ymax=1400,
axis background/.style={fill=white},
title style={font=\bfseries},
title={$\gamma\text{= 0.10}$}
]
\addplot [color=mycolor1, line width=2.0pt, forget plot]
  table[row sep=crcr]{%
0.1	1155\\
0.109090909090909	1178\\
0.118181818181818	1194\\
0.127272727272727	1206\\
0.136363636363636	1232\\
0.145454545454545	1243\\
0.154545454545455	1241\\
0.163636363636364	1250\\
0.172727272727273	1260\\
0.181818181818182	1269\\
0.190909090909091	1277\\
0.2	1287\\
0.209090909090909	1301\\
0.218181818181818	1295\\
0.227272727272727	1301\\
0.236363636363636	1308\\
0.245454545454545	1311\\
0.254545454545455	1317\\
0.263636363636364	1320\\
0.272727272727273	1319\\
0.281818181818182	1323\\
0.290909090909091	1325\\
0.3	1328\\
0.309090909090909	1321\\
0.318181818181818	1334\\
0.327272727272727	1337\\
0.336363636363636	1341\\
0.345454545454545	1331\\
0.354545454545454	1344\\
0.363636363636364	1347\\
0.372727272727273	1350\\
0.381818181818182	1342\\
0.390909090909091	1347\\
0.4	1343\\
0.409090909090909	1348\\
0.418181818181818	1350\\
0.427272727272727	1349\\
0.436363636363636	1352\\
0.445454545454545	1353\\
0.454545454545455	1351\\
0.463636363636364	1354\\
0.472727272727273	1355\\
0.481818181818182	1356\\
0.490909090909091	1356\\
0.5	1359\\
0.509090909090909	1359\\
0.518181818181818	1359\\
0.527272727272727	1362\\
0.536363636363636	1361\\
0.545454545454546	1361\\
0.554545454545455	1363\\
0.563636363636364	1367\\
0.572727272727273	1366\\
0.581818181818182	1366\\
0.590909090909091	1369\\
0.6	1369\\
0.609090909090909	1375\\
0.618181818181818	1381\\
0.627272727272727	1382\\
0.636363636363636	1381\\
0.645454545454545	1381\\
0.654545454545455	1371\\
0.663636363636364	1369\\
0.672727272727273	1384\\
0.681818181818182	1384\\
0.690909090909091	1383\\
0.7	1387\\
0.709090909090909	1387\\
0.718181818181818	1386\\
0.727272727272727	1373\\
0.736363636363636	1386\\
0.745454545454545	1388\\
0.754545454545455	1387\\
0.763636363636364	1384\\
0.772727272727273	1392\\
0.781818181818182	1391\\
0.790909090909091	1384\\
0.8	1386\\
0.809090909090909	1383\\
0.818181818181818	1389\\
0.827272727272727	1388\\
0.836363636363636	1383\\
0.845454545454545	1394\\
0.854545454545454	1397\\
0.863636363636364	1394\\
0.872727272727273	1393\\
0.881818181818182	1395\\
0.890909090909091	1395\\
0.9	1391\\
0.909090909090909	1392\\
0.918181818181818	1384\\
0.927272727272727	1388\\
0.936363636363636	1382\\
0.945454545454545	1398\\
0.954545454545455	1383\\
0.963636363636364	1386\\
0.972727272727273	1385\\
0.981818181818182	1398\\
0.990909090909091	1388\\
1	1391\\
};
\end{axis}

\begin{axis}[%
width=1.66in,
height=1.258in,
at={(3.195in,4.137in)},
scale only axis,
xmin=0,
xmax=1,
xlabel style={font=\color{white!15!black}},
xlabel={$\tau$},
ymin=1000,
ymax=1600,
axis background/.style={fill=white},
title style={font=\bfseries},
title={$\gamma\text{= 0.20}$}
]
\addplot [color=mycolor1, line width=2.0pt, forget plot]
  table[row sep=crcr]{%
0.1	1095\\
0.109090909090909	1128\\
0.118181818181818	1154\\
0.127272727272727	1173\\
0.136363636363636	1215\\
0.145454545454545	1236\\
0.154545454545455	1243\\
0.163636363636364	1257\\
0.172727272727273	1276\\
0.181818181818182	1287\\
0.190909090909091	1310\\
0.2	1313\\
0.209090909090909	1323\\
0.218181818181818	1335\\
0.227272727272727	1345\\
0.236363636363636	1353\\
0.245454545454545	1356\\
0.254545454545455	1366\\
0.263636363636364	1371\\
0.272727272727273	1388\\
0.281818181818182	1394\\
0.290909090909091	1399\\
0.3	1406\\
0.309090909090909	1401\\
0.318181818181818	1408\\
0.327272727272727	1410\\
0.336363636363636	1413\\
0.345454545454545	1420\\
0.354545454545454	1423\\
0.363636363636364	1425\\
0.372727272727273	1432\\
0.381818181818182	1437\\
0.390909090909091	1441\\
0.4	1443\\
0.409090909090909	1455\\
0.418181818181818	1457\\
0.427272727272727	1460\\
0.436363636363636	1456\\
0.445454545454545	1459\\
0.454545454545455	1458\\
0.463636363636364	1461\\
0.472727272727273	1465\\
0.481818181818182	1467\\
0.490909090909091	1467\\
0.5	1471\\
0.509090909090909	1474\\
0.518181818181818	1473\\
0.527272727272727	1476\\
0.536363636363636	1480\\
0.545454545454546	1476\\
0.554545454545455	1480\\
0.563636363636364	1480\\
0.572727272727273	1483\\
0.581818181818182	1482\\
0.590909090909091	1486\\
0.6	1485\\
0.609090909090909	1488\\
0.618181818181818	1478\\
0.627272727272727	1491\\
0.636363636363636	1491\\
0.645454545454545	1494\\
0.654545454545455	1494\\
0.663636363636364	1493\\
0.672727272727273	1498\\
0.681818181818182	1499\\
0.690909090909091	1488\\
0.7	1502\\
0.709090909090909	1501\\
0.718181818181818	1505\\
0.727272727272727	1504\\
0.736363636363636	1503\\
0.745454545454545	1507\\
0.754545454545455	1488\\
0.763636363636364	1499\\
0.772727272727273	1506\\
0.781818181818182	1504\\
0.790909090909091	1504\\
0.8	1500\\
0.809090909090909	1505\\
0.818181818181818	1505\\
0.827272727272727	1504\\
0.836363636363636	1503\\
0.845454545454545	1506\\
0.854545454545454	1506\\
0.863636363636364	1506\\
0.872727272727273	1505\\
0.881818181818182	1509\\
0.890909090909091	1510\\
0.9	1508\\
0.909090909090909	1508\\
0.918181818181818	1511\\
0.927272727272727	1511\\
0.936363636363636	1512\\
0.945454545454545	1512\\
0.954545454545455	1511\\
0.963636363636364	1513\\
0.972727272727273	1513\\
0.981818181818182	1513\\
0.990909090909091	1513\\
1	1513\\
};
\end{axis}

\begin{axis}[%
width=1.66in,
height=1.258in,
at={(5.379in,4.137in)},
scale only axis,
xmin=0,
xmax=1,
xlabel style={font=\color{white!15!black}},
xlabel={$\tau$},
ymin=1000,
ymax=1622,
axis background/.style={fill=white},
title style={font=\bfseries},
title={$\gamma\text{= 0.30}$}
]
\addplot [color=mycolor1, line width=2.0pt, forget plot]
  table[row sep=crcr]{%
0.1	1051\\
0.109090909090909	1087\\
0.118181818181818	1127\\
0.127272727272727	1150\\
0.136363636363636	1178\\
0.145454545454545	1211\\
0.154545454545455	1235\\
0.163636363636364	1265\\
0.172727272727273	1279\\
0.181818181818182	1297\\
0.190909090909091	1306\\
0.2	1340\\
0.209090909090909	1340\\
0.218181818181818	1370\\
0.227272727272727	1366\\
0.236363636363636	1380\\
0.245454545454545	1391\\
0.254545454545455	1400\\
0.263636363636364	1409\\
0.272727272727273	1417\\
0.281818181818182	1434\\
0.290909090909091	1444\\
0.3	1443\\
0.309090909090909	1449\\
0.318181818181818	1456\\
0.327272727272727	1462\\
0.336363636363636	1468\\
0.345454545454545	1475\\
0.354545454545454	1480\\
0.363636363636364	1485\\
0.372727272727273	1490\\
0.381818181818182	1493\\
0.390909090909091	1499\\
0.4	1502\\
0.409090909090909	1519\\
0.418181818181818	1521\\
0.427272727272727	1524\\
0.436363636363636	1530\\
0.445454545454545	1533\\
0.454545454545455	1537\\
0.463636363636364	1529\\
0.472727272727273	1532\\
0.481818181818182	1535\\
0.490909090909091	1537\\
0.5	1541\\
0.509090909090909	1544\\
0.518181818181818	1548\\
0.527272727272727	1551\\
0.536363636363636	1553\\
0.545454545454546	1556\\
0.554545454545455	1556\\
0.563636363636364	1561\\
0.572727272727273	1565\\
0.581818181818182	1568\\
0.590909090909091	1568\\
0.6	1571\\
0.609090909090909	1582\\
0.618181818181818	1582\\
0.627272727272727	1585\\
0.636363636363636	1588\\
0.645454545454545	1587\\
0.654545454545455	1583\\
0.663636363636364	1583\\
0.672727272727273	1586\\
0.681818181818182	1586\\
0.690909090909091	1589\\
0.7	1589\\
0.709090909090909	1592\\
0.718181818181818	1592\\
0.727272727272727	1595\\
0.736363636363636	1595\\
0.745454545454545	1598\\
0.754545454545455	1598\\
0.763636363636364	1601\\
0.772727272727273	1601\\
0.781818181818182	1601\\
0.790909090909091	1604\\
0.8	1603\\
0.809090909090909	1607\\
0.818181818181818	1603\\
0.827272727272727	1604\\
0.836363636363636	1607\\
0.845454545454545	1607\\
0.854545454545454	1607\\
0.863636363636364	1610\\
0.872727272727273	1610\\
0.881818181818182	1610\\
0.890909090909091	1613\\
0.9	1612\\
0.909090909090909	1612\\
0.918181818181818	1616\\
0.927272727272727	1615\\
0.936363636363636	1615\\
0.945454545454545	1619\\
0.954545454545455	1618\\
0.963636363636364	1618\\
0.972727272727273	1618\\
0.981818181818182	1622\\
0.990909090909091	1621\\
1	1621\\
};
\end{axis}

\begin{axis}[%
width=1.66in,
height=1.258in,
at={(1.011in,2.39in)},
scale only axis,
xmin=0,
xmax=1,
xlabel style={font=\color{white!15!black}},
xlabel={$\tau$},
ymin=1000,
ymax=1800,
axis background/.style={fill=white},
title style={font=\bfseries},
title={$\gamma\text{= 0.40}$}
]
\addplot [color=mycolor1, line width=2.0pt, forget plot]
  table[row sep=crcr]{%
0.1	1011\\
0.109090909090909	1062\\
0.118181818181818	1099\\
0.127272727272727	1135\\
0.136363636363636	1169\\
0.145454545454545	1201\\
0.154545454545455	1221\\
0.163636363636364	1248\\
0.172727272727273	1269\\
0.181818181818182	1288\\
0.190909090909091	1309\\
0.2	1342\\
0.209090909090909	1351\\
0.218181818181818	1371\\
0.227272727272727	1387\\
0.236363636363636	1395\\
0.245454545454545	1411\\
0.254545454545455	1413\\
0.263636363636364	1444\\
0.272727272727273	1456\\
0.281818181818182	1450\\
0.290909090909091	1473\\
0.3	1478\\
0.309090909090909	1480\\
0.318181818181818	1500\\
0.327272727272727	1494\\
0.336363636363636	1504\\
0.345454545454545	1510\\
0.354545454545454	1518\\
0.363636363636364	1525\\
0.372727272727273	1539\\
0.381818181818182	1544\\
0.390909090909091	1551\\
0.4	1547\\
0.409090909090909	1553\\
0.418181818181818	1557\\
0.427272727272727	1564\\
0.436363636363636	1570\\
0.445454545454545	1572\\
0.454545454545455	1579\\
0.463636363636364	1582\\
0.472727272727273	1584\\
0.481818181818182	1589\\
0.490909090909091	1591\\
0.5	1594\\
0.509090909090909	1597\\
0.518181818181818	1600\\
0.527272727272727	1606\\
0.536363636363636	1613\\
0.545454545454546	1623\\
0.554545454545455	1626\\
0.563636363636364	1629\\
0.572727272727273	1632\\
0.581818181818182	1635\\
0.590909090909091	1634\\
0.6	1637\\
0.609090909090909	1641\\
0.618181818181818	1633\\
0.627272727272727	1636\\
0.636363636363636	1639\\
0.645454545454545	1639\\
0.654545454545455	1642\\
0.663636363636364	1646\\
0.672727272727273	1645\\
0.681818181818182	1648\\
0.690909090909091	1652\\
0.7	1652\\
0.709090909090909	1654\\
0.718181818181818	1658\\
0.727272727272727	1657\\
0.736363636363636	1661\\
0.745454545454545	1660\\
0.754545454545455	1666\\
0.763636363636364	1669\\
0.772727272727273	1669\\
0.781818181818182	1672\\
0.790909090909091	1672\\
0.8	1675\\
0.809090909090909	1683\\
0.818181818181818	1686\\
0.827272727272727	1686\\
0.836363636363636	1689\\
0.845454545454545	1689\\
0.854545454545454	1688\\
0.863636363636364	1684\\
0.872727272727273	1684\\
0.881818181818182	1687\\
0.890909090909091	1687\\
0.9	1687\\
0.909090909090909	1690\\
0.918181818181818	1690\\
0.927272727272727	1694\\
0.936363636363636	1693\\
0.945454545454545	1693\\
0.954545454545455	1696\\
0.963636363636364	1695\\
0.972727272727273	1696\\
0.981818181818182	1700\\
0.990909090909091	1699\\
1	1699\\
};
\end{axis}

\begin{axis}[%
width=1.66in,
height=1.258in,
at={(3.195in,2.39in)},
scale only axis,
xmin=0,
xmax=1,
xlabel style={font=\color{white!15!black}},
xlabel={$\tau$},
ymin=996,
ymax=1756,
axis background/.style={fill=white},
title style={font=\bfseries},
title={$\gamma\text{= 0.50}$}
]
\addplot [color=mycolor1, line width=2.0pt, forget plot]
  table[row sep=crcr]{%
0.1	996\\
0.109090909090909	1038\\
0.118181818181818	1092\\
0.127272727272727	1112\\
0.136363636363636	1158\\
0.145454545454545	1185\\
0.154545454545455	1215\\
0.163636363636364	1242\\
0.172727272727273	1267\\
0.181818181818182	1290\\
0.190909090909091	1308\\
0.2	1327\\
0.209090909090909	1343\\
0.218181818181818	1361\\
0.227272727272727	1378\\
0.236363636363636	1397\\
0.245454545454545	1413\\
0.254545454545455	1425\\
0.263636363636364	1440\\
0.272727272727273	1458\\
0.281818181818182	1470\\
0.290909090909091	1475\\
0.3	1488\\
0.309090909090909	1496\\
0.318181818181818	1500\\
0.327272727272727	1528\\
0.336363636363636	1536\\
0.345454545454545	1531\\
0.354545454545454	1551\\
0.363636363636364	1560\\
0.372727272727273	1562\\
0.381818181818182	1560\\
0.390909090909091	1566\\
0.4	1590\\
0.409090909090909	1582\\
0.418181818181818	1588\\
0.427272727272727	1595\\
0.436363636363636	1601\\
0.445454545454545	1602\\
0.454545454545455	1608\\
0.463636363636364	1623\\
0.472727272727273	1625\\
0.481818181818182	1631\\
0.490909090909091	1634\\
0.5	1631\\
0.509090909090909	1633\\
0.518181818181818	1640\\
0.527272727272727	1644\\
0.536363636363636	1647\\
0.545454545454546	1654\\
0.554545454545455	1657\\
0.563636363636364	1660\\
0.572727272727273	1663\\
0.581818181818182	1666\\
0.590909090909091	1668\\
0.6	1669\\
0.609090909090909	1672\\
0.618181818181818	1674\\
0.627272727272727	1678\\
0.636363636363636	1681\\
0.645454545454545	1684\\
0.654545454545455	1687\\
0.663636363636364	1690\\
0.672727272727273	1697\\
0.681818181818182	1707\\
0.690909090909091	1706\\
0.7	1709\\
0.709090909090909	1713\\
0.718181818181818	1716\\
0.727272727272727	1715\\
0.736363636363636	1718\\
0.745454545454545	1722\\
0.754545454545455	1721\\
0.763636363636364	1724\\
0.772727272727273	1717\\
0.781818181818182	1716\\
0.790909090909091	1719\\
0.8	1723\\
0.809090909090909	1723\\
0.818181818181818	1726\\
0.827272727272727	1725\\
0.836363636363636	1729\\
0.845454545454545	1729\\
0.854545454545454	1732\\
0.863636363636364	1732\\
0.872727272727273	1736\\
0.881818181818182	1739\\
0.890909090909091	1738\\
0.9	1737\\
0.909090909090909	1741\\
0.918181818181818	1741\\
0.927272727272727	1744\\
0.936363636363636	1744\\
0.945454545454545	1750\\
0.954545454545455	1749\\
0.963636363636364	1753\\
0.972727272727273	1753\\
0.981818181818182	1752\\
0.990909090909091	1756\\
1	1756\\
};
\end{axis}

\begin{axis}[%
width=1.66in,
height=1.258in,
at={(5.379in,2.39in)},
scale only axis,
xmin=0,
xmax=1,
xlabel style={font=\color{white!15!black}},
xlabel={$\tau$},
ymin=976,
ymax=2000,
axis background/.style={fill=white},
title style={font=\bfseries},
title={$\gamma\text{= 0.60}$}
]
\addplot [color=mycolor1, line width=2.0pt, forget plot]
  table[row sep=crcr]{%
0.1	976\\
0.109090909090909	1038\\
0.118181818181818	1077\\
0.127272727272727	1114\\
0.136363636363636	1138\\
0.145454545454545	1167\\
0.154545454545455	1205\\
0.163636363636364	1233\\
0.172727272727273	1265\\
0.181818181818182	1282\\
0.190909090909091	1304\\
0.2	1333\\
0.209090909090909	1346\\
0.218181818181818	1362\\
0.227272727272727	1378\\
0.236363636363636	1399\\
0.245454545454545	1410\\
0.254545454545455	1422\\
0.263636363636364	1437\\
0.272727272727273	1451\\
0.281818181818182	1463\\
0.290909090909091	1477\\
0.3	1502\\
0.309090909090909	1501\\
0.318181818181818	1513\\
0.327272727272727	1531\\
0.336363636363636	1540\\
0.345454545454545	1542\\
0.354545454545454	1551\\
0.363636363636364	1560\\
0.372727272727273	1577\\
0.381818181818182	1570\\
0.390909090909091	1594\\
0.4	1603\\
0.409090909090909	1608\\
0.418181818181818	1600\\
0.427272727272727	1624\\
0.436363636363636	1630\\
0.445454545454545	1636\\
0.454545454545455	1627\\
0.463636363636364	1633\\
0.472727272727273	1640\\
0.481818181818182	1660\\
0.490909090909091	1648\\
0.5	1655\\
0.509090909090909	1661\\
0.518181818181818	1663\\
0.527272727272727	1670\\
0.536363636363636	1672\\
0.545454545454546	1678\\
0.554545454545455	1689\\
0.563636363636364	1691\\
0.572727272727273	1698\\
0.581818181818182	1701\\
0.590909090909091	1704\\
0.6	1697\\
0.609090909090909	1704\\
0.618181818181818	1707\\
0.627272727272727	1711\\
0.636363636363636	1714\\
0.645454545454545	1718\\
0.654545454545455	1720\\
0.663636363636364	1723\\
0.672727272727273	1726\\
0.681818181818182	1729\\
0.690909090909091	1733\\
0.7	1736\\
0.709090909090909	1738\\
0.718181818181818	1736\\
0.727272727272727	1739\\
0.736363636363636	1741\\
0.745454545454545	1745\\
0.754545454545455	1748\\
0.763636363636364	1747\\
0.772727272727273	1751\\
0.781818181818182	1754\\
0.790909090909091	1757\\
0.8	1757\\
0.809090909090909	1770\\
0.818181818181818	1773\\
0.827272727272727	1773\\
0.836363636363636	1776\\
0.845454545454545	1780\\
0.854545454545454	1779\\
0.863636363636364	1782\\
0.872727272727273	1785\\
0.881818181818182	1785\\
0.890909090909091	1788\\
0.9	1788\\
0.909090909090909	1792\\
0.918181818181818	1780\\
0.927272727272727	1784\\
0.936363636363636	1783\\
0.945454545454545	1787\\
0.954545454545455	1786\\
0.963636363636364	1790\\
0.972727272727273	1789\\
0.981818181818182	1793\\
0.990909090909091	1792\\
1	1796\\
};
\end{axis}

\begin{axis}[%
width=1.66in,
height=1.258in,
at={(1.011in,0.642in)},
scale only axis,
xmin=0,
xmax=1,
xlabel style={font=\color{white!15!black}},
xlabel={$\tau$},
ymin=968,
ymax=2000,
axis background/.style={fill=white},
title style={font=\bfseries},
title={$\gamma\text{= 0.70}$}
]
\addplot [color=mycolor1, line width=2.0pt, forget plot]
  table[row sep=crcr]{%
0.1	968\\
0.109090909090909	1009\\
0.118181818181818	1055\\
0.127272727272727	1106\\
0.136363636363636	1120\\
0.145454545454545	1163\\
0.154545454545455	1190\\
0.163636363636364	1224\\
0.172727272727273	1240\\
0.181818181818182	1273\\
0.190909090909091	1294\\
0.2	1324\\
0.209090909090909	1343\\
0.218181818181818	1354\\
0.227272727272727	1369\\
0.236363636363636	1397\\
0.245454545454545	1406\\
0.254545454545455	1424\\
0.263636363636364	1433\\
0.272727272727273	1455\\
0.281818181818182	1457\\
0.290909090909091	1474\\
0.3	1483\\
0.309090909090909	1500\\
0.318181818181818	1510\\
0.327272727272727	1518\\
0.336363636363636	1533\\
0.345454545454545	1541\\
0.354545454545454	1554\\
0.363636363636364	1563\\
0.372727272727273	1577\\
0.381818181818182	1586\\
0.390909090909091	1595\\
0.4	1606\\
0.409090909090909	1603\\
0.418181818181818	1613\\
0.427272727272727	1619\\
0.436363636363636	1636\\
0.445454545454545	1625\\
0.454545454545455	1631\\
0.463636363636364	1655\\
0.472727272727273	1661\\
0.481818181818182	1652\\
0.490909090909091	1659\\
0.5	1679\\
0.509090909090909	1685\\
0.518181818181818	1691\\
0.527272727272727	1686\\
0.536363636363636	1685\\
0.545454545454546	1692\\
0.554545454545455	1709\\
0.563636363636364	1715\\
0.572727272727273	1703\\
0.581818181818182	1710\\
0.590909090909091	1713\\
0.6	1716\\
0.609090909090909	1723\\
0.618181818181818	1725\\
0.627272727272727	1727\\
0.636363636363636	1730\\
0.645454545454545	1745\\
0.654545454545455	1748\\
0.663636363636364	1751\\
0.672727272727273	1754\\
0.681818181818182	1757\\
0.690909090909091	1760\\
0.7	1753\\
0.709090909090909	1756\\
0.718181818181818	1759\\
0.727272727272727	1762\\
0.736363636363636	1766\\
0.745454545454545	1770\\
0.754545454545455	1773\\
0.763636363636364	1776\\
0.772727272727273	1775\\
0.781818181818182	1778\\
0.790909090909091	1782\\
0.8	1785\\
0.809090909090909	1789\\
0.818181818181818	1787\\
0.827272727272727	1790\\
0.836363636363636	1792\\
0.845454545454545	1795\\
0.854545454545454	1795\\
0.863636363636364	1797\\
0.872727272727273	1801\\
0.881818181818182	1800\\
0.890909090909091	1803\\
0.9	1807\\
0.909090909090909	1806\\
0.918181818181818	1810\\
0.927272727272727	1813\\
0.936363636363636	1813\\
0.945454545454545	1826\\
0.954545454545455	1826\\
0.963636363636364	1829\\
0.972727272727273	1828\\
0.981818181818182	1832\\
0.990909090909091	1832\\
1	1835\\
};
\end{axis}

\begin{axis}[%
width=1.66in,
height=1.258in,
at={(3.195in,0.642in)},
scale only axis,
xmin=0,
xmax=1,
xlabel style={font=\color{white!15!black}},
xlabel={$\tau$},
ymin=938,
ymax=2000,
axis background/.style={fill=white},
title style={font=\bfseries},
title={$\gamma\text{= 0.80}$}
]
\addplot [color=mycolor1, line width=2.0pt, forget plot]
  table[row sep=crcr]{%
0.1	938\\
0.109090909090909	987\\
0.118181818181818	1035\\
0.127272727272727	1074\\
0.136363636363636	1125\\
0.145454545454545	1158\\
0.154545454545455	1172\\
0.163636363636364	1208\\
0.172727272727273	1243\\
0.181818181818182	1255\\
0.190909090909091	1277\\
0.2	1301\\
0.209090909090909	1318\\
0.218181818181818	1345\\
0.227272727272727	1371\\
0.236363636363636	1379\\
0.245454545454545	1396\\
0.254545454545455	1412\\
0.263636363636364	1429\\
0.272727272727273	1443\\
0.281818181818182	1465\\
0.290909090909091	1472\\
0.3	1477\\
0.309090909090909	1489\\
0.318181818181818	1501\\
0.327272727272727	1516\\
0.336363636363636	1521\\
0.345454545454545	1534\\
0.354545454545454	1547\\
0.363636363636364	1554\\
0.372727272727273	1562\\
0.381818181818182	1575\\
0.390909090909091	1586\\
0.4	1604\\
0.409090909090909	1601\\
0.418181818181818	1610\\
0.427272727272727	1625\\
0.436363636363636	1630\\
0.445454545454545	1640\\
0.454545454545455	1647\\
0.463636363636364	1644\\
0.472727272727273	1654\\
0.481818181818182	1660\\
0.490909090909091	1667\\
0.5	1680\\
0.509090909090909	1669\\
0.518181818181818	1675\\
0.527272727272727	1700\\
0.536363636363636	1706\\
0.545454545454546	1708\\
0.554545454545455	1700\\
0.563636363636364	1707\\
0.572727272727273	1723\\
0.581818181818182	1730\\
0.590909090909091	1732\\
0.6	1731\\
0.609090909090909	1726\\
0.618181818181818	1733\\
0.627272727272727	1736\\
0.636363636363636	1757\\
0.645454545454545	1760\\
0.654545454545455	1748\\
0.663636363636364	1751\\
0.672727272727273	1758\\
0.681818181818182	1761\\
0.690909090909091	1764\\
0.7	1767\\
0.709090909090909	1768\\
0.718181818181818	1772\\
0.727272727272727	1775\\
0.736363636363636	1786\\
0.745454545454545	1789\\
0.754545454545455	1792\\
0.763636363636364	1795\\
0.772727272727273	1798\\
0.781818181818182	1801\\
0.790909090909091	1804\\
0.8	1798\\
0.809090909090909	1801\\
0.818181818181818	1804\\
0.827272727272727	1803\\
0.836363636363636	1808\\
0.845454545454545	1811\\
0.854545454545454	1815\\
0.863636363636364	1818\\
0.872727272727273	1817\\
0.881818181818182	1820\\
0.890909090909091	1823\\
0.9	1827\\
0.909090909090909	1826\\
0.918181818181818	1830\\
0.927272727272727	1833\\
0.936363636363636	1832\\
0.945454545454545	1835\\
0.954545454545455	1837\\
0.963636363636364	1836\\
0.972727272727273	1840\\
0.981818181818182	1838\\
0.990909090909091	1842\\
1	1845\\
};
\end{axis}

\begin{axis}[%
width=1.66in,
height=1.258in,
at={(5.379in,0.642in)},
scale only axis,
xmin=0,
xmax=1,
xlabel style={font=\color{white!15!black}},
xlabel={$\tau$},
ymin=926,
ymax=2000,
axis background/.style={fill=white},
title style={font=\bfseries},
title={$\gamma\text{= 0.90}$}
]
\addplot [color=mycolor1, line width=2.0pt, forget plot]
  table[row sep=crcr]{%
0.1	926\\
0.109090909090909	972\\
0.118181818181818	1029\\
0.127272727272727	1070\\
0.136363636363636	1109\\
0.145454545454545	1141\\
0.154545454545455	1163\\
0.163636363636364	1207\\
0.172727272727273	1214\\
0.181818181818182	1245\\
0.190909090909091	1277\\
0.2	1288\\
0.209090909090909	1313\\
0.218181818181818	1324\\
0.227272727272727	1343\\
0.236363636363636	1359\\
0.245454545454545	1387\\
0.254545454545455	1407\\
0.263636363636364	1414\\
0.272727272727273	1429\\
0.281818181818182	1443\\
0.290909090909091	1456\\
0.3	1481\\
0.309090909090909	1485\\
0.318181818181818	1496\\
0.327272727272727	1507\\
0.336363636363636	1513\\
0.345454545454545	1525\\
0.354545454545454	1544\\
0.363636363636364	1545\\
0.372727272727273	1558\\
0.381818181818182	1564\\
0.390909090909091	1573\\
0.4	1586\\
0.409090909090909	1589\\
0.418181818181818	1599\\
0.427272727272727	1608\\
0.436363636363636	1619\\
0.445454545454545	1624\\
0.454545454545455	1643\\
0.463636363636364	1640\\
0.472727272727273	1649\\
0.481818181818182	1661\\
0.490909090909091	1666\\
0.5	1676\\
0.509090909090909	1682\\
0.518181818181818	1681\\
0.527272727272727	1688\\
0.536363636363636	1693\\
0.545454545454546	1700\\
0.554545454545455	1702\\
0.563636363636364	1716\\
0.572727272727273	1706\\
0.581818181818182	1712\\
0.590909090909091	1733\\
0.6	1739\\
0.609090909090909	1745\\
0.618181818181818	1733\\
0.627272727272727	1740\\
0.636363636363636	1757\\
0.645454545454545	1760\\
0.654545454545455	1766\\
0.663636363636364	1769\\
0.672727272727273	1764\\
0.681818181818182	1763\\
0.690909090909091	1766\\
0.7	1769\\
0.709090909090909	1773\\
0.718181818181818	1793\\
0.727272727272727	1782\\
0.736363636363636	1784\\
0.745454545454545	1787\\
0.754545454545455	1791\\
0.763636363636364	1794\\
0.772727272727273	1797\\
0.781818181818182	1800\\
0.790909090909091	1803\\
0.8	1805\\
0.809090909090909	1808\\
0.818181818181818	1811\\
0.827272727272727	1823\\
0.836363636363636	1826\\
0.845454545454545	1829\\
0.854545454545454	1832\\
0.863636363636364	1831\\
0.872727272727273	1834\\
0.881818181818182	1838\\
0.890909090909091	1841\\
0.9	1834\\
0.909090909090909	1833\\
0.918181818181818	1837\\
0.927272727272727	1840\\
0.936363636363636	1845\\
0.945454545454545	1844\\
0.954545454545455	1848\\
0.963636363636364	1851\\
0.972727272727273	1851\\
0.981818181818182	1854\\
0.990909090909091	1857\\
1	1856\\
};
\end{axis}
\end{tikzpicture}%}} 
\caption{Nei grafici viene rappresentato come varia la differenza tra gli intervalli temporali calcolati da ode45 e ode15s facendo variare $\tau$ per alcuni valori di $\gamma$ fissati.\\
Per realizzare questi grafici abbiamo utilizzato grafi di Erdos-Renyi con probabilit\`a $0.5$  e dimensione $N$}
\label{Erdos_Lenght}

\end{figure}
\begin{figure}
	\centering
 % This file was created by matlab2tikz.
%
%The latest updates can be retrieved from
%  http://www.mathworks.com/matlabcentral/fileexchange/22022-matlab2tikz-matlab2tikz
%where you can also make suggestions and rate matlab2tikz.
%
\definecolor{mycolor1}{rgb}{0.00000,0.44700,0.74100}%
\definecolor{mycolor2}{rgb}{0.85000,0.32500,0.09800}%
%
\begin{tikzpicture}[scale=0.5]


\begin{axis}[%
width=3.737in,
height=1.699in,
at={(1.452in,5.625in)},
scale only axis,
xmin=0,
xmax=100,
ymode=log,
ymin=0.023510973,
ymax=0.056310127,
yminorticks=true,
ylabel style={font=\color{white!15!black}},
ylabel={N=10},
axis background/.style={fill=white},
title style={font=\bfseries},
title={N=10},
legend style={legend cell align=left, align=left, draw=white!15!black}
]
\addplot [color=mycolor1]
  table[row sep=crcr]{%
1	0.053135833\\
2	0.051933192\\
3	0.051931381\\
4	0.052065829\\
5	0.052097173\\
6	0.052251903\\
7	0.0518648\\
8	0.052835444\\
9	0.05132012\\
10	0.05141084\\
11	0.051852224\\
12	0.051961991\\
13	0.055363165\\
14	0.050518181\\
15	0.053829209\\
16	0.050949125\\
17	0.050610291\\
18	0.05172307\\
19	0.051579368\\
20	0.051356178\\
21	0.051287683\\
22	0.051153119\\
23	0.052090543\\
24	0.052097785\\
25	0.051937302\\
26	0.051801514\\
27	0.051395662\\
28	0.050803376\\
29	0.051831679\\
30	0.056310127\\
31	0.051405206\\
32	0.051807362\\
33	0.051786538\\
34	0.050821881\\
35	0.052279083\\
36	0.051491122\\
37	0.054636211\\
38	0.052525446\\
39	0.051724792\\
40	0.052096671\\
41	0.051135504\\
42	0.051555183\\
43	0.051760704\\
44	0.053937802\\
45	0.051624878\\
46	0.051901192\\
47	0.05156277\\
48	0.051531102\\
49	0.051910852\\
50	0.051653867\\
51	0.052002345\\
52	0.050787181\\
53	0.051651078\\
54	0.051784256\\
55	0.052007009\\
56	0.051545796\\
57	0.052355296\\
58	0.052562291\\
59	0.051289405\\
60	0.051999842\\
61	0.05477453\\
62	0.052205614\\
63	0.051357104\\
64	0.050770759\\
65	0.051392637\\
66	0.053235752\\
67	0.051077748\\
68	0.053691641\\
69	0.05199781\\
70	0.051252256\\
71	0.051842428\\
72	0.051077791\\
73	0.052409155\\
74	0.050825242\\
75	0.051424696\\
76	0.051313714\\
77	0.050746295\\
78	0.051208253\\
79	0.051695961\\
80	0.050769472\\
81	0.050691876\\
82	0.050856854\\
83	0.050990479\\
84	0.051590852\\
85	0.051014502\\
86	0.054137156\\
87	0.050573125\\
88	0.052442486\\
89	0.051378839\\
90	0.051371802\\
91	0.051171348\\
92	0.05597903\\
93	0.052216607\\
94	0.051780118\\
95	0.053309441\\
96	0.051478142\\
97	0.051835657\\
98	0.051512036\\
99	0.051713856\\
100	0.051330694\\
};
\addlegendentry{ode15s}

\addplot [color=mycolor2]
  table[row sep=crcr]{%
1	0.047234009\\
2	0.024754564\\
3	0.02418504\\
4	0.024781125\\
5	0.024555127\\
6	0.02402762\\
7	0.023976518\\
8	0.024691543\\
9	0.023907458\\
10	0.0240048\\
11	0.023890861\\
12	0.023808589\\
13	0.023912193\\
14	0.024285765\\
15	0.024010567\\
16	0.02397536\\
17	0.024307188\\
18	0.02384781\\
19	0.024477261\\
20	0.023966781\\
21	0.023568343\\
22	0.024100689\\
23	0.023987228\\
24	0.024832173\\
25	0.024222687\\
26	0.024299711\\
27	0.024085204\\
28	0.023944515\\
29	0.024493056\\
30	0.026220562\\
31	0.024309354\\
32	0.024309403\\
33	0.024428813\\
34	0.023877957\\
35	0.0238673\\
36	0.024150204\\
37	0.0244681\\
38	0.024614095\\
39	0.024687812\\
40	0.024285609\\
41	0.024745243\\
42	0.024790744\\
43	0.024307709\\
44	0.024380996\\
45	0.024058748\\
46	0.024223712\\
47	0.023510973\\
48	0.024879135\\
49	0.023826982\\
50	0.02449544\\
51	0.024379198\\
52	0.023719624\\
53	0.024552983\\
54	0.024563766\\
55	0.023879079\\
56	0.024404828\\
57	0.024292669\\
58	0.024459095\\
59	0.024639413\\
60	0.024164087\\
61	0.02480616\\
62	0.024262132\\
63	0.023792652\\
64	0.024109734\\
65	0.024052875\\
66	0.024310862\\
67	0.024177397\\
68	0.024018523\\
69	0.023718979\\
70	0.024380105\\
71	0.023802639\\
72	0.02376467\\
73	0.025225996\\
74	0.02379265\\
75	0.023520557\\
76	0.023642624\\
77	0.024210153\\
78	0.02400733\\
79	0.024222249\\
80	0.023992372\\
81	0.024072563\\
82	0.023786832\\
83	0.023611163\\
84	0.024395921\\
85	0.024075185\\
86	0.023812925\\
87	0.023753497\\
88	0.024724962\\
89	0.024216718\\
90	0.024027183\\
91	0.024219908\\
92	0.024260312\\
93	0.023757134\\
94	0.023947857\\
95	0.024296566\\
96	0.024632574\\
97	0.024237335\\
98	0.024299525\\
99	0.024488713\\
100	0.024304722\\
};
\addlegendentry{ode45}

\end{axis}

\begin{axis}[%
width=3.737in,
height=1.699in,
at={(6.369in,5.625in)},
scale only axis,
xmin=0,
xmax=100,
ymode=log,
ymin=0.032823739,
ymax=0.105307099,
yminorticks=true,
ylabel style={font=\color{white!15!black}},
ylabel={N=20},
axis background/.style={fill=white},
title style={font=\bfseries},
title={N=20},
legend style={legend cell align=left, align=left, draw=white!15!black}
]
\addplot [color=mycolor1]
  table[row sep=crcr]{%
1	0.072366793\\
2	0.073472487\\
3	0.073091019\\
4	0.072910095\\
5	0.072863384\\
6	0.07472233\\
7	0.073158845\\
8	0.07351977\\
9	0.072226065\\
10	0.073282941\\
11	0.071952756\\
12	0.072185253\\
13	0.07213424\\
14	0.072997437\\
15	0.075681005\\
16	0.073180489\\
17	0.073500805\\
18	0.07431085\\
19	0.072616404\\
20	0.072533921\\
21	0.072499682\\
22	0.072084009\\
23	0.071988551\\
24	0.073188067\\
25	0.072741171\\
26	0.072615245\\
27	0.072014664\\
28	0.071959293\\
29	0.075850028\\
30	0.094560349\\
31	0.077613225\\
32	0.072321572\\
33	0.073827022\\
34	0.074281378\\
35	0.076074807\\
36	0.072425129\\
37	0.07245299\\
38	0.07256455\\
39	0.072373307\\
40	0.072027144\\
41	0.090276104\\
42	0.078990287\\
43	0.0939307\\
44	0.105307099\\
45	0.088015445\\
46	0.071632048\\
47	0.07758805\\
48	0.072770875\\
49	0.071553616\\
50	0.073818081\\
51	0.079919822\\
52	0.073237524\\
53	0.072484001\\
54	0.073019566\\
55	0.080921232\\
56	0.072388456\\
57	0.072665141\\
58	0.072264358\\
59	0.07240974\\
60	0.073922969\\
61	0.073368923\\
62	0.072215184\\
63	0.076166523\\
64	0.074939622\\
65	0.072025142\\
66	0.083769257\\
67	0.072324592\\
68	0.072951659\\
69	0.073132936\\
70	0.089427445\\
71	0.095495418\\
72	0.071966145\\
73	0.073033935\\
74	0.076276265\\
75	0.073226523\\
76	0.07605433\\
77	0.072645618\\
78	0.073066613\\
79	0.077350298\\
80	0.072773361\\
81	0.072893165\\
82	0.073199615\\
83	0.078782568\\
84	0.07951127\\
85	0.07625119\\
86	0.075656438\\
87	0.095523334\\
88	0.082551682\\
89	0.084561041\\
90	0.079357268\\
91	0.076716452\\
92	0.074332023\\
93	0.0734048\\
94	0.077415601\\
95	0.08181105\\
96	0.098810813\\
97	0.081679602\\
98	0.086884734\\
99	0.098429106\\
100	0.073288293\\
};
\addlegendentry{ode15s}

\addplot [color=mycolor2]
  table[row sep=crcr]{%
1	0.034435974\\
2	0.033885298\\
3	0.033232564\\
4	0.033387766\\
5	0.033513962\\
6	0.033202951\\
7	0.033561961\\
8	0.033298934\\
9	0.033524159\\
10	0.033097917\\
11	0.033186669\\
12	0.032847036\\
13	0.034948059\\
14	0.032890082\\
15	0.033232277\\
16	0.039809866\\
17	0.039540022\\
18	0.032823739\\
19	0.033478193\\
20	0.033271165\\
21	0.032933199\\
22	0.03341035\\
23	0.03310364\\
24	0.034789819\\
25	0.033666152\\
26	0.03309516\\
27	0.033469212\\
28	0.03325483\\
29	0.032965352\\
30	0.039672028\\
31	0.037161552\\
32	0.033771555\\
33	0.033389228\\
34	0.033769315\\
35	0.033254895\\
36	0.03333265\\
37	0.033247049\\
38	0.033791301\\
39	0.033264947\\
40	0.03351075\\
41	0.041899351\\
42	0.033525818\\
43	0.042802136\\
44	0.044188956\\
45	0.034896716\\
46	0.033739228\\
47	0.033289641\\
48	0.033329615\\
49	0.03434577\\
50	0.035478239\\
51	0.034607696\\
52	0.033567046\\
53	0.034444539\\
54	0.033886316\\
55	0.033244476\\
56	0.033338627\\
57	0.033472908\\
58	0.038239529\\
59	0.033458237\\
60	0.034089431\\
61	0.033440734\\
62	0.033349151\\
63	0.033569293\\
64	0.033828268\\
65	0.033157002\\
66	0.033473943\\
67	0.033919321\\
68	0.032930772\\
69	0.033720607\\
70	0.04955653\\
71	0.042308774\\
72	0.03368201\\
73	0.033423088\\
74	0.034012322\\
75	0.033333418\\
76	0.033377893\\
77	0.033287302\\
78	0.033643293\\
79	0.033184448\\
80	0.033988465\\
81	0.033972319\\
82	0.039139308\\
83	0.038630418\\
84	0.034645569\\
85	0.035039967\\
86	0.035176554\\
87	0.047438354\\
88	0.038091102\\
89	0.036711863\\
90	0.039370718\\
91	0.038161241\\
92	0.034944373\\
93	0.033592517\\
94	0.034288044\\
95	0.035316863\\
96	0.056560421\\
97	0.036478248\\
98	0.044887018\\
99	0.044587833\\
100	0.034380343\\
};
\addlegendentry{ode45}

\end{axis}

\begin{axis}[%
width=3.737in,
height=1.699in,
at={(1.452in,3.249in)},
scale only axis,
xmin=0,
xmax=100,
ymode=log,
ymin=0.038926351,
ymax=0.255117096,
yminorticks=true,
ylabel style={font=\color{white!15!black}},
ylabel={N=30},
axis background/.style={fill=white},
title style={font=\bfseries},
title={N=30},
legend style={legend cell align=left, align=left, draw=white!15!black}
]
\addplot [color=mycolor1]
  table[row sep=crcr]{%
1	0.096842226\\
2	0.092324999\\
3	0.101880407\\
4	0.106469146\\
5	0.105677766\\
6	0.092914108\\
7	0.096989502\\
8	0.093650241\\
9	0.098619251\\
10	0.115754523\\
11	0.115827433\\
12	0.255117096\\
13	0.157183773\\
14	0.162888131\\
15	0.187923269\\
16	0.122246545\\
17	0.106370799\\
18	0.096345825\\
19	0.123830641\\
20	0.218701419\\
21	0.181891177\\
22	0.148901654\\
23	0.181082242\\
24	0.144904524\\
25	0.094061536\\
26	0.111552801\\
27	0.093996739\\
28	0.118513226\\
29	0.096207335\\
30	0.109826136\\
31	0.162591729\\
32	0.104301985\\
33	0.105427775\\
34	0.118805047\\
35	0.12073344\\
36	0.106203299\\
37	0.116765333\\
38	0.111162412\\
39	0.102482798\\
40	0.103886122\\
41	0.111199966\\
42	0.111979688\\
43	0.125101683\\
44	0.112298373\\
45	0.110260861\\
46	0.105945486\\
47	0.099435382\\
48	0.104285605\\
49	0.146159398\\
50	0.116515649\\
51	0.124903773\\
52	0.104087502\\
53	0.151012945\\
54	0.131614526\\
55	0.095454138\\
56	0.10529986\\
57	0.121058173\\
58	0.094241177\\
59	0.092679329\\
60	0.093671879\\
61	0.103109393\\
62	0.100343514\\
63	0.106348882\\
64	0.102853862\\
65	0.093258332\\
66	0.120267216\\
67	0.095376485\\
68	0.094521009\\
69	0.109566611\\
70	0.095924535\\
71	0.114447835\\
72	0.094726423\\
73	0.093762779\\
74	0.093574783\\
75	0.095016777\\
76	0.109485414\\
77	0.161092967\\
78	0.13060977\\
79	0.117539499\\
80	0.095932662\\
81	0.097542396\\
82	0.099324364\\
83	0.117110158\\
84	0.097273911\\
85	0.101801331\\
86	0.101130541\\
87	0.145668456\\
88	0.110999362\\
89	0.094465303\\
90	0.095888819\\
91	0.100304612\\
92	0.094504734\\
93	0.093787968\\
94	0.095085927\\
95	0.093265349\\
96	0.094150957\\
97	0.093405732\\
98	0.094942207\\
99	0.093636542\\
100	0.097165218\\
};
\addlegendentry{ode15s}

\addplot [color=mycolor2]
  table[row sep=crcr]{%
1	0.039855614\\
2	0.040035684\\
3	0.043371834\\
4	0.041576723\\
5	0.042339397\\
6	0.040327304\\
7	0.052849988\\
8	0.040044504\\
9	0.038981939\\
10	0.054742812\\
11	0.046146556\\
12	0.186999874\\
13	0.074623337\\
14	0.083063194\\
15	0.056440096\\
16	0.06806865\\
17	0.052370747\\
18	0.043471012\\
19	0.050852518\\
20	0.088839554\\
21	0.088997651\\
22	0.073067948\\
23	0.061415407\\
24	0.079431866\\
25	0.039620361\\
26	0.041954891\\
27	0.039267627\\
28	0.041172835\\
29	0.04123435\\
30	0.043767695\\
31	0.044169194\\
32	0.04543884\\
33	0.042375815\\
34	0.050290944\\
35	0.056389274\\
36	0.045791548\\
37	0.043690713\\
38	0.042215847\\
39	0.049613837\\
40	0.041053331\\
41	0.043201651\\
42	0.049351915\\
43	0.053064154\\
44	0.046801364\\
45	0.046123759\\
46	0.041806436\\
47	0.039993656\\
48	0.040876439\\
49	0.066770924\\
50	0.060763482\\
51	0.052578485\\
52	0.047963085\\
53	0.050475833\\
54	0.041022543\\
55	0.040705039\\
56	0.043125971\\
57	0.047010842\\
58	0.054655196\\
59	0.042192894\\
60	0.064366759\\
61	0.039878669\\
62	0.043328409\\
63	0.05527052\\
64	0.039226536\\
65	0.039534575\\
66	0.041693034\\
67	0.042084069\\
68	0.040722068\\
69	0.10991629\\
70	0.039787106\\
71	0.057175471\\
72	0.054163996\\
73	0.041459241\\
74	0.040352963\\
75	0.042838889\\
76	0.061141887\\
77	0.070194222\\
78	0.059675504\\
79	0.04998925\\
80	0.039791639\\
81	0.038926351\\
82	0.042466161\\
83	0.049356589\\
84	0.040542876\\
85	0.041034716\\
86	0.043842115\\
87	0.053013886\\
88	0.045469254\\
89	0.042220095\\
90	0.041125103\\
91	0.042597802\\
92	0.040504179\\
93	0.04053427\\
94	0.041801306\\
95	0.040110695\\
96	0.039979689\\
97	0.040248266\\
98	0.044760913\\
99	0.040349154\\
100	0.040377198\\
};
\addlegendentry{ode45}

\end{axis}

\begin{axis}[%
width=3.737in,
height=1.699in,
at={(6.369in,3.249in)},
scale only axis,
xmin=0,
xmax=100,
ymode=log,
ymin=0.045610167,
ymax=0.156605074,
yminorticks=true,
ylabel style={font=\color{white!15!black}},
ylabel={N=40},
axis background/.style={fill=white},
title style={font=\bfseries},
title={N=40},
legend style={legend cell align=left, align=left, draw=white!15!black}
]
\addplot [color=mycolor1]
  table[row sep=crcr]{%
1	0.108763395\\
2	0.107725785\\
3	0.112154188\\
4	0.108717332\\
5	0.108911913\\
6	0.109973548\\
7	0.108150759\\
8	0.110071682\\
9	0.107059043\\
10	0.107664184\\
11	0.113837962\\
12	0.107647113\\
13	0.107916393\\
14	0.111135704\\
15	0.10750059\\
16	0.108099937\\
17	0.107942576\\
18	0.106977955\\
19	0.108791312\\
20	0.109874842\\
21	0.10838786\\
22	0.11279744\\
23	0.112773617\\
24	0.111450841\\
25	0.108944178\\
26	0.107244095\\
27	0.113869599\\
28	0.111777063\\
29	0.110075569\\
30	0.110790208\\
31	0.107351206\\
32	0.107946513\\
33	0.106719676\\
34	0.107136168\\
35	0.109097364\\
36	0.108314512\\
37	0.112166382\\
38	0.111438217\\
39	0.106394352\\
40	0.10911117\\
41	0.106146658\\
42	0.106504727\\
43	0.110477196\\
44	0.146581507\\
45	0.109398053\\
46	0.109335056\\
47	0.108011419\\
48	0.112871403\\
49	0.119026494\\
50	0.141823138\\
51	0.10613075\\
52	0.108833251\\
53	0.108371682\\
54	0.10668808\\
55	0.151511278\\
56	0.131006148\\
57	0.124595971\\
58	0.156605074\\
59	0.122897786\\
60	0.141378939\\
61	0.120415894\\
62	0.110095996\\
63	0.133973558\\
64	0.109005721\\
65	0.136961929\\
66	0.123382206\\
67	0.148330945\\
68	0.124482955\\
69	0.137765421\\
70	0.107441428\\
71	0.106685411\\
72	0.109030642\\
73	0.107225645\\
74	0.106793768\\
75	0.108907258\\
76	0.106871651\\
77	0.106940684\\
78	0.107724107\\
79	0.107530441\\
80	0.108295255\\
81	0.106482512\\
82	0.136227266\\
83	0.112301155\\
84	0.130554901\\
85	0.106797439\\
86	0.106654767\\
87	0.106718488\\
88	0.106806786\\
89	0.106864729\\
90	0.107012463\\
91	0.107339816\\
92	0.109835521\\
93	0.106680704\\
94	0.108410445\\
95	0.107122165\\
96	0.114131116\\
97	0.109024275\\
98	0.114204815\\
99	0.11400229\\
100	0.107380569\\
};
\addlegendentry{ode15s}

\addplot [color=mycolor2]
  table[row sep=crcr]{%
1	0.047378488\\
2	0.04838678\\
3	0.047321275\\
4	0.047175895\\
5	0.047355636\\
6	0.047092388\\
7	0.048881909\\
8	0.051737258\\
9	0.049035883\\
10	0.046764031\\
11	0.054342049\\
12	0.0470019\\
13	0.04714543\\
14	0.047136611\\
15	0.047732197\\
16	0.047502012\\
17	0.046882741\\
18	0.049302434\\
19	0.046396337\\
20	0.047271524\\
21	0.047150054\\
22	0.047103099\\
23	0.046859081\\
24	0.049809248\\
25	0.047409934\\
26	0.047724665\\
27	0.046894662\\
28	0.047283195\\
29	0.047100709\\
30	0.046853868\\
31	0.046834819\\
32	0.051346422\\
33	0.04734681\\
34	0.046804781\\
35	0.046210951\\
36	0.048139809\\
37	0.051684707\\
38	0.04671719\\
39	0.046085042\\
40	0.053942158\\
41	0.046794804\\
42	0.045851569\\
43	0.049405906\\
44	0.053352239\\
45	0.050543177\\
46	0.047254569\\
47	0.046909915\\
48	0.052124036\\
49	0.05434185\\
50	0.075790129\\
51	0.046285643\\
52	0.052123362\\
53	0.046973835\\
54	0.051035935\\
55	0.04721223\\
56	0.058447089\\
57	0.046833038\\
58	0.058185142\\
59	0.05886897\\
60	0.058935912\\
61	0.057529468\\
62	0.046094469\\
63	0.050354478\\
64	0.047384419\\
65	0.056986929\\
66	0.079586304\\
67	0.0682766\\
68	0.080169795\\
69	0.048013002\\
70	0.047678944\\
71	0.04605905\\
72	0.050049732\\
73	0.04631953\\
74	0.046877977\\
75	0.047727158\\
76	0.046539603\\
77	0.047083424\\
78	0.046397883\\
79	0.046211473\\
80	0.047187764\\
81	0.045610167\\
82	0.059173393\\
83	0.047856414\\
84	0.049160269\\
85	0.047436928\\
86	0.047001335\\
87	0.047115443\\
88	0.046600323\\
89	0.046032861\\
90	0.045773762\\
91	0.048960064\\
92	0.047174084\\
93	0.047879778\\
94	0.04722543\\
95	0.046747806\\
96	0.048168769\\
97	0.047207423\\
98	0.047845767\\
99	0.048931783\\
100	0.047893127\\
};
\addlegendentry{ode45}

\end{axis}

\begin{axis}[%
width=3.737in,
height=1.699in,
at={(1.452in,0.872in)},
scale only axis,
xmin=0,
xmax=100,
ymode=log,
ymin=0.052105425,
ymax=0.316240058,
yminorticks=true,
ylabel style={font=\color{white!15!black}},
ylabel={N=50},
axis background/.style={fill=white},
title style={font=\bfseries},
title={N=50},
legend style={legend cell align=left, align=left, draw=white!15!black}
]
\addplot [color=mycolor1]
  table[row sep=crcr]{%
1	0.142285542\\
2	0.139571952\\
3	0.13890928\\
4	0.139256052\\
5	0.139937723\\
6	0.138953763\\
7	0.142746884\\
8	0.137605782\\
9	0.151127942\\
10	0.139332391\\
11	0.144010912\\
12	0.147040664\\
13	0.144132378\\
14	0.145995605\\
15	0.147578164\\
16	0.163031036\\
17	0.16328365\\
18	0.151359354\\
19	0.149697588\\
20	0.17898374\\
21	0.149168861\\
22	0.149076628\\
23	0.154902551\\
24	0.187111605\\
25	0.150861298\\
26	0.139436919\\
27	0.146540259\\
28	0.139819873\\
29	0.141044413\\
30	0.149689386\\
31	0.210431574\\
32	0.145774344\\
33	0.157252448\\
34	0.147402526\\
35	0.150569533\\
36	0.151813356\\
37	0.152808011\\
38	0.162147305\\
39	0.147537059\\
40	0.148170558\\
41	0.150046774\\
42	0.151060721\\
43	0.155380085\\
44	0.144608965\\
45	0.152240592\\
46	0.148687228\\
47	0.151706028\\
48	0.154412583\\
49	0.157771928\\
50	0.159597716\\
51	0.15794481\\
52	0.173172454\\
53	0.145394737\\
54	0.149524934\\
55	0.180271043\\
56	0.151256398\\
57	0.15998874\\
58	0.154274641\\
59	0.156978727\\
60	0.154903589\\
61	0.152483083\\
62	0.169148402\\
63	0.169871954\\
64	0.154171758\\
65	0.154842189\\
66	0.144876614\\
67	0.140776694\\
68	0.147367484\\
69	0.204809527\\
70	0.156771936\\
71	0.151943964\\
72	0.150521754\\
73	0.150924492\\
74	0.165053734\\
75	0.155230445\\
76	0.158757357\\
77	0.15380155\\
78	0.178783173\\
79	0.158128612\\
80	0.161977757\\
81	0.151881818\\
82	0.155255627\\
83	0.15014689\\
84	0.220587941\\
85	0.216691482\\
86	0.166135303\\
87	0.151715115\\
88	0.15144137\\
89	0.157798117\\
90	0.144264752\\
91	0.15277459\\
92	0.161063021\\
93	0.155123338\\
94	0.156619307\\
95	0.316240058\\
96	0.150880455\\
97	0.190996425\\
98	0.165483425\\
99	0.152057291\\
100	0.309289209\\
};
\addlegendentry{ode15s}

\addplot [color=mycolor2]
  table[row sep=crcr]{%
1	0.053700495\\
2	0.053803397\\
3	0.053559817\\
4	0.053955009\\
5	0.053527107\\
6	0.053225449\\
7	0.062723102\\
8	0.060076332\\
9	0.057950985\\
10	0.052105425\\
11	0.056706163\\
12	0.055274818\\
13	0.054651944\\
14	0.054998969\\
15	0.055286638\\
16	0.054864138\\
17	0.09780907\\
18	0.0542825\\
19	0.058165787\\
20	0.069957172\\
21	0.066385442\\
22	0.058362516\\
23	0.058047929\\
24	0.059447616\\
25	0.058075056\\
26	0.052786816\\
27	0.054381303\\
28	0.054670763\\
29	0.053575993\\
30	0.054321477\\
31	0.058862443\\
32	0.060005424\\
33	0.053936827\\
34	0.057091169\\
35	0.062826715\\
36	0.059901492\\
37	0.060950871\\
38	0.063623212\\
39	0.061814632\\
40	0.059273537\\
41	0.057673509\\
42	0.057789283\\
43	0.06037708\\
44	0.060927344\\
45	0.056252962\\
46	0.058707469\\
47	0.058884397\\
48	0.059652221\\
49	0.058364904\\
50	0.055661663\\
51	0.05838607\\
52	0.0614402\\
53	0.058244697\\
54	0.056065774\\
55	0.057460504\\
56	0.060502844\\
57	0.05672579\\
58	0.060986609\\
59	0.065931984\\
60	0.057870882\\
61	0.058530726\\
62	0.057636272\\
63	0.056236209\\
64	0.062795688\\
65	0.055451651\\
66	0.053557101\\
67	0.05370632\\
68	0.053472177\\
69	0.05961671\\
70	0.060744437\\
71	0.061064146\\
72	0.06126653\\
73	0.056917958\\
74	0.062181426\\
75	0.060799839\\
76	0.05767973\\
77	0.057477607\\
78	0.064553598\\
79	0.060174484\\
80	0.060765857\\
81	0.064529571\\
82	0.057963599\\
83	0.060974876\\
84	0.057801428\\
85	0.072565186\\
86	0.076130617\\
87	0.05639738\\
88	0.057086903\\
89	0.063661874\\
90	0.057119064\\
91	0.057458002\\
92	0.063277753\\
93	0.05507521\\
94	0.057339962\\
95	0.090141697\\
96	0.107466537\\
97	0.078086347\\
98	0.063168681\\
99	0.059090691\\
100	0.057067553\\
};
\addlegendentry{ode45}

\end{axis}

\begin{axis}[%
width=3.737in,
height=1.699in,
at={(6.369in,0.872in)},
scale only axis,
xmin=0,
xmax=100,
ymode=log,
ymin=0.056850998,
ymax=0.405803008,
yminorticks=true,
ylabel style={font=\color{white!15!black}},
ylabel={N=60},
axis background/.style={fill=white},
title style={font=\bfseries},
title={N=60},
legend style={legend cell align=left, align=left, draw=white!15!black}
]
\addplot [color=mycolor1]
  table[row sep=crcr]{%
1	0.162890757\\
2	0.197433532\\
3	0.182373511\\
4	0.279917924\\
5	0.237636282\\
6	0.191843087\\
7	0.194161251\\
8	0.185722942\\
9	0.405803008\\
10	0.223605474\\
11	0.193431012\\
12	0.151030918\\
13	0.162317421\\
14	0.238662796\\
15	0.216862644\\
16	0.285687413\\
17	0.16118177\\
18	0.185238621\\
19	0.246072622\\
20	0.161484824\\
21	0.156004829\\
22	0.163724787\\
23	0.152596345\\
24	0.169482499\\
25	0.172357168\\
26	0.162883729\\
27	0.16287181\\
28	0.270993758\\
29	0.168288643\\
30	0.192727144\\
31	0.170793856\\
32	0.188338021\\
33	0.204826205\\
34	0.175742222\\
35	0.175059622\\
36	0.163917283\\
37	0.182932028\\
38	0.179265245\\
39	0.211606133\\
40	0.159689152\\
41	0.17185799\\
42	0.156092163\\
43	0.194747316\\
44	0.159354979\\
45	0.181136931\\
46	0.166188214\\
47	0.208153624\\
48	0.268218124\\
49	0.177377892\\
50	0.164230939\\
51	0.177683741\\
52	0.189100815\\
53	0.292896328\\
54	0.291947577\\
55	0.186186305\\
56	0.172132277\\
57	0.213581628\\
58	0.231826336\\
59	0.202216683\\
60	0.203513411\\
61	0.179744463\\
62	0.186081459\\
63	0.170110788\\
64	0.178322981\\
65	0.150600701\\
66	0.162603258\\
67	0.175477496\\
68	0.177144312\\
69	0.175107213\\
70	0.174221818\\
71	0.173056088\\
72	0.163158138\\
73	0.194123243\\
74	0.156601677\\
75	0.181414975\\
76	0.162687225\\
77	0.175401988\\
78	0.180930454\\
79	0.179385179\\
80	0.151995958\\
81	0.157440599\\
82	0.157029093\\
83	0.16128163\\
84	0.16964229\\
85	0.174600518\\
86	0.170988751\\
87	0.168584862\\
88	0.162423952\\
89	0.172251153\\
90	0.165109895\\
91	0.159459873\\
92	0.158158424\\
93	0.166589156\\
94	0.172031585\\
95	0.167146242\\
96	0.170235131\\
97	0.168389085\\
98	0.154152357\\
99	0.165686627\\
100	0.165988328\\
};
\addlegendentry{ode15s}

\addplot [color=mycolor2]
  table[row sep=crcr]{%
1	0.058204966\\
2	0.070739723\\
3	0.065186763\\
4	0.079167973\\
5	0.083420108\\
6	0.081765499\\
7	0.075538082\\
8	0.081692539\\
9	0.135430667\\
10	0.091438347\\
11	0.083686751\\
12	0.059883\\
13	0.061503773\\
14	0.058371758\\
15	0.088941726\\
16	0.084341606\\
17	0.063605446\\
18	0.066328998\\
19	0.060237509\\
20	0.061845203\\
21	0.068496538\\
22	0.060261138\\
23	0.057985451\\
24	0.061542828\\
25	0.068351462\\
26	0.058504962\\
27	0.064648349\\
28	0.069015205\\
29	0.075874092\\
30	0.060749549\\
31	0.101807026\\
32	0.080278291\\
33	0.059666415\\
34	0.062244534\\
35	0.060113589\\
36	0.059510767\\
37	0.091337503\\
38	0.06243063\\
39	0.066727165\\
40	0.059552583\\
41	0.069643561\\
42	0.068037344\\
43	0.065504334\\
44	0.071196584\\
45	0.064422876\\
46	0.070653657\\
47	0.104409801\\
48	0.094309935\\
49	0.10044633\\
50	0.064357047\\
51	0.06400175\\
52	0.065744093\\
53	0.06315265\\
54	0.099131729\\
55	0.085242742\\
56	0.061404103\\
57	0.088744478\\
58	0.129671612\\
59	0.065683704\\
60	0.063264077\\
61	0.092381943\\
62	0.065402345\\
63	0.084712551\\
64	0.062571163\\
65	0.058104933\\
66	0.071852679\\
67	0.057120466\\
68	0.070623784\\
69	0.062992035\\
70	0.058219996\\
71	0.086380315\\
72	0.062316591\\
73	0.064571645\\
74	0.076292926\\
75	0.059881087\\
76	0.063317484\\
77	0.057491803\\
78	0.065220442\\
79	0.103399416\\
80	0.059650651\\
81	0.059976361\\
82	0.059780836\\
83	0.058634977\\
84	0.05892466\\
85	0.061034526\\
86	0.064170048\\
87	0.056850998\\
88	0.061538156\\
89	0.059858754\\
90	0.066000783\\
91	0.065443713\\
92	0.05917012\\
93	0.071199113\\
94	0.066903378\\
95	0.059633116\\
96	0.059300715\\
97	0.066279349\\
98	0.059411286\\
99	0.066072008\\
100	0.059230059\\
};
\addlegendentry{ode45}

\end{axis}

\begin{axis}[%
width=11.167in,
height=7.931in,
at={(0in,0in)},
scale only axis,
xmin=0,
xmax=1,
ymin=0,
ymax=1,
axis line style={draw=none},
ticks=none,
axis x line*=bottom,
axis y line*=left
]
\end{axis}
\end{tikzpicture}%
 \caption{Nei grafici vengono  rappresentati i tempi (in scala logaritmica) necessari per la risoluzione del sistema di ode  utilizzando le funzioni ode45 e ode15s.  Per ottenere dei risultati attendibili,  abbiamo ripetuto la misura del tempo $100$ volte.  I parametri utilizzati sono $\tau =0.3$ e $\gamma =0.1$}
 \label{Erdos_tempo}
\end{figure}

\begin{figure}
\centering
% This file was created by matlab2tikz.
%
%The latest updates can be retrieved from
%  http://www.mathworks.com/matlabcentral/fileexchange/22022-matlab2tikz-matlab2tikz
%where you can also make suggestions and rate matlab2tikz.
%
\definecolor{mycolor1}{rgb}{0.00000,0.44700,0.74100}%
%
\begin{tikzpicture}[scale=0.8]

\begin{axis}[%
width=2.603in,
height=1.074in,
at={(1.011in,4.246in)},
scale only axis,
xmin=0,
xmax=10,
xlabel style={font=\color{white!15!black}},
xlabel={t},
ymode=log,
ymin=9.63671723985048,
ymax=100000,
yminorticks=true,
ylabel style={font=\color{white!15!black}},
ylabel={indice stiff},
axis background/.style={fill=white},
title style={font=\bfseries},
title={N=10}
]
\addplot [color=mycolor1, forget plot]
  table[row sep=crcr]{%
0	20.5711090943093\\
0.002	67669.541839912\\
0.004	33812.8243843502\\
0.006	22527.2496385432\\
0.008	16884.460561982\\
0.01	13498.7857465837\\
0.012	11241.6680563336\\
0.014	9629.44014724444\\
0.016	8420.26834733425\\
0.018	7479.80061662416\\
0.02	6727.42573113936\\
0.022	6111.84563926826\\
0.024	5598.86163995015\\
0.026	5164.79770937891\\
0.028	4792.74240198934\\
0.03	4470.29399104201\\
0.032	4188.15118149107\\
0.034	3939.20121830863\\
0.036	3717.91195870304\\
0.038	3519.91592153078\\
0.04	3341.71912188385\\
0.042	3180.49309579645\\
0.044	3033.9236454851\\
0.046	2900.09904235288\\
0.048	2777.4261792425\\
0.05	2664.56684614784\\
0.052	2560.38871153251\\
0.054	2463.92719668592\\
0.056	2374.35551986256\\
0.058	2290.96093818486\\
0.06	2213.12574116468\\
0.062	2140.31192289882\\
0.064	2072.04872822866\\
0.066	2007.92246323203\\
0.068	1947.56810386638\\
0.07	1890.66234312809\\
0.0720000000000001	1836.91779701984\\
0.0740000000000001	1786.07815009036\\
0.0760000000000001	1737.91406746753\\
0.0780000000000001	1692.21973581079\\
0.0800000000000001	1648.80992311867\\
0.0820000000000001	1607.5174688105\\
0.0840000000000001	1568.19113236629\\
0.0860000000000001	1530.6937421573\\
0.0880000000000001	1494.90059670823\\
0.0900000000000001	1460.6980791249\\
0.0920000000000001	1427.98245224865\\
0.0940000000000001	1396.65880762054\\
0.0960000000000001	1366.64014582532\\
0.0980000000000001	1337.84656944625\\
0.1	1310.20457286543\\
0.102	1283.64641561683\\
0.104	1258.10956804485\\
0.106	1233.53621971855\\
0.108	1209.87284246665\\
0.11	1187.06980108096\\
0.112	1165.08100573082\\
0.114	1143.86360096487\\
0.116	1123.37768688557\\
0.118	1103.58606867843\\
0.12	1084.4540311887\\
0.122	1065.94913567039\\
0.124	1048.04103620528\\
0.126	1030.70131360539\\
0.128	1013.9033248873\\
0.13	997.622066641084\\
0.132	981.834050820354\\
0.134	966.517191655417\\
0.136	951.650702544805\\
0.138	937.215001912729\\
0.14	923.191627135882\\
0.142	909.563155744031\\
0.144	896.31313318724\\
0.146	883.426006540114\\
0.148	870.887063581128\\
0.15	858.682376745869\\
0.152	846.798751504989\\
0.154	835.223678764439\\
0.156	823.945290927493\\
0.158	812.9523212937\\
0.16	802.234066502893\\
0.162	791.780351761172\\
0.164	781.581498611342\\
0.166	771.628295033184\\
0.168	761.911967679351\\
0.17	752.42415607123\\
0.172	743.156888594892\\
0.174	734.102560152729\\
0.176	725.25391133862\\
0.178	716.604009017228\\
0.18	708.146228197789\\
0.182	699.874235103206\\
0.184	691.781971343289\\
0.186	683.863639109025\\
0.188	676.113687311968\\
0.19	668.52679859898\\
0.192	661.097877178575\\
0.194	653.822037400247\\
0.196	646.694593033091\\
0.198	639.711047194154\\
0.2	632.867082880994\\
0.202	626.158554066885\\
0.204	619.581477319637\\
0.206	613.132023908703\\
0.208	606.806512367671\\
0.21	600.601401481713\\
0.212	594.513283672003\\
0.214	588.538878751132\\
0.216	582.675028025285\\
0.218	576.91868872133\\
0.22	571.266928717639\\
0.222	565.71692155987\\
0.224	560.265941743733\\
0.226	554.911360248151\\
0.228	549.650640303708\\
0.23	544.48133338159\\
0.232	539.401075390286\\
0.234	534.407583067137\\
0.236	529.498650553409\\
0.238	524.672146142098\\
0.24	519.926009188331\\
0.242	515.258247172891\\
0.244	510.666932910279\\
0.246	506.150201892868\\
0.248	501.70624976365\\
0.25	497.333329910184\\
0.252	493.02975117324\\
0.254	488.793875663618\\
0.256	484.624116681301\\
0.258	480.518936731442\\
0.26	476.476845631869\\
0.262	472.496398707295\\
0.264	468.576195065632\\
0.266	464.714875951977\\
0.268	460.911123176429\\
0.27	457.163657611663\\
0.272	453.47123775687\\
0.274	449.832658364512\\
0.276	446.246749126875\\
0.278	442.712373419253\\
0.28	439.228427096987\\
0.282	435.793837343755\\
0.284	432.407561568454\\
0.286	429.068586348447\\
0.288	425.775926416734\\
0.29	422.528623691145\\
0.292	419.325746343351\\
0.294	416.166387905911\\
0.296	413.049666415507\\
0.298	409.974723590679\\
0.3	406.940724042449\\
0.302	403.946854516337\\
0.304	400.99232316424\\
0.306	398.076358844925\\
0.308	395.198210451744\\
0.31	392.357146266367\\
0.312	389.552453337393\\
0.314	386.783436882694\\
0.316	384.049419714455\\
0.318	381.349741685915\\
0.32	378.683759158855\\
0.322	376.05084449092\\
0.324	373.450385541951\\
0.326	370.881785198445\\
0.328	368.344460915493\\
0.33	365.837844275269\\
0.332	363.361380561565\\
0.334	360.914528349566\\
0.336	358.496759110298\\
0.338	356.10755682909\\
0.34	353.74641763754\\
0.342	351.41284945836\\
0.344	349.106371662639\\
0.346	346.826514738962\\
0.348	344.572819973968\\
0.35	342.344839143825\\
0.352	340.142134216277\\
0.354	337.964277062727\\
0.356	335.810849180075\\
0.358	333.681441421838\\
0.36	331.575653738287\\
0.362	329.493094925128\\
0.364	327.433382380528\\
0.366	325.396141870092\\
0.368	323.381007299456\\
0.37	321.387620494336\\
0.372	319.415630987616\\
0.374	317.464695813272\\
0.376	315.534479306901\\
0.378	313.624652912531\\
0.38	311.734894995604\\
0.382	309.864890661763\\
0.384	308.014331581342\\
0.386	306.182915819297\\
0.388	304.370347670414\\
0.39	302.576337499561\\
0.392	300.80060158684\\
0.394	299.042861977473\\
0.396	297.302846336205\\
0.398	295.580287806131\\
0.4	293.874924871723\\
0.402	292.186501225973\\
0.404	290.514765641465\\
0.406	288.859471845268\\
0.408	287.22037839751\\
0.41	285.597248573476\\
0.412	283.989850249189\\
0.414	282.397955790274\\
0.416	280.821341944033\\
0.418	279.259789734614\\
0.42	277.713084361189\\
0.422	276.181015099018\\
0.424	274.663375203317\\
0.426	273.159961815821\\
0.428	271.67057587398\\
0.43	270.195022022689\\
0.432	268.733108528465\\
0.434	267.28464719599\\
0.436	265.849453286955\\
0.438	264.427345441153\\
0.44	263.018145599671\\
0.442	261.621678930215\\
0.444	260.237773754426\\
0.446	258.86626147717\\
0.448	257.506976517695\\
0.45	256.159756242661\\
0.452	254.824440900907\\
0.454	253.500873559973\\
0.456	252.188900044284\\
0.458	250.888368874933\\
0.46	249.599131211072\\
0.462	248.321040792807\\
0.464	247.053953885527\\
0.466	245.797729225775\\
0.468	244.552227968374\\
0.47	243.317313634997\\
0.472	242.092852064026\\
0.474	240.878711361643\\
0.476	239.674761854191\\
0.478	238.480876041726\\
0.48	237.29692855271\\
0.482	236.122796099852\\
0.484	234.958357437037\\
0.486	233.803493317316\\
0.488	232.65808645193\\
0.49	231.522021470338\\
0.492	230.39518488122\\
0.494	229.277465034427\\
0.496	228.168752083852\\
0.498	227.06893795119\\
0.5	225.97791629057\\
0.502	224.895582454048\\
0.504	223.821833457888\\
0.506	222.756567949681\\
0.508	221.699686176202\\
0.51	220.651089952061\\
0.512	219.610682629053\\
0.514	218.578369066257\\
0.516	217.5540556008\\
0.518	216.537650019315\\
0.52	215.529061530059\\
0.522	214.528200735661\\
0.524	213.534979606508\\
0.526	212.549311454716\\
0.528	211.571110908716\\
0.53	210.600293888406\\
0.532	209.636777580855\\
0.534	208.680480416566\\
0.536	207.731322046266\\
0.538	206.789223318205\\
0.54	205.854106255976\\
0.542	204.925894036801\\
0.544	204.004510970336\\
0.546	203.089882477887\\
0.548	202.181935072134\\
0.55	201.28059633726\\
0.552	200.385794909534\\
0.554	199.497460458301\\
0.556	198.61552366739\\
0.558	197.739916216918\\
0.56	196.870570765481\\
0.562	196.007420932735\\
0.564	195.150401282343\\
0.566	194.299447305269\\
0.568	193.454495403446\\
0.57	192.615482873784\\
0.572	191.782347892499\\
0.574	190.955029499781\\
0.576	190.133467584786\\
0.578	189.317602870923\\
0.58	188.507376901459\\
0.582	187.70273202542\\
0.584	186.903611383774\\
0.586	186.109958895898\\
0.588	185.321719246334\\
0.59	184.538837871794\\
0.592	183.761260948443\\
0.594	182.988935379433\\
0.596	182.221808782696\\
0.598	181.459829478958\\
0.6	180.702946480029\\
0.602	179.951109477298\\
0.604	179.204268830466\\
0.606	178.462375556494\\
0.608	177.725381318791\\
0.61	176.993238416594\\
0.612	176.265899774555\\
0.614	175.543318932548\\
0.616	174.825450035658\\
0.618	174.112247824381\\
0.62	173.40366762499\\
0.622	172.699665340114\\
0.624	172.000197439475\\
0.626	171.305220950817\\
0.628	170.614693451004\\
0.63	169.928573057285\\
0.632	169.246818418726\\
0.634	168.569388707805\\
0.636	167.89624361217\\
0.638	167.227343326539\\
0.64	166.562648544769\\
0.642	165.902120452056\\
0.644	165.245720717291\\
0.646	164.593411485562\\
0.648	163.945155370778\\
0.65	163.300915448445\\
0.652	162.660655248571\\
0.654	162.024338748691\\
0.656	161.391930367035\\
0.658	160.763394955818\\
0.66	160.138697794631\\
0.662	159.517804583984\\
0.664	158.900681438943\\
0.666	158.287294882887\\
0.668	157.677611841382\\
0.67	157.071599636158\\
0.672	156.469225979201\\
0.674	155.870458966935\\
0.676	155.275267074537\\
0.678	154.683619150311\\
0.68	154.095484410201\\
0.682	153.510832432376\\
0.684	152.929633151917\\
0.686	152.351856855599\\
0.688000000000001	151.777474176762\\
0.690000000000001	151.20645609027\\
0.692000000000001	150.638773907562\\
0.694000000000001	150.074399271776\\
0.696000000000001	149.513304152979\\
0.698000000000001	148.95546084346\\
0.700000000000001	148.400841953104\\
0.702000000000001	147.849420404866\\
0.704000000000001	147.301169430289\\
0.706000000000001	146.756062565129\\
0.708000000000001	146.214073645038\\
0.710000000000001	145.675176801322\\
0.712000000000001	145.139346456775\\
0.714000000000001	144.606557321572\\
0.716000000000001	144.076784389248\\
0.718000000000001	143.550002932731\\
0.720000000000001	143.026188500441\\
0.722000000000001	142.505316912464\\
0.724000000000001	141.987364256783\\
0.726000000000001	141.472306885564\\
0.728000000000001	140.960121411519\\
0.730000000000001	140.450784704314\\
0.732000000000001	139.944273887049\\
0.734000000000001	139.440566332779\\
0.736000000000001	138.939639661118\\
0.738000000000001	138.441471734861\\
0.740000000000001	137.946040656697\\
0.742000000000001	137.453324765954\\
0.744000000000001	136.963302635404\\
0.746000000000001	136.475953068113\\
0.748000000000001	135.991255094356\\
0.750000000000001	135.509187968561\\
0.752000000000001	135.029731166317\\
0.754000000000001	134.552864381426\\
0.756000000000001	134.078567522998\\
0.758000000000001	133.606820712597\\
0.760000000000001	133.13760428143\\
0.762000000000001	132.670898767577\\
0.764000000000001	132.20668491327\\
0.766000000000001	131.744943662214\\
0.768000000000001	131.285656156946\\
0.770000000000001	130.828803736233\\
0.772000000000001	130.374367932527\\
0.774000000000001	129.922330469434\\
0.776000000000001	129.472673259241\\
0.778000000000001	129.02537840048\\
0.780000000000001	128.580428175516\\
0.782000000000001	128.13780504819\\
0.784000000000001	127.69749166148\\
0.786000000000001	127.259470835219\\
0.788000000000001	126.823725563825\\
0.790000000000001	126.390239014076\\
0.792000000000001	125.958994522934\\
0.794000000000001	125.529975595365\\
0.796000000000001	125.103165902234\\
0.798000000000001	124.678549278194\\
0.800000000000001	124.256109719641\\
0.802000000000001	123.83583138267\\
0.804000000000001	123.417698581076\\
0.806000000000001	123.001695784395\\
0.808000000000001	122.587807615943\\
0.810000000000001	122.176018850923\\
0.812000000000001	121.766314414525\\
0.814000000000001	121.358679380078\\
0.816000000000001	120.953098967219\\
0.818000000000001	120.549558540089\\
0.820000000000001	120.148043605558\\
0.822000000000001	119.748539811474\\
0.824000000000001	119.35103294494\\
0.826000000000001	118.955508930617\\
0.828000000000001	118.56195382904\\
0.830000000000001	118.17035383498\\
0.832000000000001	117.780695275808\\
0.834000000000001	117.392964609895\\
0.836000000000001	117.007148425031\\
0.838000000000001	116.623233436871\\
0.840000000000001	116.241206487395\\
0.842000000000001	115.861054543398\\
0.844000000000001	115.482764694997\\
0.846000000000001	115.106324154159\\
0.848000000000001	114.731720253257\\
0.850000000000001	114.358940443634\\
0.852000000000001	113.987972294198\\
0.854000000000001	113.618803490032\\
0.856000000000001	113.251421831027\\
0.858000000000001	112.885815230524\\
0.860000000000001	112.521971713992\\
0.862000000000001	112.159879417709\\
0.864000000000001	111.799526587472\\
0.866000000000001	111.440901577313\\
0.868000000000001	111.083992848249\\
0.870000000000001	110.728788967034\\
0.872000000000001	110.375278604939\\
0.874000000000001	110.02345053654\\
0.876000000000001	109.673293638534\\
0.878000000000001	109.324796888554\\
0.880000000000001	108.977949364022\\
0.882000000000001	108.632740240999\\
0.884000000000001	108.289158793058\\
0.886000000000001	107.947194390179\\
0.888000000000001	107.606836497643\\
0.890000000000001	107.268074674958\\
0.892000000000001	106.930898574785\\
0.894000000000001	106.595297941892\\
0.896000000000001	106.261262612108\\
0.898000000000001	105.928782511307\\
0.900000000000001	105.597847654389\\
0.902000000000001	105.268448144285\\
0.904000000000001	104.940574170976\\
0.906000000000001	104.614216010517\\
0.908000000000001	104.289364024082\\
0.910000000000001	103.966008657017\\
0.912000000000001	103.644140437909\\
0.914000000000001	103.323749977662\\
0.916000000000001	103.00482796859\\
0.918000000000001	102.687365183522\\
0.920000000000001	102.371352474915\\
0.922000000000001	102.056780773979\\
0.924000000000001	101.743641089822\\
0.926000000000001	101.431924508589\\
0.928000000000001	101.121622192633\\
0.930000000000001	100.812725379676\\
0.932000000000001	100.505225382002\\
0.934000000000001	100.19911358564\\
0.936000000000001	99.8943814495704\\
0.938000000000001	99.5910205049422\\
0.940000000000001	99.2890223542868\\
0.942000000000001	98.9883786707633\\
0.944000000000001	98.6890811973922\\
0.946000000000001	98.3911217463105\\
0.948000000000001	98.0944921980379\\
0.950000000000001	97.799184500744\\
0.952000000000001	97.5051906695299\\
0.954000000000001	97.2125027857207\\
0.956000000000001	96.9211129961603\\
0.958000000000001	96.6310135125255\\
0.960000000000001	96.3421966106346\\
0.962000000000001	96.05465462978\\
0.964000000000001	95.7683799720562\\
0.966000000000001	95.4833651017063\\
0.968000000000001	95.1996025444665\\
0.970000000000001	94.9170848869315\\
0.972000000000001	94.6358047759149\\
0.974000000000001	94.3557549178261\\
0.976000000000001	94.0769280780513\\
0.978000000000001	93.7993170803419\\
0.980000000000001	93.5229148062155\\
0.982000000000001	93.2477141943556\\
0.984000000000001	92.9737082400264\\
0.986000000000001	92.700889994491\\
0.988000000000001	92.4292525644374\\
0.990000000000001	92.1587891114106\\
0.992000000000001	91.8894928512568\\
0.994000000000001	91.6213570535647\\
0.996000000000001	91.3543750411224\\
0.998000000000001	91.0885401893775\\
1	90.8238459259046\\
1.002	90.5602857298743\\
1.004	90.2978531315394\\
1.006	90.0365417117148\\
1.008	89.7763451012734\\
1.01	89.5172569806427\\
1.012	89.2592710793088\\
1.014	89.0023811753271\\
1.016	88.7465810948385\\
1.018	88.4918647115893\\
1.02	88.2382259464612\\
1.022	87.9856587670032\\
1.024	87.7341571869692\\
1.026	87.4837152658619\\
1.028	87.2343271084852\\
1.03	86.9859868644951\\
1.032	86.7386887279627\\
1.034	86.4924269369357\\
1.036	86.2471957730121\\
1.038	86.0029895609149\\
1.04	85.7598026680696\\
1.042	85.5176295041917\\
1.044	85.2764645208769\\
1.046	85.0363022111938\\
1.048	84.7971371092857\\
1.05	84.5589637899701\\
1.052	84.321776868354\\
1.054	84.0855709994421\\
1.056	83.850340877755\\
1.058	83.6160812369531\\
1.06	83.3827868494592\\
1.062	83.1504525260934\\
1.064	82.9190731157059\\
1.066	82.6886435048151\\
1.068	82.4591586172514\\
1.07	82.2306134138054\\
1.072	82.0030028918771\\
1.074	81.7763220851327\\
1.076	81.5505660631619\\
1.078	81.325729931142\\
1.08	81.1018088295045\\
1.082	80.8787979336044\\
1.084	80.6566924533965\\
1.086	80.4354876331096\\
1.088	80.2151787509321\\
1.09	79.9957611186926\\
1.092	79.7772300815534\\
1.094	79.5595810176962\\
1.096	79.3428093380224\\
1.098	79.1269104858511\\
1.1	78.9118799366179\\
1.102	78.6977131975847\\
1.104	78.4844058075441\\
1.106	78.2719533365358\\
1.108	78.0603513855566\\
1.11	77.8495955862825\\
1.112	77.6396816007884\\
1.114	77.4306051212714\\
1.116	77.2223618697802\\
1.118	77.0149475979431\\
1.12	76.8083580867033\\
1.122	76.602589146053\\
1.124	76.3976366147755\\
1.126	76.1934963601828\\
1.128	75.9901642778629\\
1.13	75.7876362914271\\
1.132	75.585908352259\\
1.134	75.3849764392696\\
1.136	75.1848365586477\\
1.138	74.9854847436248\\
1.14	74.7869170542308\\
1.142	74.5891295770569\\
1.144	74.3921184250265\\
1.146	74.1958797371589\\
1.148	74.0004096783397\\
1.15	73.8057044390971\\
1.152	73.6117602353762\\
1.154	73.4185733083156\\
1.156	73.2261399240283\\
1.158	73.0344563733865\\
1.16	72.8435189718042\\
1.162	72.6533240590246\\
1.164	72.4638679989091\\
1.166	72.2751471792323\\
1.168	72.08715801147\\
1.17	71.8998969306013\\
1.172	71.7133603949029\\
1.174	71.5275448857498\\
1.176	71.3424469074176\\
1.178	71.1580629868895\\
1.18	70.9743896736588\\
1.182	70.7914235395396\\
1.184	70.6091611784759\\
1.186	70.4275992063558\\
1.188	70.2467342608246\\
1.19	70.0665630010997\\
1.192	69.8870821077911\\
1.194	69.7082882827204\\
1.196	69.5301782487414\\
1.198	69.3527487495662\\
1.2	69.1759965495883\\
1.202	68.9999184337107\\
1.204	68.8245112071752\\
1.206	68.649771695393\\
1.208	68.4756967437761\\
1.21	68.3022832175729\\
1.212	68.1295280017035\\
1.214	67.9574280005969\\
1.216	67.7859801380302\\
1.218	67.6151813569684\\
1.22	67.4450286194092\\
1.222	67.2755189062251\\
1.224	67.106649217009\\
1.226	66.9384165699207\\
1.228	66.7708180015377\\
1.23	66.6038505667032\\
1.232	66.4375113383777\\
1.234	66.2717974074932\\
1.236	66.106705882807\\
1.238	65.9422338907576\\
1.24	65.7783785753231\\
1.242	65.6151370978787\\
1.244	65.4525066370569\\
1.246	65.2904843886101\\
1.248	65.1290675652727\\
1.25	64.9682533966261\\
1.252	64.8080391289638\\
1.254	64.648422025158\\
1.256	64.4893993645288\\
1.258	64.3309684427127\\
1.26	64.1731265715345\\
1.262	64.0158710788786\\
1.264	63.8591993085617\\
1.266	63.7031086202091\\
1.268	63.547596389129\\
1.27	63.3926600061907\\
1.272	63.2382968777007\\
1.274	63.0845044252845\\
1.276	62.9312800857647\\
1.278	62.7786213110449\\
1.28	62.6265255679913\\
1.282	62.4749903383162\\
1.284	62.324013118464\\
1.286	62.1735914194966\\
1.288	62.0237227669814\\
1.29	61.874404700878\\
1.292	61.7256347754309\\
1.294	61.5774105590544\\
1.296	61.4297296342296\\
1.298	61.2825895973943\\
1.3	61.1359880588359\\
1.302	60.9899226425881\\
1.304	60.8443909863236\\
1.306	60.6993907412527\\
1.308	60.55491957202\\
1.31	60.4109751566033\\
1.312	60.2675551862124\\
1.314	60.1246573651889\\
1.316	59.9822794109096\\
1.318	59.8404190536862\\
1.32	59.6990740366699\\
1.322	59.5582421157544\\
1.324	59.4179210594823\\
1.326	59.2781086489487\\
1.328	59.1388026777103\\
1.33	59.0000009516907\\
1.332	58.8617012890902\\
1.334	58.7239015202945\\
1.336	58.5865994877843\\
1.338	58.4497930460484\\
1.34	58.3134800614903\\
1.342	58.1776584123478\\
1.344	58.0423259886001\\
1.346	57.9074806918856\\
1.348	57.7731204354153\\
1.35	57.6392431438885\\
1.352	57.5058467534095\\
1.354	57.3729292114051\\
1.356	57.2404884765423\\
1.358	57.108522518646\\
1.36	56.9770293186211\\
1.362	56.8460068683689\\
1.364	56.7154531707113\\
1.366	56.585366239311\\
1.368	56.4557440985933\\
1.37	56.3265847836702\\
1.372	56.1978863402637\\
1.374	56.0696468246293\\
1.376	55.9418643034818\\
1.378	55.8145368539228\\
1.38	55.6876625633631\\
1.382	55.5612395294533\\
1.384	55.435265860009\\
1.386	55.3097396729422\\
1.388	55.1846590961858\\
1.39	55.0600222676278\\
1.392	54.9358273350384\\
1.394	54.8120724560025\\
1.396	54.6887557978504\\
1.398	54.5658755375892\\
1.4	54.4434298618375\\
1.402	54.321416966757\\
1.404	54.1998350579858\\
1.406	54.0786823505749\\
1.408	53.9579570689218\\
1.41	53.8376574467055\\
1.412	53.7177817268241\\
1.414	53.5983281613308\\
1.416	53.4792950113701\\
1.418	53.3606805471173\\
1.42	53.2424830477161\\
1.422	53.1247008012166\\
1.424	53.0073321045166\\
1.426	52.8903752632988\\
1.428	52.7738285919734\\
1.43	52.6576904136184\\
1.432	52.541959059921\\
1.434	52.4266328711188\\
1.436	52.3117101959438\\
1.438	52.1971893915634\\
1.44	52.0830688235257\\
1.442	51.9693468657014\\
1.444	51.8560219002299\\
1.446	51.7430923174637\\
1.448	51.6305565159131\\
1.45	51.5184129021924\\
1.452	51.4066598909655\\
1.454	51.2952959048941\\
1.456	51.1843193745831\\
1.458	51.0737287385295\\
1.46	50.9635224430693\\
1.462	50.8536989423266\\
1.464	50.744256698163\\
1.466	50.6351941801258\\
1.468	50.5265098653975\\
1.47	50.4182022387482\\
1.472	50.3102697924835\\
1.474	50.202711026397\\
1.476	50.0955244477218\\
1.478	49.9887085710808\\
1.48	49.8822619184403\\
1.482	49.776183019063\\
1.484	49.6704704094592\\
1.486	49.565122633341\\
1.488	49.4601382415775\\
1.49	49.3555157921463\\
1.492	49.2512538500893\\
1.494	49.1473509874686\\
1.496	49.0438057833204\\
1.498	48.9406168236102\\
1.5	48.8377827011906\\
1.502	48.7353020157557\\
1.504	48.6331733737995\\
1.506	48.5313953885711\\
1.508	48.4299666800346\\
1.51	48.3288858748235\\
1.512	48.2281516062024\\
1.514	48.1277625140217\\
1.516	48.0277172446806\\
1.518	47.9280144510816\\
1.52	47.8286527925938\\
1.522	47.7296309350101\\
1.524	47.6309475505086\\
1.526	47.5326013176128\\
1.528	47.4345909211511\\
1.53	47.3369150522195\\
1.532	47.2395724081424\\
1.534	47.1425616924338\\
1.536	47.0458816147584\\
1.538	46.9495308908965\\
1.54	46.8535082427041\\
1.542	46.7578123980765\\
1.544	46.6624420909114\\
1.546	46.567396061073\\
1.548	46.4726730543543\\
1.55	46.3782718224424\\
1.552	46.2841911228817\\
1.554	46.1904297190396\\
1.556	46.09698638007\\
1.558	46.00385988088\\
1.56	45.9110490020936\\
1.562	45.8185525300182\\
1.564	45.7263692566107\\
1.566	45.6344979794433\\
1.568	45.54293750167\\
1.57	45.4516866319931\\
1.572	45.360744184631\\
1.574	45.2701089792839\\
1.576	45.179779841103\\
1.578	45.0897556006568\\
1.58	45.0000350938993\\
1.582	44.9106171621397\\
1.584	44.8215006520082\\
1.586	44.7326844154268\\
1.588	44.6441673095779\\
1.59	44.555948196873\\
1.592	44.468025944921\\
1.594	44.3803994265013\\
1.596	44.2930675195289\\
1.598	44.2060291070288\\
1.6	44.119283077104\\
1.602	44.0328283229061\\
1.604	43.9466637426068\\
1.606	43.8607882393689\\
1.608	43.7752007213166\\
1.61	43.6899001015079\\
1.612	43.6048852979049\\
1.614	43.5201552333481\\
1.616	43.4357088355257\\
1.618	43.3515450369473\\
1.62	43.2676627749166\\
1.622	43.1840609915024\\
1.624	43.1007386335141\\
1.626	43.0176946524719\\
1.628	42.9349280045827\\
1.63	42.8524376507116\\
1.632	42.7702225563569\\
1.634	42.6882816916225\\
1.636	42.6066140311934\\
1.638	42.5252185543096\\
1.64	42.4440942447394\\
1.642	42.363240090756\\
1.644	42.2826550851097\\
1.646	42.2023382250065\\
1.648	42.1222885120789\\
1.65	42.0425049523653\\
1.652	41.9629865562822\\
1.654	41.8837323386026\\
1.656	41.8047413184298\\
1.658	41.7260125191753\\
1.66	41.6475449685329\\
1.662	41.569337698457\\
1.664	41.4913897451387\\
1.666	41.4137001489812\\
1.668	41.3362679545783\\
1.67	41.2590922106918\\
1.672	41.1821719702266\\
1.674	41.1055062902088\\
1.676	41.0290942317663\\
1.678	40.9529348601013\\
1.68	40.8770272444722\\
1.682	40.8013704581704\\
1.684	40.7259635784972\\
1.686	40.6508056867454\\
1.688	40.5758958681731\\
1.69	40.5012332119871\\
1.692	40.4268168113183\\
1.694	40.3526457632022\\
1.696	40.2787191685575\\
1.698	40.2050361321657\\
1.7	40.1315957626505\\
1.702	40.0583971724558\\
1.704	39.9854394778286\\
1.706	39.9127217987952\\
1.708	39.8402432591433\\
1.71	39.7680029864018\\
1.712	39.6960001118203\\
1.714	39.6242337703506\\
1.716	39.5527031006259\\
1.718	39.4814072449422\\
1.72	39.410345349239\\
1.722	39.3395165630805\\
1.724	39.2689200396358\\
1.726	39.1985549356597\\
1.728	39.1284204114766\\
1.73	39.0585156309586\\
1.732	38.9888397615091\\
1.734	38.9193919740437\\
1.736	38.8501714429719\\
1.738	38.78117734618\\
1.74	38.7124088650122\\
1.742	38.6438651842515\\
1.744	38.575545492106\\
1.746	38.5074489801865\\
1.748	38.4395748434924\\
1.75	38.3719222803924\\
1.752	38.3044904926086\\
1.754	38.2372786851985\\
1.756	38.1702860665384\\
1.758	38.1035118483062\\
1.76	38.0369552454642\\
1.762	37.970615476244\\
1.764	37.904491762128\\
1.766	37.8385833278336\\
1.768	37.7728894012982\\
1.77	37.7074092136604\\
1.772	37.642141999246\\
1.774	37.5770869955511\\
1.776	37.5122434432257\\
1.778	37.4476105860595\\
1.78	37.383187670964\\
1.782	37.3189739479591\\
1.784	37.2549686701546\\
1.786	37.1911710937388\\
1.788	37.1275804779599\\
1.79	37.064196085112\\
1.792	37.0010171805197\\
1.794	36.9380430325236\\
1.796	36.875272912465\\
1.798	36.812706094671\\
1.8	36.75034185644\\
1.802	36.6881794780268\\
1.804	36.626218242628\\
1.806	36.5644574363684\\
1.808	36.5028963482853\\
1.81	36.4415342703154\\
1.812	36.3803704972793\\
1.814	36.3194043268695\\
1.816	36.2586350596344\\
1.818	36.1980619989652\\
1.82	36.1376844510818\\
1.822	36.0775017250204\\
1.824	36.0175131326172\\
1.826	35.9577179884979\\
1.828	35.8981156100615\\
1.83	35.8387053174691\\
1.832	35.7794864336287\\
1.834	35.7204582841841\\
1.836	35.6616201974992\\
1.838	35.6029715046476\\
1.84	35.5445115393976\\
1.842	35.4862396382008\\
1.844	35.4281551401774\\
1.846	35.3702573871053\\
1.848	35.3125457234073\\
1.85	35.2550194961369\\
1.852	35.1976780549678\\
1.854	35.1405207521798\\
1.856	35.0835469426484\\
1.858	35.02675598383\\
1.86	34.970147235752\\
1.862	34.9137200609992\\
1.864	34.8574738247027\\
1.866	34.8014078945278\\
1.868	34.7455216406612\\
1.87	34.6898144357999\\
1.872	34.6342856551399\\
1.874	34.5789346763634\\
1.876	34.5237608796288\\
1.878	34.4687636475565\\
1.88	34.4139423652206\\
1.882	34.3592964201354\\
1.884	34.3048252022447\\
1.886	34.2505281039105\\
1.888	34.1964045199012\\
1.89	34.1424538473821\\
1.892	34.088675485903\\
1.894	34.0350688373871\\
1.896	33.981633306121\\
1.898	33.9283682987425\\
1.9	33.8752732242322\\
1.902	33.8223474938991\\
1.904	33.7695905213739\\
1.906	33.7170017225952\\
1.908	33.6645805158014\\
1.91	33.6123263215181\\
1.912	33.560238562549\\
1.914	33.5083166639657\\
1.916	33.4565600530962\\
1.918	33.4049681595156\\
1.92	33.353540415036\\
1.922	33.3022762536958\\
1.924	33.2511751117502\\
1.926	33.2002364276603\\
1.928	33.1494596420844\\
1.93	33.0988441978672\\
1.932	33.0483895400308\\
1.934	32.9980951157636\\
1.936	32.9479603744121\\
1.938	32.89798476747\\
1.94	32.8481677485696\\
1.942	32.7985087734713\\
1.944	32.7490073000555\\
1.946	32.6996627883111\\
1.948	32.650474700328\\
1.95	32.6014425002874\\
1.952	32.5525656544514\\
1.954	32.5038436311551\\
1.956	32.4552759007966\\
1.958	32.4068619358281\\
1.96	32.358601210748\\
1.962	32.3104932020897\\
1.964	32.2625373884135\\
1.966	32.2147332502994\\
1.968	32.1670802703357\\
1.97	32.1195779331114\\
1.972	32.0722257252073\\
1.974	32.0250231351881\\
1.976	31.9779696535918\\
1.978	31.931064772923\\
1.98	31.8843079876437\\
1.982	31.8376987941641\\
1.984	31.7912366908348\\
1.986	31.7449211779387\\
1.988	31.6987517576822\\
1.99	31.6527279341862\\
1.992	31.6068492134794\\
1.994	31.5611151034887\\
1.996	31.5155251140313\\
1.998	31.4700787568074\\
2	31.4247755453904\\
2.002	31.3796149952213\\
2.004	31.334596623598\\
2.006	31.2897199496692\\
2.008	31.244984494426\\
2.01	31.2003897806938\\
2.012	31.155935333124\\
2.014	31.1116206781881\\
2.016	31.0674453441675\\
2.018	31.0234088611473\\
2.02	30.9795107610085\\
2.022	30.9357505774199\\
2.024	30.8921278458307\\
2.026	30.8486421034633\\
2.028	30.8052928893055\\
2.03	30.7620797441031\\
2.032	30.7190022103523\\
2.034	30.6760598322924\\
2.036	30.6332521558988\\
2.038	30.5905787288758\\
2.04	30.5480391006481\\
2.042	30.5056328223548\\
2.044	30.4633594468426\\
2.046	30.4212185286571\\
2.048	30.3792096240368\\
2.05	30.3373322909059\\
2.052	30.2955860888675\\
2.054	30.253970579195\\
2.056	30.212485324828\\
2.05799999999999	30.1711298903626\\
2.05999999999999	30.129903842046\\
2.06199999999999	30.0888067477704\\
2.06399999999999	30.0478381770635\\
2.06599999999999	30.0069977010844\\
2.06799999999999	29.9662848926158\\
2.06999999999999	29.9256993260566\\
2.07199999999999	29.8852405774175\\
2.07399999999999	29.8449082243113\\
2.07599999999999	29.8047018459488\\
2.07799999999999	29.7646210231306\\
2.07999999999999	29.7246653382421\\
2.08199999999999	29.6848343752453\\
2.08399999999999	29.6451277196744\\
2.08599999999999	29.6055449586269\\
2.08799999999999	29.5660856807594\\
2.08999999999999	29.5267494762803\\
2.09199999999999	29.4875359369426\\
2.09399999999999	29.4484446560398\\
2.09599999999999	29.4094752283976\\
2.09799999999999	29.3706272503687\\
2.09999999999999	29.3319003198262\\
2.10199999999999	29.2932940361579\\
2.10399999999999	29.2548080002596\\
2.10599999999999	29.2164418145297\\
2.10799999999999	29.1781950828613\\
2.10999999999999	29.1400674106398\\
2.11199999999999	29.1020584047325\\
2.11399999999999	29.064167673486\\
2.11599999999999	29.0263948267191\\
2.11799999999999	28.9887394757162\\
2.11999999999999	28.9512012332221\\
2.12199999999999	28.9137797134365\\
2.12399999999999	28.8764745320073\\
2.12599999999999	28.8392853060258\\
2.12799999999999	28.8022116540196\\
2.12999999999999	28.7652531959481\\
2.13199999999999	28.7284095531967\\
2.13399999999999	28.6916803485705\\
2.13599999999999	28.6550652062881\\
2.13799999999999	28.6185637519779\\
2.13999999999999	28.5821756126712\\
2.14199999999999	28.5459004167964\\
2.14399999999999	28.5097377941741\\
2.14599999999999	28.4736873760112\\
2.14799999999999	28.4377487948963\\
2.14999999999998	28.4019216847927\\
2.15199999999998	28.366205681034\\
2.15399999999998	28.3306004203191\\
2.15599999999998	28.2951055407051\\
2.15799999999998	28.2597206816045\\
2.15999999999998	28.2244454837771\\
2.16199999999998	28.1892795893272\\
2.16399999999998	28.1542226416965\\
2.16599999999998	28.1192742856595\\
2.16799999999998	28.0844341673191\\
2.16999999999998	28.0497019340998\\
2.17199999999998	28.0150772347436\\
2.17399999999998	27.9805597193052\\
2.17599999999998	27.9461490391456\\
2.17799999999998	27.9118448469281\\
2.17999999999998	27.8776467966127\\
2.18199999999998	27.8435545434512\\
2.18399999999998	27.8095677439825\\
2.18599999999998	27.7756860560269\\
2.18799999999998	27.7419091386822\\
2.18999999999998	27.7082366523169\\
2.19199999999998	27.6746682585681\\
2.19399999999998	27.6412036203334\\
2.19599999999998	27.6078424017683\\
2.19799999999998	27.5745842682811\\
2.19999999999998	27.5414288865271\\
2.20199999999998	27.508375924404\\
2.20399999999998	27.4754250510483\\
2.20599999999998	27.4425759368291\\
2.20799999999998	27.4098282533438\\
2.20999999999998	27.3771816734142\\
2.21199999999998	27.3446358710805\\
2.21399999999998	27.3121905215974\\
2.21599999999998	27.2798453014299\\
2.21799999999998	27.247599888247\\
2.21999999999998	27.2154539609193\\
2.22199999999998	27.1834071995126\\
2.22399999999998	27.1514592852845\\
2.22599999999998	27.1196099006791\\
2.22799999999998	27.0878587293229\\
2.22999999999998	27.0562054560202\\
2.23199999999998	27.0246497667486\\
2.23399999999998	26.9931913486548\\
2.23599999999998	26.9618298900496\\
2.23799999999998	26.9305650804038\\
2.23999999999997	26.8993966103445\\
2.24199999999997	26.868324171649\\
2.24399999999997	26.8373474572423\\
2.24599999999997	26.8064661611917\\
2.24799999999997	26.7756799787029\\
2.24999999999997	26.7449886061149\\
2.25199999999997	26.7143917408976\\
2.25399999999997	26.6838890816454\\
2.25599999999997	26.6534803280739\\
2.25799999999997	26.6231651810162\\
2.25999999999997	26.5929433424179\\
2.26199999999997	26.5628145153331\\
2.26399999999997	26.5327784039203\\
2.26599999999997	26.5028347134388\\
2.26799999999997	26.4729831502437\\
2.26999999999997	26.4432234217818\\
2.27199999999997	26.4135552365888\\
2.27399999999997	26.3839783042834\\
2.27599999999997	26.3544923355649\\
2.27799999999997	26.3250970422079\\
2.27999999999997	26.2957921370593\\
2.28199999999997	26.2665773340332\\
2.28399999999997	26.2374523481081\\
2.28599999999997	26.2084168953224\\
2.28799999999997	26.1794706927699\\
2.28999999999997	26.150613458597\\
2.29199999999997	26.121844911998\\
2.29399999999997	26.0931647732109\\
2.29599999999997	26.064572763515\\
2.29799999999997	26.0360686052253\\
2.29999999999997	26.0076520216896\\
2.30199999999997	25.9793227372846\\
2.30399999999997	25.9510804774122\\
2.30599999999997	25.9229249684948\\
2.30799999999997	25.8948559379731\\
2.30999999999997	25.8668731143006\\
2.31199999999997	25.8389762269412\\
2.31399999999997	25.8111650063648\\
2.31599999999997	25.7834391840439\\
2.31799999999997	25.7557984924496\\
2.31999999999997	25.7282426650477\\
2.32199999999997	25.7007714362961\\
2.32399999999997	25.67338454164\\
2.32599999999997	25.6460817175086\\
2.32799999999997	25.6188627013121\\
2.32999999999996	25.5917272314366\\
2.33199999999996	25.5646750472423\\
2.33399999999996	25.5377058890587\\
2.33599999999996	25.5108194981814\\
2.33799999999996	25.4840156168687\\
2.33999999999996	25.4572939883377\\
2.34199999999996	25.4306543567608\\
2.34399999999996	25.4040964672627\\
2.34599999999996	25.3776200659168\\
2.34799999999996	25.3512248997406\\
2.34999999999996	25.3249107166936\\
2.35199999999996	25.2986772656736\\
2.35399999999996	25.2725242965125\\
2.35599999999996	25.2464515599736\\
2.35799999999996	25.2204588077479\\
2.35999999999996	25.1945457924503\\
2.36199999999996	25.1687122676175\\
2.36399999999996	25.142957987703\\
2.36599999999996	25.1172827080746\\
2.36799999999996	25.0916861850113\\
2.36999999999996	25.0661681756988\\
2.37199999999996	25.0407284382281\\
2.37399999999996	25.0153667315896\\
2.37599999999996	24.9900828156723\\
2.37799999999996	24.9648764512587\\
2.37999999999996	24.9397474000228\\
2.38199999999996	24.9146954245255\\
2.38399999999996	24.889720288213\\
2.38599999999996	24.8648217554118\\
2.38799999999996	24.8399995913264\\
2.38999999999996	24.8152535620363\\
2.39199999999996	24.7905834344921\\
2.39399999999996	24.7659889765127\\
2.39599999999996	24.7414699567823\\
2.39799999999996	24.7170261448459\\
2.39999999999996	24.6926573111088\\
2.40199999999996	24.6683632268299\\
2.40399999999996	24.6441436641218\\
2.40599999999996	24.6199983959456\\
2.40799999999996	24.5959271961086\\
2.40999999999996	24.5719298392611\\
2.41199999999996	24.5480061008928\\
2.41399999999996	24.5241557573306\\
2.41599999999996	24.5003785857342\\
2.41799999999996	24.4766743640942\\
2.41999999999996	24.453042871229\\
2.42199999999995	24.4294838867804\\
2.42399999999995	24.4059971912121\\
2.42599999999995	24.3825825658055\\
2.42799999999995	24.3592397926578\\
2.42999999999995	24.3359686546774\\
2.43199999999995	24.3127689355824\\
2.43399999999995	24.2896404198969\\
2.43599999999995	24.2665828929474\\
2.43799999999995	24.2435961408612\\
2.43999999999995	24.220679950562\\
2.44199999999995	24.197834109768\\
2.44399999999995	24.1750584069879\\
2.44599999999995	24.1523526315188\\
2.44799999999995	24.1297165734425\\
2.44999999999995	24.1071500236236\\
2.45199999999995	24.0846527737051\\
2.45399999999995	24.0622246161064\\
2.45599999999995	24.0398653440202\\
2.45799999999995	24.0175747514099\\
2.45999999999995	23.9953526330053\\
2.46199999999995	23.9731987843021\\
2.46399999999995	23.9511130015557\\
2.46599999999995	23.9290950817815\\
2.46799999999995	23.9071448227504\\
2.46999999999995	23.8852620229855\\
2.47199999999995	23.8634464817603\\
2.47399999999995	23.8416979990954\\
2.47599999999995	23.8200163757548\\
2.47799999999995	23.7984014132449\\
2.47999999999995	23.7768529138093\\
2.48199999999995	23.7553706804278\\
2.48399999999995	23.7339545168124\\
2.48599999999995	23.7126042274053\\
2.48799999999995	23.6913196173754\\
2.48999999999995	23.6701004926153\\
2.49199999999995	23.6489466597393\\
2.49399999999995	23.6278579260798\\
2.49599999999995	23.6068340996848\\
2.49799999999995	23.5858749893148\\
2.49999999999995	23.5649804044401\\
2.50199999999995	23.5441501552382\\
2.50399999999995	23.5233840525907\\
2.50599999999995	23.5026819080803\\
2.50799999999995	23.4820435339884\\
2.50999999999995	23.4614687432922\\
2.51199999999994	23.4409573496614\\
2.51399999999994	23.4205091674559\\
2.51599999999994	23.4001240117233\\
2.51799999999994	23.3798016981946\\
2.51999999999994	23.3595420432832\\
2.52199999999994	23.3393448640812\\
2.52399999999994	23.3192099783563\\
2.52599999999994	23.2991372045497\\
2.52799999999994	23.2791263617729\\
2.52999999999994	23.2591772698048\\
2.53199999999994	23.2392897490893\\
2.53399999999994	23.2194636207319\\
2.53599999999994	23.1996987064979\\
2.53799999999994	23.1799948288081\\
2.53999999999994	23.1603518107376\\
2.54199999999994	23.1407694760115\\
2.54399999999994	23.1212476490035\\
2.54599999999994	23.1017861547322\\
2.54799999999994	23.0823848188585\\
2.54999999999994	23.0630434676825\\
2.55199999999994	23.0437619281418\\
2.55399999999994	23.0245400278071\\
2.55599999999994	23.0053775948808\\
2.55799999999994	22.9862744581933\\
2.55999999999994	22.9672304472005\\
2.56199999999994	22.9482453919811\\
2.56399999999994	22.9293191232335\\
2.56599999999994	22.9104514722736\\
2.56799999999994	22.8916422710308\\
2.56999999999994	22.8728913520465\\
2.57199999999994	22.8541985484706\\
2.57399999999994	22.8355636940586\\
2.57599999999994	22.8169866231691\\
2.57799999999994	22.7984671707608\\
2.57999999999994	22.7800051723892\\
2.58199999999994	22.7616004642054\\
2.58399999999994	22.7432528829511\\
2.58599999999994	22.7249622659568\\
2.58799999999994	22.7067284511395\\
2.58999999999994	22.6885512769991\\
2.59199999999994	22.6704305826152\\
2.59399999999994	22.6523662076454\\
2.59599999999994	22.6343579923216\\
2.59799999999994	22.6164057774472\\
2.59999999999994	22.5985094043942\\
2.60199999999994	22.5806687151011\\
2.60399999999993	22.5628835520687\\
2.60599999999993	22.5451537583586\\
2.60799999999993	22.5274791775886\\
2.60999999999993	22.5098596539321\\
2.61199999999993	22.4922950321129\\
2.61399999999993	22.474785157404\\
2.61599999999993	22.4573298756232\\
2.61799999999993	22.439929033132\\
2.61999999999993	22.4225824768309\\
2.62199999999993	22.4052900541571\\
2.62399999999993	22.3880516130824\\
2.62599999999993	22.370867002109\\
2.62799999999993	22.353736070267\\
2.62999999999993	22.3366586671118\\
2.63199999999993	22.3196346427207\\
2.63399999999993	22.3026638476898\\
2.63599999999993	22.285746133132\\
2.63799999999993	22.2688813506726\\
2.63999999999993	22.2520693524471\\
2.64199999999993	22.2353099910983\\
2.64399999999993	22.2186031197732\\
2.64599999999993	22.2019485921191\\
2.64799999999993	22.1853462622823\\
2.64999999999993	22.1687959849034\\
2.65199999999993	22.1522976151149\\
2.65399999999993	22.1358510085383\\
2.65599999999993	22.1194560212816\\
2.65799999999993	22.1031125099344\\
2.65999999999993	22.0868203315666\\
2.66199999999993	22.0705793437242\\
2.66399999999993	22.0543894044274\\
2.66599999999993	22.0382503721657\\
2.66799999999993	22.0221621058967\\
2.66999999999993	22.0061244650418\\
2.67199999999993	21.9901373094831\\
2.67399999999993	21.9742004995609\\
2.67599999999993	21.9583138960694\\
2.67799999999993	21.9424773602553\\
2.67999999999993	21.9266907538125\\
2.68199999999993	21.910953938881\\
2.68399999999993	21.8952667780419\\
2.68599999999993	21.8796291343154\\
2.68799999999993	21.8640408711567\\
2.68999999999993	21.848501852454\\
2.69199999999993	21.8330119425236\\
2.69399999999992	21.8175710061079\\
2.69599999999992	21.8021789083717\\
2.69799999999992	21.7868355148994\\
2.69999999999992	21.7715406916906\\
2.70199999999992	21.7562943051578\\
2.70399999999992	21.7410962221227\\
2.70599999999992	21.7259463098133\\
2.70799999999992	21.7108444358596\\
2.70999999999992	21.6957904682915\\
2.71199999999992	21.6807842755346\\
2.71399999999992	21.6658257264069\\
2.71599999999992	21.6509146901158\\
2.71799999999992	21.6360510362544\\
2.71999999999992	21.6212346347985\\
2.72199999999992	21.6064653561026\\
2.72399999999992	21.5917430708967\\
2.72599999999992	21.5770676502831\\
2.72799999999992	21.5624389657331\\
2.72999999999992	21.5478568890827\\
2.73199999999992	21.5333212925302\\
2.73399999999992	21.518832048632\\
2.73599999999992	21.5043890302992\\
2.73799999999992	21.4899921107945\\
2.73999999999992	21.4756411637283\\
2.74199999999992	21.4613360630556\\
2.74399999999992	21.4470766830715\\
2.74599999999992	21.4328628984092\\
2.74799999999992	21.4186945840349\\
2.74999999999992	21.4045716152453\\
2.75199999999992	21.3904938676638\\
2.75399999999992	21.3764612172364\\
2.75599999999992	21.3624735402288\\
2.75799999999992	21.3485307132227\\
2.75999999999992	21.3346326131113\\
2.76199999999992	21.3207791170972\\
2.76399999999992	21.3069701026872\\
2.76599999999992	21.2932054476903\\
2.76799999999992	21.279485030212\\
2.76999999999992	21.2658087286531\\
2.77199999999992	21.2521764217039\\
2.77399999999992	21.2385879883417\\
2.77599999999992	21.2250433078266\\
2.77799999999992	21.2115422596982\\
2.77999999999992	21.1980847237719\\
2.78199999999992	21.1846705801345\\
2.78399999999991	21.1712997091417\\
2.78599999999991	21.1579719914127\\
2.78799999999991	21.1446873078288\\
2.78999999999991	21.1314455395273\\
2.79199999999991	21.1182465678992\\
2.79399999999991	21.1050902745851\\
2.79599999999991	21.0919765414712\\
2.79799999999991	21.0789052506861\\
2.79999999999991	21.0658762845963\\
2.80199999999991	21.0528895258033\\
2.80399999999991	21.0399448571393\\
2.80599999999991	21.0270421616632\\
2.80799999999991	21.0141813226574\\
2.80999999999991	21.0013622236241\\
2.81199999999991	20.9885847482805\\
2.81399999999991	20.975848780556\\
2.81599999999991	20.9631542045887\\
2.81799999999991	20.9505009047204\\
2.81999999999991	20.9378887654935\\
2.82199999999991	20.9253176716476\\
2.82399999999991	20.9127875081152\\
2.82599999999991	20.9002981600179\\
2.82799999999991	20.8878495126627\\
2.82999999999991	20.8754414515387\\
2.83199999999991	20.8630738623125\\
2.83399999999991	20.8507466308249\\
2.83599999999991	20.8384596430871\\
2.83799999999991	20.8262127852772\\
2.83999999999991	20.8140059437358\\
2.84199999999991	20.8018390049628\\
2.84399999999991	20.7897118556134\\
2.84599999999991	20.7776243824939\\
2.84799999999991	20.7655764725592\\
2.84999999999991	20.7535680129083\\
2.85199999999991	20.7415988907802\\
2.85399999999991	20.7296689935505\\
2.85599999999991	20.7177782087286\\
2.85799999999991	20.7059264239525\\
2.85999999999991	20.694113526986\\
2.86199999999991	20.6823394057148\\
2.86399999999991	20.6706039481435\\
2.86599999999991	20.6589070423904\\
2.86799999999991	20.6472485766856\\
2.86999999999991	20.6356284393663\\
2.87199999999991	20.6240465188739\\
2.87399999999991	20.6125027037492\\
2.8759999999999	20.6009968826309\\
2.8779999999999	20.5895289442496\\
2.8799999999999	20.5780987774266\\
2.8819999999999	20.5667062710684\\
2.8839999999999	20.5553513141645\\
2.8859999999999	20.5440337957838\\
2.8879999999999	20.5327536050701\\
2.8899999999999	20.5215106312407\\
2.8919999999999	20.5103047635803\\
2.8939999999999	20.4991358914404\\
2.8959999999999	20.4880039042334\\
2.8979999999999	20.4769086914317\\
2.8999999999999	20.4658501425624\\
2.9019999999999	20.454828147205\\
2.9039999999999	20.4438425949883\\
2.9059999999999	20.4328933755864\\
2.9079999999999	20.4219803787163\\
2.9099999999999	20.4111034941343\\
2.9119999999999	20.4002626116333\\
2.9139999999999	20.389457621039\\
2.9159999999999	20.3786884122075\\
2.9179999999999	20.3679548750223\\
2.9199999999999	20.3572568993903\\
2.9219999999999	20.3465943752407\\
2.9239999999999	20.3359671925203\\
2.9259999999999	20.3253752411917\\
2.9279999999999	20.3148184112296\\
2.9299999999999	20.3042965926192\\
2.9319999999999	20.2938096753525\\
2.9339999999999	20.2833575494259\\
2.9359999999999	20.2729401048375\\
2.9379999999999	20.2625572315845\\
2.9399999999999	20.2522088196608\\
2.9419999999999	20.241894759054\\
2.9439999999999	20.2316149397438\\
2.9459999999999	20.2213692516981\\
2.9479999999999	20.2111575848726\\
2.9499999999999	20.2009798292067\\
2.9519999999999	20.1908358746217\\
2.9539999999999	20.1807256110196\\
2.9559999999999	20.1706489282793\\
2.9579999999999	20.1606057162558\\
2.9599999999999	20.1505958647774\\
2.9619999999999	20.140619263644\\
2.9639999999999	20.1306758026246\\
2.96599999999989	20.1207653714569\\
2.96799999999989	20.110887859843\\
2.96999999999989	20.1010431574499\\
2.97199999999989	20.0912311539067\\
2.97399999999989	20.0814517388033\\
2.97599999999989	20.0717048016882\\
2.97799999999989	20.0619902320674\\
2.97999999999989	20.0523079194034\\
2.98199999999989	20.0426577531126\\
2.98399999999989	20.0330396225653\\
2.98599999999989	20.0234534170832\\
2.98799999999989	20.0138990259387\\
2.98999999999989	20.0043763383543\\
2.99199999999989	19.9948852435004\\
2.99399999999989	19.985425630495\\
2.99599999999989	19.9759973884028\\
2.99799999999989	19.9666004062339\\
2.99999999999989	19.9572345729432\\
3.00199999999989	19.9478997774297\\
3.00399999999989	19.9385959085357\\
3.00599999999989	19.9293228550466\\
3.00799999999989	19.9200805056892\\
3.00999999999989	19.9108687491331\\
3.01199999999989	19.9016874739889\\
3.01399999999989	19.8925365688077\\
3.01599999999989	19.8834159220824\\
3.01799999999989	19.874325422246\\
3.01999999999989	19.8652649576721\\
3.02199999999989	19.856234416675\\
3.02399999999989	19.8472336875098\\
3.02599999999989	19.8382626583718\\
3.02799999999989	19.829321217398\\
3.02999999999989	19.8204092526659\\
3.03199999999989	19.8115266521951\\
3.03399999999989	19.8026733039471\\
3.03599999999989	19.7938490958259\\
3.03799999999989	19.7850539156791\\
3.03999999999989	19.7762876512973\\
3.04199999999989	19.7675501904165\\
3.04399999999989	19.7588414207175\\
3.04599999999989	19.7501612298277\\
3.04799999999989	19.7415095053213\\
3.04999999999989	19.7328861347216\\
3.05199999999989	19.7242910055005\\
3.05399999999989	19.7157240050812\\
3.05599999999989	19.7071850208382\\
3.05799999999988	19.6986739400999\\
3.05999999999988	19.690190650149\\
3.06199999999988	19.6817350382245\\
3.06399999999988	19.6733069915229\\
3.06599999999988	19.664906397201\\
3.06799999999988	19.6565331423757\\
3.06999999999988	19.6481871141277\\
3.07199999999988	19.6398681995023\\
3.07399999999988	19.6315762855113\\
3.07599999999988	19.623311259136\\
3.07799999999988	19.6150730073282\\
3.07999999999988	19.6068614170128\\
3.08199999999988	19.5986763750903\\
3.08399999999988	19.5905177684388\\
3.08599999999988	19.5823854839163\\
3.08799999999988	19.5742794083637\\
3.08999999999988	19.5661994286064\\
3.09199999999988	19.5581454314579\\
3.09399999999988	19.5501173037216\\
3.09599999999988	19.5421149321941\\
3.09799999999988	19.5341382036679\\
3.09999999999988	19.5261870049342\\
3.10199999999988	19.5182612227853\\
3.10399999999988	19.5103607440185\\
3.10599999999988	19.5024854554386\\
3.10799999999988	19.4946352438609\\
3.10999999999988	19.4868099961147\\
3.11199999999988	19.4790095990466\\
3.11399999999988	19.4712339395229\\
3.11599999999988	19.4634829044343\\
3.11799999999988	19.4557563806983\\
3.11999999999988	19.448054255263\\
3.12199999999988	19.44037641511\\
3.12399999999988	19.4327227472595\\
3.12599999999988	19.425093138772\\
3.12799999999988	19.4174874767533\\
3.12999999999988	19.4099056483574\\
3.13199999999988	19.402347540791\\
3.13399999999988	19.3948130413164\\
3.13599999999988	19.3873020372564\\
3.13799999999988	19.3798144159971\\
3.13999999999988	19.3723500649922\\
3.14199999999988	19.3649088717678\\
3.14399999999988	19.3574907239251\\
3.14599999999988	19.3500955091449\\
3.14799999999987	19.3427231151923\\
3.14999999999987	19.3353734299199\\
3.15199999999987	19.3280463412726\\
3.15399999999987	19.3207417372913\\
3.15599999999987	19.3134595061176\\
3.15799999999987	19.3061995359972\\
3.15999999999987	19.2989617152854\\
3.16199999999987	19.2917459324502\\
3.16399999999987	19.2845520760772\\
3.16599999999987	19.2773800348743\\
3.16799999999987	19.2702296976747\\
3.16999999999987	19.2631009534429\\
3.17199999999987	19.2559936912781\\
3.17399999999987	19.2489078004188\\
3.17599999999987	19.2418431702474\\
3.17799999999987	19.2347996902941\\
3.17999999999987	19.2277772502423\\
3.18199999999987	19.2207757399323\\
3.18399999999987	19.2137950493656\\
3.18599999999987	19.2068350687104\\
3.18799999999987	19.1998956883047\\
3.18999999999987	19.1929767986621\\
3.19199999999987	19.1860782904751\\
3.19399999999987	19.1792000546207\\
3.19599999999987	19.1723419821637\\
3.19799999999987	19.1655039643623\\
3.19999999999987	19.1586858926719\\
3.20199999999987	19.1518876587495\\
3.20399999999987	19.1451091544588\\
3.20599999999987	19.1383502718739\\
3.20799999999987	19.1316109032844\\
3.20999999999987	19.1248909411989\\
3.21199999999987	19.1181902783507\\
3.21399999999987	19.1115088077012\\
3.21599999999987	19.1048464224445\\
3.21799999999987	19.0982030160123\\
3.21999999999987	19.0915784820772\\
3.22199999999987	19.084972714558\\
3.22399999999987	19.0783856076237\\
3.22599999999987	19.0718170556976\\
3.22799999999987	19.0652669534617\\
3.22999999999987	19.0587351958608\\
3.23199999999987	19.0522216781072\\
3.23399999999987	19.045726295684\\
3.23599999999987	19.0392489443499\\
3.23799999999986	19.0327895201436\\
3.23999999999986	19.0263479193867\\
3.24199999999986	19.019924038689\\
3.24399999999986	19.0135177749519\\
3.24599999999986	19.0071290253725\\
3.24799999999986	19.0007576874474\\
3.24999999999986	18.9944036589767\\
3.25199999999986	18.9880668380684\\
3.25399999999986	18.9817471231409\\
3.25599999999986	18.9754444129284\\
3.25799999999986	18.9691586064832\\
3.25999999999986	18.9628896031808\\
3.26199999999986	18.9566373027221\\
3.26399999999986	18.950401605138\\
3.26599999999986	18.9441824107928\\
3.26799999999986	18.9379796203876\\
3.26999999999986	18.9317931349637\\
3.27199999999986	18.9256228559062\\
3.27399999999986	18.919468684947\\
3.27599999999986	18.9133305241688\\
3.27799999999986	18.907208276008\\
3.27999999999986	18.9011018432575\\
3.28199999999986	18.8950111290708\\
3.28399999999986	18.8889360369642\\
3.28599999999986	18.8828764708203\\
3.28799999999986	18.8768323348915\\
3.28999999999986	18.870803533802\\
3.29199999999986	18.8647899725512\\
3.29399999999986	18.8587915565164\\
3.29599999999986	18.8528081914564\\
3.29799999999986	18.8468397835126\\
3.29999999999986	18.8408862392136\\
3.30199999999986	18.834947465476\\
3.30399999999986	18.8290233696083\\
3.30599999999986	18.8231138593129\\
3.30799999999986	18.8172188426887\\
3.30999999999986	18.8113382282332\\
3.31199999999986	18.8054719248451\\
3.31399999999986	18.7996198418267\\
3.31599999999986	18.7937818888859\\
3.31799999999986	18.7879579761384\\
3.31999999999986	18.7821480141097\\
3.32199999999986	18.7763519137377\\
3.32399999999986	18.7705695863737\\
3.32599999999986	18.7648009437856\\
3.32799999999986	18.7590458981581\\
3.32999999999985	18.7533043620968\\
3.33199999999985	18.7475762486276\\
3.33399999999985	18.7418614711998\\
3.33599999999985	18.7361599436878\\
3.33799999999985	18.7304715803917\\
3.33999999999985	18.7247962960397\\
3.34199999999985	18.719134005789\\
3.34399999999985	18.7134846252283\\
3.34599999999985	18.707848070377\\
3.34799999999985	18.7022242576891\\
3.34999999999985	18.6966131040525\\
3.35199999999985	18.691014526791\\
3.35399999999985	18.6854284436654\\
3.35599999999985	18.6798547728741\\
3.35799999999985	18.6742934330549\\
3.35999999999985	18.6687443432852\\
3.36199999999985	18.6632074230832\\
3.36399999999985	18.6576825924089\\
3.36599999999985	18.6521697716651\\
3.36799999999985	18.6466688816969\\
3.36999999999985	18.6411798437941\\
3.37199999999985	18.6357025796908\\
3.37399999999985	18.6302370115659\\
3.37599999999985	18.6247830620444\\
3.37799999999985	18.6193406541971\\
3.37999999999985	18.6139097115414\\
3.38199999999985	18.6084901580416\\
3.38399999999985	18.6030819181094\\
3.38599999999985	18.5976849166039\\
3.38799999999985	18.5922990788318\\
3.38999999999985	18.5869243305481\\
3.39199999999985	18.5815605979556\\
3.39399999999985	18.5762078077053\\
3.39599999999985	18.5708658868963\\
3.39799999999985	18.5655347630766\\
3.39999999999985	18.5602143642411\\
3.40199999999985	18.5549046188341\\
3.40399999999985	18.5496054557469\\
3.40599999999985	18.5443168043188\\
3.40799999999985	18.5390385943371\\
3.40999999999985	18.5337707560358\\
3.41199999999985	18.5285132200959\\
3.41399999999985	18.5232659176456\\
3.41599999999985	18.5180287802587\\
3.41799999999985	18.5128017399554\\
3.41999999999984	18.5075847292009\\
3.42199999999984	18.5023776809053\\
3.42399999999984	18.4971805284234\\
3.42599999999984	18.4919932055537\\
3.42799999999984	18.4868156465376\\
3.42999999999984	18.4816477860596\\
3.43199999999984	18.4764895592457\\
3.43399999999984	18.4713409016633\\
3.43599999999984	18.4662017493203\\
3.43799999999984	18.4610720386644\\
3.43999999999984	18.4559517065821\\
3.44199999999984	18.4508406903981\\
3.44399999999984	18.4457389278746\\
3.44599999999984	18.4406463572099\\
3.44799999999984	18.4355629170379\\
3.44999999999984	18.4304885464273\\
3.45199999999984	18.4254231848803\\
3.45399999999984	18.4203667723318\\
3.45599999999984	18.4153192491485\\
3.45799999999984	18.4102805561275\\
3.45999999999984	18.4052506344958\\
3.46199999999984	18.4002294259088\\
3.46399999999984	18.3952168724496\\
3.46599999999984	18.3902129166274\\
3.46799999999984	18.3852175013768\\
3.46999999999984	18.3802305700568\\
3.47199999999984	18.3752520664488\\
3.47399999999984	18.3702819347566\\
3.47599999999984	18.3653201196043\\
3.47799999999984	18.3603665660355\\
3.47999999999984	18.3554212195123\\
3.48199999999984	18.3504840259133\\
3.48399999999984	18.3455549315333\\
3.48599999999984	18.3406338830815\\
3.48799999999984	18.3357208276803\\
3.48999999999984	18.3308157128644\\
3.49199999999984	18.3259184865789\\
3.49399999999984	18.3210290971786\\
3.49599999999984	18.3161474934265\\
3.49799999999984	18.3112736244922\\
3.49999999999984	18.3064074399514\\
3.50199999999984	18.3015488897839\\
3.50399999999984	18.2966979243724\\
3.50599999999984	18.2918544945014\\
3.50799999999984	18.2870185513558\\
3.50999999999984	18.2821900465199\\
3.51199999999983	18.2773689319753\\
3.51399999999983	18.2725551601005\\
3.51599999999983	18.2677486836692\\
3.51799999999983	18.2629494558488\\
3.51999999999983	18.2581574301996\\
3.52199999999983	18.2533725606733\\
3.52399999999983	18.2485948016113\\
3.52599999999983	18.2438241077442\\
3.52799999999983	18.2390604341901\\
3.52999999999983	18.2343037364532\\
3.53199999999983	18.2295539704232\\
3.53399999999983	18.224811092373\\
3.53599999999983	18.2200750589589\\
3.53799999999983	18.2153458272179\\
3.53999999999983	18.2106233545678\\
3.54199999999983	18.2059075988051\\
3.54399999999983	18.2011985181045\\
3.54599999999983	18.1964960710172\\
3.54799999999983	18.1918002164701\\
3.54999999999983	18.187110913765\\
3.55199999999983	18.1824281225766\\
3.55399999999983	18.1777518029525\\
3.55599999999983	18.1730819153115\\
3.55799999999983	18.1684184204426\\
3.55999999999983	18.1637612795045\\
3.56199999999983	18.1591104540241\\
3.56399999999983	18.154465905896\\
3.56599999999983	18.1498275973809\\
3.56799999999983	18.1451954911061\\
3.56999999999983	18.1405695500628\\
3.57199999999983	18.1359497376071\\
3.57399999999983	18.1313360174577\\
3.57599999999983	18.1267283536964\\
3.57799999999983	18.1221267107667\\
3.57999999999983	18.1175310534731\\
3.58199999999983	18.112941346981\\
3.58399999999983	18.1083575568155\\
3.58599999999983	18.1037796488614\\
3.58799999999983	18.0992075893625\\
3.58999999999983	18.0946413449208\\
3.59199999999983	18.0900808824968\\
3.59399999999983	18.0855261694085\\
3.59599999999983	18.0809771733315\\
3.59799999999983	18.0764338622987\\
3.59999999999983	18.0718962046999\\
3.60199999999982	18.0673641692819\\
3.60399999999982	18.0628377251483\\
3.60599999999982	18.0583168417596\\
3.60799999999982	18.0538014889332\\
3.60999999999982	18.0492916368433\\
3.61199999999982	18.0447872560214\\
3.61399999999982	18.0402883173565\\
3.61599999999982	18.0357947920952\\
3.61799999999982	18.0313066518424\\
3.61999999999982	18.0268238685615\\
3.62199999999982	18.0223464145751\\
3.62399999999982	18.0178742625657\\
3.62599999999982	18.0134073855764\\
3.62799999999982	18.0089457570114\\
3.62999999999982	18.0044893506373\\
3.63199999999982	18.0000381405841\\
3.63399999999982	17.9955921013456\\
3.63599999999982	17.9911512077819\\
3.63799999999982	17.9867154351192\\
3.63999999999982	17.9822847589526\\
3.64199999999982	17.9778591552462\\
3.64399999999982	17.9734386003358\\
3.64599999999982	17.9690230709306\\
3.64799999999982	17.9646125441144\\
3.64999999999982	17.9602069973479\\
3.65199999999982	17.955806408471\\
3.65399999999982	17.9514107557051\\
3.65599999999982	17.9470200176549\\
3.65799999999982	17.942634173312\\
3.65999999999982	17.9382532020563\\
3.66199999999982	17.9338770836597\\
3.66399999999982	17.9295057982892\\
3.66599999999982	17.9251393265094\\
3.66799999999982	17.9207776492862\\
3.66999999999982	17.9164207479907\\
3.67199999999982	17.912068604402\\
3.67399999999982	17.9077212007121\\
3.67599999999982	17.9033785195286\\
3.67799999999982	17.8990405438808\\
3.67999999999982	17.8947072572227\\
3.68199999999982	17.8903786434384\\
3.68399999999982	17.8860546868467\\
3.68599999999982	17.8817353722064\\
3.68799999999982	17.8774206847219\\
3.68999999999982	17.8731106100485\\
3.69199999999981	17.8688051342981\\
3.69399999999981	17.8645042440465\\
3.69599999999981	17.8602079263381\\
3.69799999999981	17.8559161686942\\
3.69999999999981	17.8516289591191\\
3.70199999999981	17.8473462861075\\
3.70399999999981	17.8430681386525\\
3.70599999999981	17.8387945062534\\
3.70799999999981	17.8345253789236\\
3.70999999999981	17.8302607471996\\
3.71199999999981	17.8260006021503\\
3.71399999999981	17.8217449353853\\
3.71599999999981	17.8174937390661\\
3.71799999999981	17.8132470059145\\
3.71999999999981	17.8090047292249\\
3.72199999999981	17.8047669028741\\
3.72399999999981	17.8005335213331\\
3.72599999999981	17.7963045796785\\
3.72799999999981	17.7920800736057\\
3.72999999999981	17.787859999441\\
3.73199999999981	17.783644354155\\
3.73399999999981	17.7794331353765\\
3.73599999999981	17.7752263414067\\
3.73799999999981	17.7710239712342\\
3.73999999999981	17.7668260245502\\
3.74199999999981	17.7626325017649\\
3.74399999999981	17.7584434040235\\
3.74599999999981	17.7542587332239\\
3.74799999999981	17.7500784920344\\
3.74999999999981	17.7459026839118\\
3.75199999999981	17.7417313131208\\
3.75399999999981	17.7375643847539\\
3.75599999999981	17.7334019047518\\
3.75799999999981	17.7292438799242\\
3.75999999999981	17.7250903179722\\
3.76199999999981	17.7209412275108\\
3.76399999999981	17.7167966180921\\
3.76599999999981	17.7126565002295\\
3.76799999999981	17.7085208854229\\
3.76999999999981	17.7043897861841\\
3.77199999999981	17.7002632160634\\
3.77399999999981	17.696141189677\\
3.77599999999981	17.692023722735\\
3.77799999999981	17.6879108320703\\
3.77999999999981	17.6838025356687\\
3.78199999999981	17.6796988526989\\
3.7839999999998	17.6755998035447\\
3.7859999999998	17.671505409836\\
3.7879999999998	17.667415694483\\
3.7899999999998	17.663330681709\\
3.7919999999998	17.6592503970854\\
3.7939999999998	17.6551748675669\\
3.7959999999998	17.6511041215269\\
3.7979999999998	17.6470381887948\\
3.7999999999998	17.6429771006925\\
3.8019999999998	17.6389208900722\\
3.8039999999998	17.6348695913546\\
3.8059999999998	17.6308232405675\\
3.8079999999998	17.6267818753844\\
3.8099999999998	17.6227455351641\\
3.8119999999998	17.6187142609887\\
3.8139999999998	17.6146880957038\\
3.8159999999998	17.6106670839563\\
3.8179999999998	17.6066512722341\\
3.8199999999998	17.6026407089031\\
3.8219999999998	17.5986354442446\\
3.8239999999998	17.5946355304918\\
3.8259999999998	17.5906410218651\\
3.8279999999998	17.5866519746048\\
3.8299999999998	17.5826684470036\\
3.8319999999998	17.5786904994363\\
3.8339999999998	17.5747181943869\\
3.8359999999998	17.570751596473\\
3.8379999999998	17.5667907724668\\
3.8399999999998	17.5628357913137\\
3.8419999999998	17.5588867241437\\
3.8439999999998	17.5549436442825\\
3.8459999999998	17.5510066272533\\
3.8479999999998	17.5470757507753\\
3.8499999999998	17.5431510947551\\
3.8519999999998	17.5392327412709\\
3.8539999999998	17.5353207745495\\
3.8559999999998	17.5314152809342\\
3.8579999999998	17.5275163488449\\
3.8599999999998	17.523624068727\\
3.8619999999998	17.5197385329902\\
3.8639999999998	17.5158598359368\\
3.8659999999998	17.5119880736763\\
3.8679999999998	17.5081233440289\\
3.8699999999998	17.5042657464142\\
3.8719999999998	17.5004153817257\\
3.87399999999979	17.4965723521911\\
3.87599999999979	17.4927367612159\\
3.87799999999979	17.4889087132118\\
3.87999999999979	17.4850883134076\\
3.88199999999979	17.4812756676439\\
3.88399999999979	17.4774708821492\\
3.88599999999979	17.4736740632999\\
3.88799999999979	17.469885317362\\
3.88999999999979	17.4661047502151\\
3.89199999999979	17.4623324670611\\
3.89399999999979	17.458568572114\\
3.89599999999979	17.4548131682767\\
3.89799999999979	17.4510663568009\\
3.89999999999979	17.4473282369353\\
3.90199999999979	17.4435989055608\\
3.90399999999979	17.4398784568164\\
3.90599999999979	17.4361669817162\\
3.90799999999979	17.4324645677617\\
3.90999999999979	17.4287712985491\\
3.91199999999979	17.4250872533776\\
3.91399999999979	17.4214125068589\\
3.91599999999979	17.4177471285327\\
3.91799999999979	17.4140911824899\\
3.91999999999979	17.4104447270096\\
3.92199999999979	17.4068078142087\\
3.92399999999979	17.403180489713\\
3.92599999999979	17.3995627923479\\
3.92799999999979	17.395954753856\\
3.92999999999979	17.392356398642\\
3.93199999999979	17.3887677435498\\
3.93399999999979	17.3851887976719\\
3.93599999999979	17.3816195621958\\
3.93799999999979	17.3780600302885\\
3.93999999999979	17.3745101870189\\
3.94199999999979	17.3709700093223\\
3.94399999999979	17.367439466005\\
3.94599999999979	17.3639185177892\\
3.94799999999979	17.3604071174012\\
3.94999999999979	17.356905209696\\
3.95199999999979	17.3534127318229\\
3.95399999999979	17.3499296134274\\
3.95599999999979	17.3464557768876\\
3.95799999999979	17.3429911375824\\
3.95999999999979	17.3395356041905\\
3.96199999999979	17.3360890790137\\
3.96399999999979	17.3326514583261\\
3.96599999999978	17.3292226327402\\
3.96799999999978	17.3258024875911\\
3.96999999999978	17.3223909033334\\
3.97199999999978	17.3189877559466\\
3.97399999999978	17.3155929173474\\
3.97599999999978	17.3122062558036\\
3.97799999999978	17.308827636348\\
3.97999999999978	17.3054569211897\\
3.98199999999978	17.3020939701178\\
3.98399999999978	17.2987386408974\\
3.98599999999978	17.2953907896548\\
3.98799999999978	17.2920502712496\\
3.98999999999978	17.2887169396325\\
3.99199999999978	17.2853906481882\\
3.99399999999978	17.2820712500592\\
3.99599999999978	17.2787585984541\\
3.99799999999978	17.275452546936\\
3.99999999999978	17.2721529496922\\
4.00199999999978	17.2688596617853\\
4.00399999999978	17.2655725393843\\
4.00599999999978	17.2622914399775\\
4.00799999999978	17.2590162225662\\
4.00999999999978	17.25574674784\\
4.01199999999978	17.2524828783341\\
4.01399999999978	17.2492244785708\\
4.01599999999978	17.2459714151829\\
4.01799999999978	17.2427235570216\\
4.01999999999978	17.2394807752506\\
4.02199999999978	17.2362429434251\\
4.02399999999978	17.2330099375578\\
4.02599999999978	17.2297816361727\\
4.02799999999978	17.2265579203472\\
4.02999999999978	17.2233386737432\\
4.03199999999978	17.2201237826295\\
4.03399999999978	17.2169131358941\\
4.03599999999978	17.2137066250478\\
4.03799999999978	17.2105041442216\\
4.03999999999978	17.2073055901569\\
4.04199999999978	17.2041108621889\\
4.04399999999978	17.2009198622252\\
4.04599999999978	17.1977324947191\\
4.04799999999978	17.1945486666389\\
4.04999999999978	17.1913682874334\\
4.05199999999978	17.1881912689934\\
4.05399999999978	17.1850175256126\\
4.05599999999978	17.1818469739423\\
4.05799999999978	17.1786795329479\\
4.05999999999977	17.1755151238614\\
4.06199999999977	17.1723536701331\\
4.06399999999977	17.1691950973831\\
4.06599999999977	17.1660393333508\\
4.06799999999977	17.1628863078443\\
4.06999999999977	17.15973595269\\
4.07199999999977	17.1565882016813\\
4.07399999999977	17.1534429905279\\
4.07599999999977	17.1503002568052\\
4.07799999999977	17.1471599399036\\
4.07999999999977	17.1440219809792\\
4.08199999999977	17.140886322904\\
4.08399999999977	17.1377529102174\\
4.08599999999977	17.1346216890786\\
4.08799999999977	17.1314926072192\\
4.08999999999977	17.128365613897\\
4.09199999999977	17.1252406598512\\
4.09399999999977	17.1221176972572\\
4.09599999999977	17.1189966796839\\
4.09799999999977	17.1158775620514\\
4.09999999999977	17.1127603005891\\
4.10199999999977	17.1096448527958\\
4.10399999999977	17.1065311774005\\
4.10599999999977	17.1034192343239\\
4.10799999999977	17.1003089846418\\
4.10999999999977	17.0972003905478\\
4.11199999999977	17.0940934153199\\
4.11399999999977	17.0909880232845\\
4.11599999999977	17.0878841797854\\
4.11799999999977	17.0847818511497\\
4.11999999999977	17.081681004658\\
4.12199999999977	17.0785816085141\\
4.12399999999977	17.0754836318149\\
4.12599999999977	17.0723870445234\\
4.12799999999977	17.0692918174402\\
4.12999999999977	17.0661979221777\\
4.13199999999977	17.0631053311338\\
4.13399999999977	17.0600140174679\\
4.13599999999977	17.0569239550759\\
4.13799999999977	17.0538351185678\\
4.13999999999977	17.0507474832448\\
4.14199999999977	17.0476610250774\\
4.14399999999977	17.0445757206851\\
4.14599999999977	17.0414915473152\\
4.14799999999977	17.038408482824\\
4.14999999999976	17.035326505657\\
4.15199999999976	17.0322455948315\\
4.15399999999976	17.0291657299182\\
4.15599999999976	17.0260868910241\\
4.15799999999976	17.0230090587766\\
4.15999999999976	17.0199322143067\\
4.16199999999976	17.0168563392341\\
4.16399999999976	17.0137814156524\\
4.16599999999976	17.0107074261138\\
4.16799999999976	17.0076343536164\\
4.16999999999976	17.0045621815898\\
4.17199999999976	17.0014908938826\\
4.17399999999976	16.9984204747497\\
4.17599999999976	16.99535090884\\
4.17799999999976	16.9922821811849\\
4.17999999999976	16.9892142771872\\
4.18199999999976	16.9861471826094\\
4.18399999999976	16.9830808835637\\
4.18599999999976	16.9800153665018\\
4.18799999999976	16.9769506182049\\
4.18999999999976	16.9738866257735\\
4.19199999999976	16.9708233766196\\
4.19399999999976	16.9677608584563\\
4.19599999999976	16.96469905929\\
4.19799999999976	16.9616379674117\\
4.19999999999976	16.9585775713893\\
4.20199999999976	16.9555178600596\\
4.20399999999976	16.9524588225202\\
4.20599999999976	16.949400448123\\
4.20799999999976	16.946342726467\\
4.20999999999976	16.943285647391\\
4.21199999999976	16.9402292009675\\
4.21399999999976	16.9371733774955\\
4.21599999999976	16.9341181674956\\
4.21799999999976	16.9310635617031\\
4.21999999999976	16.9280095510617\\
4.22199999999976	16.9249561267193\\
4.22399999999976	16.9219032800218\\
4.22599999999976	16.9188510025073\\
4.22799999999976	16.9157992859016\\
4.22999999999976	16.9127481221136\\
4.23199999999976	16.9096975032299\\
4.23399999999976	16.9066474215102\\
4.23599999999976	16.9035978693833\\
4.23799999999976	16.9005488394418\\
4.23999999999976	16.8975003244392\\
4.24199999999975	16.8944523172847\\
4.24399999999975	16.8914048110398\\
4.24599999999975	16.888357798914\\
4.24799999999975	16.8853112742617\\
4.24999999999975	16.8822652305781\\
4.25199999999975	16.8792196614958\\
4.25399999999975	16.8761745607815\\
4.25599999999975	16.8731299223329\\
4.25799999999975	16.8700857401747\\
4.25999999999975	16.8670420084564\\
4.26199999999975	16.863998721449\\
4.26399999999975	16.8609558735417\\
4.26599999999975	16.8579134592397\\
4.26799999999975	16.8548714731609\\
4.26999999999975	16.8518299100333\\
4.27199999999975	16.848788764693\\
4.27399999999975	16.8457480320807\\
4.27599999999975	16.8427077072402\\
4.27799999999975	16.8396677853153\\
4.27999999999975	16.8366282615478\\
4.28199999999975	16.8335891312753\\
4.28399999999975	16.830550389929\\
4.28599999999975	16.8275120330312\\
4.28799999999975	16.8244740561939\\
4.28999999999975	16.8214364551163\\
4.29199999999975	16.8183992255829\\
4.29399999999975	16.8153623634619\\
4.29599999999975	16.812325864703\\
4.29799999999975	16.8092897253359\\
4.29999999999975	16.8062539414682\\
4.30199999999975	16.8032185092839\\
4.30399999999975	16.8001834250423\\
4.30599999999975	16.7971486850749\\
4.30799999999975	16.7941142857857\\
4.30999999999975	16.7910802236482\\
4.31199999999975	16.7880464952046\\
4.31399999999975	16.7850130970646\\
4.31599999999975	16.7819800259029\\
4.31799999999975	16.7789472784592\\
4.31999999999975	16.7759148515361\\
4.32199999999975	16.7728827419978\\
4.32399999999975	16.7698509467689\\
4.32599999999975	16.7668194628333\\
4.32799999999975	16.7637882872329\\
4.32999999999975	16.7607574170664\\
4.33199999999974	16.7577268494887\\
4.33399999999974	16.7546965817087\\
4.33599999999974	16.7516666109887\\
4.33799999999974	16.7486369346442\\
4.33999999999974	16.7456075500414\\
4.34199999999974	16.7425784545974\\
4.34399999999974	16.7395496457784\\
4.34599999999974	16.7365211210992\\
4.34799999999974	16.7334928781222\\
4.34999999999974	16.7304649144564\\
4.35199999999974	16.7274372277563\\
4.35399999999974	16.7244098157218\\
4.35599999999974	16.7213826760963\\
4.35799999999974	16.7183558066665\\
4.35999999999974	16.7153292052618\\
4.36199999999974	16.712302869753\\
4.36399999999974	16.7092767980514\\
4.36599999999974	16.7062509881091\\
4.36799999999974	16.7032254379166\\
4.36999999999974	16.7002001455041\\
4.37199999999974	16.6971751089387\\
4.37399999999974	16.6941503263255\\
4.37599999999974	16.6911257958057\\
4.37799999999974	16.6881015155566\\
4.37999999999974	16.6850774837907\\
4.38199999999974	16.6820536987551\\
4.38399999999974	16.6790301587312\\
4.38599999999974	16.6760068620336\\
4.38799999999974	16.6729838070095\\
4.38999999999974	16.669960992039\\
4.39199999999974	16.6669384155333\\
4.39399999999974	16.6639160759354\\
4.39599999999974	16.660893971718\\
4.39799999999974	16.657872101385\\
4.39999999999974	16.6548504634693\\
4.40199999999974	16.6518290565328\\
4.40399999999974	16.6488078791663\\
4.40599999999974	16.6457869299884\\
4.40799999999974	16.6427662076458\\
4.40999999999974	16.6397457108122\\
4.41199999999974	16.6367254381877\\
4.41399999999974	16.6337053884991\\
4.41599999999974	16.6306855604991\\
4.41799999999974	16.6276659529655\\
4.41999999999974	16.6246465647014\\
4.42199999999974	16.6216273945343\\
4.42399999999973	16.6186084413165\\
4.42599999999973	16.6155897039233\\
4.42799999999973	16.6125711812539\\
4.42999999999973	16.6095528722307\\
4.43199999999973	16.6065347757986\\
4.43399999999973	16.6035168909248\\
4.43599999999973	16.6004992165985\\
4.43799999999973	16.597481751831\\
4.43999999999973	16.5944644956541\\
4.44199999999973	16.5914474471214\\
4.44399999999973	16.5884306053065\\
4.44599999999973	16.5854139693037\\
4.44799999999973	16.5823975382273\\
4.44999999999973	16.5793813112112\\
4.45199999999973	16.5763652874088\\
4.45399999999973	16.5733494659923\\
4.45599999999973	16.5703338461533\\
4.45799999999973	16.5673184271014\\
4.45999999999973	16.5643032080648\\
4.46199999999973	16.5612881882896\\
4.46399999999973	16.5582733670395\\
4.46599999999973	16.5552587435956\\
4.46799999999973	16.5522443172565\\
4.46999999999973	16.5492300873373\\
4.47199999999973	16.5462160531702\\
4.47399999999973	16.5432022141037\\
4.47599999999973	16.5401885695022\\
4.47799999999973	16.5371751187465\\
4.47999999999973	16.5341618612328\\
4.48199999999973	16.5311487963729\\
4.48399999999973	16.528135923594\\
4.48599999999973	16.5251232423382\\
4.48799999999973	16.5221107520625\\
4.48999999999973	16.5190984522383\\
4.49199999999973	16.5160863423517\\
4.49399999999973	16.5130744219029\\
4.49599999999973	16.5100626904061\\
4.49799999999973	16.5070511473891\\
4.49999999999973	16.5040397923938\\
4.50199999999973	16.5010286249749\\
4.50399999999973	16.4980176447008\\
4.50599999999973	16.4950068511526\\
4.50799999999973	16.4919962439246\\
4.50999999999973	16.4889858226234\\
4.51199999999973	16.4859755868683\\
4.51399999999972	16.4829655362909\\
4.51599999999972	16.4799556705351\\
4.51799999999972	16.4769459892564\\
4.51999999999972	16.4739364921225\\
4.52199999999972	16.4709271788125\\
4.52399999999972	16.4679180490172\\
4.52599999999972	16.4649091024385\\
4.52799999999972	16.4619003387896\\
4.52999999999972	16.4588917577947\\
4.53199999999972	16.455883359189\\
4.53399999999972	16.4528751427183\\
4.53599999999972	16.449867108139\\
4.53799999999972	16.4468592552179\\
4.53999999999972	16.4438515837322\\
4.54199999999972	16.4408440934692\\
4.54399999999972	16.4378367842262\\
4.54599999999972	16.4348296558104\\
4.54799999999972	16.4318227080388\\
4.54999999999972	16.4288159407379\\
4.55199999999972	16.425809353744\\
4.55399999999972	16.4228029469021\\
4.55599999999972	16.4197967200673\\
4.55799999999972	16.4167906731033\\
4.55999999999972	16.4137848058828\\
4.56199999999972	16.4107791182875\\
4.56399999999972	16.4077736102079\\
4.56599999999972	16.404768281543\\
4.56799999999972	16.4017631322006\\
4.56999999999972	16.3987581620964\\
4.57199999999972	16.395753371155\\
4.57399999999972	16.3927487593088\\
4.57599999999972	16.3897443264985\\
4.57799999999972	16.3867400726726\\
4.57999999999972	16.3837359977876\\
4.58199999999972	16.380732101808\\
4.58399999999972	16.3777283847053\\
4.58599999999972	16.3747248464592\\
4.58799999999972	16.3717214870567\\
4.58999999999972	16.3687183064922\\
4.59199999999972	16.3657153047672\\
4.59399999999972	16.3627124818903\\
4.59599999999972	16.3597098378779\\
4.59799999999972	16.3567073727525\\
4.59999999999972	16.3537050865441\\
4.60199999999972	16.3507029792892\\
4.60399999999971	16.3477010510312\\
4.60599999999971	16.3446993018201\\
4.60799999999971	16.3416977317125\\
4.60999999999971	16.3386963407713\\
4.61199999999971	16.335695129066\\
4.61399999999971	16.3326940966722\\
4.61599999999971	16.329693243672\\
4.61799999999971	16.3266925701533\\
4.61999999999971	16.3236920762103\\
4.62199999999971	16.3206917619432\\
4.62399999999971	16.317691627458\\
4.62599999999971	16.3146916728665\\
4.62799999999971	16.3116918982864\\
4.62999999999971	16.3086923038409\\
4.63199999999971	16.305692889659\\
4.63399999999971	16.3026936558748\\
4.63599999999971	16.2996946026288\\
4.63799999999971	16.2966957300657\\
4.63999999999971	16.2936970383366\\
4.64199999999971	16.2906985275971\\
4.64399999999971	16.2877001980083\\
4.64599999999971	16.2847020497363\\
4.64799999999971	16.2817040829524\\
4.64999999999971	16.2787062978328\\
4.65199999999971	16.2757086945589\\
4.65399999999971	16.2727112733162\\
4.65599999999971	16.2697140342958\\
4.65799999999971	16.2667169776932\\
4.65999999999971	16.2637201037085\\
4.66199999999971	16.2607234125466\\
4.66399999999971	16.2577269044169\\
4.66599999999971	16.2547305795331\\
4.66799999999971	16.2517344381137\\
4.66999999999971	16.2487384803812\\
4.67199999999971	16.2457427065626\\
4.67399999999971	16.2427471168892\\
4.67599999999971	16.2397517115964\\
4.67799999999971	16.2367564909237\\
4.67999999999971	16.2337614551152\\
4.68199999999971	16.2307666044183\\
4.68399999999971	16.2277719390848\\
4.68599999999971	16.2247774593703\\
4.68799999999971	16.2217831655346\\
4.68999999999971	16.218789057841\\
4.69199999999971	16.2157951365566\\
4.69399999999971	16.2128014019524\\
4.6959999999997	16.209807854303\\
4.6979999999997	16.2068144938868\\
4.6999999999997	16.2038213209851\\
4.7019999999997	16.2008283358839\\
4.7039999999997	16.1978355388718\\
4.7059999999997	16.1948429302411\\
4.7079999999997	16.1918505102875\\
4.7099999999997	16.1888582793102\\
4.7119999999997	16.1858662376115\\
4.7139999999997	16.182874385497\\
4.7159999999997	16.1798827232757\\
4.7179999999997	16.1768912512595\\
4.7199999999997	16.1738999697637\\
4.7219999999997	16.1709088791067\\
4.7239999999997	16.1679179796096\\
4.7259999999997	16.1649272715969\\
4.7279999999997	16.1619367553957\\
4.7299999999997	16.1589464313367\\
4.7319999999997	16.1559562997525\\
4.7339999999997	16.1529663609794\\
4.7359999999997	16.1499766153561\\
4.7379999999997	16.1469870632242\\
4.7399999999997	16.143997704928\\
4.7419999999997	16.1410085408143\\
4.7439999999997	16.1380195712331\\
4.7459999999997	16.1350307965364\\
4.7479999999997	16.1320422170791\\
4.7499999999997	16.1290538332187\\
4.7519999999997	16.1260656453149\\
4.7539999999997	16.1230776537303\\
4.7559999999997	16.1200898588296\\
4.7579999999997	16.11710226098\\
4.7599999999997	16.1141148605513\\
4.7619999999997	16.1111276579153\\
4.7639999999997	16.1081406534463\\
4.7659999999997	16.1051538475208\\
4.7679999999997	16.1021672405174\\
4.7699999999997	16.0991808328173\\
4.7719999999997	16.0961946248035\\
4.7739999999997	16.0932086168612\\
4.7759999999997	16.090222809378\\
4.7779999999997	16.0872372027433\\
4.7799999999997	16.0842517973484\\
4.7819999999997	16.0812665935871\\
4.7839999999997	16.0782815918549\\
4.78599999999969	16.075296792549\\
4.78799999999969	16.0723121960691\\
4.78999999999969	16.0693278028164\\
4.79199999999969	16.066343613194\\
4.79399999999969	16.0633596276071\\
4.79599999999969	16.0603758464624\\
4.79799999999969	16.0573922701685\\
4.79999999999969	16.0544088991358\\
4.80199999999969	16.0514257337766\\
4.80399999999969	16.0484427745043\\
4.80599999999969	16.0454600217348\\
4.80799999999969	16.0424774758849\\
4.80999999999969	16.0394951373736\\
4.81199999999969	16.0365130066211\\
4.81399999999969	16.0335310840495\\
4.81599999999969	16.0305493700821\\
4.81799999999969	16.0275678651439\\
4.81999999999969	16.0245865696616\\
4.82199999999969	16.0216054840629\\
4.82399999999969	16.0186246087775\\
4.82599999999969	16.015643944236\\
4.82799999999969	16.0126634908707\\
4.82999999999969	16.0096832491153\\
4.83199999999969	16.0067032194047\\
4.83399999999969	16.0037234021752\\
4.83599999999969	16.0007437978644\\
4.83799999999969	15.9977644069112\\
4.83999999999969	15.9947852297558\\
4.84199999999969	15.9918062668395\\
4.84399999999969	15.988827518605\\
4.84599999999969	15.985848985496\\
4.84799999999969	15.9828706679576\\
4.84999999999969	15.9798925664358\\
4.85199999999969	15.9769146813779\\
4.85399999999969	15.9739370132324\\
4.85599999999969	15.9709595624485\\
4.85799999999969	15.967982329477\\
4.85999999999969	15.9650053147694\\
4.86199999999969	15.9620285187783\\
4.86399999999969	15.9590519419572\\
4.86599999999969	15.9560755847609\\
4.86799999999969	15.9530994476449\\
4.86999999999969	15.9501235310657\\
4.87199999999969	15.9471478354808\\
4.87399999999969	15.9441723613486\\
4.87599999999969	15.9411971091284\\
4.87799999999968	15.9382220792803\\
4.87999999999968	15.9352472722653\\
4.88199999999968	15.9322726885453\\
4.88399999999968	15.9292983285829\\
4.88599999999968	15.9263241928417\\
4.88799999999968	15.9233502817858\\
4.88999999999968	15.9203765958803\\
4.89199999999968	15.917403135591\\
4.89399999999968	15.9144299013846\\
4.89599999999968	15.9114568937279\\
4.89799999999968	15.9084841130893\\
4.89999999999968	15.9055115599371\\
4.90199999999968	15.9025392347408\\
4.90399999999968	15.8995671379703\\
4.90599999999968	15.896595270096\\
4.90799999999968	15.8936236315894\\
4.90999999999968	15.8906522229221\\
4.91199999999968	15.8876810445667\\
4.91399999999968	15.8847100969959\\
4.91599999999968	15.8817393806833\\
4.91799999999968	15.8787688961033\\
4.91999999999968	15.87579864373\\
4.92199999999968	15.872828624039\\
4.92399999999968	15.8698588375056\\
4.92599999999968	15.8668892846059\\
4.92799999999968	15.8639199658167\\
4.92999999999968	15.8609508816149\\
4.93199999999968	15.857982032478\\
4.93399999999968	15.8550134188839\\
4.93599999999968	15.8520450413109\\
4.93799999999968	15.8490769002376\\
4.93999999999968	15.8461089961433\\
4.94199999999968	15.8431413295074\\
4.94399999999968	15.8401739008099\\
4.94599999999968	15.8372067105306\\
4.94799999999968	15.8342397591505\\
4.94999999999968	15.8312730471501\\
4.95199999999968	15.8283065750107\\
4.95399999999968	15.8253403432138\\
4.95599999999968	15.8223743522409\\
4.95799999999968	15.8194086025744\\
4.95999999999968	15.8164430946962\\
4.96199999999968	15.8134778290889\\
4.96399999999968	15.8105128062355\\
4.96599999999968	15.8075480266188\\
4.96799999999967	15.804583490722\\
4.96999999999967	15.8016191990284\\
4.97199999999967	15.7986551520216\\
4.97399999999967	15.7956913501855\\
4.97599999999967	15.7927277940041\\
4.97799999999967	15.7897644839611\\
4.97999999999967	15.7868014205411\\
4.98199999999967	15.7838386042283\\
4.98399999999967	15.7808760355073\\
4.98599999999967	15.7779137148625\\
4.98799999999967	15.7749516427787\\
4.98999999999967	15.7719898197407\\
4.99199999999967	15.7690282462334\\
4.99399999999967	15.7660669227417\\
4.99599999999967	15.7631058497505\\
4.99799999999967	15.760145027745\\
4.99999999999967	15.7571844572102\\
5.00199999999967	15.7542241386312\\
5.00399999999967	15.7512640724933\\
5.00599999999967	15.7483042592816\\
5.00799999999967	15.745344699481\\
5.00999999999967	15.742385393577\\
5.01199999999967	15.7394263420545\\
5.01399999999967	15.7364675453987\\
5.01599999999967	15.7335090040947\\
5.01799999999967	15.7305507186273\\
5.01999999999967	15.7275926894817\\
5.02199999999967	15.7246349171427\\
5.02399999999967	15.7216774020951\\
5.02599999999967	15.7187201448237\\
5.02799999999967	15.7157631458132\\
5.02999999999967	15.7128064055482\\
5.03199999999967	15.709849924513\\
5.03399999999967	15.7068937031919\\
5.03599999999967	15.7039377420693\\
5.03799999999967	15.7009820416294\\
5.03999999999967	15.6980266023559\\
5.04199999999967	15.6950714247326\\
5.04399999999967	15.6921165092433\\
5.04599999999967	15.6891618563714\\
5.04799999999967	15.6862074666002\\
5.04999999999967	15.6832533404129\\
5.05199999999967	15.6802994782925\\
5.05399999999967	15.6773458807215\\
5.05599999999967	15.6743925481827\\
5.05799999999966	15.6714394811583\\
5.05999999999966	15.6684866801307\\
5.06199999999966	15.6655341455814\\
5.06399999999966	15.6625818779924\\
5.06599999999966	15.659629877845\\
5.06799999999966	15.6566781456204\\
5.06999999999966	15.6537266817996\\
5.07199999999966	15.6507754868633\\
5.07399999999966	15.6478245612918\\
5.07599999999966	15.6448739055653\\
5.07799999999966	15.6419235201638\\
5.07999999999966	15.6389734055667\\
5.08199999999966	15.6360235622535\\
5.08399999999966	15.633073990703\\
5.08599999999966	15.630124691394\\
5.08799999999966	15.6271756648047\\
5.08999999999966	15.6242269114136\\
5.09199999999966	15.621278431698\\
5.09399999999966	15.6183302261355\\
5.09599999999966	15.6153822952032\\
5.09799999999966	15.6124346393777\\
5.09999999999966	15.6094872591356\\
5.10199999999966	15.6065401549528\\
5.10399999999966	15.603593327305\\
5.10599999999966	15.6006467766676\\
5.10799999999966	15.5977005035154\\
5.10999999999966	15.5947545083231\\
5.11199999999966	15.5918087915648\\
5.11399999999966	15.5888633537145\\
5.11599999999966	15.5859181952452\\
5.11799999999966	15.5829733166302\\
5.11999999999966	15.5800287183423\\
5.12199999999966	15.5770844008532\\
5.12399999999966	15.5741403646351\\
5.12599999999966	15.5711966101593\\
5.12799999999966	15.5682531378966\\
5.12999999999966	15.5653099483176\\
5.13199999999966	15.5623670418925\\
5.13399999999966	15.5594244190909\\
5.13599999999966	15.5564820803817\\
5.13799999999966	15.5535400262341\\
5.13999999999966	15.5505982571162\\
5.14199999999966	15.5476567734959\\
5.14399999999966	15.5447155758404\\
5.14599999999966	15.541774664617\\
5.14799999999966	15.5388340402917\\
5.14999999999965	15.5358937033309\\
5.15199999999965	15.5329536541999\\
5.15399999999965	15.5300138933635\\
5.15599999999965	15.5270744212865\\
5.15799999999965	15.524135238433\\
5.15999999999965	15.5211963452661\\
5.16199999999965	15.5182577422494\\
5.16399999999965	15.515319429845\\
5.16599999999965	15.5123814085151\\
5.16799999999965	15.5094436787211\\
5.16999999999965	15.5065062409241\\
5.17199999999965	15.5035690955844\\
5.17399999999965	15.5006322431621\\
5.17599999999965	15.4976956841166\\
5.17799999999965	15.4947594189066\\
5.17999999999965	15.4918234479907\\
5.18199999999965	15.4888877718265\\
5.18399999999965	15.4859523908712\\
5.18599999999965	15.4830173055817\\
5.18799999999965	15.480082516414\\
5.18999999999965	15.4771480238239\\
5.19199999999965	15.4742138282662\\
5.19399999999965	15.4712799301955\\
5.19599999999965	15.4683463300657\\
5.19799999999965	15.4654130283301\\
5.19999999999965	15.4624800254416\\
5.20199999999965	15.4595473218522\\
5.20399999999965	15.4566149180137\\
5.20599999999965	15.453682814377\\
5.20799999999965	15.4507510113926\\
5.20999999999965	15.4478195095105\\
5.21199999999965	15.4448883091798\\
5.21399999999965	15.4419574108492\\
5.21599999999965	15.4390268149668\\
5.21799999999965	15.4360965219802\\
5.21999999999965	15.433166532336\\
5.22199999999965	15.4302368464807\\
5.22399999999965	15.42730746486\\
5.22599999999965	15.4243783879187\\
5.22799999999965	15.4214496161014\\
5.22999999999965	15.418521149852\\
5.23199999999965	15.4155929896135\\
5.23399999999965	15.4126651358288\\
5.23599999999965	15.4097375889394\\
5.23799999999965	15.4068103493869\\
5.23999999999964	15.4038834176119\\
5.24199999999964	15.4009567940546\\
5.24399999999964	15.3980304791542\\
5.24599999999964	15.3951044733497\\
5.24799999999964	15.3921787770791\\
5.24999999999964	15.38925339078\\
5.25199999999964	15.3863283148892\\
5.25399999999964	15.383403549843\\
5.25599999999964	15.3804790960768\\
5.25799999999964	15.3775549540255\\
5.25999999999964	15.3746311241236\\
5.26199999999964	15.3717076068044\\
5.26399999999964	15.3687844025009\\
5.26599999999964	15.3658615116456\\
5.26799999999964	15.36293893467\\
5.26999999999964	15.3600166720048\\
5.27199999999964	15.3570947240806\\
5.27399999999964	15.3541730913269\\
5.27599999999964	15.3512517741726\\
5.27799999999964	15.348330773046\\
5.27999999999964	15.3454100883745\\
5.28199999999964	15.3424897205855\\
5.28399999999964	15.3395696701047\\
5.28599999999964	15.336649937358\\
5.28799999999964	15.3337305227701\\
5.28999999999964	15.3308114267652\\
5.29199999999964	15.3278926497668\\
5.29399999999964	15.3249741921979\\
5.29599999999964	15.3220560544804\\
5.29799999999964	15.3191382370357\\
5.29999999999964	15.3162207402847\\
5.30199999999964	15.3133035646474\\
5.30399999999964	15.3103867105431\\
5.30599999999964	15.3074701783904\\
5.30799999999964	15.3045539686073\\
5.30999999999964	15.3016380816111\\
5.31199999999964	15.2987225178184\\
5.31399999999964	15.2958072776448\\
5.31599999999964	15.2928923615056\\
5.31799999999964	15.2899777698152\\
5.31999999999964	15.2870635029874\\
5.32199999999964	15.2841495614349\\
5.32399999999964	15.2812359455704\\
5.32599999999964	15.2783226558053\\
5.32799999999964	15.2754096925505\\
5.32999999999964	15.272497056216\\
5.33199999999963	15.2695847472115\\
5.33399999999963	15.2666727659456\\
5.33599999999963	15.2637611128262\\
5.33799999999963	15.2608497882608\\
5.33999999999963	15.2579387926558\\
5.34199999999963	15.255028126417\\
5.34399999999963	15.2521177899497\\
5.34599999999963	15.2492077836581\\
5.34799999999963	15.2462981079459\\
5.34999999999963	15.2433887632163\\
5.35199999999963	15.2404797498711\\
5.35399999999963	15.2375710683121\\
5.35599999999963	15.23466271894\\
5.35799999999963	15.2317547021546\\
5.35999999999963	15.2288470183554\\
5.36199999999963	15.225939667941\\
5.36399999999963	15.223032651309\\
5.36599999999963	15.2201259688567\\
5.36799999999963	15.2172196209803\\
5.36999999999963	15.2143136080756\\
5.37199999999963	15.2114079305372\\
5.37399999999963	15.2085025887595\\
5.37599999999963	15.2055975831358\\
5.37799999999963	15.2026929140588\\
5.37999999999963	15.1997885819204\\
5.38199999999963	15.1968845871118\\
5.38399999999963	15.1939809300234\\
5.38599999999963	15.1910776110448\\
5.38799999999963	15.1881746305653\\
5.38999999999963	15.1852719889727\\
5.39199999999963	15.1823696866547\\
5.39399999999963	15.179467723998\\
5.39599999999963	15.1765661013886\\
5.39799999999963	15.1736648192115\\
5.39999999999963	15.1707638778515\\
5.40199999999963	15.1678632776922\\
5.40399999999963	15.1649630191165\\
5.40599999999963	15.1620631025066\\
5.40799999999963	15.1591635282443\\
5.40999999999963	15.15626429671\\
5.41199999999963	15.1533654082838\\
5.41399999999963	15.1504668633452\\
5.41599999999963	15.1475686622722\\
5.41799999999963	15.1446708054429\\
5.41999999999963	15.1417732932342\\
5.42199999999962	15.1388761260222\\
5.42399999999962	15.1359793041825\\
5.42599999999962	15.1330828280898\\
5.42799999999962	15.1301866981181\\
5.42999999999962	15.1272909146406\\
5.43199999999962	15.1243954780298\\
5.43399999999962	15.1215003886573\\
5.43599999999962	15.1186056468941\\
5.43799999999962	15.1157112531104\\
5.43999999999962	15.1128172076756\\
5.44199999999962	15.1099235109586\\
5.44399999999962	15.1070301633271\\
5.44599999999962	15.1041371651484\\
5.44799999999962	15.1012445167889\\
5.44999999999962	15.0983522186142\\
5.45199999999962	15.0954602709892\\
5.45399999999962	15.0925686742781\\
5.45599999999962	15.0896774288443\\
5.45799999999962	15.0867865350505\\
5.45999999999962	15.0838959932585\\
5.46199999999962	15.0810058038293\\
5.46399999999962	15.0781159671235\\
5.46599999999962	15.0752264835006\\
5.46799999999962	15.0723373533194\\
5.46999999999962	15.069448576938\\
5.47199999999962	15.0665601547139\\
5.47399999999962	15.0636720870034\\
5.47599999999962	15.0607843741625\\
5.47799999999962	15.0578970165463\\
5.47999999999962	15.0550100145089\\
5.48199999999962	15.052123368404\\
5.48399999999962	15.0492370785843\\
5.48599999999962	15.046351145402\\
5.48799999999962	15.0434655692082\\
5.48999999999962	15.0405803503534\\
5.49199999999962	15.0376954891875\\
5.49399999999962	15.0348109860593\\
5.49599999999962	15.0319268413173\\
5.49799999999962	15.0290430553087\\
5.49999999999962	15.0261596283803\\
5.50199999999962	15.0232765608783\\
5.50399999999962	15.0203938531475\\
5.50599999999962	15.0175115055327\\
5.50799999999962	15.0146295183775\\
5.50999999999962	15.0117478920248\\
5.51199999999961	15.0088666268169\\
5.51399999999961	15.0059857230949\\
5.51599999999961	15.0031051811999\\
5.51799999999961	15.0002250014715\\
5.51999999999961	14.9973451842491\\
5.52199999999961	14.9944657298707\\
5.52399999999961	14.9915866386744\\
5.52599999999961	14.9887079109969\\
5.52799999999961	14.9858295471742\\
5.52999999999961	14.9829515475419\\
5.53199999999961	14.9800739124348\\
5.53399999999961	14.9771966421863\\
5.53599999999961	14.9743197371298\\
5.53799999999961	14.9714431975974\\
5.53999999999961	14.9685670239213\\
5.54199999999961	14.9656912164317\\
5.54399999999961	14.9628157754591\\
5.54599999999961	14.9599407013329\\
5.54799999999961	14.9570659943813\\
5.54999999999961	14.9541916549326\\
5.55199999999961	14.9513176833136\\
5.55399999999961	14.948444079851\\
5.55599999999961	14.9455708448699\\
5.55799999999961	14.9426979786956\\
5.55999999999961	14.939825481652\\
5.56199999999961	14.9369533540624\\
5.56399999999961	14.9340815962495\\
5.56599999999961	14.9312102085351\\
5.56799999999961	14.9283391912404\\
5.56999999999961	14.9254685446858\\
5.57199999999961	14.9225982691906\\
5.57399999999961	14.919728365074\\
5.57599999999961	14.916858832654\\
5.57799999999961	14.9139896722479\\
5.57999999999961	14.9111208841724\\
5.58199999999961	14.9082524687435\\
5.58399999999961	14.9053844262761\\
5.58599999999961	14.9025167570847\\
5.58799999999961	14.8996494614831\\
5.58999999999961	14.8967825397839\\
5.59199999999961	14.8939159922995\\
5.59399999999961	14.8910498193413\\
5.59599999999961	14.88818402122\\
5.59799999999961	14.8853185982453\\
5.59999999999961	14.8824535507268\\
5.60199999999961	14.8795888789726\\
5.6039999999996	14.8767245832907\\
5.6059999999996	14.8738606639879\\
5.6079999999996	14.8709971213705\\
5.6099999999996	14.868133955744\\
5.6119999999996	14.8652711674132\\
5.6139999999996	14.8624087566821\\
5.6159999999996	14.8595467238541\\
5.6179999999996	14.8566850692317\\
5.6199999999996	14.8538237931166\\
5.6219999999996	14.8509628958102\\
5.6239999999996	14.8481023776124\\
5.6259999999996	14.8452422388235\\
5.6279999999996	14.8423824797417\\
5.6299999999996	14.8395231006655\\
5.6319999999996	14.8366641018923\\
5.6339999999996	14.833805483719\\
5.6359999999996	14.8309472464413\\
5.6379999999996	14.8280893903545\\
5.6399999999996	14.8252319157532\\
5.6419999999996	14.8223748229311\\
5.6439999999996	14.8195181121814\\
5.6459999999996	14.8166617837963\\
5.6479999999996	14.8138058380675\\
5.6499999999996	14.8109502752858\\
5.6519999999996	14.8080950957415\\
5.6539999999996	14.805240299724\\
5.6559999999996	14.8023858875218\\
5.6579999999996	14.7995318594232\\
5.6599999999996	14.7966782157154\\
5.6619999999996	14.7938249566848\\
5.6639999999996	14.7909720826173\\
5.6659999999996	14.7881195937982\\
5.6679999999996	14.7852674905117\\
5.6699999999996	14.7824157730414\\
5.6719999999996	14.7795644416706\\
5.6739999999996	14.776713496681\\
5.6759999999996	14.7738629383547\\
5.6779999999996	14.7710127669723\\
5.6799999999996	14.7681629828138\\
5.6819999999996	14.7653135861587\\
5.6839999999996	14.7624645772856\\
5.6859999999996	14.7596159564725\\
5.6879999999996	14.7567677239967\\
5.6899999999996	14.7539198801347\\
5.6919999999996	14.7510724251622\\
5.69399999999959	14.7482253593546\\
5.69599999999959	14.7453786829862\\
5.69799999999959	14.7425323963306\\
5.69999999999959	14.7396864996608\\
5.70199999999959	14.7368409932493\\
5.70399999999959	14.7339958773675\\
5.70599999999959	14.7311511522864\\
5.70799999999959	14.7283068182761\\
5.70999999999959	14.7254628756062\\
5.71199999999959	14.7226193245453\\
5.71399999999959	14.7197761653617\\
5.71599999999959	14.7169333983227\\
5.71799999999959	14.7140910236949\\
5.71999999999959	14.7112490417443\\
5.72199999999959	14.7084074527361\\
5.72399999999959	14.7055662569353\\
5.72599999999959	14.7027254546053\\
5.72799999999959	14.6998850460097\\
5.72999999999959	14.6970450314106\\
5.73199999999959	14.6942054110702\\
5.73399999999959	14.6913661852493\\
5.73599999999959	14.6885273542086\\
5.73799999999959	14.6856889182076\\
5.73999999999959	14.6828508775057\\
5.74199999999959	14.6800132323608\\
5.74399999999959	14.6771759830309\\
5.74599999999959	14.6743391297728\\
5.74799999999959	14.6715026728431\\
5.74999999999959	14.6686666124971\\
5.75199999999959	14.6658309489897\\
5.75399999999959	14.6629956825754\\
5.75599999999959	14.6601608135076\\
5.75799999999959	14.6573263420393\\
5.75999999999959	14.6544922684227\\
5.76199999999959	14.6516585929091\\
5.76399999999959	14.6488253157494\\
5.76599999999959	14.645992437194\\
5.76799999999959	14.6431599574922\\
5.76999999999959	14.6403278768929\\
5.77199999999959	14.6374961956441\\
5.77399999999959	14.6346649139933\\
5.77599999999959	14.6318340321875\\
5.77799999999959	14.6290035504725\\
5.77999999999959	14.6261734690941\\
5.78199999999959	14.6233437882965\\
5.78399999999959	14.6205145083244\\
5.78599999999958	14.617685629421\\
5.78799999999958	14.6148571518289\\
5.78999999999958	14.6120290757905\\
5.79199999999958	14.6092014015469\\
5.79399999999958	14.6063741293392\\
5.79599999999958	14.6035472594073\\
5.79799999999958	14.6007207919906\\
5.79999999999958	14.5978947273281\\
5.80199999999958	14.5950690656577\\
5.80399999999958	14.5922438072169\\
5.80599999999958	14.5894189522425\\
5.80799999999958	14.5865945009706\\
5.80999999999958	14.5837704536367\\
5.81199999999958	14.5809468104757\\
5.81399999999958	14.5781235717216\\
5.81599999999958	14.5753007376081\\
5.81799999999958	14.5724783083679\\
5.81999999999958	14.5696562842333\\
5.82199999999958	14.5668346654358\\
5.82399999999958	14.5640134522062\\
5.82599999999958	14.5611926447749\\
5.82799999999958	14.5583722433714\\
5.82999999999958	14.5555522482247\\
5.83199999999958	14.5527326595631\\
5.83399999999958	14.5499134776142\\
5.83599999999958	14.5470947026051\\
5.83799999999958	14.5442763347621\\
5.83999999999958	14.5414583743109\\
5.84199999999958	14.5386408214766\\
5.84399999999958	14.5358236764837\\
5.84599999999958	14.5330069395559\\
5.84799999999958	14.5301906109164\\
5.84999999999958	14.5273746907878\\
5.85199999999958	14.5245591793918\\
5.85399999999958	14.5217440769496\\
5.85599999999958	14.5189293836821\\
5.85799999999958	14.5161150998091\\
5.85999999999958	14.5133012255498\\
5.86199999999958	14.5104877611233\\
5.86399999999958	14.5076747067472\\
5.86599999999958	14.5048620626394\\
5.86799999999958	14.5020498290163\\
5.86999999999958	14.4992380060943\\
5.87199999999958	14.496426594089\\
5.87399999999958	14.4936155932152\\
5.87599999999957	14.4908050036872\\
5.87799999999957	14.4879948257188\\
5.87999999999957	14.4851850595229\\
5.88199999999957	14.4823757053121\\
5.88399999999957	14.4795667632981\\
5.88599999999957	14.476758233692\\
5.88799999999957	14.4739501167046\\
5.88999999999957	14.4711424125457\\
5.89199999999957	14.4683351214247\\
5.89399999999957	14.4655282435501\\
5.89599999999957	14.4627217791302\\
5.89799999999957	14.4599157283726\\
5.89999999999957	14.4571100914839\\
5.90199999999957	14.4543048686705\\
5.90399999999957	14.4515000601381\\
5.90599999999957	14.4486956660915\\
5.90799999999957	14.4458916867353\\
5.90999999999957	14.4430881222732\\
5.91199999999957	14.4402849729085\\
5.91399999999957	14.4374822388438\\
5.91599999999957	14.4346799202813\\
5.91799999999957	14.4318780174219\\
5.91999999999957	14.4290765304665\\
5.92199999999957	14.4262754596157\\
5.92399999999957	14.4234748050687\\
5.92599999999957	14.4206745670246\\
5.92799999999957	14.4178747456817\\
5.92999999999957	14.4150753412378\\
5.93199999999957	14.41227635389\\
5.93399999999957	14.4094777838351\\
5.93599999999957	14.4066796312689\\
5.93799999999957	14.4038818963868\\
5.93999999999957	14.4010845793837\\
5.94199999999957	14.3982876804538\\
5.94399999999957	14.3954911997906\\
5.94599999999957	14.3926951375872\\
5.94799999999957	14.3898994940359\\
5.94999999999957	14.3871042693288\\
5.95199999999957	14.3843094636569\\
5.95399999999957	14.3815150772109\\
5.95599999999957	14.378721110181\\
5.95799999999957	14.3759275627567\\
5.95999999999957	14.3731344351268\\
5.96199999999957	14.3703417274796\\
5.96399999999957	14.3675494400029\\
5.96599999999956	14.3647575728839\\
5.96799999999956	14.3619661263092\\
5.96999999999956	14.3591751004648\\
5.97199999999956	14.3563844955361\\
5.97399999999956	14.3535943117079\\
5.97599999999956	14.3508045491644\\
5.97799999999956	14.3480152080896\\
5.97999999999956	14.3452262886662\\
5.98199999999956	14.3424377910772\\
5.98399999999956	14.3396497155044\\
5.98599999999956	14.336862062129\\
5.98799999999956	14.3340748311321\\
5.98999999999956	14.3312880226939\\
5.99199999999956	14.3285016369939\\
5.99399999999956	14.3257156742115\\
5.99599999999956	14.3229301345255\\
5.99799999999956	14.3201450181132\\
5.99999999999956	14.3173603251526\\
6.00199999999956	14.3145760558203\\
6.00399999999956	14.3117922102928\\
6.00599999999956	14.3090087887458\\
6.00799999999956	14.3062257913544\\
6.00999999999956	14.3034432182933\\
6.01199999999956	14.3006610697366\\
6.01399999999956	14.2978793458579\\
6.01599999999956	14.29509804683\\
6.01799999999956	14.2923171728253\\
6.01999999999956	14.2895367240157\\
6.02199999999956	14.2867567005727\\
6.02399999999956	14.2839771026668\\
6.02599999999956	14.2811979304681\\
6.02799999999956	14.2784191841467\\
6.02999999999956	14.2756408638713\\
6.03199999999956	14.2728629698106\\
6.03399999999956	14.2700855021325\\
6.03599999999956	14.2673084610047\\
6.03799999999956	14.264531846594\\
6.03999999999956	14.2617556590665\\
6.04199999999956	14.2589798985885\\
6.04399999999956	14.2562045653248\\
6.04599999999956	14.2534296594405\\
6.04799999999956	14.2506551810998\\
6.04999999999956	14.2478811304662\\
6.05199999999956	14.2451075077028\\
6.05399999999956	14.2423343129723\\
6.05599999999956	14.2395615464368\\
6.05799999999955	14.2367892082578\\
6.05999999999955	14.2340172985962\\
6.06199999999955	14.2312458176125\\
6.06399999999955	14.2284747654665\\
6.06599999999955	14.2257041423178\\
6.06799999999955	14.2229339483251\\
6.06999999999955	14.2201641836469\\
6.07199999999955	14.2173948484408\\
6.07399999999955	14.2146259428643\\
6.07599999999955	14.2118574670739\\
6.07799999999955	14.209089421226\\
6.07999999999955	14.2063218054763\\
6.08199999999955	14.2035546199798\\
6.08399999999955	14.2007878648914\\
6.08599999999955	14.1980215403651\\
6.08799999999955	14.1952556465545\\
6.08999999999955	14.1924901836127\\
6.09199999999955	14.1897251516924\\
6.09399999999955	14.1869605509454\\
6.09599999999955	14.1841963815235\\
6.09799999999955	14.1814326435776\\
6.09999999999955	14.1786693372583\\
6.10199999999955	14.1759064627155\\
6.10399999999955	14.1731440200987\\
6.10599999999955	14.170382009557\\
6.10799999999955	14.1676204312387\\
6.10999999999955	14.1648592852918\\
6.11199999999955	14.1620985718639\\
6.11399999999955	14.1593382911018\\
6.11599999999955	14.1565784431519\\
6.11799999999955	14.1538190281602\\
6.11999999999955	14.1510600462721\\
6.12199999999955	14.1483014976326\\
6.12399999999955	14.145543382386\\
6.12599999999955	14.1427857006761\\
6.12799999999955	14.1400284526467\\
6.12999999999955	14.1372716384405\\
6.13199999999955	14.1345152581998\\
6.13399999999955	14.1317593120667\\
6.13599999999955	14.1290038001825\\
6.13799999999955	14.1262487226882\\
6.13999999999955	14.1234940797243\\
6.14199999999955	14.1207398714305\\
6.14399999999955	14.1179860979465\\
6.14599999999955	14.1152327594111\\
6.14799999999954	14.1124798559629\\
6.14999999999954	14.1097273877397\\
6.15199999999954	14.106975354879\\
6.15399999999954	14.1042237575181\\
6.15599999999954	14.1014725957931\\
6.15799999999954	14.0987218698403\\
6.15999999999954	14.0959715797951\\
6.16199999999954	14.0932217257925\\
6.16399999999954	14.0904723079672\\
6.16599999999954	14.0877233264532\\
6.16799999999954	14.0849747813843\\
6.16999999999954	14.0822266728933\\
6.17199999999954	14.0794790011131\\
6.17399999999954	14.0767317661757\\
6.17599999999954	14.0739849682129\\
6.17799999999954	14.0712386073559\\
6.17999999999954	14.0684926837354\\
6.18199999999954	14.0657471974817\\
6.18399999999954	14.0630021487245\\
6.18599999999954	14.0602575375932\\
6.18799999999954	14.0575133642166\\
6.18999999999954	14.0547696287231\\
6.19199999999954	14.0520263312406\\
6.19399999999954	14.0492834718967\\
6.19599999999954	14.0465410508178\\
6.19799999999954	14.043799068131\\
6.19999999999954	14.0410575239622\\
6.20199999999954	14.0383164184368\\
6.20399999999954	14.0355757516799\\
6.20599999999954	14.0328355238162\\
6.20799999999954	14.0300957349699\\
6.20999999999954	14.0273563852648\\
6.21199999999954	14.0246174748239\\
6.21399999999954	14.0218790037702\\
6.21599999999954	14.0191409722259\\
6.21799999999954	14.016403380313\\
6.21999999999954	14.0136662281529\\
6.22199999999954	14.0109295158664\\
6.22399999999954	14.0081932435742\\
6.22599999999954	14.0054574113962\\
6.22799999999954	14.0027220194522\\
6.22999999999954	13.9999870678611\\
6.23199999999954	13.9972525567418\\
6.23399999999954	13.9945184862123\\
6.23599999999954	13.9917848563907\\
6.23799999999954	13.9890516673941\\
6.23999999999953	13.9863189193396\\
6.24199999999953	13.9835866123434\\
6.24399999999953	13.9808547465215\\
6.24599999999953	13.9781233219897\\
6.24799999999953	13.9753923388631\\
6.24999999999953	13.9726617972562\\
6.25199999999953	13.9699316972834\\
6.25399999999953	13.9672020390582\\
6.25599999999953	13.9644728226942\\
6.25799999999953	13.9617440483043\\
6.25999999999953	13.9590157160008\\
6.26199999999953	13.9562878258959\\
6.26399999999953	13.9535603781011\\
6.26599999999953	13.9508333727275\\
6.26799999999953	13.9481068098859\\
6.26999999999953	13.9453806896866\\
6.27199999999953	13.9426550122394\\
6.27399999999953	13.9399297776539\\
6.27599999999953	13.9372049860388\\
6.27799999999953	13.9344806375028\\
6.27999999999953	13.931756732154\\
6.28199999999953	13.9290332701002\\
6.28399999999953	13.9263102514485\\
6.28599999999953	13.9235876763058\\
6.28799999999953	13.9208655447786\\
6.28999999999953	13.9181438569727\\
6.29199999999953	13.915422612994\\
6.29399999999953	13.9127018129473\\
6.29599999999953	13.9099814569374\\
6.29799999999953	13.9072615450689\\
6.29999999999953	13.9045420774451\\
6.30199999999953	13.9018230541699\\
6.30399999999953	13.8991044753463\\
6.30599999999953	13.8963863410767\\
6.30799999999953	13.8936686514635\\
6.30999999999953	13.8909514066084\\
6.31199999999953	13.8882346066127\\
6.31399999999953	13.8855182515776\\
6.31599999999953	13.8828023416033\\
6.31799999999953	13.8800868767902\\
6.31999999999953	13.8773718572378\\
6.32199999999953	13.8746572830455\\
6.32399999999953	13.8719431543123\\
6.32599999999953	13.8692294711365\\
6.32799999999953	13.8665162336163\\
6.32999999999952	13.8638034418492\\
6.33199999999952	13.8610910959327\\
6.33399999999952	13.8583791959633\\
6.33599999999952	13.8556677420378\\
6.33799999999952	13.8529567342521\\
6.33999999999952	13.8502461727018\\
6.34199999999952	13.8475360574821\\
6.34399999999952	13.8448263886879\\
6.34599999999952	13.8421171664136\\
6.34799999999952	13.8394083907533\\
6.34999999999952	13.8367000618005\\
6.35199999999952	13.8339921796485\\
6.35399999999952	13.8312847443901\\
6.35599999999952	13.8285777561179\\
6.35799999999952	13.8258712149236\\
6.35999999999952	13.8231651208992\\
6.36199999999952	13.8204594741357\\
6.36399999999952	13.817754274724\\
6.36599999999952	13.8150495227547\\
6.36799999999952	13.8123452183176\\
6.36999999999952	13.8096413615027\\
6.37199999999952	13.8069379523991\\
6.37399999999952	13.8042349910957\\
6.37599999999952	13.801532477681\\
6.37799999999952	13.7988304122433\\
6.37999999999952	13.7961287948701\\
6.38199999999952	13.7934276256488\\
6.38399999999952	13.7907269046665\\
6.38599999999952	13.7880266320096\\
6.38799999999952	13.7853268077643\\
6.38999999999952	13.7826274320167\\
6.39199999999952	13.7799285048518\\
6.39399999999952	13.777230026355\\
6.39599999999952	13.7745319966106\\
6.39799999999952	13.7718344157034\\
6.39999999999952	13.769137283717\\
6.40199999999952	13.7664406007348\\
6.40399999999952	13.7637443668401\\
6.40599999999952	13.7610485821157\\
6.40799999999952	13.758353246644\\
6.40999999999952	13.755658360507\\
6.41199999999952	13.7529639237863\\
6.41399999999952	13.7502699365633\\
6.41599999999952	13.7475763989188\\
6.41799999999952	13.7448833109333\\
6.41999999999951	13.7421906726871\\
6.42199999999951	13.7394984842599\\
6.42399999999951	13.7368067457312\\
6.42599999999951	13.7341154571799\\
6.42799999999951	13.7314246186849\\
6.42999999999951	13.7287342303243\\
6.43199999999951	13.7260442921763\\
6.43399999999951	13.7233548043183\\
6.43599999999951	13.7206657668275\\
6.43799999999951	13.717977179781\\
6.43999999999951	13.7152890432549\\
6.44199999999951	13.7126013573257\\
6.44399999999951	13.7099141220691\\
6.44599999999951	13.7072273375603\\
6.44799999999951	13.7045410038746\\
6.44999999999951	13.7018551210865\\
6.45199999999951	13.6991696892706\\
6.45399999999951	13.6964847085007\\
6.45599999999951	13.6938001788505\\
6.45799999999951	13.691116100393\\
6.45999999999951	13.6884324732015\\
6.46199999999951	13.6857492973483\\
6.46399999999951	13.6830665729057\\
6.46599999999951	13.6803842999456\\
6.46799999999951	13.6777024785393\\
6.46999999999951	13.6750211087582\\
6.47199999999951	13.6723401906729\\
6.47399999999951	13.6696597243539\\
6.47599999999951	13.6669797098714\\
6.47799999999951	13.664300147295\\
6.47999999999951	13.6616210366943\\
6.48199999999951	13.6589423781381\\
6.48399999999951	13.6562641716953\\
6.48599999999951	13.6535864174341\\
6.48799999999951	13.6509091154227\\
6.48999999999951	13.6482322657287\\
6.49199999999951	13.6455558684194\\
6.49399999999951	13.6428799235618\\
6.49599999999951	13.6402044312226\\
6.49799999999951	13.6375293914681\\
6.49999999999951	13.6348548043642\\
6.50199999999951	13.6321806699766\\
6.50399999999951	13.6295069883706\\
6.50599999999951	13.6268337596111\\
6.50799999999951	13.6241609837628\\
6.50999999999951	13.6214886608898\\
6.5119999999995	13.6188167910562\\
6.5139999999995	13.6161453743256\\
6.5159999999995	13.6134744107612\\
6.5179999999995	13.6108039004261\\
6.5199999999995	13.6081338433827\\
6.5219999999995	13.6054642396935\\
6.5239999999995	13.6027950894202\\
6.5259999999995	13.6001263926246\\
6.5279999999995	13.5974581493678\\
6.5299999999995	13.5947903597111\\
6.5319999999995	13.5921230237148\\
6.5339999999995	13.5894561414394\\
6.5359999999995	13.5867897129448\\
6.5379999999995	13.5841237382906\\
6.5399999999995	13.5814582175362\\
6.5419999999995	13.5787931507405\\
6.5439999999995	13.5761285379624\\
6.5459999999995	13.57346437926\\
6.5479999999995	13.5708006746915\\
6.5499999999995	13.5681374243145\\
6.5519999999995	13.5654746281864\\
6.5539999999995	13.5628122863644\\
6.5559999999995	13.560150398905\\
6.5579999999995	13.5574889658648\\
6.5599999999995	13.5548279873001\\
6.5619999999995	13.5521674632664\\
6.5639999999995	13.5495073938192\\
6.5659999999995	13.5468477790137\\
6.5679999999995	13.5441886189048\\
6.5699999999995	13.5415299135471\\
6.5719999999995	13.5388716629946\\
6.5739999999995	13.5362138673013\\
6.5759999999995	13.533556526521\\
6.5779999999995	13.5308996407066\\
6.5799999999995	13.5282432099114\\
6.5819999999995	13.5255872341879\\
6.5839999999995	13.5229317135885\\
6.5859999999995	13.520276648165\\
6.5879999999995	13.5176220379695\\
6.5899999999995	13.5149678830533\\
6.5919999999995	13.5123141834675\\
6.5939999999995	13.5096609392628\\
6.5959999999995	13.5070081504899\\
6.5979999999995	13.5043558171988\\
6.5999999999995	13.5017039394396\\
6.60199999999949	13.4990525172617\\
6.60399999999949	13.4964015507145\\
6.60599999999949	13.493751039847\\
6.60799999999949	13.4911009847078\\
6.60999999999949	13.4884513853454\\
6.61199999999949	13.4858022418078\\
6.61399999999949	13.4831535541428\\
6.61599999999949	13.4805053223978\\
6.61799999999949	13.4778575466202\\
6.61999999999949	13.4752102268567\\
6.62199999999949	13.4725633631541\\
6.62399999999949	13.4699169555583\\
6.62599999999949	13.4672710041158\\
6.62799999999949	13.464625508872\\
6.62999999999949	13.4619804698723\\
6.63199999999949	13.459335887162\\
6.63399999999949	13.4566917607858\\
6.63599999999949	13.4540480907883\\
6.63799999999949	13.4514048772137\\
6.63999999999949	13.4487621201059\\
6.64199999999949	13.4461198195087\\
6.64399999999949	13.4434779754654\\
6.64599999999949	13.440836588019\\
6.64799999999949	13.4381956572124\\
6.64999999999949	13.4355551830881\\
6.65199999999949	13.4329151656883\\
6.65399999999949	13.4302756050551\\
6.65599999999949	13.4276365012297\\
6.65799999999949	13.4249978542539\\
6.65999999999949	13.4223596641686\\
6.66199999999949	13.4197219310147\\
6.66399999999949	13.4170846548326\\
6.66599999999949	13.4144478356626\\
6.66799999999949	13.4118114735446\\
6.66999999999949	13.4091755685183\\
6.67199999999949	13.406540120623\\
6.67399999999949	13.403905129898\\
6.67599999999949	13.401270596382\\
6.67799999999949	13.3986365201134\\
6.67999999999949	13.3960029011309\\
6.68199999999949	13.393369739472\\
6.68399999999949	13.3907370351747\\
6.68599999999949	13.3881047882765\\
6.68799999999949	13.3854729988144\\
6.68999999999949	13.3828416668252\\
6.69199999999949	13.3802107923459\\
6.69399999999948	13.3775803754124\\
6.69599999999948	13.3749504160611\\
6.69799999999948	13.3723209143276\\
6.69999999999948	13.3696918702476\\
6.70199999999948	13.3670632838562\\
6.70399999999948	13.3644351551884\\
6.70599999999948	13.3618074842789\\
6.70799999999948	13.3591802711623\\
6.70999999999948	13.3565535158727\\
6.71199999999948	13.3539272184439\\
6.71399999999948	13.3513013789098\\
6.71599999999948	13.3486759973033\\
6.71799999999948	13.3460510736581\\
6.71999999999948	13.3434266080066\\
6.72199999999948	13.3408026003816\\
6.72399999999948	13.3381790508153\\
6.72599999999948	13.3355559593398\\
6.72799999999948	13.3329333259871\\
6.72999999999948	13.3303111507882\\
6.73199999999948	13.3276894337748\\
6.73399999999948	13.3250681749778\\
6.73599999999948	13.3224473744278\\
6.73799999999948	13.3198270321555\\
6.73999999999948	13.317207148191\\
6.74199999999948	13.3145877225642\\
6.74399999999948	13.311968755305\\
6.74599999999948	13.3093502464427\\
6.74799999999948	13.3067321960064\\
6.74999999999948	13.3041146040252\\
6.75199999999948	13.3014974705278\\
6.75399999999948	13.2988807955425\\
6.75599999999948	13.2962645790974\\
6.75799999999948	13.2936488212207\\
6.75999999999948	13.2910335219398\\
6.76199999999948	13.2884186812821\\
6.76399999999948	13.2858042992749\\
6.76599999999948	13.2831903759449\\
6.76799999999948	13.2805769113191\\
6.76999999999948	13.2779639054236\\
6.77199999999948	13.2753513582846\\
6.77399999999948	13.272739269928\\
6.77599999999948	13.2701276403795\\
6.77799999999948	13.2675164696644\\
6.77999999999948	13.2649057578081\\
6.78199999999948	13.2622955048351\\
6.78399999999947	13.2596857107705\\
6.78599999999947	13.2570763756384\\
6.78799999999947	13.2544674994631\\
6.78999999999947	13.2518590822685\\
6.79199999999947	13.2492511240782\\
6.79399999999947	13.2466436249158\\
6.79599999999947	13.2440365848044\\
6.79799999999947	13.2414300037668\\
6.79999999999947	13.2388238818259\\
6.80199999999947	13.2362182190041\\
6.80399999999947	13.2336130153237\\
6.80599999999947	13.2310082708066\\
6.80799999999947	13.2284039854745\\
6.80999999999947	13.225800159349\\
6.81199999999947	13.2231967924513\\
6.81399999999947	13.2205938848024\\
6.81599999999947	13.2179914364231\\
6.81799999999947	13.2153894473341\\
6.81999999999947	13.2127879175556\\
6.82199999999947	13.2101868471077\\
6.82399999999947	13.2075862360101\\
6.82599999999947	13.2049860842825\\
6.82799999999947	13.2023863919444\\
6.82999999999947	13.1997871590147\\
6.83199999999947	13.1971883855126\\
6.83399999999947	13.1945900714564\\
6.83599999999947	13.191992216865\\
6.83799999999947	13.1893948217562\\
6.83999999999947	13.1867978861481\\
6.84199999999947	13.1842014100586\\
6.84399999999947	13.181605393505\\
6.84599999999947	13.1790098365048\\
6.84799999999947	13.1764147390748\\
6.84999999999947	13.173820101232\\
6.85199999999947	13.1712259229931\\
6.85399999999947	13.1686322043742\\
6.85599999999947	13.1660389453915\\
6.85799999999947	13.1634461460611\\
6.85999999999947	13.1608538063986\\
6.86199999999947	13.1582619264195\\
6.86399999999947	13.1556705061389\\
6.86599999999947	13.153079545572\\
6.86799999999947	13.1504890447335\\
6.86999999999947	13.1478990036381\\
6.87199999999947	13.1453094223\\
6.87399999999946	13.1427203007333\\
6.87599999999946	13.140131638952\\
6.87799999999946	13.1375434369697\\
6.87999999999946	13.1349556948\\
6.88199999999946	13.1323684124561\\
6.88399999999946	13.1297815899509\\
6.88599999999946	13.1271952272974\\
6.88799999999946	13.124609324508\\
6.88999999999946	13.1220238815951\\
6.89199999999946	13.1194388985711\\
6.89399999999946	13.1168543754475\\
6.89599999999946	13.1142703122363\\
6.89799999999946	13.1116867089489\\
6.89999999999946	13.1091035655967\\
6.90199999999946	13.1065208821908\\
6.90399999999946	13.1039386587418\\
6.90599999999946	13.1013568952605\\
6.90799999999946	13.0987755917574\\
6.90999999999946	13.0961947482426\\
6.91199999999946	13.0936143647261\\
6.91399999999946	13.0910344412176\\
6.91599999999946	13.0884549777271\\
6.91799999999946	13.0858759742635\\
6.91999999999946	13.0832974308361\\
6.92199999999946	13.0807193474538\\
6.92399999999946	13.0781417241255\\
6.92599999999946	13.0755645608597\\
6.92799999999946	13.0729878576645\\
6.92999999999946	13.0704116145483\\
6.93199999999946	13.0678358315187\\
6.93399999999946	13.0652605085837\\
6.93599999999946	13.0626856457506\\
6.93799999999946	13.0601112430268\\
6.93999999999946	13.0575373004192\\
6.94199999999946	13.0549638179348\\
6.94399999999946	13.0523907955803\\
6.94599999999946	13.049818233362\\
6.94799999999946	13.0472461312863\\
6.94999999999946	13.0446744893592\\
6.95199999999946	13.0421033075866\\
6.95399999999946	13.0395325859742\\
6.95599999999946	13.0369623245273\\
6.95799999999946	13.0343925232512\\
6.95999999999946	13.0318231821511\\
6.96199999999946	13.0292543012315\\
6.96399999999946	13.0266858804974\\
6.96599999999945	13.0241179199531\\
6.96799999999945	13.0215504196028\\
6.96999999999945	13.0189833794505\\
6.97199999999945	13.0164167995003\\
6.97399999999945	13.0138506797557\\
6.97599999999945	13.0112850202201\\
6.97799999999945	13.0087198208968\\
6.97999999999945	13.006155081789\\
6.98199999999945	13.0035908028994\\
6.98399999999945	13.0010269842308\\
6.98599999999945	12.9984636257855\\
6.98799999999945	12.9959007275658\\
6.98999999999945	12.9933382895739\\
6.99199999999945	12.9907763118118\\
6.99399999999945	12.988214794281\\
6.99599999999945	12.9856537369831\\
6.99799999999945	12.9830931399193\\
6.99999999999945	12.9805330030909\\
7.00199999999945	12.9779733264988\\
7.00399999999945	12.9754141101436\\
7.00599999999945	12.972855354026\\
7.00799999999945	12.9702970581463\\
7.00999999999945	12.9677392225048\\
7.01199999999945	12.9651818471012\\
7.01399999999945	12.9626249319356\\
7.01599999999945	12.9600684770075\\
7.01799999999945	12.9575124823163\\
7.01999999999945	12.9549569478612\\
7.02199999999945	12.9524018736413\\
7.02399999999945	12.9498472596556\\
7.02599999999945	12.9472931059025\\
7.02799999999945	12.9447394123806\\
7.02999999999945	12.9421861790882\\
7.03199999999945	12.9396334060236\\
7.03399999999945	12.9370810931844\\
7.03599999999945	12.9345292405687\\
7.03799999999945	12.9319778481739\\
7.03999999999945	12.9294269159973\\
7.04199999999945	12.9268764440363\\
7.04399999999945	12.9243264322878\\
7.04599999999945	12.9217768807487\\
7.04799999999945	12.9192277894156\\
7.04999999999945	12.916679158285\\
7.05199999999945	12.9141309873532\\
7.05399999999945	12.9115832766163\\
7.05599999999944	12.9090360260704\\
7.05799999999944	12.9064892357111\\
7.05999999999944	12.903942905534\\
7.06199999999944	12.9013970355345\\
7.06399999999944	12.8988516257079\\
7.06599999999944	12.8963066760492\\
7.06799999999944	12.8937621865533\\
7.06999999999944	12.8912181572149\\
7.07199999999944	12.8886745880285\\
7.07399999999944	12.8861314789885\\
7.07599999999944	12.8835888300889\\
7.07799999999944	12.8810466413239\\
7.07999999999944	12.8785049126872\\
7.08199999999944	12.8759636441726\\
7.08399999999944	12.8734228357735\\
7.08599999999944	12.8708824874831\\
7.08799999999944	12.8683425992945\\
7.08999999999944	12.865803171201\\
7.09199999999944	12.863264203195\\
7.09399999999944	12.8607256952694\\
7.09599999999944	12.8581876474164\\
7.09799999999944	12.8556500596284\\
7.09999999999944	12.8531129318976\\
7.10199999999944	12.8505762642157\\
7.10399999999944	12.8480400565747\\
7.10599999999944	12.845504308966\\
7.10799999999944	12.8429690213812\\
7.10999999999944	12.8404341938115\\
7.11199999999944	12.8378998262479\\
7.11399999999944	12.8353659186814\\
7.11599999999944	12.8328324711028\\
7.11799999999944	12.8302994835025\\
7.11999999999944	12.8277669558713\\
7.12199999999944	12.8252348881991\\
7.12399999999944	12.8227032804762\\
7.12599999999944	12.8201721326924\\
7.12799999999944	12.8176414448375\\
7.12999999999944	12.8151112169012\\
7.13199999999944	12.8125814488729\\
7.13399999999944	12.8100521407417\\
7.13599999999944	12.8075232924969\\
7.13799999999944	12.8049949041275\\
7.13999999999944	12.8024669756221\\
7.14199999999944	12.7999395069694\\
7.14399999999944	12.797412498158\\
7.14599999999944	12.794885949176\\
7.14799999999943	12.7923598600115\\
7.14999999999943	12.7898342306527\\
7.15199999999943	12.7873090610874\\
7.15399999999943	12.7847843513032\\
7.15599999999943	12.7822601012874\\
7.15799999999943	12.7797363110278\\
7.15999999999943	12.7772129805111\\
7.16199999999943	12.7746901097248\\
7.16399999999943	12.7721676986554\\
7.16599999999943	12.7696457472898\\
7.16799999999943	12.7671242556143\\
7.16999999999943	12.7646032236157\\
7.17199999999943	12.7620826512801\\
7.17399999999943	12.7595625385934\\
7.17599999999943	12.7570428855418\\
7.17799999999943	12.7545236921109\\
7.17999999999943	12.7520049582864\\
7.18199999999943	12.7494866840537\\
7.18399999999943	12.7469688693983\\
7.18599999999943	12.744451514305\\
7.18799999999943	12.7419346187593\\
7.18999999999943	12.7394181827457\\
7.19199999999943	12.7369022062491\\
7.19399999999943	12.7343866892539\\
7.19599999999943	12.7318716317446\\
7.19799999999943	12.7293570337055\\
7.19999999999943	12.7268428951206\\
7.20199999999943	12.7243292159738\\
7.20399999999943	12.7218159962491\\
7.20599999999943	12.7193032359299\\
7.20799999999943	12.7167909350001\\
7.20999999999943	12.7142790934426\\
7.21199999999943	12.7117677112408\\
7.21399999999943	12.7092567883778\\
7.21599999999943	12.7067463248364\\
7.21799999999943	12.7042363205995\\
7.21999999999943	12.7017267756497\\
7.22199999999943	12.6992176899693\\
7.22399999999943	12.6967090635407\\
7.22599999999943	12.6942008963462\\
7.22799999999943	12.6916931883678\\
7.22999999999943	12.6891859395872\\
7.23199999999943	12.6866791499863\\
7.23399999999943	12.6841728195466\\
7.23599999999943	12.6816669482497\\
7.23799999999942	12.6791615360769\\
7.23999999999942	12.6766565830091\\
7.24199999999942	12.6741520890275\\
7.24399999999942	12.671648054113\\
7.24599999999942	12.6691444782462\\
7.24799999999942	12.6666413614079\\
7.24999999999942	12.6641387035785\\
7.25199999999942	12.6616365047382\\
7.25399999999942	12.6591347648672\\
7.25599999999942	12.6566334839455\\
7.25799999999942	12.6541326619532\\
7.25999999999942	12.6516322988697\\
7.26199999999942	12.6491323946749\\
7.26399999999942	12.6466329493481\\
7.26599999999942	12.6441339628686\\
7.26799999999942	12.6416354352158\\
7.26999999999942	12.6391373663685\\
7.27199999999942	12.6366397563058\\
7.27399999999942	12.6341426050064\\
7.27599999999942	12.6316459124488\\
7.27799999999942	12.6291496786118\\
7.27999999999942	12.6266539034736\\
7.28199999999942	12.6241585870123\\
7.28399999999942	12.6216637292061\\
7.28599999999942	12.6191693300331\\
7.28799999999942	12.616675389471\\
7.28999999999942	12.6141819074974\\
7.29199999999942	12.61168888409\\
7.29399999999942	12.609196319226\\
7.29599999999942	12.6067042128829\\
7.29799999999942	12.6042125650378\\
7.29999999999942	12.6017213756676\\
7.30199999999942	12.5992306447493\\
7.30399999999942	12.5967403722596\\
7.30599999999942	12.5942505581752\\
7.30799999999942	12.5917612024724\\
7.30999999999942	12.5892723051276\\
7.31199999999942	12.5867838661172\\
7.31399999999942	12.5842958854171\\
7.31599999999942	12.5818083630033\\
7.31799999999942	12.5793212988517\\
7.31999999999942	12.576834692938\\
7.32199999999942	12.5743485452376\\
7.32399999999942	12.571862855726\\
7.32599999999942	12.5693776243787\\
7.32799999999941	12.5668928511706\\
7.32999999999941	12.564408536077\\
7.33199999999941	12.5619246790726\\
7.33399999999941	12.5594412801324\\
7.33599999999941	12.556958339231\\
7.33799999999941	12.5544758563428\\
7.33999999999941	12.5519938314423\\
7.34199999999941	12.5495122645039\\
7.34399999999941	12.5470311555016\\
7.34599999999941	12.5445505044096\\
7.34799999999941	12.5420703112016\\
7.34999999999941	12.5395905758515\\
7.35199999999941	12.5371112983329\\
7.35399999999941	12.5346324786194\\
7.35599999999941	12.5321541166845\\
7.35799999999941	12.5296762125013\\
7.35999999999941	12.5271987660429\\
7.36199999999941	12.5247217772825\\
7.36399999999941	12.5222452461931\\
7.36599999999941	12.5197691727472\\
7.36799999999941	12.5172935569177\\
7.36999999999941	12.514818398677\\
7.37199999999941	12.5123436979976\\
7.37399999999941	12.5098694548519\\
7.37599999999941	12.5073956692119\\
7.37799999999941	12.5049223410496\\
7.37999999999941	12.5024494703372\\
7.38199999999941	12.4999770570463\\
7.38399999999941	12.4975051011486\\
7.38599999999941	12.4950336026159\\
7.38799999999941	12.4925625614194\\
7.38999999999941	12.4900919775305\\
7.39199999999941	12.4876218509205\\
7.39399999999941	12.4851521815604\\
7.39599999999941	12.4826829694212\\
7.39799999999941	12.4802142144739\\
7.39999999999941	12.477745916689\\
7.40199999999941	12.4752780760374\\
7.40399999999941	12.4728106924893\\
7.40599999999941	12.4703437660153\\
7.40799999999941	12.4678772965856\\
7.40999999999941	12.4654112841704\\
7.41199999999941	12.4629457287398\\
7.41399999999941	12.4604806302635\\
7.41599999999941	12.4580159887116\\
7.41799999999941	12.4555518040536\\
7.4199999999994	12.453088076259\\
7.4219999999994	12.4506248052975\\
7.4239999999994	12.4481619911384\\
7.4259999999994	12.4456996337507\\
7.4279999999994	12.4432377331038\\
7.4299999999994	12.4407762891666\\
7.4319999999994	12.438315301908\\
7.4339999999994	12.4358547712968\\
7.4359999999994	12.4333946973016\\
7.4379999999994	12.430935079891\\
7.4399999999994	12.4284759190336\\
7.4419999999994	12.4260172146974\\
7.4439999999994	12.4235589668509\\
7.4459999999994	12.421101175462\\
7.4479999999994	12.4186438404989\\
7.4499999999994	12.4161869619293\\
7.4519999999994	12.4137305397212\\
7.4539999999994	12.411274573842\\
7.4559999999994	12.4088190642594\\
7.4579999999994	12.406364010941\\
7.4599999999994	12.4039094138538\\
7.4619999999994	12.4014552729652\\
7.4639999999994	12.3990015882424\\
7.4659999999994	12.3965483596522\\
7.4679999999994	12.3940955871618\\
7.4699999999994	12.3916432707378\\
7.4719999999994	12.3891914103467\\
7.4739999999994	12.3867400059554\\
7.4759999999994	12.3842890575303\\
7.4779999999994	12.3818385650376\\
7.4799999999994	12.3793885284438\\
7.4819999999994	12.3769389477148\\
7.4839999999994	12.3744898228169\\
7.4859999999994	12.3720411537159\\
7.4879999999994	12.3695929403775\\
7.4899999999994	12.3671451827676\\
7.4919999999994	12.3646978808518\\
7.4939999999994	12.3622510345956\\
7.4959999999994	12.3598046439644\\
7.4979999999994	12.3573587089236\\
7.4999999999994	12.3549132294382\\
7.5019999999994	12.3524682054735\\
7.5039999999994	12.3500236369943\\
7.5059999999994	12.3475795239657\\
7.5079999999994	12.3451358663524\\
7.50999999999939	12.342692664119\\
7.51199999999939	12.3402499172303\\
7.51399999999939	12.3378076256504\\
7.51599999999939	12.3353657893442\\
7.51799999999939	12.3329244082756\\
7.51999999999939	12.3304834824088\\
7.52199999999939	12.328043011708\\
7.52399999999939	12.3256029961371\\
7.52599999999939	12.3231634356601\\
7.52799999999939	12.3207243302406\\
7.52999999999939	12.3182856798424\\
7.53199999999939	12.3158474844289\\
7.53399999999939	12.3134097439638\\
7.53599999999939	12.3109724584103\\
7.53799999999939	12.3085356277318\\
7.53999999999939	12.3060992518912\\
7.54199999999939	12.303663330852\\
7.54399999999939	12.3012278645768\\
7.54599999999939	12.2987928530288\\
7.54799999999939	12.2963582961705\\
7.54999999999939	12.2939241939647\\
7.55199999999939	12.2914905463741\\
7.55399999999939	12.2890573533609\\
7.55599999999939	12.2866246148878\\
7.55799999999939	12.2841923309169\\
7.55999999999939	12.2817605014104\\
7.56199999999939	12.2793291263306\\
7.56399999999939	12.2768982056392\\
7.56599999999939	12.2744677392984\\
7.56799999999939	12.2720377272697\\
7.56999999999939	12.2696081695152\\
7.57199999999939	12.2671790659962\\
7.57399999999939	12.2647504166744\\
7.57599999999939	12.2623222215113\\
7.57799999999939	12.259894480468\\
7.57999999999939	12.2574671935059\\
7.58199999999939	12.255040360586\\
7.58399999999939	12.2526139816696\\
7.58599999999939	12.2501880567176\\
7.58799999999939	12.2477625856907\\
7.58999999999939	12.2453375685498\\
7.59199999999939	12.2429130052557\\
7.59399999999939	12.2404888957688\\
7.59599999999939	12.2380652400495\\
7.59799999999939	12.2356420380586\\
7.59999999999939	12.2332192897561\\
7.60199999999938	12.2307969951024\\
7.60399999999938	12.2283751540575\\
7.60599999999938	12.2259537665815\\
7.60799999999938	12.2235328326343\\
7.60999999999938	12.221112352176\\
7.61199999999938	12.2186923251659\\
7.61399999999938	12.2162727515641\\
7.61599999999938	12.2138536313301\\
7.61799999999938	12.2114349644233\\
7.61999999999938	12.2090167508032\\
7.62199999999938	12.206598990429\\
7.62399999999938	12.2041816832601\\
7.62599999999938	12.2017648292555\\
7.62799999999938	12.1993484283744\\
7.62999999999938	12.1969324805756\\
7.63199999999938	12.1945169858181\\
7.63399999999938	12.1921019440606\\
7.63599999999938	12.189687355262\\
7.63799999999938	12.1872732193807\\
7.63999999999938	12.1848595363753\\
7.64199999999938	12.1824463062042\\
7.64399999999938	12.1800335288259\\
7.64599999999938	12.1776212041985\\
7.64799999999938	12.1752093322802\\
7.64999999999938	12.1727979130293\\
7.65199999999938	12.1703869464036\\
7.65399999999938	12.1679764323611\\
7.65599999999938	12.1655663708596\\
7.65799999999938	12.1631567618569\\
7.65999999999938	12.1607476053106\\
7.66199999999938	12.1583389011784\\
7.66399999999938	12.1559306494177\\
7.66599999999938	12.1535228499858\\
7.66799999999938	12.1511155028403\\
7.66999999999938	12.1487086079384\\
7.67199999999938	12.1463021652371\\
7.67399999999938	12.1438961746937\\
7.67599999999938	12.1414906362648\\
7.67799999999938	12.1390855499079\\
7.67999999999938	12.1366809155792\\
7.68199999999938	12.1342767332359\\
7.68399999999938	12.1318730028345\\
7.68599999999938	12.1294697243316\\
7.68799999999938	12.1270668976837\\
7.68999999999938	12.1246645228472\\
7.69199999999937	12.1222625997786\\
7.69399999999937	12.119861128434\\
7.69599999999937	12.1174601087694\\
7.69799999999937	12.1150595407412\\
7.69999999999937	12.1126594243054\\
7.70199999999937	12.1102597594177\\
7.70399999999937	12.1078605460341\\
7.70599999999937	12.1054617841105\\
7.70799999999937	12.1030634736023\\
7.70999999999937	12.1006656144653\\
7.71199999999937	12.098268206655\\
7.71399999999937	12.0958712501269\\
7.71599999999937	12.0934747448362\\
7.71799999999937	12.0910786907385\\
7.71999999999937	12.0886830877887\\
7.72199999999937	12.0862879359421\\
7.72399999999937	12.0838932351538\\
7.72599999999937	12.0814989853787\\
7.72799999999937	12.0791051865717\\
7.72999999999937	12.0767118386877\\
7.73199999999937	12.0743189416813\\
7.73399999999937	12.0719264955074\\
7.73599999999937	12.0695345001203\\
7.73799999999937	12.0671429554749\\
7.73999999999937	12.0647518615253\\
7.74199999999937	12.062361218226\\
7.74399999999937	12.0599710255314\\
7.74599999999937	12.0575812833954\\
7.74799999999937	12.0551919917725\\
7.74999999999937	12.0528031506164\\
7.75199999999937	12.0504147598814\\
7.75399999999937	12.0480268195213\\
7.75599999999937	12.0456393294898\\
7.75799999999937	12.0432522897409\\
7.75999999999937	12.0408657002282\\
7.76199999999937	12.0384795609052\\
7.76399999999937	12.0360938717255\\
7.76599999999937	12.0337086326427\\
7.76799999999937	12.0313238436099\\
7.76999999999937	12.0289395045807\\
7.77199999999937	12.0265556155082\\
7.77399999999937	12.0241721763455\\
7.77599999999937	12.0217891870458\\
7.77799999999937	12.0194066475622\\
7.77999999999937	12.0170245578475\\
7.78199999999936	12.0146429178545\\
7.78399999999936	12.0122617275363\\
7.78599999999936	12.0098809868454\\
7.78799999999936	12.0075006957345\\
7.78999999999936	12.0051208541562\\
7.79199999999936	12.0027414620632\\
7.79399999999936	12.0003625194076\\
7.79599999999936	11.9979840261419\\
7.79799999999936	11.9956059822185\\
7.79999999999936	11.9932283875897\\
7.80199999999936	11.9908512422074\\
7.80399999999936	11.9884745460238\\
7.80599999999936	11.986098298991\\
7.80799999999936	11.9837225010608\\
7.80999999999936	11.9813471521853\\
7.81199999999936	11.9789722523161\\
7.81399999999936	11.9765978014051\\
7.81599999999936	11.9742237994039\\
7.81799999999936	11.971850246264\\
7.81999999999936	11.969477141937\\
7.82199999999936	11.9671044863745\\
7.82399999999936	11.9647322795277\\
7.82599999999936	11.962360521348\\
7.82799999999936	11.9599892117867\\
7.82999999999936	11.957618350795\\
7.83199999999936	11.9552479383239\\
7.83399999999936	11.9528779743246\\
7.83599999999936	11.950508458748\\
7.83799999999936	11.9481393915449\\
7.83999999999936	11.9457707726665\\
7.84199999999936	11.9434026020632\\
7.84399999999936	11.9410348796859\\
7.84599999999936	11.9386676054853\\
7.84799999999936	11.9363007794119\\
7.84999999999936	11.9339344014162\\
7.85199999999936	11.9315684714487\\
7.85399999999936	11.9292029894598\\
7.85599999999936	11.9268379553997\\
7.85799999999936	11.9244733692188\\
7.85999999999936	11.9221092308673\\
7.86199999999936	11.9197455402951\\
7.86399999999936	11.9173822974525\\
7.86599999999936	11.9150195022894\\
7.86799999999936	11.9126571547558\\
7.86999999999936	11.9102952548014\\
7.87199999999936	11.9079338023761\\
7.87399999999935	11.9055727974296\\
7.87599999999935	11.9032122399117\\
7.87799999999935	11.9008521297718\\
7.87999999999935	11.8984924669595\\
7.88199999999935	11.8961332514244\\
7.88399999999935	11.8937744831159\\
7.88599999999935	11.8914161619832\\
7.88799999999935	11.8890582879756\\
7.88999999999935	11.8867008610425\\
7.89199999999935	11.8843438811329\\
7.89399999999935	11.8819873481961\\
7.89599999999935	11.8796312621809\\
7.89799999999935	11.8772756230363\\
7.89999999999935	11.8749204307114\\
7.90199999999935	11.8725656851549\\
7.90399999999935	11.8702113863156\\
7.90599999999935	11.8678575341424\\
7.90799999999935	11.8655041285836\\
7.90999999999935	11.8631511695881\\
7.91199999999935	11.8607986571044\\
7.91399999999935	11.858446591081\\
7.91599999999935	11.8560949714662\\
7.91799999999935	11.8537437982085\\
7.91999999999935	11.8513930712561\\
7.92199999999935	11.8490427905572\\
7.92399999999935	11.8466929560602\\
7.92599999999935	11.8443435677129\\
7.92799999999935	11.8419946254637\\
7.92999999999935	11.8396461292604\\
7.93199999999935	11.8372980790509\\
7.93399999999935	11.8349504747832\\
7.93599999999935	11.8326033164051\\
7.93799999999935	11.8302566038643\\
7.93999999999935	11.8279103371086\\
7.94199999999935	11.8255645160855\\
7.94399999999935	11.8232191407428\\
7.94599999999935	11.8208742110279\\
7.94799999999935	11.8185297268883\\
7.94999999999935	11.8161856882713\\
7.95199999999935	11.8138420951244\\
7.95399999999935	11.8114989473948\\
7.95599999999935	11.8091562450299\\
7.95799999999935	11.8068139879766\\
7.95999999999935	11.8044721761823\\
7.96199999999935	11.8021308095939\\
7.96399999999934	11.7997898881585\\
7.96599999999934	11.797449411823\\
7.96799999999934	11.7951093805344\\
7.96999999999934	11.7927697942393\\
7.97199999999934	11.7904306528847\\
7.97399999999934	11.7880919564173\\
7.97599999999934	11.7857537047838\\
7.97799999999934	11.7834158979307\\
7.97999999999934	11.7810785358045\\
7.98199999999934	11.7787416183519\\
7.98399999999934	11.7764051455192\\
7.98599999999934	11.7740691172529\\
7.98799999999934	11.7717335334994\\
7.98999999999934	11.7693983942047\\
7.99199999999934	11.7670636993152\\
7.99399999999934	11.7647294487772\\
7.99599999999934	11.7623956425367\\
7.99799999999934	11.7600622805396\\
7.99999999999934	11.7577293627322\\
};
\addplot [color=mycolor1, forget plot]
  table[row sep=crcr]{%
7.99999999999934	11.7577293627322\\
8.00199999999934	11.7553968890603\\
8.00399999999934	11.7530648594698\\
8.00599999999934	11.7507332739066\\
8.00799999999934	11.7484021323164\\
8.00999999999934	11.746071434645\\
8.01199999999934	11.7437411808382\\
8.01399999999935	11.7414113708414\\
8.01599999999935	11.7390820046003\\
8.01799999999935	11.7367530820606\\
8.01999999999935	11.7344246031674\\
8.02199999999935	11.7320965678665\\
8.02399999999935	11.7297689761029\\
8.02599999999935	11.7274418278222\\
8.02799999999935	11.7251151229696\\
8.02999999999935	11.7227888614903\\
8.03199999999935	11.7204630433295\\
8.03399999999935	11.7181376684322\\
8.03599999999935	11.7158127367435\\
8.03799999999935	11.7134882482084\\
8.03999999999935	11.7111642027719\\
8.04199999999935	11.7088406003789\\
8.04399999999936	11.7065174409742\\
8.04599999999936	11.7041947245027\\
8.04799999999936	11.7018724509089\\
8.04999999999936	11.6995506201379\\
8.05199999999936	11.697229232134\\
8.05399999999936	11.694908286842\\
8.05599999999936	11.6925877842062\\
8.05799999999936	11.6902677241714\\
8.05999999999936	11.6879481066819\\
8.06199999999936	11.6856289316821\\
8.06399999999936	11.6833101991164\\
8.06599999999936	11.680991908929\\
8.06799999999936	11.6786740610643\\
8.06999999999936	11.6763566554663\\
8.07199999999937	11.6740396920793\\
8.07399999999937	11.6717231708474\\
8.07599999999937	11.6694070917146\\
8.07799999999937	11.6670914546248\\
8.07999999999937	11.6647762595222\\
8.08199999999937	11.6624615063505\\
8.08399999999937	11.6601471950537\\
8.08599999999937	11.6578333255755\\
8.08799999999937	11.6555198978597\\
8.08999999999937	11.65320691185\\
8.09199999999937	11.6508943674901\\
8.09399999999937	11.6485822647236\\
8.09599999999937	11.6462706034941\\
8.09799999999937	11.643959383745\\
8.09999999999937	11.64164860542\\
8.10199999999938	11.6393382684623\\
8.10399999999938	11.6370283728154\\
8.10599999999938	11.6347189184226\\
8.10799999999938	11.6324099052273\\
8.10999999999938	11.6301013331725\\
8.11199999999938	11.6277932022016\\
8.11399999999938	11.6254855122577\\
8.11599999999938	11.6231782632839\\
8.11799999999938	11.6208714552232\\
8.11999999999938	11.6185650880186\\
8.12199999999938	11.6162591616133\\
8.12399999999938	11.6139536759499\\
8.12599999999938	11.6116486309715\\
8.12799999999938	11.6093440266207\\
8.12999999999938	11.6070398628404\\
8.13199999999939	11.6047361395734\\
8.13399999999939	11.6024328567624\\
8.13599999999939	11.60013001435\\
8.13799999999939	11.5978276122787\\
8.13999999999939	11.5955256504912\\
8.14199999999939	11.5932241289299\\
8.14399999999939	11.5909230475374\\
8.14599999999939	11.588622406256\\
8.14799999999939	11.5863222050281\\
8.14999999999939	11.5840224437961\\
8.15199999999939	11.5817231225022\\
8.15399999999939	11.5794242410888\\
8.15599999999939	11.5771257994979\\
8.15799999999939	11.5748277976718\\
8.15999999999939	11.5725302355526\\
8.1619999999994	11.5702331130823\\
8.1639999999994	11.567936430203\\
8.1659999999994	11.5656401868566\\
8.1679999999994	11.5633443829853\\
8.1699999999994	11.5610490185306\\
8.1719999999994	11.5587540934347\\
8.1739999999994	11.5564596076391\\
8.1759999999994	11.554165561086\\
8.1779999999994	11.5518719537167\\
8.1799999999994	11.5495787854731\\
8.1819999999994	11.5472860562969\\
8.1839999999994	11.5449937661296\\
8.1859999999994	11.5427019149128\\
8.1879999999994	11.5404105025879\\
8.1899999999994	11.5381195290964\\
8.19199999999941	11.5358289943799\\
8.19399999999941	11.5335388983799\\
8.19599999999941	11.5312492410374\\
8.19799999999941	11.5289600222938\\
8.19999999999941	11.5266712420906\\
8.20199999999941	11.5243829003689\\
8.20399999999941	11.5220949970697\\
8.20599999999941	11.5198075321345\\
8.20799999999941	11.517520505504\\
8.20999999999941	11.5152339171197\\
8.21199999999941	11.5129477669223\\
8.21399999999941	11.5106620548529\\
8.21599999999941	11.5083767808526\\
8.21799999999941	11.506091944862\\
8.21999999999941	11.5038075468223\\
8.22199999999942	11.5015235866741\\
8.22399999999942	11.4992400643582\\
8.22599999999942	11.4969569798153\\
8.22799999999942	11.4946743329863\\
8.22999999999942	11.4923921238118\\
8.23199999999942	11.4901103522324\\
8.23399999999942	11.4878290181887\\
8.23599999999942	11.4855481216211\\
8.23799999999942	11.4832676624704\\
8.23999999999942	11.4809876406769\\
8.24199999999942	11.4787080561811\\
8.24399999999942	11.4764289089233\\
8.24599999999942	11.4741501988439\\
8.24799999999942	11.4718719258833\\
8.24999999999942	11.4695940899817\\
8.25199999999943	11.4673166910795\\
8.25399999999943	11.4650397291166\\
8.25599999999943	11.4627632040334\\
8.25799999999943	11.4604871157699\\
8.25999999999943	11.4582114642664\\
8.26199999999943	11.4559362494627\\
8.26399999999943	11.453661471299\\
8.26599999999943	11.4513871297152\\
8.26799999999943	11.4491132246513\\
8.26999999999943	11.4468397560471\\
8.27199999999943	11.4445667238425\\
8.27399999999943	11.4422941279774\\
8.27599999999943	11.4400219683917\\
8.27799999999943	11.4377502450249\\
8.27999999999943	11.4354789578169\\
8.28199999999944	11.4332081067072\\
8.28399999999944	11.4309376916357\\
8.28599999999944	11.4286677125419\\
8.28799999999944	11.4263981693653\\
8.28999999999944	11.4241290620455\\
8.29199999999944	11.421860390522\\
8.29399999999944	11.4195921547343\\
8.29599999999944	11.4173243546218\\
8.29799999999944	11.4150569901239\\
8.29999999999944	11.41279006118\\
8.30199999999944	11.4105235677294\\
8.30399999999944	11.4082575097113\\
8.30599999999944	11.4059918870652\\
8.30799999999944	11.4037266997301\\
8.30999999999944	11.4014619476452\\
8.31199999999945	11.3991976307497\\
8.31399999999945	11.3969337489828\\
8.31599999999945	11.3946703022834\\
8.31799999999945	11.3924072905908\\
8.31999999999945	11.3901447138437\\
8.32199999999945	11.3878825719813\\
8.32399999999945	11.3856208649425\\
8.32599999999945	11.3833595926661\\
8.32799999999945	11.3810987550911\\
8.32999999999945	11.3788383521564\\
8.33199999999945	11.3765783838006\\
8.33399999999945	11.3743188499626\\
8.33599999999945	11.3720597505811\\
8.33799999999945	11.369801085595\\
8.33999999999945	11.3675428549426\\
8.34199999999946	11.3652850585629\\
8.34399999999946	11.3630276963943\\
8.34599999999946	11.3607707683754\\
8.34799999999946	11.3585142744449\\
8.34999999999946	11.356258214541\\
8.35199999999946	11.3540025886024\\
8.35399999999946	11.3517473965675\\
8.35599999999946	11.3494926383746\\
8.35799999999946	11.3472383139624\\
8.35999999999946	11.3449844232688\\
8.36199999999946	11.3427309662324\\
8.36399999999946	11.3404779427915\\
8.36599999999946	11.3382253528841\\
8.36799999999946	11.3359731964486\\
8.36999999999946	11.3337214734233\\
8.37199999999947	11.3314701837462\\
8.37399999999947	11.3292193273552\\
8.37599999999947	11.3269689041888\\
8.37799999999947	11.3247189141848\\
8.37999999999947	11.3224693572813\\
8.38199999999947	11.3202202334163\\
8.38399999999947	11.3179715425277\\
8.38599999999947	11.3157232845536\\
8.38799999999947	11.3134754594317\\
8.38999999999947	11.3112280670999\\
8.39199999999947	11.3089811074962\\
8.39399999999947	11.3067345805581\\
8.39599999999947	11.3044884862238\\
8.39799999999947	11.3022428244306\\
8.39999999999947	11.2999975951165\\
8.40199999999948	11.2977527982191\\
8.40399999999948	11.2955084336761\\
8.40599999999948	11.2932645014251\\
8.40799999999948	11.2910210014036\\
8.40999999999948	11.2887779335494\\
8.41199999999948	11.2865352977997\\
8.41399999999948	11.2842930940923\\
8.41599999999948	11.2820513223645\\
8.41799999999948	11.2798099825537\\
8.41999999999948	11.2775690745976\\
8.42199999999948	11.2753285984333\\
8.42399999999948	11.2730885539982\\
8.42599999999948	11.2708489412297\\
8.42799999999948	11.2686097600651\\
8.42999999999948	11.2663710104417\\
8.43199999999949	11.2641326922966\\
8.43399999999949	11.2618948055671\\
8.43599999999949	11.2596573501904\\
8.43799999999949	11.2574203261036\\
8.43999999999949	11.2551837332438\\
8.44199999999949	11.2529475715482\\
8.44399999999949	11.2507118409538\\
8.44599999999949	11.2484765413976\\
8.44799999999949	11.2462416728167\\
8.44999999999949	11.244007235148\\
8.45199999999949	11.2417732283285\\
8.45399999999949	11.2395396522951\\
8.45599999999949	11.2373065069846\\
8.45799999999949	11.2350737923341\\
8.45999999999949	11.2328415082802\\
8.4619999999995	11.2306096547599\\
8.4639999999995	11.2283782317099\\
8.4659999999995	11.2261472390671\\
8.4679999999995	11.2239166767679\\
8.4699999999995	11.2216865447493\\
8.4719999999995	11.2194568429477\\
8.4739999999995	11.2172275713001\\
8.4759999999995	11.2149987297429\\
8.4779999999995	11.2127703182128\\
8.4799999999995	11.2105423366461\\
8.4819999999995	11.2083147849798\\
8.4839999999995	11.20608766315\\
8.4859999999995	11.2038609710934\\
8.4879999999995	11.2016347087464\\
8.4899999999995	11.1994088760454\\
8.49199999999951	11.1971834729268\\
8.49399999999951	11.1949584993271\\
8.49599999999951	11.1927339551825\\
8.49799999999951	11.1905098404295\\
8.49999999999951	11.1882861550043\\
8.50199999999951	11.1860628988431\\
8.50399999999951	11.1838400718824\\
8.50599999999951	11.1816176740581\\
8.50799999999951	11.1793957053066\\
8.50999999999951	11.177174165564\\
8.51199999999951	11.1749530547665\\
8.51399999999951	11.1727323728502\\
8.51599999999951	11.1705121197512\\
8.51799999999951	11.1682922954056\\
8.51999999999951	11.1660728997493\\
8.52199999999952	11.1638539327184\\
8.52399999999952	11.1616353942491\\
8.52599999999952	11.1594172842771\\
8.52799999999952	11.1571996027383\\
8.52999999999952	11.1549823495689\\
8.53199999999952	11.1527655247046\\
8.53399999999952	11.1505491280814\\
8.53599999999952	11.148333159635\\
8.53799999999952	11.1461176193012\\
8.53999999999952	11.143902507016\\
8.54199999999952	11.1416878227151\\
8.54399999999952	11.1394735663341\\
8.54599999999952	11.1372597378089\\
8.54799999999952	11.135046337075\\
8.54999999999952	11.1328333640683\\
8.55199999999953	11.1306208187244\\
8.55399999999953	11.1284087009788\\
8.55599999999953	11.1261970107671\\
8.55799999999953	11.1239857480251\\
8.55999999999953	11.1217749126882\\
8.56199999999953	11.1195645046919\\
8.56399999999953	11.1173545239718\\
8.56599999999953	11.1151449704633\\
8.56799999999953	11.112935844102\\
8.56999999999953	11.1107271448232\\
8.57199999999953	11.1085188725624\\
8.57399999999953	11.1063110272548\\
8.57599999999953	11.104103608836\\
8.57799999999953	11.1018966172413\\
8.57999999999953	11.099690052406\\
8.58199999999954	11.0974839142654\\
8.58399999999954	11.0952782027547\\
8.58599999999954	11.0930729178093\\
8.58799999999954	11.0908680593643\\
8.58999999999954	11.0886636273549\\
8.59199999999954	11.0864596217165\\
8.59399999999954	11.0842560423841\\
8.59599999999954	11.0820528892927\\
8.59799999999954	11.0798501623778\\
8.59999999999954	11.0776478615741\\
8.60199999999954	11.075445986817\\
8.60399999999954	11.0732445380413\\
8.60599999999954	11.0710435151822\\
8.60799999999954	11.0688429181747\\
8.60999999999954	11.0666427469536\\
8.61199999999955	11.0644430014541\\
8.61399999999955	11.062243681611\\
8.61599999999955	11.0600447873593\\
8.61799999999955	11.0578463186339\\
8.61999999999955	11.0556482753695\\
8.62199999999955	11.0534506575013\\
8.62399999999955	11.0512534649639\\
8.62599999999955	11.0490566976923\\
8.62799999999955	11.046860355621\\
8.62999999999955	11.044664438685\\
8.63199999999955	11.042468946819\\
8.63399999999955	11.0402738799576\\
8.63599999999955	11.0380792380358\\
8.63799999999955	11.0358850209881\\
8.63999999999955	11.0336912287491\\
8.64199999999956	11.0314978612537\\
8.64399999999956	11.0293049184362\\
8.64599999999956	11.0271124002315\\
8.64799999999956	11.024920306574\\
8.64999999999956	11.0227286373984\\
8.65199999999956	11.0205373926392\\
8.65399999999956	11.0183465722309\\
8.65599999999956	11.016156176108\\
8.65799999999956	11.013966204205\\
8.65999999999956	11.0117766564564\\
8.66199999999956	11.0095875327966\\
8.66399999999956	11.0073988331601\\
8.66599999999956	11.0052105574813\\
8.66799999999956	11.0030227056943\\
8.66999999999956	11.0008352777339\\
8.67199999999957	10.9986482735343\\
8.67399999999957	10.9964616930297\\
8.67599999999957	10.9942755361545\\
8.67799999999957	10.992089802843\\
8.67999999999957	10.9899044930295\\
8.68199999999957	10.9877196066482\\
8.68399999999957	10.9855351436333\\
8.68599999999957	10.9833511039191\\
8.68799999999957	10.9811674874397\\
8.68999999999957	10.9789842941294\\
8.69199999999957	10.9768015239222\\
8.69399999999957	10.9746191767523\\
8.69599999999957	10.9724372525539\\
8.69799999999957	10.9702557512611\\
8.69999999999957	10.9680746728078\\
8.70199999999958	10.9658940171282\\
8.70399999999958	10.9637137841563\\
8.70599999999958	10.9615339738263\\
8.70799999999958	10.9593545860719\\
8.70999999999958	10.9571756208273\\
8.71199999999958	10.9549970780264\\
8.71399999999958	10.9528189576032\\
8.71599999999958	10.9506412594916\\
8.71799999999958	10.9484639836255\\
8.71999999999958	10.9462871299388\\
8.72199999999958	10.9441106983654\\
8.72399999999958	10.9419346888393\\
8.72599999999958	10.939759101294\\
8.72799999999958	10.9375839356636\\
8.72999999999958	10.9354091918819\\
8.73199999999959	10.9332348698826\\
8.73399999999959	10.9310609695995\\
8.73599999999959	10.9288874909664\\
8.73799999999959	10.926714433917\\
8.73999999999959	10.924541798385\\
8.74199999999959	10.9223695843041\\
8.74399999999959	10.9201977916081\\
8.74599999999959	10.9180264202306\\
8.74799999999959	10.9158554701052\\
8.74999999999959	10.9136849411655\\
8.75199999999959	10.9115148333454\\
8.75399999999959	10.9093451465781\\
8.75599999999959	10.9071758807974\\
8.75799999999959	10.905007035937\\
8.75999999999959	10.9028386119303\\
8.7619999999996	10.9006706087107\\
8.7639999999996	10.8985030262119\\
8.7659999999996	10.8963358643675\\
8.7679999999996	10.8941691231107\\
8.7699999999996	10.8920028023753\\
8.7719999999996	10.8898369020944\\
8.7739999999996	10.8876714222017\\
8.7759999999996	10.8855063626304\\
8.7779999999996	10.8833417233142\\
8.7799999999996	10.8811775041863\\
8.7819999999996	10.8790137051801\\
8.7839999999996	10.8768503262289\\
8.7859999999996	10.8746873672661\\
8.7879999999996	10.872524828225\\
8.7899999999996	10.870362709039\\
8.79199999999961	10.8682010096414\\
8.79399999999961	10.8660397299653\\
8.79599999999961	10.863878869944\\
8.79799999999961	10.8617184295108\\
8.79999999999961	10.8595584085991\\
8.80199999999961	10.8573988071418\\
8.80399999999961	10.8552396250724\\
8.80599999999961	10.8530808623238\\
8.80799999999961	10.8509225188294\\
8.80999999999961	10.8487645945223\\
8.81199999999961	10.8466070893356\\
8.81399999999961	10.8444500032023\\
8.81599999999961	10.8422933360557\\
8.81799999999961	10.8401370878289\\
8.81999999999961	10.8379812584548\\
8.82199999999962	10.8358258478667\\
8.82399999999962	10.8336708559975\\
8.82599999999962	10.8315162827802\\
8.82799999999962	10.8293621281479\\
8.82999999999962	10.8272083920336\\
8.83199999999962	10.8250550743703\\
8.83399999999962	10.8229021750909\\
8.83599999999962	10.8207496941284\\
8.83799999999962	10.8185976314158\\
8.83999999999962	10.8164459868861\\
8.84199999999962	10.814294760472\\
8.84399999999962	10.8121439521066\\
8.84599999999962	10.8099935617227\\
8.84799999999962	10.8078435892531\\
8.84999999999962	10.8056940346309\\
8.85199999999963	10.8035448977887\\
8.85399999999963	10.8013961786594\\
8.85599999999963	10.7992478771759\\
8.85799999999963	10.797099993271\\
8.85999999999963	10.7949525268775\\
8.86199999999963	10.7928054779281\\
8.86399999999963	10.7906588463556\\
8.86599999999963	10.7885126320928\\
8.86799999999963	10.7863668350723\\
8.86999999999963	10.7842214552269\\
8.87199999999963	10.7820764924894\\
8.87399999999963	10.7799319467925\\
8.87599999999963	10.7777878180688\\
8.87799999999963	10.775644106251\\
8.87999999999963	10.7735008112717\\
8.88199999999964	10.7713579330635\\
8.88399999999964	10.7692154715594\\
8.88599999999964	10.7670734266915\\
8.88799999999964	10.7649317983927\\
8.88999999999964	10.7627905865956\\
8.89199999999964	10.7606497912327\\
8.89399999999964	10.7585094122366\\
8.89599999999964	10.7563694495399\\
8.89799999999964	10.754229903075\\
8.89999999999964	10.7520907727747\\
8.90199999999964	10.7499520585711\\
8.90399999999964	10.7478137603972\\
8.90599999999964	10.7456758781851\\
8.90799999999964	10.7435384118674\\
8.90999999999964	10.7414013613766\\
8.91199999999965	10.7392647266452\\
8.91399999999965	10.7371285076056\\
8.91599999999965	10.7349927041901\\
8.91799999999965	10.7328573163313\\
8.91999999999965	10.7307223439616\\
8.92199999999965	10.7285877870133\\
8.92399999999965	10.7264536454187\\
8.92599999999965	10.7243199191103\\
8.92799999999965	10.7221866080206\\
8.92999999999965	10.7200537120817\\
8.93199999999965	10.717921231226\\
8.93399999999965	10.7157891653859\\
8.93599999999965	10.7136575144936\\
8.93799999999965	10.7115262784815\\
8.93999999999965	10.7093954572818\\
8.94199999999966	10.7072650508269\\
8.94399999999966	10.7051350590488\\
8.94599999999966	10.7030054818801\\
8.94799999999966	10.7008763192528\\
8.94999999999966	10.6987475710993\\
8.95199999999966	10.6966192373516\\
8.95399999999966	10.6944913179422\\
8.95599999999966	10.692363812803\\
8.95799999999966	10.6902367218663\\
8.95999999999966	10.6881100450644\\
8.96199999999966	10.6859837823293\\
8.96399999999966	10.6838579335932\\
8.96599999999966	10.6817324987882\\
8.96799999999966	10.6796074778465\\
8.96999999999966	10.6774828707002\\
8.97199999999967	10.6753586772814\\
8.97399999999967	10.6732348975222\\
8.97599999999967	10.6711115313547\\
8.97799999999967	10.668988578711\\
8.97999999999967	10.666866039523\\
8.98199999999967	10.664743913723\\
8.98399999999967	10.6626222012428\\
8.98599999999967	10.6605009020147\\
8.98799999999967	10.6583800159706\\
8.98999999999967	10.6562595430424\\
8.99199999999967	10.6541394831623\\
8.99399999999967	10.6520198362621\\
8.99599999999967	10.6499006022738\\
8.99799999999967	10.6477817811296\\
8.99999999999967	10.6456633727613\\
9.00199999999968	10.6435453771008\\
9.00399999999968	10.6414277940801\\
9.00599999999968	10.6393106236311\\
9.00799999999968	10.6371938656859\\
9.00999999999968	10.6350775201761\\
9.01199999999968	10.6329615870338\\
9.01399999999968	10.6308460661909\\
9.01599999999968	10.6287309575792\\
9.01799999999968	10.6266162611307\\
9.01999999999968	10.6245019767771\\
9.02199999999968	10.6223881044503\\
9.02399999999968	10.6202746440823\\
9.02599999999968	10.6181615956047\\
9.02799999999968	10.6160489589495\\
9.02999999999968	10.6139367340483\\
9.03199999999969	10.6118249208333\\
9.03399999999969	10.6097135192359\\
9.03599999999969	10.607602529188\\
9.03799999999969	10.6054919506215\\
9.03999999999969	10.603381783468\\
9.04199999999969	10.6012720276595\\
9.04399999999969	10.5991626831275\\
9.04599999999969	10.5970537498038\\
9.04799999999969	10.5949452276201\\
9.04999999999969	10.5928371165084\\
9.05199999999969	10.5907294164\\
9.05399999999969	10.5886221272269\\
9.05599999999969	10.5865152489207\\
9.05799999999969	10.5844087814131\\
9.05999999999969	10.5823027246358\\
9.0619999999997	10.5801970785204\\
9.0639999999997	10.5780918429986\\
9.0659999999997	10.5759870180021\\
9.0679999999997	10.5738826034624\\
9.0699999999997	10.5717785993113\\
9.0719999999997	10.5696750054804\\
9.0739999999997	10.5675718219013\\
9.0759999999997	10.5654690485055\\
9.0779999999997	10.5633666852247\\
9.0799999999997	10.5612647319905\\
9.0819999999997	10.5591631887346\\
9.0839999999997	10.5570620553883\\
9.0859999999997	10.5549613318833\\
9.0879999999997	10.5528610181513\\
9.0899999999997	10.5507611141236\\
9.09199999999971	10.548661619732\\
9.09399999999971	10.5465625349078\\
9.09599999999971	10.5444638595828\\
9.09799999999971	10.5423655936882\\
9.09999999999971	10.5402677371558\\
9.10199999999971	10.5381702899168\\
9.10399999999971	10.5360732519031\\
9.10599999999971	10.5339766230458\\
9.10799999999971	10.5318804032767\\
9.10999999999971	10.529784592527\\
9.11199999999971	10.5276891907283\\
9.11399999999971	10.525594197812\\
9.11599999999971	10.5234996137097\\
9.11799999999971	10.5214054383527\\
9.11999999999972	10.5193116716725\\
9.12199999999972	10.5172183136005\\
9.12399999999972	10.515125364068\\
9.12599999999972	10.5130328230067\\
9.12799999999972	10.5109406903477\\
9.12999999999972	10.5088489660226\\
9.13199999999972	10.5067576499629\\
9.13399999999972	10.5046667420996\\
9.13599999999972	10.5025762423644\\
9.13799999999972	10.5004861506886\\
9.13999999999972	10.4983964670034\\
9.14199999999972	10.4963071912404\\
9.14399999999972	10.4942183233308\\
9.14599999999972	10.492129863206\\
9.14799999999972	10.4900418107973\\
9.14999999999973	10.4879541660361\\
9.15199999999973	10.4858669288536\\
9.15399999999973	10.4837800991812\\
9.15599999999973	10.4816936769501\\
9.15799999999973	10.4796076620919\\
9.15999999999973	10.4775220545375\\
9.16199999999973	10.4754368542184\\
9.16399999999973	10.4733520610659\\
9.16599999999973	10.4712676750113\\
9.16799999999973	10.4691836959857\\
9.16999999999973	10.4671001239205\\
9.17199999999973	10.465016958747\\
9.17399999999973	10.4629342003962\\
9.17599999999973	10.4608518487996\\
9.17799999999973	10.4587699038883\\
9.17999999999974	10.4566883655937\\
9.18199999999974	10.4546072338467\\
9.18399999999974	10.4525265085789\\
9.18599999999974	10.4504461897213\\
9.18799999999974	10.4483662772051\\
9.18999999999974	10.4462867709615\\
9.19199999999974	10.4442076709218\\
9.19399999999974	10.4421289770171\\
9.19599999999974	10.4400506891786\\
9.19799999999974	10.4379728073375\\
9.19999999999974	10.4358953314248\\
9.20199999999974	10.4338182613719\\
9.20399999999974	10.4317415971099\\
9.20599999999974	10.42966533857\\
9.20799999999974	10.4275894856832\\
9.20999999999975	10.4255140383806\\
9.21199999999975	10.4234389965935\\
9.21399999999975	10.421364360253\\
9.21599999999975	10.4192901292903\\
9.21799999999975	10.4172163036362\\
9.21999999999975	10.4151428832222\\
9.22199999999975	10.4130698679791\\
9.22399999999975	10.4109972578382\\
9.22599999999975	10.4089250527306\\
9.22799999999975	10.4068532525873\\
9.22999999999975	10.4047818573393\\
9.23199999999975	10.4027108669179\\
9.23399999999975	10.400640281254\\
9.23599999999975	10.3985701002788\\
9.23799999999975	10.3965003239233\\
9.23999999999976	10.3944309521185\\
9.24199999999976	10.3923619847956\\
9.24399999999976	10.3902934218855\\
9.24599999999976	10.3882252633193\\
9.24799999999976	10.3861575090281\\
9.24999999999976	10.3840901589428\\
9.25199999999976	10.3820232129947\\
9.25399999999976	10.3799566711144\\
9.25599999999976	10.3778905332334\\
9.25799999999976	10.3758247992823\\
9.25999999999976	10.3737594691924\\
9.26199999999976	10.3716945428945\\
9.26399999999976	10.3696300203198\\
9.26599999999976	10.3675659013991\\
9.26799999999976	10.3655021860635\\
9.26999999999977	10.363438874244\\
9.27199999999977	10.3613759658715\\
9.27399999999977	10.359313460877\\
9.27599999999977	10.3572513591916\\
9.27799999999977	10.3551896607461\\
9.27999999999977	10.3531283654715\\
9.28199999999977	10.3510674732989\\
9.28399999999977	10.3490069841591\\
9.28599999999977	10.3469468979831\\
9.28799999999977	10.3448872147018\\
9.28999999999977	10.3428279342463\\
9.29199999999977	10.3407690565474\\
9.29399999999977	10.338710581536\\
9.29599999999977	10.3366525091431\\
9.29799999999977	10.3345948392997\\
9.29999999999978	10.3325375719366\\
9.30199999999978	10.3304807069847\\
9.30399999999978	10.3284242443751\\
9.30599999999978	10.3263681840385\\
9.30799999999978	10.3243125259059\\
9.30999999999978	10.3222572699082\\
9.31199999999978	10.3202024159763\\
9.31399999999978	10.3181479640411\\
9.31599999999978	10.3160939140334\\
9.31799999999978	10.3140402658842\\
9.31999999999978	10.3119870195244\\
9.32199999999978	10.3099341748848\\
9.32399999999978	10.3078817318963\\
9.32599999999978	10.3058296904897\\
9.32799999999978	10.3037780505961\\
9.32999999999979	10.3017268121461\\
9.33199999999979	10.2996759750707\\
9.33399999999979	10.2976255393007\\
9.33599999999979	10.2955755047669\\
9.33799999999979	10.2935258714004\\
9.33999999999979	10.2914766391317\\
9.34199999999979	10.289427807892\\
9.34399999999979	10.2873793776118\\
9.34599999999979	10.2853313482221\\
9.34799999999979	10.2832837196539\\
9.34999999999979	10.2812364918377\\
9.35199999999979	10.2791896647046\\
9.35399999999979	10.2771432381852\\
9.35599999999979	10.2750972122106\\
9.35799999999979	10.2730515867113\\
9.3599999999998	10.2710063616183\\
9.3619999999998	10.2689615368624\\
9.3639999999998	10.2669171123744\\
9.3659999999998	10.2648730880851\\
9.3679999999998	10.2628294639252\\
9.3699999999998	10.2607862398257\\
9.3719999999998	10.2587434157173\\
9.3739999999998	10.2567009915307\\
9.3759999999998	10.2546589671969\\
9.3779999999998	10.2526173426465\\
9.3799999999998	10.2505761178103\\
9.3819999999998	10.2485352926192\\
9.3839999999998	10.2464948670039\\
9.3859999999998	10.2444548408952\\
9.3879999999998	10.2424152142239\\
9.38999999999981	10.2403759869206\\
9.39199999999981	10.2383371589163\\
9.39399999999981	10.2362987301417\\
9.39599999999981	10.2342607005275\\
9.39799999999981	10.2322230700045\\
9.39999999999981	10.2301858385035\\
9.40199999999981	10.2281490059552\\
9.40399999999981	10.2261125722903\\
9.40599999999981	10.2240765374397\\
9.40799999999981	10.2220409013341\\
9.40999999999981	10.2200056639041\\
9.41199999999981	10.2179708250807\\
9.41399999999981	10.2159363847944\\
9.41599999999981	10.2139023429761\\
9.41799999999981	10.2118686995565\\
9.41999999999982	10.2098354544663\\
9.42199999999982	10.2078026076362\\
9.42399999999982	10.2057701589971\\
9.42599999999982	10.2037381084795\\
9.42799999999982	10.2017064560143\\
9.42999999999982	10.1996752015322\\
9.43199999999982	10.1976443449639\\
9.43399999999982	10.1956138862401\\
9.43599999999982	10.1935838252914\\
9.43799999999982	10.1915541620488\\
9.43999999999982	10.1895248964428\\
9.44199999999982	10.1874960284043\\
9.44399999999982	10.1854675578638\\
9.44599999999982	10.1834394847521\\
9.44799999999982	10.181411809\\
9.44999999999983	10.179384530538\\
9.45199999999983	10.177357649297\\
9.45399999999983	10.1753311652077\\
9.45599999999983	10.1733050782006\\
9.45799999999983	10.1712793882066\\
9.45999999999983	10.1692540951563\\
9.46199999999983	10.1672291989804\\
9.46399999999983	10.1652046996097\\
9.46599999999983	10.1631805969748\\
9.46799999999983	10.1611568910063\\
9.46999999999983	10.1591335816351\\
9.47199999999983	10.1571106687918\\
9.47399999999983	10.1550881524071\\
9.47599999999983	10.1530660324116\\
9.47799999999983	10.1510443087361\\
9.47999999999984	10.1490229813112\\
9.48199999999984	10.1470020500676\\
9.48399999999984	10.1449815149361\\
9.48599999999984	10.1429613758471\\
9.48799999999984	10.1409416327316\\
9.48999999999984	10.1389222855201\\
9.49199999999984	10.1369033341433\\
9.49399999999984	10.1348847785319\\
9.49599999999984	10.1328666186165\\
9.49799999999984	10.1308488543279\\
9.49999999999984	10.1288314855968\\
9.50199999999984	10.1268145123536\\
9.50399999999984	10.1247979345293\\
9.50599999999984	10.1227817520543\\
9.50799999999984	10.1207659648595\\
9.50999999999985	10.1187505728754\\
9.51199999999985	10.1167355760328\\
9.51399999999985	10.1147209742621\\
9.51599999999985	10.1127067674943\\
9.51799999999985	10.1106929556599\\
9.51999999999985	10.1086795386896\\
9.52199999999985	10.1066665165141\\
9.52399999999985	10.1046538890638\\
9.52599999999985	10.1026416562697\\
9.52799999999985	10.1006298180623\\
9.52999999999985	10.0986183743724\\
9.53199999999985	10.0966073251304\\
9.53399999999985	10.0945966702672\\
9.53599999999985	10.0925864097133\\
9.53799999999985	10.0905765433995\\
9.53999999999986	10.0885670712563\\
9.54199999999986	10.0865579932145\\
9.54399999999986	10.0845493092047\\
9.54599999999986	10.0825410191575\\
9.54799999999986	10.0805331230036\\
9.54999999999986	10.0785256206737\\
9.55199999999986	10.0765185120984\\
9.55399999999986	10.0745117972084\\
9.55599999999986	10.0725054759342\\
9.55799999999986	10.0704995482067\\
9.55999999999986	10.0684940139564\\
9.56199999999986	10.066488873114\\
9.56399999999986	10.0644841256101\\
9.56599999999986	10.0624797713754\\
9.56799999999986	10.0604758103405\\
9.56999999999987	10.0584722424362\\
9.57199999999987	10.0564690675929\\
9.57399999999987	10.0544662857416\\
9.57599999999987	10.0524638968126\\
9.57799999999987	10.0504619007367\\
9.57999999999987	10.0484602974446\\
9.58199999999987	10.0464590868669\\
9.58399999999987	10.0444582689343\\
9.58599999999987	10.0424578435774\\
9.58799999999987	10.0404578107268\\
9.58999999999987	10.0384581703133\\
9.59199999999987	10.0364589222674\\
9.59399999999987	10.0344600665198\\
9.59599999999987	10.0324616030013\\
9.59799999999987	10.0304635316424\\
9.59999999999988	10.0284658523736\\
9.60199999999988	10.0264685651259\\
9.60399999999988	10.0244716698298\\
9.60599999999988	10.022475166416\\
9.60799999999988	10.020479054815\\
9.60999999999988	10.0184833349576\\
9.61199999999988	10.0164880067744\\
9.61399999999988	10.014493070196\\
9.61599999999988	10.0124985251532\\
9.61799999999988	10.0105043715766\\
9.61999999999988	10.0085106093968\\
9.62199999999988	10.0065172385445\\
9.62399999999988	10.0045242589503\\
9.62599999999988	10.002531670545\\
9.62799999999988	10.0005394732591\\
9.62999999999989	9.99854766702346\\
9.63199999999989	9.99655625176852\\
9.63399999999989	9.99456522742509\\
9.63599999999989	9.99257459392371\\
9.63799999999989	9.99058435119517\\
9.63999999999989	9.98859449917002\\
9.64199999999989	9.986605037779\\
9.64399999999989	9.98461596695276\\
9.64599999999989	9.98262728662193\\
9.64799999999989	9.98063899671726\\
9.64999999999989	9.97865109716932\\
9.65199999999989	9.97666358790884\\
9.65399999999989	9.97467646886645\\
9.65599999999989	9.97268973997288\\
9.65799999999989	9.97070340115873\\
9.6599999999999	9.9687174523547\\
9.6619999999999	9.96673189349151\\
9.6639999999999	9.96474672449973\\
9.6659999999999	9.96276194531015\\
9.6679999999999	9.96077755585337\\
9.6699999999999	9.95879355606007\\
9.6719999999999	9.95680994586095\\
9.6739999999999	9.95482672518669\\
9.6759999999999	9.95284389396797\\
9.6779999999999	9.95086145213545\\
9.6799999999999	9.9488793996199\\
9.6819999999999	9.94689773635184\\
9.6839999999999	9.94491646226207\\
9.6859999999999	9.94293557728132\\
9.6879999999999	9.94095508134013\\
9.68999999999991	9.93897497436934\\
9.69199999999991	9.93699525629948\\
9.69399999999991	9.93501592706142\\
9.69599999999991	9.93303698658575\\
9.69799999999991	9.93105843480316\\
9.69999999999991	9.92908027164437\\
9.70199999999991	9.9271024970401\\
9.70399999999991	9.92512511092095\\
9.70599999999991	9.92314811321775\\
9.70799999999991	9.92117150386112\\
9.70999999999991	9.9191952827818\\
9.71199999999991	9.91721944991045\\
9.71399999999991	9.91524400517785\\
9.71599999999991	9.9132689485146\\
9.71799999999991	9.91129427985154\\
9.71999999999992	9.90931999911927\\
9.72199999999992	9.90734610624852\\
9.72399999999992	9.90537260117013\\
9.72599999999992	9.90339948381459\\
9.72799999999992	9.90142675411282\\
9.72999999999992	9.89945441199542\\
9.73199999999992	9.8974824573932\\
9.73399999999992	9.89551089023677\\
9.73599999999992	9.89353971045698\\
9.73799999999992	9.89156891798444\\
9.73999999999992	9.88959851274996\\
9.74199999999992	9.8876284946842\\
9.74399999999992	9.885658863718\\
9.74599999999992	9.883689619782\\
9.74799999999992	9.88172076280693\\
9.74999999999993	9.87975229272358\\
9.75199999999993	9.87778420946267\\
9.75399999999993	9.87581651295495\\
9.75599999999993	9.87384920313115\\
9.75799999999993	9.87188227992201\\
9.75999999999993	9.86991574325832\\
9.76199999999993	9.8679495930707\\
9.76399999999993	9.86598382929006\\
9.76599999999993	9.86401845184705\\
9.76799999999993	9.86205346067255\\
9.76999999999993	9.86008885569719\\
9.77199999999993	9.85812463685171\\
9.77399999999993	9.856160804067\\
9.77599999999993	9.85419735727381\\
9.77799999999993	9.85223429640272\\
9.77999999999994	9.8502716213847\\
9.78199999999994	9.84830933215041\\
9.78399999999994	9.84634742863075\\
9.78599999999994	9.84438591075633\\
9.78799999999994	9.84242477845804\\
9.78999999999994	9.84046403166665\\
9.79199999999994	9.83850367031291\\
9.79399999999994	9.83654369432761\\
9.79599999999994	9.8345841036415\\
9.79799999999994	9.83262489818549\\
9.79999999999994	9.83066607789025\\
9.80199999999994	9.82870764268665\\
9.80399999999994	9.82674959250542\\
9.80599999999994	9.82479192727742\\
9.80799999999994	9.82283464693337\\
9.80999999999995	9.82087775140416\\
9.81199999999995	9.81892124062061\\
9.81399999999995	9.81696511451347\\
9.81599999999995	9.81500937301354\\
9.81799999999995	9.81305401605164\\
9.81999999999995	9.81109904355866\\
9.82199999999995	9.80914445546536\\
9.82399999999995	9.80719025170254\\
9.82599999999995	9.80523643220106\\
9.82799999999995	9.80328299689178\\
9.82999999999995	9.80132994570548\\
9.83199999999995	9.79937727857297\\
9.83399999999995	9.79742499542517\\
9.83599999999995	9.79547309619286\\
9.83799999999995	9.7935215808069\\
9.83999999999996	9.7915704491981\\
9.84199999999996	9.78961970129737\\
9.84399999999996	9.78766933703553\\
9.84599999999996	9.78571935634339\\
9.84799999999996	9.78376975915187\\
9.84999999999996	9.78182054539181\\
9.85199999999996	9.77987171499398\\
9.85399999999996	9.77792326788944\\
9.85599999999996	9.77597520400888\\
9.85799999999996	9.77402752328324\\
9.85999999999996	9.77208022564338\\
9.86199999999996	9.77013331102027\\
9.86399999999996	9.76818677934459\\
9.86599999999996	9.7662406305474\\
9.86799999999996	9.76429486455949\\
9.86999999999997	9.76234948131183\\
9.87199999999997	9.7604044807352\\
9.87399999999997	9.75845986276061\\
9.87599999999997	9.75651562731889\\
9.87799999999997	9.75457177434095\\
9.87999999999997	9.75262830375759\\
9.88199999999997	9.75068521550001\\
9.88399999999997	9.74874250949884\\
9.88599999999997	9.74680018568508\\
9.88799999999997	9.74485824398969\\
9.88999999999997	9.74291668434353\\
9.89199999999997	9.74097550667758\\
9.89399999999997	9.7390347109227\\
9.89599999999997	9.7370942970099\\
9.89799999999997	9.73515426487007\\
9.89999999999998	9.73321461443411\\
9.90199999999998	9.73127534563299\\
9.90399999999998	9.72933645839768\\
9.90599999999998	9.72739795265918\\
9.90799999999998	9.72545982834822\\
9.90999999999998	9.72352208539602\\
9.91199999999998	9.72158472373339\\
9.91399999999998	9.7196477432913\\
9.91599999999998	9.71771114400074\\
9.91799999999998	9.71577492579264\\
9.91999999999998	9.71383908859806\\
9.92199999999998	9.71190363234791\\
9.92399999999998	9.70996855697318\\
9.92599999999998	9.70803386240474\\
9.92799999999998	9.7060995485738\\
9.92999999999999	9.70416561541118\\
9.93199999999999	9.70223206284787\\
9.93399999999999	9.70029889081503\\
9.93599999999999	9.69836609924347\\
9.93799999999999	9.69643368806434\\
9.93999999999999	9.69450165720855\\
9.94199999999999	9.69257000660708\\
9.94399999999999	9.69063873619107\\
9.94599999999999	9.68870784589142\\
9.94799999999999	9.6867773356393\\
9.94999999999999	9.68484720536554\\
9.95199999999999	9.68291745500128\\
9.95399999999999	9.68098808447755\\
9.95599999999999	9.67905909372541\\
9.95799999999999	9.67713048267579\\
9.96	9.67520225125992\\
9.962	9.67327439940871\\
9.964	9.67134692705322\\
9.966	9.66941983412455\\
9.968	9.66749312055376\\
9.97	9.66556678627182\\
9.972	9.66364083120992\\
9.974	9.66171525529909\\
9.976	9.65979005847039\\
9.978	9.65786524065495\\
9.98	9.65594080178382\\
9.982	9.65401674178807\\
9.984	9.65209306059878\\
9.986	9.65016975814708\\
9.988	9.64824683436405\\
9.99000000000001	9.64632428918083\\
9.99200000000001	9.64440212252852\\
9.99400000000001	9.64248033433818\\
9.99600000000001	9.64055892454099\\
9.99800000000001	9.63863789306802\\
10	9.63671723985048\\
};
\end{axis}

\begin{axis}[%
width=2.603in,
height=1.074in,
at={(4.436in,4.246in)},
scale only axis,
xmin=0,
xmax=10,
xlabel style={font=\color{white!15!black}},
xlabel={t},
ymode=log,
ymin=19.7369854446475,
ymax=100000,
yminorticks=true,
ylabel style={font=\color{white!15!black}},
ylabel={indice stiff},
axis background/.style={fill=white},
title style={font=\bfseries},
title={N=20}
]
\addplot [color=mycolor1, forget plot]
  table[row sep=crcr]{%
0	158.69469631342\\
0.002	81680.1831811359\\
0.004	40775.880051423\\
0.006	27141.2753120178\\
0.008	20324.0931633983\\
0.01	16233.8784631964\\
0.012	13507.1461783628\\
0.014	11559.5455927448\\
0.016	10098.9013589079\\
0.018	8962.89384877125\\
0.02	8054.13129385822\\
0.022	7310.63712178418\\
0.024	6691.09362448647\\
0.026	6166.89624113383\\
0.028	5717.61314835009\\
0.03	5328.26101435962\\
0.032	4987.60234531898\\
0.034	4687.04376824723\\
0.036	4419.90155297469\\
0.038	4180.89907453312\\
0.04	3965.81503682545\\
0.042	3771.23220382272\\
0.044	3594.35465861705\\
0.046	3432.87273402365\\
0.048	3284.86171053439\\
0.05	3148.70482676334\\
0.052	3023.03405671299\\
0.054	2906.68404764668\\
0.056	2798.65592842076\\
0.058	2698.08860574449\\
0.06	2604.23580118927\\
0.062	2516.44753265005\\
0.064	2434.15506803232\\
0.066	2356.85861462829\\
0.068	2284.1171809636\\
0.07	2215.54017660771\\
0.0720000000000001	2150.780412019\\
0.0740000000000001	2089.52823354727\\
0.0760000000000001	2031.50658449284\\
0.0780000000000001	1976.46682599925\\
0.0800000000000001	1924.18518481549\\
0.0820000000000001	1874.45972089291\\
0.0840000000000001	1827.10772818821\\
0.0860000000000001	1781.96349814283\\
0.0880000000000001	1738.87638814129\\
0.0900000000000001	1697.70914751088\\
0.0920000000000001	1658.33646186769\\
0.0940000000000001	1620.64368329118\\
0.0960000000000001	1584.52571922764\\
0.0980000000000001	1549.88605744517\\
0.1	1516.63590799582\\
0.102	1484.69344612135\\
0.104	1453.98314251771\\
0.106	1424.43516941806\\
0.108	1395.98487266633\\
0.11	1368.57230138319\\
0.112	1342.14178802507\\
0.114	1316.64157264695\\
0.116	1292.0234660349\\
0.118	1268.24254709673\\
0.12	1245.25689051259\\
0.122	1223.02732117525\\
0.124	1201.517192393\\
0.126	1180.69218521732\\
0.128	1160.52012658235\\
0.13	1140.9708242324\\
0.132	1122.01591665623\\
0.134	1103.62873645833\\
0.136	1085.78418578844\\
0.138	1068.45862260044\\
0.14	1051.62975666251\\
0.142	1035.27655435442\\
0.144	1019.37915139796\\
0.146	1003.91877276086\\
0.148	988.877659054864\\
0.15	974.238998821645\\
0.152	959.986866164712\\
0.154	946.106163241385\\
0.156	932.582567177747\\
0.158	919.402481016461\\
0.16	906.552988342535\\
0.162	894.02181127022\\
0.164	881.797271504946\\
0.166	869.868254218686\\
0.168	858.224174506696\\
0.17	846.854946211234\\
0.172	835.750952921163\\
0.174	824.903020971022\\
0.176	814.302394280752\\
0.178	803.940710892681\\
0.18	793.809981071333\\
0.182	783.902566848416\\
0.184	774.2111629008\\
0.186	764.72877866316\\
0.188	755.448721581322\\
0.19	746.364581424254\\
0.192	737.470215575429\\
0.194	728.759735235232\\
0.196	720.227492467198\\
0.198	711.868068029633\\
0.2	703.676259937227\\
0.202	695.647072702365\\
0.204	687.775707209022\\
0.206	680.057551176662\\
0.208	672.488170174251\\
0.21	665.063299147862\\
0.212	657.77883442747\\
0.214	650.630826182478\\
0.216	643.615471295808\\
0.218	636.729106630769\\
0.22	629.968202664746\\
0.222	623.329357467369\\
0.224	616.809291000666\\
0.226	610.404839722853\\
0.228	604.112951474946\\
0.23	597.930680635179\\
0.232	591.855183523318\\
0.234	585.883714041088\\
0.236	580.013619533937\\
0.238	574.242336861814\\
0.24	568.567388666025\\
0.242	562.986379821441\\
0.244	557.496994063041\\
0.246	552.096990777211\\
0.248	546.784201947858\\
0.25	541.556529249753\\
0.252	536.411941279704\\
0.254	531.348470918822\\
0.256	526.364212818386\\
0.258	521.457321002461\\
0.26	516.626006581598\\
0.262	511.868535570692\\
0.264	507.183226806448\\
0.266	502.568449958622\\
0.268	498.022623630229\\
0.27	493.544213542446\\
0.272	489.131730799314\\
0.274	484.783730228544\\
0.276	480.49880879456\\
0.278	476.275604079803\\
0.28	472.112792831505\\
0.282	468.009089569809\\
0.284	463.963245255094\\
0.286	459.974046011144\\
0.288	456.040311901281\\
0.29	452.160895755494\\
0.292	448.334682045489\\
0.294	444.560585805802\\
0.296	440.837551598592\\
0.298	437.164552520267\\
0.3	433.540589247525\\
0.302	429.964689121657\\
0.304	426.435905268815\\
0.306	422.953315754773\\
0.308	419.516022772803\\
0.31	416.123151862649\\
0.312	412.773851159961\\
0.314	409.46729067408\\
0.316	406.202661593433\\
0.318	402.979175617121\\
0.32	399.796064311531\\
0.322	396.652578490939\\
0.324	393.547987621136\\
0.326	390.481579244867\\
0.328	387.452658428509\\
0.33	384.460547228659\\
0.332	381.504584178111\\
0.334	378.584123790315\\
0.336	375.698536081572\\
0.338	372.847206110129\\
0.34	370.029533531543\\
0.342	367.2449321697\\
0.344	364.492829602837\\
0.346	361.772666763859\\
0.348	359.083897554518\\
0.35	356.425988472685\\
0.352	353.798418252587\\
0.354	351.200677516965\\
0.356	348.63226844121\\
0.358	346.092704428605\\
0.36	343.581509796476\\
0.362	341.098219472753\\
0.364	338.642378702486\\
0.366	336.213542764104\\
0.368	333.811276694822\\
0.37	331.435155025013\\
0.372	329.084761521225\\
0.374	326.759688937291\\
0.376	324.459538773585\\
0.378	322.183921043757\\
0.38	319.932454049007\\
0.382	317.704764159271\\
0.384	315.500485601536\\
0.386	313.319260254362\\
0.388	311.160737449172\\
0.39	309.024573777375\\
0.392	306.910432903548\\
0.394	304.817985384256\\
0.396	302.746908492437\\
0.398	300.696886047008\\
0.4	298.667608247698\\
0.402	296.658771514722\\
0.404	294.670078333303\\
0.406	292.701237102729\\
0.408	290.751961989961\\
0.41	288.821972787473\\
0.412	286.910994775187\\
0.414	285.018758586574\\
0.416	283.14500007849\\
0.418	281.28946020482\\
0.42	279.451884893786\\
0.422	277.632024928617\\
0.424	275.829635831762\\
0.426	274.04447775232\\
0.428	272.276315356609\\
0.43	270.524917721789\\
0.432	268.79005823258\\
0.434	267.071514480658\\
0.436	265.369068166971\\
0.438	263.682505006699\\
0.44	262.011614636812\\
0.442	260.356190526184\\
0.444	258.716029888062\\
0.446	257.090933595006\\
0.448	255.480706096067\\
0.45	253.885155336188\\
0.452	252.304092677759\\
0.454	250.737332824258\\
0.456	249.184693745922\\
0.458	247.645996607353\\
0.46	246.121065697052\\
0.462	244.6097283588\\
0.464	243.111814924764\\
0.466	241.627158650431\\
0.468	240.155595651134\\
0.47	238.696964840273\\
0.472	237.251107869083\\
0.474	235.817869067941\\
0.476	234.397095389182\\
0.478	232.988636351354\\
0.48	231.592343984868\\
0.482	230.208072779022\\
0.484	228.83567963034\\
0.486	227.475023792223\\
0.488	226.12596682578\\
0.49	224.788372552014\\
0.492	223.462107005014\\
0.494	222.147038386387\\
0.496	220.843037020843\\
0.498	219.549975312748\\
0.5	218.267727703831\\
0.502	216.996170631861\\
0.504	215.735182490335\\
0.506	214.484643589105\\
0.508	213.244436116011\\
0.51	212.014444099344\\
0.512	210.794553371262\\
0.514	209.584651532048\\
0.516	208.384627915186\\
0.518	207.19437355329\\
0.52	206.013781144841\\
0.522	204.84274502164\\
0.524	203.68116111706\\
0.526	202.528926935024\\
0.528	201.38594151971\\
0.53	200.252105425923\\
0.532	199.127320690154\\
0.534	198.01149080232\\
0.536	196.904520678102\\
0.538	195.806316631969\\
0.54	194.71678635076\\
0.542	193.635838867873\\
0.544	192.563384538051\\
0.546	191.49933501271\\
0.548	190.443603215806\\
0.55	189.396103320291\\
0.552	188.356750725013\\
0.554	187.325462032193\\
0.556	186.302155025344\\
0.558	185.286748647719\\
0.56	184.279162981185\\
0.562	183.279319225591\\
0.564	182.287139678573\\
0.566	181.30254771576\\
0.568	180.325467771498\\
0.57	179.355825319861\\
0.572	178.393546856172\\
0.574	177.438559878886\\
0.576	176.490792871819\\
0.578	175.550175286814\\
0.58	174.616637526742\\
0.582	173.690110928855\\
0.584	172.770527748502\\
0.586	171.85782114318\\
0.588	170.951925156924\\
0.59	170.052774705011\\
0.592	169.16030555898\\
0.594	168.274454331991\\
0.596	167.395158464423\\
0.598	166.52235620985\\
0.6	165.655986621229\\
0.602	164.795989537403\\
0.604	163.94230556991\\
0.606	163.09487608998\\
0.608	162.253643215881\\
0.61	161.418549800445\\
0.612	160.58953941892\\
0.614	159.766556356989\\
0.616	158.949545599089\\
0.618	158.13845281692\\
0.62	157.333224358229\\
0.622	156.533807235754\\
0.624	155.740149116447\\
0.626	154.952198310864\\
0.628	154.1699037628\\
0.63	153.393215039093\\
0.632	152.622082319649\\
0.634	151.856456387657\\
0.636	151.096288619984\\
0.638	150.341530977779\\
0.64	149.59213599722\\
0.642	148.84805678048\\
0.644	148.109246986842\\
0.646	147.375660823993\\
0.648	146.647253039475\\
0.65	145.923978912309\\
0.652	145.205794244771\\
0.654	144.492655354328\\
0.656	143.784519065723\\
0.658	143.081342703213\\
0.66	142.383084082944\\
0.662	141.689701505483\\
0.664	141.001153748469\\
0.666	140.317400059419\\
0.668	139.638400148663\\
0.67	138.964114182402\\
0.672	138.294502775903\\
0.674	137.629526986806\\
0.676	136.969148308584\\
0.678	136.313328664076\\
0.68	135.66203039919\\
0.682	135.01521627667\\
0.684	134.372849470023\\
0.686	133.734893557512\\
0.688000000000001	133.1013125163\\
0.690000000000001	132.472070716664\\
0.692000000000001	131.847132916326\\
0.694000000000001	131.226464254905\\
0.696000000000001	130.610030248428\\
0.698000000000001	129.997796783976\\
0.700000000000001	129.389730114397\\
0.702000000000001	128.78579685314\\
0.704000000000001	128.185963969153\\
0.706000000000001	127.590198781891\\
0.708000000000001	126.998468956405\\
0.710000000000001	126.410742498504\\
0.712000000000001	125.826987750038\\
0.714000000000001	125.247173384213\\
0.716000000000001	124.671268401032\\
0.718000000000001	124.099242122792\\
0.720000000000001	123.531064189654\\
0.722000000000001	122.966704555322\\
0.724000000000001	122.406133482747\\
0.726000000000001	121.849321539953\\
0.728000000000001	121.296239595902\\
0.730000000000001	120.746858816438\\
0.732000000000001	120.201150660316\\
0.734000000000001	119.65908687527\\
0.736000000000001	119.120639494176\\
0.738000000000001	118.585780831256\\
0.740000000000001	118.054483478363\\
0.742000000000001	117.526720301318\\
0.744000000000001	117.002464436324\\
0.746000000000001	116.48168928641\\
0.748000000000001	115.964368517971\\
0.750000000000001	115.450476057336\\
0.752000000000001	114.939986087414\\
0.754000000000001	114.432873044378\\
0.756000000000001	113.92911161442\\
0.758000000000001	113.428676730544\\
0.760000000000001	112.931543569429\\
0.762000000000001	112.437687548326\\
0.764000000000001	111.947084322017\\
0.766000000000001	111.45970977982\\
0.768000000000001	110.975540042647\\
0.770000000000001	110.494551460105\\
0.772000000000001	110.016720607637\\
0.774000000000001	109.54202428373\\
0.776000000000001	109.070439507149\\
0.778000000000001	108.601943514224\\
0.780000000000001	108.136513756175\\
0.782000000000001	107.67412789649\\
0.784000000000001	107.214763808334\\
0.786000000000001	106.75839957201\\
0.788000000000001	106.30501347244\\
0.790000000000001	105.854583996719\\
0.792000000000001	105.407089831677\\
0.794000000000001	104.962509861497\\
0.796000000000001	104.520823165358\\
0.798000000000001	104.082009015139\\
0.800000000000001	103.646046873116\\
0.802000000000001	103.212916389749\\
0.804000000000001	102.782597401459\\
0.806000000000001	102.355069928468\\
0.808000000000001	101.930314172644\\
0.810000000000001	101.508310515416\\
0.812000000000001	101.089039515698\\
0.814000000000001	100.672481907838\\
0.816000000000001	100.258618599624\\
0.818000000000001	99.8474306702979\\
0.820000000000001	99.4388993686154\\
0.822000000000001	99.0330061109242\\
0.824000000000001	98.6297324792781\\
0.826000000000001	98.2290602195807\\
0.828000000000001	97.8309712397491\\
0.830000000000001	97.435447607922\\
0.832000000000001	97.0424715506737\\
0.834000000000001	96.6520254512725\\
0.836000000000001	96.2640918479571\\
0.838000000000001	95.878653432244\\
0.840000000000001	95.4956930472518\\
0.842000000000001	95.115193686065\\
0.844000000000001	94.7371384901031\\
0.846000000000001	94.361510747536\\
0.848000000000001	93.9882938917037\\
0.850000000000001	93.6174714995746\\
0.852000000000001	93.2490272902135\\
0.854000000000001	92.8829451232882\\
0.856000000000001	92.519208997581\\
0.858000000000001	92.1578030495357\\
0.860000000000001	91.7987115518188\\
0.862000000000001	91.441918911908\\
0.864000000000001	91.0874096706927\\
0.866000000000001	90.7351685011089\\
0.868000000000001	90.3851802067785\\
0.870000000000001	90.0374297206764\\
0.872000000000001	89.6919021038249\\
0.874000000000001	89.3485825439902\\
0.876000000000001	89.0074563544139\\
0.878000000000001	88.6685089725549\\
0.880000000000001	88.3317259588498\\
0.882000000000001	87.9970929954952\\
0.884000000000001	87.664595885244\\
0.886000000000001	87.3342205502245\\
0.888000000000001	87.0059530307704\\
0.890000000000001	86.6797794842726\\
0.892000000000001	86.3556861840487\\
0.894000000000001	86.0336595182184\\
0.896000000000001	85.7136859886134\\
0.898000000000001	85.3957522096865\\
0.900000000000001	85.0798449074493\\
0.902000000000001	84.7659509184112\\
0.904000000000001	84.4540571885522\\
0.906000000000001	84.1441507722921\\
0.908000000000001	83.8362188314927\\
0.910000000000001	83.5302486344527\\
0.912000000000001	83.2262275549423\\
0.914000000000001	82.9241430712329\\
0.916000000000001	82.6239827651474\\
0.918000000000001	82.3257343211296\\
0.920000000000001	82.0293855253158\\
0.922000000000001	81.7349242646309\\
0.924000000000001	81.44233852589\\
0.926000000000001	81.1516163949204\\
0.928000000000001	80.8627460556892\\
0.930000000000001	80.5757157894464\\
0.932000000000001	80.2905139738863\\
0.934000000000001	80.0071290823105\\
0.936000000000001	79.7255496828103\\
0.938000000000001	79.4457644374646\\
0.940000000000001	79.1677621015372\\
0.942000000000001	78.8915315227011\\
0.944000000000001	78.6170616402622\\
0.946000000000001	78.3443414843983\\
0.948000000000001	78.0733601754197\\
0.950000000000001	77.8041069230163\\
0.952000000000001	77.536571025549\\
0.954000000000001	77.270741869317\\
0.956000000000001	77.0066089278666\\
0.958000000000001	76.7441617612924\\
0.960000000000001	76.4833900155512\\
0.962000000000001	76.224283421791\\
0.964000000000001	75.9668317956906\\
0.966000000000001	75.7110250368055\\
0.968000000000001	75.4568531279206\\
0.970000000000001	75.2043061344271\\
0.972000000000001	74.9533742036914\\
0.974000000000001	74.7040475644449\\
0.976000000000001	74.4563165261818\\
0.978000000000001	74.210171478564\\
0.980000000000001	73.965602890834\\
0.982000000000001	73.7226013112437\\
0.984000000000001	73.4811573664856\\
0.986000000000001	73.2412617611363\\
0.988000000000001	73.0029052771072\\
0.990000000000001	72.7660787731071\\
0.992000000000001	72.5307731841111\\
0.994000000000001	72.2969795208353\\
0.996000000000001	72.064688869228\\
0.998000000000001	71.8338923899656\\
1	71.60458131795\\
1.002	71.3767469618266\\
1.004	71.1503807034987\\
1.006	70.925473997663\\
1.008	70.7020183713405\\
1.01	70.4800054234209\\
1.012	70.2594268242168\\
1.014	70.0402743150255\\
1.016	69.8225397076875\\
1.018	69.6062148841746\\
1.02	69.391291796163\\
1.022	69.1777624646282\\
1.024	68.9656189794403\\
1.026	68.7548534989705\\
1.028	68.5454582497033\\
1.03	68.3374255258542\\
1.032	68.1307476889968\\
1.034	67.9254171676981\\
1.036	67.7214264571555\\
1.038	67.5187681188458\\
1.04	67.317434780175\\
1.042	67.1174191341469\\
1.044	66.918713939023\\
1.046	66.7213120179944\\
1.048	66.5252062588683\\
1.05	66.3303896137478\\
1.052	66.1368550987247\\
1.054	65.9445957935819\\
1.056	65.7536048414907\\
1.058	65.5638754487241\\
1.06	65.3754008843737\\
1.062	65.1881744800657\\
1.064	65.0021896296923\\
1.066	64.817439789144\\
1.068	64.6339184760364\\
1.07	64.4516192694696\\
1.072	64.2705358097581\\
1.074	64.0906617981952\\
1.076	63.9119909968027\\
1.078	63.7345172280947\\
1.08	63.5582343748423\\
1.082	63.3831363798424\\
1.084	63.2092172456941\\
1.086	63.0364710345729\\
1.088	62.864891868011\\
1.09	62.6944739266893\\
1.092	62.525211450214\\
1.094	62.3570987369164\\
1.096	62.1901301436419\\
1.098	62.0243000855477\\
1.1	61.8596030359001\\
1.102	61.6960335258737\\
1.104	61.5335861443555\\
1.106	61.3722555377503\\
1.108	61.2120364097776\\
1.11	61.0529235212874\\
1.112	60.8949116900557\\
1.114	60.7379957905985\\
1.116	60.582170753973\\
1.118	60.4274315675847\\
1.12	60.2737732749911\\
1.122	60.1211909757036\\
1.124	59.9696798249881\\
1.126	59.8192350336669\\
1.128	59.669851867907\\
1.13	59.5215256490197\\
1.132	59.3742517532455\\
1.134	59.2280256115371\\
1.136	59.0828427093392\\
1.138	58.9386985863641\\
1.14	58.7955888363543\\
1.142	58.6535091068459\\
1.144	58.5124550989188\\
1.146	58.3724225669436\\
1.148	58.2334073183155\\
1.15	58.0954052131763\\
1.152	57.9584121641367\\
1.154	57.8224241359714\\
1.156	57.6874371453156\\
1.158	57.5534472603427\\
1.16	57.4204506004216\\
1.162	57.2884433357752\\
1.164	57.1574216871017\\
1.166	57.0273819251997\\
1.168	56.8983203705578\\
1.17	56.7702333929386\\
1.172	56.6431174109282\\
1.174	56.51696889148\\
1.176	56.3917843494215\\
1.178	56.2675603469437\\
1.18	56.1442934930671\\
1.182	56.0219804430743\\
1.184	55.9006178979222\\
1.186	55.7802026036185\\
1.188	55.6607313505703\\
1.19	55.5422009729014\\
1.192	55.4246083477321\\
1.194	55.3079503944314\\
1.196	55.1922240738201\\
1.198	55.0774263873517\\
1.2	54.963554376239\\
1.202	54.8506051205483\\
1.204	54.7385757382487\\
1.206	54.6274633842164\\
1.208	54.517265249193\\
1.21	54.4079785586993\\
1.212	54.2996005718961\\
1.214	54.1921285803985\\
1.216	54.0855599070354\\
1.218	53.9798919045604\\
1.22	53.8751219543022\\
1.222	53.7712474647657\\
1.224	53.6682658701686\\
1.226	53.5661746289281\\
1.228	53.4649712220843\\
1.23	53.3646531516589\\
1.232	53.2652179389629\\
1.234	53.1666631228339\\
1.236	53.0689862578138\\
1.238	52.9721849122652\\
1.24	52.8762566664212\\
1.242	52.7811991103735\\
1.244	52.6870098419974\\
1.246	52.5936864648137\\
1.248	52.5012265857845\\
1.25	52.4096278130509\\
1.252	52.3188877536075\\
1.254	52.2290040109165\\
1.256	52.1399741824644\\
1.258	52.0517958572576\\
1.26	51.9644666132689\\
1.262	51.8779840148272\\
1.264	51.7923456099553\\
1.266	51.7075489276659\\
1.268	51.6235914752097\\
1.27	51.5404707352806\\
1.272	51.4581841631899\\
1.274	51.3767291840081\\
1.276	51.2961031896696\\
1.278	51.2163035360673\\
1.28	51.1373275401245\\
1.282	51.0591724768548\\
1.284	50.9818355764135\\
1.286	50.9053140211621\\
1.288	50.8296049427239\\
1.29	50.7547054190683\\
1.292	50.6806124716132\\
1.294	50.6073230623537\\
1.296	50.5348340910336\\
1.298	50.4631423923601\\
1.3	50.3922447332776\\
1.302	50.3221378102989\\
1.304	50.2528182469112\\
1.306	50.1842825910632\\
1.308	50.1165273127435\\
1.31	50.0495488016502\\
1.312	49.9833433649782\\
1.314	49.9179072253149\\
1.316	49.8532365186666\\
1.318	49.7893272926115\\
1.32	49.726175504601\\
1.322	49.6637770204082\\
1.324	49.6021276127333\\
1.326	49.5412229599774\\
1.328	49.4810586451799\\
1.33	49.4216301551479\\
1.332	49.3629328797546\\
1.334	49.3049621114372\\
1.336	49.2477130448889\\
1.338	49.1911807769391\\
1.34	49.1353603066526\\
1.342	49.0802465356156\\
1.344	49.0258342684384\\
1.346	48.9721182134653\\
1.348	48.9190929836827\\
1.35	48.8667530978484\\
1.352	48.815092981817\\
1.354	48.7641069700689\\
1.356	48.7137893074458\\
1.358	48.6641341510788\\
1.36	48.6151355725102\\
1.362	48.5667875600017\\
1.364	48.5190840210225\\
1.366	48.4720187849151\\
1.368	48.4255856057226\\
1.37	48.3797781651815\\
1.372	48.3345900758541\\
1.374	48.2900148844112\\
1.376	48.2460460750407\\
1.378	48.2026770729751\\
1.38	48.1599012481344\\
1.382	48.1177119188625\\
1.384	48.0761023557554\\
1.386	48.0350657855637\\
1.388	47.9945953951612\\
1.39	47.9546843355682\\
1.392	47.9153257260151\\
1.394	47.8765126580368\\
1.396	47.8382381995898\\
1.398	47.800495399174\\
1.4	47.7632772899505\\
1.402	47.7265768938524\\
1.404	47.6903872256672\\
1.406	47.654701297084\\
1.408	47.6195121207027\\
1.41	47.584812713987\\
1.412	47.5505961031539\\
1.414	47.5168553270065\\
1.416	47.4835834406691\\
1.418	47.4507735192592\\
1.42	47.4184186614541\\
1.422	47.3865119929673\\
1.424	47.3550466699234\\
1.426	47.324015882121\\
1.428	47.2934128561891\\
1.43	47.2632308586234\\
1.432	47.2334631987106\\
1.434	47.2041032313217\\
1.436	47.1751443595927\\
1.438	47.1465800374709\\
1.44	47.1184037721391\\
1.442	47.0906091263115\\
1.444	47.063189720403\\
1.446	47.0361392345684\\
1.448	47.0094514106163\\
1.45	46.9831200537991\\
1.452	46.9571390344712\\
1.454	46.9315022896321\\
1.456	46.9062038243424\\
1.458	46.8812377130201\\
1.46	46.8565981006261\\
1.462	46.8322792037267\\
1.464	46.8082753114489\\
1.466	46.7845807863292\\
1.468	46.7611900650522\\
1.47	46.7380976590865\\
1.472	46.7152981552286\\
1.474	46.6927862160412\\
1.476	46.6705565802095\\
1.478	46.6486040627996\\
1.48	46.6269235554394\\
1.482	46.6055100264142\\
1.484	46.5843585206854\\
1.486	46.5634641598363\\
1.488	46.5428221419479\\
1.49	46.522427741403\\
1.492	46.502276308636\\
1.494	46.4823632698111\\
1.496	46.4626841264589\\
1.498	46.4432344550419\\
1.5	46.4240099064895\\
1.502	46.40500620567\\
1.504	46.3862191508293\\
1.506	46.3676446129854\\
1.508	46.3492785352847\\
1.51	46.3311169323274\\
1.512	46.3131558894547\\
1.514	46.2953915620154\\
1.516	46.2778201745964\\
1.518	46.2604380202348\\
1.52	46.2432414596076\\
1.522	46.2262269202002\\
1.524	46.2093908954574\\
1.526	46.1927299439198\\
1.528	46.1762406883465\\
1.53	46.1599198148235\\
1.532	46.1437640718706\\
1.534	46.1277702695266\\
1.536	46.1119352784392\\
1.538	46.0962560289461\\
1.54	46.0807295101513\\
1.542	46.0653527689984\\
1.544	46.050122909345\\
1.546	46.0350370910363\\
1.548	46.0200925289795\\
1.55	46.0052864922204\\
1.552	45.9906163030224\\
1.554	45.9760793359502\\
1.556	45.9616730169587\\
1.558	45.9473948224873\\
1.56	45.9332422785586\\
1.562	45.9192129598871\\
1.564	45.9053044889927\\
1.566	45.8915145353257\\
1.568	45.8778408143939\\
1.57	45.8642810869069\\
1.572	45.8508331579244\\
1.574	45.8374948760155\\
1.576	45.8242641324296\\
1.578	45.8111388602752\\
1.58	45.7981170337115\\
1.582	45.7851966671518\\
1.584	45.7723758144729\\
1.586	45.7596525682438\\
1.588	45.7470250589588\\
1.59	45.7344914542848\\
1.592	45.7220499583228\\
1.594	45.7096988108757\\
1.596	45.6974362867344\\
1.598	45.6852606949667\\
1.6	45.6731703782321\\
1.602	45.6611637120914\\
1.604	45.6492391043423\\
1.606	45.6373949943591\\
1.608	45.6256298524453\\
1.61	45.613942179202\\
1.612	45.6023305049019\\
1.614	45.5907933888785\\
1.616	45.5793294189266\\
1.618	45.5679372107115\\
1.62	45.5566154071953\\
1.622	45.5453626780662\\
1.624	45.5341777191859\\
1.626	45.5230592520438\\
1.628	45.5120060232231\\
1.63	45.5010168038789\\
1.632	45.4900903892242\\
1.634	45.4792255980291\\
1.636	45.4684212721268\\
1.638	45.457676275931\\
1.64	45.4469894959684\\
1.642	45.4363598404094\\
1.644	45.4257862386214\\
1.646	45.4152676407213\\
1.648	45.4048030171423\\
1.65	45.3943913582095\\
1.652	45.3840316737239\\
1.654	45.3737229925542\\
1.656	45.3634643622387\\
1.658	45.3532548485959\\
1.66	45.343093535339\\
1.662	45.3329795237083\\
1.664	45.3229119320992\\
1.666	45.3128898957113\\
1.668	45.3029125661902\\
1.67	45.2929791112914\\
1.672	45.2830887145431\\
1.674	45.2732405749193\\
1.676	45.2634339065165\\
1.678	45.2536679382417\\
1.68	45.243941913507\\
1.682	45.2342550899276\\
1.684	45.2246067390281\\
1.686	45.2149961459567\\
1.688	45.2054226092017\\
1.69	45.1958854403207\\
1.692	45.1863839636688\\
1.694	45.1769175161348\\
1.696	45.1674854468893\\
1.698	45.1580871171282\\
1.7	45.1487218998283\\
1.702	45.1393891795084\\
1.704	45.1300883519914\\
1.706	45.1208188241773\\
1.708	45.1115800138157\\
1.71	45.1023713492869\\
1.712	45.0931922693857\\
1.714	45.0840422231141\\
1.716	45.0749206694705\\
1.718	45.0658270772536\\
1.72	45.0567609248608\\
1.722	45.0477217001\\
1.724	45.0387088999959\\
1.726	45.0297220306124\\
1.728	45.0207606068658\\
1.73	45.011824152355\\
1.732	45.0029121991824\\
1.734	44.9940242877914\\
1.736	44.9851599667975\\
1.738	44.9763187928281\\
1.74	44.9675003303648\\
1.742	44.9587041515901\\
1.744	44.949929836234\\
1.746	44.9411769714286\\
1.748	44.9324451515617\\
1.75	44.9237339781366\\
1.752	44.9150430596331\\
1.754	44.9063720113739\\
1.756	44.897720455388\\
1.758	44.8890880202864\\
1.76	44.8804743411318\\
1.762	44.8718790593169\\
1.764	44.8633018224386\\
1.766	44.8547422841865\\
1.768	44.8462001042197\\
1.77	44.8376749480568\\
1.772	44.8291664869638\\
1.774	44.8206743978458\\
1.776	44.8121983631385\\
1.778	44.8037380707067\\
1.78	44.795293213739\\
1.782	44.7868634906512\\
1.784	44.7784486049857\\
1.786	44.7700482653166\\
1.788	44.7616621851581\\
1.79	44.7532900828688\\
1.792	44.7449316815644\\
1.794	44.7365867090306\\
1.796	44.7282548976341\\
1.798	44.7199359842416\\
1.8	44.7116297101346\\
1.802	44.7033358209291\\
1.804	44.6950540664989\\
1.806	44.6867842008934\\
1.808	44.6785259822687\\
1.81	44.6702791728051\\
1.812	44.6620435386408\\
1.814	44.6538188497988\\
1.816	44.6456048801135\\
1.818	44.6374014071692\\
1.82	44.6292082122271\\
1.822	44.6210250801625\\
1.824	44.6128517994006\\
1.826	44.6046881618531\\
1.828	44.5965339628558\\
1.83	44.5883890011095\\
1.832	44.5802530786208\\
1.834	44.5721260006423\\
1.836	44.5640075756179\\
1.838	44.555897615127\\
1.84	44.547795933828\\
1.842	44.5397023494071\\
1.844	44.5316166825257\\
1.846	44.5235387567667\\
1.848	44.5154683985866\\
1.85	44.5074054372664\\
1.852	44.4993497048616\\
1.854	44.4913010361559\\
1.856	44.4832592686141\\
1.858	44.4752242423378\\
1.86	44.4671958000201\\
1.862	44.4591737869003\\
1.864	44.4511580507247\\
1.866	44.4431484417002\\
1.868	44.435144812457\\
1.87	44.4271470180064\\
1.872	44.4191549156995\\
1.874	44.4111683651921\\
1.876	44.4031872284036\\
1.878	44.3952113694802\\
1.88	44.3872406547605\\
1.882	44.379274952735\\
1.884	44.3713141340159\\
1.886	44.363358071299\\
1.888	44.3554066393304\\
1.89	44.3474597148763\\
1.892	44.3395171766847\\
1.894	44.3315789054588\\
1.896	44.323644783821\\
1.898	44.3157146962872\\
1.9	44.3077885292313\\
1.902	44.2998661708589\\
1.904	44.2919475111772\\
1.906	44.2840324419659\\
1.908	44.2761208567504\\
1.91	44.268212650773\\
1.912	44.2603077209661\\
1.914	44.2524059659269\\
1.916	44.2445072858895\\
1.918	44.2366115827005\\
1.92	44.2287187597942\\
1.922	44.2208287221669\\
1.924	44.2129413763533\\
1.926	44.205056630403\\
1.928	44.1971743938569\\
1.93	44.1892945777242\\
1.932	44.1814170944602\\
1.934	44.1735418579443\\
1.936	44.1656687834576\\
1.938	44.1577977876623\\
1.94	44.1499287885815\\
1.942	44.1420617055765\\
1.944	44.1341964593281\\
1.946	44.1263329718183\\
1.948	44.118471166307\\
1.95	44.1106109673169\\
1.952	44.1027523006107\\
1.954	44.0948950931776\\
1.956	44.08703927321\\
1.958	44.0791847700889\\
1.96	44.0713315143655\\
1.962	44.0634794377453\\
1.964	44.0556284730681\\
1.966	44.0477785542943\\
1.968	44.0399296164862\\
1.97	44.0320815957958\\
1.972	44.0242344294435\\
1.974	44.0163880557078\\
1.976	44.0085424139069\\
1.978	44.0006974443847\\
1.98	43.9928530884957\\
1.982	43.985009288592\\
1.984	43.9771659880066\\
1.986	43.9693231310405\\
1.988	43.9614806629505\\
1.99	43.9536385299328\\
1.992	43.9457966791113\\
1.994	43.9379550585253\\
1.996	43.9301136171147\\
1.998	43.9222723047087\\
2	43.9144310720119\\
2.002	43.9065898705943\\
2.004	43.8987486528771\\
2.006	43.8909073721225\\
2.008	43.8830659824188\\
2.01	43.8752244386749\\
2.012	43.867382696603\\
2.014	43.8595407127094\\
2.016	43.8516984442851\\
2.018	43.8438558493938\\
2.02	43.8360128868605\\
2.022	43.8281695162651\\
2.024	43.8203256979228\\
2.026	43.8124813928863\\
2.028	43.8046365629262\\
2.03	43.796791170525\\
2.032	43.788945178869\\
2.034	43.7810985518346\\
2.036	43.7732512539824\\
2.038	43.7654032505469\\
2.04	43.7575545074257\\
2.042	43.7497049911752\\
2.044	43.7418546689961\\
2.046	43.7340035087281\\
2.048	43.7261514788416\\
2.05	43.7182985484282\\
2.052	43.7104446871901\\
2.054	43.7025898654385\\
2.056	43.6947340540772\\
2.05799999999999	43.6868772246023\\
2.05999999999999	43.6790193490894\\
2.06199999999999	43.6711604001881\\
2.06399999999999	43.6633003511126\\
2.06599999999999	43.6554391756374\\
2.06799999999999	43.6475768480863\\
2.06999999999999	43.6397133433274\\
2.07199999999999	43.6318486367659\\
2.07399999999999	43.6239827043354\\
2.07599999999999	43.6161155224938\\
2.07799999999999	43.6082470682125\\
2.07999999999999	43.6003773189733\\
2.08199999999999	43.5925062527598\\
2.08399999999999	43.5846338480517\\
2.08599999999999	43.5767600838175\\
2.08799999999999	43.5688849395094\\
2.08999999999999	43.561008395054\\
2.09199999999999	43.5531304308511\\
2.09399999999999	43.5452510277626\\
2.09599999999999	43.5373701671094\\
2.09799999999999	43.5294878306647\\
2.09999999999999	43.5216040006476\\
2.10199999999999	43.5137186597183\\
2.10399999999999	43.5058317909709\\
2.10599999999999	43.4979433779297\\
2.10799999999999	43.4900534045415\\
2.10999999999999	43.4821618551732\\
2.11199999999999	43.4742687146013\\
2.11399999999999	43.4663739680128\\
2.11599999999999	43.4584776009938\\
2.11799999999999	43.4505795995301\\
2.11999999999999	43.4426799499977\\
2.12199999999999	43.4347786391579\\
2.12399999999999	43.4268756541568\\
2.12599999999999	43.4189709825141\\
2.12799999999999	43.411064612124\\
2.12999999999999	43.4031565312453\\
2.13199999999999	43.3952467285004\\
2.13399999999999	43.3873351928689\\
2.13599999999999	43.3794219136825\\
2.13799999999999	43.3715068806235\\
2.13999999999999	43.3635900837154\\
2.14199999999999	43.3556715133217\\
2.14399999999999	43.3477511601423\\
2.14599999999999	43.3398290152056\\
2.14799999999999	43.3319050698674\\
2.14999999999998	43.3239793158058\\
2.15199999999998	43.3160517450161\\
2.15399999999998	43.3081223498073\\
2.15599999999998	43.3001911227979\\
2.15799999999998	43.2922580569134\\
2.15999999999998	43.2843231453787\\
2.16199999999998	43.2763863817172\\
2.16399999999998	43.2684477597472\\
2.16599999999998	43.2605072735745\\
2.16799999999998	43.2525649175935\\
2.16999999999998	43.2446206864785\\
2.17199999999998	43.2366745751844\\
2.17399999999998	43.2287265789406\\
2.17599999999998	43.220776693247\\
2.17799999999998	43.2128249138712\\
2.17999999999998	43.2048712368463\\
2.18199999999998	43.1969156584647\\
2.18399999999998	43.1889581752771\\
2.18599999999998	43.1809987840872\\
2.18799999999998	43.1730374819497\\
2.18999999999998	43.165074266165\\
2.19199999999998	43.1571091342803\\
2.19399999999998	43.1491420840795\\
2.19599999999998	43.1411731135872\\
2.19799999999998	43.13320222106\\
2.19999999999998	43.1252294049855\\
2.20199999999998	43.117254664081\\
2.20399999999998	43.1092779972861\\
2.20599999999998	43.1012994037635\\
2.20799999999998	43.0933188828951\\
2.20999999999998	43.0853364342771\\
2.21199999999998	43.0773520577186\\
2.21399999999998	43.0693657532402\\
2.21599999999998	43.0613775210676\\
2.21799999999998	43.0533873616322\\
2.21999999999998	43.0453952755649\\
2.22199999999998	43.0374012636965\\
2.22399999999998	43.0294053270535\\
2.22599999999998	43.021407466855\\
2.22799999999998	43.0134076845098\\
2.22999999999998	43.0054059816156\\
2.23199999999998	42.9974023599551\\
2.23399999999998	42.989396821492\\
2.23599999999998	42.9813893683716\\
2.23799999999998	42.9733800029157\\
2.23999999999997	42.9653687276205\\
2.24199999999997	42.9573555451549\\
2.24399999999997	42.949340458358\\
2.24599999999997	42.9413234702357\\
2.24799999999997	42.9333045839586\\
2.24999999999997	42.9252838028605\\
2.25199999999997	42.917261130435\\
2.25399999999997	42.909236570334\\
2.25599999999997	42.901210126364\\
2.25799999999997	42.8931818024857\\
2.25999999999997	42.8851516028107\\
2.26199999999997	42.8771195315988\\
2.26399999999997	42.8690855932561\\
2.26599999999997	42.8610497923341\\
2.26799999999997	42.8530121335246\\
2.26999999999997	42.8449726216623\\
2.27199999999997	42.8369312617173\\
2.27399999999997	42.8288880587955\\
2.27599999999997	42.8208430181384\\
2.27799999999997	42.8127961451173\\
2.27999999999997	42.8047474452335\\
2.28199999999997	42.796696924116\\
2.28399999999997	42.7886445875204\\
2.28599999999997	42.7805904413244\\
2.28799999999997	42.7725344915267\\
2.28999999999997	42.7644767442482\\
2.29199999999997	42.7564172057265\\
2.29399999999997	42.7483558823135\\
2.29599999999997	42.7402927804772\\
2.29799999999997	42.7322279067975\\
2.29999999999997	42.7241612679626\\
2.30199999999997	42.7160928707733\\
2.30399999999997	42.7080227221322\\
2.30599999999997	42.6999508290501\\
2.30799999999997	42.6918771986401\\
2.30999999999997	42.683801838117\\
2.31199999999997	42.6757247547955\\
2.31399999999997	42.6676459560864\\
2.31599999999997	42.6595654494992\\
2.31799999999997	42.6514832426376\\
2.31999999999997	42.6433993431968\\
2.32199999999997	42.6353137589645\\
2.32399999999997	42.6272264978189\\
2.32599999999997	42.6191375677251\\
2.32799999999997	42.611046976735\\
2.32999999999996	42.6029547329856\\
2.33199999999996	42.5948608446978\\
2.33399999999996	42.5867653201735\\
2.33599999999996	42.578668167795\\
2.33799999999996	42.5705693960252\\
2.33999999999996	42.5624690134023\\
2.34199999999996	42.5543670285412\\
2.34399999999996	42.5462634501317\\
2.34599999999996	42.5381582869354\\
2.34799999999996	42.5300515477876\\
2.34999999999996	42.5219432415916\\
2.35199999999996	42.5138333773201\\
2.35399999999996	42.5057219640146\\
2.35599999999996	42.4976090107803\\
2.35799999999996	42.4894945267891\\
2.35999999999996	42.481378521275\\
2.36199999999996	42.4732610035342\\
2.36399999999996	42.4651419829245\\
2.36599999999996	42.4570214688621\\
2.36799999999996	42.4488994708218\\
2.36999999999996	42.4407759983355\\
2.37199999999996	42.4326510609891\\
2.37399999999996	42.4245246684247\\
2.37599999999996	42.4163968303376\\
2.37799999999996	42.4082675564733\\
2.37999999999996	42.4001368566302\\
2.38199999999996	42.3920047406538\\
2.38399999999996	42.3838712184401\\
2.38599999999996	42.3757362999311\\
2.38799999999996	42.3675999951146\\
2.38999999999996	42.3594623140248\\
2.39199999999996	42.3513232667377\\
2.39399999999996	42.3431828633736\\
2.39599999999996	42.3350411140933\\
2.39799999999996	42.3268980290986\\
2.39999999999996	42.3187536186292\\
2.40199999999996	42.3106078929666\\
2.40399999999996	42.3024608624251\\
2.40599999999996	42.2943125373591\\
2.40799999999996	42.2861629281552\\
2.40999999999996	42.2780120452361\\
2.41199999999996	42.269859899057\\
2.41399999999996	42.2617065001051\\
2.41599999999996	42.2535518588989\\
2.41799999999996	42.2453959859868\\
2.41999999999996	42.2372388919467\\
2.42199999999995	42.2290805873852\\
2.42399999999995	42.2209210829349\\
2.42599999999995	42.2127603892565\\
2.42799999999995	42.2045985170348\\
2.42999999999995	42.1964354769796\\
2.43199999999995	42.1882712798245\\
2.43399999999995	42.1801059363259\\
2.43599999999995	42.1719394572616\\
2.43799999999995	42.163771853431\\
2.43999999999995	42.155603135653\\
2.44199999999995	42.1474333147659\\
2.44399999999995	42.1392624016275\\
2.44599999999995	42.1310904071117\\
2.44799999999995	42.1229173421106\\
2.44999999999995	42.1147432175311\\
2.45199999999995	42.1065680442954\\
2.45399999999995	42.0983918333407\\
2.45599999999995	42.0902145956185\\
2.45799999999995	42.0820363420906\\
2.45999999999995	42.0738570837337\\
2.46199999999995	42.0656768315335\\
2.46399999999995	42.0574955964882\\
2.46599999999995	42.0493133896049\\
2.46799999999995	42.0411302218988\\
2.46999999999995	42.0329461043956\\
2.47199999999995	42.0247610481264\\
2.47399999999995	42.0165750641318\\
2.47599999999995	42.0083881634563\\
2.47799999999995	42.0002003571509\\
2.47999999999995	41.9920116562725\\
2.48199999999995	41.98382207188\\
2.48399999999995	41.9756316150379\\
2.48599999999995	41.9674402968123\\
2.48799999999995	41.9592481282723\\
2.48999999999995	41.951055120489\\
2.49199999999995	41.9428612845332\\
2.49399999999995	41.9346666314772\\
2.49599999999995	41.926471172392\\
2.49799999999995	41.9182749183492\\
2.49999999999995	41.9100778804183\\
2.50199999999995	41.9018800696662\\
2.50399999999995	41.8936814971579\\
2.50599999999995	41.8854821739555\\
2.50799999999995	41.8772821111157\\
2.50999999999995	41.8690813196927\\
2.51199999999994	41.8608798107355\\
2.51399999999994	41.8526775952874\\
2.51599999999994	41.8444746843853\\
2.51799999999994	41.8362710890605\\
2.51999999999994	41.8280668203365\\
2.52199999999994	41.8198618892305\\
2.52399999999994	41.8116563067502\\
2.52599999999994	41.8034500838955\\
2.52799999999994	41.7952432316579\\
2.52999999999994	41.787035761018\\
2.53199999999994	41.7788276829472\\
2.53399999999994	41.7706190084071\\
2.53599999999994	41.7624097483475\\
2.53799999999994	41.7541999137061\\
2.53999999999994	41.745989515411\\
2.54199999999994	41.7377785643756\\
2.54399999999994	41.7295670715023\\
2.54599999999994	41.7213550476793\\
2.54799999999994	41.713142503782\\
2.54999999999994	41.7049294506706\\
2.55199999999994	41.6967158991918\\
2.55399999999994	41.6885018601779\\
2.55599999999994	41.6802873444445\\
2.55799999999994	41.6720723627931\\
2.55999999999994	41.6638569260075\\
2.56199999999994	41.6556410448564\\
2.56399999999994	41.6474247300907\\
2.56599999999994	41.6392079924453\\
2.56799999999994	41.6309908426366\\
2.56999999999994	41.6227732913636\\
2.57199999999994	41.6145553493055\\
2.57399999999994	41.6063370271255\\
2.57599999999994	41.5981183354658\\
2.57799999999994	41.5898992849494\\
2.57999999999994	41.5816798861808\\
2.58199999999994	41.5734601497435\\
2.58399999999994	41.565240086201\\
2.58599999999994	41.5570197060959\\
2.58799999999994	41.5487990199501\\
2.58999999999994	41.5405780382638\\
2.59199999999994	41.5323567715167\\
2.59399999999994	41.5241352301655\\
2.59599999999994	41.5159134246449\\
2.59799999999994	41.5076913653677\\
2.59999999999994	41.4994690627235\\
2.60199999999994	41.491246527079\\
2.60399999999993	41.4830237687777\\
2.60599999999993	41.4748007981386\\
2.60799999999993	41.4665776254586\\
2.60999999999993	41.458354261009\\
2.61199999999993	41.4501307150374\\
2.61399999999993	41.4419069977666\\
2.61599999999993	41.4336831193949\\
2.61799999999993	41.4254590900944\\
2.61999999999993	41.4172349200133\\
2.62199999999993	41.4090106192734\\
2.62399999999993	41.4007861979712\\
2.62599999999993	41.3925616661762\\
2.62799999999993	41.3843370339328\\
2.62999999999993	41.3761123112582\\
2.63199999999993	41.367887508143\\
2.63399999999993	41.3596626345509\\
2.63599999999993	41.3514377004193\\
2.63799999999993	41.3432127156568\\
2.63999999999993	41.3349876901461\\
2.64199999999993	41.3267626337414\\
2.64399999999993	41.3185375562685\\
2.64599999999993	41.3103124675261\\
2.64799999999993	41.3020873772844\\
2.64999999999993	41.293862295285\\
2.65199999999993	41.2856372312408\\
2.65399999999993	41.2774121948369\\
2.65599999999993	41.2691871957281\\
2.65799999999993	41.2609622435407\\
2.65999999999993	41.2527373478726\\
2.66199999999993	41.2445125182907\\
2.66399999999993	41.2362877643336\\
2.66599999999993	41.2280630955096\\
2.66799999999993	41.219838521298\\
2.66999999999993	41.2116140511467\\
2.67199999999993	41.2033896944747\\
2.67399999999993	41.1951654606704\\
2.67599999999993	41.1869413590919\\
2.67799999999993	41.1787173990661\\
2.67999999999993	41.1704935898903\\
2.68199999999993	41.162269940831\\
2.68399999999993	41.1540464611229\\
2.68599999999993	41.1458231599711\\
2.68799999999993	41.1376000465477\\
2.68999999999993	41.1293771299953\\
2.69199999999993	41.121154419425\\
2.69399999999992	41.1129319239157\\
2.69599999999992	41.1047096525152\\
2.69799999999992	41.0964876142401\\
2.69999999999992	41.0882658180744\\
2.70199999999992	41.0800442729719\\
2.70399999999992	41.0718229878532\\
2.70599999999992	41.0636019716067\\
2.70799999999992	41.0553812330902\\
2.70999999999992	41.0471607811282\\
2.71199999999992	41.0389406245134\\
2.71399999999992	41.0307207720065\\
2.71599999999992	41.022501232336\\
2.71799999999992	41.014282014197\\
2.71999999999992	41.006063126253\\
2.72199999999992	40.9978445771353\\
2.72399999999992	40.989626375442\\
2.72599999999992	40.9814085297383\\
2.72799999999992	40.9731910485572\\
2.72999999999992	40.9649739403996\\
2.73199999999992	40.9567572137323\\
2.73399999999992	40.9485408769902\\
2.73599999999992	40.9403249385754\\
2.73799999999992	40.9321094068559\\
2.73999999999992	40.9238942901679\\
2.74199999999992	40.915679596815\\
2.74399999999992	40.9074653350661\\
2.74599999999992	40.899251513158\\
2.74799999999992	40.8910381392944\\
2.74999999999992	40.8828252216463\\
2.75199999999992	40.8746127683513\\
2.75399999999992	40.8664007875129\\
2.75599999999992	40.8581892872022\\
2.75799999999992	40.849978275458\\
2.75999999999992	40.8417677602838\\
2.76199999999992	40.8335577496511\\
2.76399999999992	40.8253482514985\\
2.76599999999992	40.8171392737302\\
2.76799999999992	40.8089308242177\\
2.76999999999992	40.8007229107997\\
2.77199999999992	40.7925155412805\\
2.77399999999992	40.7843087234322\\
2.77599999999992	40.7761024649924\\
2.77799999999992	40.7678967736671\\
2.77999999999992	40.7596916571268\\
2.78199999999992	40.7514871230105\\
2.78399999999991	40.7432831789229\\
2.78599999999991	40.735079832436\\
2.78799999999991	40.7268770910879\\
2.78999999999991	40.7186749623836\\
2.79199999999991	40.7104734537956\\
2.79399999999991	40.7022725727625\\
2.79599999999991	40.6940723266889\\
2.79799999999991	40.6858727229472\\
2.79999999999991	40.6776737688769\\
2.80199999999991	40.669475471783\\
2.80399999999991	40.6612778389386\\
2.80599999999991	40.6530808775824\\
2.80799999999991	40.6448845949216\\
2.80999999999991	40.6366889981287\\
2.81199999999991	40.6284940943441\\
2.81399999999991	40.6202998906745\\
2.81599999999991	40.6121063941944\\
2.81799999999991	40.603913611945\\
2.81999999999991	40.5957215509334\\
2.82199999999991	40.5875302181358\\
2.82399999999991	40.5793396204939\\
2.82599999999991	40.5711497649176\\
2.82799999999991	40.5629606582826\\
2.82999999999991	40.5547723074343\\
2.83199999999991	40.5465847191826\\
2.83399999999991	40.5383979003057\\
2.83599999999991	40.5302118575501\\
2.83799999999991	40.5220265976288\\
2.83999999999991	40.5138421272224\\
2.84199999999991	40.5056584529787\\
2.84399999999991	40.4974755815134\\
2.84599999999991	40.4892935194102\\
2.84799999999991	40.4811122732191\\
2.84999999999991	40.4729318494586\\
2.85199999999991	40.4647522546157\\
2.85399999999991	40.4565734951436\\
2.85599999999991	40.4483955774647\\
2.85799999999991	40.4402185079678\\
2.85999999999991	40.4320422930118\\
2.86199999999991	40.4238669389215\\
2.86399999999991	40.4156924519905\\
2.86599999999991	40.4075188384813\\
2.86799999999991	40.399346104623\\
2.86999999999991	40.3911742566147\\
2.87199999999991	40.3830033006226\\
2.87399999999991	40.3748332427809\\
2.8759999999999	40.366664089194\\
2.8779999999999	40.358495845933\\
2.8799999999999	40.3503285190381\\
2.8819999999999	40.342162114519\\
2.8839999999999	40.3339966383523\\
2.8859999999999	40.3258320964848\\
2.8879999999999	40.317668494832\\
2.8899999999999	40.3095058392775\\
2.8919999999999	40.3013441356745\\
2.8939999999999	40.2931833898452\\
2.8959999999999	40.2850236075799\\
2.8979999999999	40.2768647946396\\
2.8999999999999	40.2687069567541\\
2.9019999999999	40.2605500996211\\
2.9039999999999	40.2523942289101\\
2.9059999999999	40.2442393502579\\
2.9079999999999	40.2360854692715\\
2.9099999999999	40.2279325915284\\
2.9119999999999	40.2197807225746\\
2.9139999999999	40.2116298679269\\
2.9159999999999	40.2034800330702\\
2.9179999999999	40.1953312234614\\
2.9199999999999	40.1871834445265\\
2.9219999999999	40.1790367016608\\
2.9239999999999	40.1708910002315\\
2.9259999999999	40.1627463455738\\
2.9279999999999	40.1546027429952\\
2.9299999999999	40.1464601977725\\
2.9319999999999	40.1383187151539\\
2.9339999999999	40.130178300357\\
2.9359999999999	40.1220389585703\\
2.9379999999999	40.1139006949539\\
2.9399999999999	40.1057635146373\\
2.9419999999999	40.0976274227225\\
2.9439999999999	40.089492424281\\
2.9459999999999	40.0813585243565\\
2.9479999999999	40.0732257279626\\
2.9499999999999	40.0650940400863\\
2.9519999999999	40.0569634656829\\
2.9539999999999	40.0488340096814\\
2.9559999999999	40.0407056769815\\
2.9579999999999	40.0325784724549\\
2.9599999999999	40.0244524009436\\
2.9619999999999	40.0163274672632\\
2.9639999999999	40.0082036761998\\
2.96599999999989	40.0000810325122\\
2.96799999999989	39.9919595409307\\
2.96999999999989	39.9838392061582\\
2.97199999999989	39.9757200328692\\
2.97399999999989	39.9676020257105\\
2.97599999999989	39.9594851893022\\
2.97799999999989	39.9513695282357\\
2.97999999999989	39.9432550470758\\
2.98199999999989	39.9351417503591\\
2.98399999999989	39.9270296425959\\
2.98599999999989	39.918918728269\\
2.98799999999989	39.9108090118328\\
2.98999999999989	39.9027004977173\\
2.99199999999989	39.8945931903233\\
2.99399999999989	39.8864870940264\\
2.99599999999989	39.8783822131735\\
2.99799999999989	39.8702785520871\\
2.99999999999989	39.8621761150616\\
3.00199999999989	39.8540749063661\\
3.00399999999989	39.8459749302422\\
3.00599999999989	39.8378761909061\\
3.00799999999989	39.8297786925475\\
3.00999999999989	39.8216824393304\\
3.01199999999989	39.813587435392\\
3.01399999999989	39.8054936848443\\
3.01599999999989	39.7974011917735\\
3.01799999999989	39.7893099602399\\
3.01999999999989	39.7812199942781\\
3.02199999999989	39.7731312978973\\
3.02399999999989	39.7650438750817\\
3.02599999999989	39.7569577297893\\
3.02799999999989	39.7488728659537\\
3.02999999999989	39.7407892874827\\
3.03199999999989	39.7327069982598\\
3.03399999999989	39.7246260021434\\
3.03599999999989	39.7165463029663\\
3.03799999999989	39.708467904537\\
3.03999999999989	39.7003908106398\\
3.04199999999989	39.6923150250342\\
3.04399999999989	39.6842405514545\\
3.04599999999989	39.6761673936118\\
3.04799999999989	39.668095555192\\
3.04999999999989	39.6600250398572\\
3.05199999999989	39.6519558512454\\
3.05399999999989	39.6438879929706\\
3.05599999999989	39.6358214686227\\
3.05799999999988	39.6277562817685\\
3.05999999999988	39.61969243595\\
3.06199999999988	39.6116299346863\\
3.06399999999988	39.6035687814729\\
3.06599999999988	39.595508979782\\
3.06799999999988	39.5874505330619\\
3.06999999999988	39.5793934447379\\
3.07199999999988	39.5713377182131\\
3.07399999999988	39.5632833568661\\
3.07599999999988	39.5552303640532\\
3.07799999999988	39.547178743108\\
3.07999999999988	39.5391284973414\\
3.08199999999988	39.5310796300409\\
3.08399999999988	39.5230321444723\\
3.08599999999988	39.5149860438787\\
3.08799999999988	39.5069413314806\\
3.08999999999988	39.498898010476\\
3.09199999999988	39.4908560840409\\
3.09399999999988	39.4828155553298\\
3.09599999999988	39.4747764274743\\
3.09799999999988	39.4667387035847\\
3.09999999999988	39.458702386749\\
3.10199999999988	39.4506674800334\\
3.10399999999988	39.442633986484\\
3.10599999999988	39.4346019091235\\
3.10799999999988	39.4265712509531\\
3.10999999999988	39.4185420149542\\
3.11199999999988	39.4105142040853\\
3.11399999999988	39.4024878212855\\
3.11599999999988	39.3944628694707\\
3.11799999999988	39.386439351538\\
3.11999999999988	39.3784172703622\\
3.12199999999988	39.3703966287972\\
3.12399999999988	39.3623774296769\\
3.12599999999988	39.354359675815\\
3.12799999999988	39.3463433700029\\
3.12999999999988	39.3383285150135\\
3.13199999999988	39.330315113598\\
3.13399999999988	39.3223031684886\\
3.13599999999988	39.3142926823961\\
3.13799999999988	39.3062836580116\\
3.13999999999988	39.2982760980072\\
3.14199999999988	39.2902700050331\\
3.14399999999988	39.2822653817225\\
3.14599999999988	39.2742622306859\\
3.14799999999987	39.2662605545161\\
3.14999999999987	39.2582603557865\\
3.15199999999987	39.25026163705\\
3.15399999999987	39.2422644008404\\
3.15599999999987	39.2342686496722\\
3.15799999999987	39.2262743860425\\
3.15999999999987	39.2182816124261\\
3.16199999999987	39.2102903312816\\
3.16399999999987	39.202300545047\\
3.16599999999987	39.1943122561428\\
3.16799999999987	39.1863254669697\\
3.16999999999987	39.1783401799104\\
3.17199999999987	39.1703563973289\\
3.17399999999987	39.1623741215705\\
3.17599999999987	39.1543933549627\\
3.17799999999987	39.1464140998139\\
3.17999999999987	39.1384363584152\\
3.18199999999987	39.1304601330391\\
3.18399999999987	39.1224854259404\\
3.18599999999987	39.1145122393547\\
3.18799999999987	39.1065405755016\\
3.18999999999987	39.0985704365825\\
3.19199999999987	39.09060182478\\
3.19399999999987	39.0826347422604\\
3.19599999999987	39.0746691911713\\
3.19799999999987	39.0667051736445\\
3.19999999999987	39.058742691793\\
3.20199999999987	39.0507817477132\\
3.20399999999987	39.042822343484\\
3.20599999999987	39.0348644811677\\
3.20799999999987	39.0269081628095\\
3.20999999999987	39.018953390437\\
3.21199999999987	39.0110001660622\\
3.21399999999987	39.0030484916789\\
3.21599999999987	38.9950983692664\\
3.21799999999987	38.9871498007843\\
3.21999999999987	38.9792027881792\\
3.22199999999987	38.971257333378\\
3.22399999999987	38.9633134382941\\
3.22599999999987	38.9553711048224\\
3.22799999999987	38.9474303348436\\
3.22999999999987	38.9394911302205\\
3.23199999999987	38.9315534928013\\
3.23399999999987	38.9236174244175\\
3.23599999999987	38.9156829268854\\
3.23799999999986	38.9077500020042\\
3.23999999999986	38.8998186515593\\
3.24199999999986	38.8918888773194\\
3.24399999999986	38.8839606810377\\
3.24599999999986	38.8760340644521\\
3.24799999999986	38.8681090292852\\
3.24999999999986	38.8601855772442\\
3.25199999999986	38.8522637100217\\
3.25399999999986	38.844343429294\\
3.25599999999986	38.8364247367237\\
3.25799999999986	38.8285076339569\\
3.25999999999986	38.8205921226268\\
3.26199999999986	38.8126782043495\\
3.26399999999986	38.8047658807283\\
3.26599999999986	38.7968551533509\\
3.26799999999986	38.7889460237908\\
3.26999999999986	38.7810384936064\\
3.27199999999986	38.7731325643423\\
3.27399999999986	38.7652282375289\\
3.27599999999986	38.7573255146817\\
3.27799999999986	38.7494243973018\\
3.27999999999986	38.7415248868775\\
3.28199999999986	38.7336269848819\\
3.28399999999986	38.725730692774\\
3.28599999999986	38.717836012\\
3.28799999999986	38.7099429439913\\
3.28999999999986	38.7020514901656\\
3.29199999999986	38.6941616519278\\
3.29399999999986	38.6862734306685\\
3.29599999999986	38.6783868277642\\
3.29799999999986	38.6705018445795\\
3.29999999999986	38.6626184824643\\
3.30199999999986	38.6547367427555\\
3.30399999999986	38.6468566267773\\
3.30599999999986	38.6389781358401\\
3.30799999999986	38.6311012712411\\
3.30999999999986	38.6232260342654\\
3.31199999999986	38.6153524261842\\
3.31399999999986	38.6074804482565\\
3.31599999999986	38.5996101017275\\
3.31799999999986	38.5917413878311\\
3.31999999999986	38.5838743077871\\
3.32199999999986	38.5760088628036\\
3.32399999999986	38.5681450540766\\
3.32599999999986	38.5602828827876\\
3.32799999999986	38.552422350108\\
3.32999999999985	38.5445634571958\\
3.33199999999985	38.536706205197\\
3.33399999999985	38.528850595245\\
3.33599999999985	38.5209966284615\\
3.33799999999985	38.5131443059559\\
3.33999999999985	38.5052936288262\\
3.34199999999985	38.4974445981577\\
3.34399999999985	38.4895972150239\\
3.34599999999985	38.4817514804874\\
3.34799999999985	38.473907395598\\
3.34999999999985	38.4660649613947\\
3.35199999999985	38.4582241789042\\
3.35399999999985	38.4503850491423\\
3.35599999999985	38.4425475731129\\
3.35799999999985	38.4347117518087\\
3.35999999999985	38.426877586211\\
3.36199999999985	38.41904507729\\
3.36399999999985	38.4112142260045\\
3.36599999999985	38.4033850333024\\
3.36799999999985	38.3955575001202\\
3.36999999999985	38.3877316273839\\
3.37199999999985	38.379907416008\\
3.37399999999985	38.3720848668961\\
3.37599999999985	38.364263980942\\
3.37799999999985	38.356444759027\\
3.37999999999985	38.3486272020236\\
3.38199999999985	38.3408113107925\\
3.38399999999985	38.3329970861839\\
3.38599999999985	38.3251845290376\\
3.38799999999985	38.3173736401836\\
3.38999999999985	38.3095644204402\\
3.39199999999985	38.3017568706169\\
3.39399999999985	38.2939509915119\\
3.39599999999985	38.2861467839133\\
3.39799999999985	38.2783442485993\\
3.39999999999985	38.2705433863382\\
3.40199999999985	38.2627441978876\\
3.40399999999985	38.2549466839956\\
3.40599999999985	38.2471508454005\\
3.40799999999985	38.2393566828309\\
3.40999999999985	38.2315641970047\\
3.41199999999985	38.2237733886307\\
3.41399999999985	38.2159842584084\\
3.41599999999985	38.2081968070272\\
3.41799999999985	38.2004110351669\\
3.41999999999984	38.1926269434978\\
3.42199999999984	38.1848445326808\\
3.42399999999984	38.1770638033676\\
3.42599999999984	38.1692847562005\\
3.42799999999984	38.1615073918125\\
3.42999999999984	38.1537317108269\\
3.43199999999984	38.1459577138586\\
3.43399999999984	38.1381854015132\\
3.43599999999984	38.1304147743871\\
3.43799999999984	38.1226458330672\\
3.43999999999984	38.1148785781322\\
3.44199999999984	38.1071130101517\\
3.44399999999984	38.0993491296864\\
3.44599999999984	38.0915869372881\\
3.44799999999984	38.0838264335001\\
3.44999999999984	38.0760676188572\\
3.45199999999984	38.0683104938849\\
3.45399999999984	38.0605550591002\\
3.45599999999984	38.0528013150126\\
3.45799999999984	38.0450492621218\\
3.45999999999984	38.03729890092\\
3.46199999999984	38.029550231891\\
3.46399999999984	38.0218032555096\\
3.46599999999984	38.0140579722427\\
3.46799999999984	38.0063143825497\\
3.46999999999984	37.9985724868804\\
3.47199999999984	37.9908322856774\\
3.47399999999984	37.9830937793758\\
3.47599999999984	37.9753569684011\\
3.47799999999984	37.9676218531721\\
3.47999999999984	37.9598884340999\\
3.48199999999984	37.9521567115862\\
3.48399999999984	37.9444266860267\\
3.48599999999984	37.9366983578084\\
3.48799999999984	37.9289717273099\\
3.48999999999984	37.9212467949039\\
3.49199999999984	37.913523560954\\
3.49399999999984	37.9058020258165\\
3.49599999999984	37.8980821898413\\
3.49799999999984	37.8903640533691\\
3.49999999999984	37.8826476167344\\
3.50199999999984	37.8749328802639\\
3.50399999999984	37.8672198442772\\
3.50599999999984	37.8595085090858\\
3.50799999999984	37.8517988749953\\
3.50999999999984	37.8440909423031\\
3.51199999999983	37.8363847112997\\
3.51399999999983	37.8286801822686\\
3.51599999999983	37.8209773554864\\
3.51799999999983	37.8132762312219\\
3.51999999999983	37.805576809738\\
3.52199999999983	37.7978790912902\\
3.52399999999983	37.790183076127\\
3.52599999999983	37.7824887644901\\
3.52799999999983	37.7747961566146\\
3.52999999999983	37.767105252729\\
3.53199999999983	37.7594160530544\\
3.53399999999983	37.7517285578062\\
3.53599999999983	37.744042767192\\
3.53799999999983	37.7363586814144\\
3.53999999999983	37.7286763006678\\
3.54199999999983	37.7209956251416\\
3.54399999999983	37.7133166550177\\
3.54599999999983	37.7056393904717\\
3.54799999999983	37.6979638316737\\
3.54999999999983	37.6902899787865\\
3.55199999999983	37.6826178319671\\
3.55399999999983	37.6749473913667\\
3.55599999999983	37.6672786571292\\
3.55799999999983	37.6596116293926\\
3.55999999999983	37.6519463082903\\
3.56199999999983	37.6442826939477\\
3.56399999999983	37.6366207864849\\
3.56599999999983	37.6289605860166\\
3.56799999999983	37.6213020926508\\
3.56999999999983	37.6136453064895\\
3.57199999999983	37.6059902276293\\
3.57399999999983	37.5983368561608\\
3.57599999999983	37.5906851921693\\
3.57799999999983	37.5830352357332\\
3.57999999999983	37.575386986926\\
3.58199999999983	37.5677404458159\\
3.58399999999983	37.5600956124643\\
3.58599999999983	37.5524524869275\\
3.58799999999983	37.5448110692578\\
3.58999999999983	37.5371713594988\\
3.59199999999983	37.5295333576915\\
3.59399999999983	37.52189706387\\
3.59599999999983	37.5142624780636\\
3.59799999999983	37.5066296002952\\
3.59999999999983	37.4989984305838\\
3.60199999999982	37.4913689689426\\
3.60399999999982	37.483741215379\\
3.60599999999982	37.4761151698954\\
3.60799999999982	37.4684908324894\\
3.60999999999982	37.460868203153\\
3.61199999999982	37.4532472818737\\
3.61399999999982	37.4456280686331\\
3.61599999999982	37.4380105634083\\
3.61799999999982	37.4303947661711\\
3.61999999999982	37.4227806768886\\
3.62199999999982	37.4151682955226\\
3.62399999999982	37.4075576220306\\
3.62599999999982	37.3999486563644\\
3.62799999999982	37.3923413984718\\
3.62999999999982	37.3847358482951\\
3.63199999999982	37.3771320057722\\
3.63399999999982	37.3695298708362\\
3.63599999999982	37.3619294434151\\
3.63799999999982	37.3543307234333\\
3.63999999999982	37.3467337108096\\
3.64199999999982	37.3391384054579\\
3.64399999999982	37.3315448072886\\
3.64599999999982	37.3239529162073\\
3.64799999999982	37.316362732114\\
3.64999999999982	37.3087742549055\\
3.65199999999982	37.3011874844742\\
3.65399999999982	37.293602420707\\
3.65599999999982	37.2860190634871\\
3.65799999999982	37.2784374126932\\
3.65999999999982	37.2708574682001\\
3.66199999999982	37.2632792298777\\
3.66399999999982	37.255702697592\\
3.66599999999982	37.2481278712045\\
3.66799999999982	37.2405547505729\\
3.66999999999982	37.2329833355504\\
3.67199999999982	37.225413625986\\
3.67399999999982	37.2178456217252\\
3.67599999999982	37.2102793226089\\
3.67799999999982	37.2027147284741\\
3.67999999999982	37.1951518391536\\
3.68199999999982	37.1875906544767\\
3.68399999999982	37.1800311742679\\
3.68599999999982	37.1724733983489\\
3.68799999999982	37.1649173265362\\
3.68999999999982	37.1573629586437\\
3.69199999999981	37.1498102944809\\
3.69399999999981	37.1422593338532\\
3.69599999999981	37.1347100765622\\
3.69799999999981	37.1271625224071\\
3.69999999999981	37.1196166711813\\
3.70199999999981	37.112072522676\\
3.70399999999981	37.1045300766781\\
3.70599999999981	37.096989332971\\
3.70799999999981	37.0894502913349\\
3.70999999999981	37.0819129515453\\
3.71199999999981	37.0743773133758\\
3.71399999999981	37.0668433765945\\
3.71599999999981	37.0593111409677\\
3.71799999999981	37.0517806062572\\
3.71999999999981	37.0442517722221\\
3.72199999999981	37.0367246386168\\
3.72399999999981	37.029199205194\\
3.72599999999981	37.021675471702\\
3.72799999999981	37.0141534378855\\
3.72999999999981	37.0066331034867\\
3.73199999999981	36.9991144682443\\
3.73399999999981	36.9915975318928\\
3.73599999999981	36.9840822941652\\
3.73799999999981	36.9765687547892\\
3.73999999999981	36.9690569134909\\
3.74199999999981	36.961546769993\\
3.74399999999981	36.9540383240142\\
3.74599999999981	36.9465315752707\\
3.74799999999981	36.9390265234761\\
3.74999999999981	36.9315231683397\\
3.75199999999981	36.9240215095689\\
3.75399999999981	36.9165215468672\\
3.75599999999981	36.9090232799358\\
3.75799999999981	36.9015267084722\\
3.75999999999981	36.894031832172\\
3.76199999999981	36.8865386507265\\
3.76399999999981	36.8790471638256\\
3.76599999999981	36.8715573711547\\
3.76799999999981	36.8640692723978\\
3.76999999999981	36.8565828672356\\
3.77199999999981	36.8490981553451\\
3.77399999999981	36.8416151364016\\
3.77599999999981	36.8341338100774\\
3.77799999999981	36.8266541760419\\
3.77999999999981	36.8191762339616\\
3.78199999999981	36.8116999835006\\
3.7839999999998	36.8042254243202\\
3.7859999999998	36.7967525560794\\
3.7879999999998	36.7892813784339\\
3.7899999999998	36.7818118910369\\
3.7919999999998	36.7743440935395\\
3.7939999999998	36.7668779855898\\
3.7959999999998	36.7594135668341\\
3.7979999999998	36.7519508369147\\
3.7999999999998	36.7444897954723\\
3.8019999999998	36.737030442146\\
3.8039999999998	36.7295727765705\\
3.8059999999998	36.7221167983796\\
3.8079999999998	36.7146625072039\\
3.8099999999998	36.7072099026725\\
3.8119999999998	36.6997589844105\\
3.8139999999998	36.6923097520421\\
3.8159999999998	36.6848622051882\\
3.8179999999998	36.6774163434687\\
3.8199999999998	36.6699721664995\\
3.8219999999998	36.6625296738952\\
3.8239999999998	36.6550888652679\\
3.8259999999998	36.6476497402284\\
3.8279999999998	36.6402122983831\\
3.8299999999998	36.6327765393384\\
3.8319999999998	36.6253424626971\\
3.8339999999998	36.6179100680603\\
3.8359999999998	36.6104793550276\\
3.8379999999998	36.6030503231957\\
3.8399999999998	36.5956229721586\\
3.8419999999998	36.5881973015097\\
3.8439999999998	36.5807733108393\\
3.8459999999998	36.573350999736\\
3.8479999999998	36.5659303677862\\
3.8499999999998	36.5585114145741\\
3.8519999999998	36.5510941396829\\
3.8539999999998	36.5436785426928\\
3.8559999999998	36.5362646231816\\
3.8579999999998	36.5288523807267\\
3.8599999999998	36.5214418149025\\
3.8619999999998	36.5140329252815\\
3.8639999999998	36.506625711435\\
3.8659999999998	36.4992201729312\\
3.8679999999998	36.4918163093378\\
3.8699999999998	36.48441412022\\
3.8719999999998	36.4770136051407\\
3.87399999999979	36.4696147636621\\
3.87599999999979	36.4622175953434\\
3.87799999999979	36.4548220997434\\
3.87999999999979	36.4474282764178\\
3.88199999999979	36.4400361249212\\
3.88399999999979	36.4326456448063\\
3.88599999999979	36.4252568356246\\
3.88799999999979	36.417869696925\\
3.88999999999979	36.4104842282554\\
3.89199999999979	36.4031004291618\\
3.89399999999979	36.3957182991888\\
3.89599999999979	36.3883378378786\\
3.89799999999979	36.3809590447733\\
3.89999999999979	36.3735819194118\\
3.90199999999979	36.3662064613316\\
3.90399999999979	36.3588326700701\\
3.90599999999979	36.3514605451615\\
3.90799999999979	36.344090086139\\
3.90999999999979	36.3367212925348\\
3.91199999999979	36.3293541638792\\
3.91399999999979	36.3219886997005\\
3.91599999999979	36.3146248995261\\
3.91799999999979	36.307262762882\\
3.91999999999979	36.2999022892924\\
3.92199999999979	36.2925434782803\\
3.92399999999979	36.2851863293672\\
3.92599999999979	36.2778308420733\\
3.92799999999979	36.2704770159173\\
3.92999999999979	36.263124850416\\
3.93199999999979	36.2557743450861\\
3.93399999999979	36.248425499442\\
3.93599999999979	36.2410783129967\\
3.93799999999979	36.2337327852621\\
3.93999999999979	36.2263889157493\\
3.94199999999979	36.2190467039676\\
3.94399999999979	36.2117061494244\\
3.94599999999979	36.2043672516272\\
3.94799999999979	36.197030010081\\
3.94999999999979	36.1896944242906\\
3.95199999999979	36.1823604937589\\
3.95399999999979	36.1750282179875\\
3.95599999999979	36.1676975964772\\
3.95799999999979	36.1603686287275\\
3.95999999999979	36.1530413142367\\
3.96199999999979	36.1457156525016\\
3.96399999999979	36.1383916430184\\
3.96599999999978	36.1310692852822\\
3.96799999999978	36.123748578786\\
3.96999999999978	36.1164295230231\\
3.97199999999978	36.1091121174846\\
3.97399999999978	36.1017963616606\\
3.97599999999978	36.0944822550408\\
3.97799999999978	36.0871697971135\\
3.97999999999978	36.0798589873659\\
3.98199999999978	36.0725498252835\\
3.98399999999978	36.0652423103525\\
3.98599999999978	36.0579364420559\\
3.98799999999978	36.0506322198774\\
3.98999999999978	36.0433296432991\\
3.99199999999978	36.0360287118017\\
3.99399999999978	36.0287294248655\\
3.99599999999978	36.0214317819704\\
3.99799999999978	36.0141357825937\\
3.99999999999978	36.0068414262132\\
4.00199999999978	35.9995487123049\\
4.00399999999978	35.9922576403444\\
4.00599999999978	35.9849682098065\\
4.00799999999978	35.9776804201645\\
4.00999999999978	35.9703942708913\\
4.01199999999978	35.963109761459\\
4.01399999999978	35.9558268913382\\
4.01599999999978	35.9485456599995\\
4.01799999999978	35.9412660669122\\
4.01999999999978	35.9339881115447\\
4.02199999999978	35.9267117933645\\
4.02399999999978	35.919437111839\\
4.02599999999978	35.912164066434\\
4.02799999999978	35.9048926566147\\
4.02999999999978	35.8976228818457\\
4.03199999999978	35.8903547415911\\
4.03399999999978	35.8830882353137\\
4.03599999999978	35.8758233624757\\
4.03799999999978	35.8685601225383\\
4.03999999999978	35.8612985149632\\
4.04199999999978	35.8540385392097\\
4.04399999999978	35.8467801947373\\
4.04599999999978	35.839523481005\\
4.04799999999978	35.8322683974707\\
4.04999999999978	35.8250149435916\\
4.05199999999978	35.8177631188242\\
4.05399999999978	35.8105129226246\\
4.05599999999978	35.8032643544484\\
4.05799999999978	35.79601741375\\
4.05999999999977	35.7887720999835\\
4.06199999999977	35.7815284126023\\
4.06399999999977	35.7742863510593\\
4.06599999999977	35.767045914807\\
4.06799999999977	35.7598071032967\\
4.06999999999977	35.7525699159793\\
4.07199999999977	35.7453343523055\\
4.07399999999977	35.7381004117253\\
4.07599999999977	35.7308680936878\\
4.07799999999977	35.7236373976421\\
4.07999999999977	35.716408323036\\
4.08199999999977	35.7091808693175\\
4.08399999999977	35.7019550359333\\
4.08599999999977	35.6947308223311\\
4.08799999999977	35.6875082279559\\
4.08999999999977	35.6802872522541\\
4.09199999999977	35.6730678946708\\
4.09399999999977	35.6658501546501\\
4.09599999999977	35.6586340316369\\
4.09799999999977	35.6514195250743\\
4.09999999999977	35.6442066344062\\
4.10199999999977	35.6369953590751\\
4.10399999999977	35.6297856985234\\
4.10599999999977	35.6225776521931\\
4.10799999999977	35.6153712195256\\
4.10999999999977	35.6081663999617\\
4.11199999999977	35.6009631929429\\
4.11399999999977	35.5937615979088\\
4.11599999999977	35.5865616142993\\
4.11799999999977	35.5793632415542\\
4.11999999999977	35.5721664791125\\
4.12199999999977	35.5649713264126\\
4.12399999999977	35.5577777828934\\
4.12599999999977	35.5505858479924\\
4.12799999999977	35.5433955211474\\
4.12999999999977	35.5362068017961\\
4.13199999999977	35.529019689375\\
4.13399999999977	35.5218341833211\\
4.13599999999977	35.5146502830699\\
4.13799999999977	35.5074679880585\\
4.13999999999977	35.5002872977217\\
4.14199999999977	35.4931082114951\\
4.14399999999977	35.4859307288143\\
4.14599999999977	35.4787548491132\\
4.14799999999977	35.4715805718269\\
4.14999999999976	35.4644078963894\\
4.15199999999976	35.4572368222347\\
4.15399999999976	35.4500673487966\\
4.15599999999976	35.4428994755088\\
4.15799999999976	35.4357332018037\\
4.15999999999976	35.4285685271149\\
4.16199999999976	35.4214054508746\\
4.16399999999976	35.4142439725156\\
4.16599999999976	35.4070840914701\\
4.16799999999976	35.3999258071707\\
4.16999999999976	35.3927691190477\\
4.17199999999976	35.3856140265341\\
4.17399999999976	35.3784605290609\\
4.17599999999976	35.3713086260594\\
4.17799999999976	35.3641583169603\\
4.17999999999976	35.3570096011945\\
4.18199999999976	35.3498624781932\\
4.18399999999976	35.3427169473865\\
4.18599999999976	35.3355730082046\\
4.18799999999976	35.3284306600779\\
4.18999999999976	35.3212899024365\\
4.19199999999976	35.31415073471\\
4.19399999999976	35.3070131563283\\
4.19599999999976	35.2998771667214\\
4.19799999999976	35.2927427653182\\
4.19999999999976	35.285609951548\\
4.20199999999976	35.2784787248405\\
4.20399999999976	35.2713490846245\\
4.20599999999976	35.2642210303295\\
4.20799999999976	35.2570945613838\\
4.20999999999976	35.2499696772167\\
4.21199999999976	35.2428463772566\\
4.21399999999976	35.235724660932\\
4.21599999999976	35.2286045276722\\
4.21799999999976	35.2214859769048\\
4.21999999999976	35.2143690080587\\
4.22199999999976	35.2072536205625\\
4.22399999999976	35.2001398138438\\
4.22599999999976	35.1930275873316\\
4.22799999999976	35.1859169404534\\
4.22999999999976	35.1788078726377\\
4.23199999999976	35.1717003833126\\
4.23399999999976	35.1645944719062\\
4.23599999999976	35.1574901378462\\
4.23799999999976	35.1503873805609\\
4.23999999999976	35.1432861994785\\
4.24199999999975	35.1361865940267\\
4.24399999999975	35.1290885636336\\
4.24599999999975	35.1219921077269\\
4.24799999999975	35.1148972257347\\
4.24999999999975	35.1078039170851\\
4.25199999999975	35.1007121812061\\
4.25399999999975	35.0936220175254\\
4.25599999999975	35.0865334254713\\
4.25799999999975	35.0794464044713\\
4.25999999999975	35.0723609539543\\
4.26199999999975	35.0652770733479\\
4.26399999999975	35.05819476208\\
4.26599999999975	35.0511140195787\\
4.26799999999975	35.0440348452729\\
4.26999999999975	35.0369572385903\\
4.27199999999975	35.0298811989594\\
4.27399999999975	35.0228067258082\\
4.27599999999975	35.0157338185655\\
4.27799999999975	35.0086624766595\\
4.27999999999975	35.0015926995189\\
4.28199999999975	34.9945244865722\\
4.28399999999975	34.9874578372483\\
4.28599999999975	34.9803927509755\\
4.28799999999975	34.9733292271836\\
4.28999999999975	34.9662672653002\\
4.29199999999975	34.9592068647553\\
4.29399999999975	34.9521480249777\\
4.29599999999975	34.945090745397\\
4.29799999999975	34.9380350254421\\
4.29999999999975	34.9309808645423\\
4.30199999999975	34.9239282621274\\
4.30399999999975	34.9168772176274\\
4.30599999999975	34.9098277304718\\
4.30799999999975	34.9027798000899\\
4.30999999999975	34.8957334259128\\
4.31199999999975	34.8886886073699\\
4.31399999999975	34.8816453438918\\
4.31599999999975	34.8746036349087\\
4.31799999999975	34.8675634798514\\
4.31999999999975	34.8605248781506\\
4.32199999999975	34.8534878292371\\
4.32399999999975	34.846452332542\\
4.32599999999975	34.8394183874961\\
4.32799999999975	34.8323859935309\\
4.32999999999975	34.8253551500782\\
4.33199999999974	34.8183258565693\\
4.33399999999974	34.8112981124361\\
4.33599999999974	34.8042719171106\\
4.33799999999974	34.7972472700246\\
4.33999999999974	34.790224170611\\
4.34199999999974	34.7832026183021\\
4.34399999999974	34.7761826125306\\
4.34599999999974	34.7691641527293\\
4.34799999999974	34.7621472383308\\
4.34999999999974	34.7551318687692\\
4.35199999999974	34.7481180434776\\
4.35399999999974	34.7411057618897\\
4.35599999999974	34.7340950234388\\
4.35799999999974	34.7270858275599\\
4.35999999999974	34.7200781736865\\
4.36199999999974	34.7130720612534\\
4.36399999999974	34.7060674896952\\
4.36599999999974	34.6990644584467\\
4.36799999999974	34.6920629669432\\
4.36999999999974	34.6850630146204\\
4.37199999999974	34.678064600913\\
4.37399999999974	34.6710677252571\\
4.37599999999974	34.6640723870892\\
4.37799999999974	34.6570785858447\\
4.37999999999974	34.6500863209608\\
4.38199999999974	34.643095591874\\
4.38399999999974	34.6361063980211\\
4.38599999999974	34.6291187388393\\
4.38799999999974	34.6221326137667\\
4.38999999999974	34.61514802224\\
4.39199999999974	34.6081649636976\\
4.39399999999974	34.6011834375777\\
4.39599999999974	34.5942034433189\\
4.39799999999974	34.5872249803596\\
4.39999999999974	34.580248048139\\
4.40199999999974	34.5732726460961\\
4.40399999999974	34.5662987736707\\
4.40599999999974	34.559326430302\\
4.40799999999974	34.5523556154309\\
4.40999999999974	34.5453863284971\\
4.41199999999974	34.5384185689413\\
4.41399999999974	34.5314523362044\\
4.41599999999974	34.5244876297273\\
4.41799999999974	34.5175244489516\\
4.41999999999974	34.5105627933191\\
4.42199999999974	34.5036026622715\\
4.42399999999973	34.4966440552517\\
4.42599999999973	34.4896869717014\\
4.42799999999973	34.4827314110638\\
4.42999999999973	34.4757773727823\\
4.43199999999973	34.4688248563\\
4.43399999999973	34.4618738610608\\
4.43599999999973	34.4549243865085\\
4.43799999999973	34.4479764320878\\
4.43999999999973	34.4410299972431\\
4.44199999999973	34.4340850814195\\
4.44399999999973	34.4271416840621\\
4.44599999999973	34.4201998046168\\
4.44799999999973	34.4132594425286\\
4.44999999999973	34.406320597245\\
4.45199999999973	34.3993832682114\\
4.45399999999973	34.3924474548751\\
4.45599999999973	34.3855131566835\\
4.45799999999973	34.3785803730835\\
4.45999999999973	34.3716491035231\\
4.46199999999973	34.3647193474505\\
4.46399999999973	34.3577911043139\\
4.46599999999973	34.3508643735626\\
4.46799999999973	34.3439391546453\\
4.46999999999973	34.3370154470112\\
4.47199999999973	34.3300932501108\\
4.47399999999973	34.3231725633935\\
4.47599999999973	34.3162533863097\\
4.47799999999973	34.309335718311\\
4.47999999999973	34.3024195588477\\
4.48199999999973	34.2955049073715\\
4.48399999999973	34.2885917633343\\
4.48599999999973	34.2816801261882\\
4.48799999999973	34.2747699953856\\
4.48999999999973	34.2678613703794\\
4.49199999999973	34.2609542506229\\
4.49399999999973	34.2540486355695\\
4.49599999999973	34.2471445246732\\
4.49799999999973	34.2402419173881\\
4.49999999999973	34.2333408131689\\
4.50199999999973	34.2264412114707\\
4.50399999999973	34.2195431117485\\
4.50599999999973	34.2126465134579\\
4.50799999999973	34.2057514160555\\
4.50999999999973	34.1988578189973\\
4.51199999999973	34.1919657217403\\
4.51399999999972	34.1850751237413\\
4.51599999999972	34.1781860244582\\
4.51799999999972	34.1712984233483\\
4.51999999999972	34.1644123198702\\
4.52199999999972	34.1575277134828\\
4.52399999999972	34.1506446036446\\
4.52599999999972	34.1437629898152\\
4.52799999999972	34.1368828714542\\
4.52999999999972	34.1300042480217\\
4.53199999999972	34.1231271189782\\
4.53399999999972	34.116251483785\\
4.53599999999972	34.1093773419025\\
4.53799999999972	34.1025046927927\\
4.53999999999972	34.0956335359181\\
4.54199999999972	34.0887638707402\\
4.54399999999972	34.0818956967223\\
4.54599999999972	34.0750290133275\\
4.54799999999972	34.0681638200194\\
4.54999999999972	34.0613001162615\\
4.55199999999972	34.0544379015187\\
4.55399999999972	34.0475771752552\\
4.55599999999972	34.0407179369364\\
4.55799999999972	34.0338601860278\\
4.55999999999972	34.0270039219949\\
4.56199999999972	34.0201491443045\\
4.56399999999972	34.0132958524231\\
4.56599999999972	34.0064440458173\\
4.56799999999972	33.9995937239552\\
4.56999999999972	33.9927448863042\\
4.57199999999972	33.9858975323331\\
4.57399999999972	33.9790516615097\\
4.57599999999972	33.9722072733038\\
4.57799999999972	33.9653643671846\\
4.57999999999972	33.9585229426219\\
4.58199999999972	33.951682999086\\
4.58399999999972	33.9448445360473\\
4.58599999999972	33.9380075529775\\
4.58799999999972	33.9311720493477\\
4.58999999999972	33.9243380246297\\
4.59199999999972	33.9175054782959\\
4.59399999999972	33.9106744098188\\
4.59599999999972	33.903844818672\\
4.59799999999972	33.8970167043282\\
4.59999999999972	33.8901900662621\\
4.60199999999972	33.8833649039476\\
4.60399999999971	33.8765412168597\\
4.60599999999971	33.8697190044734\\
4.60799999999971	33.8628982662641\\
4.60999999999971	33.8560790017083\\
4.61199999999971	33.849261210282\\
4.61399999999971	33.842444891462\\
4.61599999999971	33.8356300447256\\
4.61799999999971	33.8288166695505\\
4.61999999999971	33.8220047654148\\
4.62199999999971	33.8151943317971\\
4.62399999999971	33.808385368176\\
4.62599999999971	33.8015778740309\\
4.62799999999971	33.7947718488415\\
4.62999999999971	33.7879672920882\\
4.63199999999971	33.7811642032514\\
4.63399999999971	33.7743625818121\\
4.63599999999971	33.767562427252\\
4.63799999999971	33.7607637390524\\
4.63999999999971	33.753966516696\\
4.64199999999971	33.7471707596653\\
4.64399999999971	33.7403764674439\\
4.64599999999971	33.7335836395146\\
4.64799999999971	33.7267922753617\\
4.64999999999971	33.7200023744698\\
4.65199999999971	33.7132139363235\\
4.65399999999971	33.7064269604082\\
4.65599999999971	33.6996414462093\\
4.65799999999971	33.6928573932131\\
4.65999999999971	33.6860748009062\\
4.66199999999971	33.6792936687755\\
4.66399999999971	33.6725139963084\\
4.66599999999971	33.6657357829926\\
4.66799999999971	33.6589590283167\\
4.66999999999971	33.652183731769\\
4.67199999999971	33.6454098928389\\
4.67399999999971	33.6386375110158\\
4.67599999999971	33.6318665857897\\
4.67799999999971	33.6250971166507\\
4.67999999999971	33.6183291030904\\
4.68199999999971	33.6115625445994\\
4.68399999999971	33.6047974406697\\
4.68599999999971	33.5980337907936\\
4.68799999999971	33.5912715944635\\
4.68999999999971	33.584510851172\\
4.69199999999971	33.5777515604131\\
4.69399999999971	33.5709937216804\\
4.6959999999997	33.5642373344688\\
4.6979999999997	33.5574823982721\\
4.6999999999997	33.5507289125863\\
4.7019999999997	33.5439768769066\\
4.7039999999997	33.5372262907289\\
4.7059999999997	33.5304771535505\\
4.7079999999997	33.5237294648674\\
4.7099999999997	33.5169832241773\\
4.7119999999997	33.5102384309786\\
4.7139999999997	33.5034950847688\\
4.7159999999997	33.4967531850467\\
4.7179999999997	33.490012731312\\
4.7199999999997	33.4832737230631\\
4.7219999999997	33.4765361598012\\
4.7239999999997	33.4698000410262\\
4.7259999999997	33.4630653662387\\
4.7279999999997	33.4563321349406\\
4.7299999999997	33.4496003466336\\
4.7319999999997	33.4428700008198\\
4.7339999999997	33.4361410970013\\
4.7359999999997	33.429413634682\\
4.7379999999997	33.4226876133648\\
4.7399999999997	33.4159630325539\\
4.7419999999997	33.4092398917537\\
4.7439999999997	33.4025181904692\\
4.7459999999997	33.3957979282055\\
4.7479999999997	33.3890791044684\\
4.7499999999997	33.3823617187638\\
4.7519999999997	33.3756457705985\\
4.7539999999997	33.3689312594797\\
4.7559999999997	33.3622181849145\\
4.7579999999997	33.3555065464111\\
4.7599999999997	33.3487963434778\\
4.7619999999997	33.3420875756234\\
4.7639999999997	33.3353802423574\\
4.7659999999997	33.328674343189\\
4.7679999999997	33.3219698776285\\
4.7699999999997	33.3152668451864\\
4.7719999999997	33.3085652453742\\
4.7739999999997	33.3018650777028\\
4.7759999999997	33.2951663416845\\
4.7779999999997	33.2884690368312\\
4.7799999999997	33.2817731626561\\
4.7819999999997	33.2750787186721\\
4.7839999999997	33.2683857043929\\
4.78599999999969	33.2616941193327\\
4.78799999999969	33.2550039630062\\
4.78999999999969	33.2483152349281\\
4.79199999999969	33.241627934614\\
4.79399999999969	33.2349420615795\\
4.79599999999969	33.2282576153414\\
4.79799999999969	33.221574595416\\
4.79999999999969	33.2148930013207\\
4.80199999999969	33.2082128325732\\
4.80399999999969	33.2015340886913\\
4.80599999999969	33.1948567691936\\
4.80799999999969	33.1881808735995\\
4.80999999999969	33.1815064014278\\
4.81199999999969	33.1748333521985\\
4.81399999999969	33.168161725432\\
4.81599999999969	33.1614915206492\\
4.81799999999969	33.1548227373708\\
4.81999999999969	33.1481553751187\\
4.82199999999969	33.1414894334149\\
4.82399999999969	33.1348249117817\\
4.82599999999969	33.1281618097422\\
4.82799999999969	33.12150012682\\
4.82999999999969	33.1148398625382\\
4.83199999999969	33.108181016422\\
4.83399999999969	33.1015235879952\\
4.83599999999969	33.0948675767836\\
4.83799999999969	33.0882129823122\\
4.83999999999969	33.0815598041072\\
4.84199999999969	33.0749080416953\\
4.84399999999969	33.068257694603\\
4.84599999999969	33.0616087623578\\
4.84799999999969	33.0549612444876\\
4.84999999999969	33.0483151405204\\
4.85199999999969	33.0416704499848\\
4.85399999999969	33.0350271724099\\
4.85599999999969	33.0283853073254\\
4.85799999999969	33.0217448542608\\
4.85999999999969	33.0151058127472\\
4.86199999999969	33.0084681823147\\
4.86399999999969	33.0018319624951\\
4.86599999999969	32.9951971528197\\
4.86799999999969	32.9885637528208\\
4.86999999999969	32.9819317620311\\
4.87199999999969	32.9753011799832\\
4.87399999999969	32.9686720062111\\
4.87599999999969	32.9620442402482\\
4.87799999999968	32.9554178816291\\
4.87999999999968	32.9487929298885\\
4.88199999999968	32.9421693845615\\
4.88399999999968	32.9355472451839\\
4.88599999999968	32.9289265112914\\
4.88799999999968	32.9223071824209\\
4.88999999999968	32.9156892581092\\
4.89199999999968	32.9090727378935\\
4.89399999999968	32.9024576213118\\
4.89599999999968	32.895843907902\\
4.89799999999968	32.8892315972031\\
4.89999999999968	32.8826206887539\\
4.90199999999968	32.8760111820944\\
4.90399999999968	32.8694030767645\\
4.90599999999968	32.8627963723036\\
4.90799999999968	32.8561910682537\\
4.90999999999968	32.8495871641556\\
4.91199999999968	32.8429846595511\\
4.91399999999968	32.8363835539822\\
4.91599999999968	32.8297838469916\\
4.91799999999968	32.8231855381217\\
4.91999999999968	32.8165886269168\\
4.92199999999968	32.8099931129201\\
4.92399999999968	32.8033989956764\\
4.92599999999968	32.7968062747296\\
4.92799999999968	32.7902149496257\\
4.92999999999968	32.7836250199095\\
4.93199999999968	32.7770364851277\\
4.93399999999968	32.7704493448262\\
4.93599999999968	32.763863598552\\
4.93799999999968	32.7572792458524\\
4.93999999999968	32.750696286275\\
4.94199999999968	32.7441147193678\\
4.94399999999968	32.7375345446798\\
4.94599999999968	32.7309557617597\\
4.94799999999968	32.724378370157\\
4.94999999999968	32.7178023694209\\
4.95199999999968	32.7112277591028\\
4.95399999999968	32.7046545387525\\
4.95599999999968	32.698082707921\\
4.95799999999968	32.6915122661608\\
4.95999999999968	32.6849432130229\\
4.96199999999968	32.67837554806\\
4.96399999999968	32.671809270825\\
4.96599999999968	32.6652443808709\\
4.96799999999967	32.6586808777511\\
4.96999999999967	32.6521187610205\\
4.97199999999967	32.645558030233\\
4.97399999999967	32.6389986849438\\
4.97599999999967	32.6324407247077\\
4.97799999999967	32.6258841490809\\
4.97999999999967	32.6193289576192\\
4.98199999999967	32.6127751498796\\
4.98399999999967	32.606222725419\\
4.98599999999967	32.5996716837945\\
4.98799999999967	32.5931220245642\\
4.98999999999967	32.5865737472862\\
4.99199999999967	32.5800268515196\\
4.99399999999967	32.5734813368232\\
4.99599999999967	32.5669372027564\\
4.99799999999967	32.5603944488794\\
4.99999999999967	32.5538530747522\\
5.00199999999967	32.5473130799357\\
5.00399999999967	32.5407744639913\\
5.00599999999967	32.5342372264802\\
5.00799999999967	32.5277013669649\\
5.00999999999967	32.5211668850072\\
5.01199999999967	32.5146337801703\\
5.01399999999967	32.5081020520175\\
5.01599999999967	32.5015717001124\\
5.01799999999967	32.4950427240191\\
5.01999999999967	32.4885151233017\\
5.02199999999967	32.4819888975257\\
5.02399999999967	32.4754640462559\\
5.02599999999967	32.4689405690585\\
5.02799999999967	32.4624184654994\\
5.02999999999967	32.4558977351447\\
5.03199999999967	32.4493783775619\\
5.03399999999967	32.4428603923181\\
5.03599999999967	32.4363437789813\\
5.03799999999967	32.4298285371193\\
5.03999999999967	32.4233146663012\\
5.04199999999967	32.4168021660955\\
5.04399999999967	32.4102910360716\\
5.04599999999967	32.4037812757996\\
5.04799999999967	32.3972728848496\\
5.04999999999967	32.3907658627921\\
5.05199999999967	32.3842602091982\\
5.05399999999967	32.3777559236392\\
5.05599999999967	32.3712530056873\\
5.05799999999966	32.3647514549142\\
5.05999999999966	32.358251270893\\
5.06199999999966	32.3517524531965\\
5.06399999999966	32.345255001398\\
5.06599999999966	32.3387589150715\\
5.06799999999966	32.3322641937913\\
5.06999999999966	32.3257708371317\\
5.07199999999966	32.3192788446682\\
5.07399999999966	32.3127882159761\\
5.07599999999966	32.3062989506311\\
5.07799999999966	32.2998110482095\\
5.07999999999966	32.2933245082879\\
5.08199999999966	32.2868393304435\\
5.08399999999966	32.2803555142535\\
5.08599999999966	32.2738730592959\\
5.08799999999966	32.267391965149\\
5.08999999999966	32.260912231391\\
5.09199999999966	32.2544338576013\\
5.09399999999966	32.2479568433592\\
5.09599999999966	32.2414811882444\\
5.09799999999966	32.2350068918375\\
5.09999999999966	32.2285339537185\\
5.10199999999966	32.2220623734688\\
5.10399999999966	32.2155921506696\\
5.10599999999966	32.2091232849029\\
5.10799999999966	32.2026557757505\\
5.10999999999966	32.1961896227952\\
5.11199999999966	32.1897248256203\\
5.11399999999966	32.1832613838083\\
5.11599999999966	32.1767992969433\\
5.11799999999966	32.1703385646099\\
5.11999999999966	32.163879186392\\
5.12199999999966	32.1574211618744\\
5.12399999999966	32.1509644906428\\
5.12599999999966	32.144509172283\\
5.12799999999966	32.1380552063807\\
5.12999999999966	32.1316025925221\\
5.13199999999966	32.1251513302946\\
5.13399999999966	32.1187014192852\\
5.13599999999966	32.1122528590813\\
5.13799999999966	32.1058056492711\\
5.13999999999966	32.099359789443\\
5.14199999999966	32.0929152791855\\
5.14399999999966	32.086472118088\\
5.14599999999966	32.0800303057398\\
5.14799999999966	32.0735898417309\\
5.14999999999965	32.0671507256515\\
5.15199999999965	32.0607129570924\\
5.15399999999965	32.0542765356446\\
5.15599999999965	32.0478414608994\\
5.15799999999965	32.0414077324486\\
5.15999999999965	32.0349753498847\\
5.16199999999965	32.0285443127998\\
5.16399999999965	32.0221146207872\\
5.16599999999965	32.0156862734398\\
5.16799999999965	32.0092592703516\\
5.16999999999965	32.0028336111166\\
5.17199999999965	31.9964092953294\\
5.17399999999965	31.9899863225844\\
5.17599999999965	31.9835646924772\\
5.17799999999965	31.977144404603\\
5.17999999999965	31.9707254585582\\
5.18199999999965	31.9643078539387\\
5.18399999999965	31.9578915903413\\
5.18599999999965	31.9514766673629\\
5.18799999999965	31.9450630846014\\
5.18999999999965	31.938650841654\\
5.19199999999965	31.9322399381193\\
5.19399999999965	31.9258303735958\\
5.19599999999965	31.9194221476823\\
5.19799999999965	31.9130152599778\\
5.19999999999965	31.9066097100821\\
5.20199999999965	31.9002054975958\\
5.20399999999965	31.8938026221185\\
5.20599999999965	31.8874010832512\\
5.20799999999965	31.8810008805952\\
5.20999999999965	31.8746020137517\\
5.21199999999965	31.8682044823228\\
5.21399999999965	31.8618082859107\\
5.21599999999965	31.8554134241176\\
5.21799999999965	31.8490198965466\\
5.21999999999965	31.8426277028014\\
5.22199999999965	31.8362368424851\\
5.22399999999965	31.8298473152022\\
5.22599999999965	31.8234591205567\\
5.22799999999965	31.8170722581536\\
5.22999999999965	31.8106867275979\\
5.23199999999965	31.804302528495\\
5.23399999999965	31.7979196604506\\
5.23599999999965	31.7915381230715\\
5.23799999999965	31.7851579159637\\
5.23999999999964	31.7787790387341\\
5.24199999999964	31.7724014909903\\
5.24399999999964	31.7660252723395\\
5.24599999999964	31.75965038239\\
5.24799999999964	31.7532768207499\\
5.24999999999964	31.7469045870281\\
5.25199999999964	31.7405336808335\\
5.25399999999964	31.7341641017755\\
5.25599999999964	31.7277958494641\\
5.25799999999964	31.7214289235089\\
5.25999999999964	31.7150633235206\\
5.26199999999964	31.7086990491102\\
5.26399999999964	31.7023360998887\\
5.26599999999964	31.6959744754674\\
5.26799999999964	31.6896141754584\\
5.26999999999964	31.6832551994739\\
5.27199999999964	31.6768975471265\\
5.27399999999964	31.6705412180288\\
5.27599999999964	31.6641862117944\\
5.27799999999964	31.6578325280367\\
5.27999999999964	31.65148016637\\
5.28199999999964	31.6451291264081\\
5.28399999999964	31.6387794077657\\
5.28599999999964	31.6324310100582\\
5.28799999999964	31.6260839329003\\
5.28999999999964	31.6197381759085\\
5.29199999999964	31.6133937386981\\
5.29399999999964	31.6070506208861\\
5.29599999999964	31.6007088220886\\
5.29799999999964	31.5943683419227\\
5.29999999999964	31.5880291800062\\
5.30199999999964	31.5816913359567\\
5.30399999999964	31.5753548093922\\
5.30599999999964	31.569019599931\\
5.30799999999964	31.562685707192\\
5.30999999999964	31.5563531307943\\
5.31199999999964	31.5500218703574\\
5.31399999999964	31.5436919255011\\
5.31599999999964	31.537363295845\\
5.31799999999964	31.5310359810102\\
5.31999999999964	31.5247099806172\\
5.32199999999964	31.5183852942871\\
5.32399999999964	31.5120619216415\\
5.32599999999964	31.5057398623021\\
5.32799999999964	31.4994191158908\\
5.32999999999964	31.4930996820306\\
5.33199999999963	31.4867815603437\\
5.33399999999963	31.4804647504535\\
5.33599999999963	31.4741492519838\\
5.33799999999963	31.4678350645576\\
5.33999999999963	31.4615221877995\\
5.34199999999963	31.4552106213343\\
5.34399999999963	31.4489003647861\\
5.34599999999963	31.4425914177805\\
5.34799999999963	31.4362837799427\\
5.34999999999963	31.4299774508984\\
5.35199999999963	31.4236724302739\\
5.35399999999963	31.4173687176955\\
5.35599999999963	31.4110663127903\\
5.35799999999963	31.4047652151848\\
5.35999999999963	31.3984654245068\\
5.36199999999963	31.392166940384\\
5.36399999999963	31.3858697624443\\
5.36599999999963	31.3795738903162\\
5.36799999999963	31.3732793236282\\
5.36999999999963	31.3669860620098\\
5.37199999999963	31.3606941050901\\
5.37399999999963	31.3544034524985\\
5.37599999999963	31.3481141038651\\
5.37799999999963	31.3418260588207\\
5.37999999999963	31.3355393169954\\
5.38199999999963	31.3292538780203\\
5.38399999999963	31.3229697415268\\
5.38599999999963	31.3166869071464\\
5.38799999999963	31.3104053745111\\
5.38999999999963	31.3041251432529\\
5.39199999999963	31.2978462130048\\
5.39399999999963	31.2915685833993\\
5.39599999999963	31.2852922540696\\
5.39799999999963	31.2790172246494\\
5.39999999999963	31.2727434947723\\
5.40199999999963	31.2664710640727\\
5.40399999999963	31.2601999321848\\
5.40599999999963	31.2539300987433\\
5.40799999999963	31.2476615633834\\
5.40999999999963	31.2413943257407\\
5.41199999999963	31.2351283854508\\
5.41399999999963	31.2288637421496\\
5.41599999999963	31.2226003954734\\
5.41799999999963	31.2163383450589\\
5.41999999999963	31.210077590543\\
5.42199999999962	31.2038181315628\\
5.42399999999962	31.1975599677564\\
5.42599999999962	31.1913030987611\\
5.42799999999962	31.1850475242157\\
5.42999999999962	31.1787932437578\\
5.43199999999962	31.1725402570269\\
5.43399999999962	31.1662885636618\\
5.43599999999962	31.1600381633023\\
5.43799999999962	31.1537890555876\\
5.43999999999962	31.147541240158\\
5.44199999999962	31.1412947166536\\
5.44399999999962	31.1350494847154\\
5.44599999999962	31.1288055439843\\
5.44799999999962	31.1225628941011\\
5.44999999999962	31.1163215347077\\
5.45199999999962	31.110081465446\\
5.45399999999962	31.1038426859577\\
5.45599999999962	31.0976051958859\\
5.45799999999962	31.0913689948727\\
5.45999999999962	31.0851340825616\\
5.46199999999962	31.0789004585957\\
5.46399999999962	31.0726681226188\\
5.46599999999962	31.0664370742748\\
5.46799999999962	31.0602073132078\\
5.46999999999962	31.0539788390624\\
5.47199999999962	31.0477516514835\\
5.47399999999962	31.0415257501163\\
5.47599999999962	31.035301134606\\
5.47799999999962	31.0290778045987\\
5.47999999999962	31.02285575974\\
5.48199999999962	31.0166349996765\\
5.48399999999962	31.0104155240546\\
5.48599999999962	31.0041973325215\\
5.48799999999962	30.9979804247239\\
5.48999999999962	30.9917648003099\\
5.49199999999962	30.9855504589267\\
5.49399999999962	30.9793374002225\\
5.49599999999962	30.9731256238463\\
5.49799999999962	30.9669151294457\\
5.49999999999962	30.9607059166704\\
5.50199999999962	30.9544979851695\\
5.50399999999962	30.9482913345924\\
5.50599999999962	30.9420859645889\\
5.50799999999962	30.9358818748092\\
5.50999999999962	30.9296790649035\\
5.51199999999961	30.9234775345226\\
5.51399999999961	30.9172772833174\\
5.51599999999961	30.9110783109393\\
5.51799999999961	30.9048806170395\\
5.51999999999961	30.8986842012701\\
5.52199999999961	30.8924890632832\\
5.52399999999961	30.8862952027311\\
5.52599999999961	30.8801026192666\\
5.52799999999961	30.8739113125422\\
5.52999999999961	30.8677212822116\\
5.53199999999961	30.8615325279281\\
5.53399999999961	30.8553450493457\\
5.53599999999961	30.8491588461181\\
5.53799999999961	30.8429739178999\\
5.53999999999961	30.8367902643458\\
5.54199999999961	30.8306078851108\\
5.54399999999961	30.8244267798497\\
5.54599999999961	30.8182469482185\\
5.54799999999961	30.8120683898722\\
5.54999999999961	30.8058911044675\\
5.55199999999961	30.7997150916606\\
5.55399999999961	30.7935403511079\\
5.55599999999961	30.787366882466\\
5.55799999999961	30.7811946853928\\
5.55999999999961	30.7750237595449\\
5.56199999999961	30.7688541045807\\
5.56399999999961	30.7626857201577\\
5.56599999999961	30.7565186059341\\
5.56799999999961	30.7503527615688\\
5.56999999999961	30.7441881867201\\
5.57199999999961	30.7380248810476\\
5.57399999999961	30.7318628442099\\
5.57599999999961	30.7257020758669\\
5.57799999999961	30.7195425756789\\
5.57999999999961	30.7133843433057\\
5.58199999999961	30.7072273784075\\
5.58399999999961	30.7010716806453\\
5.58599999999961	30.6949172496799\\
5.58799999999961	30.6887640851724\\
5.58999999999961	30.6826121867849\\
5.59199999999961	30.6764615541781\\
5.59399999999961	30.6703121870147\\
5.59599999999961	30.664164084957\\
5.59799999999961	30.6580172476672\\
5.59999999999961	30.6518716748084\\
5.60199999999961	30.6457273660436\\
5.6039999999996	30.6395843210362\\
5.6059999999996	30.6334425394496\\
5.6079999999996	30.627302020948\\
5.6099999999996	30.6211627651951\\
5.6119999999996	30.6150247718555\\
5.6139999999996	30.6088880405942\\
5.6159999999996	30.6027525710757\\
5.6179999999996	30.5966183629653\\
5.6199999999996	30.5904854159285\\
5.6219999999996	30.5843537296307\\
5.6239999999996	30.5782233037385\\
5.6259999999996	30.5720941379175\\
5.6279999999996	30.5659662318343\\
5.6299999999996	30.5598395851561\\
5.6319999999996	30.5537141975493\\
5.6339999999996	30.5475900686815\\
5.6359999999996	30.5414671982202\\
5.6379999999996	30.5353455858331\\
5.6399999999996	30.5292252311882\\
5.6419999999996	30.5231061339536\\
5.6439999999996	30.5169882937982\\
5.6459999999996	30.5108717103908\\
5.6479999999996	30.5047563834002\\
5.6499999999996	30.4986423124957\\
5.6519999999996	30.4925294973473\\
5.6539999999996	30.486417937624\\
5.6559999999996	30.4803076329967\\
5.6579999999996	30.4741985831352\\
5.6599999999996	30.4680907877101\\
5.6619999999996	30.4619842463924\\
5.6639999999996	30.455878958853\\
5.6659999999996	30.4497749247636\\
5.6679999999996	30.4436721437952\\
5.6699999999996	30.4375706156199\\
5.6719999999996	30.4314703399101\\
5.6739999999996	30.4253713163375\\
5.6759999999996	30.4192735445751\\
5.6779999999996	30.4131770242955\\
5.6799999999996	30.4070817551717\\
5.6819999999996	30.4009877368772\\
5.6839999999996	30.3948949690856\\
5.6859999999996	30.3888034514706\\
5.6879999999996	30.3827131837058\\
5.6899999999996	30.3766241654664\\
5.6919999999996	30.3705363964262\\
5.69399999999959	30.3644498762603\\
5.69599999999959	30.3583646046436\\
5.69799999999959	30.3522805812515\\
5.69999999999959	30.3461978057593\\
5.70199999999959	30.3401162778429\\
5.70399999999959	30.3340359971783\\
5.70599999999959	30.3279569634415\\
5.70799999999959	30.3218791763095\\
5.70999999999959	30.3158026354586\\
5.71199999999959	30.3097273405657\\
5.71399999999959	30.3036532913083\\
5.71599999999959	30.2975804873636\\
5.71799999999959	30.2915089284094\\
5.71999999999959	30.2854386141234\\
5.72199999999959	30.2793695441841\\
5.72399999999959	30.2733017182695\\
5.72599999999959	30.2672351360586\\
5.72799999999959	30.26116979723\\
5.72999999999959	30.2551057014629\\
5.73199999999959	30.2490428484367\\
5.73399999999959	30.2429812378308\\
5.73599999999959	30.2369208693253\\
5.73799999999959	30.2308617425997\\
5.73999999999959	30.2248038573349\\
5.74199999999959	30.2187472132112\\
5.74399999999959	30.2126918099091\\
5.74599999999959	30.2066376471096\\
5.74799999999959	30.2005847244944\\
5.74999999999959	30.1945330417444\\
5.75199999999959	30.1884825985415\\
5.75399999999959	30.1824333945676\\
5.75599999999959	30.1763854295047\\
5.75799999999959	30.1703387030354\\
5.75999999999959	30.164293214842\\
5.76199999999959	30.1582489646077\\
5.76399999999959	30.1522059520151\\
5.76599999999959	30.1461641767479\\
5.76799999999959	30.1401236384894\\
5.76999999999959	30.1340843369236\\
5.77199999999959	30.128046271734\\
5.77399999999959	30.1220094426049\\
5.77599999999959	30.1159738492211\\
5.77799999999959	30.1099394912668\\
5.77999999999959	30.1039063684272\\
5.78199999999959	30.0978744803873\\
5.78399999999959	30.0918438268323\\
5.78599999999958	30.0858144074481\\
5.78799999999958	30.0797862219197\\
5.78999999999958	30.0737592699339\\
5.79199999999958	30.0677335511768\\
5.79399999999958	30.0617090653344\\
5.79599999999958	30.0556858120937\\
5.79799999999958	30.0496637911416\\
5.79999999999958	30.0436430021652\\
5.80199999999958	30.0376234448517\\
5.80399999999958	30.0316051188889\\
5.80599999999958	30.0255880239641\\
5.80799999999958	30.0195721597657\\
5.80999999999958	30.0135575259817\\
5.81199999999958	30.0075441223008\\
5.81399999999958	30.0015319484113\\
5.81599999999958	29.9955210040024\\
5.81799999999958	29.9895112887629\\
5.81999999999958	29.9835028023825\\
5.82199999999958	29.9774955445502\\
5.82399999999958	29.9714895149561\\
5.82599999999958	29.9654847132901\\
5.82799999999958	29.9594811392425\\
5.82999999999958	29.9534787925032\\
5.83199999999958	29.9474776727633\\
5.83399999999958	29.9414777797134\\
5.83599999999958	29.9354791130448\\
5.83799999999958	29.9294816724482\\
5.83999999999958	29.9234854576157\\
5.84199999999958	29.9174904682388\\
5.84399999999958	29.9114967040088\\
5.84599999999958	29.9055041646186\\
5.84799999999958	29.8995128497602\\
5.84999999999958	29.893522759126\\
5.85199999999958	29.8875338924088\\
5.85399999999958	29.8815462493018\\
5.85599999999958	29.8755598294977\\
5.85799999999958	29.8695746326904\\
5.85999999999958	29.8635906585733\\
5.86199999999958	29.8576079068397\\
5.86399999999958	29.8516263771843\\
5.86599999999958	29.845646069301\\
5.86799999999958	29.8396669828843\\
5.86999999999958	29.8336891176287\\
5.87199999999958	29.827712473229\\
5.87399999999958	29.8217370493806\\
5.87599999999957	29.8157628457783\\
5.87799999999957	29.8097898621178\\
5.87999999999957	29.8038180980945\\
5.88199999999957	29.7978475534048\\
5.88399999999957	29.7918782277442\\
5.88599999999957	29.7859101208094\\
5.88799999999957	29.7799432322966\\
5.88999999999957	29.7739775619027\\
5.89199999999957	29.7680131093245\\
5.89399999999957	29.762049874259\\
5.89599999999957	29.7560878564036\\
5.89799999999957	29.7501270554559\\
5.89999999999957	29.7441674711134\\
5.90199999999957	29.7382091030741\\
5.90399999999957	29.7322519510362\\
5.90599999999957	29.7262960146978\\
5.90799999999957	29.7203412937576\\
5.90999999999957	29.7143877879141\\
5.91199999999957	29.7084354968666\\
5.91399999999957	29.7024844203138\\
5.91599999999957	29.6965345579552\\
5.91799999999957	29.6905859094904\\
5.91999999999957	29.6846384746187\\
5.92199999999957	29.6786922530407\\
5.92399999999957	29.6727472444558\\
5.92599999999957	29.6668034485648\\
5.92799999999957	29.660860865068\\
5.92999999999957	29.6549194936662\\
5.93199999999957	29.6489793340601\\
5.93399999999957	29.6430403859509\\
5.93599999999957	29.6371026490402\\
5.93799999999957	29.6311661230289\\
5.93999999999957	29.6252308076191\\
5.94199999999957	29.6192967025124\\
5.94399999999957	29.613363807411\\
5.94599999999957	29.6074321220172\\
5.94799999999957	29.6015016460335\\
5.94999999999957	29.5955723791626\\
5.95199999999957	29.5896443211071\\
5.95399999999957	29.5837174715701\\
5.95599999999957	29.5777918302552\\
5.95799999999957	29.5718673968654\\
5.95999999999957	29.5659441711043\\
5.96199999999957	29.560022152676\\
5.96399999999957	29.5541013412842\\
5.96599999999956	29.5481817366334\\
5.96799999999956	29.5422633384276\\
5.96999999999956	29.5363461463719\\
5.97199999999956	29.5304301601705\\
5.97399999999956	29.5245153795286\\
5.97599999999956	29.5186018041515\\
5.97799999999956	29.5126894337442\\
5.97999999999956	29.5067782680124\\
5.98199999999956	29.5008683066619\\
5.98399999999956	29.4949595493984\\
5.98599999999956	29.4890519959279\\
5.98799999999956	29.4831456459571\\
5.98999999999956	29.4772404991919\\
5.99199999999956	29.4713365553393\\
5.99399999999956	29.465433814106\\
5.99599999999956	29.4595322751991\\
5.99799999999956	29.4536319383258\\
5.99999999999956	29.4477328031933\\
6.00199999999956	29.4418348695093\\
6.00399999999956	29.4359381369814\\
6.00599999999956	29.4300426053177\\
6.00799999999956	29.4241482742263\\
6.00999999999956	29.4182551434157\\
6.01199999999956	29.4123632125938\\
6.01399999999956	29.4064724814701\\
6.01599999999956	29.4005829497525\\
6.01799999999956	29.3946946171508\\
6.01999999999956	29.3888074833739\\
6.02199999999956	29.3829215481311\\
6.02399999999956	29.3770368111323\\
6.02599999999956	29.371153272087\\
6.02799999999956	29.3652709307051\\
6.02999999999956	29.359389786697\\
6.03199999999956	29.3535098397724\\
6.03399999999956	29.3476310896424\\
6.03599999999956	29.3417535360176\\
6.03799999999956	29.3358771786084\\
6.03999999999956	29.3300020171262\\
6.04199999999956	29.3241280512822\\
6.04399999999956	29.3182552807876\\
6.04599999999956	29.3123837053539\\
6.04799999999956	29.3065133246931\\
6.04999999999956	29.3006441385166\\
6.05199999999956	29.2947761465371\\
6.05399999999956	29.2889093484664\\
6.05599999999956	29.2830437440172\\
6.05799999999955	29.277179332902\\
6.05999999999955	29.2713161148334\\
6.06199999999955	29.2654540895247\\
6.06399999999955	29.2595932566889\\
6.06599999999955	29.2537336160391\\
6.06799999999955	29.247875167289\\
6.06999999999955	29.2420179101522\\
6.07199999999955	29.2361618443424\\
6.07399999999955	29.2303069695738\\
6.07599999999955	29.2244532855607\\
6.07799999999955	29.218600792017\\
6.07999999999955	29.2127494886577\\
6.08199999999955	29.2068993751971\\
6.08399999999955	29.2010504513502\\
6.08599999999955	29.1952027168323\\
6.08799999999955	29.189356171358\\
6.08999999999955	29.1835108146431\\
6.09199999999955	29.177666646403\\
6.09399999999955	29.1718236663538\\
6.09599999999955	29.1659818742107\\
6.09799999999955	29.1601412696903\\
6.09999999999955	29.1543018525086\\
6.10199999999955	29.1484636223821\\
6.10399999999955	29.1426265790273\\
6.10599999999955	29.136790722161\\
6.10799999999955	29.1309560514998\\
6.10999999999955	29.1251225667613\\
6.11199999999955	29.1192902676621\\
6.11399999999955	29.1134591539201\\
6.11599999999955	29.1076292252528\\
6.11799999999955	29.1018004813777\\
6.11999999999955	29.0959729220128\\
6.12199999999955	29.0901465468763\\
6.12399999999955	29.0843213556863\\
6.12599999999955	29.0784973481612\\
6.12799999999955	29.0726745240196\\
6.12999999999955	29.0668528829801\\
6.13199999999955	29.0610324247617\\
6.13399999999955	29.0552131490837\\
6.13599999999955	29.0493950556648\\
6.13799999999955	29.0435781442246\\
6.13999999999955	29.0377624144829\\
6.14199999999955	29.0319478661592\\
6.14399999999955	29.0261344989734\\
6.14599999999955	29.0203223126454\\
6.14799999999954	29.0145113068957\\
6.14999999999954	29.0087014814443\\
6.15199999999954	29.0028928360119\\
6.15399999999954	28.9970853703192\\
6.15599999999954	28.991279084087\\
6.15799999999954	28.9854739770364\\
6.15999999999954	28.9796700488885\\
6.16199999999954	28.9738672993645\\
6.16399999999954	28.9680657281859\\
6.16599999999954	28.9622653350746\\
6.16799999999954	28.9564661197522\\
6.16999999999954	28.9506680819407\\
6.17199999999954	28.9448712213622\\
6.17399999999954	28.939075537739\\
6.17599999999954	28.9332810307934\\
6.17799999999954	28.9274877002484\\
6.17999999999954	28.921695545826\\
6.18199999999954	28.9159045672501\\
6.18399999999954	28.910114764243\\
6.18599999999954	28.9043261365282\\
6.18799999999954	28.8985386838291\\
6.18999999999954	28.8927524058692\\
6.19199999999954	28.8869673023722\\
6.19399999999954	28.8811833730618\\
6.19599999999954	28.8754006176623\\
6.19799999999954	28.8696190358977\\
6.19999999999954	28.8638386274922\\
6.20199999999954	28.8580593921704\\
6.20399999999954	28.8522813296571\\
6.20599999999954	28.8465044396767\\
6.20799999999954	28.8407287219543\\
6.20999999999954	28.834954176215\\
6.21199999999954	28.829180802184\\
6.21399999999954	28.8234085995869\\
6.21599999999954	28.8176375681491\\
6.21799999999954	28.8118677075963\\
6.21999999999954	28.8060990176539\\
6.22199999999954	28.8003314980487\\
6.22399999999954	28.7945651485062\\
6.22599999999954	28.7887999687532\\
6.22799999999954	28.7830359585159\\
6.22999999999954	28.7772731175209\\
6.23199999999954	28.771511445495\\
6.23399999999954	28.7657509421651\\
6.23599999999954	28.7599916072583\\
6.23799999999954	28.7542334405017\\
6.23999999999953	28.7484764416226\\
6.24199999999953	28.7427206103489\\
6.24399999999953	28.736965946408\\
6.24599999999953	28.7312124495274\\
6.24799999999953	28.7254601194356\\
6.24999999999953	28.7197089558602\\
6.25199999999953	28.7139589585299\\
6.25399999999953	28.7082101271729\\
6.25599999999953	28.7024624615175\\
6.25799999999953	28.6967159612928\\
6.25999999999953	28.6909706262275\\
6.26199999999953	28.6852264560504\\
6.26399999999953	28.6794834504909\\
6.26599999999953	28.6737416092781\\
6.26799999999953	28.6680009321415\\
6.26999999999953	28.6622614188106\\
6.27199999999953	28.6565230690152\\
6.27399999999953	28.6507858824852\\
6.27599999999953	28.6450498589506\\
6.27799999999953	28.6393149981413\\
6.27999999999953	28.6335812997879\\
6.28199999999953	28.6278487636207\\
6.28399999999953	28.6221173893702\\
6.28599999999953	28.6163871767676\\
6.28799999999953	28.6106581255429\\
6.28999999999953	28.6049302354279\\
6.29199999999953	28.5992035061535\\
6.29399999999953	28.593477937451\\
6.29599999999953	28.5877535290518\\
6.29799999999953	28.5820302806875\\
6.29999999999953	28.5763081920898\\
6.30199999999953	28.5705872629907\\
6.30399999999953	28.564867493122\\
6.30599999999953	28.5591488822159\\
6.30799999999953	28.5534314300051\\
6.30999999999953	28.5477151362214\\
6.31199999999953	28.5420000005978\\
6.31399999999953	28.536286022867\\
6.31599999999953	28.5305732027616\\
6.31799999999953	28.5248615400148\\
6.31999999999953	28.5191510343596\\
6.32199999999953	28.5134416855295\\
6.32399999999953	28.5077334932574\\
6.32599999999953	28.5020264572778\\
6.32799999999953	28.4963205773233\\
6.32999999999952	28.4906158531285\\
6.33199999999952	28.484912284427\\
6.33399999999952	28.4792098709532\\
6.33599999999952	28.4735086124412\\
6.33799999999952	28.4678085086251\\
6.33999999999952	28.4621095592397\\
6.34199999999952	28.4564117640198\\
6.34399999999952	28.4507151226999\\
6.34599999999952	28.4450196350151\\
6.34799999999952	28.4393253007003\\
6.34999999999952	28.4336321194909\\
6.35199999999952	28.4279400911223\\
6.35399999999952	28.4222492153297\\
6.35599999999952	28.4165594918488\\
6.35799999999952	28.4108709204154\\
6.35999999999952	28.4051835007654\\
6.36199999999952	28.3994972326348\\
6.36399999999952	28.3938121157597\\
6.36599999999952	28.3881281498764\\
6.36799999999952	28.3824453347216\\
6.36999999999952	28.3767636700312\\
6.37199999999952	28.3710831555424\\
6.37399999999952	28.3654037909919\\
6.37599999999952	28.3597255761166\\
6.37799999999952	28.3540485106537\\
6.37999999999952	28.3483725943407\\
6.38199999999952	28.3426978269139\\
6.38399999999952	28.3370242081119\\
6.38599999999952	28.3313517376718\\
6.38799999999952	28.3256804153315\\
6.38999999999952	28.320010240829\\
6.39199999999952	28.3143412139018\\
6.39399999999952	28.3086733342887\\
6.39599999999952	28.3030066017273\\
6.39799999999952	28.2973410159567\\
6.39999999999952	28.2916765767148\\
6.40199999999952	28.2860132837408\\
6.40399999999952	28.2803511367731\\
6.40599999999952	28.2746901355508\\
6.40799999999952	28.269030279813\\
6.40999999999952	28.2633715692986\\
6.41199999999952	28.2577140037473\\
6.41399999999952	28.2520575828985\\
6.41599999999952	28.2464023064916\\
6.41799999999952	28.2407481742661\\
6.41999999999951	28.2350951859624\\
6.42199999999951	28.22944334132\\
6.42399999999951	28.223792640079\\
6.42599999999951	28.2181430819798\\
6.42799999999951	28.2124946667627\\
6.42999999999951	28.2068473941682\\
6.43199999999951	28.2012012639366\\
6.43399999999951	28.1955562758091\\
6.43599999999951	28.1899124295261\\
6.43799999999951	28.1842697248291\\
6.43999999999951	28.1786281614584\\
6.44199999999951	28.172987739156\\
6.44399999999951	28.1673484576628\\
6.44599999999951	28.1617103167207\\
6.44799999999951	28.1560733160711\\
6.44999999999951	28.1504374554554\\
6.45199999999951	28.1448027346157\\
6.45399999999951	28.1391691532942\\
6.45599999999951	28.1335367112327\\
6.45799999999951	28.1279054081737\\
6.45999999999951	28.1222752438592\\
6.46199999999951	28.1166462180319\\
6.46399999999951	28.1110183304345\\
6.46599999999951	28.1053915808096\\
6.46799999999951	28.0997659689002\\
6.46999999999951	28.0941414944487\\
6.47199999999951	28.088518157199\\
6.47399999999951	28.0828959568937\\
6.47599999999951	28.0772748932766\\
6.47799999999951	28.0716549660907\\
6.47999999999951	28.0660361750801\\
6.48199999999951	28.0604185199882\\
6.48399999999951	28.0548020005588\\
6.48599999999951	28.0491866165359\\
6.48799999999951	28.0435723676636\\
6.48999999999951	28.0379592536859\\
6.49199999999951	28.0323472743477\\
6.49399999999951	28.0267364293929\\
6.49599999999951	28.021126718566\\
6.49799999999951	28.0155181416122\\
6.49999999999951	28.0099106982756\\
6.50199999999951	28.0043043883018\\
6.50399999999951	27.9986992114355\\
6.50599999999951	27.9930951674217\\
6.50799999999951	27.9874922560061\\
6.50999999999951	27.9818904769338\\
6.5119999999995	27.9762898299503\\
6.5139999999995	27.9706903148011\\
6.5159999999995	27.9650919312326\\
6.5179999999995	27.95949467899\\
6.5199999999995	27.9538985578196\\
6.5219999999995	27.9483035674677\\
6.5239999999995	27.9427097076799\\
6.5259999999995	27.937116978203\\
6.5279999999995	27.9315253787833\\
6.5299999999995	27.9259349091676\\
6.5319999999995	27.9203455691025\\
6.5339999999995	27.9147573583346\\
6.5359999999995	27.9091702766111\\
6.5379999999995	27.9035843236789\\
6.5399999999995	27.8979994992853\\
6.5419999999995	27.8924158031774\\
6.5439999999995	27.8868332351027\\
6.5459999999995	27.8812517948087\\
6.5479999999995	27.8756714820431\\
6.5499999999995	27.8700922965536\\
6.5519999999995	27.864514238088\\
6.5539999999995	27.8589373063943\\
6.5559999999995	27.8533615012208\\
6.5579999999995	27.8477868223152\\
6.5599999999995	27.8422132694262\\
6.5619999999995	27.8366408423022\\
6.5639999999995	27.8310695406917\\
6.5659999999995	27.8254993643436\\
6.5679999999995	27.8199303130061\\
6.5699999999995	27.8143623864287\\
6.5719999999995	27.80879558436\\
6.5739999999995	27.8032299065493\\
6.5759999999995	27.7976653527458\\
6.5779999999995	27.7921019226989\\
6.5799999999995	27.7865396161579\\
6.5819999999995	27.7809784328726\\
6.5839999999995	27.7754183725924\\
6.5859999999995	27.7698594350673\\
6.5879999999995	27.7643016200473\\
6.5899999999995	27.758744927282\\
6.5919999999995	27.7531893565221\\
6.5939999999995	27.7476349075173\\
6.5959999999995	27.7420815800183\\
6.5979999999995	27.7365293737753\\
6.5999999999995	27.7309782885396\\
6.60199999999949	27.7254283240607\\
6.60399999999949	27.7198794800903\\
6.60599999999949	27.7143317563789\\
6.60799999999949	27.7087851526779\\
6.60999999999949	27.703239668738\\
6.61199999999949	27.6976953043107\\
6.61399999999949	27.6921520591473\\
6.61599999999949	27.6866099329991\\
6.61799999999949	27.681068925618\\
6.61999999999949	27.6755290367553\\
6.62199999999949	27.669990266163\\
6.62399999999949	27.6644526135932\\
6.62599999999949	27.6589160787974\\
6.62799999999949	27.6533806615281\\
6.62999999999949	27.6478463615374\\
6.63199999999949	27.6423131785776\\
6.63399999999949	27.6367811124013\\
6.63599999999949	27.631250162761\\
6.63799999999949	27.625720329409\\
6.63999999999949	27.6201916120985\\
6.64199999999949	27.6146640105824\\
6.64399999999949	27.6091375246134\\
6.64599999999949	27.6036121539445\\
6.64799999999949	27.5980878983293\\
6.64999999999949	27.5925647575209\\
6.65199999999949	27.5870427312726\\
6.65399999999949	27.5815218193381\\
6.65599999999949	27.576002021471\\
6.65799999999949	27.5704833374247\\
6.65999999999949	27.5649657669535\\
6.66199999999949	27.559449309811\\
6.66399999999949	27.5539339657514\\
6.66599999999949	27.5484197345289\\
6.66799999999949	27.5429066158976\\
6.66999999999949	27.5373946096121\\
6.67199999999949	27.5318837154265\\
6.67399999999949	27.5263739330957\\
6.67599999999949	27.5208652623743\\
6.67799999999949	27.5153577030169\\
6.67999999999949	27.5098512547784\\
6.68199999999949	27.5043459174143\\
6.68399999999949	27.498841690679\\
6.68599999999949	27.4933385743282\\
6.68799999999949	27.487836568117\\
6.68999999999949	27.4823356718007\\
6.69199999999949	27.4768358851349\\
6.69399999999948	27.4713372078753\\
6.69599999999948	27.4658396397776\\
6.69799999999948	27.4603431805976\\
6.69999999999948	27.4548478300911\\
6.70199999999948	27.4493535880141\\
6.70399999999948	27.4438604541231\\
6.70599999999948	27.438368428174\\
6.70799999999948	27.4328775099232\\
6.70999999999948	27.4273876991272\\
6.71199999999948	27.4218989955421\\
6.71399999999948	27.4164113989254\\
6.71599999999948	27.4109249090332\\
6.71799999999948	27.4054395256226\\
6.71999999999948	27.3999552484503\\
6.72199999999948	27.3944720772735\\
6.72399999999948	27.3889900118494\\
6.72599999999948	27.3835090519352\\
6.72799999999948	27.3780291972883\\
6.72999999999948	27.3725504476661\\
6.73199999999948	27.367072802826\\
6.73399999999948	27.3615962625259\\
6.73599999999948	27.3561208265232\\
6.73799999999948	27.3506464945762\\
6.73999999999948	27.3451732664425\\
6.74199999999948	27.3397011418805\\
6.74399999999948	27.334230120648\\
6.74599999999948	27.3287602025035\\
6.74799999999948	27.3232913872052\\
6.74999999999948	27.3178236745115\\
6.75199999999948	27.3123570641811\\
6.75399999999948	27.3068915559725\\
6.75599999999948	27.3014271496446\\
6.75799999999948	27.2959638449561\\
6.75999999999948	27.2905016416659\\
6.76199999999948	27.2850405395335\\
6.76399999999948	27.2795805383174\\
6.76599999999948	27.2741216377772\\
6.76799999999948	27.2686638376722\\
6.76999999999948	27.2632071377617\\
6.77199999999948	27.2577515378055\\
6.77399999999948	27.2522970375629\\
6.77599999999948	27.2468436367937\\
6.77799999999948	27.2413913352578\\
6.77999999999948	27.2359401327153\\
6.78199999999948	27.2304900289258\\
6.78399999999947	27.2250410236499\\
6.78599999999947	27.2195931166475\\
6.78799999999947	27.214146307679\\
6.78999999999947	27.2087005965045\\
6.79199999999947	27.2032559828849\\
6.79399999999947	27.1978124665806\\
6.79599999999947	27.1923700473525\\
6.79799999999947	27.1869287249611\\
6.79999999999947	27.1814884991676\\
6.80199999999947	27.1760493697328\\
6.80399999999947	27.1706113364175\\
6.80599999999947	27.1651743989831\\
6.80799999999947	27.1597385571913\\
6.80999999999947	27.1543038108029\\
6.81199999999947	27.1488701595793\\
6.81399999999947	27.1434376032826\\
6.81599999999947	27.1380061416737\\
6.81799999999947	27.132575774515\\
6.81999999999947	27.1271465015682\\
6.82199999999947	27.1217183225949\\
6.82399999999947	27.1162912373574\\
6.82599999999947	27.1108652456177\\
6.82799999999947	27.1054403471381\\
6.82999999999947	27.1000165416808\\
6.83199999999947	27.0945938290081\\
6.83399999999947	27.0891722088829\\
6.83599999999947	27.0837516810674\\
6.83799999999947	27.0783322453245\\
6.83999999999947	27.0729139014168\\
6.84199999999947	27.067496649107\\
6.84399999999947	27.0620804881584\\
6.84599999999947	27.0566654183341\\
6.84799999999947	27.0512514393969\\
6.84999999999947	27.0458385511102\\
6.85199999999947	27.0404267532374\\
6.85399999999947	27.0350160455417\\
6.85599999999947	27.0296064277868\\
6.85799999999947	27.0241978997363\\
6.85999999999947	27.0187904611537\\
6.86199999999947	27.0133841118029\\
6.86399999999947	27.0079788514479\\
6.86599999999947	27.0025746798525\\
6.86799999999947	26.9971715967809\\
6.86999999999947	26.991769601997\\
6.87199999999947	26.9863686952652\\
6.87399999999946	26.9809688763499\\
6.87599999999946	26.9755701450155\\
6.87799999999946	26.9701725010262\\
6.87999999999946	26.9647759441471\\
6.88199999999946	26.9593804741427\\
6.88399999999946	26.9539860907774\\
6.88599999999946	26.9485927938168\\
6.88799999999946	26.9432005830251\\
6.88999999999946	26.9378094581679\\
6.89199999999946	26.9324194190102\\
6.89399999999946	26.927030465317\\
6.89599999999946	26.9216425968541\\
6.89799999999946	26.9162558133863\\
6.89999999999946	26.9108701146797\\
6.90199999999946	26.9054855004994\\
6.90399999999946	26.9001019706115\\
6.90599999999946	26.8947195247812\\
6.90799999999946	26.889338162775\\
6.90999999999946	26.8839578843585\\
6.91199999999946	26.8785786892977\\
6.91399999999946	26.8732005773587\\
6.91599999999946	26.867823548308\\
6.91799999999946	26.8624476019115\\
6.91999999999946	26.8570727379361\\
6.92199999999946	26.8516989561476\\
6.92399999999946	26.8463262563132\\
6.92599999999946	26.8409546381991\\
6.92799999999946	26.8355841015724\\
6.92999999999946	26.8302146461995\\
6.93199999999946	26.8248462718475\\
6.93399999999946	26.8194789782837\\
6.93599999999946	26.8141127652746\\
6.93799999999946	26.8087476325878\\
6.93999999999946	26.8033835799905\\
6.94199999999946	26.79802060725\\
6.94399999999946	26.7926587141336\\
6.94599999999946	26.7872979004092\\
6.94799999999946	26.781938165844\\
6.94999999999946	26.7765795102057\\
6.95199999999946	26.7712219332624\\
6.95399999999946	26.7658654347817\\
6.95599999999946	26.7605100145315\\
6.95799999999946	26.7551556722801\\
6.95999999999946	26.7498024077955\\
6.96199999999946	26.7444502208458\\
6.96399999999946	26.7390991111993\\
6.96599999999945	26.7337490786246\\
6.96799999999945	26.7284001228898\\
6.96999999999945	26.7230522437638\\
6.97199999999945	26.7177054410151\\
6.97399999999945	26.7123597144124\\
6.97599999999945	26.7070150637243\\
6.97799999999945	26.7016714887201\\
6.97999999999945	26.6963289891686\\
6.98199999999945	26.6909875648386\\
6.98399999999945	26.6856472154996\\
6.98599999999945	26.6803079409207\\
6.98799999999945	26.6749697408708\\
6.98999999999945	26.6696326151202\\
6.99199999999945	26.6642965634377\\
6.99399999999945	26.6589615855928\\
6.99599999999945	26.6536276813556\\
6.99799999999945	26.6482948504952\\
6.99999999999945	26.6429630927821\\
7.00199999999945	26.6376324079859\\
7.00399999999945	26.6323027958764\\
7.00599999999945	26.6269742562239\\
7.00799999999945	26.6216467887984\\
7.00999999999945	26.6163203933701\\
7.01199999999945	26.6109950697094\\
7.01399999999945	26.6056708175869\\
7.01599999999945	26.6003476367726\\
7.01799999999945	26.5950255270375\\
7.01999999999945	26.5897044881518\\
7.02199999999945	26.5843845198867\\
7.02399999999945	26.5790656220125\\
7.02599999999945	26.5737477943004\\
7.02799999999945	26.5684310365215\\
7.02999999999945	26.5631153484465\\
7.03199999999945	26.5578007298467\\
7.03399999999945	26.5524871804933\\
7.03599999999945	26.5471747001576\\
7.03799999999945	26.541863288611\\
7.03999999999945	26.5365529456248\\
7.04199999999945	26.5312436709709\\
7.04399999999945	26.5259354644204\\
7.04599999999945	26.5206283257455\\
7.04799999999945	26.5153222547177\\
7.04999999999945	26.5100172511089\\
7.05199999999945	26.5047133146909\\
7.05399999999945	26.4994104452361\\
7.05599999999944	26.4941086425161\\
7.05799999999944	26.4888079063036\\
7.05999999999944	26.4835082363706\\
7.06199999999944	26.4782096324893\\
7.06399999999944	26.4729120944324\\
7.06599999999944	26.4676156219724\\
7.06799999999944	26.4623202148814\\
7.06999999999944	26.4570258729329\\
7.07199999999944	26.4517325958989\\
7.07399999999944	26.4464403835526\\
7.07599999999944	26.4411492356666\\
7.07799999999944	26.4358591520144\\
7.07999999999944	26.4305701323683\\
7.08199999999944	26.4252821765023\\
7.08399999999944	26.4199952841889\\
7.08599999999944	26.414709455202\\
7.08799999999944	26.4094246893144\\
7.08999999999944	26.4041409862999\\
7.09199999999944	26.398858345932\\
7.09399999999944	26.3935767679843\\
7.09599999999944	26.3882962522304\\
7.09799999999944	26.383016798444\\
7.09999999999944	26.3777384063991\\
7.10199999999944	26.3724610758697\\
7.10399999999944	26.3671848066297\\
7.10599999999944	26.3619095984531\\
7.10799999999944	26.3566354511143\\
7.10999999999944	26.3513623643873\\
7.11199999999944	26.3460903380465\\
7.11399999999944	26.340819371866\\
7.11599999999944	26.3355494656209\\
7.11799999999944	26.3302806190854\\
7.11999999999944	26.325012832034\\
7.12199999999944	26.3197461042416\\
7.12399999999944	26.3144804354827\\
7.12599999999944	26.3092158255324\\
7.12799999999944	26.3039522741658\\
7.12999999999944	26.2986897811576\\
7.13199999999944	26.2934283462828\\
7.13399999999944	26.288167969317\\
7.13599999999944	26.282908650035\\
7.13799999999944	26.2776503882125\\
7.13999999999944	26.2723931836246\\
7.14199999999944	26.2671370360468\\
7.14399999999944	26.2618819452547\\
7.14599999999944	26.256627911024\\
7.14799999999943	26.2513749331304\\
7.14999999999943	26.2461230113493\\
7.15199999999943	26.2408721454572\\
7.15399999999943	26.2356223352294\\
7.15599999999943	26.2303735804424\\
7.15799999999943	26.2251258808719\\
7.15999999999943	26.2198792362943\\
7.16199999999943	26.2146336464855\\
7.16399999999943	26.2093891112222\\
7.16599999999943	26.2041456302805\\
7.16799999999943	26.1989032034372\\
7.16999999999943	26.1936618304683\\
7.17199999999943	26.1884215111508\\
7.17399999999943	26.1831822452612\\
7.17599999999943	26.1779440325764\\
7.17799999999943	26.1727068728729\\
7.17999999999943	26.1674707659281\\
7.18199999999943	26.1622357115186\\
7.18399999999943	26.1570017094216\\
7.18599999999943	26.1517687594141\\
7.18799999999943	26.1465368612734\\
7.18999999999943	26.1413060147768\\
7.19199999999943	26.1360762197015\\
7.19399999999943	26.1308474758251\\
7.19599999999943	26.1256197829251\\
7.19799999999943	26.1203931407789\\
7.19999999999943	26.1151675491641\\
7.20199999999943	26.1099430078585\\
7.20399999999943	26.1047195166402\\
7.20599999999943	26.0994970752865\\
7.20799999999943	26.0942756835756\\
7.20999999999943	26.0890553412858\\
7.21199999999943	26.0838360481947\\
7.21399999999943	26.0786178040807\\
7.21599999999943	26.0734006087221\\
7.21799999999943	26.0681844618969\\
7.21999999999943	26.0629693633838\\
7.22199999999943	26.057755312961\\
7.22399999999943	26.0525423104072\\
7.22599999999943	26.0473303555008\\
7.22799999999943	26.0421194480208\\
7.22999999999943	26.0369095877455\\
7.23199999999943	26.031700774454\\
7.23399999999943	26.026493007925\\
7.23599999999943	26.0212862879376\\
7.23799999999942	26.0160806142709\\
7.23999999999942	26.0108759867039\\
7.24199999999942	26.0056724050156\\
7.24399999999942	26.0004698689853\\
7.24599999999942	25.9952683783926\\
7.24799999999942	25.9900679330167\\
7.24999999999942	25.9848685326371\\
7.25199999999942	25.9796701770329\\
7.25399999999942	25.9744728659842\\
7.25599999999942	25.9692765992708\\
7.25799999999942	25.9640813766719\\
7.25999999999942	25.9588871979677\\
7.26199999999942	25.9536940629379\\
7.26399999999942	25.9485019713625\\
7.26599999999942	25.9433109230216\\
7.26799999999942	25.9381209176952\\
7.26999999999942	25.9329319551636\\
7.27199999999942	25.927744035207\\
7.27399999999942	25.9225571576056\\
7.27599999999942	25.9173713221398\\
7.27799999999942	25.9121865285899\\
7.27999999999942	25.9070027767367\\
7.28199999999942	25.9018200663608\\
7.28399999999942	25.8966383972427\\
7.28599999999942	25.8914577691632\\
7.28799999999942	25.886278181903\\
7.28999999999942	25.8810996352432\\
7.29199999999942	25.8759221289642\\
7.29399999999942	25.8707456628479\\
7.29599999999942	25.8655702366745\\
7.29799999999942	25.8603958502257\\
7.29999999999942	25.8552225032826\\
7.30199999999942	25.8500501956262\\
7.30399999999942	25.8448789270383\\
7.30599999999942	25.8397086973\\
7.30799999999942	25.8345395061929\\
7.30999999999942	25.8293713534988\\
7.31199999999942	25.8242042389989\\
7.31399999999942	25.8190381624751\\
7.31599999999942	25.8138731237092\\
7.31799999999942	25.8087091224831\\
7.31999999999942	25.8035461585786\\
7.32199999999942	25.7983842317778\\
7.32399999999942	25.7932233418624\\
7.32599999999942	25.788063488615\\
7.32799999999941	25.7829046718175\\
7.32999999999941	25.7777468912523\\
7.33199999999941	25.7725901467016\\
7.33399999999941	25.7674344379476\\
7.33599999999941	25.7622797647735\\
7.33799999999941	25.7571261269607\\
7.33999999999941	25.7519735242927\\
7.34199999999941	25.7468219565519\\
7.34399999999941	25.741671423521\\
7.34599999999941	25.7365219249827\\
7.34799999999941	25.7313734607199\\
7.34999999999941	25.7262260305157\\
7.35199999999941	25.721079634153\\
7.35399999999941	25.7159342714148\\
7.35599999999941	25.7107899420843\\
7.35799999999941	25.7056466459448\\
7.35999999999941	25.7005043827793\\
7.36199999999941	25.6953631523713\\
7.36399999999941	25.6902229545044\\
7.36599999999941	25.6850837889619\\
7.36799999999941	25.6799456555271\\
7.36999999999941	25.6748085539842\\
7.37199999999941	25.6696724841161\\
7.37399999999941	25.6645374457073\\
7.37599999999941	25.6594034385407\\
7.37799999999941	25.6542704624013\\
7.37999999999941	25.6491385170722\\
7.38199999999941	25.6440076023376\\
7.38399999999941	25.6388777179818\\
7.38599999999941	25.6337488637887\\
7.38799999999941	25.6286210395426\\
7.38999999999941	25.6234942450277\\
7.39199999999941	25.6183684800284\\
7.39399999999941	25.6132437443292\\
7.39599999999941	25.6081200377146\\
7.39799999999941	25.6029973599687\\
7.39999999999941	25.5978757108766\\
7.40199999999941	25.5927550902226\\
7.40399999999941	25.5876354977916\\
7.40599999999941	25.5825169333686\\
7.40799999999941	25.5773993967381\\
7.40999999999941	25.5722828876854\\
7.41199999999941	25.5671674059951\\
7.41399999999941	25.5620529514527\\
7.41599999999941	25.556939523843\\
7.41799999999941	25.551827122951\\
7.4199999999994	25.5467157485626\\
7.4219999999994	25.5416054004626\\
7.4239999999994	25.5364960784368\\
7.4259999999994	25.5313877822701\\
7.4279999999994	25.5262805117483\\
7.4299999999994	25.5211742666572\\
7.4319999999994	25.5160690467821\\
7.4339999999994	25.5109648519088\\
7.4359999999994	25.5058616818233\\
7.4379999999994	25.5007595363113\\
7.4399999999994	25.4956584151585\\
7.4419999999994	25.4905583181512\\
7.4439999999994	25.4854592450753\\
7.4459999999994	25.4803611957168\\
7.4479999999994	25.4752641698621\\
7.4499999999994	25.4701681672973\\
7.4519999999994	25.4650731878085\\
7.4539999999994	25.4599792311822\\
7.4559999999994	25.4548862972051\\
7.4579999999994	25.4497943856633\\
7.4599999999994	25.4447034963436\\
7.4619999999994	25.4396136290323\\
7.4639999999994	25.4345247835162\\
7.4659999999994	25.4294369595824\\
7.4679999999994	25.4243501570173\\
7.4699999999994	25.4192643756077\\
7.4719999999994	25.4141796151409\\
7.4739999999994	25.4090958754036\\
7.4759999999994	25.404013156183\\
7.4779999999994	25.3989314572662\\
7.4799999999994	25.3938507784403\\
7.4819999999994	25.3887711194926\\
7.4839999999994	25.3836924802106\\
7.4859999999994	25.3786148603813\\
7.4879999999994	25.3735382597925\\
7.4899999999994	25.3684626782314\\
7.4919999999994	25.3633881154858\\
7.4939999999994	25.3583145713431\\
7.4959999999994	25.3532420455912\\
7.4979999999994	25.3481705380176\\
7.4999999999994	25.3431000484105\\
7.5019999999994	25.3380305765573\\
7.5039999999994	25.3329621222463\\
7.5059999999994	25.3278946852653\\
7.5079999999994	25.3228282654025\\
7.50999999999939	25.317762862446\\
7.51199999999939	25.3126984761839\\
7.51399999999939	25.3076351064045\\
7.51599999999939	25.3025727528961\\
7.51799999999939	25.2975114154471\\
7.51999999999939	25.292451093846\\
7.52199999999939	25.2873917878813\\
7.52399999999939	25.2823334973412\\
7.52599999999939	25.2772762220148\\
7.52799999999939	25.2722199616904\\
7.52999999999939	25.2671647161572\\
7.53199999999939	25.2621104852035\\
7.53399999999939	25.2570572686186\\
7.53599999999939	25.252005066191\\
7.53799999999939	25.2469538777102\\
7.53999999999939	25.2419037029649\\
7.54199999999939	25.2368545417443\\
7.54399999999939	25.2318063938374\\
7.54599999999939	25.2267592590337\\
7.54799999999939	25.2217131371224\\
7.54999999999939	25.2166680278928\\
7.55199999999939	25.2116239311346\\
7.55399999999939	25.2065808466369\\
7.55599999999939	25.2015387741894\\
7.55799999999939	25.1964977135816\\
7.55999999999939	25.1914576646035\\
7.56199999999939	25.1864186270445\\
7.56399999999939	25.1813806006945\\
7.56599999999939	25.1763435853433\\
7.56799999999939	25.1713075807807\\
7.56999999999939	25.1662725867969\\
7.57199999999939	25.1612386031819\\
7.57399999999939	25.1562056297256\\
7.57599999999939	25.1511736662184\\
7.57799999999939	25.1461427124504\\
7.57999999999939	25.1411127682118\\
7.58199999999939	25.1360838332926\\
7.58399999999939	25.131055907484\\
7.58599999999939	25.126028990576\\
7.58799999999939	25.1210030823591\\
7.58999999999939	25.1159781826236\\
7.59199999999939	25.1109542911607\\
7.59399999999939	25.1059314077606\\
7.59599999999939	25.1009095322142\\
7.59799999999939	25.0958886643126\\
7.59999999999939	25.0908688038464\\
7.60199999999938	25.0858499506064\\
7.60399999999938	25.080832104384\\
7.60599999999938	25.0758152649697\\
7.60799999999938	25.0707994321553\\
7.60999999999938	25.0657846057312\\
7.61199999999938	25.0607707854889\\
7.61399999999938	25.05575797122\\
7.61599999999938	25.0507461627158\\
7.61799999999938	25.0457353597673\\
7.61999999999938	25.0407255621662\\
7.62199999999938	25.035716769704\\
7.62399999999938	25.0307089821723\\
7.62599999999938	25.0257021993627\\
7.62799999999938	25.0206964210671\\
7.62999999999938	25.0156916470768\\
7.63199999999938	25.0106878771841\\
7.63399999999938	25.0056851111806\\
7.63599999999938	25.0006833488584\\
7.63799999999938	24.9956825900093\\
7.63999999999938	24.9906828344254\\
7.64199999999938	24.9856840818989\\
7.64399999999938	24.980686332222\\
7.64599999999938	24.9756895851869\\
7.64799999999938	24.9706938405858\\
7.64999999999938	24.965699098211\\
7.65199999999938	24.9607053578551\\
7.65399999999938	24.9557126193103\\
7.65599999999938	24.9507208823695\\
7.65799999999938	24.9457301468249\\
7.65999999999938	24.9407404124691\\
7.66199999999938	24.9357516790952\\
7.66399999999938	24.9307639464956\\
7.66599999999938	24.9257772144632\\
7.66799999999938	24.9207914827911\\
7.66999999999938	24.9158067512719\\
7.67199999999938	24.9108230196986\\
7.67399999999938	24.9058402878645\\
7.67599999999938	24.9008585555623\\
7.67799999999938	24.8958778225856\\
7.67999999999938	24.8908980887273\\
7.68199999999938	24.8859193537807\\
7.68399999999938	24.8809416175394\\
7.68599999999938	24.8759648797964\\
7.68799999999938	24.8709891403455\\
7.68999999999938	24.8660143989797\\
7.69199999999937	24.8610406554933\\
7.69399999999937	24.8560679096793\\
7.69599999999937	24.8510961613314\\
7.69799999999937	24.8461254102436\\
7.69999999999937	24.8411556562098\\
7.70199999999937	24.8361868990234\\
7.70399999999937	24.8312191384784\\
7.70599999999937	24.8262523743691\\
7.70799999999937	24.8212866064892\\
7.70999999999937	24.8163218346329\\
7.71199999999937	24.8113580585943\\
7.71399999999937	24.8063952781675\\
7.71599999999937	24.8014334931469\\
7.71799999999937	24.7964727033266\\
7.71999999999937	24.7915129085011\\
7.72199999999937	24.7865541084647\\
7.72399999999937	24.781596303012\\
7.72599999999937	24.7766394919375\\
7.72799999999937	24.7716836750356\\
7.72999999999937	24.7667288521012\\
7.73199999999937	24.7617750229288\\
7.73399999999937	24.7568221873133\\
7.73599999999937	24.7518703450493\\
7.73799999999937	24.7469194959317\\
7.73999999999937	24.7419696397557\\
7.74199999999937	24.7370207763159\\
7.74399999999937	24.7320729054076\\
7.74599999999937	24.7271260268257\\
7.74799999999937	24.7221801403655\\
7.74999999999937	24.717235245822\\
7.75199999999937	24.7122913429905\\
7.75399999999937	24.7073484316666\\
7.75599999999937	24.7024065116453\\
7.75799999999937	24.6974655827223\\
7.75999999999937	24.6925256446927\\
7.76199999999937	24.6875866973525\\
7.76399999999937	24.6826487404968\\
7.76599999999937	24.6777117739216\\
7.76799999999937	24.6727757974224\\
7.76999999999937	24.667840810795\\
7.77199999999937	24.6629068138354\\
7.77399999999937	24.6579738063391\\
7.77599999999937	24.6530417881026\\
7.77799999999937	24.6481107589211\\
7.77999999999937	24.6431807185912\\
7.78199999999936	24.6382516669088\\
7.78399999999936	24.6333236036699\\
7.78599999999936	24.628396528671\\
7.78799999999936	24.6234704417082\\
7.78999999999936	24.6185453425777\\
7.79199999999936	24.613621231076\\
7.79399999999936	24.6086981069994\\
7.79599999999936	24.6037759701446\\
7.79799999999936	24.5988548203078\\
7.79999999999936	24.5939346572857\\
7.80199999999936	24.5890154808748\\
7.80399999999936	24.5840972908723\\
7.80599999999936	24.5791800870744\\
7.80799999999936	24.5742638692782\\
7.80999999999936	24.5693486372804\\
7.81199999999936	24.5644343908779\\
7.81399999999936	24.5595211298677\\
7.81599999999936	24.5546088540466\\
7.81799999999936	24.5496975632122\\
7.81999999999936	24.5447872571611\\
7.82199999999936	24.5398779356908\\
7.82399999999936	24.5349695985982\\
7.82599999999936	24.530062245681\\
7.82799999999936	24.5251558767364\\
7.82999999999936	24.5202504915615\\
7.83199999999936	24.5153460899541\\
7.83399999999936	24.5104426717116\\
7.83599999999936	24.5055402366313\\
7.83799999999936	24.5006387845113\\
7.83999999999936	24.4957383151489\\
7.84199999999936	24.4908388283418\\
7.84399999999936	24.4859403238879\\
7.84599999999936	24.4810428015852\\
7.84799999999936	24.4761462612311\\
7.84999999999936	24.471250702624\\
7.85199999999936	24.4663561255616\\
7.85399999999936	24.4614625298421\\
7.85599999999936	24.4565699152635\\
7.85799999999936	24.4516782816238\\
7.85999999999936	24.4467876287216\\
7.86199999999936	24.4418979563549\\
7.86399999999936	24.4370092643218\\
7.86599999999936	24.4321215524212\\
7.86799999999936	24.427234820451\\
7.86999999999936	24.4223490682098\\
7.87199999999936	24.4174642954963\\
7.87399999999935	24.4125805021089\\
7.87599999999935	24.4076976878462\\
7.87799999999935	24.402815852507\\
7.87999999999935	24.3979349958898\\
7.88199999999935	24.3930551177938\\
7.88399999999935	24.3881762180176\\
7.88599999999935	24.3832982963599\\
7.88799999999935	24.37842135262\\
7.88999999999935	24.3735453865965\\
7.89199999999935	24.3686703980887\\
7.89399999999935	24.3637963868958\\
7.89599999999935	24.3589233528167\\
7.89799999999935	24.3540512956508\\
7.89999999999935	24.3491802151972\\
7.90199999999935	24.3443101112554\\
7.90399999999935	24.3394409836244\\
7.90599999999935	24.334572832104\\
7.90799999999935	24.3297056564934\\
7.90999999999935	24.3248394565926\\
7.91199999999935	24.3199742322006\\
7.91399999999935	24.315109983117\\
7.91599999999935	24.310246709142\\
7.91799999999935	24.305384410075\\
7.91999999999935	24.3005230857158\\
7.92199999999935	24.2956627358643\\
7.92399999999935	24.2908033603203\\
7.92599999999935	24.2859449588839\\
7.92799999999935	24.2810875313549\\
7.92999999999935	24.2762310775333\\
7.93199999999935	24.2713755972193\\
7.93399999999935	24.2665210902132\\
7.93599999999935	24.261667556315\\
7.93799999999935	24.2568149953249\\
7.93999999999935	24.2519634070433\\
7.94199999999935	24.2471127912707\\
7.94399999999935	24.2422631478073\\
7.94599999999935	24.2374144764534\\
7.94799999999935	24.2325667770099\\
7.94999999999935	24.2277200492772\\
7.95199999999935	24.2228742930558\\
7.95399999999935	24.2180295081464\\
7.95599999999935	24.2131856943498\\
7.95799999999935	24.2083428514666\\
7.95999999999935	24.2035009792979\\
7.96199999999935	24.1986600776445\\
7.96399999999934	24.193820146307\\
7.96599999999934	24.1889811850867\\
7.96799999999934	24.1841431937845\\
7.96999999999934	24.1793061722013\\
7.97199999999934	24.1744701201387\\
7.97399999999934	24.1696350373974\\
7.97599999999934	24.164800923779\\
7.97799999999934	24.1599677790843\\
7.97999999999934	24.1551356031152\\
7.98199999999934	24.1503043956725\\
7.98399999999934	24.145474156558\\
7.98599999999934	24.1406448855733\\
7.98799999999934	24.1358165825195\\
7.98999999999934	24.1309892471985\\
7.99199999999934	24.1261628794119\\
7.99399999999934	24.1213374789612\\
7.99599999999934	24.1165130456485\\
7.99799999999934	24.1116895792751\\
7.99999999999934	24.1068670796432\\
};
\addplot [color=mycolor1, forget plot]
  table[row sep=crcr]{%
7.99999999999934	24.1068670796432\\
8.00199999999934	24.1020455465545\\
8.00399999999934	24.0972249798112\\
8.00599999999934	24.0924053792149\\
8.00799999999934	24.0875867445678\\
8.00999999999934	24.0827690756721\\
8.01199999999934	24.0779523723301\\
8.01399999999935	24.0731366343434\\
8.01599999999935	24.0683218615146\\
8.01799999999935	24.0635080536461\\
8.01999999999935	24.0586952105401\\
8.02199999999935	24.0538833319992\\
8.02399999999935	24.0490724178254\\
8.02599999999935	24.0442624678216\\
8.02799999999935	24.03945348179\\
8.02999999999935	24.0346454595335\\
8.03199999999935	24.0298384008547\\
8.03399999999935	24.025032305556\\
8.03599999999935	24.0202271734406\\
8.03799999999935	24.0154230043108\\
8.03999999999935	24.01061979797\\
8.04199999999935	24.0058175542203\\
8.04399999999936	24.0010162728655\\
8.04599999999936	23.9962159537082\\
8.04799999999936	23.9914165965513\\
8.04999999999936	23.986618201198\\
8.05199999999936	23.9818207674515\\
8.05399999999936	23.9770242951152\\
8.05599999999936	23.9722287839918\\
8.05799999999936	23.967434233885\\
8.05999999999936	23.9626406445981\\
8.06199999999936	23.9578480159343\\
8.06399999999936	23.9530563476973\\
8.06599999999936	23.9482656396902\\
8.06799999999936	23.943475891717\\
8.06999999999936	23.9386871035811\\
8.07199999999937	23.933899275086\\
8.07399999999937	23.9291124060355\\
8.07599999999937	23.9243264962334\\
8.07799999999937	23.9195415454833\\
8.07999999999937	23.9147575535893\\
8.08199999999937	23.9099745203549\\
8.08399999999937	23.9051924455845\\
8.08599999999937	23.9004113290815\\
8.08799999999937	23.8956311706504\\
8.08999999999937	23.8908519700953\\
8.09199999999937	23.8860737272201\\
8.09399999999937	23.881296441829\\
8.09599999999937	23.8765201137261\\
8.09799999999937	23.8717447427161\\
8.09999999999937	23.866970328603\\
8.10199999999938	23.8621968711913\\
8.10399999999938	23.8574243702851\\
8.10599999999938	23.8526528256894\\
8.10799999999938	23.8478822372083\\
8.10999999999938	23.8431126046465\\
8.11199999999938	23.8383439278086\\
8.11399999999938	23.8335762064993\\
8.11599999999938	23.8288094405236\\
8.11799999999938	23.8240436296856\\
8.11999999999938	23.8192787737908\\
8.12199999999938	23.8145148726435\\
8.12399999999938	23.809751926049\\
8.12599999999938	23.804989933812\\
8.12799999999938	23.8002288957374\\
8.12999999999938	23.7954688116308\\
8.13199999999939	23.7907096812969\\
8.13399999999939	23.7859515045408\\
8.13599999999939	23.7811942811678\\
8.13799999999939	23.7764380109833\\
8.13999999999939	23.7716826937922\\
8.14199999999939	23.7669283294003\\
8.14399999999939	23.7621749176126\\
8.14599999999939	23.757422458235\\
8.14799999999939	23.7526709510724\\
8.14999999999939	23.7479203959309\\
8.15199999999939	23.7431707926156\\
8.15399999999939	23.7384221409324\\
8.15599999999939	23.733674440687\\
8.15799999999939	23.7289276916852\\
8.15999999999939	23.7241818937324\\
8.1619999999994	23.7194370466348\\
8.1639999999994	23.714693150198\\
8.1659999999994	23.7099502042281\\
8.1679999999994	23.7052082085311\\
8.1699999999994	23.7004671629129\\
8.1719999999994	23.6957270671795\\
8.1739999999994	23.6909879211373\\
8.1759999999994	23.6862497245921\\
8.1779999999994	23.6815124773501\\
8.1799999999994	23.676776179218\\
8.1819999999994	23.6720408300017\\
8.1839999999994	23.6673064295078\\
8.1859999999994	23.6625729775424\\
8.1879999999994	23.6578404739122\\
8.1899999999994	23.6531089184236\\
8.19199999999941	23.6483783108832\\
8.19399999999941	23.6436486510973\\
8.19599999999941	23.638919938873\\
8.19799999999941	23.6341921740165\\
8.19999999999941	23.6294653563348\\
8.20199999999941	23.6247394856348\\
8.20399999999941	23.6200145617231\\
8.20599999999941	23.6152905844065\\
8.20799999999941	23.6105675534923\\
8.20999999999941	23.605845468787\\
8.21199999999941	23.6011243300978\\
8.21399999999941	23.5964041372319\\
8.21599999999941	23.5916848899962\\
8.21799999999941	23.5869665881979\\
8.21999999999941	23.5822492316441\\
8.22199999999942	23.5775328201426\\
8.22399999999942	23.5728173534999\\
8.22599999999942	23.5681028315238\\
8.22799999999942	23.5633892540218\\
8.22999999999942	23.5586766208009\\
8.23199999999942	23.5539649316688\\
8.23399999999942	23.5492541864332\\
8.23599999999942	23.5445443849014\\
8.23799999999942	23.5398355268812\\
8.23999999999942	23.5351276121802\\
8.24199999999942	23.530420640606\\
8.24399999999942	23.5257146119668\\
8.24599999999942	23.5210095260698\\
8.24799999999942	23.5163053827232\\
8.24999999999942	23.5116021817348\\
8.25199999999943	23.5068999229127\\
8.25399999999943	23.5021986060647\\
8.25599999999943	23.4974982309989\\
8.25799999999943	23.4927987975236\\
8.25999999999943	23.4881003054466\\
8.26199999999943	23.4834027545763\\
8.26399999999943	23.4787061447207\\
8.26599999999943	23.4740104756884\\
8.26799999999943	23.4693157472873\\
8.26999999999943	23.4646219593262\\
8.27199999999943	23.4599291116134\\
8.27399999999943	23.4552372039571\\
8.27599999999943	23.4505462361661\\
8.27799999999943	23.4458562080488\\
8.27999999999943	23.4411671194138\\
8.28199999999944	23.4364789700698\\
8.28399999999944	23.4317917598253\\
8.28599999999944	23.4271054884895\\
8.28799999999944	23.4224201558707\\
8.28999999999944	23.417735761778\\
8.29199999999944	23.4130523060202\\
8.29399999999944	23.408369788406\\
8.29599999999944	23.4036882087447\\
8.29799999999944	23.3990075668452\\
8.29999999999944	23.3943278625165\\
8.30199999999944	23.3896490955677\\
8.30399999999944	23.3849712658081\\
8.30599999999944	23.3802943730466\\
8.30799999999944	23.3756184170925\\
8.30999999999944	23.3709433977554\\
8.31199999999945	23.3662693148446\\
8.31399999999945	23.3615961681691\\
8.31599999999945	23.3569239575386\\
8.31799999999945	23.3522526827624\\
8.31999999999945	23.3475823436502\\
8.32199999999945	23.3429129400114\\
8.32399999999945	23.3382444716558\\
8.32599999999945	23.3335769383928\\
8.32799999999945	23.3289103400324\\
8.32999999999945	23.3242446763839\\
8.33199999999945	23.3195799472576\\
8.33399999999945	23.314916152463\\
8.33599999999945	23.3102532918102\\
8.33799999999945	23.3055913651087\\
8.33999999999945	23.3009303721689\\
8.34199999999946	23.2962703128006\\
8.34399999999946	23.2916111868141\\
8.34599999999946	23.2869529940193\\
8.34799999999946	23.2822957342262\\
8.34999999999946	23.2776394072455\\
8.35199999999946	23.2729840128868\\
8.35399999999946	23.2683295509609\\
8.35599999999946	23.2636760212781\\
8.35799999999946	23.2590234236482\\
8.35999999999946	23.2543717578825\\
8.36199999999946	23.2497210237907\\
8.36399999999946	23.2450712211835\\
8.36599999999946	23.2404223498719\\
8.36799999999946	23.2357744096659\\
8.36999999999946	23.2311274003766\\
8.37199999999947	23.2264813218144\\
8.37399999999947	23.2218361737901\\
8.37599999999947	23.2171919561147\\
8.37799999999947	23.2125486685988\\
8.37999999999947	23.2079063110532\\
8.38199999999947	23.203264883289\\
8.38399999999947	23.1986243851172\\
8.38599999999947	23.1939848163486\\
8.38799999999947	23.1893461767944\\
8.38999999999947	23.1847084662655\\
8.39199999999947	23.1800716845733\\
8.39399999999947	23.1754358315289\\
8.39599999999947	23.1708009069433\\
8.39799999999947	23.1661669106283\\
8.39999999999947	23.1615338423948\\
8.40199999999948	23.1569017020538\\
8.40399999999948	23.1522704894176\\
8.40599999999948	23.147640204297\\
8.40799999999948	23.1430108465039\\
8.40999999999948	23.1383824158495\\
8.41199999999948	23.1337549121456\\
8.41399999999948	23.1291283352038\\
8.41599999999948	23.1245026848356\\
8.41799999999948	23.119877960853\\
8.41999999999948	23.1152541630674\\
8.42199999999948	23.1106312912909\\
8.42399999999948	23.1060093453352\\
8.42599999999948	23.1013883250124\\
8.42799999999948	23.096768230134\\
8.42999999999948	23.0921490605127\\
8.43199999999949	23.0875308159599\\
8.43399999999949	23.0829134962878\\
8.43599999999949	23.0782971013087\\
8.43799999999949	23.0736816308347\\
8.43999999999949	23.0690670846777\\
8.44199999999949	23.0644534626505\\
8.44399999999949	23.0598407645649\\
8.44599999999949	23.0552289902335\\
8.44799999999949	23.0506181394686\\
8.44999999999949	23.0460082120827\\
8.45199999999949	23.0413992078881\\
8.45399999999949	23.0367911266976\\
8.45599999999949	23.0321839683235\\
8.45799999999949	23.0275777325784\\
8.45999999999949	23.0229724192749\\
8.4619999999995	23.0183680282259\\
8.4639999999995	23.0137645592443\\
8.4659999999995	23.0091620121423\\
8.4679999999995	23.004560386733\\
8.4699999999995	22.9999596828294\\
8.4719999999995	22.9953599002443\\
8.4739999999995	22.9907610387907\\
8.4759999999995	22.9861630982814\\
8.4779999999995	22.9815660785297\\
8.4799999999995	22.9769699793483\\
8.4819999999995	22.9723748005509\\
8.4839999999995	22.9677805419503\\
8.4859999999995	22.9631872033596\\
8.4879999999995	22.9585947845923\\
8.4899999999995	22.9540032854617\\
8.49199999999951	22.9494127057809\\
8.49399999999951	22.9448230453635\\
8.49599999999951	22.9402343040228\\
8.49799999999951	22.9356464815725\\
8.49999999999951	22.9310595778258\\
8.50199999999951	22.9264735925965\\
8.50399999999951	22.9218885256981\\
8.50599999999951	22.9173043769443\\
8.50799999999951	22.9127211461487\\
8.50999999999951	22.9081388331251\\
8.51199999999951	22.9035574376873\\
8.51399999999951	22.898976959649\\
8.51599999999951	22.8943973988241\\
8.51799999999951	22.8898187550266\\
8.51999999999951	22.8852410280703\\
8.52199999999952	22.8806642177695\\
8.52399999999952	22.8760883239378\\
8.52599999999952	22.8715133463897\\
8.52799999999952	22.8669392849389\\
8.52999999999952	22.8623661394\\
8.53199999999952	22.8577939095867\\
8.53399999999952	22.8532225953139\\
8.53599999999952	22.8486521963952\\
8.53799999999952	22.8440827126454\\
8.53999999999952	22.8395141438786\\
8.54199999999952	22.8349464899096\\
8.54399999999952	22.8303797505524\\
8.54599999999952	22.8258139256218\\
8.54799999999952	22.8212490149323\\
8.54999999999952	22.8166850182985\\
8.55199999999953	22.812121935535\\
8.55399999999953	22.8075597664565\\
8.55599999999953	22.8029985108778\\
8.55799999999953	22.7984381686133\\
8.55999999999953	22.7938787394784\\
8.56199999999953	22.7893202232874\\
8.56399999999953	22.7847626198554\\
8.56599999999953	22.7802059289976\\
8.56799999999953	22.7756501505287\\
8.56999999999953	22.7710952842636\\
8.57199999999953	22.7665413300174\\
8.57399999999953	22.7619882876057\\
8.57599999999953	22.7574361568429\\
8.57799999999953	22.7528849375448\\
8.57999999999953	22.7483346295262\\
8.58199999999954	22.7437852326028\\
8.58399999999954	22.7392367465895\\
8.58599999999954	22.7346891713019\\
8.58799999999954	22.7301425065554\\
8.58999999999954	22.7255967521651\\
8.59199999999954	22.7210519079472\\
8.59399999999954	22.7165079737165\\
8.59599999999954	22.7119649492889\\
8.59799999999954	22.7074228344803\\
8.59999999999954	22.7028816291057\\
8.60199999999954	22.6983413329812\\
8.60399999999954	22.6938019459226\\
8.60599999999954	22.6892634677455\\
8.60799999999954	22.6847258982658\\
8.60999999999954	22.6801892372993\\
8.61199999999955	22.675653484662\\
8.61399999999955	22.6711186401699\\
8.61599999999955	22.6665847036389\\
8.61799999999955	22.6620516748851\\
8.61999999999955	22.6575195537244\\
8.62199999999955	22.652988339973\\
8.62399999999955	22.6484580334475\\
8.62599999999955	22.6439286339633\\
8.62799999999955	22.6394001413373\\
8.62999999999955	22.6348725553855\\
8.63199999999955	22.6303458759242\\
8.63399999999955	22.6258201027699\\
8.63599999999955	22.621295235739\\
8.63799999999955	22.6167712746479\\
8.63999999999955	22.612248219313\\
8.64199999999956	22.6077260695511\\
8.64399999999956	22.6032048251785\\
8.64599999999956	22.5986844860121\\
8.64799999999956	22.5941650518685\\
8.64999999999956	22.5896465225642\\
8.65199999999956	22.585128897916\\
8.65399999999956	22.580612177741\\
8.65599999999956	22.5760963618556\\
8.65799999999956	22.571581450077\\
8.65999999999956	22.5670674422218\\
8.66199999999956	22.5625543381075\\
8.66399999999956	22.5580421375505\\
8.66599999999956	22.553530840368\\
8.66799999999956	22.5490204463776\\
8.66999999999956	22.5445109553955\\
8.67199999999957	22.5400023672396\\
8.67399999999957	22.5354946817268\\
8.67599999999957	22.5309878986745\\
8.67799999999957	22.5264820178999\\
8.67999999999957	22.5219770392201\\
8.68199999999957	22.517472962453\\
8.68399999999957	22.5129697874155\\
8.68599999999957	22.5084675139253\\
8.68799999999957	22.5039661418\\
8.68999999999957	22.499465670857\\
8.69199999999957	22.4949661009139\\
8.69399999999957	22.4904674317882\\
8.69599999999957	22.4859696632977\\
8.69799999999957	22.48147279526\\
8.69999999999957	22.4769768274931\\
8.70199999999958	22.4724817598144\\
8.70399999999958	22.467987592042\\
8.70599999999958	22.4634943239937\\
8.70799999999958	22.4590019554875\\
8.70999999999958	22.4545104863411\\
8.71199999999958	22.4500199163728\\
8.71399999999958	22.4455302454004\\
8.71599999999958	22.441041473242\\
8.71799999999958	22.4365535997159\\
8.71999999999958	22.4320666246401\\
8.72199999999958	22.4275805478328\\
8.72399999999958	22.4230953691122\\
8.72599999999958	22.4186110882966\\
8.72799999999958	22.4141277052044\\
8.72999999999958	22.409645219654\\
8.73199999999959	22.4051636314636\\
8.73399999999959	22.400682940452\\
8.73599999999959	22.3962031464373\\
8.73799999999959	22.391724249238\\
8.73999999999959	22.387246248673\\
8.74199999999959	22.3827691445609\\
8.74399999999959	22.3782929367199\\
8.74599999999959	22.3738176249692\\
8.74799999999959	22.3693432091272\\
8.74999999999959	22.3648696890128\\
8.75199999999959	22.360397064445\\
8.75399999999959	22.3559253352421\\
8.75599999999959	22.3514545012235\\
8.75799999999959	22.3469845622081\\
8.75999999999959	22.3425155180146\\
8.7619999999996	22.338047368462\\
8.7639999999996	22.3335801133698\\
8.7659999999996	22.3291137525567\\
8.7679999999996	22.3246482858417\\
8.7699999999996	22.3201837130446\\
8.7719999999996	22.3157200339841\\
8.7739999999996	22.3112572484796\\
8.7759999999996	22.3067953563503\\
8.7779999999996	22.3023343574157\\
8.7799999999996	22.2978742514953\\
8.7819999999996	22.293415038408\\
8.7839999999996	22.2889567179737\\
8.7859999999996	22.2844992900119\\
8.7879999999996	22.2800427543419\\
8.7899999999996	22.2755871107835\\
8.79199999999961	22.2711323591564\\
8.79399999999961	22.2666784992798\\
8.79599999999961	22.2622255309739\\
8.79799999999961	22.2577734540582\\
8.79999999999961	22.2533222683525\\
8.80199999999961	22.2488719736766\\
8.80399999999961	22.2444225698507\\
8.80599999999961	22.2399740566942\\
8.80799999999961	22.2355264340273\\
8.80999999999961	22.2310797016699\\
8.81199999999961	22.2266338594422\\
8.81399999999961	22.2221889071641\\
8.81599999999961	22.2177448446558\\
8.81799999999961	22.2133016717373\\
8.81999999999961	22.208859388229\\
8.82199999999962	22.2044179939509\\
8.82399999999962	22.1999774887235\\
8.82599999999962	22.1955378723669\\
8.82799999999962	22.1910991447016\\
8.82999999999962	22.1866613055478\\
8.83199999999962	22.1822243547263\\
8.83399999999962	22.1777882920571\\
8.83599999999962	22.1733531173608\\
8.83799999999962	22.1689188304583\\
8.83999999999962	22.1644854311698\\
8.84199999999962	22.1600529193163\\
8.84399999999962	22.1556212947181\\
8.84599999999962	22.151190557196\\
8.84799999999962	22.146760706571\\
8.84999999999962	22.1423317426636\\
8.85199999999963	22.1379036652945\\
8.85399999999963	22.1334764742851\\
8.85599999999963	22.1290501694559\\
8.85799999999963	22.1246247506278\\
8.85999999999963	22.1202002176221\\
8.86199999999963	22.1157765702595\\
8.86399999999963	22.1113538083612\\
8.86599999999963	22.1069319317484\\
8.86799999999963	22.1025109402422\\
8.86999999999963	22.0980908336636\\
8.87199999999963	22.0936716118341\\
8.87399999999963	22.0892532745746\\
8.87599999999963	22.084835821707\\
8.87799999999963	22.0804192530521\\
8.87999999999963	22.0760035684314\\
8.88199999999964	22.0715887676665\\
8.88399999999964	22.0671748505786\\
8.88599999999964	22.0627618169894\\
8.88799999999964	22.0583496667206\\
8.88999999999964	22.0539383995932\\
8.89199999999964	22.0495280154296\\
8.89399999999964	22.0451185140509\\
8.89599999999964	22.0407098952788\\
8.89799999999964	22.0363021589353\\
8.89999999999964	22.0318953048423\\
8.90199999999964	22.0274893328208\\
8.90399999999964	22.0230842426937\\
8.90599999999964	22.0186800342822\\
8.90799999999964	22.0142767074088\\
8.90999999999964	22.009874261895\\
8.91199999999965	22.0054726975629\\
8.91399999999965	22.0010720142346\\
8.91599999999965	21.9966722117322\\
8.91799999999965	21.9922732898781\\
8.91999999999965	21.987875248494\\
8.92199999999965	21.9834780874025\\
8.92399999999965	21.9790818064254\\
8.92599999999965	21.9746864053855\\
8.92799999999965	21.9702918841049\\
8.92999999999965	21.9658982424059\\
8.93199999999965	21.961505480111\\
8.93399999999965	21.9571135970426\\
8.93599999999965	21.9527225930232\\
8.93799999999965	21.9483324678754\\
8.93999999999965	21.9439432214216\\
8.94199999999966	21.9395548534846\\
8.94399999999966	21.935167363887\\
8.94599999999966	21.9307807524515\\
8.94799999999966	21.9263950190005\\
8.94999999999966	21.9220101633572\\
8.95199999999966	21.9176261853439\\
8.95399999999966	21.9132430847842\\
8.95599999999966	21.9088608615002\\
8.95799999999966	21.9044795153153\\
8.95999999999966	21.9000990460521\\
8.96199999999966	21.8957194535339\\
8.96399999999966	21.8913407375839\\
8.96599999999966	21.8869628980246\\
8.96799999999966	21.8825859346794\\
8.96999999999966	21.8782098473717\\
8.97199999999967	21.8738346359241\\
8.97399999999967	21.8694603001604\\
8.97599999999967	21.8650868399036\\
8.97799999999967	21.860714254977\\
8.97999999999967	21.856342545204\\
8.98199999999967	21.851971710408\\
8.98399999999967	21.8476017504124\\
8.98599999999967	21.8432326650407\\
8.98799999999967	21.8388644541162\\
8.98999999999967	21.8344971174628\\
8.99199999999967	21.8301306549036\\
8.99399999999967	21.8257650662627\\
8.99599999999967	21.8214003513634\\
8.99799999999967	21.8170365100297\\
8.99999999999967	21.812673542085\\
9.00199999999968	21.8083114473532\\
9.00399999999968	21.8039502256582\\
9.00599999999968	21.7995898768238\\
9.00799999999968	21.7952304006739\\
9.00999999999968	21.7908717970324\\
9.01199999999968	21.7865140657232\\
9.01399999999968	21.7821572065704\\
9.01599999999968	21.777801219398\\
9.01799999999968	21.7734461040301\\
9.01999999999968	21.7690918602907\\
9.02199999999968	21.7647384880041\\
9.02399999999968	21.7603859869946\\
9.02599999999968	21.756034357086\\
9.02799999999968	21.7516835981031\\
9.02999999999968	21.7473337098699\\
9.03199999999969	21.7429846922108\\
9.03399999999969	21.7386365449501\\
9.03599999999969	21.7342892679123\\
9.03799999999969	21.7299428609221\\
9.03999999999969	21.7255973238037\\
9.04199999999969	21.7212526563816\\
9.04399999999969	21.7169088584806\\
9.04599999999969	21.7125659299253\\
9.04799999999969	21.7082238705401\\
9.04999999999969	21.7038826801501\\
9.05199999999969	21.6995423585796\\
9.05399999999969	21.6952029056537\\
9.05599999999969	21.6908643211969\\
9.05799999999969	21.6865266050343\\
9.05999999999969	21.6821897569907\\
9.0619999999997	21.6778537768911\\
9.0639999999997	21.6735186645604\\
9.0659999999997	21.6691844198235\\
9.0679999999997	21.6648510425056\\
9.0699999999997	21.6605185324317\\
9.0719999999997	21.6561868894269\\
9.0739999999997	21.6518561133164\\
9.0759999999997	21.6475262039253\\
9.0779999999997	21.643197161079\\
9.0799999999997	21.6388689846025\\
9.0819999999997	21.6345416743212\\
9.0839999999997	21.6302152300607\\
9.0859999999997	21.625889651646\\
9.0879999999997	21.6215649389028\\
9.0899999999997	21.6172410916565\\
9.09199999999971	21.6129181097325\\
9.09399999999971	21.6085959929562\\
9.09599999999971	21.6042747411535\\
9.09799999999971	21.5999543541499\\
9.09999999999971	21.5956348317708\\
9.10199999999971	21.5913161738423\\
9.10399999999971	21.5869983801898\\
9.10599999999971	21.5826814506392\\
9.10799999999971	21.5783653850163\\
9.10999999999971	21.5740501831469\\
9.11199999999971	21.5697358448568\\
9.11399999999971	21.565422369972\\
9.11599999999971	21.5611097583184\\
9.11799999999971	21.556798009722\\
9.11999999999972	21.5524871240089\\
9.12199999999972	21.548177101005\\
9.12399999999972	21.5438679405367\\
9.12599999999972	21.5395596424296\\
9.12799999999972	21.5352522065104\\
9.12999999999972	21.5309456326051\\
9.13199999999972	21.5266399205401\\
9.13399999999972	21.5223350701414\\
9.13599999999972	21.5180310812356\\
9.13799999999972	21.5137279536488\\
9.13999999999972	21.5094256872077\\
9.14199999999972	21.5051242817386\\
9.14399999999972	21.5008237370679\\
9.14599999999972	21.4965240530221\\
9.14799999999972	21.492225229428\\
9.14999999999973	21.487927266112\\
9.15199999999973	21.4836301629005\\
9.15399999999973	21.4793339196207\\
9.15599999999973	21.4750385360989\\
9.15799999999973	21.4707440121618\\
9.15999999999973	21.4664503476366\\
9.16199999999973	21.4621575423496\\
9.16399999999973	21.4578655961281\\
9.16599999999973	21.4535745087987\\
9.16799999999973	21.4492842801884\\
9.16999999999973	21.4449949101242\\
9.17199999999973	21.4407063984331\\
9.17399999999973	21.436418744942\\
9.17599999999973	21.4321319494781\\
9.17799999999973	21.4278460118686\\
9.17999999999974	21.4235609319405\\
9.18199999999974	21.4192767095211\\
9.18399999999974	21.4149933444375\\
9.18599999999974	21.4107108365171\\
9.18799999999974	21.406429185587\\
9.18999999999974	21.4021483914747\\
9.19199999999974	21.3978684540076\\
9.19399999999974	21.3935893730131\\
9.19599999999974	21.3893111483184\\
9.19799999999974	21.3850337797513\\
9.19999999999974	21.3807572671392\\
9.20199999999974	21.3764816103096\\
9.20399999999974	21.3722068090904\\
9.20599999999974	21.3679328633089\\
9.20799999999974	21.3636597727929\\
9.20999999999975	21.35938753737\\
9.21199999999975	21.355116156868\\
9.21399999999975	21.3508456311148\\
9.21599999999975	21.3465759599381\\
9.21799999999975	21.3423071431658\\
9.21999999999975	21.3380391806258\\
9.22199999999975	21.3337720721461\\
9.22399999999975	21.3295058175544\\
9.22599999999975	21.3252404166792\\
9.22799999999975	21.320975869348\\
9.22999999999975	21.3167121753892\\
9.23199999999975	21.3124493346308\\
9.23399999999975	21.308187346901\\
9.23599999999975	21.303926212028\\
9.23799999999975	21.29966592984\\
9.23999999999976	21.2954065001653\\
9.24199999999976	21.2911479228322\\
9.24399999999976	21.2868901976689\\
9.24599999999976	21.282633324504\\
9.24799999999976	21.2783773031657\\
9.24999999999976	21.2741221334827\\
9.25199999999976	21.2698678152832\\
9.25399999999976	21.265614348396\\
9.25599999999976	21.2613617326494\\
9.25799999999976	21.2571099678722\\
9.25999999999976	21.252859053893\\
9.26199999999976	21.2486089905404\\
9.26399999999976	21.2443597776432\\
9.26599999999976	21.2401114150298\\
9.26799999999976	21.2358639025296\\
9.26999999999977	21.2316172399709\\
9.27199999999977	21.2273714271828\\
9.27399999999977	21.223126463994\\
9.27599999999977	21.2188823502338\\
9.27799999999977	21.2146390857307\\
9.27999999999977	21.2103966703139\\
9.28199999999977	21.2061551038125\\
9.28399999999977	21.2019143860556\\
9.28599999999977	21.1976745168722\\
9.28799999999977	21.1934354960916\\
9.28999999999977	21.1891973235427\\
9.29199999999977	21.184959999055\\
9.29399999999977	21.1807235224576\\
9.29599999999977	21.1764878935799\\
9.29799999999977	21.1722531122511\\
9.29999999999978	21.1680191783007\\
9.30199999999978	21.163786091558\\
9.30399999999978	21.1595538518526\\
9.30599999999978	21.1553224590138\\
9.30799999999978	21.1510919128711\\
9.30999999999978	21.1468622132544\\
9.31199999999978	21.1426333599928\\
9.31399999999978	21.1384053529162\\
9.31599999999978	21.1341781918541\\
9.31799999999978	21.1299518766363\\
9.31999999999978	21.1257264070926\\
9.32199999999978	21.1215017830527\\
9.32399999999978	21.1172780043465\\
9.32599999999978	21.1130550708037\\
9.32799999999978	21.1088329822541\\
9.32999999999979	21.1046117385279\\
9.33199999999979	21.1003913394548\\
9.33399999999979	21.0961717848649\\
9.33599999999979	21.0919530745883\\
9.33799999999979	21.0877352084547\\
9.33999999999979	21.0835181862946\\
9.34199999999979	21.0793020079381\\
9.34399999999979	21.0750866732151\\
9.34599999999979	21.070872181956\\
9.34799999999979	21.066658533991\\
9.34999999999979	21.0624457291504\\
9.35199999999979	21.0582337672646\\
9.35399999999979	21.0540226481637\\
9.35599999999979	21.0498123716781\\
9.35799999999979	21.0456029376386\\
9.3599999999998	21.0413943458754\\
9.3619999999998	21.0371865962188\\
9.3639999999998	21.0329796884998\\
9.3659999999998	21.0287736225485\\
9.3679999999998	21.0245683981958\\
9.3699999999998	21.0203640152722\\
9.3719999999998	21.0161604736085\\
9.3739999999998	21.0119577730353\\
9.3759999999998	21.0077559133833\\
9.3779999999998	21.0035548944835\\
9.3799999999998	20.9993547161666\\
9.3819999999998	20.9951553782632\\
9.3839999999998	20.9909568806046\\
9.3859999999998	20.9867592230216\\
9.3879999999998	20.9825624053451\\
9.38999999999981	20.978366427406\\
9.39199999999981	20.9741712890355\\
9.39399999999981	20.9699769900649\\
9.39599999999981	20.9657835303247\\
9.39799999999981	20.9615909096465\\
9.39999999999981	20.9573991278614\\
9.40199999999981	20.9532081848005\\
9.40399999999981	20.9490180802953\\
9.40599999999981	20.9448288141767\\
9.40799999999981	20.9406403862764\\
9.40999999999981	20.9364527964254\\
9.41199999999981	20.9322660444555\\
9.41399999999981	20.9280801301978\\
9.41599999999981	20.9238950534839\\
9.41799999999981	20.9197108141454\\
9.41999999999982	20.9155274120136\\
9.42199999999982	20.9113448469203\\
9.42399999999982	20.907163118697\\
9.42599999999982	20.9029822271755\\
9.42799999999982	20.8988021721871\\
9.42999999999982	20.8946229535639\\
9.43199999999982	20.8904445711375\\
9.43399999999982	20.8862670247396\\
9.43599999999982	20.8820903142023\\
9.43799999999982	20.8779144393571\\
9.43999999999982	20.8737394000362\\
9.44199999999982	20.8695651960714\\
9.44399999999982	20.8653918272947\\
9.44599999999982	20.8612192935381\\
9.44799999999982	20.8570475946336\\
9.44999999999983	20.8528767304133\\
9.45199999999983	20.8487067007093\\
9.45399999999983	20.8445375053538\\
9.45599999999983	20.8403691441787\\
9.45799999999983	20.8362016170168\\
9.45999999999983	20.8320349236998\\
9.46199999999983	20.8278690640602\\
9.46399999999983	20.8237040379305\\
9.46599999999983	20.8195398451427\\
9.46799999999983	20.8153764855294\\
9.46999999999983	20.811213958923\\
9.47199999999983	20.807052265156\\
9.47399999999983	20.8028914040608\\
9.47599999999983	20.7987313754701\\
9.47799999999983	20.7945721792163\\
9.47999999999984	20.7904138151322\\
9.48199999999984	20.7862562830502\\
9.48399999999984	20.7820995828031\\
9.48599999999984	20.7779437142236\\
9.48799999999984	20.7737886771447\\
9.48999999999984	20.7696344713987\\
9.49199999999984	20.7654810968188\\
9.49399999999984	20.7613285532377\\
9.49599999999984	20.7571768404882\\
9.49799999999984	20.7530259584036\\
9.49999999999984	20.7488759068164\\
9.50199999999984	20.7447266855599\\
9.50399999999984	20.740578294467\\
9.50599999999984	20.7364307333708\\
9.50799999999984	20.7322840021045\\
9.50999999999985	20.7281381005011\\
9.51199999999985	20.7239930283939\\
9.51399999999985	20.7198487856158\\
9.51599999999985	20.7157053720004\\
9.51799999999985	20.7115627873809\\
9.51999999999985	20.7074210315904\\
9.52199999999985	20.7032801044626\\
9.52399999999985	20.6991400058306\\
9.52599999999985	20.695000735528\\
9.52799999999985	20.6908622933882\\
9.52999999999985	20.6867246792446\\
9.53199999999985	20.6825878929309\\
9.53399999999985	20.6784519342805\\
9.53599999999985	20.674316803127\\
9.53799999999985	20.6701824993041\\
9.53999999999986	20.6660490226454\\
9.54199999999986	20.6619163729848\\
9.54399999999986	20.6577845501558\\
9.54599999999986	20.6536535539922\\
9.54799999999986	20.6495233843279\\
9.54999999999986	20.6453940409969\\
9.55199999999986	20.6412655238327\\
9.55399999999986	20.6371378326693\\
9.55599999999986	20.6330109673409\\
9.55799999999986	20.6288849276813\\
9.55999999999986	20.6247597135247\\
9.56199999999986	20.6206353247047\\
9.56399999999986	20.6165117610559\\
9.56599999999986	20.6123890224121\\
9.56799999999986	20.6082671086077\\
9.56999999999987	20.6041460194766\\
9.57199999999987	20.6000257548531\\
9.57399999999987	20.5959063145717\\
9.57599999999987	20.5917876984666\\
9.57799999999987	20.587669906372\\
9.57999999999987	20.5835529381224\\
9.58199999999987	20.579436793552\\
9.58399999999987	20.5753214724955\\
9.58599999999987	20.5712069747872\\
9.58799999999987	20.5670933002617\\
9.58999999999987	20.5629804487535\\
9.59199999999987	20.558868420097\\
9.59399999999987	20.5547572141273\\
9.59599999999987	20.5506468306785\\
9.59799999999987	20.5465372695856\\
9.59999999999988	20.5424285306831\\
9.60199999999988	20.5383206138061\\
9.60399999999988	20.534213518789\\
9.60599999999988	20.5301072454668\\
9.60799999999988	20.5260017936744\\
9.60999999999988	20.5218971632465\\
9.61199999999988	20.5177933540183\\
9.61399999999988	20.5136903658245\\
9.61599999999988	20.5095881985003\\
9.61799999999988	20.5054868518806\\
9.61999999999988	20.5013863258003\\
9.62199999999988	20.4972866200949\\
9.62399999999988	20.4931877345992\\
9.62599999999988	20.4890896691486\\
9.62799999999988	20.4849924235781\\
9.62999999999989	20.4808959977231\\
9.63199999999989	20.4768003914187\\
9.63399999999989	20.4727056045005\\
9.63599999999989	20.4686116368035\\
9.63799999999989	20.4645184881632\\
9.63999999999989	20.4604261584151\\
9.64199999999989	20.4563346473944\\
9.64399999999989	20.4522439549369\\
9.64599999999989	20.4481540808779\\
9.64799999999989	20.4440650250531\\
9.64999999999989	20.4399767872979\\
9.65199999999989	20.435889367448\\
9.65399999999989	20.431802765339\\
9.65599999999989	20.4277169808067\\
9.65799999999989	20.4236320136868\\
9.6599999999999	20.4195478638151\\
9.6619999999999	20.4154645310271\\
9.6639999999999	20.4113820151589\\
9.6659999999999	20.4073003160464\\
9.6679999999999	20.4032194335252\\
9.6699999999999	20.3991393674316\\
9.6719999999999	20.3950601176013\\
9.6739999999999	20.3909816838702\\
9.6759999999999	20.3869040660746\\
9.6779999999999	20.3828272640506\\
9.6799999999999	20.3787512776339\\
9.6819999999999	20.3746761066611\\
9.6839999999999	20.370601750968\\
9.6859999999999	20.3665282103911\\
9.6879999999999	20.3624554847664\\
9.68999999999991	20.3583835739302\\
9.69199999999991	20.3543124777189\\
9.69399999999991	20.350242195969\\
9.69599999999991	20.3461727285165\\
9.69799999999991	20.342104075198\\
9.69999999999991	20.3380362358499\\
9.70199999999991	20.3339692103088\\
9.70399999999991	20.3299029984111\\
9.70599999999991	20.3258375999933\\
9.70799999999991	20.3217730148921\\
9.70999999999991	20.3177092429439\\
9.71199999999991	20.3136462839858\\
9.71399999999991	20.3095841378539\\
9.71599999999991	20.3055228043853\\
9.71799999999991	20.3014622834167\\
9.71999999999992	20.2974025747848\\
9.72199999999992	20.2933436783264\\
9.72399999999992	20.2892855938785\\
9.72599999999992	20.2852283212778\\
9.72799999999992	20.2811718603615\\
9.72999999999992	20.2771162109661\\
9.73199999999992	20.273061372929\\
9.73399999999992	20.269007346087\\
9.73599999999992	20.2649541302774\\
9.73799999999992	20.2609017253369\\
9.73999999999992	20.256850131103\\
9.74199999999992	20.2527993474126\\
9.74399999999992	20.248749374103\\
9.74599999999992	20.2447002110115\\
9.74799999999992	20.2406518579753\\
9.74999999999993	20.2366043148316\\
9.75199999999993	20.232557581418\\
9.75399999999993	20.2285116575716\\
9.75599999999993	20.2244665431298\\
9.75799999999993	20.2204222379302\\
9.75999999999993	20.2163787418101\\
9.76199999999993	20.2123360546072\\
9.76399999999993	20.2082941761587\\
9.76599999999993	20.2042531063025\\
9.76799999999993	20.2002128448761\\
9.76999999999993	20.1961733917171\\
9.77199999999993	20.1921347466632\\
9.77399999999993	20.188096909552\\
9.77599999999993	20.1840598802213\\
9.77799999999993	20.180023658509\\
9.77999999999994	20.1759882442527\\
9.78199999999994	20.1719536372904\\
9.78399999999994	20.1679198374599\\
9.78599999999994	20.1638868445992\\
9.78799999999994	20.159854658546\\
9.78999999999994	20.1558232791386\\
9.79199999999994	20.1517927062146\\
9.79399999999994	20.1477629396125\\
9.79599999999994	20.14373397917\\
9.79799999999994	20.1397058247255\\
9.79999999999994	20.1356784761168\\
9.80199999999994	20.1316519331825\\
9.80399999999994	20.1276261957605\\
9.80599999999994	20.123601263689\\
9.80799999999994	20.1195771368067\\
9.80999999999995	20.1155538149514\\
9.81199999999995	20.1115312979617\\
9.81399999999995	20.107509585676\\
9.81599999999995	20.1034886779326\\
9.81799999999995	20.0994685745702\\
9.81999999999995	20.0954492754268\\
9.82199999999995	20.0914307803414\\
9.82399999999995	20.0874130891524\\
9.82599999999995	20.0833962016981\\
9.82799999999995	20.0793801178175\\
9.82999999999995	20.075364837349\\
9.83199999999995	20.0713503601314\\
9.83399999999995	20.0673366860035\\
9.83599999999995	20.0633238148039\\
9.83799999999995	20.0593117463714\\
9.83999999999996	20.0553004805449\\
9.84199999999996	20.051290017163\\
9.84399999999996	20.0472803560648\\
9.84599999999996	20.0432714970893\\
9.84799999999996	20.0392634400753\\
9.84999999999996	20.0352561848619\\
9.85199999999996	20.0312497312881\\
9.85399999999996	20.0272440791927\\
9.85599999999996	20.0232392284151\\
9.85799999999996	20.0192351787942\\
9.85999999999996	20.0152319301694\\
9.86199999999996	20.0112294823795\\
9.86399999999996	20.0072278352641\\
9.86599999999996	20.0032269886623\\
9.86799999999996	19.9992269424133\\
9.86999999999997	19.9952276963566\\
9.87199999999997	19.9912292503314\\
9.87399999999997	19.9872316041773\\
9.87599999999997	19.9832347577334\\
9.87799999999997	19.9792387108393\\
9.87999999999997	19.9752434633345\\
9.88199999999997	19.9712490150586\\
9.88399999999997	19.9672553658511\\
9.88599999999997	19.9632625155515\\
9.88799999999997	19.9592704639994\\
9.88999999999997	19.9552792110346\\
9.89199999999997	19.9512887564968\\
9.89399999999997	19.9472991002256\\
9.89599999999997	19.9433102420608\\
9.89799999999997	19.9393221818421\\
9.89999999999998	19.9353349194094\\
9.90199999999998	19.9313484546026\\
9.90399999999998	19.9273627872615\\
9.90599999999998	19.9233779172262\\
9.90799999999998	19.9193938443363\\
9.90999999999998	19.9154105684322\\
9.91199999999998	19.9114280893536\\
9.91399999999998	19.9074464069405\\
9.91599999999998	19.9034655210334\\
9.91799999999998	19.8994854314719\\
9.91999999999998	19.8955061380967\\
9.92199999999998	19.8915276407476\\
9.92399999999998	19.8875499392648\\
9.92599999999998	19.8835730334889\\
9.92799999999998	19.8795969232597\\
9.92999999999999	19.8756216084179\\
9.93199999999999	19.8716470888038\\
9.93399999999999	19.8676733642575\\
9.93599999999999	19.8637004346199\\
9.93799999999999	19.859728299731\\
9.93999999999999	19.8557569594316\\
9.94199999999999	19.8517864135617\\
9.94399999999999	19.8478166619627\\
9.94599999999999	19.8438477044745\\
9.94799999999999	19.8398795409379\\
9.94999999999999	19.8359121711935\\
9.95199999999999	19.8319455950822\\
9.95399999999999	19.8279798124445\\
9.95599999999999	19.8240148231213\\
9.95799999999999	19.8200506269532\\
9.96	19.8160872237812\\
9.962	19.8121246134459\\
9.964	19.8081627957885\\
9.966	19.8042017706497\\
9.968	19.8002415378704\\
9.97	19.7962820972918\\
9.972	19.7923234487548\\
9.974	19.7883655921003\\
9.976	19.7844085271697\\
9.978	19.7804522538038\\
9.98	19.7764967718438\\
9.982	19.772542081131\\
9.984	19.7685881815065\\
9.986	19.7646350728116\\
9.988	19.7606827548878\\
9.99000000000001	19.7567312275758\\
9.99200000000001	19.7527804907176\\
9.99400000000001	19.7488305441541\\
9.99600000000001	19.7448813877271\\
9.99800000000001	19.7409330212778\\
10	19.7369854446475\\
};
\end{axis}

\begin{axis}[%
width=2.603in,
height=1.074in,
at={(1.011in,2.499in)},
scale only axis,
xmin=0,
xmax=10,
xlabel style={font=\color{white!15!black}},
xlabel={t},
ymode=log,
ymin=25.1138353313008,
ymax=124207.41312983,
yminorticks=true,
ylabel style={font=\color{white!15!black}},
ylabel={indice stiff},
axis background/.style={fill=white},
title style={font=\bfseries},
title={N=30}
]
\addplot [color=mycolor1, forget plot]
  table[row sep=crcr]{%
0	583.571199705396\\
0.002	124207.41312983\\
0.004	61922.0531264968\\
0.006	41160.9764373461\\
0.008	30780.9606297128\\
0.01	24553.3613145156\\
0.012	20401.9637761634\\
0.014	17436.9617769385\\
0.016	15213.4522580588\\
0.018	13484.266906721\\
0.02	12101.1047829455\\
0.022	10969.59260449\\
0.024	10026.8148893385\\
0.026	9229.21480165202\\
0.028	8545.6803494588\\
0.03	7953.39610328392\\
0.032	7435.25052028186\\
0.034	6978.1583304602\\
0.036	6571.94212748115\\
0.038	6208.56714133065\\
0.04	5881.60557918034\\
0.042	5585.85401172365\\
0.044	5317.05510869108\\
0.046	5071.69196510272\\
0.048	4846.83384595553\\
0.05	4640.01895218895\\
0.052	4449.16424065653\\
0.054	4272.4952841036\\
0.056	4108.49116114899\\
0.058	3955.84074833859\\
0.06	3813.40775378967\\
0.062	3680.20251851263\\
0.064	3555.35910498565\\
0.066	3438.1165514425\\
0.068	3327.80343422411\\
0.07	3223.82507658454\\
0.0720000000000001	3125.65288936145\\
0.0740000000000001	3032.81544018311\\
0.0760000000000001	2944.89093280038\\
0.0780000000000001	2861.50084344527\\
0.0800000000000001	2782.30451172927\\
0.0820000000000001	2706.99452312192\\
0.0840000000000001	2635.29275106722\\
0.0860000000000001	2566.94695136896\\
0.0880000000000001	2501.7278209699\\
0.0900000000000001	2439.42644889598\\
0.0920000000000001	2379.8520996824\\
0.0940000000000001	2322.83027976766\\
0.0960000000000001	2268.20104558423\\
0.0980000000000001	2215.81751882191\\
0.1	2165.54457985756\\
0.102	2117.25771489851\\
0.104	2070.84199614399\\
0.106	2026.19117740193\\
0.108	1983.20689018938\\
0.11	1941.79792752965\\
0.112	1901.87960448538\\
0.114	1863.37318600157\\
0.116	1826.20537393707\\
0.118	1790.3078462601\\
0.12	1755.61684232331\\
0.122	1722.07278893131\\
0.124	1689.61996259282\\
0.126	1658.20618393933\\
0.128	1627.78254078941\\
0.13	1598.3031367775\\
0.132	1569.7248628322\\
0.134	1542.00718911839\\
0.136	1515.11197533674\\
0.138	1489.00329751882\\
0.14	1463.64728966579\\
0.142	1439.01199877055\\
0.144	1415.06725192044\\
0.146	1391.78453432113\\
0.148	1369.13687721105\\
0.15	1347.09875474187\\
0.152	1325.64598899856\\
0.154	1304.75566242068\\
0.156	1284.4060369602\\
0.158	1264.57647937759\\
0.16	1245.24739214123\\
0.162	1226.40014944507\\
0.164	1208.01703790613\\
0.166	1190.08120155027\\
0.168	1172.57659072629\\
0.17	1155.48791462555\\
0.172	1138.80059711451\\
0.174	1122.50073561281\\
0.176	1106.57506277458\\
0.178	1091.01091075299\\
0.18	1075.79617784728\\
0.182	1060.91929734818\\
0.184	1046.36920841557\\
0.186	1032.13532883415\\
0.188	1018.20752950852\\
0.19	1004.57611056951\\
0.192	991.231778972044\\
0.194	978.165627481102\\
0.196	965.369114942031\\
0.198	952.834047747728\\
0.2	940.552562417822\\
0.202	928.517109212305\\
0.204	916.720436709445\\
0.206	905.155577281857\\
0.208	893.815833411108\\
0.21	882.694764783664\\
0.212	871.786176118165\\
0.214	861.084105674553\\
0.216	850.582814402647\\
0.218	840.276775687351\\
0.22	830.160665653742\\
0.222	820.229353996713\\
0.224	810.47789530121\\
0.226	800.901520824277\\
0.228	791.495630709614\\
0.23	782.255786608236\\
0.232	773.177704681625\\
0.234	764.257248963179\\
0.236	755.490425058202\\
0.238	746.873374161144\\
0.24	738.402367372585\\
0.242	730.073800298249\\
0.244	721.884187913647\\
0.246	713.830159679805\\
0.248	705.908454895341\\
0.25	698.115918272052\\
0.252	690.449495721562\\
0.254	682.906230340785\\
0.256	675.483258586676\\
0.258	668.177806628366\\
0.26	660.987186868634\\
0.262	653.908794624436\\
0.264	646.940104958937\\
0.266	640.078669656663\\
0.268	633.322114334453\\
0.27	626.668135680926\\
0.272	620.114498818244\\
0.274	613.659034779664\\
0.276	607.29963809689\\
0.278	601.034264492195\\
0.28	594.860928669422\\
0.282	588.777702199488\\
0.284	582.782711495515\\
0.286	576.874135873078\\
0.288	571.050205691786\\
0.29	565.309200573933\\
0.292	559.64944769669\\
0.294	554.069320154244\\
0.296	548.567235386621\\
0.298	543.141653672419\\
0.3	537.791076681433\\
0.302	532.514046085833\\
0.304	527.309142225924\\
0.306	522.174982828753\\
0.308	517.110221776887\\
0.31	512.113547925151\\
0.312	507.183683963071\\
0.314	502.319385321202\\
0.316	497.519439119159\\
0.318	492.782663153704\\
0.32	488.107904924924\\
0.322	483.494040699182\\
0.324	478.93997460682\\
0.326	474.444637773465\\
0.328	470.006987483403\\
0.33	465.626006373513\\
0.332	461.300701656682\\
0.334	457.030104373402\\
0.336	452.813268670274\\
0.338	448.649271104392\\
0.34	444.537209972564\\
0.342	440.476204664289\\
0.344	436.465395037481\\
0.346	432.503940816244\\
0.348	428.591021009457\\
0.35	424.725833349629\\
0.352	420.90759375112\\
0.354	417.135535786881\\
0.356	413.408910183122\\
0.358	409.726984331085\\
0.36	406.089041815369\\
0.362	402.494381958019\\
0.364	398.942319377926\\
0.366	395.432183564825\\
0.368	391.963318467399\\
0.37	388.535082094998\\
0.372	385.14684613232\\
0.374	381.79799556672\\
0.376	378.487928327452\\
0.378	375.216054936844\\
0.38	371.981798172299\\
0.382	368.784592739442\\
0.384	365.623884955414\\
0.386	362.499132442285\\
0.388	359.409803830132\\
0.39	356.355378469391\\
0.392	353.335346152171\\
0.394	350.349206842242\\
0.396	347.396470413359\\
0.398	344.476656395631\\
0.4	341.589293729623\\
0.402	338.733920528052\\
0.404	335.910083844508\\
0.406	333.117339449409\\
0.408	330.355251612516\\
0.41	327.62339289191\\
0.412	324.921343929491\\
0.414	322.248693252269\\
0.416	319.605037079685\\
0.418	316.989979136534\\
0.42	314.403130471377\\
0.422	311.844109280199\\
0.424	309.312540735275\\
0.426	306.80805681882\\
0.428	304.330296161615\\
0.43	301.878903886063\\
0.432	299.453531453844\\
0.434	297.053836517807\\
0.436	294.679482778145\\
0.438	292.330139842491\\
0.44	290.005483090046\\
0.442	287.705193539389\\
0.444	285.428957720061\\
0.446	283.176467547621\\
0.448	280.947420202193\\
0.45	278.741518010301\\
0.452	276.558468329959\\
0.454	274.397983438884\\
0.456	272.259780425688\\
0.458	270.143581084106\\
0.46	268.049111809986\\
0.462	265.976103500998\\
0.464	263.924291459114\\
0.466	261.89341529562\\
0.468	259.883218838627\\
0.47	257.893450043049\\
0.472	255.923860902925\\
0.474	253.974207365964\\
0.476	252.044249250471\\
0.478	250.133750164248\\
0.48	248.242477425692\\
0.482	246.370201986877\\
0.484	244.516698358627\\
0.486	242.681744537469\\
0.488	240.865121934471\\
0.49	239.06661530582\\
0.492	237.286012685274\\
0.494	235.52310531811\\
0.496	233.777687596928\\
0.498	232.049556998887\\
0.5	230.338514024604\\
0.502	228.64436213851\\
0.504	226.966907710673\\
0.506	225.305959960052\\
0.508	223.661330899148\\
0.51	222.032835279977\\
0.512	220.42029054136\\
0.514	218.823516757482\\
0.516	217.242336587684\\
0.518	215.676575227453\\
0.52	214.126060360594\\
0.522	212.590622112488\\
0.524	211.070093004549\\
0.526	209.564307909626\\
0.528	208.073104008576\\
0.53	206.596320747739\\
0.532	205.133799797502\\
0.534	203.685385011737\\
0.536	202.250922388232\\
0.538	200.830260030013\\
0.54	199.423248107535\\
0.542	198.029738821776\\
0.544	196.649586368102\\
0.546	195.282646901038\\
0.548	193.92877849975\\
0.55	192.587841134359\\
0.552	191.259696632963\\
0.554	189.944208649448\\
0.556	188.641242631947\\
0.558	187.350665792064\\
0.56	186.072347074707\\
0.562	184.806157128656\\
0.564	183.551968277694\\
0.566	182.309654492442\\
0.568	181.079091362746\\
0.57	179.8601560707\\
0.572	178.652727364216\\
0.574	177.456685531191\\
0.576	176.271912374194\\
0.578	175.098291185729\\
0.58	173.93570672398\\
0.582	172.784045189093\\
0.584	171.643194199966\\
0.586	170.513042771473\\
0.588	169.393481292243\\
0.59	168.284401502822\\
0.592	167.18569647434\\
0.594	166.097260587582\\
0.596	165.018989512518\\
0.598	163.950780188248\\
0.6	162.892530803304\\
0.602	161.844140776448\\
0.604	160.805510737759\\
0.606	159.776542510169\\
0.608	158.75713909134\\
0.61	157.747204635906\\
0.612	156.746644438071\\
0.614	155.755364914553\\
0.616	154.773273587871\\
0.618	153.800279069942\\
0.62	152.83629104601\\
0.622	151.881220258917\\
0.624	150.934978493624\\
0.626	149.997478562061\\
0.628	149.068634288303\\
0.63	148.148360493969\\
0.632	147.236572983939\\
0.634	146.333188532347\\
0.636	145.438124868811\\
0.638	144.551300664953\\
0.64	143.672635521175\\
0.642	142.802049953657\\
0.644	141.939465381621\\
0.646	141.084804114838\\
0.648	140.237989341338\\
0.65	139.398945115388\\
0.652	138.567596345663\\
0.654	137.743868783655\\
0.656	136.927689012275\\
0.658	136.118984434685\\
0.66	135.317683263321\\
0.662	134.523714509117\\
0.664	133.737007970948\\
0.666	132.957494225203\\
0.668	132.185104615642\\
0.67	131.419771243339\\
0.672	130.661426956876\\
0.674	129.910005342665\\
0.676	129.165440715486\\
0.678	128.427668109148\\
0.68	127.69662326737\\
0.682	126.972242634763\\
0.684	126.254463348026\\
0.686	125.543223227262\\
0.688000000000001	124.838460767472\\
0.690000000000001	124.140115130169\\
0.692000000000001	123.448126135171\\
0.694000000000001	122.762434252529\\
0.696000000000001	122.082980594578\\
0.698000000000001	121.40970690814\\
0.700000000000001	120.742555566886\\
0.702000000000001	120.081469563782\\
0.704000000000001	119.426392503713\\
0.706000000000001	118.777268596211\\
0.708000000000001	118.13404264831\\
0.710000000000001	117.496660057545\\
0.712000000000001	116.865066805043\\
0.714000000000001	116.239209448757\\
0.716000000000001	115.619035116816\\
0.718000000000001	115.004491500973\\
0.720000000000001	114.395526850199\\
0.722000000000001	113.792089964338\\
0.724000000000001	113.194130187942\\
0.726000000000001	112.601597404146\\
0.728000000000001	112.014442028698\\
0.730000000000001	111.432615004057\\
0.732000000000001	110.85606779364\\
0.734000000000001	110.284752376121\\
0.736000000000001	109.718621239866\\
0.738000000000001	109.157627377438\\
0.740000000000001	108.60172428024\\
0.742000000000001	108.050865933214\\
0.744000000000001	107.50500680965\\
0.746000000000001	106.964101866089\\
0.748000000000001	106.428106537336\\
0.750000000000001	105.896976731512\\
0.752000000000001	105.370668825267\\
0.754000000000001	104.84913965902\\
0.756000000000001	104.33234653232\\
0.758000000000001	103.820247199287\\
0.760000000000001	103.312799864138\\
0.762000000000001	102.80996317679\\
0.764000000000001	102.311696228573\\
0.766000000000001	101.817958547991\\
0.768000000000001	101.328710096584\\
0.770000000000001	100.843911264882\\
0.772000000000001	100.363522868415\\
0.774000000000001	99.8875061438371\\
0.776000000000001	99.4158227450861\\
0.778000000000001	98.9484347396643\\
0.780000000000001	98.4853046049796\\
0.782000000000001	98.0263952247574\\
0.784000000000001	97.5716698855476\\
0.786000000000001	97.1210922733041\\
0.788000000000001	96.6746264700251\\
0.790000000000001	96.2322369504994\\
0.792000000000001	95.7938885791011\\
0.794000000000001	95.3595466066867\\
0.796000000000001	94.9291766675402\\
0.798000000000001	94.5027447764194\\
0.800000000000001	94.0802173256655\\
0.802000000000001	93.6615610823936\\
0.804000000000001	93.2467431857478\\
0.806000000000001	92.83573114426\\
0.808000000000001	92.4284928332367\\
0.810000000000001	92.0249964922875\\
0.812000000000001	91.625210722864\\
0.814000000000001	91.2291044859167\\
0.816000000000001	90.8366470996224\\
0.818000000000001	90.447808237163\\
0.820000000000001	90.0625579246291\\
0.822000000000001	89.6808665389508\\
0.824000000000001	89.3027048059374\\
0.826000000000001	88.9280437983888\\
0.828000000000001	88.556854934278\\
0.830000000000001	88.1891099750228\\
0.832000000000001	87.8247810238306\\
0.834000000000001	87.4638405241299\\
0.836000000000001	87.106261258067\\
0.838000000000001	86.7520163451052\\
0.840000000000001	86.4010792406927\\
0.842000000000001	86.0534237350082\\
0.844000000000001	85.709023951811\\
0.846000000000001	85.3678543473445\\
0.848000000000001	85.0298897093431\\
0.850000000000001	84.6951051561293\\
0.852000000000001	84.3634761357692\\
0.854000000000001	84.0349784253485\\
0.856000000000001	83.7095881302983\\
0.858000000000001	83.3872816838436\\
0.860000000000001	83.0680358465063\\
0.862000000000001	82.7518277057151\\
0.864000000000001	82.4386346755\\
0.866000000000001	82.12843449627\\
0.868000000000001	81.8212052346795\\
0.870000000000001	81.5169252835818\\
0.872000000000001	81.2155733620787\\
0.874000000000001	80.9171285156458\\
0.876000000000001	80.6215701163463\\
0.878000000000001	80.3288778631381\\
0.880000000000001	80.0390317822601\\
0.882000000000001	79.7520122276955\\
0.884000000000001	79.4677998817354\\
0.886000000000001	79.1863757556109\\
0.888000000000001	78.9077211901904\\
0.890000000000001	78.6318178567836\\
0.892000000000001	78.3586477579867\\
0.894000000000001	78.0881932286194\\
0.896000000000001	77.8204369367024\\
0.898000000000001	77.555361884523\\
0.900000000000001	77.2929514097336\\
0.902000000000001	77.0331891864992\\
0.904000000000001	76.7760592267\\
0.906000000000001	76.5215458811428\\
0.908000000000001	76.2696338408305\\
0.910000000000001	76.0203081382088\\
0.912000000000001	75.7735541484346\\
0.914000000000001	75.5293575906295\\
0.916000000000001	75.2877045291208\\
0.918000000000001	75.0485813746147\\
0.920000000000001	74.8119748853395\\
0.922000000000001	74.5778721681045\\
0.924000000000001	74.346260679253\\
0.926000000000001	74.1171282255063\\
0.928000000000001	73.8904629646541\\
0.930000000000001	73.6662534060895\\
0.932000000000001	73.4444884111114\\
0.934000000000001	73.2251571930131\\
0.936000000000001	73.0082493169012\\
0.938000000000001	72.7937546991882\\
0.940000000000001	72.5816636067402\\
0.942000000000001	72.3719666556416\\
0.944000000000001	72.1646548094755\\
0.946000000000001	71.9597193771614\\
0.948000000000001	71.7571520101671\\
0.950000000000001	71.5569446991794\\
0.952000000000001	71.3590897700295\\
0.954000000000001	71.1635798789219\\
0.956000000000001	70.9704080068087\\
0.958000000000001	70.7795674528831\\
0.960000000000001	70.5910518271056\\
0.962000000000001	70.4048550416443\\
0.964000000000001	70.220971301208\\
0.966000000000001	70.0393950921204\\
0.968000000000001	69.860121170059\\
0.970000000000001	69.6831445464112\\
0.972000000000001	69.5084604730755\\
0.974000000000001	69.3360644256876\\
0.976000000000001	69.165952085113\\
0.978000000000001	68.9981193171636\\
0.980000000000001	68.8325621504252\\
0.982000000000001	68.6692767521191\\
0.984000000000001	68.5082594019092\\
0.986000000000001	68.3495064635995\\
0.988000000000001	68.1930143546808\\
0.990000000000001	68.0387795136354\\
0.992000000000001	67.8867983650142\\
0.994000000000001	67.7370672822671\\
0.996000000000001	67.5895825483281\\
0.998000000000001	67.4443403140001\\
1	67.301336554204\\
1.002	67.1605670221832\\
1.004	67.0220272017907\\
1.006	66.8857122580152\\
1.008	66.7516169859623\\
1.01	66.619735758498\\
1.012	66.4900624728827\\
1.014	66.3625904966486\\
1.016	66.2373126131537\\
1.018	66.1142209671474\\
1.02	65.9933070108409\\
1.022	65.8745614509129\\
1.024	65.7579741969689\\
1.026	65.6435343119999\\
1.028	65.5312299653479\\
1.03	65.4210483887609\\
1.032	65.3129758361058\\
1.034	65.2069975472458\\
1.036	65.1030977166581\\
1.038	65.0012594672767\\
1.04	64.9014648300192\\
1.042	64.8036947294252\\
1.044	64.707928975737\\
1.046	64.6141462637396\\
1.048	64.5223241785253\\
1.05	64.4324392083333\\
1.052	64.3444667644818\\
1.054	64.2583812083167\\
1.056	64.1741558850529\\
1.058	64.0917631642249\\
1.06	64.0111744864185\\
1.062	63.9323604158781\\
1.064	63.8552906984503\\
1.066	63.779934324358\\
1.068	63.7062595951302\\
1.07	63.6342341940254\\
1.072	63.5638252593345\\
1.074	63.4949994597305\\
1.076	63.427723071021\\
1.078	63.3619620536148\\
1.08	63.2976821299306\\
1.082	63.2348488611714\\
1.084	63.1734277228053\\
1.086	63.1133841781667\\
1.088	63.0546837497126\\
1.09	62.9972920874291\\
1.092	62.9411750340007\\
1.094	62.8862986864435\\
1.096	62.8326294538741\\
1.098	62.7801341112753\\
1.1	62.7287798490594\\
1.102	62.6785343183652\\
1.104	62.6293656720657\\
1.106	62.5812426014597\\
1.108	62.5341343687462\\
1.11	62.4880108353556\\
1.112	62.4428424862502\\
1.114	62.3986004504046\\
1.116	62.3552565175606\\
1.118	62.3127831515169\\
1.12	62.2711535001516\\
1.122	62.2303414023559\\
1.124	62.190321392154\\
1.126	62.1510687001847\\
1.128	62.1125592528024\\
1.13	62.0747696690069\\
1.132	62.0376772553972\\
1.134	62.0012599993776\\
1.136	61.965496560815\\
1.138	61.9303662623129\\
1.14	61.8958490782915\\
1.142	61.8619256230512\\
1.144	61.8285771379628\\
1.146	61.7957854779168\\
1.148	61.7635330971851\\
1.15	61.7318030348187\\
1.152	61.700578899656\\
1.154	61.6698448550928\\
1.156	61.6395856036562\\
1.158	61.6097863715092\\
1.16	61.5804328929147\\
1.162	61.5515113947529\\
1.164	61.5230085811473\\
1.166	61.4949116182158\\
1.168	61.467208119043\\
1.17	61.4398861288626\\
1.172	61.4129341105052\\
1.174	61.3863409301398\\
1.176	61.3600958433083\\
1.178	61.3341884812962\\
1.18	61.3086088378494\\
1.182	61.2833472562251\\
1.184	61.2583944166098\\
1.186	61.2337413239035\\
1.188	61.2093792958638\\
1.19	61.1852999516225\\
1.192	61.1614952005643\\
1.194	61.1379572315628\\
1.196	61.1146785025993\\
1.198	61.0916517307069\\
1.2	61.0688698822873\\
1.202	61.0463261637618\\
1.204	61.0240140125531\\
1.206	61.0019270884028\\
1.208	60.9800592650056\\
1.21	60.9584046219609\\
1.212	60.9369574370092\\
1.214	60.9157121785836\\
1.216	60.894663498635\\
1.218	60.8738062257351\\
1.22	60.8531353584427\\
1.222	60.8326460589373\\
1.224	60.812333646888\\
1.226	60.7921935935768\\
1.228	60.7722215162391\\
1.23	60.7524131726439\\
1.232	60.7327644558706\\
1.234	60.7132713893088\\
1.236	60.6939301218419\\
1.238	60.6747369232376\\
1.24	60.6556881797102\\
1.242	60.6367803896613\\
1.244	60.6180101595978\\
1.246	60.5993742001964\\
1.248	60.5808693225514\\
1.25	60.5624924345351\\
1.252	60.5442405373441\\
1.254	60.5261107221429\\
1.256	60.508100166872\\
1.258	60.4902061331624\\
1.26	60.4724259633821\\
1.262	60.4547570778006\\
1.264	60.4371969718483\\
1.266	60.4197432135091\\
1.268	60.4023934407945\\
1.27	60.3851453593254\\
1.272	60.3679967400092\\
1.274	60.350945416798\\
1.276	60.3339892845468\\
1.278	60.3171262969453\\
1.28	60.3003544645336\\
1.282	60.2836718527887\\
1.284	60.2670765802927\\
1.286	60.2505668169619\\
1.288	60.234140782352\\
1.29	60.2177967440198\\
1.292	60.201533015954\\
1.294	60.1853479570537\\
1.296	60.1692399696813\\
1.298	60.1532074982532\\
1.3	60.1372490278887\\
1.302	60.1213630831108\\
1.304	60.1055482265954\\
1.306	60.0898030579656\\
1.308	60.0741262126241\\
1.31	60.0585163606372\\
1.312	60.0429722056551\\
1.314	60.0274924838695\\
1.316	60.0120759630161\\
1.318	59.9967214413981\\
1.32	59.9814277469676\\
1.322	59.9661937364143\\
1.324	59.9510182943022\\
1.326	59.935900332237\\
1.328	59.9208387880485\\
1.33	59.9058326250279\\
1.332	59.8908808311552\\
1.334	59.875982418391\\
1.336	59.8611364219679\\
1.338	59.8463418997123\\
1.34	59.831597931393\\
1.342	59.8169036180905\\
1.344	59.8022580815836\\
1.346	59.7876604637646\\
1.348	59.7731099260645\\
1.35	59.7586056489074\\
1.352	59.7441468311746\\
1.354	59.7297326896868\\
1.356	59.7153624587191\\
1.358	59.7010353895027\\
1.36	59.6867507497707\\
1.362	59.6725078233044\\
1.364	59.6583059094981\\
1.366	59.6441443229329\\
1.368	59.6300223929755\\
1.37	59.6159394633768\\
1.372	59.6018948918898\\
1.374	59.5878880499078\\
1.376	59.5739183220909\\
1.378	59.5599851060282\\
1.38	59.5460878118981\\
1.382	59.5322258621428\\
1.384	59.5183986911502\\
1.386	59.5046057449511\\
1.388	59.4908464809166\\
1.39	59.4771203674738\\
1.392	59.4634268838277\\
1.394	59.4497655196846\\
1.396	59.4361357749964\\
1.398	59.4225371596975\\
1.4	59.4089691934676\\
1.402	59.3954314054869\\
1.404	59.3819233341943\\
1.406	59.3684445270774\\
1.408	59.3549945404465\\
1.41	59.3415729392152\\
1.412	59.3281792967059\\
1.414	59.3148131944331\\
1.416	59.301474221929\\
1.418	59.2881619765313\\
1.42	59.2748760632176\\
1.422	59.2616160944167\\
1.424	59.2483816898381\\
1.426	59.2351724763031\\
1.428	59.2219880875775\\
1.43	59.2088281642178\\
1.432	59.1956923534096\\
1.434	59.1825803088235\\
1.436	59.1694916904571\\
1.438	59.1564261645113\\
1.44	59.1433834032296\\
1.442	59.1303630847798\\
1.444	59.1173648931115\\
1.446	59.1043885178397\\
1.448	59.0914336541049\\
1.45	59.0785000024688\\
1.452	59.0655872687813\\
1.454	59.0526951640783\\
1.456	59.0398234044605\\
1.458	59.0269717109904\\
1.46	59.0141398095819\\
1.462	59.0013274309024\\
1.464	58.9885343102659\\
1.466	58.9757601875406\\
1.468	58.9630048070494\\
1.47	58.9502679174775\\
1.472	58.937549271786\\
1.474	58.9248486271167\\
1.476	58.9121657447107\\
1.478	58.899500389823\\
1.48	58.8868523316381\\
1.482	58.8742213431978\\
1.484	58.8616072013131\\
1.486	58.8490096864949\\
1.488	58.8364285828786\\
1.49	58.8238636781504\\
1.492	58.8113147634772\\
1.494	58.7987816334401\\
1.496	58.7862640859623\\
1.498	58.7737619222481\\
1.5	58.7612749467199\\
1.502	58.748802966946\\
1.504	58.7363457935958\\
1.506	58.7239032403632\\
1.508	58.71147512392\\
1.51	58.6990612638565\\
1.512	58.6866614826208\\
1.514	58.6742756054724\\
1.516	58.6619034604251\\
1.518	58.6495448781942\\
1.52	58.6371996921485\\
1.522	58.6248677382595\\
1.524	58.6125488550549\\
1.526	58.6002428835678\\
1.528	58.5879496672949\\
1.53	58.5756690521473\\
1.532	58.5634008864121\\
1.534	58.5511450207017\\
1.536	58.5389013079181\\
1.538	58.5266696032087\\
1.54	58.5144497639261\\
1.542	58.5022416495905\\
1.544	58.4900451218482\\
1.546	58.4778600444353\\
1.548	58.4656862831436\\
1.55	58.4535237057783\\
1.552	58.4413721821314\\
1.554	58.4292315839349\\
1.556	58.4171017848374\\
1.558	58.4049826603687\\
1.56	58.3928740879069\\
1.562	58.3807759466416\\
1.564	58.368688117551\\
1.566	58.3566104833672\\
1.568	58.3445429285437\\
1.57	58.3324853392314\\
1.572	58.3204376032501\\
1.574	58.3083996100536\\
1.576	58.2963712507107\\
1.578	58.2843524178726\\
1.58	58.2723430057485\\
1.582	58.2603429100804\\
1.584	58.2483520281171\\
1.586	58.2363702585892\\
1.588	58.2243975016861\\
1.59	58.2124336590276\\
1.592	58.2004786336498\\
1.594	58.1885323299693\\
1.596	58.1765946537761\\
1.598	58.1646655121982\\
1.6	58.1527448136858\\
1.602	58.1408324679903\\
1.604	58.1289283861435\\
1.606	58.117032480437\\
1.608	58.1051446643983\\
1.61	58.0932648527803\\
1.612	58.0813929615315\\
1.614	58.0695289077877\\
1.616	58.0576726098417\\
1.618	58.0458239871358\\
1.62	58.033982960239\\
1.622	58.0221494508305\\
1.624	58.0103233816798\\
1.626	57.9985046766361\\
1.628	57.9866932606046\\
1.63	57.9748890595348\\
1.632	57.9630920004042\\
1.634	57.9513020112045\\
1.636	57.9395190209172\\
1.638	57.92774295951\\
1.64	57.9159737579156\\
1.642	57.904211348018\\
1.644	57.8924556626405\\
1.646	57.880706635526\\
1.648	57.8689642013345\\
1.65	57.8572282956134\\
1.652	57.8454988547988\\
1.654	57.8337758161961\\
1.656	57.8220591179653\\
1.658	57.810348699113\\
1.66	57.798644499476\\
1.662	57.7869464597144\\
1.664	57.7752545212911\\
1.666	57.7635686264693\\
1.668	57.7518887182948\\
1.67	57.740214740587\\
1.672	57.7285466379276\\
1.674	57.7168843556476\\
1.676	57.7052278398211\\
1.678	57.6935770372485\\
1.68	57.6819318954524\\
1.682	57.6702923626613\\
1.684	57.6586583878038\\
1.686	57.6470299204984\\
1.688	57.6354069110415\\
1.69	57.6237893103976\\
1.692	57.6121770701949\\
1.694	57.6005701427064\\
1.696	57.5889684808529\\
1.698	57.577372038183\\
1.7	57.5657807688709\\
1.702	57.5541946277074\\
1.704	57.5426135700854\\
1.706	57.5310375519998\\
1.708	57.5194665300323\\
1.71	57.5079004613469\\
1.712	57.496339303683\\
1.714	57.4847830153421\\
1.716	57.4732315551846\\
1.718	57.461684882622\\
1.72	57.4501429576079\\
1.722	57.4386057406304\\
1.724	57.4270731927037\\
1.726	57.4155452753666\\
1.728	57.4040219506676\\
1.73	57.3925031811643\\
1.732	57.3809889299109\\
1.734	57.3694791604559\\
1.736	57.3579738368346\\
1.738	57.3464729235622\\
1.74	57.3349763856249\\
1.742	57.3234841884763\\
1.744	57.311996298031\\
1.746	57.3005126806579\\
1.748	57.2890333031738\\
1.75	57.2775581328371\\
1.752	57.2660871373429\\
1.754	57.2546202848174\\
1.756	57.2431575438095\\
1.758	57.2316988832907\\
1.76	57.2202442726412\\
1.762	57.2087936816528\\
1.764	57.1973470805195\\
1.766	57.1859044398295\\
1.768	57.1744657305671\\
1.77	57.1630309241012\\
1.772	57.1515999921817\\
1.774	57.1401729069376\\
1.776	57.1287496408665\\
1.778	57.1173301668363\\
1.78	57.1059144580736\\
1.782	57.0945024881648\\
1.784	57.0830942310482\\
1.786	57.07168966101\\
1.788	57.0602887526785\\
1.79	57.0488914810224\\
1.792	57.0374978213452\\
1.794	57.0261077492801\\
1.796	57.0147212407857\\
1.798	57.003338272144\\
1.8	56.9919588199508\\
1.802	56.9805828611203\\
1.804	56.9692103728717\\
1.806	56.9578413327322\\
1.808	56.9464757185287\\
1.81	56.9351135083858\\
1.812	56.923754680723\\
1.814	56.9123992142478\\
1.816	56.9010470879551\\
1.818	56.8896982811226\\
1.82	56.8783527733051\\
1.822	56.8670105443313\\
1.824	56.8556715743063\\
1.826	56.8443358436015\\
1.828	56.83300333285\\
1.83	56.8216740229491\\
1.832	56.8103478950536\\
1.834	56.7990249305702\\
1.836	56.7877051111637\\
1.838	56.7763884187409\\
1.84	56.7650748354536\\
1.842	56.7537643437005\\
1.844	56.7424569261127\\
1.846	56.7311525655638\\
1.848	56.719851245153\\
1.85	56.708552948216\\
1.852	56.6972576583104\\
1.854	56.6859653592183\\
1.856	56.6746760349462\\
1.858	56.663389669715\\
1.86	56.6521062479641\\
1.862	56.6408257543426\\
1.864	56.6295481737123\\
1.866	56.6182734911404\\
1.868	56.6070016919005\\
1.87	56.595732761467\\
1.872	56.5844666855143\\
1.874	56.5732034499131\\
1.876	56.5619430407301\\
1.878	56.5506854442221\\
1.88	56.539430646839\\
1.882	56.5281786352118\\
1.884	56.5169293961627\\
1.886	56.5056829166928\\
1.888	56.4944391839831\\
1.89	56.4831981853943\\
1.892	56.4719599084617\\
1.894	56.4607243408941\\
1.896	56.4494914705705\\
1.898	56.4382612855412\\
1.9	56.4270337740204\\
1.902	56.4158089243897\\
1.904	56.4045867251918\\
1.906	56.3933671651303\\
1.908	56.3821502330673\\
1.91	56.3709359180211\\
1.912	56.3597242091664\\
1.914	56.3485150958271\\
1.916	56.3373085674804\\
1.918	56.3261046137522\\
1.92	56.3149032244125\\
1.922	56.3037043893789\\
1.924	56.2925080987121\\
1.926	56.2813143426128\\
1.928	56.2701231114231\\
1.93	56.2589343956189\\
1.932	56.2477481858172\\
1.934	56.2365644727655\\
1.936	56.2253832473454\\
1.938	56.2142045005694\\
1.94	56.2030282235778\\
1.942	56.191854407641\\
1.944	56.1806830441511\\
1.946	56.169514124631\\
1.948	56.1583476407197\\
1.95	56.147183584181\\
1.952	56.1360219468969\\
1.954	56.1248627208704\\
1.956	56.1137058982137\\
1.958	56.1025514711657\\
1.96	56.0913994320672\\
1.962	56.0802497733789\\
1.964	56.0691024876686\\
1.966	56.0579575676153\\
1.968	56.0468150060045\\
1.97	56.035674795728\\
1.972	56.0245369297854\\
1.974	56.0134014012775\\
1.976	56.0022682034066\\
1.978	55.9911373294804\\
1.98	55.9800087729036\\
1.982	55.9688825271787\\
1.984	55.9577585859076\\
1.986	55.9466369427859\\
1.988	55.9355175916079\\
1.99	55.9244005262571\\
1.992	55.9132857407122\\
1.994	55.9021732290418\\
1.996	55.8910629854045\\
1.998	55.8799550040488\\
2	55.8688492793105\\
2.002	55.8577458056115\\
2.004	55.8466445774596\\
2.006	55.8355455894459\\
2.008	55.8244488362459\\
2.01	55.8133543126185\\
2.012	55.8022620134001\\
2.014	55.7911719335086\\
2.016	55.7800840679446\\
2.018	55.7689984117809\\
2.02	55.7579149601697\\
2.022	55.7468337083406\\
2.024	55.7357546515964\\
2.026	55.7246777853144\\
2.028	55.7136031049446\\
2.03	55.7025306060093\\
2.032	55.6914602841014\\
2.034	55.6803921348841\\
2.036	55.6693261540912\\
2.038	55.6582623375232\\
2.04	55.6472006810483\\
2.042	55.636141180602\\
2.044	55.6250838321846\\
2.046	55.6140286318606\\
2.048	55.6029755757607\\
2.05	55.5919246600761\\
2.052	55.580875881062\\
2.054	55.5698292350337\\
2.056	55.5587847183704\\
2.05799999999999	55.5477423275057\\
2.05999999999999	55.5367020589363\\
2.06199999999999	55.5256639092141\\
2.06399999999999	55.5146278749527\\
2.06599999999999	55.5035939528184\\
2.06799999999999	55.4925621395353\\
2.06999999999999	55.4815324318801\\
2.07199999999999	55.4705048266882\\
2.07399999999999	55.4594793208464\\
2.07599999999999	55.4484559112928\\
2.07799999999999	55.4374345950214\\
2.07999999999999	55.4264153690735\\
2.08199999999999	55.4153982305437\\
2.08399999999999	55.4043831765793\\
2.08599999999999	55.3933702043695\\
2.08799999999999	55.3823593111609\\
2.08999999999999	55.3713504942435\\
2.09199999999999	55.3603437509545\\
2.09399999999999	55.3493390786784\\
2.09599999999999	55.3383364748475\\
2.09799999999999	55.3273359369403\\
2.09999999999999	55.3163374624761\\
2.10199999999999	55.30534104902\\
2.10399999999999	55.2943466941852\\
2.10599999999999	55.283354395621\\
2.10799999999999	55.2723641510249\\
2.10999999999999	55.2613759581334\\
2.11199999999999	55.2503898147242\\
2.11399999999999	55.2394057186188\\
2.11599999999999	55.2284236676737\\
2.11799999999999	55.2174436597899\\
2.11999999999999	55.2064656929052\\
2.12199999999999	55.1954897649956\\
2.12399999999999	55.1845158740761\\
2.12599999999999	55.1735440181995\\
2.12799999999999	55.1625741954536\\
2.12999999999999	55.1516064039644\\
2.13199999999999	55.1406406418923\\
2.13399999999999	55.1296769074365\\
2.13599999999999	55.1187151988251\\
2.13799999999999	55.1077555143264\\
2.13999999999999	55.0967978522392\\
2.14199999999999	55.0858422108968\\
2.14399999999999	55.0748885886654\\
2.14599999999999	55.0639369839449\\
2.14799999999999	55.052987395164\\
2.14999999999998	55.042039820786\\
2.15199999999998	55.0310942593037\\
2.15399999999998	55.0201507092409\\
2.15599999999998	55.009209169152\\
2.15799999999998	54.9982696376211\\
2.15999999999998	54.9873321132602\\
2.16199999999998	54.9763965947134\\
2.16399999999998	54.9654630806487\\
2.16599999999998	54.9545315697666\\
2.16799999999998	54.9436020607923\\
2.16999999999998	54.9326745524802\\
2.17199999999998	54.9217490436086\\
2.17399999999998	54.9108255329865\\
2.17599999999998	54.8999040194466\\
2.17799999999998	54.8889845018448\\
2.17999999999998	54.8780669790669\\
2.18199999999998	54.8671514500213\\
2.18399999999998	54.8562379136416\\
2.18599999999998	54.8453263688836\\
2.18799999999998	54.8344168147295\\
2.18999999999998	54.8235092501842\\
2.19199999999998	54.8126036742735\\
2.19399999999998	54.8017000860493\\
2.19599999999998	54.7907984845849\\
2.19799999999998	54.7798988689741\\
2.19999999999998	54.7690012383333\\
2.20199999999998	54.7581055918004\\
2.20399999999998	54.7472119285349\\
2.20599999999998	54.7363202477162\\
2.20799999999998	54.7254305485447\\
2.20999999999998	54.7145428302399\\
2.21199999999998	54.7036570920426\\
2.21399999999998	54.6927733332108\\
2.21599999999998	54.6818915530251\\
2.21799999999998	54.6710117507813\\
2.21999999999998	54.6601339257966\\
2.22199999999998	54.6492580774039\\
2.22399999999998	54.6383842049559\\
2.22599999999998	54.6275123078238\\
2.22799999999998	54.616642385393\\
2.22999999999998	54.6057744370705\\
2.23199999999998	54.5949084622747\\
2.23399999999998	54.5840444604457\\
2.23599999999998	54.5731824310374\\
2.23799999999998	54.5623223735197\\
2.23999999999997	54.5514642873786\\
2.24199999999997	54.5406081721164\\
2.24399999999997	54.5297540272494\\
2.24599999999997	54.5189018523097\\
2.24799999999997	54.5080516468439\\
2.24999999999997	54.4972034104152\\
2.25199999999997	54.4863571425964\\
2.25399999999997	54.4755128429787\\
2.25599999999997	54.4646705111648\\
2.25799999999997	54.4538301467714\\
2.25999999999997	54.442991749431\\
2.26199999999997	54.4321553187851\\
2.26399999999997	54.4213208544908\\
2.26599999999997	54.4104883562172\\
2.26799999999997	54.3996578236456\\
2.26999999999997	54.3888292564701\\
2.27199999999997	54.3780026543968\\
2.27399999999997	54.3671780171427\\
2.27599999999997	54.3563553444372\\
2.27799999999997	54.3455346360216\\
2.27999999999997	54.3347158916479\\
2.28199999999997	54.3238991110791\\
2.28399999999997	54.3130842940882\\
2.28599999999997	54.3022714404588\\
2.28799999999997	54.2914605499878\\
2.28999999999997	54.2806516224797\\
2.29199999999997	54.2698446577493\\
2.29399999999997	54.2590396556205\\
2.29599999999997	54.2482366159293\\
2.29799999999997	54.2374355385204\\
2.29999999999997	54.2266364232466\\
2.30199999999997	54.2158392699699\\
2.30399999999997	54.2050440785622\\
2.30599999999997	54.1942508489057\\
2.30799999999997	54.1834595808874\\
2.30999999999997	54.1726702744063\\
2.31199999999997	54.161882929367\\
2.31399999999997	54.1510975456854\\
2.31599999999997	54.140314123282\\
2.31799999999997	54.1295326620879\\
2.31999999999997	54.1187531620405\\
2.32199999999997	54.107975623084\\
2.32399999999997	54.097200045173\\
2.32599999999997	54.0864264282654\\
2.32799999999997	54.0756547723281\\
2.32999999999996	54.0648850773366\\
2.33199999999996	54.0541173432707\\
2.33399999999996	54.0433515701174\\
2.33599999999996	54.0325877578702\\
2.33799999999996	54.0218259065307\\
2.33999999999996	54.011066016104\\
2.34199999999996	54.0003080866044\\
2.34399999999996	53.9895521180493\\
2.34599999999996	53.9787981104621\\
2.34799999999996	53.9680460638743\\
2.34999999999996	53.9572959783227\\
2.35199999999996	53.946547853847\\
2.35399999999996	53.9358016904938\\
2.35599999999996	53.9250574883156\\
2.35799999999996	53.91431524737\\
2.35999999999996	53.9035749677182\\
2.36199999999996	53.8928366494262\\
2.36399999999996	53.8821002925681\\
2.36599999999996	53.8713658972194\\
2.36799999999996	53.8606334634585\\
2.36999999999996	53.8499029913728\\
2.37199999999996	53.839174481053\\
2.37399999999996	53.8284479325911\\
2.37599999999996	53.8177233460843\\
2.37799999999996	53.8070007216373\\
2.37999999999996	53.7962800593548\\
2.38199999999996	53.7855613593463\\
2.38399999999996	53.7748446217253\\
2.38599999999996	53.7641298466086\\
2.38799999999996	53.7534170341174\\
2.38999999999996	53.7427061843753\\
2.39199999999996	53.7319972975094\\
2.39399999999996	53.7212903736517\\
2.39599999999996	53.7105854129346\\
2.39799999999996	53.6998824154947\\
2.39999999999996	53.6891813814719\\
2.40199999999996	53.6784823110092\\
2.40399999999996	53.6677852042519\\
2.40599999999996	53.6570900613489\\
2.40799999999996	53.64639688245\\
2.40999999999996	53.6357056677079\\
2.41199999999996	53.6250164172804\\
2.41399999999996	53.6143291313248\\
2.41599999999996	53.6036438100006\\
2.41799999999996	53.5929604534723\\
2.41999999999996	53.5822790619025\\
2.42199999999995	53.5715996354606\\
2.42399999999995	53.5609221743142\\
2.42599999999995	53.550246678634\\
2.42799999999995	53.5395731485947\\
2.42999999999995	53.5289015843678\\
2.43199999999995	53.5182319861315\\
2.43399999999995	53.5075643540638\\
2.43599999999995	53.4968986883443\\
2.43799999999995	53.4862349891531\\
2.43999999999995	53.4755732566737\\
2.44199999999995	53.464913491088\\
2.44399999999995	53.4542556925853\\
2.44599999999995	53.4435998613489\\
2.44799999999995	53.4329459975661\\
2.44999999999995	53.4222941014279\\
2.45199999999995	53.4116441731223\\
2.45399999999995	53.4009962128414\\
2.45599999999995	53.3903502207771\\
2.45799999999995	53.3797061971222\\
2.45999999999995	53.3690641420694\\
2.46199999999995	53.3584240558145\\
2.46399999999995	53.3477859385521\\
2.46599999999995	53.3371497904783\\
2.46799999999995	53.3265156117898\\
2.46999999999995	53.3158834026829\\
2.47199999999995	53.3052531633562\\
2.47399999999995	53.2946248940074\\
2.47599999999995	53.2839985948358\\
2.47799999999995	53.2733742660392\\
2.47999999999995	53.2627519078187\\
2.48199999999995	53.2521315203728\\
2.48399999999995	53.2415131039013\\
2.48599999999995	53.2308966586043\\
2.48799999999995	53.2202821846825\\
2.48999999999995	53.2096696823365\\
2.49199999999995	53.1990591517661\\
2.49399999999995	53.1884505931727\\
2.49599999999995	53.1778440067555\\
2.49799999999995	53.1672393927162\\
2.49999999999995	53.1566367512546\\
2.50199999999995	53.1460360825701\\
2.50399999999995	53.1354373868646\\
2.50599999999995	53.1248406643358\\
2.50799999999995	53.1142459151848\\
2.50999999999995	53.103653139611\\
2.51199999999994	53.093062337813\\
2.51399999999994	53.0824735099885\\
2.51599999999994	53.0718866563377\\
2.51799999999994	53.0613017770564\\
2.51999999999994	53.0507188723429\\
2.52199999999994	53.040137942395\\
2.52399999999994	53.0295589874081\\
2.52599999999994	53.0189820075772\\
2.52799999999994	53.0084070030994\\
2.52999999999994	52.9978339741683\\
2.53199999999994	52.9872629209772\\
2.53399999999994	52.9766938437211\\
2.53599999999994	52.9661267425914\\
2.53799999999994	52.9555616177807\\
2.53999999999994	52.9449984694795\\
2.54199999999994	52.9344372978777\\
2.54399999999994	52.9238781031654\\
2.54599999999994	52.9133208855315\\
2.54799999999994	52.9027656451638\\
2.54999999999994	52.8922123822489\\
2.55199999999994	52.8816610969725\\
2.55399999999994	52.8711117895196\\
2.55599999999994	52.8605644600755\\
2.55799999999994	52.8500191088217\\
2.55999999999994	52.8394757359425\\
2.56199999999994	52.828934341617\\
2.56399999999994	52.8183949260251\\
2.56599999999994	52.8078574893468\\
2.56799999999994	52.7973220317592\\
2.56999999999994	52.7867885534389\\
2.57199999999994	52.7762570545618\\
2.57399999999994	52.765727535303\\
2.57599999999994	52.7551999958342\\
2.57799999999994	52.744674436328\\
2.57999999999994	52.7341508569567\\
2.58199999999994	52.7236292578871\\
2.58399999999994	52.7131096392889\\
2.58599999999994	52.7025920013291\\
2.58799999999994	52.6920763441744\\
2.58999999999994	52.6815626679881\\
2.59199999999994	52.6710509729333\\
2.59399999999994	52.6605412591715\\
2.59599999999994	52.6500335268646\\
2.59799999999994	52.6395277761708\\
2.59999999999994	52.6290240072474\\
2.60199999999994	52.6185222202502\\
2.60399999999993	52.6080224153363\\
2.60599999999993	52.5975245926564\\
2.60799999999993	52.5870287523642\\
2.60999999999993	52.5765348946094\\
2.61199999999993	52.5660430195413\\
2.61399999999993	52.5555531273091\\
2.61599999999993	52.545065218056\\
2.61799999999993	52.5345792919289\\
2.61999999999993	52.5240953490704\\
2.62199999999993	52.5136133896207\\
2.62399999999993	52.5031334137217\\
2.62599999999993	52.4926554215113\\
2.62799999999993	52.4821794131262\\
2.62999999999993	52.4717053887011\\
2.63199999999993	52.4612333483711\\
2.63399999999993	52.4507632922677\\
2.63599999999993	52.4402952205214\\
2.63799999999993	52.4298291332618\\
2.63999999999993	52.4193650306157\\
2.64199999999993	52.4089029127092\\
2.64399999999993	52.3984427796657\\
2.64599999999993	52.387984631609\\
2.64799999999993	52.3775284686592\\
2.64999999999993	52.3670742909359\\
2.65199999999993	52.356622098556\\
2.65399999999993	52.3461718916348\\
2.65599999999993	52.335723670289\\
2.65799999999993	52.3252774346288\\
2.65999999999993	52.3148331847654\\
2.66199999999993	52.3043909208092\\
2.66399999999993	52.2939506428659\\
2.66599999999993	52.2835123510433\\
2.66799999999993	52.2730760454422\\
2.66999999999993	52.2626417261679\\
2.67199999999993	52.2522093933208\\
2.67399999999993	52.241779046998\\
2.67599999999993	52.2313506872986\\
2.67799999999993	52.2209243143149\\
2.67999999999993	52.2104999281448\\
2.68199999999993	52.2000775288771\\
2.68399999999993	52.189657116604\\
2.68599999999993	52.1792386914117\\
2.68799999999993	52.1688222533886\\
2.68999999999993	52.1584078026188\\
2.69199999999993	52.1479953391865\\
2.69399999999992	52.1375848631717\\
2.69599999999992	52.1271763746554\\
2.69799999999992	52.1167698737138\\
2.69999999999992	52.1063653604247\\
2.70199999999992	52.0959628348618\\
2.70399999999992	52.0855622970975\\
2.70599999999992	52.0751637472031\\
2.70799999999992	52.0647671852468\\
2.70999999999992	52.0543726112963\\
2.71199999999992	52.043980025417\\
2.71399999999992	52.0335894276722\\
2.71599999999992	52.0232008181237\\
2.71799999999992	52.0128141968332\\
2.71999999999992	52.002429563857\\
2.72199999999992	51.9920469192532\\
2.72399999999992	51.981666263075\\
2.72599999999992	51.9712875953766\\
2.72799999999992	51.9609109162092\\
2.72999999999992	51.9505362256216\\
2.73199999999992	51.9401635236611\\
2.73399999999992	51.9297928103748\\
2.73599999999992	51.9194240858059\\
2.73799999999992	51.9090573499953\\
2.73999999999992	51.8986926029855\\
2.74199999999992	51.8883298448164\\
2.74399999999992	51.8779690755214\\
2.74599999999992	51.8676102951366\\
2.74799999999992	51.8572535036982\\
2.74999999999992	51.8468987012347\\
2.75199999999992	51.8365458877761\\
2.75399999999992	51.8261950633515\\
2.75599999999992	51.8158462279863\\
2.75799999999992	51.8054993817063\\
2.75999999999992	51.7951545245323\\
2.76199999999992	51.7848116564864\\
2.76399999999992	51.7744707775876\\
2.76599999999992	51.764131887853\\
2.76799999999992	51.7537949872987\\
2.76999999999992	51.7434600759384\\
2.77199999999992	51.733127153784\\
2.77399999999992	51.7227962208455\\
2.77599999999992	51.7124672771316\\
2.77799999999992	51.7021403226499\\
2.77999999999992	51.6918153574049\\
2.78199999999992	51.6814923813994\\
2.78399999999991	51.6711713946357\\
2.78599999999991	51.6608523971133\\
2.78799999999991	51.6505353888317\\
2.78999999999991	51.6402203697842\\
2.79199999999991	51.6299073399669\\
2.79399999999991	51.6195962993737\\
2.79599999999991	51.6092872479952\\
2.79799999999991	51.5989801858208\\
2.79999999999991	51.588675112838\\
2.80199999999991	51.5783720290312\\
2.80399999999991	51.5680709343876\\
2.80599999999991	51.5577718288884\\
2.80799999999991	51.5474747125145\\
2.80999999999991	51.5371795852449\\
2.81199999999991	51.5268864470568\\
2.81399999999991	51.5165952979265\\
2.81599999999991	51.5063061378278\\
2.81799999999991	51.4960189667337\\
2.81999999999991	51.4857337846139\\
2.82199999999991	51.4754505914377\\
2.82399999999991	51.4651693871729\\
2.82599999999991	51.4548901717844\\
2.82799999999991	51.4446129452372\\
2.82999999999991	51.4343377074924\\
2.83199999999991	51.4240644585114\\
2.83399999999991	51.4137931982531\\
2.83599999999991	51.4035239266747\\
2.83799999999991	51.3932566437317\\
2.83999999999991	51.3829913493791\\
2.84199999999991	51.3727280435684\\
2.84399999999991	51.3624667262508\\
2.84599999999991	51.3522073973753\\
2.84799999999991	51.341950056889\\
2.84999999999991	51.3316947047384\\
2.85199999999991	51.3214413408676\\
2.85399999999991	51.3111899652197\\
2.85599999999991	51.3009405777345\\
2.85799999999991	51.290693178353\\
2.85999999999991	51.2804477670126\\
2.86199999999991	51.2702043436499\\
2.86399999999991	51.2599629081974\\
2.86599999999991	51.2497234605914\\
2.86799999999991	51.2394860007609\\
2.86999999999991	51.2292505286366\\
2.87199999999991	51.2190170441482\\
2.87399999999991	51.2087855472212\\
2.8759999999999	51.1985560377801\\
2.8779999999999	51.1883285157503\\
2.8799999999999	51.1781029810538\\
2.8819999999999	51.1678794336111\\
2.8839999999999	51.1576578733401\\
2.8859999999999	51.14743830016\\
2.8879999999999	51.1372207139857\\
2.8899999999999	51.1270051147313\\
2.8919999999999	51.1167915023115\\
2.8939999999999	51.1065798766371\\
2.8959999999999	51.0963702376171\\
2.8979999999999	51.0861625851604\\
2.8999999999999	51.0759569191754\\
2.9019999999999	51.065753239566\\
2.9039999999999	51.0555515462363\\
2.9059999999999	51.0453518390896\\
2.9079999999999	51.0351541180259\\
2.9099999999999	51.0249583829463\\
2.9119999999999	51.014764633747\\
2.9139999999999	51.004572870327\\
2.9159999999999	50.994383092579\\
2.9179999999999	50.9841953003992\\
2.9199999999999	50.9740094936779\\
2.9219999999999	50.9638256723067\\
2.9239999999999	50.9536438361761\\
2.9259999999999	50.9434639851725\\
2.9279999999999	50.933286119183\\
2.9299999999999	50.9231102380933\\
2.9319999999999	50.912936341787\\
2.9339999999999	50.9027644301462\\
2.9359999999999	50.8925945030529\\
2.9379999999999	50.8824265603846\\
2.9399999999999	50.8722606020216\\
2.9419999999999	50.86209662784\\
2.9439999999999	50.8519346377151\\
2.9459999999999	50.8417746315207\\
2.9479999999999	50.8316166091301\\
2.9499999999999	50.8214605704146\\
2.9519999999999	50.8113065152434\\
2.9539999999999	50.8011544434865\\
2.9559999999999	50.7910043550102\\
2.9579999999999	50.7808562496816\\
2.9599999999999	50.7707101273642\\
2.9619999999999	50.7605659879218\\
2.9639999999999	50.750423831216\\
2.96599999999989	50.740283657108\\
2.96799999999989	50.730145465457\\
2.96999999999989	50.7200092561221\\
2.97199999999989	50.7098750289584\\
2.97399999999989	50.6997427838222\\
2.97599999999989	50.6896125205683\\
2.97799999999989	50.6794842390479\\
2.97999999999989	50.6693579391138\\
2.98199999999989	50.6592336206171\\
2.98399999999989	50.6491112834061\\
2.98599999999989	50.6389909273287\\
2.98799999999989	50.6288725522321\\
2.98999999999989	50.6187561579597\\
2.99199999999989	50.6086417443584\\
2.99399999999989	50.5985293112696\\
2.99599999999989	50.5884188585353\\
2.99799999999989	50.5783103859953\\
2.99999999999989	50.56820389349\\
3.00199999999989	50.5580993808567\\
3.00399999999989	50.5479968479322\\
3.00599999999989	50.5378962945517\\
3.00799999999989	50.5277977205503\\
3.00999999999989	50.5177011257609\\
3.01199999999989	50.5076065100165\\
3.01399999999989	50.497513873147\\
3.01599999999989	50.4874232149818\\
3.01799999999989	50.4773345353497\\
3.01999999999989	50.4672478340777\\
3.02199999999989	50.4571631109933\\
3.02399999999989	50.4470803659203\\
3.02599999999989	50.4369995986829\\
3.02799999999989	50.4269208091037\\
3.02999999999989	50.4168439970054\\
3.03199999999989	50.4067691622061\\
3.03399999999989	50.3966963045285\\
3.03599999999989	50.3866254237873\\
3.03799999999989	50.3765565198022\\
3.03999999999989	50.3664895923888\\
3.04199999999989	50.3564246413603\\
3.04399999999989	50.3463616665322\\
3.04599999999989	50.3363006677178\\
3.04799999999989	50.326241644726\\
3.04999999999989	50.3161845973703\\
3.05199999999989	50.3061295254589\\
3.05399999999989	50.2960764287997\\
3.05599999999989	50.2860253072005\\
3.05799999999988	50.2759761604683\\
3.05999999999988	50.265928988407\\
3.06199999999988	50.2558837908214\\
3.06399999999988	50.2458405675158\\
3.06599999999988	50.2357993182899\\
3.06799999999988	50.2257600429459\\
3.06999999999988	50.2157227412842\\
3.07199999999988	50.2056874131036\\
3.07399999999988	50.1956540582023\\
3.07599999999988	50.1856226763764\\
3.07799999999988	50.1755932674228\\
3.07999999999988	50.1655658311357\\
3.08199999999988	50.1555403673094\\
3.08399999999988	50.145516875737\\
3.08599999999988	50.1354953562103\\
3.08799999999988	50.1254758085208\\
3.08999999999988	50.1154582324565\\
3.09199999999988	50.1054426278093\\
3.09399999999988	50.095428994366\\
3.09599999999988	50.0854173319126\\
3.09799999999988	50.0754076402372\\
3.09999999999988	50.0653999191243\\
3.10199999999988	50.0553941683564\\
3.10399999999988	50.0453903877197\\
3.10599999999988	50.0353885769949\\
3.10799999999988	50.0253887359637\\
3.10999999999988	50.0153908644067\\
3.11199999999988	50.0053949621037\\
3.11399999999988	49.9954010288326\\
3.11599999999988	49.9854090643718\\
3.11799999999988	49.9754190684983\\
3.11999999999988	49.9654310409873\\
3.12199999999988	49.9554449816152\\
3.12399999999988	49.9454608901542\\
3.12599999999988	49.935478766379\\
3.12799999999988	49.9254986100613\\
3.12999999999988	49.9155204209736\\
3.13199999999988	49.9055441988848\\
3.13399999999988	49.8955699435671\\
3.13599999999988	49.8855976547865\\
3.13799999999988	49.875627332313\\
3.13999999999988	49.8656589759135\\
3.14199999999988	49.855692585354\\
3.14399999999988	49.8457281604001\\
3.14599999999988	49.8357657008165\\
3.14799999999987	49.8258052063661\\
3.14999999999987	49.8158466768141\\
3.15199999999987	49.8058901119197\\
3.15399999999987	49.7959355114465\\
3.15599999999987	49.7859828751539\\
3.15799999999987	49.7760322028025\\
3.15999999999987	49.7660834941496\\
3.16199999999987	49.7561367489551\\
3.16399999999987	49.7461919669748\\
3.16599999999987	49.7362491479667\\
3.16799999999987	49.7263082916851\\
3.16999999999987	49.7163693978848\\
3.17199999999987	49.7064324663213\\
3.17399999999987	49.696497496747\\
3.17599999999987	49.6865644889137\\
3.17799999999987	49.6766334425741\\
3.17999999999987	49.6667043574803\\
3.18199999999987	49.6567772333812\\
3.18399999999987	49.6468520700254\\
3.18599999999987	49.6369288671646\\
3.18799999999987	49.6270076245434\\
3.18999999999987	49.6170883419118\\
3.19199999999987	49.6071710190148\\
3.19399999999987	49.5972556555987\\
3.19599999999987	49.5873422514085\\
3.19799999999987	49.577430806188\\
3.19999999999987	49.5675213196817\\
3.20199999999987	49.5576137916315\\
3.20399999999987	49.5477082217819\\
3.20599999999987	49.5378046098712\\
3.20799999999987	49.5279029556417\\
3.20999999999987	49.5180032588341\\
3.21199999999987	49.5081055191867\\
3.21399999999987	49.4982097364383\\
3.21599999999987	49.4883159103277\\
3.21799999999987	49.4784240405911\\
3.21999999999987	49.4685341269654\\
3.22199999999987	49.4586461691871\\
3.22399999999987	49.4487601669914\\
3.22599999999987	49.4388761201126\\
3.22799999999987	49.4289940282834\\
3.22999999999987	49.4191138912407\\
3.23199999999987	49.4092357087135\\
3.23399999999987	49.3993594804347\\
3.23599999999987	49.3894852061367\\
3.23799999999986	49.3796128855496\\
3.23999999999986	49.3697425184029\\
3.24199999999986	49.3598741044256\\
3.24399999999986	49.3500076433481\\
3.24599999999986	49.3401431348971\\
3.24799999999986	49.3302805788008\\
3.24999999999986	49.3204199747858\\
3.25199999999986	49.3105613225786\\
3.25399999999986	49.3007046219048\\
3.25599999999986	49.290849872489\\
3.25799999999986	49.2809970740561\\
3.25999999999986	49.2711462263303\\
3.26199999999986	49.2612973290335\\
3.26399999999986	49.2514503818894\\
3.26599999999986	49.2416053846197\\
3.26799999999986	49.2317623369446\\
3.26999999999986	49.2219212385881\\
3.27199999999986	49.2120820892683\\
3.27399999999986	49.202244888704\\
3.27599999999986	49.1924096366169\\
3.27799999999986	49.1825763327228\\
3.27999999999986	49.1727449767418\\
3.28199999999986	49.1629155683905\\
3.28399999999986	49.1530881073859\\
3.28599999999986	49.1432625934438\\
3.28799999999986	49.133439026281\\
3.28999999999986	49.1236174056117\\
3.29199999999986	49.1137977311498\\
3.29399999999986	49.1039800026122\\
3.29599999999986	49.09416421971\\
3.29799999999986	49.0843503821565\\
3.29999999999986	49.0745384896649\\
3.30199999999986	49.0647285419472\\
3.30399999999986	49.0549205387136\\
3.30599999999986	49.0451144796774\\
3.30799999999986	49.035310364546\\
3.30999999999986	49.0255081930319\\
3.31199999999986	49.0157079648427\\
3.31399999999986	49.0059096796879\\
3.31599999999986	48.9961133372764\\
3.31799999999986	48.986318937315\\
3.31999999999986	48.9765264795116\\
3.32199999999986	48.9667359635737\\
3.32399999999986	48.9569473892059\\
3.32599999999986	48.9471607561154\\
3.32799999999986	48.9373760640071\\
3.32999999999985	48.9275933125858\\
3.33199999999985	48.9178125015572\\
3.33399999999985	48.9080336306231\\
3.33599999999985	48.8982566994877\\
3.33799999999985	48.8884817078548\\
3.33999999999985	48.878708655426\\
3.34199999999985	48.8689375419042\\
3.34399999999985	48.8591683669903\\
3.34599999999985	48.849401130386\\
3.34799999999985	48.8396358317909\\
3.34999999999985	48.8298724709052\\
3.35199999999985	48.8201110474304\\
3.35399999999985	48.8103515610643\\
3.35599999999985	48.8005940115057\\
3.35799999999985	48.7908383984537\\
3.35999999999985	48.7810847216062\\
3.36199999999985	48.7713329806603\\
3.36399999999985	48.7615831753137\\
3.36599999999985	48.7518353052625\\
3.36799999999985	48.7420893702042\\
3.36999999999985	48.7323453698317\\
3.37199999999985	48.7226033038439\\
3.37399999999985	48.7128631719335\\
3.37599999999985	48.703124973796\\
3.37799999999985	48.6933887091247\\
3.37999999999985	48.6836543776146\\
3.38199999999985	48.6739219789582\\
3.38399999999985	48.664191512848\\
3.38599999999985	48.6544629789785\\
3.38799999999985	48.6447363770389\\
3.38999999999985	48.6350117067227\\
3.39199999999985	48.6252889677211\\
3.39399999999985	48.615568159725\\
3.39599999999985	48.6058492824252\\
3.39799999999985	48.5961323355113\\
3.39999999999985	48.5864173186732\\
3.40199999999985	48.5767042315998\\
3.40399999999985	48.5669930739812\\
3.40599999999985	48.5572838455051\\
3.40799999999985	48.5475765458605\\
3.40999999999985	48.5378711747353\\
3.41199999999985	48.5281677318156\\
3.41399999999985	48.5184662167907\\
3.41599999999985	48.5087666293459\\
3.41799999999985	48.4990689691681\\
3.41999999999984	48.4893732359434\\
3.42199999999984	48.4796794293566\\
3.42399999999984	48.4699875490943\\
3.42599999999984	48.4602975948409\\
3.42799999999984	48.4506095662817\\
3.42999999999984	48.4409234631006\\
3.43199999999984	48.4312392849817\\
3.43399999999984	48.4215570316075\\
3.43599999999984	48.4118767026627\\
3.43799999999984	48.40219829783\\
3.43999999999984	48.392521816792\\
3.44199999999984	48.3828472592297\\
3.44399999999984	48.373174624827\\
3.44599999999984	48.3635039132648\\
3.44799999999984	48.3538351242236\\
3.44999999999984	48.3441682573856\\
3.45199999999984	48.3345033124306\\
3.45399999999984	48.3248402890398\\
3.45599999999984	48.3151791868917\\
3.45799999999984	48.3055200056675\\
3.45999999999984	48.295862745046\\
3.46199999999984	48.2862074047062\\
3.46399999999984	48.2765539843272\\
3.46599999999984	48.2669024835871\\
3.46799999999984	48.2572529021646\\
3.46999999999984	48.2476052397369\\
3.47199999999984	48.2379594959819\\
3.47399999999984	48.2283156705773\\
3.47599999999984	48.2186737631994\\
3.47799999999984	48.2090337735243\\
3.47999999999984	48.1993957012307\\
3.48199999999984	48.1897595459926\\
3.48399999999984	48.1801253074866\\
3.48599999999984	48.1704929853886\\
3.48799999999984	48.1608625793725\\
3.48999999999984	48.1512340891141\\
3.49199999999984	48.1416075142887\\
3.49399999999984	48.1319828545703\\
3.49599999999984	48.122360109633\\
3.49799999999984	48.1127392791506\\
3.49999999999984	48.103120362797\\
3.50199999999984	48.0935033602447\\
3.50399999999984	48.0838882711675\\
3.50599999999984	48.0742750952385\\
3.50799999999984	48.0646638321292\\
3.50999999999984	48.0550544815131\\
3.51199999999983	48.0454470430618\\
3.51399999999983	48.035841516447\\
3.51599999999983	48.0262379013401\\
3.51799999999983	48.0166361974123\\
3.51999999999983	48.007036404335\\
3.52199999999983	47.9974385217788\\
3.52399999999983	47.9878425494151\\
3.52599999999983	47.978248486912\\
3.52799999999983	47.9686563339421\\
3.52999999999983	47.9590660901725\\
3.53199999999983	47.9494777552752\\
3.53399999999983	47.9398913289191\\
3.53599999999983	47.9303068107721\\
3.53799999999983	47.9207242005047\\
3.53999999999983	47.9111434977836\\
3.54199999999983	47.9015647022782\\
3.54399999999983	47.8919878136574\\
3.54599999999983	47.8824128315883\\
3.54799999999983	47.8728397557386\\
3.54999999999983	47.863268585776\\
3.55199999999983	47.8536993213672\\
3.55399999999983	47.8441319621804\\
3.55599999999983	47.834566507882\\
3.55799999999983	47.8250029581381\\
3.55999999999983	47.8154413126162\\
3.56199999999983	47.8058815709805\\
3.56399999999983	47.7963237328998\\
3.56599999999983	47.7867677980379\\
3.56799999999983	47.7772137660606\\
3.56999999999983	47.7676616366338\\
3.57199999999983	47.7581114094226\\
3.57399999999983	47.7485630840912\\
3.57599999999983	47.7390166603052\\
3.57799999999983	47.7294721377293\\
3.57999999999983	47.7199295160274\\
3.58199999999983	47.7103887948632\\
3.58399999999983	47.7008499739014\\
3.58599999999983	47.6913130528059\\
3.58799999999983	47.6817780312403\\
3.58999999999983	47.6722449088673\\
3.59199999999983	47.6627136853503\\
3.59399999999983	47.6531843603531\\
3.59599999999983	47.6436569335379\\
3.59799999999983	47.6341314045673\\
3.59999999999983	47.6246077731044\\
3.60199999999982	47.6150860388111\\
3.60399999999982	47.6055662013484\\
3.60599999999982	47.5960482603803\\
3.60799999999982	47.5865322155678\\
3.60999999999982	47.5770180665711\\
3.61199999999982	47.5675058130542\\
3.61399999999982	47.5579954546761\\
3.61599999999982	47.5484869910992\\
3.61799999999982	47.5389804219833\\
3.61999999999982	47.5294757469902\\
3.62199999999982	47.5199729657803\\
3.62399999999982	47.5104720780135\\
3.62599999999982	47.5009730833497\\
3.62799999999982	47.4914759814512\\
3.62999999999982	47.4819807719748\\
3.63199999999982	47.4724874545824\\
3.63399999999982	47.4629960289332\\
3.63599999999982	47.4535064946861\\
3.63799999999982	47.444018851501\\
3.63999999999982	47.4345330990373\\
3.64199999999982	47.4250492369528\\
3.64399999999982	47.4155672649074\\
3.64599999999982	47.4060871825597\\
3.64799999999982	47.3966089895676\\
3.64999999999982	47.38713268559\\
3.65199999999982	47.377658270285\\
3.65399999999982	47.3681857433103\\
3.65599999999982	47.3587151043248\\
3.65799999999982	47.3492463529865\\
3.65999999999982	47.3397794889518\\
3.66199999999982	47.3303145118785\\
3.66399999999982	47.3208514214251\\
3.66599999999982	47.3113902172477\\
3.66799999999982	47.3019308990047\\
3.66999999999982	47.2924734663525\\
3.67199999999982	47.2830179189463\\
3.67399999999982	47.2735642564452\\
3.67599999999982	47.2641124785045\\
3.67799999999982	47.2546625847817\\
3.67999999999982	47.2452145749319\\
3.68199999999982	47.2357684486127\\
3.68399999999982	47.226324205479\\
3.68599999999982	47.2168818451872\\
3.68799999999982	47.2074413673939\\
3.68999999999982	47.1980027717535\\
3.69199999999981	47.1885660579223\\
3.69399999999981	47.1791312255562\\
3.69599999999981	47.1696982743095\\
3.69799999999981	47.1602672038385\\
3.69999999999981	47.1508380137976\\
3.70199999999981	47.1414107038427\\
3.70399999999981	47.1319852736275\\
3.70599999999981	47.1225617228071\\
3.70799999999981	47.1131400510373\\
3.70999999999981	47.1037202579713\\
3.71199999999981	47.0943023432643\\
3.71399999999981	47.0848863065705\\
3.71599999999981	47.0754721475444\\
3.71799999999981	47.0660598658396\\
3.71999999999981	47.0566494611106\\
3.72199999999981	47.0472409330111\\
3.72399999999981	47.0378342811945\\
3.72599999999981	47.0284295053158\\
3.72799999999981	47.0190266050271\\
3.72999999999981	47.0096255799828\\
3.73199999999981	47.0002264298361\\
3.73399999999981	46.9908291542402\\
3.73599999999981	46.9814337528487\\
3.73799999999981	46.9720402253151\\
3.73999999999981	46.962648571291\\
3.74199999999981	46.9532587904309\\
3.74399999999981	46.9438708823862\\
3.74599999999981	46.9344848468107\\
3.74799999999981	46.9251006833569\\
3.74999999999981	46.9157183916769\\
3.75199999999981	46.9063379714243\\
3.75399999999981	46.8969594222498\\
3.75599999999981	46.8875827438072\\
3.75799999999981	46.8782079357485\\
3.75999999999981	46.8688349977247\\
3.76199999999981	46.8594639293892\\
3.76399999999981	46.8500947303937\\
3.76599999999981	46.8407274003892\\
3.76799999999981	46.8313619390283\\
3.76999999999981	46.8219983459631\\
3.77199999999981	46.8126366208444\\
3.77399999999981	46.8032767633236\\
3.77599999999981	46.7939187730532\\
3.77799999999981	46.7845626496845\\
3.77999999999981	46.7752083928671\\
3.78199999999981	46.7658560022546\\
3.7839999999998	46.7565054774965\\
3.7859999999998	46.7471568182456\\
3.7879999999998	46.7378100241509\\
3.7899999999998	46.728465094865\\
3.7919999999998	46.7191220300381\\
3.7939999999998	46.7097808293219\\
3.7959999999998	46.7004414923649\\
3.7979999999998	46.6911040188202\\
3.7999999999998	46.681768408337\\
3.8019999999998	46.6724346605665\\
3.8039999999998	46.6631027751593\\
3.8059999999998	46.6537727517651\\
3.8079999999998	46.6444445900352\\
3.8099999999998	46.6351182896197\\
3.8119999999998	46.6257938501684\\
3.8139999999998	46.6164712713307\\
3.8159999999998	46.6071505527585\\
3.8179999999998	46.5978316941007\\
3.8199999999998	46.5885146950086\\
3.8219999999998	46.5791995551297\\
3.8239999999998	46.5698862741157\\
3.8259999999998	46.5605748516165\\
3.8279999999998	46.551265287281\\
3.8299999999998	46.5419575807598\\
3.8319999999998	46.5326517317009\\
3.8339999999998	46.5233477397563\\
3.8359999999998	46.514045604574\\
3.8379999999998	46.5047453258034\\
3.8399999999998	46.4954469030949\\
3.8419999999998	46.4861503360962\\
3.8439999999998	46.4768556244586\\
3.8459999999998	46.4675627678305\\
3.8479999999998	46.4582717658613\\
3.8499999999998	46.4489826181999\\
3.8519999999998	46.4396953244955\\
3.8539999999998	46.4304098843979\\
3.8559999999998	46.421126297555\\
3.8579999999998	46.4118445636166\\
3.8599999999998	46.4025646822317\\
3.8619999999998	46.393286653048\\
3.8639999999998	46.3840104757169\\
3.8659999999998	46.3747361498849\\
3.8679999999998	46.3654636752011\\
3.8699999999998	46.3561930513149\\
3.8719999999998	46.3469242778748\\
3.87399999999979	46.3376573545303\\
3.87599999999979	46.3283922809284\\
3.87799999999979	46.3191290567188\\
3.87999999999979	46.3098676815501\\
3.88199999999979	46.3006081550702\\
3.88399999999979	46.2913504769274\\
3.88599999999979	46.2820946467715\\
3.88799999999979	46.2728406642491\\
3.88999999999979	46.2635885290104\\
3.89199999999979	46.2543382407026\\
3.89399999999979	46.2450897989742\\
3.89599999999979	46.2358432034733\\
3.89799999999979	46.2265984538487\\
3.89999999999979	46.2173555497478\\
3.90199999999979	46.2081144908198\\
3.90399999999979	46.1988752767123\\
3.90599999999979	46.1896379070725\\
3.90799999999979	46.1804023815502\\
3.90999999999979	46.1711686997929\\
3.91199999999979	46.1619368614476\\
3.91399999999979	46.1527068661643\\
3.91599999999979	46.1434787135887\\
3.91799999999979	46.1342524033706\\
3.91999999999979	46.1250279351576\\
3.92199999999979	46.1158053085964\\
3.92399999999979	46.106584523336\\
3.92599999999979	46.0973655790234\\
3.92799999999979	46.0881484753075\\
3.92999999999979	46.0789332118356\\
3.93199999999979	46.0697197882551\\
3.93399999999979	46.0605082042144\\
3.93599999999979	46.0512984593608\\
3.93799999999979	46.0420905533423\\
3.93999999999979	46.0328844858069\\
3.94199999999979	46.0236802564015\\
3.94399999999979	46.0144778647745\\
3.94599999999979	46.0052773105733\\
3.94799999999979	45.9960785934449\\
3.94999999999979	45.9868817130376\\
3.95199999999979	45.977686668999\\
3.95399999999979	45.9684934609764\\
3.95599999999979	45.9593020886178\\
3.95799999999979	45.9501125515706\\
3.95999999999979	45.9409248494819\\
3.96199999999979	45.9317389819997\\
3.96399999999979	45.9225549487718\\
3.96599999999978	45.9133727494441\\
3.96799999999978	45.9041923836657\\
3.96999999999978	45.8950138510839\\
3.97199999999978	45.8858371513455\\
3.97399999999978	45.8766622840981\\
3.97599999999978	45.8674892489899\\
3.97799999999978	45.8583180456679\\
3.97999999999978	45.8491486737787\\
3.98199999999978	45.8399811329706\\
3.98399999999978	45.8308154228909\\
3.98599999999978	45.8216515431868\\
3.98799999999978	45.8124894935059\\
3.98999999999978	45.8033292734954\\
3.99199999999978	45.7941708828031\\
3.99399999999978	45.7850143210752\\
3.99599999999978	45.7758595879602\\
3.99799999999978	45.7667066831051\\
3.99999999999978	45.7575556061574\\
4.00199999999978	45.7484063567637\\
4.00399999999978	45.7392589345728\\
4.00599999999978	45.7301133392307\\
4.00799999999978	45.7209695703852\\
4.00999999999978	45.7118276276834\\
4.01199999999978	45.7026875107733\\
4.01399999999978	45.6935492193015\\
4.01599999999978	45.6844127529153\\
4.01799999999978	45.6752781112628\\
4.01999999999978	45.666145293991\\
4.02199999999978	45.6570143007465\\
4.02399999999978	45.6478851311773\\
4.02599999999978	45.6387577849314\\
4.02799999999978	45.6296322616549\\
4.02999999999978	45.6205085609949\\
4.03199999999978	45.6113866826002\\
4.03399999999978	45.6022666261168\\
4.03599999999978	45.5931483911932\\
4.03799999999978	45.5840319774748\\
4.03999999999978	45.5749173846118\\
4.04199999999978	45.5658046122495\\
4.04399999999978	45.5566936600353\\
4.04599999999978	45.5475845276173\\
4.04799999999978	45.5384772146421\\
4.04999999999978	45.529371720758\\
4.05199999999978	45.5202680456119\\
4.05399999999978	45.5111661888514\\
4.05599999999978	45.502066150123\\
4.05799999999978	45.4929679290746\\
4.05999999999977	45.4838715253546\\
4.06199999999977	45.4747769386091\\
4.06399999999977	45.4656841684857\\
4.06599999999977	45.4565932146322\\
4.06799999999977	45.4475040766957\\
4.06999999999977	45.4384167543244\\
4.07199999999977	45.4293312471644\\
4.07399999999977	45.4202475548646\\
4.07599999999977	45.4111656770719\\
4.07799999999977	45.4020856134334\\
4.07999999999977	45.3930073635966\\
4.08199999999977	45.3839309272096\\
4.08399999999977	45.3748563039199\\
4.08599999999977	45.365783493374\\
4.08799999999977	45.3567124952208\\
4.08999999999977	45.347643309107\\
4.09199999999977	45.3385759346804\\
4.09399999999977	45.3295103715888\\
4.09599999999977	45.3204466194797\\
4.09799999999977	45.3113846779996\\
4.09999999999977	45.3023245467976\\
4.10199999999977	45.2932662255211\\
4.10399999999977	45.2842097138174\\
4.10599999999977	45.2751550113342\\
4.10799999999977	45.2661021177195\\
4.10999999999977	45.2570510326201\\
4.11199999999977	45.2480017556848\\
4.11399999999977	45.238954286561\\
4.11599999999977	45.2299086248959\\
4.11799999999977	45.2208647703381\\
4.11999999999977	45.2118227225343\\
4.12199999999977	45.2027824811335\\
4.12399999999977	45.1937440457826\\
4.12599999999977	45.1847074161305\\
4.12799999999977	45.1756725918238\\
4.12999999999977	45.1666395725112\\
4.13199999999977	45.1576083578407\\
4.13399999999977	45.1485789474592\\
4.13599999999977	45.1395513410161\\
4.13799999999977	45.1305255381579\\
4.13999999999977	45.1215015385338\\
4.14199999999977	45.1124793417907\\
4.14399999999977	45.1034589475774\\
4.14599999999977	45.094440355542\\
4.14799999999977	45.0854235653318\\
4.14999999999976	45.0764085765958\\
4.15199999999976	45.0673953889815\\
4.15399999999976	45.0583840021376\\
4.15599999999976	45.0493744157112\\
4.15799999999976	45.0403666293514\\
4.15999999999976	45.0313606427061\\
4.16199999999976	45.0223564554236\\
4.16399999999976	45.0133540671523\\
4.16599999999976	45.0043534775399\\
4.16799999999976	44.9953546862353\\
4.16999999999976	44.9863576928862\\
4.17199999999976	44.977362497142\\
4.17399999999976	44.96836909865\\
4.17599999999976	44.9593774970596\\
4.17799999999976	44.9503876920181\\
4.17999999999976	44.9413996831742\\
4.18199999999976	44.9324134701771\\
4.18399999999976	44.9234290526751\\
4.18599999999976	44.9144464303165\\
4.18799999999976	44.9054656027491\\
4.18999999999976	44.8964865696232\\
4.19199999999976	44.8875093305858\\
4.19399999999976	44.878533885286\\
4.19599999999976	44.8695602333734\\
4.19799999999976	44.8605883744957\\
4.19999999999976	44.8516183083015\\
4.20199999999976	44.8426500344402\\
4.20399999999976	44.8336835525607\\
4.20599999999976	44.8247188623106\\
4.20799999999976	44.8157559633401\\
4.20999999999976	44.806794855298\\
4.21199999999976	44.7978355378323\\
4.21399999999976	44.7888780105922\\
4.21599999999976	44.7799222732265\\
4.21799999999976	44.7709683253856\\
4.21999999999976	44.7620161667167\\
4.22199999999976	44.7530657968702\\
4.22399999999976	44.7441172154946\\
4.22599999999976	44.7351704222387\\
4.22799999999976	44.7262254167519\\
4.22999999999976	44.7172821986836\\
4.23199999999976	44.7083407676829\\
4.23399999999976	44.6994011233991\\
4.23599999999976	44.6904632654814\\
4.23799999999976	44.681527193579\\
4.23999999999976	44.6725929073415\\
4.24199999999975	44.6636604064185\\
4.24399999999975	44.6547296904585\\
4.24599999999975	44.6458007591113\\
4.24799999999975	44.6368736120279\\
4.24999999999975	44.6279482488548\\
4.25199999999975	44.6190246692449\\
4.25399999999975	44.6101028728452\\
4.25599999999975	44.6011828593063\\
4.25799999999975	44.5922646282782\\
4.25999999999975	44.5833481794102\\
4.26199999999975	44.5744335123521\\
4.26399999999975	44.5655206267528\\
4.26599999999975	44.5566095222636\\
4.26799999999975	44.5477001985336\\
4.26999999999975	44.5387926552128\\
4.27199999999975	44.5298868919508\\
4.27399999999975	44.5209829083978\\
4.27599999999975	44.5120807042039\\
4.27799999999975	44.5031802790183\\
4.27999999999975	44.4942816324924\\
4.28199999999975	44.4853847642757\\
4.28399999999975	44.4764896740177\\
4.28599999999975	44.467596361369\\
4.28799999999975	44.4587048259802\\
4.28999999999975	44.4498150675014\\
4.29199999999975	44.4409270855825\\
4.29399999999975	44.4320408798745\\
4.29599999999975	44.4231564500266\\
4.29799999999975	44.4142737956902\\
4.29999999999975	44.4053929165151\\
4.30199999999975	44.3965138121523\\
4.30399999999975	44.3876364822523\\
4.30599999999975	44.3787609264652\\
4.30799999999975	44.3698871444414\\
4.30999999999975	44.3610151358332\\
4.31199999999975	44.3521449002896\\
4.31399999999975	44.3432764374613\\
4.31599999999975	44.3344097469998\\
4.31799999999975	44.3255448285555\\
4.31999999999975	44.3166816817797\\
4.32199999999975	44.3078203063225\\
4.32399999999975	44.298960701835\\
4.32599999999975	44.2901028679689\\
4.32799999999975	44.2812468043746\\
4.32999999999975	44.2723925107031\\
4.33199999999974	44.2635399866058\\
4.33399999999974	44.2546892317332\\
4.33599999999974	44.2458402457372\\
4.33799999999974	44.2369930282684\\
4.33999999999974	44.2281475789786\\
4.34199999999974	44.2193038975189\\
4.34399999999974	44.2104619835403\\
4.34599999999974	44.2016218366949\\
4.34799999999974	44.1927834566338\\
4.34999999999974	44.1839468430077\\
4.35199999999974	44.175111995469\\
4.35399999999974	44.1662789136685\\
4.35599999999974	44.1574475972588\\
4.35799999999974	44.1486180458911\\
4.35999999999974	44.1397902592166\\
4.36199999999974	44.130964236888\\
4.36399999999974	44.1221399785557\\
4.36599999999974	44.113317483873\\
4.36799999999974	44.10449675249\\
4.36999999999974	44.0956777840603\\
4.37199999999974	44.0868605782358\\
4.37399999999974	44.0780451346674\\
4.37599999999974	44.069231453007\\
4.37799999999974	44.0604195329074\\
4.37999999999974	44.0516093740212\\
4.38199999999974	44.0428009759998\\
4.38399999999974	44.0339943384956\\
4.38599999999974	44.0251894611609\\
4.38799999999974	44.0163863436468\\
4.38999999999974	44.0075849856082\\
4.39199999999974	43.9987853866957\\
4.39399999999974	43.9899875465624\\
4.39599999999974	43.9811914648602\\
4.39799999999974	43.972397141242\\
4.39999999999974	43.9636045753602\\
4.40199999999974	43.9548137668675\\
4.40399999999974	43.946024715417\\
4.40599999999974	43.9372374206606\\
4.40799999999974	43.9284518822518\\
4.40999999999974	43.9196680998433\\
4.41199999999974	43.9108860730872\\
4.41399999999974	43.902105801638\\
4.41599999999974	43.8933272851471\\
4.41799999999974	43.8845505232677\\
4.41999999999974	43.8757755156541\\
4.42199999999974	43.867002261958\\
4.42399999999973	43.8582307618333\\
4.42599999999973	43.8494610149326\\
4.42799999999973	43.84069302091\\
4.42999999999973	43.8319267794179\\
4.43199999999973	43.8231622901101\\
4.43399999999973	43.8143995526401\\
4.43599999999973	43.8056385666609\\
4.43799999999973	43.7968793318264\\
4.43999999999973	43.7881218477904\\
4.44199999999973	43.779366114205\\
4.44399999999973	43.7706121307259\\
4.44599999999973	43.7618598970053\\
4.44799999999973	43.7531094126977\\
4.44999999999973	43.7443606774568\\
4.45199999999973	43.7356136909361\\
4.45399999999973	43.7268684527891\\
4.45599999999973	43.7181249626701\\
4.45799999999973	43.7093832202333\\
4.45999999999973	43.7006432251332\\
4.46199999999973	43.6919049770231\\
4.46399999999973	43.6831684755569\\
4.46599999999973	43.6744337203901\\
4.46799999999973	43.6657007111753\\
4.46999999999973	43.6569694475675\\
4.47199999999973	43.6482399292218\\
4.47399999999973	43.6395121557903\\
4.47599999999973	43.6307861269302\\
4.47799999999973	43.6220618422948\\
4.47999999999973	43.6133393015375\\
4.48199999999973	43.6046185043148\\
4.48399999999973	43.5958994502805\\
4.48599999999973	43.5871821390888\\
4.48799999999973	43.5784665703953\\
4.48999999999973	43.5697527438549\\
4.49199999999973	43.5610406591201\\
4.49399999999973	43.5523303158485\\
4.49599999999973	43.5436217136942\\
4.49799999999973	43.5349148523123\\
4.49999999999973	43.5262097313575\\
4.50199999999973	43.5175063504851\\
4.50399999999973	43.5088047093501\\
4.50599999999973	43.5001048076081\\
4.50799999999973	43.4914066449139\\
4.50999999999973	43.4827102209226\\
4.51199999999973	43.4740155352899\\
4.51399999999972	43.4653225876719\\
4.51599999999972	43.4566313777231\\
4.51799999999972	43.4479419050993\\
4.51999999999972	43.4392541694556\\
4.52199999999972	43.4305681704486\\
4.52399999999972	43.4218839077335\\
4.52599999999972	43.4132013809657\\
4.52799999999972	43.4045205898011\\
4.52999999999972	43.3958415338959\\
4.53199999999972	43.3871642129052\\
4.53399999999972	43.3784886264861\\
4.53599999999972	43.3698147742939\\
4.53799999999972	43.3611426559844\\
4.53999999999972	43.3524722712148\\
4.54199999999972	43.3438036196398\\
4.54399999999972	43.3351367009159\\
4.54599999999972	43.3264715147001\\
4.54799999999972	43.3178080606487\\
4.54999999999972	43.3091463384172\\
4.55199999999972	43.3004863476631\\
4.55399999999972	43.2918280880415\\
4.55599999999972	43.2831715592101\\
4.55799999999972	43.2745167608249\\
4.55999999999972	43.2658636925425\\
4.56199999999972	43.2572123540207\\
4.56399999999972	43.2485627449145\\
4.56599999999972	43.2399148648815\\
4.56799999999972	43.2312687135793\\
4.56999999999972	43.2226242906633\\
4.57199999999972	43.213981595792\\
4.57399999999972	43.2053406286216\\
4.57599999999972	43.1967013888081\\
4.57799999999972	43.1880638760107\\
4.57999999999972	43.1794280898851\\
4.58199999999972	43.1707940300892\\
4.58399999999972	43.1621616962801\\
4.58599999999972	43.1535310881156\\
4.58799999999972	43.1449022052527\\
4.58999999999972	43.1362750473482\\
4.59199999999972	43.1276496140605\\
4.59399999999972	43.119025905048\\
4.59599999999972	43.1104039199657\\
4.59799999999972	43.1017836584733\\
4.59999999999972	43.0931651202279\\
4.60199999999972	43.0845483048878\\
4.60399999999971	43.0759332121102\\
4.60599999999971	43.0673198415534\\
4.60799999999971	43.0587081928745\\
4.60999999999971	43.0500982657334\\
4.61199999999971	43.041490059786\\
4.61399999999971	43.0328835746916\\
4.61599999999971	43.0242788101087\\
4.61799999999971	43.0156757656941\\
4.61999999999971	43.0070744411073\\
4.62199999999971	42.9984748360072\\
4.62399999999971	42.9898769500505\\
4.62599999999971	42.9812807828962\\
4.62799999999971	42.9726863342042\\
4.62999999999971	42.9640936036306\\
4.63199999999971	42.9555025908362\\
4.63399999999971	42.9469132954785\\
4.63599999999971	42.9383257172174\\
4.63799999999971	42.9297398557099\\
4.63999999999971	42.9211557106162\\
4.64199999999971	42.9125732815949\\
4.64399999999971	42.9039925683049\\
4.64599999999971	42.8954135704049\\
4.64799999999971	42.8868362875551\\
4.64999999999971	42.8782607194135\\
4.65199999999971	42.8696868656392\\
4.65399999999971	42.8611147258926\\
4.65599999999971	42.8525442998318\\
4.65799999999971	42.8439755871177\\
4.65999999999971	42.8354085874071\\
4.66199999999971	42.8268433003626\\
4.66399999999971	42.8182797256411\\
4.66599999999971	42.8097178629038\\
4.66799999999971	42.80115771181\\
4.66999999999971	42.7925992720186\\
4.67199999999971	42.7840425431905\\
4.67399999999971	42.7754875249848\\
4.67599999999971	42.7669342170618\\
4.67799999999971	42.7583826190813\\
4.67999999999971	42.7498327307034\\
4.68199999999971	42.7412845515876\\
4.68399999999971	42.7327380813945\\
4.68599999999971	42.7241933197846\\
4.68799999999971	42.7156502664164\\
4.68999999999971	42.7071089209534\\
4.69199999999971	42.6985692830524\\
4.69399999999971	42.6900313523768\\
4.6959999999997	42.6814951285852\\
4.6979999999997	42.6729606113378\\
4.6999999999997	42.6644278002973\\
4.7019999999997	42.6558966951223\\
4.7039999999997	42.6473672954741\\
4.7059999999997	42.6388396010137\\
4.7079999999997	42.6303136114018\\
4.7099999999997	42.6217893262994\\
4.7119999999997	42.6132667453668\\
4.7139999999997	42.6047458682661\\
4.7159999999997	42.5962266946573\\
4.7179999999997	42.5877092242012\\
4.7199999999997	42.5791934565605\\
4.7219999999997	42.5706793913955\\
4.7239999999997	42.5621670283675\\
4.7259999999997	42.5536563671378\\
4.7279999999997	42.5451474073674\\
4.7299999999997	42.5366401487187\\
4.7319999999997	42.5281345908524\\
4.7339999999997	42.5196307334311\\
4.7359999999997	42.5111285761153\\
4.7379999999997	42.5026281185671\\
4.7399999999997	42.4941293604481\\
4.7419999999997	42.4856323014198\\
4.7439999999997	42.4771369411448\\
4.7459999999997	42.4686432792854\\
4.7479999999997	42.4601513155021\\
4.7499999999997	42.4516610494582\\
4.7519999999997	42.4431724808146\\
4.7539999999997	42.4346856092352\\
4.7559999999997	42.4262004343807\\
4.7579999999997	42.417716955914\\
4.7599999999997	42.4092351734976\\
4.7619999999997	42.4007550867928\\
4.7639999999997	42.3922766954635\\
4.7659999999997	42.3837999991722\\
4.7679999999997	42.3753249975796\\
4.7699999999997	42.3668516903503\\
4.7719999999997	42.3583800771462\\
4.7739999999997	42.3499101576301\\
4.7759999999997	42.3414419314648\\
4.7779999999997	42.3329753983134\\
4.7799999999997	42.3245105578388\\
4.7819999999997	42.3160474097036\\
4.7839999999997	42.307585953571\\
4.78599999999969	42.2991261891043\\
4.78799999999969	42.2906681159672\\
4.78999999999969	42.2822117338212\\
4.79199999999969	42.2737570423314\\
4.79399999999969	42.2653040411606\\
4.79599999999969	42.256852729972\\
4.79799999999969	42.2484031084291\\
4.79999999999969	42.2399551761949\\
4.80199999999969	42.2315089329342\\
4.80399999999969	42.2230643783096\\
4.80599999999969	42.2146215119852\\
4.80799999999969	42.2061803336249\\
4.80999999999969	42.1977408428926\\
4.81199999999969	42.189303039452\\
4.81399999999969	42.1808669229665\\
4.81599999999969	42.1724324931011\\
4.81799999999969	42.1639997495195\\
4.81999999999969	42.155568691885\\
4.82199999999969	42.1471393198631\\
4.82399999999969	42.1387116331174\\
4.82599999999969	42.1302856313124\\
4.82799999999969	42.1218613141124\\
4.82999999999969	42.1134386811816\\
4.83199999999969	42.1050177321845\\
4.83399999999969	42.0965984667865\\
4.83599999999969	42.0881808846508\\
4.83799999999969	42.0797649854434\\
4.83999999999969	42.0713507688281\\
4.84199999999969	42.0629382344704\\
4.84399999999969	42.054527382034\\
4.84599999999969	42.0461182111859\\
4.84799999999969	42.0377107215888\\
4.84999999999969	42.0293049129087\\
4.85199999999969	42.0209007848109\\
4.85399999999969	42.0124983369604\\
4.85599999999969	42.0040975690223\\
4.85799999999969	41.9956984806625\\
4.85999999999969	41.9873010715456\\
4.86199999999969	41.9789053413368\\
4.86399999999969	41.9705112897022\\
4.86599999999969	41.9621189163072\\
4.86799999999969	41.9537282208172\\
4.86999999999969	41.9453392028979\\
4.87199999999969	41.9369518622145\\
4.87399999999969	41.928566198434\\
4.87599999999969	41.9201822112213\\
4.87799999999968	41.911799900242\\
4.87999999999968	41.9034192651633\\
4.88199999999968	41.8950403056497\\
4.88399999999968	41.8866630213679\\
4.88599999999968	41.8782874119854\\
4.88799999999968	41.8699134771662\\
4.88999999999968	41.8615412165773\\
4.89199999999968	41.8531706298853\\
4.89399999999968	41.8448017167569\\
4.89599999999968	41.8364344768579\\
4.89799999999968	41.8280689098552\\
4.89999999999968	41.8197050154143\\
4.90199999999968	41.8113427932035\\
4.90399999999968	41.8029822428885\\
4.90599999999968	41.794623364136\\
4.90799999999968	41.7862661566131\\
4.90999999999968	41.7779106199862\\
4.91199999999968	41.7695567539231\\
4.91399999999968	41.7612045580896\\
4.91599999999968	41.7528540321542\\
4.91799999999968	41.7445051757828\\
4.91999999999968	41.7361579886429\\
4.92199999999968	41.7278124704016\\
4.92399999999968	41.719468620727\\
4.92599999999968	41.7111264392851\\
4.92799999999968	41.7027859257446\\
4.92999999999968	41.6944470797723\\
4.93199999999968	41.6861099010359\\
4.93399999999968	41.6777743892027\\
4.93599999999968	41.6694405439411\\
4.93799999999968	41.661108364918\\
4.93999999999968	41.6527778518021\\
4.94199999999968	41.6444490042595\\
4.94399999999968	41.6361218219603\\
4.94599999999968	41.627796304571\\
4.94799999999968	41.6194724517601\\
4.94999999999968	41.6111502631956\\
4.95199999999968	41.6028297385458\\
4.95399999999968	41.5945108774791\\
4.95599999999968	41.5861936796629\\
4.95799999999968	41.5778781447666\\
4.95999999999968	41.5695642724581\\
4.96199999999968	41.5612520624058\\
4.96399999999968	41.5529415142782\\
4.96599999999968	41.5446326277441\\
4.96799999999967	41.536325402472\\
4.96999999999967	41.5280198381306\\
4.97199999999967	41.5197159343887\\
4.97399999999967	41.5114136909151\\
4.97599999999967	41.5031131073787\\
4.97799999999967	41.4948141834486\\
4.97999999999967	41.4865169187938\\
4.98199999999967	41.478221313083\\
4.98399999999967	41.4699273659855\\
4.98599999999967	41.4616350771709\\
4.98799999999967	41.453344446308\\
4.98999999999967	41.4450554730663\\
4.99199999999967	41.4367681571151\\
4.99399999999967	41.428482498124\\
4.99599999999967	41.4201984957624\\
4.99799999999967	41.4119161496999\\
4.99999999999967	41.4036354596061\\
5.00199999999967	41.3953564251505\\
5.00399999999967	41.3870790460033\\
5.00599999999967	41.3788033218339\\
5.00799999999967	41.3705292523127\\
5.00999999999967	41.3622568371091\\
5.01199999999967	41.3539860758931\\
5.01399999999967	41.3457169683352\\
5.01599999999967	41.3374495141052\\
5.01799999999967	41.3291837128733\\
5.01999999999967	41.3209195643097\\
5.02199999999967	41.3126570680851\\
5.02399999999967	41.3043962238694\\
5.02599999999967	41.2961370313334\\
5.02799999999967	41.2878794901477\\
5.02999999999967	41.2796235999819\\
5.03199999999967	41.2713693605079\\
5.03399999999967	41.2631167713957\\
5.03599999999967	41.2548658323161\\
5.03799999999967	41.2466165429398\\
5.03999999999967	41.2383689029378\\
5.04199999999967	41.230122911981\\
5.04399999999967	41.2218785697409\\
5.04599999999967	41.2136358758881\\
5.04799999999967	41.2053948300934\\
5.04999999999967	41.1971554320282\\
5.05199999999967	41.1889176813642\\
5.05399999999967	41.1806815777722\\
5.05599999999967	41.1724471209236\\
5.05799999999966	41.1642143104901\\
5.05999999999966	41.155983146143\\
5.06199999999966	41.1477536275538\\
5.06399999999966	41.139525754394\\
5.06599999999966	41.1312995263358\\
5.06799999999966	41.1230749430506\\
5.06999999999966	41.1148520042098\\
5.07199999999966	41.1066307094865\\
5.07399999999966	41.0984110585512\\
5.07599999999966	41.0901930510766\\
5.07799999999966	41.0819766867344\\
5.07999999999966	41.0737619651973\\
5.08199999999966	41.0655488861371\\
5.08399999999966	41.0573374492259\\
5.08599999999966	41.0491276541363\\
5.08799999999966	41.0409195005399\\
5.08999999999966	41.0327129881106\\
5.09199999999966	41.0245081165194\\
5.09399999999966	41.0163048854399\\
5.09599999999966	41.008103294544\\
5.09799999999966	40.9999033435049\\
5.09999999999966	40.9917050319948\\
5.10199999999966	40.9835083596867\\
5.10399999999966	40.9753133262538\\
5.10599999999966	40.9671199313685\\
5.10799999999966	40.9589281747041\\
5.10999999999966	40.9507380559338\\
5.11199999999966	40.9425495747302\\
5.11399999999966	40.9343627307667\\
5.11599999999966	40.9261775237167\\
5.11799999999966	40.9179939532538\\
5.11999999999966	40.9098120190501\\
5.12199999999966	40.9016317207802\\
5.12399999999966	40.8934530581174\\
5.12599999999966	40.8852760307352\\
5.12799999999966	40.8771006383064\\
5.12999999999966	40.8689268805057\\
5.13199999999966	40.8607547570065\\
5.13399999999966	40.8525842674825\\
5.13599999999966	40.8444154116075\\
5.13799999999966	40.8362481890558\\
5.13999999999966	40.8280825995008\\
5.14199999999966	40.8199186426171\\
5.14399999999966	40.8117563180784\\
5.14599999999966	40.8035956255587\\
5.14799999999966	40.7954365647328\\
5.14999999999965	40.7872791352744\\
5.15199999999965	40.7791233368591\\
5.15399999999965	40.7709691691596\\
5.15599999999965	40.7628166318516\\
5.15799999999965	40.7546657246086\\
5.15999999999965	40.7465164471062\\
5.16199999999965	40.7383687990188\\
5.16399999999965	40.7302227800208\\
5.16599999999965	40.7220783897873\\
5.16799999999965	40.7139356279924\\
5.16999999999965	40.7057944943123\\
5.17199999999965	40.6976549884212\\
5.17399999999965	40.6895171099941\\
5.17599999999965	40.6813808587062\\
5.17799999999965	40.6732462342331\\
5.17999999999965	40.6651132362496\\
5.18199999999965	40.6569818644303\\
5.18399999999965	40.6488521184521\\
5.18599999999965	40.6407239979893\\
5.18799999999965	40.6325975027178\\
5.18999999999965	40.6244726323123\\
5.19199999999965	40.6163493864499\\
5.19399999999965	40.6082277648047\\
5.19599999999965	40.6001077670539\\
5.19799999999965	40.5919893928722\\
5.19999999999965	40.5838726419352\\
5.20199999999965	40.5757575139201\\
5.20399999999965	40.5676440085026\\
5.20599999999965	40.5595321253578\\
5.20799999999965	40.5514218641622\\
5.20999999999965	40.5433132245922\\
5.21199999999965	40.535206206324\\
5.21399999999965	40.5271008090337\\
5.21599999999965	40.5189970323974\\
5.21799999999965	40.5108948760923\\
5.21999999999965	40.5027943397937\\
5.22199999999965	40.4946954231797\\
5.22399999999965	40.4865981259254\\
5.22599999999965	40.4785024477087\\
5.22799999999965	40.4704083882053\\
5.22999999999965	40.4623159470926\\
5.23199999999965	40.4542251240469\\
5.23399999999965	40.446135918746\\
5.23599999999965	40.4380483308658\\
5.23799999999965	40.429962360084\\
5.23999999999964	40.4218780060773\\
5.24199999999964	40.4137952685233\\
5.24399999999964	40.4057141470991\\
5.24599999999964	40.3976346414818\\
5.24799999999964	40.3895567513485\\
5.24999999999964	40.3814804763777\\
5.25199999999964	40.3734058162454\\
5.25399999999964	40.3653327706303\\
5.25599999999964	40.3572613392091\\
5.25799999999964	40.3491915216604\\
5.25999999999964	40.3411233176609\\
5.26199999999964	40.3330567268894\\
5.26399999999964	40.3249917490229\\
5.26599999999964	40.3169283837397\\
5.26799999999964	40.3088666307176\\
5.26999999999964	40.3008064896346\\
5.27199999999964	40.2927479601689\\
5.27399999999964	40.2846910419992\\
5.27599999999964	40.2766357348029\\
5.27799999999964	40.2685820382586\\
5.27999999999964	40.2605299520448\\
5.28199999999964	40.2524794758391\\
5.28399999999964	40.244430609321\\
5.28599999999964	40.2363833521686\\
5.28799999999964	40.2283377040599\\
5.28999999999964	40.2202936646748\\
5.29199999999964	40.2122512336914\\
5.29399999999964	40.2042104107882\\
5.29599999999964	40.1961711956444\\
5.29799999999964	40.1881335879385\\
5.29999999999964	40.1800975873499\\
5.30199999999964	40.1720631935578\\
5.30399999999964	40.1640304062404\\
5.30599999999964	40.155999225078\\
5.30799999999964	40.1479696497484\\
5.30999999999964	40.1399416799324\\
5.31199999999964	40.1319153153082\\
5.31399999999964	40.123890555556\\
5.31599999999964	40.1158674003548\\
5.31799999999964	40.107845849384\\
5.31999999999964	40.0998259023235\\
5.32199999999964	40.0918075588528\\
5.32399999999964	40.0837908186515\\
5.32599999999964	40.0757756813998\\
5.32799999999964	40.0677621467776\\
5.32999999999964	40.0597502144636\\
5.33199999999963	40.0517398841392\\
5.33399999999963	40.0437311554837\\
5.33599999999963	40.0357240281777\\
5.33799999999963	40.0277185019004\\
5.33999999999963	40.0197145763326\\
5.34199999999963	40.011712251155\\
5.34399999999963	40.0037115260473\\
5.34599999999963	39.9957124006902\\
5.34799999999963	39.9877148747637\\
5.34999999999963	39.9797189479489\\
5.35199999999963	39.9717246199264\\
5.35399999999963	39.9637318903762\\
5.35599999999963	39.9557407589794\\
5.35799999999963	39.9477512254168\\
5.35999999999963	39.9397632893693\\
5.36199999999963	39.9317769505177\\
5.36399999999963	39.9237922085431\\
5.36599999999963	39.915809063126\\
5.36799999999963	39.9078275139482\\
5.36999999999963	39.8998475606901\\
5.37199999999963	39.8918692030335\\
5.37399999999963	39.8838924406596\\
5.37599999999963	39.8759172732491\\
5.37799999999963	39.8679437004843\\
5.37999999999963	39.8599717220462\\
5.38199999999963	39.8520013376164\\
5.38399999999963	39.8440325468763\\
5.38599999999963	39.8360653495075\\
5.38799999999963	39.8280997451922\\
5.38999999999963	39.8201357336118\\
5.39199999999963	39.8121733144478\\
5.39399999999963	39.8042124873828\\
5.39599999999963	39.796253252098\\
5.39799999999963	39.7882956082759\\
5.39999999999963	39.7803395555986\\
5.40199999999963	39.7723850937481\\
5.40399999999963	39.7644322224063\\
5.40599999999963	39.7564809412561\\
5.40799999999963	39.7485312499797\\
5.40999999999963	39.7405831482587\\
5.41199999999963	39.7326366357767\\
5.41399999999963	39.7246917122153\\
5.41599999999963	39.7167483772575\\
5.41799999999963	39.7088066305854\\
5.41999999999963	39.7008664718828\\
5.42199999999962	39.6929279008313\\
5.42399999999962	39.6849909171145\\
5.42599999999962	39.6770555204147\\
5.42799999999962	39.6691217104154\\
5.42999999999962	39.6611894867994\\
5.43199999999962	39.6532588492494\\
5.43399999999962	39.6453297974488\\
5.43599999999962	39.6374023310809\\
5.43799999999962	39.6294764498292\\
5.43999999999962	39.6215521533764\\
5.44199999999962	39.613629441406\\
5.44399999999962	39.6057083136021\\
5.44599999999962	39.5977887696473\\
5.44799999999962	39.5898708092256\\
5.44999999999962	39.5819544320205\\
5.45199999999962	39.5740396377159\\
5.45399999999962	39.5661264259956\\
5.45599999999962	39.5582147965431\\
5.45799999999962	39.5503047490424\\
5.45999999999962	39.5423962831775\\
5.46199999999962	39.5344893986324\\
5.46399999999962	39.5265840950909\\
5.46599999999962	39.518680372237\\
5.46799999999962	39.5107782297561\\
5.46999999999962	39.5028776673308\\
5.47199999999962	39.4949786846467\\
5.47399999999962	39.487081281387\\
5.47599999999962	39.4791854572371\\
5.47799999999962	39.471291211881\\
5.47999999999962	39.4633985450036\\
5.48199999999962	39.4555074562892\\
5.48399999999962	39.4476179454226\\
5.48599999999962	39.4397300120884\\
5.48799999999962	39.431843655972\\
5.48999999999962	39.4239588767573\\
5.49199999999962	39.41607567413\\
5.49399999999962	39.4081940477743\\
5.49599999999962	39.4003139973765\\
5.49799999999962	39.3924355226208\\
5.49999999999962	39.3845586231922\\
5.50199999999962	39.3766832987768\\
5.50399999999962	39.3688095490588\\
5.50599999999962	39.360937373724\\
5.50799999999962	39.3530667724585\\
5.50999999999962	39.3451977449471\\
5.51199999999961	39.3373302908751\\
5.51399999999961	39.329464409929\\
5.51599999999961	39.3216001017935\\
5.51799999999961	39.3137373661548\\
5.51999999999961	39.305876202699\\
5.52199999999961	39.2980166111114\\
5.52399999999961	39.2901585910785\\
5.52599999999961	39.2823021422854\\
5.52799999999961	39.2744472644186\\
5.52999999999961	39.2665939571646\\
5.53199999999961	39.2587422202089\\
5.53399999999961	39.2508920532385\\
5.53599999999961	39.2430434559385\\
5.53799999999961	39.2351964279965\\
5.53999999999961	39.2273509690984\\
5.54199999999961	39.2195070789303\\
5.54399999999961	39.2116647571794\\
5.54599999999961	39.2038240035313\\
5.54799999999961	39.1959848176738\\
5.54999999999961	39.1881471992931\\
5.55199999999961	39.1803111480757\\
5.55399999999961	39.1724766637094\\
5.55599999999961	39.1646437458799\\
5.55799999999961	39.156812394275\\
5.55999999999961	39.1489826085812\\
5.56199999999961	39.1411543884859\\
5.56399999999961	39.1333277336759\\
5.56599999999961	39.125502643839\\
5.56799999999961	39.1176791186621\\
5.56999999999961	39.1098571578325\\
5.57199999999961	39.1020367610372\\
5.57399999999961	39.0942179279647\\
5.57599999999961	39.0864006583017\\
5.57799999999961	39.0785849517365\\
5.57999999999961	39.0707708079558\\
5.58199999999961	39.0629582266478\\
5.58399999999961	39.0551472075\\
5.58599999999961	39.0473377502008\\
5.58799999999961	39.0395298544376\\
5.58999999999961	39.0317235198984\\
5.59199999999961	39.0239187462712\\
5.59399999999961	39.0161155332443\\
5.59599999999961	39.0083138805055\\
5.59799999999961	39.0005137877433\\
5.59999999999961	38.9927152546463\\
5.60199999999961	38.9849182809012\\
5.6039999999996	38.9771228661987\\
5.6059999999996	38.9693290102254\\
5.6079999999996	38.9615367126708\\
5.6099999999996	38.9537459732229\\
5.6119999999996	38.9459567915707\\
5.6139999999996	38.9381691674027\\
5.6159999999996	38.9303831004074\\
5.6179999999996	38.922598590274\\
5.6199999999996	38.9148156366911\\
5.6219999999996	38.9070342393484\\
5.6239999999996	38.8992543979336\\
5.6259999999996	38.8914761121366\\
5.6279999999996	38.8836993816459\\
5.6299999999996	38.8759242061524\\
5.6319999999996	38.8681505853436\\
5.6339999999996	38.8603785189085\\
5.6359999999996	38.8526080065377\\
5.6379999999996	38.8448390479203\\
5.6399999999996	38.8370716427452\\
5.6419999999996	38.8293057907026\\
5.6439999999996	38.8215414914817\\
5.6459999999996	38.8137787447726\\
5.6479999999996	38.8060175502643\\
5.6499999999996	38.7982579076476\\
5.6519999999996	38.7904998166113\\
5.6539999999996	38.7827432768463\\
5.6559999999996	38.7749882880418\\
5.6579999999996	38.7672348498879\\
5.6599999999996	38.7594829620749\\
5.6619999999996	38.7517326242929\\
5.6639999999996	38.7439838362329\\
5.6659999999996	38.7362365975841\\
5.6679999999996	38.7284909080372\\
5.6699999999996	38.7207467672823\\
5.6719999999996	38.7130041750105\\
5.6739999999996	38.7052631309113\\
5.6759999999996	38.6975236346768\\
5.6779999999996	38.6897856859967\\
5.6799999999996	38.6820492845613\\
5.6819999999996	38.6743144300621\\
5.6839999999996	38.6665811221896\\
5.6859999999996	38.6588493606349\\
5.6879999999996	38.6511191450892\\
5.6899999999996	38.6433904752426\\
5.6919999999996	38.6356633507873\\
5.69399999999959	38.6279377714135\\
5.69599999999959	38.6202137368125\\
5.69799999999959	38.6124912466762\\
5.69999999999959	38.6047703006957\\
5.70199999999959	38.5970508985618\\
5.70399999999959	38.5893330399668\\
5.70599999999959	38.5816167246014\\
5.70799999999959	38.5739019521575\\
5.70999999999959	38.5661887223267\\
5.71199999999959	38.5584770348008\\
5.71399999999959	38.5507668892713\\
5.71599999999959	38.5430582854304\\
5.71799999999959	38.5353512229699\\
5.71999999999959	38.5276457015811\\
5.72199999999959	38.5199417209572\\
5.72399999999959	38.5122392807888\\
5.72599999999959	38.5045383807689\\
5.72799999999959	38.4968390205894\\
5.72999999999959	38.4891411999427\\
5.73199999999959	38.481444918521\\
5.73399999999959	38.4737501760166\\
5.73599999999959	38.4660569721219\\
5.73799999999959	38.4583653065296\\
5.73999999999959	38.450675178932\\
5.74199999999959	38.4429865890215\\
5.74399999999959	38.4352995364916\\
5.74599999999959	38.4276140210342\\
5.74799999999959	38.4199300423428\\
5.74999999999959	38.4122476001088\\
5.75199999999959	38.4045666940266\\
5.75399999999959	38.3968873237887\\
5.75599999999959	38.3892094890875\\
5.75799999999959	38.3815331896174\\
5.75999999999959	38.3738584250703\\
5.76199999999959	38.3661851951404\\
5.76399999999959	38.35851349952\\
5.76599999999959	38.350843337903\\
5.76799999999959	38.3431747099828\\
5.76999999999959	38.3355076154527\\
5.77199999999959	38.3278420540058\\
5.77399999999959	38.3201780253365\\
5.77599999999959	38.3125155291379\\
5.77799999999959	38.3048545651039\\
5.77999999999959	38.2971951329281\\
5.78199999999959	38.2895372323042\\
5.78399999999959	38.2818808629266\\
5.78599999999958	38.2742260244887\\
5.78799999999958	38.2665727166847\\
5.78999999999958	38.2589209392091\\
5.79199999999958	38.2512706917548\\
5.79399999999958	38.2436219740172\\
5.79599999999958	38.2359747856899\\
5.79799999999958	38.2283291264672\\
5.79999999999958	38.2206849960438\\
5.80199999999958	38.2130423941136\\
5.80399999999958	38.205401320372\\
5.80599999999958	38.1977617745124\\
5.80799999999958	38.1901237562304\\
5.80999999999958	38.1824872652202\\
5.81199999999958	38.1748523011758\\
5.81399999999958	38.1672188637935\\
5.81599999999958	38.1595869527669\\
5.81799999999958	38.151956567792\\
5.81999999999958	38.1443277085629\\
5.82199999999958	38.1367003747743\\
5.82399999999958	38.1290745661221\\
5.82599999999958	38.1214502823013\\
5.82799999999958	38.1138275230069\\
5.82999999999958	38.1062062879344\\
5.83199999999958	38.0985865767789\\
5.83399999999958	38.0909683892358\\
5.83599999999958	38.0833517250007\\
5.83799999999958	38.0757365837691\\
5.83999999999958	38.0681229652361\\
5.84199999999958	38.0605108690981\\
5.84399999999958	38.0529002950505\\
5.84599999999958	38.0452912427885\\
5.84799999999958	38.0376837120082\\
5.84999999999958	38.0300777024059\\
5.85199999999958	38.0224732136774\\
5.85399999999958	38.014870245518\\
5.85599999999958	38.007268797625\\
5.85799999999958	37.9996688696932\\
5.85999999999958	37.9920704614198\\
5.86199999999958	37.9844735725009\\
5.86399999999958	37.9768782026316\\
5.86599999999958	37.9692843515098\\
5.86799999999958	37.9616920188312\\
5.86999999999958	37.954101204292\\
5.87199999999958	37.9465119075893\\
5.87399999999958	37.9389241284195\\
5.87599999999957	37.9313378664791\\
5.87799999999957	37.9237531214654\\
5.87999999999957	37.9161698930739\\
5.88199999999957	37.9085881810024\\
5.88399999999957	37.901007984948\\
5.88599999999957	37.8934293046071\\
5.88799999999957	37.8858521396767\\
5.88999999999957	37.8782764898541\\
5.89199999999957	37.8707023548367\\
5.89399999999957	37.863129734321\\
5.89599999999957	37.8555586280049\\
5.89799999999957	37.8479890355853\\
5.89999999999957	37.8404209567598\\
5.90199999999957	37.832854391226\\
5.90399999999957	37.8252893386806\\
5.90599999999957	37.8177257988224\\
5.90799999999957	37.8101637713483\\
5.90999999999957	37.8026032559554\\
5.91199999999957	37.7950442523421\\
5.91399999999957	37.7874867602067\\
5.91599999999957	37.7799307792458\\
5.91799999999957	37.7723763091586\\
5.91999999999957	37.7648233496423\\
5.92199999999957	37.7572719003954\\
5.92399999999957	37.7497219611154\\
5.92599999999957	37.7421735315008\\
5.92799999999957	37.7346266112497\\
5.92999999999957	37.7270812000604\\
5.93199999999957	37.7195372976316\\
5.93399999999957	37.7119949036612\\
5.93599999999957	37.704454017848\\
5.93799999999957	37.69691463989\\
5.93999999999957	37.6893767694864\\
5.94199999999957	37.6818404063354\\
5.94399999999957	37.674305550136\\
5.94599999999957	37.6667722005871\\
5.94799999999957	37.6592403573867\\
5.94999999999957	37.6517100202344\\
5.95199999999957	37.6441811888288\\
5.95399999999957	37.6366538628696\\
5.95599999999957	37.6291280420544\\
5.95799999999957	37.621603726084\\
5.95999999999957	37.614080914656\\
5.96199999999957	37.6065596074714\\
5.96399999999957	37.5990398042276\\
5.96599999999956	37.5915215046252\\
5.96799999999956	37.5840047083634\\
5.96999999999956	37.5764894151416\\
5.97199999999956	37.5689756246588\\
5.97399999999956	37.5614633366153\\
5.97599999999956	37.5539525507101\\
5.97799999999956	37.5464432666438\\
5.97999999999956	37.5389354841156\\
5.98199999999956	37.5314292028251\\
5.98399999999956	37.5239244224732\\
5.98599999999956	37.5164211427591\\
5.98799999999956	37.5089193633823\\
5.98999999999956	37.5014190840437\\
5.99199999999956	37.493920304443\\
5.99399999999956	37.4864230242808\\
5.99599999999956	37.478927243257\\
5.99799999999956	37.4714329610718\\
5.99999999999956	37.463940177426\\
6.00199999999956	37.4564488920197\\
6.00399999999956	37.4489591045536\\
6.00599999999956	37.441470814728\\
6.00799999999956	37.4339840222435\\
6.00999999999956	37.4264987268013\\
6.01199999999956	37.4190149281014\\
6.01399999999956	37.411532625845\\
6.01599999999956	37.404051819733\\
6.01799999999956	37.3965725094661\\
6.01999999999956	37.3890946947451\\
6.02199999999956	37.3816183752712\\
6.02399999999956	37.3741435507456\\
6.02599999999956	37.3666702208697\\
6.02799999999956	37.3591983853439\\
6.02999999999956	37.3517280438703\\
6.03199999999956	37.3442591961494\\
6.03399999999956	37.3367918418832\\
6.03599999999956	37.3293259807733\\
6.03799999999956	37.3218616125205\\
6.03999999999956	37.3143987368268\\
6.04199999999956	37.3069373533935\\
6.04399999999956	37.2994774619224\\
6.04599999999956	37.2920190621155\\
6.04799999999956	37.2845621536749\\
6.04999999999956	37.2771067363014\\
6.05199999999956	37.2696528096978\\
6.05399999999956	37.2622003735656\\
6.05599999999956	37.2547494276071\\
6.05799999999955	37.2472999715239\\
6.05999999999955	37.2398520050187\\
6.06199999999955	37.2324055277937\\
6.06399999999955	37.2249605395514\\
6.06599999999955	37.2175170399932\\
6.06799999999955	37.2100750288221\\
6.06999999999955	37.2026345057402\\
6.07199999999955	37.1951954704506\\
6.07399999999955	37.1877579226555\\
6.07599999999955	37.1803218620577\\
6.07799999999955	37.1728872883595\\
6.07999999999955	37.1654542012643\\
6.08199999999955	37.1580226004741\\
6.08399999999955	37.1505924856923\\
6.08599999999955	37.1431638566216\\
6.08799999999955	37.1357367129649\\
6.08999999999955	37.1283110544255\\
6.09199999999955	37.1208868807061\\
6.09399999999955	37.1134641915107\\
6.09599999999955	37.1060429865418\\
6.09799999999955	37.0986232655027\\
6.09999999999955	37.0912050280966\\
6.10199999999955	37.0837882740277\\
6.10399999999955	37.0763730029981\\
6.10599999999955	37.0689592147127\\
6.10799999999955	37.0615469088745\\
6.10999999999955	37.0541360851871\\
6.11199999999955	37.0467267433541\\
6.11399999999955	37.0393188830792\\
6.11599999999955	37.0319125040669\\
6.11799999999955	37.0245076060198\\
6.11999999999955	37.0171041886427\\
6.12199999999955	37.0097022516395\\
6.12399999999955	37.0023017947143\\
6.12599999999955	36.9949028175707\\
6.12799999999955	36.9875053199136\\
6.12999999999955	36.9801093014466\\
6.13199999999955	36.972714761874\\
6.13399999999955	36.9653217009011\\
6.13599999999955	36.9579301182311\\
6.13799999999955	36.9505400135686\\
6.13999999999955	36.9431513866185\\
6.14199999999955	36.9357642370857\\
6.14399999999955	36.9283785646742\\
6.14599999999955	36.9209943690886\\
6.14799999999954	36.9136116500343\\
6.14999999999954	36.9062304072155\\
6.15199999999954	36.8988506403378\\
6.15399999999954	36.891472349105\\
6.15599999999954	36.8840955332231\\
6.15799999999954	36.8767201923968\\
6.15999999999954	36.869346326331\\
6.16199999999954	36.8619739347307\\
6.16399999999954	36.8546030173019\\
6.16599999999954	36.8472335737494\\
6.16799999999954	36.8398656037782\\
6.16999999999954	36.8324991070945\\
6.17199999999954	36.8251340834029\\
6.17399999999954	36.8177705324095\\
6.17599999999954	36.8104084538195\\
6.17799999999954	36.8030478473389\\
6.17999999999954	36.7956887126731\\
6.18199999999954	36.7883310495278\\
6.18399999999954	36.7809748576093\\
6.18599999999954	36.7736201366226\\
6.18799999999954	36.7662668862746\\
6.18999999999954	36.7589151062708\\
6.19199999999954	36.7515647963172\\
6.19399999999954	36.7442159561199\\
6.19599999999954	36.7368685853847\\
6.19799999999954	36.7295226838185\\
6.19999999999954	36.7221782511275\\
6.20199999999954	36.7148352870174\\
6.20399999999954	36.7074937911951\\
6.20599999999954	36.7001537633674\\
6.20799999999954	36.6928152032399\\
6.20999999999954	36.6854781105197\\
6.21199999999954	36.6781424849135\\
6.21399999999954	36.6708083261277\\
6.21599999999954	36.6634756338693\\
6.21799999999954	36.6561444078449\\
6.21999999999954	36.6488146477614\\
6.22199999999954	36.6414863533258\\
6.22399999999954	36.6341595242448\\
6.22599999999954	36.6268341602262\\
6.22799999999954	36.619510260976\\
6.22999999999954	36.6121878262017\\
6.23199999999954	36.6048668556114\\
6.23399999999954	36.597547348911\\
6.23599999999954	36.590229305809\\
6.23799999999954	36.5829127260123\\
6.23999999999953	36.5755976092282\\
6.24199999999953	36.5682839551639\\
6.24399999999953	36.5609717635279\\
6.24599999999953	36.5536610340272\\
6.24799999999953	36.5463517663694\\
6.24999999999953	36.5390439602623\\
6.25199999999953	36.5317376154142\\
6.25399999999953	36.5244327315323\\
6.25599999999953	36.5171293083244\\
6.25799999999953	36.5098273454992\\
6.25999999999953	36.5025268427646\\
6.26199999999953	36.4952277998282\\
6.26399999999953	36.4879302163983\\
6.26599999999953	36.480634092183\\
6.26799999999953	36.4733394268906\\
6.26999999999953	36.4660462202297\\
6.27199999999953	36.4587544719087\\
6.27399999999953	36.4514641816357\\
6.27599999999953	36.4441753491194\\
6.27799999999953	36.4368879740678\\
6.27999999999953	36.4296020561903\\
6.28199999999953	36.4223175951957\\
6.28399999999953	36.4150345907916\\
6.28599999999953	36.4077530426875\\
6.28799999999953	36.4004729505928\\
6.28999999999953	36.3931943142151\\
6.29199999999953	36.3859171332645\\
6.29399999999953	36.378641407449\\
6.29599999999953	36.3713671364786\\
6.29799999999953	36.3640943200618\\
6.29999999999953	36.3568229579083\\
6.30199999999953	36.3495530497269\\
6.30399999999953	36.342284595227\\
6.30599999999953	36.3350175941183\\
6.30799999999953	36.3277520461097\\
6.30999999999953	36.320487950911\\
6.31199999999953	36.3132253082313\\
6.31399999999953	36.3059641177815\\
6.31599999999953	36.2987043792697\\
6.31799999999953	36.291446092406\\
6.31999999999953	36.2841892569008\\
6.32199999999953	36.2769338724627\\
6.32399999999953	36.2696799388032\\
6.32599999999953	36.262427455631\\
6.32799999999953	36.2551764226567\\
6.32999999999952	36.2479268395899\\
6.33199999999952	36.2406787061413\\
6.33399999999952	36.2334320220206\\
6.33599999999952	36.2261867869383\\
6.33799999999952	36.2189430006044\\
6.33999999999952	36.2117006627296\\
6.34199999999952	36.2044597730237\\
6.34399999999952	36.1972203311979\\
6.34599999999952	36.1899823369625\\
6.34799999999952	36.1827457900274\\
6.34999999999952	36.1755106901042\\
6.35199999999952	36.1682770369029\\
6.35399999999952	36.1610448301353\\
6.35599999999952	36.1538140695106\\
6.35799999999952	36.1465847547405\\
6.35999999999952	36.1393568855364\\
6.36199999999952	36.1321304616085\\
6.36399999999952	36.1249054826682\\
6.36599999999952	36.1176819484267\\
6.36799999999952	36.110459858595\\
6.36999999999952	36.1032392128838\\
6.37199999999952	36.0960200110051\\
6.37399999999952	36.0888022526699\\
6.37599999999952	36.0815859375899\\
6.37799999999952	36.074371065476\\
6.37999999999952	36.0671576360403\\
6.38199999999952	36.0599456489936\\
6.38399999999952	36.0527351040483\\
6.38599999999952	36.0455260009155\\
6.38799999999952	36.0383183393076\\
6.38999999999952	36.0311121189356\\
6.39199999999952	36.0239073395114\\
6.39399999999952	36.0167040007478\\
6.39599999999952	36.0095021023556\\
6.39799999999952	36.0023016440477\\
6.39999999999952	35.9951026255357\\
6.40199999999952	35.9879050465317\\
6.40399999999952	35.9807089067487\\
6.40599999999952	35.9735142058976\\
6.40799999999952	35.9663209436918\\
6.40999999999952	35.9591291198426\\
6.41199999999952	35.9519387340632\\
6.41399999999952	35.9447497860667\\
6.41599999999952	35.9375622755641\\
6.41799999999952	35.9303762022693\\
6.41999999999951	35.9231915658937\\
6.42199999999951	35.9160083661514\\
6.42399999999951	35.9088266027539\\
6.42599999999951	35.9016462754151\\
6.42799999999951	35.8944673838469\\
6.42999999999951	35.8872899277628\\
6.43199999999951	35.880113906876\\
6.43399999999951	35.8729393208991\\
6.43599999999951	35.8657661695452\\
6.43799999999951	35.8585944525275\\
6.43999999999951	35.8514241695595\\
6.44199999999951	35.8442553203541\\
6.44399999999951	35.837087904625\\
6.44599999999951	35.8299219220852\\
6.44799999999951	35.8227573724484\\
6.44999999999951	35.8155942554283\\
6.45199999999951	35.808432570738\\
6.45399999999951	35.8012723180916\\
6.45599999999951	35.7941134972022\\
6.45799999999951	35.7869561077842\\
6.45999999999951	35.7798001495508\\
6.46199999999951	35.7726456222158\\
6.46399999999951	35.7654925254936\\
6.46599999999951	35.7583408590986\\
6.46799999999951	35.7511906227431\\
6.46999999999951	35.7440418161428\\
6.47199999999951	35.7368944390113\\
6.47399999999951	35.7297484910627\\
6.47599999999951	35.7226039720115\\
6.47799999999951	35.7154608815713\\
6.47999999999951	35.7083192194576\\
6.48199999999951	35.7011789853836\\
6.48399999999951	35.6940401790652\\
6.48599999999951	35.6869028002157\\
6.48799999999951	35.6797668485494\\
6.48999999999951	35.6726323237825\\
6.49199999999951	35.6654992256285\\
6.49399999999951	35.6583675538029\\
6.49599999999951	35.6512373080194\\
6.49799999999951	35.6441084879943\\
6.49999999999951	35.6369810934414\\
6.50199999999951	35.6298551240762\\
6.50399999999951	35.6227305796139\\
6.50599999999951	35.615607459769\\
6.50799999999951	35.6084857642573\\
6.50999999999951	35.6013654927933\\
6.5119999999995	35.5942466450932\\
6.5139999999995	35.5871292208719\\
6.5159999999995	35.5800132198444\\
6.5179999999995	35.5728986417266\\
6.5199999999995	35.5657854862339\\
6.5219999999995	35.5586737530814\\
6.5239999999995	35.5515634419854\\
6.5259999999995	35.5444545526617\\
6.5279999999995	35.5373470848252\\
6.5299999999995	35.5302410381921\\
6.5319999999995	35.5231364124784\\
6.5339999999995	35.5160332073998\\
6.5359999999995	35.5089314226721\\
6.5379999999995	35.5018310580116\\
6.5399999999995	35.4947321131344\\
6.5419999999995	35.4876345877566\\
6.5439999999995	35.4805384815938\\
6.5459999999995	35.4734437943633\\
6.5479999999995	35.4663505257803\\
6.5499999999995	35.4592586755622\\
6.5519999999995	35.4521682434247\\
6.5539999999995	35.4450792290841\\
6.5559999999995	35.4379916322575\\
6.5579999999995	35.4309054526609\\
6.5599999999995	35.4238206900116\\
6.5619999999995	35.4167373440256\\
6.5639999999995	35.4096554144204\\
6.5659999999995	35.4025749009122\\
6.5679999999995	35.395495803218\\
6.5699999999995	35.3884181210549\\
6.5719999999995	35.3813418541396\\
6.5739999999995	35.3742670021892\\
6.5759999999995	35.3671935649211\\
6.5779999999995	35.3601215420522\\
6.5799999999995	35.3530509332991\\
6.5819999999995	35.3459817383804\\
6.5839999999995	35.3389139570123\\
6.5859999999995	35.3318475889127\\
6.5879999999995	35.3247826337984\\
6.5899999999995	35.3177190913879\\
6.5919999999995	35.3106569613979\\
6.5939999999995	35.3035962435465\\
6.5959999999995	35.2965369375505\\
6.5979999999995	35.2894790431286\\
6.5999999999995	35.2824225599985\\
6.60199999999949	35.275367487877\\
6.60399999999949	35.2683138264832\\
6.60599999999949	35.261261575534\\
6.60799999999949	35.2542107347483\\
6.60999999999949	35.2471613038437\\
6.61199999999949	35.2401132825382\\
6.61399999999949	35.2330666705501\\
6.61599999999949	35.2260214675977\\
6.61799999999949	35.2189776733991\\
6.61999999999949	35.2119352876725\\
6.62199999999949	35.2048943101371\\
6.62399999999949	35.1978547405099\\
6.62599999999949	35.1908165785106\\
6.62799999999949	35.1837798238574\\
6.62999999999949	35.1767444762689\\
6.63199999999949	35.1697105354635\\
6.63399999999949	35.1626780011601\\
6.63599999999949	35.155646873078\\
6.63799999999949	35.148617150935\\
6.63999999999949	35.141588834451\\
6.64199999999949	35.1345619233439\\
6.64399999999949	35.1275364173335\\
6.64599999999949	35.1205123161391\\
6.64799999999949	35.1134896194789\\
6.64999999999949	35.1064683270726\\
6.65199999999949	35.0994484386393\\
6.65399999999949	35.0924299538981\\
6.65599999999949	35.0854128725691\\
6.65799999999949	35.0783971943708\\
6.65999999999949	35.0713829190227\\
6.66199999999949	35.0643700462451\\
6.66399999999949	35.0573585757569\\
6.66599999999949	35.0503485072779\\
6.66799999999949	35.0433398405277\\
6.66999999999949	35.0363325752256\\
6.67199999999949	35.0293267110928\\
6.67399999999949	35.0223222478473\\
6.67599999999949	35.0153191852103\\
6.67799999999949	35.0083175229015\\
6.67999999999949	35.0013172606407\\
6.68199999999949	34.9943183981477\\
6.68399999999949	34.9873209351431\\
6.68599999999949	34.9803248713468\\
6.68799999999949	34.9733302064793\\
6.68999999999949	34.9663369402608\\
6.69199999999949	34.9593450724115\\
6.69399999999948	34.9523546026515\\
6.69599999999948	34.9453655307019\\
6.69799999999948	34.9383778562835\\
6.69999999999948	34.9313915791154\\
6.70199999999948	34.9244066989196\\
6.70399999999948	34.9174232154161\\
6.70599999999948	34.9104411283259\\
6.70799999999948	34.9034604373698\\
6.70999999999948	34.8964811422683\\
6.71199999999948	34.8895032427429\\
6.71399999999948	34.882526738514\\
6.71599999999948	34.8755516293029\\
6.71799999999948	34.8685779148308\\
6.71999999999948	34.8616055948186\\
6.72199999999948	34.8546346689871\\
6.72399999999948	34.8476651370585\\
6.72599999999948	34.8406969987532\\
6.72799999999948	34.8337302537931\\
6.72999999999948	34.826764901899\\
6.73199999999948	34.8198009427927\\
6.73399999999948	34.8128383761961\\
6.73599999999948	34.8058772018304\\
6.73799999999948	34.7989174194173\\
6.73999999999948	34.7919590286781\\
6.74199999999948	34.7850020293353\\
6.74399999999948	34.7780464211098\\
6.74599999999948	34.7710922037244\\
6.74799999999948	34.7641393769002\\
6.74999999999948	34.7571879403593\\
6.75199999999948	34.7502378938241\\
6.75399999999948	34.7432892370165\\
6.75599999999948	34.7363419696583\\
6.75799999999948	34.7293960914725\\
6.75999999999948	34.7224516021803\\
6.76199999999948	34.7155085015047\\
6.76399999999948	34.7085667891678\\
6.76599999999948	34.7016264648919\\
6.76799999999948	34.6946875283997\\
6.76999999999948	34.6877499794133\\
6.77199999999948	34.6808138176563\\
6.77399999999948	34.6738790428499\\
6.77599999999948	34.6669456547177\\
6.77799999999948	34.6600136529823\\
6.77999999999948	34.6530830373664\\
6.78199999999948	34.6461538075929\\
6.78399999999947	34.6392259633845\\
6.78599999999947	34.6322995044642\\
6.78799999999947	34.6253744305552\\
6.78999999999947	34.6184507413805\\
6.79199999999947	34.6115284366631\\
6.79399999999947	34.6046075161264\\
6.79599999999947	34.5976879794934\\
6.79799999999947	34.5907698264878\\
6.79999999999947	34.5838530568323\\
6.80199999999947	34.5769376702509\\
6.80399999999947	34.5700236664664\\
6.80599999999947	34.5631110452032\\
6.80799999999947	34.556199806184\\
6.80999999999947	34.5492899491332\\
6.81199999999947	34.5423814737738\\
6.81399999999947	34.5354743798296\\
6.81599999999947	34.5285686670247\\
6.81799999999947	34.5216643350829\\
6.81999999999947	34.5147613837279\\
6.82199999999947	34.5078598126839\\
6.82399999999947	34.5009596216747\\
6.82599999999947	34.4940608104247\\
6.82799999999947	34.4871633786575\\
6.82999999999947	34.4802673260973\\
6.83199999999947	34.4733726524688\\
6.83399999999947	34.4664793574957\\
6.83599999999947	34.4595874409034\\
6.83799999999947	34.452696902415\\
6.83999999999947	34.4458077417555\\
6.84199999999947	34.4389199586496\\
6.84399999999947	34.4320335528216\\
6.84599999999947	34.4251485239962\\
6.84799999999947	34.4182648718979\\
6.84999999999947	34.4113825962514\\
6.85199999999947	34.4045016967821\\
6.85399999999947	34.3976221732145\\
6.85599999999947	34.390744025273\\
6.85799999999947	34.3838672526828\\
6.85999999999947	34.376991855169\\
6.86199999999947	34.3701178324567\\
6.86399999999947	34.3632451842712\\
6.86599999999947	34.356373910337\\
6.86799999999947	34.3495040103799\\
6.86999999999947	34.3426354841251\\
6.87199999999947	34.3357683312978\\
6.87399999999946	34.3289025516234\\
6.87599999999946	34.3220381448272\\
6.87799999999946	34.3151751106352\\
6.87999999999946	34.3083134487721\\
6.88199999999946	34.3014531589643\\
6.88399999999946	34.2945942409368\\
6.88599999999946	34.2877366944163\\
6.88799999999946	34.2808805191273\\
6.88999999999946	34.2740257147967\\
6.89199999999946	34.2671722811497\\
6.89399999999946	34.2603202179122\\
6.89599999999946	34.2534695248108\\
6.89799999999946	34.246620201571\\
6.89999999999946	34.2397722479189\\
6.90199999999946	34.232925663581\\
6.90399999999946	34.226080448283\\
6.90599999999946	34.2192366017516\\
6.90799999999946	34.2123941237132\\
6.90999999999946	34.2055530138938\\
6.91199999999946	34.1987132720198\\
6.91399999999946	34.1918748978181\\
6.91599999999946	34.185037891015\\
6.91799999999946	34.1782022513368\\
6.91999999999946	34.1713679785106\\
6.92199999999946	34.1645350722629\\
6.92399999999946	34.1577035323202\\
6.92599999999946	34.1508733584096\\
6.92799999999946	34.1440445502581\\
6.92999999999946	34.1372171075922\\
6.93199999999946	34.1303910301393\\
6.93399999999946	34.1235663176261\\
6.93599999999946	34.1167429697797\\
6.93799999999946	34.1099209863273\\
6.93999999999946	34.1031003669961\\
6.94199999999946	34.0962811115136\\
6.94399999999946	34.0894632196065\\
6.94599999999946	34.0826466910028\\
6.94799999999946	34.0758315254292\\
6.94999999999946	34.0690177226136\\
6.95199999999946	34.0622052822836\\
6.95399999999946	34.0553942041669\\
6.95599999999946	34.0485844879904\\
6.95799999999946	34.0417761334825\\
6.95999999999946	34.0349691403709\\
6.96199999999946	34.0281635083823\\
6.96399999999946	34.021359237246\\
6.96599999999945	34.0145563266893\\
6.96799999999945	34.0077547764398\\
6.96999999999945	34.000954586226\\
6.97199999999945	33.9941557557757\\
6.97399999999945	33.9873582848172\\
6.97599999999945	33.9805621730783\\
6.97799999999945	33.9737674202878\\
6.97999999999945	33.9669740261737\\
6.98199999999945	33.9601819904642\\
6.98399999999945	33.9533913128874\\
6.98599999999945	33.9466019931726\\
6.98799999999945	33.9398140310476\\
6.98999999999945	33.933027426241\\
6.99199999999945	33.9262421784815\\
6.99399999999945	33.9194582874978\\
6.99599999999945	33.912675753019\\
6.99799999999945	33.9058945747729\\
6.99999999999945	33.899114752489\\
7.00199999999945	33.8923362858964\\
7.00399999999945	33.8855591747237\\
7.00599999999945	33.8787834186995\\
7.00799999999945	33.8720090175528\\
7.00999999999945	33.865235971014\\
7.01199999999945	33.8584642788109\\
7.01399999999945	33.8516939406731\\
7.01599999999945	33.8449249563298\\
7.01799999999945	33.8381573255104\\
7.01999999999945	33.8313910479445\\
7.02199999999945	33.8246261233608\\
7.02399999999945	33.8178625514892\\
7.02599999999945	33.8111003320594\\
7.02799999999945	33.8043394648007\\
7.02999999999945	33.797579949443\\
7.03199999999945	33.7908217857155\\
7.03399999999945	33.7840649733484\\
7.03599999999945	33.7773095120715\\
7.03799999999945	33.7705554016144\\
7.03999999999945	33.7638026417071\\
7.04199999999945	33.7570512320793\\
7.04399999999945	33.7503011724614\\
7.04599999999945	33.7435524625831\\
7.04799999999945	33.7368051021748\\
7.04999999999945	33.7300590909669\\
7.05199999999945	33.7233144286889\\
7.05399999999945	33.7165711150717\\
7.05599999999944	33.7098291498453\\
7.05799999999944	33.70308853274\\
7.05999999999944	33.6963492634869\\
7.06199999999944	33.6896113418156\\
7.06399999999944	33.682874767457\\
7.06599999999944	33.6761395401421\\
7.06799999999944	33.6694056596011\\
7.06999999999944	33.662673125565\\
7.07199999999944	33.6559419377638\\
7.07399999999944	33.6492120959293\\
7.07599999999944	33.6424835997921\\
7.07799999999944	33.6357564490826\\
7.07999999999944	33.6290306435325\\
7.08199999999944	33.6223061828723\\
7.08399999999944	33.6155830668333\\
7.08599999999944	33.6088612951468\\
7.08799999999944	33.6021408675436\\
7.08999999999944	33.5954217837551\\
7.09199999999944	33.5887040435128\\
7.09399999999944	33.5819876465476\\
7.09599999999944	33.5752725925917\\
7.09799999999944	33.5685588813756\\
7.09999999999944	33.5618465126313\\
7.10199999999944	33.5551354860908\\
7.10399999999944	33.5484258014851\\
7.10599999999944	33.5417174585459\\
7.10799999999944	33.5350104570049\\
7.10999999999944	33.5283047965941\\
7.11199999999944	33.5216004770454\\
7.11399999999944	33.5148974980908\\
7.11599999999944	33.5081958594617\\
7.11799999999944	33.5014955608903\\
7.11999999999944	33.494796602109\\
7.12199999999944	33.4880989828497\\
7.12399999999944	33.4814027028444\\
7.12599999999944	33.4747077618255\\
7.12799999999944	33.468014159525\\
7.12999999999944	33.4613218956757\\
7.13199999999944	33.4546309700095\\
7.13399999999944	33.4479413822589\\
7.13599999999944	33.4412531321565\\
7.13799999999944	33.4345662194348\\
7.13999999999944	33.4278806438266\\
7.14199999999944	33.4211964050642\\
7.14399999999944	33.4145135028804\\
7.14599999999944	33.4078319370078\\
7.14799999999943	33.4011517071795\\
7.14999999999943	33.3944728131286\\
7.15199999999943	33.387795254587\\
7.15399999999943	33.3811190312888\\
7.15599999999943	33.3744441429662\\
7.15799999999943	33.3677705893526\\
7.15999999999943	33.3610983701809\\
7.16199999999943	33.3544274851846\\
7.16399999999943	33.3477579340967\\
7.16599999999943	33.3410897166505\\
7.16799999999943	33.3344228325794\\
7.16999999999943	33.3277572816172\\
7.17199999999943	33.3210930634962\\
7.17399999999943	33.3144301779509\\
7.17599999999943	33.3077686247146\\
7.17799999999943	33.301108403521\\
7.17999999999943	33.2944495141032\\
7.18199999999943	33.2877919561954\\
7.18399999999943	33.2811357295314\\
7.18599999999943	33.2744808338448\\
7.18799999999943	33.2678272688693\\
7.18999999999943	33.2611750343393\\
7.19199999999943	33.2545241299885\\
7.19399999999943	33.2478745555507\\
7.19599999999943	33.2412263107606\\
7.19799999999943	33.2345793953515\\
7.19999999999943	33.2279338090582\\
7.20199999999943	33.2212895516145\\
7.20399999999943	33.2146466227554\\
7.20599999999943	33.2080050222147\\
7.20799999999943	33.201364749727\\
7.20999999999943	33.1947258050263\\
7.21199999999943	33.1880881878474\\
7.21399999999943	33.1814518979249\\
7.21599999999943	33.1748169349937\\
7.21799999999943	33.168183298788\\
7.21999999999943	33.1615509890426\\
7.22199999999943	33.1549200054924\\
7.22399999999943	33.1482903478721\\
7.22599999999943	33.1416620159163\\
7.22799999999943	33.135035009361\\
7.22999999999943	33.12840932794\\
7.23199999999943	33.1217849713889\\
7.23399999999943	33.1151619394426\\
7.23599999999943	33.1085402318363\\
7.23799999999942	33.101919848305\\
7.23999999999942	33.0953007885841\\
7.24199999999942	33.0886830524094\\
7.24399999999942	33.0820666395152\\
7.24599999999942	33.0754515496379\\
7.24799999999942	33.0688377825122\\
7.24999999999942	33.0622253378741\\
7.25199999999942	33.055614215459\\
7.25399999999942	33.0490044150026\\
7.25599999999942	33.0423959362401\\
7.25799999999942	33.0357887789079\\
7.25999999999942	33.0291829427417\\
7.26199999999942	33.0225784274766\\
7.26399999999942	33.0159752328491\\
7.26599999999942	33.0093733585952\\
7.26799999999942	33.0027728044505\\
7.26999999999942	32.9961735701512\\
7.27199999999942	32.9895756554335\\
7.27399999999942	32.9829790600335\\
7.27599999999942	32.9763837836873\\
7.27799999999942	32.969789826131\\
7.27999999999942	32.9631971871015\\
7.28199999999942	32.9566058663343\\
7.28399999999942	32.9500158635665\\
7.28599999999942	32.9434271785341\\
7.28799999999942	32.9368398109741\\
7.28999999999942	32.9302537606227\\
7.29199999999942	32.9236690272166\\
7.29399999999942	32.9170856104925\\
7.29599999999942	32.910503510187\\
7.29799999999942	32.9039227260369\\
7.29999999999942	32.8973432577791\\
7.30199999999942	32.8907651051505\\
7.30399999999942	32.8841882678881\\
7.30599999999942	32.8776127457287\\
7.30799999999942	32.8710385384095\\
7.30999999999942	32.8644656456673\\
7.31199999999942	32.8578940672397\\
7.31399999999942	32.8513238028635\\
7.31599999999942	32.8447548522761\\
7.31799999999942	32.8381872152148\\
7.31999999999942	32.8316208914169\\
7.32199999999942	32.8250558806202\\
7.32399999999942	32.8184921825615\\
7.32599999999942	32.8119297969787\\
7.32799999999941	32.8053687236089\\
7.32999999999941	32.7988089621906\\
7.33199999999941	32.7922505124606\\
7.33399999999941	32.7856933741573\\
7.33599999999941	32.779137547018\\
7.33799999999941	32.7725830307808\\
7.33999999999941	32.7660298251833\\
7.34199999999941	32.7594779299634\\
7.34399999999941	32.7529273448595\\
7.34599999999941	32.7463780696094\\
7.34799999999941	32.7398301039512\\
7.34999999999941	32.7332834476228\\
7.35199999999941	32.7267381003628\\
7.35399999999941	32.7201940619096\\
7.35599999999941	32.7136513320004\\
7.35799999999941	32.7071099103748\\
7.35999999999941	32.7005697967707\\
7.36199999999941	32.6940309909262\\
7.36399999999941	32.6874934925806\\
7.36599999999941	32.6809573014718\\
7.36799999999941	32.6744224173386\\
7.36999999999941	32.6678888399196\\
7.37199999999941	32.6613565689538\\
7.37399999999941	32.6548256041798\\
7.37599999999941	32.648295945336\\
7.37799999999941	32.6417675921621\\
7.37999999999941	32.6352405443963\\
7.38199999999941	32.628714801778\\
7.38399999999941	32.6221903640458\\
7.38599999999941	32.6156672309392\\
7.38799999999941	32.6091454021974\\
7.38999999999941	32.6026248775594\\
7.39199999999941	32.5961056567641\\
7.39399999999941	32.5895877395512\\
7.39599999999941	32.58307112566\\
7.39799999999941	32.5765558148297\\
7.39999999999941	32.5700418068\\
7.40199999999941	32.5635291013102\\
7.40399999999941	32.5570176981\\
7.40599999999941	32.5505075969087\\
7.40799999999941	32.5439987974766\\
7.40999999999941	32.5374912995424\\
7.41199999999941	32.5309851028468\\
7.41399999999941	32.524480207129\\
7.41599999999941	32.5179766121293\\
7.41799999999941	32.5114743175874\\
7.4199999999994	32.5049733232432\\
7.4219999999994	32.4984736288369\\
7.4239999999994	32.4919752341082\\
7.4259999999994	32.4854781387979\\
7.4279999999994	32.4789823426454\\
7.4299999999994	32.4724878453915\\
7.4319999999994	32.4659946467759\\
7.4339999999994	32.4595027465395\\
7.4359999999994	32.4530121444227\\
7.4379999999994	32.4465228401653\\
7.4399999999994	32.4400348335083\\
7.4419999999994	32.4335481241922\\
7.4439999999994	32.4270627119574\\
7.4459999999994	32.4205785965446\\
7.4479999999994	32.4140957776945\\
7.4499999999994	32.407614255148\\
7.4519999999994	32.4011340286457\\
7.4539999999994	32.3946550979287\\
7.4559999999994	32.3881774627374\\
7.4579999999994	32.3817011228133\\
7.4599999999994	32.375226077897\\
7.4619999999994	32.3687523277297\\
7.4639999999994	32.3622798720528\\
7.4659999999994	32.3558087106069\\
7.4679999999994	32.3493388431335\\
7.4699999999994	32.342870269374\\
7.4719999999994	32.3364029890694\\
7.4739999999994	32.3299370019614\\
7.4759999999994	32.3234723077914\\
7.4779999999994	32.3170089063004\\
7.4799999999994	32.3105467972303\\
7.4819999999994	32.3040859803227\\
7.4839999999994	32.2976264553189\\
7.4859999999994	32.2911682219609\\
7.4879999999994	32.2847112799904\\
7.4899999999994	32.278255629149\\
7.4919999999994	32.2718012691786\\
7.4939999999994	32.2653481998215\\
7.4959999999994	32.2588964208189\\
7.4979999999994	32.252445931913\\
7.4999999999994	32.2459967328462\\
7.5019999999994	32.2395488233602\\
7.5039999999994	32.2331022031974\\
7.5059999999994	32.2266568721\\
7.5079999999994	32.2202128298099\\
7.50999999999939	32.2137700760697\\
7.51199999999939	32.2073286106217\\
7.51399999999939	32.2008884332083\\
7.51599999999939	32.1944495435719\\
7.51799999999939	32.1880119414546\\
7.51999999999939	32.1815756265999\\
7.52199999999939	32.1751405987494\\
7.52399999999939	32.1687068576467\\
7.52599999999939	32.1622744030337\\
7.52799999999939	32.1558432346536\\
7.52999999999939	32.1494133522489\\
7.53199999999939	32.1429847555628\\
7.53399999999939	32.1365574443381\\
7.53599999999939	32.1301314183176\\
7.53799999999939	32.1237066772446\\
7.53999999999939	32.1172832208619\\
7.54199999999939	32.1108610489127\\
7.54399999999939	32.10444016114\\
7.54599999999939	32.0980205572873\\
7.54799999999939	32.091602237098\\
7.54999999999939	32.0851852003148\\
7.55199999999939	32.0787694466815\\
7.55399999999939	32.0723549759417\\
7.55599999999939	32.0659417878388\\
7.55799999999939	32.0595298821158\\
7.55999999999939	32.053119258517\\
7.56199999999939	32.0467099167856\\
7.56399999999939	32.0403018566654\\
7.56599999999939	32.0338950778999\\
7.56799999999939	32.0274895802334\\
7.56999999999939	32.0210853634092\\
7.57199999999939	32.0146824271715\\
7.57399999999939	32.008280771264\\
7.57599999999939	32.001880395431\\
7.57799999999939	31.9954812994165\\
7.57999999999939	31.9890834829642\\
7.58199999999939	31.9826869458186\\
7.58399999999939	31.9762916877237\\
7.58599999999939	31.9698977084242\\
7.58799999999939	31.9635050076637\\
7.58999999999939	31.9571135851868\\
7.59199999999939	31.9507234407384\\
7.59399999999939	31.9443345740624\\
7.59599999999939	31.937946984903\\
7.59799999999939	31.9315606730053\\
7.59999999999939	31.9251756381144\\
7.60199999999938	31.918791879974\\
7.60399999999938	31.9124093983292\\
7.60599999999938	31.9060281929244\\
7.60799999999938	31.8996482635049\\
7.60999999999938	31.8932696098155\\
7.61199999999938	31.8868922316007\\
7.61399999999938	31.8805161286059\\
7.61599999999938	31.874141300576\\
7.61799999999938	31.8677677472558\\
7.61999999999938	31.8613954683906\\
7.62199999999938	31.8550244637257\\
7.62399999999938	31.848654733006\\
7.62599999999938	31.842286275977\\
7.62799999999938	31.8359190923843\\
7.62999999999938	31.8295531819725\\
7.63199999999938	31.8231885444876\\
7.63399999999938	31.8168251796751\\
7.63599999999938	31.8104630872799\\
7.63799999999938	31.8041022670486\\
7.63999999999938	31.7977427187257\\
7.64199999999938	31.7913844420575\\
7.64399999999938	31.7850274367899\\
7.64599999999938	31.778671702668\\
7.64799999999938	31.7723172394382\\
7.64999999999938	31.7659640468465\\
7.65199999999938	31.7596121246382\\
7.65399999999938	31.7532614725595\\
7.65599999999938	31.7469120903567\\
7.65799999999938	31.7405639777757\\
7.65999999999938	31.7342171345622\\
7.66199999999938	31.7278715604635\\
7.66399999999938	31.7215272552248\\
7.66599999999938	31.715184218593\\
7.66799999999938	31.7088424503135\\
7.66999999999938	31.7025019501337\\
7.67199999999938	31.6961627178001\\
7.67399999999938	31.6898247530581\\
7.67599999999938	31.683488055655\\
7.67799999999938	31.6771526253378\\
7.67999999999938	31.6708184618518\\
7.68199999999938	31.664485564945\\
7.68399999999938	31.6581539343637\\
7.68599999999938	31.6518235698542\\
7.68799999999938	31.6454944711641\\
7.68999999999938	31.63916663804\\
7.69199999999937	31.6328400702286\\
7.69399999999937	31.626514767477\\
7.69599999999937	31.6201907295328\\
7.69799999999937	31.6138679561421\\
7.69999999999937	31.6075464470528\\
7.70199999999937	31.6012262020116\\
7.70399999999937	31.5949072207662\\
7.70599999999937	31.5885895030638\\
7.70799999999937	31.5822730486514\\
7.70999999999937	31.5759578572766\\
7.71199999999937	31.5696439286867\\
7.71399999999937	31.5633312626297\\
7.71599999999937	31.5570198588527\\
7.71799999999937	31.5507097171036\\
7.71999999999937	31.5444008371297\\
7.72199999999937	31.5380932186789\\
7.72399999999937	31.5317868614987\\
7.72599999999937	31.525481765337\\
7.72799999999937	31.5191779299424\\
7.72999999999937	31.5128753550613\\
7.73199999999937	31.5065740404431\\
7.73399999999937	31.5002739858351\\
7.73599999999937	31.4939751909848\\
7.73799999999937	31.4876776556419\\
7.73999999999937	31.4813813795532\\
7.74199999999937	31.4750863624674\\
7.74399999999937	31.4687926041323\\
7.74599999999937	31.4625001042968\\
7.74799999999937	31.456208862709\\
7.74999999999937	31.4499188791174\\
7.75199999999937	31.4436301532704\\
7.75399999999937	31.4373426849161\\
7.75599999999937	31.4310564738038\\
7.75799999999937	31.4247715196817\\
7.75999999999937	31.4184878222982\\
7.76199999999937	31.4122053814024\\
7.76399999999937	31.4059241967429\\
7.76599999999937	31.3996442680689\\
7.76799999999937	31.3933655951284\\
7.76999999999937	31.3870881776711\\
7.77199999999937	31.3808120154455\\
7.77399999999937	31.3745371082009\\
7.77599999999937	31.3682634556864\\
7.77799999999937	31.3619910576505\\
7.77999999999937	31.3557199138427\\
7.78199999999936	31.3494500240128\\
7.78399999999936	31.3431813879094\\
7.78599999999936	31.3369140052818\\
7.78799999999936	31.3306478758795\\
7.78999999999936	31.324382999452\\
7.79199999999936	31.3181193757488\\
7.79399999999936	31.3118570045191\\
7.79599999999936	31.3055958855127\\
7.79799999999936	31.2993360184792\\
7.79999999999936	31.293077403168\\
7.80199999999936	31.2868200393289\\
7.80399999999936	31.2805639267122\\
7.80599999999936	31.2743090650671\\
7.80799999999936	31.2680554541437\\
7.80999999999936	31.2618030936917\\
7.81199999999936	31.2555519834614\\
7.81399999999936	31.2493021232023\\
7.81599999999936	31.243053512665\\
7.81799999999936	31.2368061515993\\
7.81999999999936	31.2305600397556\\
7.82199999999936	31.2243151768835\\
7.82399999999936	31.2180715627343\\
7.82599999999936	31.2118291970571\\
7.82799999999936	31.2055880796032\\
7.82999999999936	31.1993482101229\\
7.83199999999936	31.1931095883659\\
7.83399999999936	31.1868722140835\\
7.83599999999936	31.1806360870258\\
7.83799999999936	31.1744012069438\\
7.83999999999936	31.1681675735875\\
7.84199999999936	31.1619351867086\\
7.84399999999936	31.1557040460572\\
7.84599999999936	31.1494741513843\\
7.84799999999936	31.1432455024405\\
7.84999999999936	31.1370180989769\\
7.85199999999936	31.1307919407445\\
7.85399999999936	31.1245670274941\\
7.85599999999936	31.1183433589771\\
7.85799999999936	31.1121209349443\\
7.85999999999936	31.105899755147\\
7.86199999999936	31.0996798193365\\
7.86399999999936	31.0934611272635\\
7.86599999999936	31.08724367868\\
7.86799999999936	31.0810274733371\\
7.86999999999936	31.0748125109861\\
7.87199999999936	31.0685987913784\\
7.87399999999935	31.0623863142656\\
7.87599999999935	31.0561750793996\\
7.87799999999935	31.0499650865316\\
7.87999999999935	31.0437563354136\\
7.88199999999935	31.0375488257961\\
7.88399999999935	31.0313425574328\\
7.88599999999935	31.0251375300741\\
7.88799999999935	31.0189337434723\\
7.88999999999935	31.0127311973791\\
7.89199999999935	31.0065298915467\\
7.89399999999935	31.0003298257268\\
7.89599999999935	30.9941309996717\\
7.89799999999935	30.9879334131336\\
7.89999999999935	30.9817370658642\\
7.90199999999935	30.9755419576165\\
7.90399999999935	30.969348088142\\
7.90599999999935	30.9631554571931\\
7.90799999999935	30.9569640645225\\
7.90999999999935	30.950773909882\\
7.91199999999935	30.9445849930245\\
7.91399999999935	30.9383973137026\\
7.91599999999935	30.9322108716685\\
7.91799999999935	30.9260256666748\\
7.91999999999935	30.9198416984748\\
7.92199999999935	30.91365896682\\
7.92399999999935	30.9074774714643\\
7.92599999999935	30.9012972121601\\
7.92799999999935	30.8951181886598\\
7.92999999999935	30.8889404007168\\
7.93199999999935	30.8827638480838\\
7.93399999999935	30.8765885305141\\
7.93599999999935	30.8704144477601\\
7.93799999999935	30.8642415995757\\
7.93999999999935	30.8580699857132\\
7.94199999999935	30.8518996059267\\
7.94399999999935	30.8457304599685\\
7.94599999999935	30.8395625475928\\
7.94799999999935	30.8333958685523\\
7.94999999999935	30.8272304226006\\
7.95199999999935	30.821066209491\\
7.95399999999935	30.8149032289771\\
7.95599999999935	30.8087414808124\\
7.95799999999935	30.8025809647506\\
7.95999999999935	30.7964216805451\\
7.96199999999935	30.7902636279495\\
7.96399999999934	30.7841068067179\\
7.96599999999934	30.7779512166042\\
7.96799999999934	30.7717968573615\\
7.96999999999934	30.7656437287446\\
7.97199999999934	30.7594918305063\\
7.97399999999934	30.7533411624017\\
7.97599999999934	30.747191724184\\
7.97799999999934	30.741043515608\\
7.97999999999934	30.734896536427\\
7.98199999999934	30.7287507863958\\
7.98399999999934	30.7226062652685\\
7.98599999999934	30.7164629727989\\
7.98799999999934	30.7103209087422\\
7.98999999999934	30.7041800728521\\
7.99199999999934	30.6980404648832\\
7.99399999999934	30.6919020845898\\
7.99599999999934	30.6857649317268\\
7.99799999999934	30.6796290060483\\
7.99999999999934	30.6734943073093\\
};
\addplot [color=mycolor1, forget plot]
  table[row sep=crcr]{%
7.99999999999934	30.6734943073093\\
8.00199999999934	30.6673608352644\\
8.00399999999934	30.6612285896681\\
8.00599999999934	30.6550975702753\\
8.00799999999934	30.648967776841\\
8.00999999999934	30.6428392091194\\
8.01199999999934	30.6367118668663\\
8.01399999999935	30.6305857498359\\
8.01599999999935	30.6244608577839\\
8.01799999999935	30.6183371904648\\
8.01999999999935	30.6122147476335\\
8.02199999999935	30.6060935290461\\
8.02399999999935	30.599973534457\\
8.02599999999935	30.5938547636216\\
8.02799999999935	30.5877372162954\\
8.02999999999935	30.5816208922336\\
8.03199999999935	30.5755057911917\\
8.03399999999935	30.5693919129252\\
8.03599999999935	30.5632792571892\\
8.03799999999935	30.5571678237398\\
8.03999999999935	30.5510576123321\\
8.04199999999935	30.544948622722\\
8.04399999999936	30.5388408546651\\
8.04599999999936	30.532734307917\\
8.04799999999936	30.5266289822342\\
8.04999999999936	30.5205248773717\\
8.05199999999936	30.5144219930857\\
8.05399999999936	30.5083203291322\\
8.05599999999936	30.502219885267\\
8.05799999999936	30.4961206612464\\
8.05999999999936	30.490022656826\\
8.06199999999936	30.4839258717624\\
8.06399999999936	30.4778303058118\\
8.06599999999936	30.4717359587303\\
8.06799999999936	30.465642830274\\
8.06999999999936	30.4595509201993\\
8.07199999999937	30.4534602282627\\
8.07399999999937	30.4473707542208\\
8.07599999999937	30.4412824978295\\
8.07799999999937	30.4351954588458\\
8.07999999999937	30.429109637026\\
8.08199999999937	30.423025032127\\
8.08399999999937	30.4169416439054\\
8.08599999999937	30.4108594721175\\
8.08799999999937	30.4047785165207\\
8.08999999999937	30.3986987768712\\
8.09199999999937	30.3926202529261\\
8.09399999999937	30.3865429444427\\
8.09599999999937	30.3804668511774\\
8.09799999999937	30.3743919728873\\
8.09999999999937	30.3683183093297\\
8.10199999999938	30.3622458602617\\
8.10399999999938	30.3561746254402\\
8.10599999999938	30.3501046046227\\
8.10799999999938	30.344035797566\\
8.10999999999938	30.3379682040279\\
8.11199999999938	30.3319018237655\\
8.11399999999938	30.3258366565366\\
8.11599999999938	30.3197727020975\\
8.11799999999938	30.3137099602069\\
8.11999999999938	30.3076484306216\\
8.12199999999938	30.3015881130997\\
8.12399999999938	30.2955290073986\\
8.12599999999938	30.2894711132761\\
8.12799999999938	30.2834144304894\\
8.12999999999938	30.2773589587973\\
8.13199999999939	30.2713046979569\\
8.13399999999939	30.2652516477261\\
8.13599999999939	30.2591998078629\\
8.13799999999939	30.2531491781258\\
8.13999999999939	30.2470997582719\\
8.14199999999939	30.2410515480598\\
8.14399999999939	30.2350045472476\\
8.14599999999939	30.2289587555936\\
8.14799999999939	30.2229141728556\\
8.14999999999939	30.2168707987923\\
8.15199999999939	30.2108286331615\\
8.15399999999939	30.2047876757223\\
8.15599999999939	30.1987479262321\\
8.15799999999939	30.1927093844504\\
8.15999999999939	30.186672050135\\
8.1619999999994	30.1806359230452\\
8.1639999999994	30.1746010029388\\
8.1659999999994	30.1685672895745\\
8.1679999999994	30.1625347827117\\
8.1699999999994	30.1565034821086\\
8.1719999999994	30.1504733875241\\
8.1739999999994	30.1444444987171\\
8.1759999999994	30.1384168154462\\
8.1779999999994	30.132390337471\\
8.1799999999994	30.1263650645501\\
8.1819999999994	30.1203409964422\\
8.1839999999994	30.114318132907\\
8.1859999999994	30.1082964737032\\
8.1879999999994	30.1022760185901\\
8.1899999999994	30.096256767327\\
8.19199999999941	30.0902387196735\\
8.19399999999941	30.084221875388\\
8.19599999999941	30.0782062342307\\
8.19799999999941	30.0721917959609\\
8.19999999999941	30.0661785603378\\
8.20199999999941	30.060166527121\\
8.20399999999941	30.0541556960703\\
8.20599999999941	30.0481460669449\\
8.20799999999941	30.0421376395049\\
8.20999999999941	30.0361304135098\\
8.21199999999941	30.0301243887193\\
8.21399999999941	30.0241195648936\\
8.21599999999941	30.0181159417918\\
8.21799999999941	30.0121135191746\\
8.21999999999941	30.0061122968013\\
8.22199999999942	30.0001122744324\\
8.22399999999942	29.9941134518276\\
8.22599999999942	29.9881158287473\\
8.22799999999942	29.9821194049515\\
8.22999999999942	29.9761241802003\\
8.23199999999942	29.970130154254\\
8.23399999999942	29.964137326873\\
8.23599999999942	29.9581456978173\\
8.23799999999942	29.9521552668481\\
8.23999999999942	29.9461660337249\\
8.24199999999942	29.9401779982084\\
8.24399999999942	29.9341911600592\\
8.24599999999942	29.9282055190382\\
8.24799999999942	29.9222210749059\\
8.24999999999942	29.9162378274226\\
8.25199999999943	29.9102557763494\\
8.25399999999943	29.9042749214467\\
8.25599999999943	29.8982952624756\\
8.25799999999943	29.8923167991971\\
8.25999999999943	29.8863395313716\\
8.26199999999943	29.8803634587604\\
8.26399999999943	29.8743885811248\\
8.26599999999943	29.8684148982254\\
8.26799999999943	29.8624424098232\\
8.26999999999943	29.85647111568\\
8.27199999999943	29.8505010155561\\
8.27399999999943	29.8445321092136\\
8.27599999999943	29.8385643964133\\
8.27799999999943	29.8325978769164\\
8.27999999999943	29.8266325504849\\
8.28199999999944	29.8206684168797\\
8.28399999999944	29.8147054758628\\
8.28599999999944	29.8087437271949\\
8.28799999999944	29.8027831706382\\
8.28999999999944	29.7968238059541\\
8.29199999999944	29.7908656329048\\
8.29399999999944	29.784908651251\\
8.29599999999944	29.7789528607554\\
8.29799999999944	29.7729982611797\\
8.29999999999944	29.767044852285\\
8.30199999999944	29.7610926338339\\
8.30399999999944	29.7551416055886\\
8.30599999999944	29.7491917673102\\
8.30799999999944	29.7432431187617\\
8.30999999999944	29.7372956597046\\
8.31199999999945	29.7313493899013\\
8.31399999999945	29.7254043091138\\
8.31599999999945	29.7194604171048\\
8.31799999999945	29.7135177136356\\
8.31999999999945	29.7075761984698\\
8.32199999999945	29.7016358713688\\
8.32399999999945	29.6956967320958\\
8.32599999999945	29.6897587804127\\
8.32799999999945	29.683822016082\\
8.32999999999945	29.6778864388669\\
8.33199999999945	29.6719520485295\\
8.33399999999945	29.6660188448328\\
8.33599999999945	29.660086827539\\
8.33799999999945	29.6541559964112\\
8.33999999999945	29.6482263512124\\
8.34199999999946	29.6422978917055\\
8.34399999999946	29.636370617653\\
8.34599999999946	29.6304445288179\\
8.34799999999946	29.6245196249635\\
8.34999999999946	29.6185959058526\\
8.35199999999946	29.6126733712484\\
8.35399999999946	29.6067520209141\\
8.35599999999946	29.6008318546127\\
8.35799999999946	29.5949128721077\\
8.35999999999946	29.5889950731623\\
8.36199999999946	29.5830784575399\\
8.36399999999946	29.5771630250038\\
8.36599999999946	29.5712487753173\\
8.36799999999946	29.5653357082439\\
8.36999999999946	29.5594238235474\\
8.37199999999947	29.5535131209911\\
8.37399999999947	29.5476036003388\\
8.37599999999947	29.5416952613542\\
8.37799999999947	29.5357881038007\\
8.37999999999947	29.5298821274422\\
8.38199999999947	29.5239773320426\\
8.38399999999947	29.5180737173659\\
8.38599999999947	29.5121712831757\\
8.38799999999947	29.5062700292361\\
8.38999999999947	29.5003699553112\\
8.39199999999947	29.4944710611649\\
8.39399999999947	29.4885733465612\\
8.39599999999947	29.4826768112644\\
8.39799999999947	29.4767814550386\\
8.39999999999947	29.4708872776482\\
8.40199999999948	29.4649942788573\\
8.40399999999948	29.4591024584303\\
8.40599999999948	29.4532118161312\\
8.40799999999948	29.4473223517253\\
8.40999999999948	29.4414340649765\\
8.41199999999948	29.4355469556491\\
8.41399999999948	29.4296610235082\\
8.41599999999948	29.4237762683179\\
8.41799999999948	29.4178926898432\\
8.41999999999948	29.4120102878486\\
8.42199999999948	29.4061290620991\\
8.42399999999948	29.4002490123594\\
8.42599999999948	29.3943701383941\\
8.42799999999948	29.3884924399685\\
8.42999999999948	29.3826159168472\\
8.43199999999949	29.3767405687954\\
8.43399999999949	29.370866395578\\
8.43599999999949	29.36499339696\\
8.43799999999949	29.3591215727069\\
8.43999999999949	29.3532509225836\\
8.44199999999949	29.3473814463551\\
8.44399999999949	29.3415131437872\\
8.44599999999949	29.3356460146448\\
8.44799999999949	29.3297800586937\\
8.44999999999949	29.3239152756985\\
8.45199999999949	29.3180516654256\\
8.45399999999949	29.3121892276398\\
8.45599999999949	29.3063279621072\\
8.45799999999949	29.3004678685928\\
8.45999999999949	29.2946089468628\\
8.4619999999995	29.2887511966826\\
8.4639999999995	29.2828946178177\\
8.4659999999995	29.2770392100346\\
8.4679999999995	29.2711849730986\\
8.4699999999995	29.2653319067754\\
8.4719999999995	29.2594800108317\\
8.4739999999995	29.2536292850325\\
8.4759999999995	29.2477797291445\\
8.4779999999995	29.2419313429336\\
8.4799999999995	29.2360841261659\\
8.4819999999995	29.2302380786075\\
8.4839999999995	29.2243932000247\\
8.4859999999995	29.2185494901836\\
8.4879999999995	29.2127069488507\\
8.4899999999995	29.2068655757921\\
8.49199999999951	29.2010253707744\\
8.49399999999951	29.1951863335639\\
8.49599999999951	29.1893484639273\\
8.49799999999951	29.1835117616309\\
8.49999999999951	29.1776762264414\\
8.50199999999951	29.1718418581253\\
8.50399999999951	29.1660086564494\\
8.50599999999951	29.1601766211803\\
8.50799999999951	29.154345752085\\
8.50999999999951	29.14851604893\\
8.51199999999951	29.1426875114826\\
8.51399999999951	29.1368601395092\\
8.51599999999951	29.131033932777\\
8.51799999999951	29.1252088910531\\
8.51999999999951	29.1193850141044\\
8.52199999999952	29.1135623016978\\
8.52399999999952	29.1077407536008\\
8.52599999999952	29.1019203695806\\
8.52799999999952	29.0961011494039\\
8.52999999999952	29.0902830928388\\
8.53199999999952	29.0844661996517\\
8.53399999999952	29.0786504696108\\
8.53599999999952	29.072835902483\\
8.53799999999952	29.0670224980362\\
8.53999999999952	29.0612102560376\\
8.54199999999952	29.0553991762544\\
8.54399999999952	29.0495892584547\\
8.54599999999952	29.0437805024063\\
8.54799999999952	29.0379729078763\\
8.54999999999952	29.0321664746332\\
8.55199999999953	29.0263612024443\\
8.55399999999953	29.0205570910773\\
8.55599999999953	29.0147541403004\\
8.55799999999953	29.0089523498814\\
8.55999999999953	29.0031517195883\\
8.56199999999953	28.9973522491893\\
8.56399999999953	28.991553938452\\
8.56599999999953	28.9857567871449\\
8.56799999999953	28.9799607950363\\
8.56999999999953	28.974165961894\\
8.57199999999953	28.9683722874864\\
8.57399999999953	28.9625797715819\\
8.57599999999953	28.9567884139487\\
8.57799999999953	28.9509982143553\\
8.57999999999953	28.9452091725703\\
8.58199999999954	28.9394212883618\\
8.58399999999954	28.9336345614982\\
8.58599999999954	28.9278489917489\\
8.58799999999954	28.9220645788818\\
8.58999999999954	28.9162813226658\\
8.59199999999954	28.9104992228697\\
8.59399999999954	28.904718279262\\
8.59599999999954	28.8989384916119\\
8.59799999999954	28.8931598596878\\
8.59999999999954	28.8873823832593\\
8.60199999999954	28.8816060620943\\
8.60399999999954	28.8758308959627\\
8.60599999999954	28.8700568846332\\
8.60799999999954	28.864284027875\\
8.60999999999954	28.8585123254569\\
8.61199999999955	28.8527417771485\\
8.61399999999955	28.846972382719\\
8.61599999999955	28.841204141937\\
8.61799999999955	28.8354370545728\\
8.61999999999955	28.829671120395\\
8.62199999999955	28.8239063391735\\
8.62399999999955	28.8181427106775\\
8.62599999999955	28.8123802346766\\
8.62799999999955	28.8066189109402\\
8.62999999999955	28.800858739238\\
8.63199999999955	28.7950997193398\\
8.63399999999955	28.7893418510149\\
8.63599999999955	28.7835851340334\\
8.63799999999955	28.7778295681651\\
8.63999999999955	28.7720751531795\\
8.64199999999956	28.7663218888466\\
8.64399999999956	28.7605697749363\\
8.64599999999956	28.7548188112187\\
8.64799999999956	28.7490689974637\\
8.64999999999956	28.7433203334415\\
8.65199999999956	28.7375728189217\\
8.65399999999956	28.7318264536752\\
8.65599999999956	28.7260812374715\\
8.65799999999956	28.7203371700813\\
8.65999999999956	28.7145942512747\\
8.66199999999956	28.7088524808221\\
8.66399999999956	28.703111858494\\
8.66599999999956	28.6973723840603\\
8.66799999999956	28.691634057292\\
8.66999999999956	28.6858968779594\\
8.67199999999957	28.6801608458332\\
8.67399999999957	28.674425960684\\
8.67599999999957	28.6686922222819\\
8.67799999999957	28.6629596303984\\
8.67999999999957	28.6572281848039\\
8.68199999999957	28.651497885269\\
8.68399999999957	28.6457687315648\\
8.68599999999957	28.640040723462\\
8.68799999999957	28.6343138607317\\
8.68999999999957	28.6285881431451\\
8.69199999999957	28.6228635704725\\
8.69399999999957	28.6171401424854\\
8.69599999999957	28.611417858955\\
8.69799999999957	28.6056967196524\\
8.69999999999957	28.5999767243486\\
8.70199999999958	28.5942578728149\\
8.70399999999958	28.5885401648229\\
8.70599999999958	28.5828236001435\\
8.70799999999958	28.5771081785488\\
8.70999999999958	28.5713938998092\\
8.71199999999958	28.5656807636966\\
8.71399999999958	28.5599687699829\\
8.71599999999958	28.5542579184393\\
8.71799999999958	28.5485482088378\\
8.71999999999958	28.5428396409494\\
8.72199999999958	28.5371322145462\\
8.72399999999958	28.5314259294003\\
8.72599999999958	28.5257207852826\\
8.72799999999958	28.5200167819655\\
8.72999999999958	28.5143139192211\\
8.73199999999959	28.5086121968207\\
8.73399999999959	28.502911614537\\
8.73599999999959	28.4972121721414\\
8.73799999999959	28.4915138694065\\
8.73999999999959	28.485816706104\\
8.74199999999959	28.4801206820061\\
8.74399999999959	28.4744257968851\\
8.74599999999959	28.4687320505135\\
8.74799999999959	28.4630394426634\\
8.74999999999959	28.4573479731067\\
8.75199999999959	28.4516576416165\\
8.75399999999959	28.4459684479646\\
8.75599999999959	28.4402803919243\\
8.75799999999959	28.4345934732672\\
8.75999999999959	28.4289076917664\\
8.7619999999996	28.4232230471945\\
8.7639999999996	28.417539539324\\
8.7659999999996	28.4118571679276\\
8.7679999999996	28.4061759327782\\
8.7699999999996	28.4004958336484\\
8.7719999999996	28.394816870311\\
8.7739999999996	28.3891390425396\\
8.7759999999996	28.3834623501061\\
8.7779999999996	28.3777867927842\\
8.7799999999996	28.3721123703466\\
8.7819999999996	28.3664390825665\\
8.7839999999996	28.360766929217\\
8.7859999999996	28.355095910071\\
8.7879999999996	28.349426024902\\
8.7899999999996	28.3437572734833\\
8.79199999999961	28.338089655588\\
8.79399999999961	28.3324231709894\\
8.79599999999961	28.3267578194611\\
8.79799999999961	28.3210936007766\\
8.79999999999961	28.315430514709\\
8.80199999999961	28.3097685610319\\
8.80399999999961	28.3041077395193\\
8.80599999999961	28.2984480499444\\
8.80799999999961	28.292789492081\\
8.80999999999961	28.2871320657026\\
8.81199999999961	28.2814757705832\\
8.81399999999961	28.2758206064966\\
8.81599999999961	28.2701665732165\\
8.81799999999961	28.2645136705167\\
8.81999999999961	28.2588618981717\\
8.82199999999962	28.2532112559548\\
8.82399999999962	28.2475617436401\\
8.82599999999962	28.241913361002\\
8.82799999999962	28.2362661078143\\
8.82999999999962	28.2306199838514\\
8.83199999999962	28.2249749888874\\
8.83399999999962	28.2193311226963\\
8.83599999999962	28.213688385053\\
8.83799999999962	28.2080467757313\\
8.83999999999962	28.2024062945055\\
8.84199999999962	28.1967669411507\\
8.84399999999962	28.1911287154408\\
8.84599999999962	28.1854916171505\\
8.84799999999962	28.1798556460541\\
8.84999999999962	28.1742208019264\\
8.85199999999963	28.1685870845424\\
8.85399999999963	28.1629544936763\\
8.85599999999963	28.1573230291029\\
8.85799999999963	28.1516926905973\\
8.85999999999963	28.146063477934\\
8.86199999999963	28.1404353908882\\
8.86399999999963	28.1348084292345\\
8.86599999999963	28.1291825927482\\
8.86799999999963	28.1235578812039\\
8.86999999999963	28.1179342943768\\
8.87199999999963	28.112311832042\\
8.87399999999963	28.1066904939751\\
8.87599999999963	28.1010702799507\\
8.87799999999963	28.0954511897442\\
8.87999999999963	28.0898332231311\\
8.88199999999964	28.0842163798862\\
8.88399999999964	28.0786006597857\\
8.88599999999964	28.0729860626043\\
8.88799999999964	28.0673725881177\\
8.88999999999964	28.0617602361014\\
8.89199999999964	28.0561490063311\\
8.89399999999964	28.0505388985822\\
8.89599999999964	28.0449299126302\\
8.89799999999964	28.0393220482512\\
8.89999999999964	28.0337153052206\\
8.90199999999964	28.0281096833144\\
8.90399999999964	28.0225051823081\\
8.90599999999964	28.0169018019779\\
8.90799999999964	28.0112995420996\\
8.90999999999964	28.0056984024492\\
8.91199999999965	28.0000983828027\\
8.91399999999965	27.9944994829361\\
8.91599999999965	27.9889017026254\\
8.91799999999965	27.9833050416467\\
8.91999999999965	27.9777094997763\\
8.92199999999965	27.9721150767906\\
8.92399999999965	27.9665217724654\\
8.92599999999965	27.9609295865775\\
8.92799999999965	27.9553385189028\\
8.92999999999965	27.9497485692182\\
8.93199999999965	27.9441597373\\
8.93399999999965	27.9385720229244\\
8.93599999999965	27.9329854258681\\
8.93799999999965	27.9273999459077\\
8.93999999999965	27.92181558282\\
8.94199999999966	27.9162323363812\\
8.94399999999966	27.9106502063685\\
8.94599999999966	27.9050691925586\\
8.94799999999966	27.8994892947281\\
8.94999999999966	27.8939105126538\\
8.95199999999966	27.8883328461129\\
8.95399999999966	27.8827562948821\\
8.95599999999966	27.8771808587384\\
8.95799999999966	27.871606537459\\
8.95999999999966	27.8660333308206\\
8.96199999999966	27.8604612386007\\
8.96399999999966	27.8548902605764\\
8.96599999999966	27.8493203965248\\
8.96799999999966	27.843751646223\\
8.96999999999966	27.8381840094488\\
8.97199999999967	27.832617485979\\
8.97399999999967	27.8270520755912\\
8.97599999999967	27.8214877780629\\
8.97799999999967	27.8159245931715\\
8.97999999999967	27.8103625206944\\
8.98199999999967	27.8048015604094\\
8.98399999999967	27.7992417120938\\
8.98599999999967	27.7936829755257\\
8.98799999999967	27.7881253504823\\
8.98999999999967	27.7825688367417\\
8.99199999999967	27.7770134340814\\
8.99399999999967	27.7714591422795\\
8.99599999999967	27.7659059611138\\
8.99799999999967	27.7603538903621\\
8.99999999999967	27.7548029298024\\
9.00199999999968	27.749253079213\\
9.00399999999968	27.7437043383713\\
9.00599999999968	27.7381567070562\\
9.00799999999968	27.732610185045\\
9.00999999999968	27.7270647721166\\
9.01199999999968	27.7215204680487\\
9.01399999999968	27.7159772726201\\
9.01599999999968	27.7104351856087\\
9.01799999999968	27.7048942067929\\
9.01999999999968	27.6993543359513\\
9.02199999999968	27.6938155728623\\
9.02399999999968	27.6882779173044\\
9.02599999999968	27.6827413690558\\
9.02799999999968	27.6772059278955\\
9.02999999999968	27.6716715936019\\
9.03199999999969	27.666138365954\\
9.03399999999969	27.66060624473\\
9.03599999999969	27.6550752297089\\
9.03799999999969	27.6495453206697\\
9.03999999999969	27.644016517391\\
9.04199999999969	27.638488819652\\
9.04399999999969	27.6329622272311\\
9.04599999999969	27.6274367399078\\
9.04799999999969	27.6219123574607\\
9.04999999999969	27.6163890796694\\
9.05199999999969	27.6108669063124\\
9.05399999999969	27.6053458371694\\
9.05599999999969	27.5998258720191\\
9.05799999999969	27.5943070106412\\
9.05999999999969	27.5887892528149\\
9.0619999999997	27.5832725983192\\
9.0639999999997	27.5777570469339\\
9.0659999999997	27.5722425984379\\
9.0679999999997	27.5667292526113\\
9.0699999999997	27.5612170092333\\
9.0719999999997	27.5557058680834\\
9.0739999999997	27.550195828941\\
9.0759999999997	27.5446868915864\\
9.0779999999997	27.5391790557986\\
9.0799999999997	27.5336723213578\\
9.0819999999997	27.5281666880435\\
9.0839999999997	27.522662155636\\
9.0859999999997	27.5171587239141\\
9.0879999999997	27.5116563926587\\
9.0899999999997	27.5061551616497\\
9.09199999999971	27.5006550306664\\
9.09399999999971	27.4951559994895\\
9.09599999999971	27.489658067899\\
9.09799999999971	27.4841612356745\\
9.09999999999971	27.4786655025968\\
9.10199999999971	27.4731708684456\\
9.10399999999971	27.4676773330017\\
9.10599999999971	27.4621848960449\\
9.10799999999971	27.456693557356\\
9.10999999999971	27.451203316715\\
9.11199999999971	27.4457141739028\\
9.11399999999971	27.4402261286991\\
9.11599999999971	27.434739180885\\
9.11799999999971	27.4292533302413\\
9.11999999999972	27.4237685765481\\
9.12199999999972	27.4182849195862\\
9.12399999999972	27.4128023591364\\
9.12599999999972	27.4073208949793\\
9.12799999999972	27.4018405268957\\
9.12999999999972	27.3963612546667\\
9.13199999999972	27.3908830780727\\
9.13399999999972	27.3854059968948\\
9.13599999999972	27.3799300109145\\
9.13799999999972	27.3744551199119\\
9.13999999999972	27.3689813236687\\
9.14199999999972	27.3635086219657\\
9.14399999999972	27.3580370145844\\
9.14599999999972	27.3525665013051\\
9.14799999999972	27.34709708191\\
9.14999999999973	27.3416287561797\\
9.15199999999973	27.3361615238963\\
9.15399999999973	27.3306953848404\\
9.15599999999973	27.3252303387934\\
9.15799999999973	27.3197663855375\\
9.15999999999973	27.3143035248532\\
9.16199999999973	27.3088417565227\\
9.16399999999973	27.3033810803274\\
9.16599999999973	27.2979214960488\\
9.16799999999973	27.2924630034686\\
9.16999999999973	27.2870056023689\\
9.17199999999973	27.2815492925308\\
9.17399999999973	27.2760940737367\\
9.17599999999973	27.2706399457676\\
9.17799999999973	27.2651869084064\\
9.17999999999974	27.2597349614343\\
9.18199999999974	27.2542841046338\\
9.18399999999974	27.2488343377862\\
9.18599999999974	27.2433856606743\\
9.18799999999974	27.2379380730795\\
9.18999999999974	27.2324915747846\\
9.19199999999974	27.2270461655713\\
9.19399999999974	27.2216018452222\\
9.19599999999974	27.2161586135193\\
9.19799999999974	27.2107164702449\\
9.19999999999974	27.2052754151816\\
9.20199999999974	27.1998354481116\\
9.20399999999974	27.1943965688175\\
9.20599999999974	27.1889587770816\\
9.20799999999974	27.1835220726866\\
9.20999999999975	27.1780864554148\\
9.21199999999975	27.1726519250491\\
9.21399999999975	27.1672184813723\\
9.21599999999975	27.1617861241665\\
9.21799999999975	27.1563548532152\\
9.21999999999975	27.1509246683005\\
9.22199999999975	27.1454955692058\\
9.22399999999975	27.1400675557136\\
9.22599999999975	27.1346406276072\\
9.22799999999975	27.129214784669\\
9.22999999999975	27.1237900266825\\
9.23199999999975	27.1183663534307\\
9.23399999999975	27.1129437646964\\
9.23599999999975	27.107522260263\\
9.23799999999975	27.1021018399138\\
9.23999999999976	27.0966825034317\\
9.24199999999976	27.0912642506002\\
9.24399999999976	27.0858470812024\\
9.24599999999976	27.0804309950219\\
9.24799999999976	27.0750159918417\\
9.24999999999976	27.0696020714458\\
9.25199999999976	27.0641892336174\\
9.25399999999976	27.0587774781399\\
9.25599999999976	27.0533668047972\\
9.25799999999976	27.0479572133726\\
9.25999999999976	27.0425487036498\\
9.26199999999976	27.037141275413\\
9.26399999999976	27.0317349284451\\
9.26599999999976	27.0263296625306\\
9.26799999999976	27.0209254774529\\
9.26999999999977	27.0155223729963\\
9.27199999999977	27.0101203489441\\
9.27399999999977	27.0047194050808\\
9.27599999999977	26.9993195411902\\
9.27799999999977	26.9939207570564\\
9.27999999999977	26.9885230524635\\
9.28199999999977	26.9831264271956\\
9.28399999999977	26.9777308810368\\
9.28599999999977	26.9723364137713\\
9.28799999999977	26.9669430251837\\
9.28999999999977	26.961550715058\\
9.29199999999977	26.9561594831785\\
9.29399999999977	26.9507693293299\\
9.29599999999977	26.9453802532965\\
9.29799999999977	26.9399922548625\\
9.29999999999978	26.9346053338129\\
9.30199999999978	26.929219489932\\
9.30399999999978	26.9238347230043\\
9.30599999999978	26.9184510328147\\
9.30799999999978	26.9130684191478\\
9.30999999999978	26.9076868817883\\
9.31199999999978	26.9023064205208\\
9.31399999999978	26.8969270351305\\
9.31599999999978	26.8915487254022\\
9.31799999999978	26.8861714911205\\
9.31999999999978	26.8807953320707\\
9.32199999999978	26.8754202480375\\
9.32399999999978	26.8700462388063\\
9.32599999999978	26.8646733041621\\
9.32799999999978	26.8593014438895\\
9.32999999999979	26.8539306577743\\
9.33199999999979	26.8485609456016\\
9.33399999999979	26.8431923071563\\
9.33599999999979	26.837824742224\\
9.33799999999979	26.8324582505901\\
9.33999999999979	26.8270928320398\\
9.34199999999979	26.8217284863587\\
9.34399999999979	26.816365213332\\
9.34599999999979	26.8110030127454\\
9.34799999999979	26.8056418843844\\
9.34999999999979	26.800281828035\\
9.35199999999979	26.794922843482\\
9.35399999999979	26.7895649305115\\
9.35599999999979	26.7842080889095\\
9.35799999999979	26.7788523184616\\
9.3599999999998	26.7734976189536\\
9.3619999999998	26.7681439901711\\
9.3639999999998	26.7627914319004\\
9.3659999999998	26.7574399439271\\
9.3679999999998	26.7520895260374\\
9.3699999999998	26.746740178017\\
9.3719999999998	26.7413918996526\\
9.3739999999998	26.7360446907299\\
9.3759999999998	26.7306985510349\\
9.3779999999998	26.7253534803541\\
9.3799999999998	26.7200094784736\\
9.3819999999998	26.7146665451798\\
9.3839999999998	26.7093246802589\\
9.3859999999998	26.7039838834973\\
9.3879999999998	26.6986441546814\\
9.38999999999981	26.6933054935977\\
9.39199999999981	26.6879679000325\\
9.39399999999981	26.6826313737728\\
9.39599999999981	26.6772959146047\\
9.39799999999981	26.6719615223152\\
9.39999999999981	26.6666281966906\\
9.40199999999981	26.6612959375178\\
9.40399999999981	26.6559647445837\\
9.40599999999981	26.6506346176749\\
9.40799999999981	26.6453055565782\\
9.40999999999981	26.6399775610805\\
9.41199999999981	26.6346506309689\\
9.41399999999981	26.6293247660301\\
9.41599999999981	26.6239999660515\\
9.41799999999981	26.6186762308197\\
9.41999999999982	26.613353560122\\
9.42199999999982	26.6080319537457\\
9.42399999999982	26.6027114114777\\
9.42599999999982	26.5973919331053\\
9.42799999999982	26.5920735184159\\
9.42999999999982	26.5867561671967\\
9.43199999999982	26.5814398792349\\
9.43399999999982	26.5761246543181\\
9.43599999999982	26.5708104922338\\
9.43799999999982	26.5654973927692\\
9.43999999999982	26.5601853557122\\
9.44199999999982	26.55487438085\\
9.44399999999982	26.5495644679701\\
9.44599999999982	26.5442556168608\\
9.44799999999982	26.5389478273091\\
9.44999999999983	26.5336410991031\\
9.45199999999983	26.5283354320306\\
9.45399999999983	26.5230308258788\\
9.45599999999983	26.5177272804365\\
9.45799999999983	26.5124247954912\\
9.45999999999983	26.5071233708305\\
9.46199999999983	26.5018230062427\\
9.46399999999983	26.4965237015161\\
9.46599999999983	26.4912254564383\\
9.46799999999983	26.4859282707975\\
9.46999999999983	26.480632144382\\
9.47199999999983	26.4753370769801\\
9.47399999999983	26.4700430683798\\
9.47599999999983	26.4647501183693\\
9.47799999999983	26.4594582267372\\
9.47999999999984	26.4541673932719\\
9.48199999999984	26.4488776177615\\
9.48399999999984	26.4435888999945\\
9.48599999999984	26.4383012397597\\
9.48799999999984	26.4330146368457\\
9.48999999999984	26.4277290910402\\
9.49199999999984	26.4224446021329\\
9.49399999999984	26.4171611699118\\
9.49599999999984	26.4118787941661\\
9.49799999999984	26.4065974746838\\
9.49999999999984	26.4013172112546\\
9.50199999999984	26.3960380036666\\
9.50399999999984	26.3907598517091\\
9.50599999999984	26.3854827551708\\
9.50799999999984	26.3802067138409\\
9.50999999999985	26.374931727508\\
9.51199999999985	26.3696577959614\\
9.51399999999985	26.3643849189903\\
9.51599999999985	26.3591130963834\\
9.51799999999985	26.3538423279303\\
9.51999999999985	26.3485726134202\\
9.52199999999985	26.3433039526422\\
9.52399999999985	26.3380363453857\\
9.52599999999985	26.3327697914396\\
9.52799999999985	26.3275042905939\\
9.52999999999985	26.3222398426378\\
9.53199999999985	26.3169764473606\\
9.53399999999985	26.311714104552\\
9.53599999999985	26.3064528140014\\
9.53799999999985	26.3011925754985\\
9.53999999999986	26.2959333888328\\
9.54199999999986	26.2906752537942\\
9.54399999999986	26.2854181701722\\
9.54599999999986	26.2801621377568\\
9.54799999999986	26.2749071563376\\
9.54999999999986	26.2696532257045\\
9.55199999999986	26.2644003456471\\
9.55399999999986	26.2591485159559\\
9.55599999999986	26.2538977364206\\
9.55799999999986	26.2486480068311\\
9.55999999999986	26.2433993269776\\
9.56199999999986	26.23815169665\\
9.56399999999986	26.2329051156386\\
9.56599999999986	26.2276595837336\\
9.56799999999986	26.2224151007251\\
9.56999999999987	26.2171716664035\\
9.57199999999987	26.2119292805592\\
9.57399999999987	26.2066879429821\\
9.57599999999987	26.2014476534629\\
9.57799999999987	26.1962084117921\\
9.57999999999987	26.1909702177601\\
9.58199999999987	26.1857330711573\\
9.58399999999987	26.1804969717743\\
9.58599999999987	26.1752619194019\\
9.58799999999987	26.1700279138305\\
9.58999999999987	26.1647949548509\\
9.59199999999987	26.1595630422537\\
9.59399999999987	26.1543321758299\\
9.59599999999987	26.14910235537\\
9.59799999999987	26.1438735806651\\
9.59999999999988	26.138645851506\\
9.60199999999988	26.1334191676837\\
9.60399999999988	26.1281935289889\\
9.60599999999988	26.122968935213\\
9.60799999999988	26.1177453861468\\
9.60999999999988	26.1125228815815\\
9.61199999999988	26.1073014213082\\
9.61399999999988	26.1020810051181\\
9.61599999999988	26.0968616328026\\
9.61799999999988	26.0916433041527\\
9.61999999999988	26.0864260189598\\
9.62199999999988	26.0812097770153\\
9.62399999999988	26.0759945781105\\
9.62599999999988	26.0707804220369\\
9.62799999999988	26.0655673085858\\
9.62999999999989	26.0603552375491\\
9.63199999999989	26.055144208718\\
9.63399999999989	26.0499342218844\\
9.63599999999989	26.0447252768396\\
9.63799999999989	26.0395173733755\\
9.63999999999989	26.0343105112838\\
9.64199999999989	26.0291046903563\\
9.64399999999989	26.0238999103849\\
9.64599999999989	26.0186961711611\\
9.64799999999989	26.0134934724771\\
9.64999999999989	26.0082918141247\\
9.65199999999989	26.003091195896\\
9.65399999999989	25.9978916175829\\
9.65599999999989	25.9926930789773\\
9.65799999999989	25.9874955798717\\
9.6599999999999	25.982299120058\\
9.6619999999999	25.9771036993284\\
9.6639999999999	25.9719093174753\\
9.6659999999999	25.9667159742905\\
9.6679999999999	25.9615236695666\\
9.6699999999999	25.9563324030962\\
9.6719999999999	25.9511421746717\\
9.6739999999999	25.9459529840847\\
9.6759999999999	25.9407648311285\\
9.6779999999999	25.9355777155955\\
9.6799999999999	25.930391637278\\
9.6819999999999	25.9252065959689\\
9.6839999999999	25.9200225914606\\
9.6859999999999	25.9148396235458\\
9.6879999999999	25.9096576920172\\
9.68999999999991	25.9044767966678\\
9.69199999999991	25.8992969372902\\
9.69399999999991	25.8941181136774\\
9.69599999999991	25.888940325622\\
9.69799999999991	25.8837635729171\\
9.69999999999991	25.8785878553558\\
9.70199999999991	25.8734131727311\\
9.70399999999991	25.8682395248358\\
9.70599999999991	25.8630669114632\\
9.70799999999991	25.8578953324062\\
9.70999999999991	25.8527247874584\\
9.71199999999991	25.8475552764127\\
9.71399999999991	25.8423867990625\\
9.71599999999991	25.8372193552011\\
9.71799999999991	25.8320529446215\\
9.71999999999992	25.8268875671175\\
9.72199999999992	25.8217232224824\\
9.72399999999992	25.8165599105098\\
9.72599999999992	25.811397630993\\
9.72799999999992	25.8062363837257\\
9.72999999999992	25.8010761685014\\
9.73199999999992	25.7959169851137\\
9.73399999999992	25.7907588333563\\
9.73599999999992	25.7856017130229\\
9.73799999999992	25.7804456239074\\
9.73999999999992	25.7752905658035\\
9.74199999999992	25.7701365385049\\
9.74399999999992	25.7649835418058\\
9.74599999999992	25.7598315755\\
9.74799999999992	25.7546806393812\\
9.74999999999993	25.7495307332435\\
9.75199999999993	25.7443818568812\\
9.75399999999993	25.7392340100882\\
9.75599999999993	25.7340871926587\\
9.75799999999993	25.7289414043869\\
9.75999999999993	25.7237966450668\\
9.76199999999993	25.7186529144929\\
9.76399999999993	25.7135102124594\\
9.76599999999993	25.7083685387604\\
9.76799999999993	25.7032278931906\\
9.76999999999993	25.6980882755445\\
9.77199999999993	25.6929496856162\\
9.77399999999993	25.6878121232005\\
9.77599999999993	25.6826755880914\\
9.77799999999993	25.6775400800842\\
9.77999999999994	25.6724055989731\\
9.78199999999994	25.667272144553\\
9.78399999999994	25.6621397166183\\
9.78599999999994	25.6570083149641\\
9.78799999999994	25.6518779393848\\
9.78999999999994	25.6467485896754\\
9.79199999999994	25.6416202656309\\
9.79399999999994	25.6364929670462\\
9.79599999999994	25.6313666937158\\
9.79799999999994	25.6262414454353\\
9.79999999999994	25.6211172219995\\
9.80199999999994	25.6159940232032\\
9.80399999999994	25.610871848842\\
9.80599999999994	25.6057506987107\\
9.80799999999994	25.6006305726046\\
9.80999999999995	25.5955114703189\\
9.81199999999995	25.5903933916488\\
9.81399999999995	25.5852763363898\\
9.81599999999995	25.5801603043372\\
9.81799999999995	25.5750452952866\\
9.81999999999995	25.5699313090331\\
9.82199999999995	25.5648183453722\\
9.82399999999995	25.5597064040997\\
9.82599999999995	25.5545954850109\\
9.82799999999995	25.5494855879015\\
9.82999999999995	25.5443767125671\\
9.83199999999995	25.5392688588036\\
9.83399999999995	25.5341620264064\\
9.83599999999995	25.5290562151714\\
9.83799999999995	25.5239514248945\\
9.83999999999996	25.5188476553714\\
9.84199999999996	25.5137449063982\\
9.84399999999996	25.5086431777706\\
9.84599999999996	25.5035424692848\\
9.84799999999996	25.4984427807365\\
9.84999999999996	25.4933441119221\\
9.85199999999996	25.4882464626375\\
9.85399999999996	25.4831498326789\\
9.85599999999996	25.4780542218425\\
9.85799999999996	25.4729596299242\\
9.85999999999996	25.4678660567209\\
9.86199999999996	25.4627735020281\\
9.86399999999996	25.4576819656427\\
9.86599999999996	25.4525914473612\\
9.86799999999996	25.4475019469795\\
9.86999999999997	25.4424134642944\\
9.87199999999997	25.4373259991023\\
9.87399999999997	25.4322395511997\\
9.87599999999997	25.4271541203833\\
9.87799999999997	25.4220697064496\\
9.87999999999997	25.4169863091955\\
9.88199999999997	25.4119039284174\\
9.88399999999997	25.406822563912\\
9.88599999999997	25.4017422154766\\
9.88799999999997	25.3966628829075\\
9.88999999999997	25.3915845660019\\
9.89199999999997	25.3865072645565\\
9.89399999999997	25.3814309783683\\
9.89599999999997	25.3763557072344\\
9.89799999999997	25.3712814509516\\
9.89999999999998	25.3662082093171\\
9.90199999999998	25.361135982128\\
9.90399999999998	25.3560647691816\\
9.90599999999998	25.3509945702749\\
9.90799999999998	25.3459253852051\\
9.90999999999998	25.3408572137699\\
9.91199999999998	25.335790055766\\
9.91399999999998	25.3307239109911\\
9.91599999999998	25.3256587792424\\
9.91799999999998	25.3205946603176\\
9.91999999999998	25.3155315540141\\
9.92199999999998	25.3104694601293\\
9.92399999999998	25.3054083784608\\
9.92599999999998	25.3003483088061\\
9.92799999999998	25.2952892509632\\
9.92999999999999	25.2902312047294\\
9.93199999999999	25.2851741699025\\
9.93399999999999	25.2801181462803\\
9.93599999999999	25.2750631336606\\
9.93799999999999	25.2700091318414\\
9.93999999999999	25.2649561406202\\
9.94199999999999	25.2599041597952\\
9.94399999999999	25.2548531891644\\
9.94599999999999	25.2498032285258\\
9.94799999999999	25.2447542776772\\
9.94999999999999	25.2397063364168\\
9.95199999999999	25.2346594045426\\
9.95399999999999	25.2296134818532\\
9.95599999999999	25.2245685681467\\
9.95799999999999	25.2195246632208\\
9.96	25.2144817668742\\
9.962	25.2094398789053\\
9.964	25.2043989991123\\
9.966	25.1993591272936\\
9.968	25.1943202632478\\
9.97	25.1892824067731\\
9.972	25.1842455576682\\
9.974	25.1792097157319\\
9.976	25.1741748807625\\
9.978	25.1691410525586\\
9.98	25.1641082309192\\
9.982	25.1590764156425\\
9.984	25.154045606528\\
9.986	25.1490158033739\\
9.988	25.1439870059794\\
9.99000000000001	25.1389592141431\\
9.99200000000001	25.1339324276641\\
9.99400000000001	25.1289066463414\\
9.99600000000001	25.1238818699739\\
9.99800000000001	25.1188580983608\\
10	25.1138353313008\\
};
\end{axis}

\begin{axis}[%
width=2.603in,
height=1.074in,
at={(4.436in,2.499in)},
scale only axis,
xmin=0,
xmax=10,
xlabel style={font=\color{white!15!black}},
xlabel={t},
ymode=log,
ymin=31.7114629739591,
ymax=100000,
yminorticks=true,
ylabel style={font=\color{white!15!black}},
ylabel={indice stiff},
axis background/.style={fill=white},
title style={font=\bfseries},
title={N=40}
]
\addplot [color=mycolor1, forget plot]
  table[row sep=crcr]{%
0	1782.7038554666\\
0.002	90465.8012910994\\
0.004	45298.8419602203\\
0.006	30241.7112351806\\
0.008	22712.0258124563\\
0.01	18193.3095663582\\
0.012	15180.0708967882\\
0.014	13027.0994930544\\
0.016	11411.7905245739\\
0.018	10154.9193709511\\
0.02	9148.95151387286\\
0.022	8325.45612419839\\
0.024	7638.81285543032\\
0.026	7057.43857854958\\
0.028	6558.77408758418\\
0.03	6126.27612287615\\
0.032	5747.53738557895\\
0.034	5413.07007640177\\
0.036	5115.49492140729\\
0.038	4848.98629488552\\
0.04	4608.88380534926\\
0.042	4391.41485693005\\
0.044	4193.49287583122\\
0.046	4012.56817334574\\
0.048	3846.51609310663\\
0.05	3693.55200298508\\
0.052	3552.16590423998\\
0.054	3421.07157195893\\
0.056	3299.16659397488\\
0.058	3185.50067758935\\
0.06	3079.25029495235\\
0.062	2979.69823578675\\
0.064	2886.21699397985\\
0.066	2798.25517479679\\
0.068	2715.32630082422\\
0.07	2636.99953689932\\
0.0720000000000001	2562.89196088804\\
0.0740000000000001	2492.66208785736\\
0.0760000000000001	2426.00441675098\\
0.0780000000000001	2362.64481604415\\
0.0800000000000001	2302.33660155764\\
0.0820000000000001	2244.85718825631\\
0.0840000000000001	2190.00522036931\\
0.0860000000000001	2137.59810196549\\
0.0880000000000001	2087.46986427689\\
0.0900000000000001	2039.46931738581\\
0.0920000000000001	1993.45844300453\\
0.0940000000000001	1949.31099243888\\
0.0960000000000001	1906.91125981596\\
0.0980000000000001	1866.1530055351\\
0.1	1826.93850891397\\
0.102	1789.1777322951\\
0.104	1752.78758161076\\
0.106	1717.69125066458\\
0.108	1683.81763827914\\
0.11	1651.10082903572\\
0.112	1619.47962965673\\
0.114	1588.89715419789\\
0.116	1559.30045215933\\
0.118	1530.64017442474\\
0.12	1502.87027261268\\
0.122	1475.94772800886\\
0.124	1449.83230673802\\
0.126	1424.48633826066\\
0.128	1399.87451464233\\
0.13	1375.96370836021\\
0.132	1352.72280667953\\
0.134	1330.12256086951\\
0.136	1308.13544873082\\
0.138	1286.73554908409\\
0.14	1265.89842702324\\
0.142	1245.60102887407\\
0.144	1225.82158591058\\
0.146	1206.53952599435\\
0.148	1187.73539238282\\
0.15	1169.39076904075\\
0.152	1151.48821185391\\
0.154	1134.01118520904\\
0.156	1116.94400345894\\
0.158	1100.27177683889\\
0.16	1083.98036144573\\
0.162	1068.05631292771\\
0.164	1052.4868435691\\
0.166	1037.25978248282\\
0.168	1022.36353865177\\
0.17	1007.78706658479\\
0.172	993.519834374295\\
0.174	979.551793961909\\
0.176	965.87335343676\\
0.178	952.475351206643\\
0.18	939.349031895895\\
0.182	926.486023837824\\
0.184	913.878318039306\\
0.186	901.51824850779\\
0.188	889.398473838701\\
0.19	877.5119599699\\
0.192	865.851964019118\\
0.194	854.412019125056\\
0.196	843.185920221292\\
0.198	832.167710676433\\
0.2	821.351669739919\\
0.202	810.732300738158\\
0.204	800.30431996835\\
0.206	790.062646243759\\
0.208	780.002391045669\\
0.21	770.118849241833\\
0.212	760.407490334089\\
0.214	750.863950200274\\
0.216	741.484023298285\\
0.218	732.263655302989\\
0.22	723.198936147935\\
0.222	714.286093446316\\
0.224	705.521486267841\\
0.226	696.901599248898\\
0.228	688.423037016096\\
0.23	680.082518903204\\
0.232	671.876873944783\\
0.234	663.803036129029\\
0.236	655.85803989486\\
0.238	648.039015858657\\
0.24	640.343186757458\\
0.242	632.767863595739\\
0.244	625.310441984073\\
0.246	617.968398658985\\
0.248	610.739288173591\\
0.25	603.620739749115\\
0.252	596.610454278764\\
0.254	589.706201475342\\
0.256	582.905817154599\\
0.258	576.207200646869\\
0.26	569.608312330416\\
0.262	563.107171279515\\
0.264	556.70185302128\\
0.266	550.390487395674\\
0.268	544.171256513011\\
0.27	538.042392803862\\
0.272	532.002177156795\\
0.274	526.048937139294\\
0.276	520.181045297459\\
0.278	514.396917530634\\
0.28	508.695011537404\\
0.282	503.073825328706\\
0.284	497.531895805581\\
0.286	492.067797397563\\
0.288	486.68014075932\\
0.29	481.367571522448\\
0.292	476.128769099519\\
0.294	470.962445538499\\
0.296	465.867344424376\\
0.298	460.842239826182\\
0.3	455.885935287365\\
0.302	450.997262857026\\
0.304	446.17508216047\\
0.306	441.418279507175\\
0.308	436.725767034289\\
0.31	432.096481884264\\
0.312	427.529385414912\\
0.314	423.023462440351\\
0.316	418.577720501681\\
0.318	414.191189165796\\
0.32	409.862919351264\\
0.322	405.591982679979\\
0.324	401.377470853392\\
0.326	397.218495052508\\
0.328	393.114185360181\\
0.33	389.063690205199\\
0.332	385.066175826818\\
0.334	381.120825759164\\
0.336	377.226840334501\\
0.338	373.383436204598\\
0.34	369.589845879387\\
0.342	365.8453172823\\
0.344	362.149113321457\\
0.346	358.50051147617\\
0.348	354.89880339796\\
0.35	351.343294525643\\
0.352	347.833303713906\\
0.354	344.368162874694\\
0.356	340.947216630973\\
0.358	337.569821982435\\
0.36	334.235347982475\\
0.362	330.943175426216\\
0.364	327.692696548984\\
0.366	324.483314734851\\
0.368	321.314444234905\\
0.37	318.185509894801\\
0.372	315.095946891256\\
0.374	312.045200477193\\
0.376	309.032725735071\\
0.378	306.057987338172\\
0.38	303.120459319648\\
0.382	300.219624848745\\
0.384	297.354976014244\\
0.386	294.526013614602\\
0.388	291.732246954766\\
0.39	288.973193649148\\
0.392	286.2483794308\\
0.394	283.557337966332\\
0.396	280.899610676611\\
0.398	278.274746562654\\
0.4	275.682302037013\\
0.402	273.121840759975\\
0.404	270.592933480764\\
0.406	268.09515788337\\
0.408	265.628098436861\\
0.41	263.191346250126\\
0.412	260.784498930762\\
0.414	258.407160447983\\
0.416	256.058940999551\\
0.418	253.739456882354\\
0.42	251.448330366693\\
0.422	249.185189574072\\
0.424	246.949668358382\\
0.426	244.741406190382\\
0.428	242.560048045259\\
0.43	240.405244293399\\
0.432	238.276650593977\\
0.434	236.17392779147\\
0.436	234.0967418149\\
0.438	232.044763579799\\
0.44	230.017668892686\\
0.442	228.015138358097\\
0.444	226.036857287985\\
0.446	224.082515613487\\
0.448	222.151807798947\\
0.45	220.244432758121\\
0.452	218.360093772486\\
0.454	216.49849841163\\
0.456	214.659358455583\\
0.458	212.842389819118\\
0.46	211.047312477908\\
0.462	209.273850396434\\
0.464	207.521731457688\\
0.466	205.790687394592\\
0.468	204.080453722985\\
0.47	202.39076967627\\
0.472	200.721378141564\\
0.474	199.072025597399\\
0.476	197.442462052776\\
0.478	195.832440987741\\
0.48	194.241719295241\\
0.482	192.670057224395\\
0.484	191.117218324942\\
0.486	189.58296939303\\
0.488	188.067080418182\\
0.49	186.569324531463\\
0.492	185.089477954765\\
0.494	183.627319951232\\
0.496	182.182632776746\\
0.498	180.755201632492\\
0.5	179.344814618506\\
0.502	177.951262688276\\
0.504	176.574339604258\\
0.506	175.213841894339\\
0.508	173.869568809271\\
0.51	172.541322280917\\
0.512	171.22890688144\\
0.514	169.932129783264\\
0.516	168.650800719931\\
0.518	167.38473194772\\
0.52	166.133738208032\\
0.522	164.89763669058\\
0.524	163.676246997293\\
0.526	162.469391106947\\
0.528	161.276893340525\\
0.53	160.098580327216\\
0.532	158.934280971142\\
0.534	157.783826418713\\
0.536	156.64705002662\\
0.538	155.523787330461\\
0.54	154.413876013975\\
0.542	153.3171558789\\
0.544	152.233468815348\\
0.546	151.162658772821\\
0.548	150.104571731766\\
0.55	149.059055675646\\
0.552	148.025960563573\\
0.554	147.005138303485\\
0.556	145.996442725768\\
0.558	144.999729557457\\
0.56	144.014856396886\\
0.562	143.041682688835\\
0.564	142.080069700121\\
0.566	141.129880495715\\
0.568	140.190979915276\\
0.57	139.263234550114\\
0.572	138.346512720641\\
0.574	137.440684454231\\
0.576	136.545621463496\\
0.578	135.661197125011\\
0.58	134.787286458439\\
0.582	133.923766106052\\
0.584	133.070514312666\\
0.586	132.227410905983\\
0.588	131.394337277312\\
0.59	130.57117636267\\
0.592	129.757812624305\\
0.594	128.95413203254\\
0.596	128.160022048047\\
0.598	127.375371604476\\
0.6	126.600071091438\\
0.602	125.834012337881\\
0.604	125.07708859582\\
0.606	124.329194524428\\
0.608	123.590226174519\\
0.61	122.860080973332\\
0.612	122.138657709768\\
0.614	121.425856519899\\
0.616	120.721578872893\\
0.618	120.025727557295\\
0.62	119.338206667668\\
0.622	118.658921591602\\
0.624	117.987778997105\\
0.626	117.324686820325\\
0.628	116.669554253723\\
0.63	116.022291734553\\
0.632	115.382810933783\\
0.634	114.751024745355\\
0.636	114.126847275922\\
0.638	113.510193834887\\
0.64	112.900980924942\\
0.642	112.299126232991\\
0.644	111.704548621495\\
0.646	111.117168120284\\
0.648	110.536905918798\\
0.65	109.963684358816\\
0.652	109.39742692763\\
0.654	108.838058251753\\
0.656	108.285504091093\\
0.658	107.739691333693\\
0.66	107.20054799099\\
0.662	106.668003193612\\
0.664	106.141987187849\\
0.666	105.622431332604\\
0.668	105.109268097105\\
0.67	104.602431059172\\
0.672	104.101854904234\\
0.674	103.607475425031\\
0.676	103.119229522088\\
0.678	102.637055204918\\
0.68	102.160891594148\\
0.682	101.690678924385\\
0.684	101.226358548051\\
0.686	100.767872940135\\
0.688000000000001	100.315165703954\\
0.690000000000001	99.8681815779299\\
0.692000000000001	99.4268664434465\\
0.694000000000001	98.9911673338705\\
0.696000000000001	98.5610324447685\\
0.698000000000001	98.1364111453501\\
0.700000000000001	97.7172539912469\\
0.702000000000001	97.3035127386758\\
0.704000000000001	96.8951403600302\\
0.706000000000001	96.4920910609743\\
0.708000000000001	96.0943202991229\\
0.710000000000001	95.701784804369\\
0.712000000000001	95.3144426008675\\
0.714000000000001	94.9322530308569\\
0.716000000000001	94.5551767802339\\
0.718000000000001	94.1831759060719\\
0.720000000000001	93.8162138660186\\
0.722000000000001	93.4542555497001\\
0.724000000000001	93.097267312082\\
0.726000000000001	92.7452170088584\\
0.728000000000001	92.39807403382\\
0.730000000000001	92.0558093581713\\
0.732000000000001	91.7183955717582\\
0.734000000000001	91.3858069260536\\
0.736000000000001	91.0580193787766\\
0.738000000000001	90.7350106399685\\
0.740000000000001	90.4167602191738\\
0.742000000000001	90.1032494734619\\
0.744000000000001	89.7944616557817\\
0.746000000000001	89.4903819631291\\
0.748000000000001	89.1909975838259\\
0.750000000000001	88.8962977431288\\
0.752000000000001	88.6062737460812\\
0.754000000000001	88.3209190165338\\
0.756000000000001	88.0402291308531\\
0.758000000000001	87.7642018446993\\
0.760000000000001	87.4928371109524\\
0.762000000000001	87.2261370866004\\
0.764000000000001	86.9641061260634\\
0.766000000000001	86.7067507580931\\
0.768000000000001	86.454079643118\\
0.770000000000001	86.2061035074235\\
0.772000000000001	85.9628350504619\\
0.774000000000001	85.7242888210714\\
0.776000000000001	85.4904810583057\\
0.778000000000001	85.2614294924837\\
0.780000000000001	85.0371531018965\\
0.782000000000001	84.8176718209828\\
0.784000000000001	84.6030061960629\\
0.786000000000001	84.3931769853793\\
0.788000000000001	84.1882047011783\\
0.790000000000001	83.9881090928266\\
0.792000000000001	83.7929085717323\\
0.794000000000001	83.6026195807298\\
0.796000000000001	83.4172559131614\\
0.798000000000001	83.2368279895453\\
0.800000000000001	83.0613421026707\\
0.802000000000001	82.8907996450024\\
0.804000000000001	82.7251963352484\\
0.806000000000001	82.5645214634908\\
0.808000000000001	82.4087571764857\\
0.810000000000001	82.2578778259047\\
0.812000000000001	82.1118494027018\\
0.814000000000001	81.9706290797557\\
0.816000000000001	81.8341648828726\\
0.818000000000001	81.7023955067136\\
0.820000000000001	81.5752502875556\\
0.822000000000001	81.452649339359\\
0.824000000000001	81.3345038533698\\
0.826000000000001	81.2207165551859\\
0.828000000000001	81.1111823072077\\
0.830000000000001	81.0057888388472\\
0.832000000000001	80.9044175825577\\
0.834000000000001	80.8069445906392\\
0.836000000000001	80.7132415059981\\
0.838000000000001	80.6231765598415\\
0.840000000000001	80.5366155703363\\
0.842000000000001	80.4534229184771\\
0.844000000000001	80.3734624804638\\
0.846000000000001	80.2965984995319\\
0.848000000000001	80.2226963841791\\
0.850000000000001	80.1516234235835\\
0.852000000000001	80.0832494148428\\
0.854000000000001	80.0174472000183\\
0.856000000000001	79.9540931138109\\
0.858000000000001	79.8930673451602\\
0.860000000000001	79.8342542177925\\
0.862000000000001	79.7775423960681\\
0.864000000000001	79.7228250233326\\
0.866000000000001	79.6699998003681\\
0.868000000000001	79.6189690115935\\
0.870000000000001	79.5696395066024\\
0.872000000000001	79.5219226441356\\
0.874000000000001	79.4757342051268\\
0.876000000000001	79.4309942809262\\
0.878000000000001	79.3876271420499\\
0.880000000000001	79.3455610923053\\
0.882000000000001	79.3047283124086\\
0.884000000000001	79.2650646966433\\
0.886000000000001	79.2265096855843\\
0.888000000000001	79.1890060974151\\
0.890000000000001	79.1524999598067\\
0.892000000000001	79.1169403440964\\
0.894000000000001	79.0822792030331\\
0.896000000000001	79.0484712131015\\
0.898000000000001	79.0154736221835\\
0.900000000000001	78.9832461030887\\
0.902000000000001	78.9517506133412\\
0.904000000000001	78.920951261398\\
0.906000000000001	78.8908141794281\\
0.908000000000001	78.8613074026431\\
0.910000000000001	78.8324007550918\\
0.912000000000001	78.8040657417824\\
0.914000000000001	78.7762754469995\\
0.916000000000001	78.7490044385068\\
0.918000000000001	78.7222286775\\
0.920000000000001	78.695925433985\\
0.922000000000001	78.6700732073862\\
0.924000000000001	78.644651652025\\
0.926000000000001	78.6196415073263\\
0.928000000000001	78.5950245323878\\
0.930000000000001	78.5707834447078\\
0.932000000000001	78.5469018628231\\
0.934000000000001	78.5233642526177\\
0.936000000000001	78.5001558770548\\
0.938000000000001	78.4772627491415\\
0.940000000000001	78.4546715879046\\
0.942000000000001	78.4323697771887\\
0.944000000000001	78.4103453271109\\
0.946000000000001	78.3885868379318\\
0.948000000000001	78.3670834662867\\
0.950000000000001	78.3458248935157\\
0.952000000000001	78.3248012960268\\
0.954000000000001	78.3040033175236\\
0.956000000000001	78.2834220429756\\
0.958000000000001	78.2630489742195\\
0.960000000000001	78.2428760070824\\
0.962000000000001	78.2228954099097\\
0.964000000000001	78.2030998034494\\
0.966000000000001	78.1834821419279\\
0.968000000000001	78.1640356953052\\
0.970000000000001	78.1447540326008\\
0.972000000000001	78.1256310062311\\
0.974000000000001	78.1066607372679\\
0.976000000000001	78.0878376015803\\
0.978000000000001	78.0691562168127\\
0.980000000000001	78.0506114301028\\
0.982000000000001	78.0321983065281\\
0.984000000000001	78.0139121182216\\
0.986000000000001	77.9957483341224\\
0.988000000000001	77.9777026102831\\
0.990000000000001	77.9597707807701\\
0.992000000000001	77.9419488490361\\
0.994000000000001	77.9242329797986\\
0.996000000000001	77.9066194913702\\
0.998000000000001	77.8891048484026\\
1	77.8716856550198\\
1.002	77.8543586483574\\
1.004	77.8371206924137\\
1.006	77.8199687722509\\
1.008	77.8028999885031\\
1.01	77.7859115521662\\
1.012	77.7690007796688\\
1.014	77.7521650881981\\
1.016	77.73540199126\\
1.018	77.7187090944843\\
1.02	77.7020840916213\\
1.022	77.6855247607529\\
1.024	77.6690289607006\\
1.026	77.6525946275914\\
1.028	77.6362197716219\\
1.03	77.6199024739526\\
1.032	77.6036408837717\\
1.034	77.5874332154915\\
1.036	77.5712777460997\\
1.038	77.5551728125957\\
1.04	77.5391168095955\\
1.042	77.5231081870149\\
1.044	77.5071454478814\\
1.046	77.4912271462325\\
1.048	77.4753518851201\\
1.05	77.4595183147088\\
1.052	77.44372513045\\
1.054	77.4279710713398\\
1.056	77.4122549182674\\
1.058	77.3965754924196\\
1.06	77.3809316537795\\
1.062	77.3653222996501\\
1.064	77.3497463633008\\
1.066	77.3342028126042\\
1.068	77.3186906487956\\
1.07	77.3032089052403\\
1.072	77.2877566462796\\
1.074	77.2723329661057\\
1.076	77.2569369876917\\
1.078	77.2415678617914\\
1.08	77.2262247659226\\
1.082	77.210906903449\\
1.084	77.1956135026649\\
1.086	77.1803438159404\\
1.088	77.1650971188776\\
1.09	77.1498727095146\\
1.092	77.1346699075701\\
1.094	77.1194880536897\\
1.096	77.1043265087464\\
1.098	77.0891846531636\\
1.1	77.0740618862582\\
1.102	77.0589576256028\\
1.104	77.0438713064472\\
1.106	77.0288023811084\\
1.108	77.0137503184234\\
1.11	76.9987146032132\\
1.112	76.9836947357585\\
1.114	76.9686902313079\\
1.116	76.9537006195904\\
1.118	76.938725444359\\
1.12	76.9237642629376\\
1.122	76.9088166457962\\
1.124	76.8938821761341\\
1.126	76.8789604494834\\
1.128	76.8640510733255\\
1.13	76.8491536667076\\
1.132	76.8342678599075\\
1.134	76.8193932940599\\
1.136	76.8045296208469\\
1.138	76.7896765021608\\
1.14	76.7748336097985\\
1.142	76.7600006251671\\
1.144	76.7451772389827\\
1.146	76.7303631510034\\
1.148	76.7155580697422\\
1.15	76.7007617122217\\
1.152	76.6859738037132\\
1.154	76.671194077496\\
1.156	76.6564222746172\\
1.158	76.6416581436675\\
1.16	76.6269014405618\\
1.162	76.6121519283183\\
1.164	76.5974093768552\\
1.166	76.5826735628031\\
1.168	76.5679442692871\\
1.17	76.5532212857607\\
1.172	76.5385044078218\\
1.174	76.5237934370199\\
1.176	76.5090881807096\\
1.178	76.4943884518697\\
1.18	76.4796940689542\\
1.182	76.4650048557308\\
1.184	76.4503206411322\\
1.186	76.4356412591187\\
1.188	76.4209665485215\\
1.19	76.4062963529277\\
1.192	76.3916305205337\\
1.194	76.3769689040184\\
1.196	76.3623113604224\\
1.198	76.347657751029\\
1.2	76.3330079412449\\
1.202	76.3183618004875\\
1.204	76.3037192020733\\
1.206	76.2890800231133\\
1.208	76.2744441444094\\
1.21	76.2598114503584\\
1.212	76.2451818288388\\
1.214	76.2305551711391\\
1.216	76.2159313718495\\
1.218	76.2013103287783\\
1.22	76.1866919428543\\
1.222	76.1720761180714\\
1.224	76.1574627613751\\
1.226	76.1428517825976\\
1.228	76.1282430943808\\
1.23	76.1136366121058\\
1.232	76.0990322538018\\
1.234	76.0844299400965\\
1.236	76.0698295941365\\
1.238	76.0552311415201\\
1.24	76.0406345102407\\
1.242	76.0260396306115\\
1.244	76.0114464352086\\
1.246	75.9968548588223\\
1.248	75.9822648383774\\
1.25	75.9676763128996\\
1.252	75.9530892234407\\
1.254	75.9385035130406\\
1.256	75.9239191266673\\
1.258	75.9093360111613\\
1.26	75.8947541152039\\
1.262	75.8801733892498\\
1.264	75.8655937854908\\
1.266	75.8510152578119\\
1.268	75.83643776174\\
1.27	75.8218612544019\\
1.272	75.807285694492\\
1.274	75.7927110422213\\
1.276	75.7781372592789\\
1.278	75.7635643087981\\
1.28	75.7489921553176\\
1.282	75.7344207647385\\
1.284	75.7198501042989\\
1.286	75.7052801425265\\
1.288	75.6907108492193\\
1.29	75.6761421954049\\
1.292	75.6615741533021\\
1.294	75.6470066963011\\
1.296	75.6324397989281\\
1.298	75.6178734368171\\
1.3	75.6033075866718\\
1.302	75.5887422262528\\
1.304	75.5741773343416\\
1.306	75.559612890707\\
1.308	75.5450488760968\\
1.31	75.5304852721929\\
1.312	75.5159220616013\\
1.314	75.5013592278195\\
1.316	75.4867967552162\\
1.318	75.472234629002\\
1.32	75.457672835225\\
1.322	75.4431113607238\\
1.324	75.4285501931281\\
1.326	75.4139893208204\\
1.328	75.3994287329277\\
1.33	75.3848684192971\\
1.332	75.3703083704834\\
1.334	75.3557485777113\\
1.336	75.3411890328814\\
1.338	75.3266297285348\\
1.34	75.3120706578402\\
1.342	75.2975118145889\\
1.344	75.2829531931516\\
1.346	75.2683947884904\\
1.348	75.2538365961246\\
1.35	75.2392786121249\\
1.352	75.2247208330907\\
1.354	75.2101632561428\\
1.356	75.1956058789052\\
1.358	75.1810486994891\\
1.36	75.1664917164828\\
1.362	75.1519349289426\\
1.364	75.137378336361\\
1.366	75.1228219386873\\
1.368	75.108265736277\\
1.37	75.0937097299083\\
1.372	75.0791539207578\\
1.374	75.0645983103901\\
1.376	75.0500429007489\\
1.378	75.0354876941454\\
1.38	75.0209326932474\\
1.382	75.0063779010628\\
1.384	74.9918233209469\\
1.386	74.9772689565703\\
1.388	74.9627148119211\\
1.39	74.9481608913012\\
1.392	74.9336071992997\\
1.394	74.9190537407993\\
1.396	74.9045005209679\\
1.398	74.8899475452299\\
1.4	74.8753948192834\\
1.402	74.8608423490772\\
1.404	74.8462901408092\\
1.406	74.8317382009091\\
1.408	74.8171865360412\\
1.41	74.8026351530956\\
1.412	74.7880840591736\\
1.414	74.7735332615929\\
1.416	74.758982767862\\
1.418	74.7444325856919\\
1.42	74.7298827229834\\
1.422	74.7153331878144\\
1.424	74.7007839884435\\
1.426	74.6862351332964\\
1.428	74.6716866309597\\
1.43	74.6571384901861\\
1.432	74.6425907198729\\
1.434	74.6280433290683\\
1.436	74.6134963269585\\
1.438	74.5989497228684\\
1.44	74.5844035262514\\
1.442	74.5698577466879\\
1.444	74.5553123938816\\
1.446	74.5407674776433\\
1.448	74.5262230079057\\
1.45	74.5116789947026\\
1.452	74.4971354481649\\
1.454	74.4825923785354\\
1.456	74.4680497961367\\
1.458	74.4535077113851\\
1.46	74.4389661347861\\
1.462	74.424425076923\\
1.464	74.4098845484537\\
1.466	74.3953445601156\\
1.468	74.3808051227116\\
1.47	74.3662662471157\\
1.472	74.3517279442593\\
1.474	74.3371902251376\\
1.476	74.3226531007999\\
1.478	74.3081165823497\\
1.48	74.2935806809367\\
1.482	74.2790454077628\\
1.484	74.2645107740686\\
1.486	74.2499767911391\\
1.488	74.2354434702962\\
1.49	74.220910822895\\
1.492	74.2063788603262\\
1.494	74.1918475940062\\
1.496	74.1773170353821\\
1.498	74.1627871959233\\
1.5	74.1482580871238\\
1.502	74.1337297204942\\
1.504	74.1192021075661\\
1.506	74.1046752598813\\
1.508	74.0901491890025\\
1.51	74.0756239064938\\
1.512	74.0610994239343\\
1.514	74.0465757529087\\
1.516	74.032052905006\\
1.518	74.0175308918161\\
1.52	74.0030097249311\\
1.522	73.9884894159441\\
1.524	73.9739699764453\\
1.526	73.9594514180136\\
1.528	73.9449337522302\\
1.53	73.9304169906672\\
1.532	73.9159011448818\\
1.534	73.9013862264263\\
1.536	73.886872246836\\
1.538	73.8723592176343\\
1.54	73.8578471503285\\
1.542	73.843336056412\\
1.544	73.8288259473581\\
1.546	73.814316834614\\
1.548	73.7998087296227\\
1.55	73.7853016437826\\
1.552	73.7707955884927\\
1.554	73.7562905751098\\
1.556	73.7417866149732\\
1.558	73.7272837193925\\
1.56	73.7127818996562\\
1.562	73.6982811670127\\
1.564	73.6837815326901\\
1.566	73.6692830078829\\
1.568	73.654785603756\\
1.57	73.6402893314371\\
1.572	73.6257942020221\\
1.574	73.611300226577\\
1.576	73.5968074161264\\
1.578	73.5823157816659\\
1.58	73.5678253341487\\
1.582	73.5533360844908\\
1.584	73.5388480435726\\
1.586	73.5243612222371\\
1.588	73.5098756312821\\
1.59	73.4953912814708\\
1.592	73.4809081835217\\
1.594	73.4664263481176\\
1.596	73.4519457858884\\
1.598	73.4374665074348\\
1.6	73.4229885233048\\
1.602	73.408511844006\\
1.604	73.3940364800046\\
1.606	73.3795624417151\\
1.608	73.3650897395135\\
1.61	73.3506183837246\\
1.612	73.3361483846371\\
1.614	73.3216797524809\\
1.616	73.3072124974448\\
1.618	73.2927466296762\\
1.62	73.2782821592643\\
1.622	73.2638190962563\\
1.624	73.2493574506537\\
1.626	73.2348972324021\\
1.628	73.2204384514048\\
1.63	73.2059811175122\\
1.632	73.1915252405258\\
1.634	73.1770708302039\\
1.636	73.1626178962455\\
1.638	73.1481664483023\\
1.64	73.1337164959799\\
1.642	73.1192680488306\\
1.644	73.1048211163527\\
1.646	73.0903757079989\\
1.648	73.0759318331704\\
1.65	73.0614895012134\\
1.652	73.0470487214269\\
1.654	73.0326095030555\\
1.656	73.0181718552946\\
1.658	73.0037357872839\\
1.66	72.9893013081153\\
1.662	72.9748684268329\\
1.664	72.9604371524159\\
1.666	72.9460074938028\\
1.668	72.931579459877\\
1.67	72.9171530594687\\
1.672	72.9027283013562\\
1.674	72.8883051942643\\
1.676	72.87388374687\\
1.678	72.8594639677937\\
1.68	72.845045865607\\
1.682	72.8306294488243\\
1.684	72.8162147259117\\
1.686	72.8018017052824\\
1.688	72.787390395299\\
1.69	72.7729808042663\\
1.692	72.7585729404441\\
1.694	72.7441668120378\\
1.696	72.7297624271917\\
1.698	72.715359794014\\
1.7	72.7009589205498\\
1.702	72.6865598147976\\
1.704	72.6721624846988\\
1.706	72.6577669381473\\
1.708	72.6433731829863\\
1.71	72.6289812270024\\
1.712	72.6145910779382\\
1.714	72.6002027434746\\
1.716	72.5858162312517\\
1.718	72.5714315488493\\
1.72	72.5570487038056\\
1.722	72.5426677035993\\
1.724	72.5282885556658\\
1.726	72.5139112673819\\
1.728	72.4995358460812\\
1.73	72.4851622990418\\
1.732	72.4707906334932\\
1.734	72.4564208566168\\
1.736	72.4420529755401\\
1.738	72.4276869973447\\
1.74	72.4133229290576\\
1.742	72.3989607776621\\
1.744	72.3846005500886\\
1.746	72.3702422532194\\
1.748	72.3558858938868\\
1.75	72.341531478873\\
1.752	72.327179014916\\
1.754	72.3128285087013\\
1.756	72.2984799668676\\
1.758	72.2841333960054\\
1.76	72.2697888026563\\
1.762	72.2554461933151\\
1.764	72.2411055744263\\
1.766	72.2267669523941\\
1.768	72.2124303335676\\
1.77	72.198095724251\\
1.772	72.1837631307044\\
1.774	72.1694325591371\\
1.776	72.1551040157162\\
1.778	72.1407775065601\\
1.78	72.1264530377393\\
1.782	72.1121306152844\\
1.784	72.0978102451738\\
1.786	72.0834919333438\\
1.788	72.0691756856855\\
1.79	72.0548615080425\\
1.792	72.0405494062181\\
1.794	72.0262393859673\\
1.796	72.011931453\\
1.798	71.9976256129831\\
1.8	71.9833218715425\\
1.802	71.9690202342588\\
1.804	71.9547207066633\\
1.806	71.9404232942544\\
1.808	71.9261280024768\\
1.81	71.9118348367385\\
1.812	71.8975438024076\\
1.814	71.8832549047984\\
1.816	71.8689681491956\\
1.818	71.8546835408351\\
1.82	71.8404010849119\\
1.822	71.8261207865806\\
1.824	71.811842650953\\
1.826	71.7975666831012\\
1.828	71.7832928880553\\
1.83	71.7690212708063\\
1.832	71.7547518363028\\
1.834	71.7404845894514\\
1.836	71.7262195351226\\
1.838	71.7119566781472\\
1.84	71.6976960233173\\
1.842	71.6834375753768\\
1.844	71.6691813390428\\
1.846	71.6549273189854\\
1.848	71.6406755198358\\
1.85	71.6264259461948\\
1.852	71.612178602614\\
1.854	71.5979334936149\\
1.856	71.583690623679\\
1.858	71.5694499972509\\
1.86	71.5552116187358\\
1.862	71.5409754925019\\
1.864	71.5267416228828\\
1.866	71.5125100141731\\
1.868	71.4982806706324\\
1.87	71.4840535964873\\
1.872	71.4698287959211\\
1.874	71.4556062730875\\
1.876	71.4413860320986\\
1.878	71.4271680770371\\
1.88	71.4129524119482\\
1.882	71.3987390408439\\
1.884	71.3845279676974\\
1.886	71.3703191964527\\
1.888	71.3561127310144\\
1.89	71.341908575258\\
1.892	71.3277067330197\\
1.894	71.3135072081078\\
1.896	71.2993100042916\\
1.898	71.2851151253145\\
1.9	71.2709225748799\\
1.902	71.2567323566619\\
1.904	71.2425444743023\\
1.906	71.2283589314072\\
1.908	71.2141757315571\\
1.91	71.1999948782951\\
1.912	71.1858163751325\\
1.914	71.1716402255531\\
1.916	71.157466433006\\
1.918	71.1432950009115\\
1.92	71.1291259326571\\
1.922	71.1149592316031\\
1.924	71.1007949010748\\
1.926	71.0866329443695\\
1.928	71.0724733647563\\
1.93	71.0583161654707\\
1.932	71.0441613497211\\
1.934	71.0300089206867\\
1.936	71.0158588815166\\
1.938	71.0017112353313\\
1.94	70.9875659852218\\
1.942	70.9734231342519\\
1.944	70.9592826854547\\
1.946	70.9451446418385\\
1.948	70.9310090063806\\
1.95	70.9168757820313\\
1.952	70.9027449717148\\
1.954	70.8886165783254\\
1.956	70.8744906047313\\
1.958	70.8603670537753\\
1.96	70.8462459282707\\
1.962	70.8321272310064\\
1.964	70.8180109647415\\
1.966	70.8038971322129\\
1.968	70.7897857361298\\
1.97	70.7756767791757\\
1.972	70.7615702640056\\
1.974	70.7474661932539\\
1.976	70.7333645695259\\
1.978	70.7192653954031\\
1.98	70.705168673443\\
1.982	70.6910744061765\\
1.984	70.6769825961097\\
1.986	70.6628932457258\\
1.988	70.6488063574818\\
1.99	70.6347219338157\\
1.992	70.6206399771341\\
1.994	70.6065604898269\\
1.996	70.5924834742553\\
1.998	70.5784089327561\\
2	70.5643368676494\\
2.002	70.5502672812287\\
2.004	70.5362001757633\\
2.006	70.5221355535006\\
2.008	70.5080734166672\\
2.01	70.4940137674661\\
2.012	70.4799566080774\\
2.014	70.4659019406582\\
2.016	70.4518497673487\\
2.018	70.4378000902628\\
2.02	70.423752911492\\
2.022	70.4097082331121\\
2.024	70.3956660571698\\
2.026	70.3816263856994\\
2.028	70.3675892207095\\
2.03	70.3535545641851\\
2.032	70.3395224180977\\
2.034	70.3254927843906\\
2.036	70.3114656649949\\
2.038	70.2974410618151\\
2.04	70.2834189767402\\
2.042	70.2693994116355\\
2.044	70.2553823683495\\
2.046	70.2413678487094\\
2.048	70.2273558545248\\
2.05	70.2133463875838\\
2.052	70.199339449658\\
2.054	70.1853350424977\\
2.056	70.1713331678381\\
2.05799999999999	70.1573338273922\\
2.05999999999999	70.1433370228537\\
2.06199999999999	70.129342755902\\
2.06399999999999	70.1153510281942\\
2.06599999999999	70.1013618413737\\
2.06799999999999	70.0873751970619\\
2.06999999999999	70.0733910968651\\
2.07199999999999	70.0594095423706\\
2.07399999999999	70.0454305351469\\
2.07599999999999	70.0314540767499\\
2.07799999999999	70.017480168713\\
2.07999999999999	70.0035088125549\\
2.08199999999999	69.989540009779\\
2.08399999999999	69.9755737618673\\
2.08599999999999	69.9616100702895\\
2.08799999999999	69.9476489364981\\
2.08999999999999	69.9336903619262\\
2.09199999999999	69.9197343479936\\
2.09399999999999	69.9057808960997\\
2.09599999999999	69.8918300076362\\
2.09799999999999	69.877881683971\\
2.09999999999999	69.8639359264561\\
2.10199999999999	69.8499927364388\\
2.10399999999999	69.8360521152326\\
2.10599999999999	69.8221140641535\\
2.10799999999999	69.8081785844909\\
2.10999999999999	69.7942456775223\\
2.11199999999999	69.7803153445115\\
2.11399999999999	69.7663875867057\\
2.11599999999999	69.7524624053369\\
2.11799999999999	69.7385398016265\\
2.11999999999999	69.724619776774\\
2.12199999999999	69.7107023319713\\
2.12399999999999	69.6967874683926\\
2.12599999999999	69.6828751871971\\
2.12799999999999	69.6689654895325\\
2.12999999999999	69.6550583765315\\
2.13199999999999	69.6411538493121\\
2.13399999999999	69.6272519089794\\
2.13599999999999	69.6133525566237\\
2.13799999999999	69.5994557933238\\
2.13999999999999	69.5855616201422\\
2.14199999999999	69.5716700381326\\
2.14399999999999	69.5577810483285\\
2.14599999999999	69.5438946517569\\
2.14799999999999	69.5300108494297\\
2.14999999999998	69.5161296423417\\
2.15199999999998	69.5022510314825\\
2.15399999999998	69.4883750178247\\
2.15599999999998	69.4745016023258\\
2.15799999999998	69.460630785936\\
2.15999999999998	69.4467625695907\\
2.16199999999998	69.4328969542133\\
2.16399999999998	69.4190339407117\\
2.16599999999998	69.4051735299876\\
2.16799999999998	69.3913157229293\\
2.16999999999998	69.377460520407\\
2.17199999999998	69.3636079232886\\
2.17399999999998	69.3497579324209\\
2.17599999999998	69.3359105486496\\
2.17799999999998	69.3220657727982\\
2.17999999999998	69.3082236056841\\
2.18199999999998	69.294384048116\\
2.18399999999998	69.2805471008849\\
2.18599999999998	69.2667127647755\\
2.18799999999998	69.2528810405586\\
2.18999999999998	69.2390519289987\\
2.19199999999998	69.2252254308444\\
2.19399999999998	69.2114015468311\\
2.19599999999998	69.1975802776928\\
2.19799999999998	69.1837616241462\\
2.19999999999998	69.1699455869006\\
2.20199999999998	69.1561321666495\\
2.20399999999998	69.1423213640828\\
2.20599999999998	69.1285131798748\\
2.20799999999998	69.1147076146955\\
2.20999999999998	69.1009046691999\\
2.21199999999998	69.0871043440317\\
2.21399999999998	69.0733066398329\\
2.21599999999998	69.0595115572243\\
2.21799999999998	69.0457190968276\\
2.21999999999998	69.0319292592459\\
2.22199999999998	69.0181420450799\\
2.22399999999998	69.0043574549175\\
2.22599999999998	68.990575489334\\
2.22799999999998	68.9767961489039\\
2.22999999999998	68.9630194341851\\
2.23199999999998	68.9492453457247\\
2.23399999999998	68.9354738840697\\
2.23599999999998	68.9217050497518\\
2.23799999999998	68.9079388432928\\
2.23999999999997	68.8941752652083\\
2.24199999999997	68.8804143160047\\
2.24399999999997	68.8666559961789\\
2.24599999999997	68.85290030622\\
2.24799999999997	68.8391472466065\\
2.24999999999997	68.8253968178124\\
2.25199999999997	68.8116490202979\\
2.25399999999997	68.7979038545181\\
2.25599999999997	68.7841613209209\\
2.25799999999997	68.7704214199426\\
2.25999999999997	68.7566841520146\\
2.26199999999997	68.7429495175578\\
2.26399999999997	68.7292175169849\\
2.26599999999997	68.7154881507032\\
2.26799999999997	68.7017614191095\\
2.26999999999997	68.6880373225944\\
2.27199999999997	68.6743158615399\\
2.27399999999997	68.66059703632\\
2.27599999999997	68.6468808473013\\
2.27799999999997	68.6331672948436\\
2.27999999999997	68.6194563793003\\
2.28199999999997	68.6057481010112\\
2.28399999999997	68.5920424603178\\
2.28599999999997	68.5783394575474\\
2.28799999999997	68.564639093023\\
2.28999999999997	68.5509413670574\\
2.29199999999997	68.5372462799629\\
2.29399999999997	68.5235538320381\\
2.29599999999997	68.509864023575\\
2.29799999999997	68.4961768548635\\
2.29999999999997	68.4824923261814\\
2.30199999999997	68.468810437803\\
2.30399999999997	68.4551311899954\\
2.30599999999997	68.4414545830167\\
2.30799999999997	68.4277806171186\\
2.30999999999997	68.4141092925512\\
2.31199999999997	68.4004406095514\\
2.31399999999997	68.3867745683542\\
2.31599999999997	68.3731111691837\\
2.31799999999997	68.3594504122639\\
2.31999999999997	68.3457922978075\\
2.32199999999997	68.3321368260204\\
2.32399999999997	68.3184839971073\\
2.32599999999997	68.3048338112611\\
2.32799999999997	68.2911862686732\\
2.32999999999996	68.2775413695252\\
2.33199999999996	68.2638991139945\\
2.33399999999996	68.2502595022528\\
2.33599999999996	68.2366225344662\\
2.33799999999996	68.22298821079\\
2.33999999999996	68.2093565313856\\
2.34199999999996	68.1957274963944\\
2.34399999999996	68.1821011059592\\
2.34599999999996	68.168477360218\\
2.34799999999996	68.1548562593023\\
2.34999999999996	68.141237803338\\
2.35199999999996	68.1276219924413\\
2.35399999999996	68.1140088267285\\
2.35599999999996	68.1003983063086\\
2.35799999999996	68.0867904312863\\
2.35999999999996	68.0731852017565\\
2.36199999999996	68.0595826178152\\
2.36399999999996	68.0459826795479\\
2.36599999999996	68.0323853870388\\
2.36799999999996	68.0187907403631\\
2.36999999999996	68.0051987395939\\
2.37199999999996	67.9916093847985\\
2.37399999999996	67.9780226760403\\
2.37599999999996	67.9644386133743\\
2.37799999999996	67.950857196853\\
2.37999999999996	67.9372784265238\\
2.38199999999996	67.9237023024301\\
2.38399999999996	67.9101288246073\\
2.38599999999996	67.8965579930905\\
2.38799999999996	67.882989807907\\
2.38999999999996	67.8694242690803\\
2.39199999999996	67.8558613766275\\
2.39399999999996	67.8423011305643\\
2.39599999999996	67.8287435309008\\
2.39799999999996	67.8151885776415\\
2.39999999999996	67.8016362707855\\
2.40199999999996	67.7880866103311\\
2.40399999999996	67.7745395962679\\
2.40599999999996	67.7609952285861\\
2.40799999999996	67.7474535072668\\
2.40999999999996	67.7339144322884\\
2.41199999999996	67.720378003626\\
2.41399999999996	67.7068442212497\\
2.41599999999996	67.6933130851266\\
2.41799999999996	67.6797845952166\\
2.41999999999996	67.6662587514785\\
2.42199999999995	67.6527355538658\\
2.42399999999995	67.6392150023293\\
2.42599999999995	67.6256970968117\\
2.42799999999995	67.6121818372593\\
2.42999999999995	67.598669223605\\
2.43199999999995	67.5851592557854\\
2.43399999999995	67.5716519337297\\
2.43599999999995	67.5581472573648\\
2.43799999999995	67.5446452266114\\
2.43999999999995	67.5311458413908\\
2.44199999999995	67.5176491016149\\
2.44399999999995	67.5041550071969\\
2.44599999999995	67.4906635580431\\
2.44799999999995	67.4771747540574\\
2.44999999999995	67.4636885951401\\
2.45199999999995	67.4502050811882\\
2.45399999999995	67.4367242120947\\
2.45599999999995	67.4232459877493\\
2.45799999999995	67.4097704080385\\
2.45999999999995	67.3962974728423\\
2.46199999999995	67.3828271820442\\
2.46399999999995	67.3693595355161\\
2.46599999999995	67.3558945331322\\
2.46799999999995	67.3424321747621\\
2.46999999999995	67.328972460271\\
2.47199999999995	67.3155153895198\\
2.47399999999995	67.3020609623714\\
2.47599999999995	67.2886091786796\\
2.47799999999995	67.2751600382978\\
2.47999999999995	67.2617135410756\\
2.48199999999995	67.2482696868598\\
2.48399999999995	67.2348284754931\\
2.48599999999995	67.221389906816\\
2.48799999999995	67.2079539806654\\
2.48999999999995	67.1945206968805\\
2.49199999999995	67.1810900552848\\
2.49399999999995	67.1676620557132\\
2.49599999999995	67.1542366979869\\
2.49799999999995	67.1408139819304\\
2.49999999999995	67.1273939073623\\
2.50199999999995	67.1139764740996\\
2.50399999999995	67.1005616819567\\
2.50599999999995	67.0871495307468\\
2.50799999999995	67.0737400202746\\
2.50999999999995	67.0603331503438\\
2.51199999999994	67.0469289207626\\
2.51399999999994	67.0335273313296\\
2.51599999999994	67.0201283818399\\
2.51799999999994	67.0067320720888\\
2.51999999999994	66.9933384018694\\
2.52199999999994	66.979947370972\\
2.52399999999994	66.96655897918\\
2.52599999999994	66.9531732262807\\
2.52799999999994	66.9397901120531\\
2.52999999999994	66.9264096362794\\
2.53199999999994	66.9130317987343\\
2.53399999999994	66.8996565991917\\
2.53599999999994	66.886284037424\\
2.53799999999994	66.8729141132009\\
2.53999999999994	66.8595468262866\\
2.54199999999994	66.8461821764484\\
2.54399999999994	66.8328201634469\\
2.54599999999994	66.8194607870441\\
2.54799999999994	66.8061040469937\\
2.54999999999994	66.792749943051\\
2.55199999999994	66.7793984749696\\
2.55399999999994	66.7660496425024\\
2.55599999999994	66.7527034453925\\
2.55799999999994	66.7393598833909\\
2.55999999999994	66.7260189562371\\
2.56199999999994	66.7126806636754\\
2.56399999999994	66.6993450054437\\
2.56599999999994	66.6860119812775\\
2.56799999999994	66.6726815909159\\
2.56999999999994	66.6593538340899\\
2.57199999999994	66.646028710529\\
2.57399999999994	66.632706219963\\
2.57599999999994	66.619386362118\\
2.57799999999994	66.6060691367203\\
2.57999999999994	66.5927545434907\\
2.58199999999994	66.5794425821497\\
2.58399999999994	66.5661332524157\\
2.58599999999994	66.5528265540079\\
2.58799999999994	66.5395224866363\\
2.58999999999994	66.5262210500176\\
2.59199999999994	66.5129222438595\\
2.59399999999994	66.4996260678747\\
2.59599999999994	66.4863325217667\\
2.59799999999994	66.473041605242\\
2.59999999999994	66.4597533180031\\
2.60199999999994	66.4464676597509\\
2.60399999999993	66.4331846301885\\
2.60599999999993	66.4199042290092\\
2.60799999999993	66.4066264559114\\
2.60999999999993	66.3933513105884\\
2.61199999999993	66.3800787927348\\
2.61399999999993	66.3668089020375\\
2.61599999999993	66.3535416381886\\
2.61799999999993	66.3402770008731\\
2.61999999999993	66.3270149897786\\
2.62199999999993	66.3137556045875\\
2.62399999999993	66.3004988449845\\
2.62599999999993	66.2872447106477\\
2.62799999999993	66.273993201255\\
2.62999999999993	66.2607443164882\\
2.63199999999993	66.2474980560167\\
2.63399999999993	66.2342544195215\\
2.63599999999993	66.2210134066697\\
2.63799999999993	66.2077750171351\\
2.63999999999993	66.194539250587\\
2.64199999999993	66.1813061066921\\
2.64399999999993	66.1680755851176\\
2.64599999999993	66.1548476855294\\
2.64799999999993	66.1416224075883\\
2.64999999999993	66.1283997509564\\
2.65199999999993	66.1151797152985\\
2.65399999999993	66.1019623002674\\
2.65599999999993	66.0887475055265\\
2.65799999999993	66.075535330728\\
2.65999999999993	66.0623257755278\\
2.66199999999993	66.0491188395781\\
2.66399999999993	66.0359145225347\\
2.66599999999993	66.0227128240429\\
2.66799999999993	66.0095137437543\\
2.66999999999993	65.9963172813171\\
2.67199999999993	65.9831234363789\\
2.67399999999993	65.9699322085812\\
2.67599999999993	65.9567435975708\\
2.67799999999993	65.943557602991\\
2.67999999999993	65.9303742244786\\
2.68199999999993	65.9171934616787\\
2.68399999999993	65.9040153142286\\
2.68599999999993	65.8908397817645\\
2.68799999999993	65.8776668639216\\
2.68999999999993	65.8644965603371\\
2.69199999999993	65.8513288706443\\
2.69399999999992	65.8381637944756\\
2.69599999999992	65.8250013314618\\
2.69799999999992	65.8118414812327\\
2.69999999999992	65.7986842434197\\
2.70199999999992	65.7855296176471\\
2.70399999999992	65.7723776035466\\
2.70599999999992	65.7592282007386\\
2.70799999999992	65.7460814088506\\
2.70999999999992	65.7329372275034\\
2.71199999999992	65.7197956563215\\
2.71399999999992	65.7066566949254\\
2.71599999999992	65.6935203429338\\
2.71799999999992	65.6803865999661\\
2.71999999999992	65.6672554656394\\
2.72199999999992	65.654126939573\\
2.72399999999992	65.6410010213787\\
2.72599999999992	65.6278777106753\\
2.72799999999992	65.6147570070758\\
2.72999999999992	65.6016389101884\\
2.73199999999992	65.5885234196286\\
2.73399999999992	65.575410535008\\
2.73599999999992	65.5623002559326\\
2.73799999999992	65.5491925820137\\
2.73999999999992	65.5360875128544\\
2.74199999999992	65.5229850480672\\
2.74399999999992	65.5098851872539\\
2.74599999999992	65.496787930022\\
2.74799999999992	65.4836932759721\\
2.74999999999992	65.4706012247101\\
2.75199999999992	65.4575117758352\\
2.75399999999992	65.4444249289526\\
2.75599999999992	65.4313406836581\\
2.75799999999992	65.4182590395514\\
2.75999999999992	65.4051799962309\\
2.76199999999992	65.3921035532952\\
2.76399999999992	65.3790297103417\\
2.76599999999992	65.3659584669628\\
2.76799999999992	65.352889822756\\
2.76999999999992	65.3398237773157\\
2.77199999999992	65.3267603302323\\
2.77399999999992	65.3136994810993\\
2.77599999999992	65.3006412295094\\
2.77799999999992	65.2875855750518\\
2.77999999999992	65.2745325173162\\
2.78199999999992	65.2614820558941\\
2.78399999999991	65.2484341903718\\
2.78599999999991	65.2353889203334\\
2.78799999999991	65.2223462453722\\
2.78999999999991	65.2093061650703\\
2.79199999999991	65.1962686790124\\
2.79399999999991	65.1832337867844\\
2.79599999999991	65.1702014879698\\
2.79799999999991	65.15717178215\\
2.79999999999991	65.144144668908\\
2.80199999999991	65.1311201478253\\
2.80399999999991	65.1180982184839\\
2.80599999999991	65.1050788804612\\
2.80799999999991	65.0920621333379\\
2.80999999999991	65.0790479766925\\
2.81199999999991	65.0660364101025\\
2.81399999999991	65.0530274331449\\
2.81599999999991	65.0400210453968\\
2.81799999999991	65.0270172464335\\
2.81999999999991	65.014016035831\\
2.82199999999991	65.001017413162\\
2.82399999999991	64.9880213780022\\
2.82599999999991	64.9750279299231\\
2.82799999999991	64.962037068498\\
2.82999999999991	64.9490487933006\\
2.83199999999991	64.936063103898\\
2.83399999999991	64.9230799998648\\
2.83599999999991	64.9100994807691\\
2.83799999999991	64.8971215461809\\
2.83999999999991	64.8841461956677\\
2.84199999999991	64.8711734287987\\
2.84399999999991	64.8582032451418\\
2.84599999999991	64.8452356442637\\
2.84799999999991	64.8322706257287\\
2.84999999999991	64.819308189107\\
2.85199999999991	64.8063483339604\\
2.85399999999991	64.7933910598554\\
2.85599999999991	64.7804363663542\\
2.85799999999991	64.7674842530226\\
2.85999999999991	64.754534719421\\
2.86199999999991	64.7415877651154\\
2.86399999999991	64.7286433896643\\
2.86599999999991	64.7157015926315\\
2.86799999999991	64.7027623735756\\
2.86999999999991	64.6898257320579\\
2.87199999999991	64.6768916676403\\
2.87399999999991	64.6639601798794\\
2.8759999999999	64.6510312683334\\
2.8779999999999	64.6381049325639\\
2.8799999999999	64.6251811721268\\
2.8819999999999	64.6122599865778\\
2.8839999999999	64.5993413754774\\
2.8859999999999	64.5864253383792\\
2.8879999999999	64.5735118748381\\
2.8899999999999	64.5606009844134\\
2.8919999999999	64.5476926666568\\
2.8939999999999	64.5347869211229\\
2.8959999999999	64.5218837473666\\
2.8979999999999	64.5089831449418\\
2.8999999999999	64.4960851134014\\
2.9019999999999	64.4831896522957\\
2.9039999999999	64.4702967611803\\
2.9059999999999	64.4574064396054\\
2.9079999999999	64.4445186871217\\
2.9099999999999	64.4316335032809\\
2.9119999999999	64.4187508876319\\
2.9139999999999	64.4058708397268\\
2.9159999999999	64.3929933591141\\
2.9179999999999	64.3801184453454\\
2.9199999999999	64.3672460979648\\
2.9219999999999	64.3543763165246\\
2.9239999999999	64.3415091005719\\
2.9259999999999	64.3286444496542\\
2.9279999999999	64.3157823633162\\
2.9299999999999	64.3029228411087\\
2.9319999999999	64.2900658825764\\
2.9339999999999	64.2772114872644\\
2.9359999999999	64.2643596547204\\
2.9379999999999	64.2515103844864\\
2.9399999999999	64.2386636761129\\
2.9419999999999	64.2258195291396\\
2.9439999999999	64.2129779431124\\
2.9459999999999	64.2001389175734\\
2.9479999999999	64.1873024520686\\
2.9499999999999	64.1744685461421\\
2.9519999999999	64.1616371993341\\
2.9539999999999	64.1488084111854\\
2.9559999999999	64.1359821812441\\
2.9579999999999	64.1231585090462\\
2.9599999999999	64.1103373941379\\
2.9619999999999	64.0975188360564\\
2.9639999999999	64.0847028343442\\
2.96599999999989	64.0718893885414\\
2.96799999999989	64.0590784981891\\
2.96999999999989	64.0462701628256\\
2.97199999999989	64.0334643819923\\
2.97399999999989	64.0206611552274\\
2.97599999999989	64.007860482071\\
2.97799999999989	63.9950623620604\\
2.97999999999989	63.9822667947342\\
2.98199999999989	63.9694737796308\\
2.98399999999989	63.9566833162867\\
2.98599999999989	63.9438954042421\\
2.98799999999989	63.9311100430334\\
2.98999999999989	63.9183272321932\\
2.99199999999989	63.905546971265\\
2.99399999999989	63.8927692597825\\
2.99599999999989	63.8799940972799\\
2.99799999999989	63.8672214832952\\
2.99999999999989	63.8544514173608\\
3.00199999999989	63.8416838990156\\
3.00399999999989	63.8289189277941\\
3.00599999999989	63.8161565032305\\
3.00799999999989	63.8033966248598\\
3.00999999999989	63.7906392922148\\
3.01199999999989	63.7778845048309\\
3.01399999999989	63.7651322622424\\
3.01599999999989	63.7523825639808\\
3.01799999999989	63.7396354095808\\
3.01999999999989	63.7268907985754\\
3.02199999999989	63.7141487304971\\
3.02399999999989	63.7014092048797\\
3.02599999999989	63.6886722212558\\
3.02799999999989	63.6759377791544\\
3.02999999999989	63.6632058781092\\
3.03199999999989	63.6504765176537\\
3.03399999999989	63.6377496973173\\
3.03599999999989	63.6250254166323\\
3.03799999999989	63.6123036751291\\
3.03999999999989	63.599584472339\\
3.04199999999989	63.5868678077906\\
3.04399999999989	63.5741536810189\\
3.04599999999989	63.5614420915526\\
3.04799999999989	63.5487330389174\\
3.04999999999989	63.5360265226504\\
3.05199999999989	63.5233225422749\\
3.05399999999989	63.5106210973217\\
3.05599999999989	63.4979221873227\\
3.05799999999988	63.4852258118045\\
3.05999999999988	63.4725319702965\\
3.06199999999988	63.4598406623299\\
3.06399999999988	63.4471518874313\\
3.06599999999988	63.4344656451278\\
3.06799999999988	63.4217819349485\\
3.06999999999988	63.4091007564211\\
3.07199999999988	63.3964221090759\\
3.07399999999988	63.3837459924393\\
3.07599999999988	63.3710724060366\\
3.07799999999988	63.3584013493967\\
3.07999999999988	63.3457328220483\\
3.08199999999988	63.3330668235155\\
3.08399999999988	63.3204033533266\\
3.08599999999988	63.307742411007\\
3.08799999999988	63.2950839960853\\
3.08999999999988	63.2824281080877\\
3.09199999999988	63.2697747465407\\
3.09399999999988	63.2571239109682\\
3.09599999999988	63.2444756008983\\
3.09799999999988	63.2318298158542\\
3.09999999999988	63.2191865553643\\
3.10199999999988	63.2065458189532\\
3.10399999999988	63.1939076061452\\
3.10599999999988	63.1812719164664\\
3.10799999999988	63.1686387494433\\
3.10999999999988	63.1560081045981\\
3.11199999999988	63.1433799814591\\
3.11399999999988	63.1307543795459\\
3.11599999999988	63.1181312983877\\
3.11799999999988	63.1055107375064\\
3.11999999999988	63.0928926964286\\
3.12199999999988	63.0802771746745\\
3.12399999999988	63.0676641717746\\
3.12599999999988	63.0550536872466\\
3.12799999999988	63.0424457206165\\
3.12999999999988	63.0298402714103\\
3.13199999999988	63.0172373391471\\
3.13399999999988	63.0046369233539\\
3.13599999999988	62.9920390235531\\
3.13799999999988	62.9794436392675\\
3.13999999999988	62.9668507700215\\
3.14199999999988	62.9542604153367\\
3.14399999999988	62.9416725747362\\
3.14599999999988	62.9290872477451\\
3.14799999999987	62.9165044338823\\
3.14999999999987	62.9039241326752\\
3.15199999999987	62.8913463436411\\
3.15399999999987	62.8787710663063\\
3.15599999999987	62.8661983001911\\
3.15799999999987	62.8536280448199\\
3.15999999999987	62.8410602997122\\
3.16199999999987	62.8284950643926\\
3.16399999999987	62.8159323383815\\
3.16599999999987	62.8033721212007\\
3.16799999999987	62.7908144123733\\
3.16999999999987	62.7782592114204\\
3.17199999999987	62.7657065178638\\
3.17399999999987	62.7531563312236\\
3.17599999999987	62.7406086510241\\
3.17799999999987	62.7280634767844\\
3.17999999999987	62.7155208080272\\
3.18199999999987	62.7029806442738\\
3.18399999999987	62.6904429850431\\
3.18599999999987	62.6779078298584\\
3.18799999999987	62.6653751782416\\
3.18999999999987	62.6528450297112\\
3.19199999999987	62.6403173837904\\
3.19399999999987	62.6277922399971\\
3.19599999999987	62.6152695978564\\
3.19799999999987	62.6027494568846\\
3.19999999999987	62.590231816605\\
3.20199999999987	62.5777166765386\\
3.20399999999987	62.5652040362034\\
3.20599999999987	62.5526938951208\\
3.20799999999987	62.5401862528121\\
3.20999999999987	62.5276811087975\\
3.21199999999987	62.5151784625974\\
3.21399999999987	62.5026783137312\\
3.21599999999987	62.4901806617206\\
3.21799999999987	62.4776855060838\\
3.21999999999987	62.4651928463411\\
3.22199999999987	62.4527026820135\\
3.22399999999987	62.4402150126217\\
3.22599999999987	62.4277298376849\\
3.22799999999987	62.4152471567223\\
3.22999999999987	62.4027669692525\\
3.23199999999987	62.3902892747991\\
3.23399999999987	62.3778140728778\\
3.23599999999987	62.3653413630113\\
3.23799999999986	62.3528711447185\\
3.23999999999986	62.340403417518\\
3.24199999999986	62.3279381809319\\
3.24399999999986	62.3154754344761\\
3.24599999999986	62.3030151776706\\
3.24799999999986	62.2905574100398\\
3.24999999999986	62.2781021310977\\
3.25199999999986	62.2656493403659\\
3.25399999999986	62.2531990373653\\
3.25599999999986	62.2407512216109\\
3.25799999999986	62.2283058926251\\
3.25999999999986	62.2158630499273\\
3.26199999999986	62.2034226930365\\
3.26399999999986	62.1909848214702\\
3.26599999999986	62.1785494347509\\
3.26799999999986	62.1661165323946\\
3.26999999999986	62.1536861139225\\
3.27199999999986	62.1412581788525\\
3.27399999999986	62.1288327267036\\
3.27599999999986	62.1164097569964\\
3.27799999999986	62.1039892692494\\
3.27999999999986	62.0915712629808\\
3.28199999999986	62.0791557377099\\
3.28399999999986	62.0667426929565\\
3.28599999999986	62.0543321282382\\
3.28799999999986	62.0419240430755\\
3.28999999999986	62.0295184369874\\
3.29199999999986	62.0171153094903\\
3.29399999999986	62.0047146601067\\
3.29599999999986	61.9923164883536\\
3.29799999999986	61.9799207937487\\
3.29999999999986	61.9675275758129\\
3.30199999999986	61.9551368340654\\
3.30399999999986	61.9427485680229\\
3.30599999999986	61.9303627772065\\
3.30799999999986	61.9179794611337\\
3.30999999999986	61.9055986193245\\
3.31199999999986	61.8932202512963\\
3.31399999999986	61.8808443565683\\
3.31599999999986	61.8684709346603\\
3.31799999999986	61.856099985091\\
3.31999999999986	61.8437315073778\\
3.32199999999986	61.8313655010406\\
3.32399999999986	61.8190019655996\\
3.32599999999986	61.8066409005701\\
3.32799999999986	61.7942823054739\\
3.32999999999985	61.7819261798291\\
3.33199999999985	61.7695725231526\\
3.33399999999985	61.7572213349663\\
3.33599999999985	61.7448726147884\\
3.33799999999985	61.7325263621356\\
3.33999999999985	61.7201825765299\\
3.34199999999985	61.7078412574876\\
3.34399999999985	61.695502404528\\
3.34599999999985	61.6831660171711\\
3.34799999999985	61.6708320949344\\
3.34999999999985	61.6585006373373\\
3.35199999999985	61.6461716438977\\
3.35399999999985	61.6338451141381\\
3.35599999999985	61.6215210475737\\
3.35799999999985	61.6091994437234\\
3.35999999999985	61.5968803021083\\
3.36199999999985	61.5845636222476\\
3.36399999999985	61.5722494036564\\
3.36599999999985	61.559937645858\\
3.36799999999985	61.5476283483692\\
3.36999999999985	61.5353215107095\\
3.37199999999985	61.5230171323981\\
3.37399999999985	61.5107152129541\\
3.37599999999985	61.4984157518943\\
3.37799999999985	61.4861187487417\\
3.37999999999985	61.4738242030133\\
3.38199999999985	61.4615321142259\\
3.38399999999985	61.4492424819032\\
3.38599999999985	61.4369553055609\\
3.38799999999985	61.4246705847203\\
3.38999999999985	61.412388318897\\
3.39199999999985	61.4001085076158\\
3.39399999999985	61.3878311503912\\
3.39599999999985	61.375556246746\\
3.39799999999985	61.3632837961958\\
3.39999999999985	61.3510137982623\\
3.40199999999985	61.338746252463\\
3.40399999999985	61.3264811583203\\
3.40599999999985	61.3142185153505\\
3.40799999999985	61.3019583230736\\
3.40999999999985	61.2897005810092\\
3.41199999999985	61.27744528868\\
3.41399999999985	61.2651924455986\\
3.41599999999985	61.2529420512911\\
3.41799999999985	61.240694105273\\
3.41999999999984	61.2284486070665\\
3.42199999999984	61.2162055561881\\
3.42399999999984	61.2039649521606\\
3.42599999999984	61.1917267945016\\
3.42799999999984	61.1794910827309\\
3.42999999999984	61.1672578163695\\
3.43199999999984	61.1550269949354\\
3.43399999999984	61.1427986179494\\
3.43599999999984	61.1305726849311\\
3.43799999999984	61.1183491953994\\
3.43999999999984	61.1061281488765\\
3.44199999999984	61.0939095448809\\
3.44399999999984	61.0816933829327\\
3.44599999999984	61.0694796625511\\
3.44799999999984	61.0572683832565\\
3.44999999999984	61.0450595445698\\
3.45199999999984	61.0328531460085\\
3.45399999999984	61.0206491870956\\
3.45599999999984	61.0084476673512\\
3.45799999999984	60.9962485862938\\
3.45999999999984	60.984051943444\\
3.46199999999984	60.9718577383224\\
3.46399999999984	60.9596659704503\\
3.46599999999984	60.9474766393474\\
3.46799999999984	60.9352897445322\\
3.46999999999984	60.9231052855263\\
3.47199999999984	60.9109232618526\\
3.47399999999984	60.8987436730274\\
3.47599999999984	60.8865665185742\\
3.47799999999984	60.8743917980111\\
3.47999999999984	60.8622195108634\\
3.48199999999984	60.8500496566465\\
3.48399999999984	60.8378822348838\\
3.48599999999984	60.8257172450957\\
3.48799999999984	60.8135546868031\\
3.48999999999984	60.8013945595264\\
3.49199999999984	60.7892368627856\\
3.49399999999984	60.7770815961046\\
3.49599999999984	60.7649287590011\\
3.49799999999984	60.7527783509987\\
3.49999999999984	60.7406303716169\\
3.50199999999984	60.7284848203771\\
3.50399999999984	60.7163416968001\\
3.50599999999984	60.7042010004084\\
3.50799999999984	60.6920627307211\\
3.50999999999984	60.6799268872623\\
3.51199999999983	60.6677934695514\\
3.51399999999983	60.6556624771105\\
3.51599999999983	60.6435339094603\\
3.51799999999983	60.6314077661225\\
3.51999999999983	60.6192840466207\\
3.52199999999983	60.6071627504741\\
3.52399999999983	60.5950438772049\\
3.52599999999983	60.5829274263351\\
3.52799999999983	60.570813397385\\
3.52999999999983	60.5587017898802\\
3.53199999999983	60.5465926033388\\
3.53399999999983	60.5344858372838\\
3.53599999999983	60.5223814912376\\
3.53799999999983	60.5102795647225\\
3.53999999999983	60.4981800572599\\
3.54199999999983	60.4860829683706\\
3.54399999999983	60.4739882975797\\
3.54599999999983	60.4618960444074\\
3.54799999999983	60.4498062083771\\
3.54999999999983	60.4377187890096\\
3.55199999999983	60.4256337858291\\
3.55399999999983	60.4135511983574\\
3.55599999999983	60.401471026117\\
3.55799999999983	60.3893932686288\\
3.55999999999983	60.3773179254195\\
3.56199999999983	60.3652449960074\\
3.56399999999983	60.3531744799162\\
3.56599999999983	60.3411063766718\\
3.56799999999983	60.3290406857921\\
3.56999999999983	60.3169774068039\\
3.57199999999983	60.3049165392297\\
3.57399999999983	60.29285808259\\
3.57599999999983	60.280802036411\\
3.57799999999983	60.2687484002139\\
3.57999999999983	60.2566971735221\\
3.58199999999983	60.2446483558596\\
3.58399999999983	60.2326019467494\\
3.58599999999983	60.220557945714\\
3.58799999999983	60.2085163522773\\
3.58999999999983	60.1964771659633\\
3.59199999999983	60.1844403862959\\
3.59399999999983	60.1724060127972\\
3.59599999999983	60.1603740449916\\
3.59799999999983	60.1483444824039\\
3.59999999999983	60.1363173245567\\
3.60199999999982	60.124292570974\\
3.60399999999982	60.1122702211797\\
3.60599999999982	60.1002502746974\\
3.60799999999982	60.0882327310521\\
3.60999999999982	60.0762175897671\\
3.61199999999982	60.0642048503673\\
3.61399999999982	60.0521945123764\\
3.61599999999982	60.040186575318\\
3.61799999999982	60.0281810387185\\
3.61999999999982	60.0161779021005\\
3.62199999999982	60.0041771649894\\
3.62399999999982	59.9921788269086\\
3.62599999999982	59.9801828873837\\
3.62799999999982	59.9681893459398\\
3.62999999999982	59.9561982021004\\
3.63199999999982	59.9442094553906\\
3.63399999999982	59.9322231053361\\
3.63599999999982	59.920239151461\\
3.63799999999982	59.908257593291\\
3.63999999999982	59.8962784303499\\
3.64199999999982	59.8843016621648\\
3.64399999999982	59.8723272882582\\
3.64599999999982	59.8603553081584\\
3.64799999999982	59.8483857213883\\
3.64999999999982	59.8364185274741\\
3.65199999999982	59.8244537259431\\
3.65399999999982	59.8124913163168\\
3.65599999999982	59.8005312981253\\
3.65799999999982	59.7885736708902\\
3.65999999999982	59.7766184341408\\
3.66199999999982	59.7646655874002\\
3.66399999999982	59.7527151301948\\
3.66599999999982	59.7407670620511\\
3.66799999999982	59.728821382496\\
3.66999999999982	59.7168780910541\\
3.67199999999982	59.7049371872515\\
3.67399999999982	59.6929986706149\\
3.67599999999982	59.6810625406716\\
3.67799999999982	59.6691287969467\\
3.67999999999982	59.657197438966\\
3.68199999999982	59.6452684662573\\
3.68399999999982	59.633341878347\\
3.68599999999982	59.6214176747614\\
3.68799999999982	59.6094958550276\\
3.68999999999982	59.5975764186703\\
3.69199999999981	59.5856593652201\\
3.69399999999981	59.5737446942\\
3.69599999999981	59.5618324051399\\
3.69799999999981	59.549922497565\\
3.69999999999981	59.5380149710035\\
3.70199999999981	59.526109824983\\
3.70399999999981	59.5142070590294\\
3.70599999999981	59.5023066726711\\
3.70799999999981	59.490408665435\\
3.70999999999981	59.4785130368481\\
3.71199999999981	59.4666197864383\\
3.71399999999981	59.4547289137344\\
3.71599999999981	59.442840418262\\
3.71799999999981	59.4309542995498\\
3.71999999999981	59.4190705571266\\
3.72199999999981	59.4071891905201\\
3.72399999999981	59.3953101992562\\
3.72599999999981	59.3834335828654\\
3.72799999999981	59.3715593408753\\
3.72999999999981	59.3596874728128\\
3.73199999999981	59.3478179782069\\
3.73399999999981	59.3359508565874\\
3.73599999999981	59.3240861074803\\
3.73799999999981	59.3122237304155\\
3.73999999999981	59.3003637249219\\
3.74199999999981	59.2885060905269\\
3.74399999999981	59.2766508267604\\
3.74599999999981	59.2647979331501\\
3.74799999999981	59.2529474092267\\
3.74999999999981	59.2410992545157\\
3.75199999999981	59.2292534685494\\
3.75399999999981	59.2174100508562\\
3.75599999999981	59.2055690009648\\
3.75799999999981	59.1937303184045\\
3.75999999999981	59.1818940027038\\
3.76199999999981	59.1700600533939\\
3.76399999999981	59.1582284700034\\
3.76599999999981	59.1463992520615\\
3.76799999999981	59.1345723990982\\
3.76999999999981	59.1227479106425\\
3.77199999999981	59.1109257862249\\
3.77399999999981	59.0991060253758\\
3.77599999999981	59.0872886276244\\
3.77799999999981	59.0754735925002\\
3.77999999999981	59.0636609195339\\
3.78199999999981	59.0518506082552\\
3.7839999999998	59.0400426581953\\
3.7859999999998	59.028237068884\\
3.7879999999998	59.016433839851\\
3.7899999999998	59.0046329706284\\
3.7919999999998	58.9928344607439\\
3.7939999999998	58.9810383097315\\
3.7959999999998	58.9692445171196\\
3.7979999999998	58.9574530824403\\
3.7999999999998	58.9456640052229\\
3.8019999999998	58.9338772850003\\
3.8039999999998	58.9220929213016\\
3.8059999999998	58.9103109136588\\
3.8079999999998	58.8985312616041\\
3.8099999999998	58.8867539646674\\
3.8119999999998	58.8749790223789\\
3.8139999999998	58.8632064342722\\
3.8159999999998	58.851436199878\\
3.8179999999998	58.8396683187281\\
3.8199999999998	58.8279027903532\\
3.8219999999998	58.8161396142856\\
3.8239999999998	58.8043787900577\\
3.8259999999998	58.7926203172002\\
3.8279999999998	58.7808641952462\\
3.8299999999998	58.7691104237272\\
3.8319999999998	58.7573590021754\\
3.8339999999998	58.7456099301218\\
3.8359999999998	58.7338632071007\\
3.8379999999998	58.7221188326431\\
3.8399999999998	58.7103768062823\\
3.8419999999998	58.6986371275494\\
3.8439999999998	58.6868997959784\\
3.8459999999998	58.6751648111016\\
3.8479999999998	58.6634321724505\\
3.8499999999998	58.65170187956\\
3.8519999999998	58.6399739319611\\
3.8539999999998	58.6282483291886\\
3.8559999999998	58.6165250707738\\
3.8579999999998	58.6048041562517\\
3.8599999999998	58.5930855851537\\
3.8619999999998	58.5813693570138\\
3.8639999999998	58.5696554713659\\
3.8659999999998	58.5579439277422\\
3.8679999999998	58.5462347256781\\
3.8699999999998	58.5345278647063\\
3.8719999999998	58.5228233443589\\
3.87399999999979	58.5111211641724\\
3.87599999999979	58.4994213236787\\
3.87799999999979	58.4877238224129\\
3.87999999999979	58.4760286599085\\
3.88199999999979	58.4643358357002\\
3.88399999999979	58.4526453493216\\
3.88599999999979	58.4409572003053\\
3.88799999999979	58.4292713881905\\
3.88999999999979	58.4175879125057\\
3.89199999999979	58.4059067727897\\
3.89399999999979	58.3942279685744\\
3.89599999999979	58.3825514993972\\
3.89799999999979	58.3708773647912\\
3.89999999999979	58.3592055642912\\
3.90199999999979	58.3475360974323\\
3.90399999999979	58.3358689637496\\
3.90599999999979	58.3242041627776\\
3.90799999999979	58.3125416940527\\
3.90999999999979	58.3008815571093\\
3.91199999999979	58.2892237514828\\
3.91399999999979	58.2775682767089\\
3.91599999999979	58.2659151323222\\
3.91799999999979	58.2542643178593\\
3.91999999999979	58.2426158328564\\
3.92199999999979	58.2309696768476\\
3.92399999999979	58.2193258493697\\
3.92599999999979	58.2076843499585\\
3.92799999999979	58.1960451781487\\
3.92999999999979	58.1844083334792\\
3.93199999999979	58.1727738154847\\
3.93399999999979	58.1611416236994\\
3.93599999999979	58.1495117576623\\
3.93799999999979	58.1378842169099\\
3.93999999999979	58.1262590009764\\
3.94199999999979	58.1146361093997\\
3.94399999999979	58.1030155417182\\
3.94599999999979	58.0913972974653\\
3.94799999999979	58.0797813761804\\
3.94999999999979	58.068167777399\\
3.95199999999979	58.0565565006584\\
3.95399999999979	58.0449475454966\\
3.95599999999979	58.033340911449\\
3.95799999999979	58.0217365980545\\
3.95999999999979	58.0101346048493\\
3.96199999999979	57.9985349313707\\
3.96399999999979	57.9869375771578\\
3.96599999999978	57.9753425417454\\
3.96799999999978	57.9637498246749\\
3.96999999999978	57.9521594254799\\
3.97199999999978	57.940571343702\\
3.97399999999978	57.928985578876\\
3.97599999999978	57.9174021305419\\
3.97799999999978	57.9058209982373\\
3.97999999999978	57.8942421814997\\
3.98199999999978	57.8826656798676\\
3.98399999999978	57.8710914928806\\
3.98599999999978	57.8595196200758\\
3.98799999999978	57.8479500609907\\
3.98999999999978	57.8363828151654\\
3.99199999999978	57.8248178821396\\
3.99399999999978	57.8132552614496\\
3.99599999999978	57.8016949526355\\
3.99799999999978	57.7901369552355\\
3.99999999999978	57.7785812687897\\
4.00199999999978	57.7670278928364\\
4.00399999999978	57.7554768269145\\
4.00599999999978	57.7439280705646\\
4.00799999999978	57.7323816233249\\
4.00999999999978	57.720837484734\\
4.01199999999978	57.7092956543333\\
4.01399999999978	57.697756131662\\
4.01599999999978	57.6862189162591\\
4.01799999999978	57.6746840076634\\
4.01999999999978	57.6631514054174\\
4.02199999999978	57.6516211090584\\
4.02399999999978	57.6400931181282\\
4.02599999999978	57.6285674321665\\
4.02799999999978	57.617044050711\\
4.02999999999978	57.6055229733062\\
4.03199999999978	57.5940041994898\\
4.03399999999978	57.5824877288031\\
4.03599999999978	57.5709735607862\\
4.03799999999978	57.5594616949795\\
4.03999999999978	57.5479521309246\\
4.04199999999978	57.5364448681614\\
4.04399999999978	57.5249399062313\\
4.04599999999978	57.5134372446749\\
4.04799999999978	57.5019368830333\\
4.04999999999978	57.4904388208473\\
4.05199999999978	57.478943057658\\
4.05399999999978	57.4674495930079\\
4.05599999999978	57.4559584264377\\
4.05799999999978	57.4444695574882\\
4.05999999999977	57.4329829857009\\
4.06199999999977	57.4214987106189\\
4.06399999999977	57.4100167317829\\
4.06599999999977	57.3985370487344\\
4.06799999999977	57.3870596610161\\
4.06999999999977	57.3755845681689\\
4.07199999999977	57.3641117697359\\
4.07399999999977	57.3526412652584\\
4.07599999999977	57.34117305428\\
4.07799999999977	57.3297071363413\\
4.07999999999977	57.318243510985\\
4.08199999999977	57.3067821777542\\
4.08399999999977	57.2953231361917\\
4.08599999999977	57.283866385839\\
4.08799999999977	57.2724119262403\\
4.08999999999977	57.2609597569372\\
4.09199999999977	57.2495098774733\\
4.09399999999977	57.2380622873909\\
4.09599999999977	57.226616986234\\
4.09799999999977	57.215173973544\\
4.09999999999977	57.2037332488667\\
4.10199999999977	57.192294811745\\
4.10399999999977	57.1808586617196\\
4.10599999999977	57.1694247983365\\
4.10799999999977	57.157993221139\\
4.10999999999977	57.1465639296697\\
4.11199999999977	57.1351369234736\\
4.11399999999977	57.1237122020929\\
4.11599999999977	57.1122897650743\\
4.11799999999977	57.1008696119596\\
4.11999999999977	57.0894517422933\\
4.12199999999977	57.0780361556204\\
4.12399999999977	57.0666228514837\\
4.12599999999977	57.0552118294289\\
4.12799999999977	57.0438030890006\\
4.12999999999977	57.0323966297416\\
4.13199999999977	57.0209924511972\\
4.13399999999977	57.0095905529134\\
4.13599999999977	56.9981909344348\\
4.13799999999977	56.9867935953039\\
4.13999999999977	56.9753985350696\\
4.14199999999977	56.9640057532727\\
4.14399999999977	56.9526152494608\\
4.14599999999977	56.9412270231799\\
4.14799999999977	56.9298410739727\\
4.14999999999976	56.9184574013869\\
4.15199999999976	56.9070760049662\\
4.15399999999976	56.8956968842573\\
4.15599999999976	56.8843200388056\\
4.15799999999976	56.872945468157\\
4.15999999999976	56.8615731718573\\
4.16199999999976	56.8502031494514\\
4.16399999999976	56.8388354004874\\
4.16599999999976	56.8274699245082\\
4.16799999999976	56.816106721064\\
4.16999999999976	56.8047457896979\\
4.17199999999976	56.7933871299586\\
4.17399999999976	56.7820307413896\\
4.17599999999976	56.7706766235399\\
4.17799999999976	56.7593247759547\\
4.17999999999976	56.747975198182\\
4.18199999999976	56.7366278897673\\
4.18399999999976	56.725282850258\\
4.18599999999976	56.7139400792012\\
4.18799999999976	56.7025995761438\\
4.18999999999976	56.6912613406332\\
4.19199999999976	56.6799253722158\\
4.19399999999976	56.6685916704389\\
4.19599999999976	56.6572602348511\\
4.19799999999976	56.6459310649998\\
4.19999999999976	56.6346041604315\\
4.20199999999976	56.6232795206938\\
4.20399999999976	56.6119571453343\\
4.20599999999976	56.6006370339028\\
4.20799999999976	56.5893191859449\\
4.20999999999976	56.5780036010095\\
4.21199999999976	56.5666902786454\\
4.21399999999976	56.5553792183992\\
4.21599999999976	56.5440704198207\\
4.21799999999976	56.5327638824563\\
4.21999999999976	56.5214596058566\\
4.22199999999976	56.5101575895687\\
4.22399999999976	56.4988578331419\\
4.22599999999976	56.4875603361243\\
4.22799999999976	56.4762650980655\\
4.22999999999976	56.4649721185125\\
4.23199999999976	56.4536813970168\\
4.23399999999976	56.4423929331262\\
4.23599999999976	56.4311067263893\\
4.23799999999976	56.4198227763557\\
4.23999999999976	56.4085410825747\\
4.24199999999975	56.3972616445958\\
4.24399999999975	56.385984461969\\
4.24599999999975	56.3747095342425\\
4.24799999999975	56.3634368609672\\
4.24999999999975	56.3521664416917\\
4.25199999999975	56.3408982759672\\
4.25399999999975	56.3296323633422\\
4.25599999999975	56.3183687033664\\
4.25799999999975	56.3071072955926\\
4.25999999999975	56.2958481395684\\
4.26199999999975	56.2845912348447\\
4.26399999999975	56.2733365809715\\
4.26599999999975	56.2620841774999\\
4.26799999999975	56.2508340239799\\
4.26999999999975	56.2395861199622\\
4.27199999999975	56.2283404649976\\
4.27399999999975	56.2170970586371\\
4.27599999999975	56.205855900431\\
4.27799999999975	56.1946169899305\\
4.27999999999975	56.1833803266864\\
4.28199999999975	56.1721459102497\\
4.28399999999975	56.1609137401726\\
4.28599999999975	56.1496838160059\\
4.28799999999975	56.1384561373\\
4.28999999999975	56.127230703607\\
4.29199999999975	56.1160075144789\\
4.29399999999975	56.1047865694668\\
4.29599999999975	56.0935678681229\\
4.29799999999975	56.0823514099981\\
4.29999999999975	56.0711371946459\\
4.30199999999975	56.059925221616\\
4.30399999999975	56.0487154904623\\
4.30599999999975	56.0375080007356\\
4.30799999999975	56.0263027519901\\
4.30999999999975	56.0150997437762\\
4.31199999999975	56.0038989756476\\
4.31399999999975	55.9927004471544\\
4.31599999999975	55.9815041578534\\
4.31799999999975	55.9703101072941\\
4.31999999999975	55.9591182950286\\
4.32199999999975	55.9479287206121\\
4.32399999999975	55.9367413835976\\
4.32599999999975	55.9255562835364\\
4.32799999999975	55.9143734199818\\
4.32999999999975	55.9031927924879\\
4.33199999999974	55.8920144006067\\
4.33399999999974	55.8808382438942\\
4.33599999999974	55.8696643218993\\
4.33799999999974	55.858492634181\\
4.33999999999974	55.8473231802894\\
4.34199999999974	55.8361559597782\\
4.34399999999974	55.8249909722036\\
4.34599999999974	55.8138282171171\\
4.34799999999974	55.8026676940728\\
4.34999999999974	55.7915094026267\\
4.35199999999974	55.7803533423311\\
4.35399999999974	55.7691995127401\\
4.35599999999974	55.758047913409\\
4.35799999999974	55.746898543893\\
4.35999999999974	55.735751403745\\
4.36199999999974	55.7246064925202\\
4.36399999999974	55.7134638097726\\
4.36599999999974	55.7023233550581\\
4.36799999999974	55.6911851279306\\
4.36999999999974	55.680049127946\\
4.37199999999974	55.6689153546576\\
4.37399999999974	55.6577838076225\\
4.37599999999974	55.6466544863959\\
4.37799999999974	55.63552739053\\
4.37999999999974	55.6244025195837\\
4.38199999999974	55.6132798731106\\
4.38399999999974	55.6021594506668\\
4.38599999999974	55.5910412518074\\
4.38799999999974	55.5799252760899\\
4.38999999999974	55.5688115230675\\
4.39199999999974	55.5576999922965\\
4.39399999999974	55.5465906833354\\
4.39599999999974	55.5354835957378\\
4.39799999999974	55.5243787290606\\
4.39999999999974	55.5132760828599\\
4.40199999999974	55.5021756566924\\
4.40399999999974	55.4910774501138\\
4.40599999999974	55.479981462681\\
4.40799999999974	55.4688876939506\\
4.40999999999974	55.457796143479\\
4.41199999999974	55.4467068108238\\
4.41399999999974	55.4356196955408\\
4.41599999999974	55.4245347971873\\
4.41799999999974	55.4134521153199\\
4.41999999999974	55.4023716494957\\
4.42199999999974	55.3912933992737\\
4.42399999999973	55.3802173642089\\
4.42599999999973	55.3691435438601\\
4.42799999999973	55.3580719377833\\
4.42999999999973	55.3470025455366\\
4.43199999999973	55.3359353666787\\
4.43399999999973	55.3248704007668\\
4.43599999999973	55.3138076473582\\
4.43799999999973	55.3027471060099\\
4.43999999999973	55.2916887762825\\
4.44199999999973	55.2806326577312\\
4.44399999999973	55.2695787499151\\
4.44599999999973	55.2585270523937\\
4.44799999999973	55.2474775647242\\
4.44999999999973	55.2364302864646\\
4.45199999999973	55.2253852171742\\
4.45399999999973	55.2143423564107\\
4.45599999999973	55.2033017037342\\
4.45799999999973	55.1922632587016\\
4.45999999999973	55.1812270208726\\
4.46199999999973	55.1701929898067\\
4.46399999999973	55.1591611650625\\
4.46599999999973	55.1481315461985\\
4.46799999999973	55.1371041327739\\
4.46999999999973	55.1260789243499\\
4.47199999999973	55.1150559204823\\
4.47399999999973	55.1040351207339\\
4.47599999999973	55.0930165246621\\
4.47799999999973	55.0820001318281\\
4.47999999999973	55.0709859417904\\
4.48199999999973	55.0599739541089\\
4.48399999999973	55.0489641683441\\
4.48599999999973	55.0379565840548\\
4.48799999999973	55.026951200803\\
4.48999999999973	55.0159480181462\\
4.49199999999973	55.0049470356472\\
4.49399999999973	54.9939482528645\\
4.49599999999973	54.9829516693595\\
4.49799999999973	54.9719572846917\\
4.49999999999973	54.960965098423\\
4.50199999999973	54.9499751101128\\
4.50399999999973	54.9389873193211\\
4.50599999999973	54.9280017256113\\
4.50799999999973	54.9170183285427\\
4.50999999999973	54.9060371276763\\
4.51199999999973	54.895058122573\\
4.51399999999972	54.8840813127954\\
4.51599999999972	54.8731066979023\\
4.51799999999972	54.8621342774569\\
4.51999999999972	54.8511640510205\\
4.52199999999972	54.8401960181546\\
4.52399999999972	54.8292301784193\\
4.52599999999972	54.8182665313776\\
4.52799999999972	54.8073050765916\\
4.52999999999972	54.7963458136224\\
4.53199999999972	54.7853887420308\\
4.53399999999972	54.7744338613819\\
4.53599999999972	54.763481171235\\
4.53799999999972	54.7525306711552\\
4.53999999999972	54.7415823607016\\
4.54199999999972	54.730636239438\\
4.54399999999972	54.7196923069262\\
4.54599999999972	54.7087505627296\\
4.54799999999972	54.6978110064111\\
4.54999999999972	54.6868736375325\\
4.55199999999972	54.6759384556568\\
4.55399999999972	54.6650054603471\\
4.55599999999972	54.6540746511664\\
4.55799999999972	54.6431460276773\\
4.55999999999972	54.6322195894439\\
4.56199999999972	54.6212953360278\\
4.56399999999972	54.6103732669945\\
4.56599999999972	54.5994533819059\\
4.56799999999972	54.5885356803249\\
4.56999999999972	54.5776201618173\\
4.57199999999972	54.5667068259451\\
4.57399999999972	54.5557956722724\\
4.57599999999972	54.5448867003627\\
4.57799999999972	54.5339799097813\\
4.57999999999972	54.5230753000895\\
4.58199999999972	54.512172870853\\
4.58399999999972	54.5012726216377\\
4.58599999999972	54.4903745520057\\
4.58799999999972	54.4794786615213\\
4.58999999999972	54.4685849497489\\
4.59199999999972	54.4576934162559\\
4.59399999999972	54.4468040606031\\
4.59599999999972	54.4359168823561\\
4.59799999999972	54.4250318810812\\
4.59999999999972	54.4141490563411\\
4.60199999999972	54.4032684077036\\
4.60399999999971	54.3923899347301\\
4.60599999999971	54.3815136369904\\
4.60799999999971	54.3706395140454\\
4.60999999999971	54.359767565462\\
4.61199999999971	54.3488977908073\\
4.61399999999971	54.3380301896438\\
4.61599999999971	54.3271647615373\\
4.61799999999971	54.316301506056\\
4.61999999999971	54.3054404227628\\
4.62199999999971	54.2945815112266\\
4.62399999999971	54.283724771011\\
4.62599999999971	54.2728702016821\\
4.62799999999971	54.2620178028061\\
4.62999999999971	54.2511675739501\\
4.63199999999971	54.2403195146793\\
4.63399999999971	54.2294736245596\\
4.63599999999971	54.2186299031596\\
4.63799999999971	54.207788350043\\
4.63999999999971	54.1969489647783\\
4.64199999999971	54.1861117469327\\
4.64399999999971	54.1752766960697\\
4.64599999999971	54.1644438117612\\
4.64799999999971	54.1536130935698\\
4.64999999999971	54.1427845410634\\
4.65199999999971	54.1319581538093\\
4.65399999999971	54.1211339313759\\
4.65599999999971	54.1103118733297\\
4.65799999999971	54.0994919792378\\
4.65999999999971	54.0886742486685\\
4.66199999999971	54.0778586811882\\
4.66399999999971	54.0670452763635\\
4.66599999999971	54.0562340337643\\
4.66799999999971	54.0454249529585\\
4.66999999999971	54.034618033512\\
4.67199999999971	54.0238132749949\\
4.67399999999971	54.0130106769736\\
4.67599999999971	54.0022102390176\\
4.67799999999971	53.991411960694\\
4.67999999999971	53.9806158415697\\
4.68199999999971	53.969821881216\\
4.68399999999971	53.9590300791993\\
4.68599999999971	53.9482404350909\\
4.68799999999971	53.9374529484563\\
4.68999999999971	53.9266676188652\\
4.69199999999971	53.9158844458874\\
4.69399999999971	53.9051034290899\\
4.6959999999997	53.8943245680438\\
4.6979999999997	53.8835478623176\\
4.6999999999997	53.8727733114798\\
4.7019999999997	53.8620009150994\\
4.7039999999997	53.8512306727457\\
4.7059999999997	53.8404625839889\\
4.7079999999997	53.8296966483996\\
4.7099999999997	53.8189328655435\\
4.7119999999997	53.8081712349941\\
4.7139999999997	53.7974117563199\\
4.7159999999997	53.7866544290897\\
4.7179999999997	53.7758992528754\\
4.7199999999997	53.7651462272459\\
4.7219999999997	53.7543953517701\\
4.7239999999997	53.7436466260204\\
4.7259999999997	53.7329000495656\\
4.7279999999997	53.7221556219759\\
4.7299999999997	53.7114133428238\\
4.7319999999997	53.7006732116773\\
4.7339999999997	53.689935228108\\
4.7359999999997	53.6791993916874\\
4.7379999999997	53.6684657019833\\
4.7399999999997	53.6577341585718\\
4.7419999999997	53.6470047610189\\
4.7439999999997	53.6362775088971\\
4.7459999999997	53.6255524017796\\
4.7479999999997	53.6148294392343\\
4.7499999999997	53.6041086208357\\
4.7519999999997	53.5933899461526\\
4.7539999999997	53.5826734147569\\
4.7559999999997	53.5719590262208\\
4.7579999999997	53.5612467801151\\
4.7599999999997	53.5505366760141\\
4.7619999999997	53.5398287134871\\
4.7639999999997	53.5291228921062\\
4.7659999999997	53.5184192114417\\
4.7679999999997	53.5077176710698\\
4.7699999999997	53.4970182705603\\
4.7719999999997	53.4863210094845\\
4.7739999999997	53.4756258874162\\
4.7759999999997	53.4649329039272\\
4.7779999999997	53.4542420585912\\
4.7799999999997	53.443553350978\\
4.7819999999997	53.432866780664\\
4.7839999999997	53.4221823472191\\
4.78599999999969	53.4115000502162\\
4.78799999999969	53.4008198892293\\
4.78999999999969	53.39014186383\\
4.79199999999969	53.3794659735931\\
4.79399999999969	53.3687922180902\\
4.79599999999969	53.3581205968961\\
4.79799999999969	53.3474511095832\\
4.79999999999969	53.3367837557244\\
4.80199999999969	53.3261185348936\\
4.80399999999969	53.3154554466656\\
4.80599999999969	53.3047944906124\\
4.80799999999969	53.2941356663082\\
4.80999999999969	53.2834789733267\\
4.81199999999969	53.2728244112425\\
4.81399999999969	53.2621719796292\\
4.81599999999969	53.2515216780607\\
4.81799999999969	53.2408735061107\\
4.81999999999969	53.2302274633553\\
4.82199999999969	53.2195835493676\\
4.82399999999969	53.2089417637207\\
4.82599999999969	53.1983021059912\\
4.82799999999969	53.1876645757523\\
4.82999999999969	53.1770291725793\\
4.83199999999969	53.1663958960472\\
4.83399999999969	53.155764745731\\
4.83599999999969	53.1451357212051\\
4.83799999999969	53.1345088220434\\
4.83999999999969	53.1238840478239\\
4.84199999999969	53.1132613981194\\
4.84399999999969	53.1026408725061\\
4.84599999999969	53.0920224705583\\
4.84799999999969	53.0814061918525\\
4.84999999999969	53.0707920359649\\
4.85199999999969	53.0601800024695\\
4.85399999999969	53.0495700909421\\
4.85599999999969	53.0389623009597\\
4.85799999999969	53.0283566320967\\
4.85999999999969	53.0177530839295\\
4.86199999999969	53.0071516560366\\
4.86399999999969	52.9965523479901\\
4.86599999999969	52.9859551593687\\
4.86799999999969	52.9753600897473\\
4.86999999999969	52.9647671387028\\
4.87199999999969	52.9541763058127\\
4.87399999999969	52.9435875906518\\
4.87599999999969	52.9330009927985\\
4.87799999999968	52.9224165118273\\
4.87999999999968	52.9118341473172\\
4.88199999999968	52.9012538988432\\
4.88399999999968	52.8906757659841\\
4.88599999999968	52.8800997483164\\
4.88799999999968	52.8695258454145\\
4.88999999999968	52.8589540568594\\
4.89199999999968	52.8483843822269\\
4.89399999999968	52.8378168210943\\
4.89599999999968	52.8272513730398\\
4.89799999999968	52.8166880376384\\
4.89999999999968	52.806126814471\\
4.90199999999968	52.7955677031134\\
4.90399999999968	52.7850107031438\\
4.90599999999968	52.7744558141402\\
4.90799999999968	52.7639030356795\\
4.90999999999968	52.7533523673419\\
4.91199999999968	52.7428038087045\\
4.91399999999968	52.7322573593448\\
4.91599999999968	52.7217130188422\\
4.91799999999968	52.7111707867742\\
4.91999999999968	52.7006306627196\\
4.92199999999968	52.6900926462573\\
4.92399999999968	52.6795567369649\\
4.92599999999968	52.6690229344227\\
4.92799999999968	52.6584912382072\\
4.92999999999968	52.6479616478995\\
4.93199999999968	52.6374341630785\\
4.93399999999968	52.6269087833215\\
4.93599999999968	52.6163855082089\\
4.93799999999968	52.6058643373197\\
4.93999999999968	52.5953452702325\\
4.94199999999968	52.5848283065278\\
4.94399999999968	52.5743134457845\\
4.94599999999968	52.5638006875819\\
4.94799999999968	52.553290031501\\
4.94999999999968	52.5427814771198\\
4.95199999999968	52.5322750240191\\
4.95399999999968	52.5217706717782\\
4.95599999999968	52.5112684199777\\
4.95799999999968	52.5007682681974\\
4.95999999999968	52.4902702160172\\
4.96199999999968	52.4797742630165\\
4.96399999999968	52.469280408777\\
4.96599999999968	52.4587886528796\\
4.96799999999967	52.4482989949035\\
4.96999999999967	52.4378114344297\\
4.97199999999967	52.4273259710375\\
4.97399999999967	52.4168426043099\\
4.97599999999967	52.4063613338268\\
4.97799999999967	52.3958821591688\\
4.97999999999967	52.3854050799168\\
4.98199999999967	52.3749300956527\\
4.98399999999967	52.3644572059563\\
4.98599999999967	52.3539864104095\\
4.98799999999967	52.343517708595\\
4.98999999999967	52.3330511000911\\
4.99199999999967	52.3225865844817\\
4.99399999999967	52.3121241613473\\
4.99599999999967	52.301663830271\\
4.99799999999967	52.2912055908319\\
4.99999999999967	52.2807494426156\\
5.00199999999967	52.2702953851992\\
5.00399999999967	52.2598434181684\\
5.00599999999967	52.2493935411044\\
5.00799999999967	52.2389457535891\\
5.00999999999967	52.228500055204\\
5.01199999999967	52.2180564455317\\
5.01399999999967	52.2076149241558\\
5.01599999999967	52.1971754906573\\
5.01799999999967	52.1867381446198\\
5.01999999999967	52.1763028856256\\
5.02199999999967	52.1658697132567\\
5.02399999999967	52.1554386270963\\
5.02599999999967	52.1450096267284\\
5.02799999999967	52.134582711735\\
5.02999999999967	52.1241578816996\\
5.03199999999967	52.1137351362032\\
5.03399999999967	52.1033144748327\\
5.03599999999967	52.0928958971688\\
5.03799999999967	52.0824794027956\\
5.03999999999967	52.0720649912966\\
5.04199999999967	52.0616526622558\\
5.04399999999967	52.0512424152562\\
5.04599999999967	52.0408342498816\\
5.04799999999967	52.0304281657153\\
5.04999999999967	52.0200241623426\\
5.05199999999967	52.0096222393463\\
5.05399999999967	51.9992223963105\\
5.05599999999967	51.9888246328203\\
5.05799999999966	51.978428948458\\
5.05999999999966	51.9680353428099\\
5.06199999999966	51.9576438154607\\
5.06399999999966	51.9472543659923\\
5.06599999999966	51.9368669939912\\
5.06799999999966	51.9264816990418\\
5.06999999999966	51.9160984807277\\
5.07199999999966	51.9057173386352\\
5.07399999999966	51.895338272349\\
5.07599999999966	51.8849612814537\\
5.07799999999966	51.8745863655337\\
5.07999999999966	51.8642135241751\\
5.08199999999966	51.8538427569623\\
5.08399999999966	51.8434740634817\\
5.08599999999966	51.8331074433173\\
5.08799999999966	51.8227428960563\\
5.08999999999966	51.8123804212815\\
5.09199999999966	51.8020200185813\\
5.09399999999966	51.7916616875404\\
5.09599999999966	51.7813054277442\\
5.09799999999966	51.7709512387787\\
5.09999999999966	51.7605991202302\\
5.10199999999966	51.7502490716846\\
5.10399999999966	51.7399010927283\\
5.10599999999966	51.7295551829465\\
5.10799999999966	51.7192113419267\\
5.10999999999966	51.7088695692555\\
5.11199999999966	51.698529864518\\
5.11399999999966	51.688192227301\\
5.11599999999966	51.6778566571906\\
5.11799999999966	51.6675231537761\\
5.11999999999966	51.6571917166414\\
5.12199999999966	51.6468623453749\\
5.12399999999966	51.6365350395631\\
5.12599999999966	51.6262097987928\\
5.12799999999966	51.615886622652\\
5.12999999999966	51.6055655107269\\
5.13199999999966	51.5952464626053\\
5.13399999999966	51.5849294778743\\
5.13599999999966	51.5746145561219\\
5.13799999999966	51.5643016969341\\
5.13999999999966	51.5539908999002\\
5.14199999999966	51.543682164607\\
5.14399999999966	51.5333754906425\\
5.14599999999966	51.5230708775949\\
5.14799999999966	51.5127683250516\\
5.14999999999965	51.5024678326011\\
5.15199999999965	51.4921693998306\\
5.15399999999965	51.4818730263296\\
5.15599999999965	51.4715787116853\\
5.15799999999965	51.4612864554872\\
5.15999999999965	51.4509962573216\\
5.16199999999965	51.4407081167789\\
5.16399999999965	51.4304220334475\\
5.16599999999965	51.4201380069148\\
5.16799999999965	51.409856036771\\
5.16999999999965	51.3995761226041\\
5.17199999999965	51.3892982640038\\
5.17399999999965	51.3790224605586\\
5.17599999999965	51.3687487118569\\
5.17799999999965	51.3584770174895\\
5.17999999999965	51.3482073770436\\
5.18199999999965	51.3379397901099\\
5.18399999999965	51.3276742562787\\
5.18599999999965	51.3174107751377\\
5.18799999999965	51.3071493462764\\
5.18999999999965	51.2968899692854\\
5.19199999999965	51.286632643754\\
5.19399999999965	51.2763773692729\\
5.19599999999965	51.2661241454307\\
5.19799999999965	51.2558729718186\\
5.19999999999965	51.2456238480258\\
5.20199999999965	51.2353767736425\\
5.20399999999965	51.2251317482593\\
5.20599999999965	51.2148887714661\\
5.20799999999965	51.2046478428548\\
5.20999999999965	51.1944089620125\\
5.21199999999965	51.184172128533\\
5.21399999999965	51.173937342006\\
5.21599999999965	51.1637046020213\\
5.21799999999965	51.1534739081717\\
5.21999999999965	51.1432452600457\\
5.22199999999965	51.1330186572358\\
5.22399999999965	51.1227940993323\\
5.22599999999965	51.1125715859268\\
5.22799999999965	51.1023511166102\\
5.22999999999965	51.0921326909737\\
5.23199999999965	51.0819163086091\\
5.23399999999965	51.0717019691079\\
5.23599999999965	51.061489672061\\
5.23799999999965	51.0512794170618\\
5.23999999999964	51.0410712036985\\
5.24199999999964	51.0308650315663\\
5.24399999999964	51.0206609002558\\
5.24599999999964	51.0104588093587\\
5.24799999999964	51.0002587584674\\
5.24999999999964	50.9900607471728\\
5.25199999999964	50.9798647750695\\
5.25399999999964	50.9696708417478\\
5.25599999999964	50.9594789468004\\
5.25799999999964	50.9492890898203\\
5.25999999999964	50.9391012703993\\
5.26199999999964	50.9289154881308\\
5.26399999999964	50.9187317426065\\
5.26599999999964	50.9085500334201\\
5.26799999999964	50.8983703601639\\
5.26999999999964	50.8881927224308\\
5.27199999999964	50.878017119814\\
5.27399999999964	50.8678435519069\\
5.27599999999964	50.8576720183029\\
5.27799999999964	50.8475025185929\\
5.27999999999964	50.8373350523731\\
5.28199999999964	50.8271696192354\\
5.28399999999964	50.817006218774\\
5.28599999999964	50.8068448505815\\
5.28799999999964	50.7966855142539\\
5.28999999999964	50.7865282093816\\
5.29199999999964	50.7763729355603\\
5.29399999999964	50.7662196923842\\
5.29599999999964	50.756068479446\\
5.29799999999964	50.7459192963405\\
5.29999999999964	50.7357721426623\\
5.30199999999964	50.7256270180043\\
5.30399999999964	50.7154839219626\\
5.30599999999964	50.7053428541299\\
5.30799999999964	50.6952038141013\\
5.30999999999964	50.6850668014716\\
5.31199999999964	50.6749318158352\\
5.31399999999964	50.6647988567867\\
5.31599999999964	50.6546679239214\\
5.31799999999964	50.6445390168328\\
5.31999999999964	50.6344121351175\\
5.32199999999964	50.6242872783703\\
5.32399999999964	50.6141644461848\\
5.32599999999964	50.6040436381576\\
5.32799999999964	50.5939248538841\\
5.32999999999964	50.5838080929586\\
5.33199999999963	50.5736933549773\\
5.33399999999963	50.5635806395353\\
5.33599999999963	50.5534699462288\\
5.33799999999963	50.5433612746527\\
5.33999999999963	50.5332546244032\\
5.34199999999963	50.523149995076\\
5.34399999999963	50.5130473862672\\
5.34599999999963	50.5029467975722\\
5.34799999999963	50.4928482285882\\
5.34999999999963	50.4827516789096\\
5.35199999999963	50.4726571481349\\
5.35399999999963	50.4625646358586\\
5.35599999999963	50.4524741416784\\
5.35799999999963	50.4423856651887\\
5.35999999999963	50.432299205989\\
5.36199999999963	50.4222147636738\\
5.36399999999963	50.4121323378404\\
5.36599999999963	50.4020519280852\\
5.36799999999963	50.391973534007\\
5.36999999999963	50.3818971552\\
5.37199999999963	50.3718227912625\\
5.37399999999963	50.3617504417922\\
5.37599999999963	50.3516801063859\\
5.37799999999963	50.3416117846407\\
5.37999999999963	50.331545476153\\
5.38199999999963	50.3214811805227\\
5.38399999999963	50.3114188973454\\
5.38599999999963	50.3013586262183\\
5.38799999999963	50.2913003667415\\
5.38999999999963	50.2812441185102\\
5.39199999999963	50.2711898811238\\
5.39399999999963	50.2611376541796\\
5.39599999999963	50.2510874372756\\
5.39799999999963	50.2410392300098\\
5.39999999999963	50.230993031981\\
5.40199999999963	50.2209488427876\\
5.40399999999963	50.2109066620255\\
5.40599999999963	50.2008664892965\\
5.40799999999963	50.190828324198\\
5.40999999999963	50.1807921663266\\
5.41199999999963	50.1707580152844\\
5.41399999999963	50.1607258706665\\
5.41599999999963	50.1506957320739\\
5.41799999999963	50.1406675991056\\
5.41999999999963	50.1306414713595\\
5.42199999999962	50.1206173484351\\
5.42399999999962	50.1105952299312\\
5.42599999999962	50.1005751154488\\
5.42799999999962	50.0905570045845\\
5.42999999999962	50.080540896939\\
5.43199999999962	50.0705267921128\\
5.43399999999962	50.0605146897026\\
5.43599999999962	50.0505045893105\\
5.43799999999962	50.0404964905362\\
5.43999999999962	50.0304903929787\\
5.44199999999962	50.0204862962365\\
5.44399999999962	50.0104841999121\\
5.44599999999962	50.0004841036039\\
5.44799999999962	49.990486006912\\
5.44999999999962	49.9804899094378\\
5.45199999999962	49.9704958107808\\
5.45399999999962	49.9605037105395\\
5.45599999999962	49.9505136083179\\
5.45799999999962	49.9405255037141\\
5.45999999999962	49.930539396329\\
5.46199999999962	49.920555285764\\
5.46399999999962	49.9105731716182\\
5.46599999999962	49.9005930534945\\
5.46799999999962	49.8906149309926\\
5.46999999999962	49.8806388037138\\
5.47199999999962	49.8706646712583\\
5.47399999999962	49.8606925332291\\
5.47599999999962	49.8507223892248\\
5.47799999999962	49.8407542388494\\
5.47999999999962	49.8307880817025\\
5.48199999999962	49.8208239173853\\
5.48399999999962	49.8108617455016\\
5.48599999999962	49.8009015656507\\
5.48799999999962	49.7909433774347\\
5.48999999999962	49.780987180456\\
5.49199999999962	49.7710329743165\\
5.49399999999962	49.7610807586179\\
5.49599999999962	49.7511305329615\\
5.49799999999962	49.7411822969501\\
5.49999999999962	49.7312360501862\\
5.50199999999962	49.7212917922712\\
5.50399999999962	49.7113495228076\\
5.50599999999962	49.7014092413987\\
5.50799999999962	49.6914709476454\\
5.50999999999962	49.6815346411522\\
5.51199999999961	49.6716003215197\\
5.51399999999961	49.661667988352\\
5.51599999999961	49.6517376412523\\
5.51799999999961	49.6418092798217\\
5.51999999999961	49.6318829036647\\
5.52199999999961	49.6219585123832\\
5.52399999999961	49.6120361055817\\
5.52599999999961	49.6021156828621\\
5.52799999999961	49.5921972438279\\
5.52999999999961	49.5822807880828\\
5.53199999999961	49.5723663152308\\
5.53399999999961	49.5624538248739\\
5.53599999999961	49.5525433166176\\
5.53799999999961	49.5426347900636\\
5.53999999999961	49.5327282448173\\
5.54199999999961	49.5228236804816\\
5.54399999999961	49.5129210966604\\
5.54599999999961	49.5030204929584\\
5.54799999999961	49.4931218689784\\
5.54999999999961	49.4832252243257\\
5.55199999999961	49.4733305586044\\
5.55399999999961	49.4634378714179\\
5.55599999999961	49.4535471623719\\
5.55799999999961	49.4436584310694\\
5.55999999999961	49.4337716771164\\
5.56199999999961	49.4238869001167\\
5.56399999999961	49.4140040996755\\
5.56599999999961	49.4041232753968\\
5.56799999999961	49.3942444268855\\
5.56999999999961	49.3843675537474\\
5.57199999999961	49.3744926555869\\
5.57399999999961	49.3646197320088\\
5.57599999999961	49.3547487826204\\
5.57799999999961	49.3448798070235\\
5.57999999999961	49.3350128048255\\
5.58199999999961	49.3251477756319\\
5.58399999999961	49.3152847190477\\
5.58599999999961	49.3054236346785\\
5.58799999999961	49.2955645221303\\
5.58999999999961	49.2857073810079\\
5.59199999999961	49.2758522109177\\
5.59399999999961	49.2659990114661\\
5.59599999999961	49.2561477822584\\
5.59799999999961	49.2462985229004\\
5.59999999999961	49.2364512329989\\
5.60199999999961	49.2266059121601\\
5.6039999999996	49.2167625599898\\
5.6059999999996	49.2069211760942\\
5.6079999999996	49.1970817600805\\
5.6099999999996	49.187244311555\\
5.6119999999996	49.1774088301233\\
5.6139999999996	49.1675753153932\\
5.6159999999996	49.1577437669709\\
5.6179999999996	49.1479141844645\\
5.6199999999996	49.138086567478\\
5.6219999999996	49.1282609156219\\
5.6239999999996	49.1184372285007\\
5.6259999999996	49.1086155057227\\
5.6279999999996	49.0987957468946\\
5.6299999999996	49.0889779516239\\
5.6319999999996	49.0791621195185\\
5.6339999999996	49.0693482501851\\
5.6359999999996	49.0595363432313\\
5.6379999999996	49.0497263982649\\
5.6399999999996	49.0399184148934\\
5.6419999999996	49.0301123927254\\
5.6439999999996	49.020308331368\\
5.6459999999996	49.0105062304284\\
5.6479999999996	49.0007060895163\\
5.6499999999996	48.9909079082389\\
5.6519999999996	48.9811116862034\\
5.6539999999996	48.9713174230197\\
5.6559999999996	48.9615251182952\\
5.6579999999996	48.9517347716395\\
5.6599999999996	48.9419463826593\\
5.6619999999996	48.9321599509639\\
5.6639999999996	48.9223754761611\\
5.6659999999996	48.9125929578622\\
5.6679999999996	48.9028123956732\\
5.6699999999996	48.8930337892049\\
5.6719999999996	48.8832571380645\\
5.6739999999996	48.873482441862\\
5.6759999999996	48.8637097002065\\
5.6779999999996	48.8539389127073\\
5.6799999999996	48.8441700789732\\
5.6819999999996	48.834403198614\\
5.6839999999996	48.8246382712385\\
5.6859999999996	48.8148752964569\\
5.6879999999996	48.8051142738784\\
5.6899999999996	48.7953552031123\\
5.6919999999996	48.7855980837687\\
5.69399999999959	48.7758429154579\\
5.69599999999959	48.7660896977886\\
5.69799999999959	48.7563384303714\\
5.69999999999959	48.7465891128173\\
5.70199999999959	48.7368417447344\\
5.70399999999959	48.7270963257347\\
5.70599999999959	48.717352855427\\
5.70799999999959	48.707611333423\\
5.70999999999959	48.6978717593327\\
5.71199999999959	48.6881341327658\\
5.71399999999959	48.6783984533339\\
5.71599999999959	48.6686647206468\\
5.71799999999959	48.658932934316\\
5.71999999999959	48.6492030939522\\
5.72199999999959	48.6394751991655\\
5.72399999999959	48.629749249568\\
5.72599999999959	48.6200252447696\\
5.72799999999959	48.6103031843825\\
5.72999999999959	48.600583068017\\
5.73199999999959	48.5908648952847\\
5.73399999999959	48.5811486657983\\
5.73599999999959	48.5714343791672\\
5.73799999999959	48.5617220350032\\
5.73999999999959	48.5520116329183\\
5.74199999999959	48.5423031725255\\
5.74399999999959	48.5325966534342\\
5.74599999999959	48.5228920752583\\
5.74799999999959	48.513189437607\\
5.74999999999959	48.5034887400962\\
5.75199999999959	48.4937899823348\\
5.75399999999959	48.4840931639358\\
5.75599999999959	48.4743982845109\\
5.75799999999959	48.4647053436742\\
5.75999999999959	48.4550143410362\\
5.76199999999959	48.4453252762104\\
5.76399999999959	48.4356381488087\\
5.76599999999959	48.4259529584436\\
5.76799999999959	48.4162697047285\\
5.76999999999959	48.4065883872754\\
5.77199999999959	48.3969090056985\\
5.77399999999959	48.3872315596086\\
5.77599999999959	48.37755604862\\
5.77799999999959	48.3678824723462\\
5.77999999999959	48.3582108303991\\
5.78199999999959	48.3485411223925\\
5.78399999999959	48.3388733479397\\
5.78599999999958	48.329207506654\\
5.78799999999958	48.3195435981501\\
5.78999999999958	48.3098816220386\\
5.79199999999958	48.3002215779358\\
5.79399999999958	48.2905634654545\\
5.79599999999958	48.2809072842076\\
5.79799999999958	48.2712530338104\\
5.79999999999958	48.2616007138757\\
5.80199999999958	48.2519503240189\\
5.80399999999958	48.2423018638523\\
5.80599999999958	48.2326553329909\\
5.80799999999958	48.2230107310488\\
5.80999999999958	48.2133680576407\\
5.81199999999958	48.2037273123804\\
5.81399999999958	48.1940884948831\\
5.81599999999958	48.184451604762\\
5.81799999999958	48.1748166416335\\
5.81999999999958	48.1651836051105\\
5.82199999999958	48.1555524948086\\
5.82399999999958	48.145923310343\\
5.82599999999958	48.136296051328\\
5.82799999999958	48.1266707173797\\
5.82999999999958	48.1170473081109\\
5.83199999999958	48.1074258231384\\
5.83399999999958	48.0978062620777\\
5.83599999999958	48.0881886245431\\
5.83799999999958	48.0785729101506\\
5.83999999999958	48.0689591185157\\
5.84199999999958	48.0593472492535\\
5.84399999999958	48.0497373019793\\
5.84599999999958	48.0401292763107\\
5.84799999999958	48.0305231718607\\
5.84999999999958	48.0209189882472\\
5.85199999999958	48.0113167250854\\
5.85399999999958	48.0017163819913\\
5.85599999999958	47.9921179585804\\
5.85799999999958	47.9825214544707\\
5.85999999999958	47.9729268692759\\
5.86199999999958	47.9633342026143\\
5.86399999999958	47.9537434541009\\
5.86599999999958	47.9441546233527\\
5.86799999999958	47.9345677099872\\
5.86999999999958	47.9249827136195\\
5.87199999999958	47.9153996338674\\
5.87399999999958	47.9058184703463\\
5.87599999999957	47.8962392226742\\
5.87799999999957	47.8866618904679\\
5.87999999999957	47.8770864733441\\
5.88199999999957	47.8675129709197\\
5.88399999999957	47.8579413828122\\
5.88599999999957	47.8483717086386\\
5.88799999999957	47.8388039480168\\
5.88999999999957	47.8292381005635\\
5.89199999999957	47.8196741658957\\
5.89399999999957	47.8101121436323\\
5.89599999999957	47.8005520333897\\
5.89799999999957	47.7909938347859\\
5.89999999999957	47.7814375474391\\
5.90199999999957	47.7718831709667\\
5.90399999999957	47.7623307049865\\
5.90599999999957	47.7527801491171\\
5.90799999999957	47.7432315029755\\
5.90999999999957	47.7336847661808\\
5.91199999999957	47.7241399383514\\
5.91399999999957	47.7145970191043\\
5.91599999999957	47.7050560080587\\
5.91799999999957	47.6955169048336\\
5.91999999999957	47.6859797090464\\
5.92199999999957	47.6764444203167\\
5.92399999999957	47.6669110382619\\
5.92599999999957	47.6573795625017\\
5.92799999999957	47.6478499926551\\
5.92999999999957	47.6383223283402\\
5.93199999999957	47.6287965691771\\
5.93399999999957	47.6192727147835\\
5.93599999999957	47.6097507647785\\
5.93799999999957	47.6002307187831\\
5.93999999999957	47.5907125764152\\
5.94199999999957	47.5811963372941\\
5.94399999999957	47.57168200104\\
5.94599999999957	47.5621695672716\\
5.94799999999957	47.5526590356091\\
5.94999999999957	47.5431504056712\\
5.95199999999957	47.53364367708\\
5.95399999999957	47.5241388494517\\
5.95599999999957	47.5146359224088\\
5.95799999999957	47.5051348955705\\
5.95999999999957	47.4956357685569\\
5.96199999999957	47.4861385409886\\
5.96399999999957	47.4766432124852\\
5.96599999999956	47.4671497826668\\
5.96799999999956	47.4576582511537\\
5.96999999999956	47.4481686175668\\
5.97199999999956	47.4386808815265\\
5.97399999999956	47.4291950426535\\
5.97599999999956	47.4197111005677\\
5.97799999999956	47.4102290548909\\
5.97999999999956	47.4007489052441\\
5.98199999999956	47.3912706512463\\
5.98399999999956	47.38179429252\\
5.98599999999956	47.372319828686\\
5.98799999999956	47.3628472593651\\
5.98999999999956	47.3533765841788\\
5.99199999999956	47.3439078027479\\
5.99399999999956	47.3344409146932\\
5.99599999999956	47.3249759196381\\
5.99799999999956	47.3155128172019\\
5.99999999999956	47.306051607008\\
6.00199999999956	47.2965922886765\\
6.00399999999956	47.2871348618303\\
6.00599999999956	47.2776793260905\\
6.00799999999956	47.2682256810784\\
6.00999999999956	47.2587739264176\\
6.01199999999956	47.2493240617286\\
6.01399999999956	47.2398760866336\\
6.01599999999956	47.2304300007558\\
6.01799999999956	47.2209858037169\\
6.01999999999956	47.211543495139\\
6.02199999999956	47.2021030746443\\
6.02399999999956	47.192664541856\\
6.02599999999956	47.1832278963957\\
6.02799999999956	47.1737931378868\\
6.02999999999956	47.1643602659515\\
6.03199999999956	47.1549292802132\\
6.03399999999956	47.1455001802945\\
6.03599999999956	47.1360729658177\\
6.03799999999956	47.1266476364065\\
6.03999999999956	47.1172241916835\\
6.04199999999956	47.1078026312717\\
6.04399999999956	47.0983829547945\\
6.04599999999956	47.088965161876\\
6.04799999999956	47.0795492521387\\
6.04999999999956	47.0701352252063\\
6.05199999999956	47.060723080702\\
6.05399999999956	47.0513128182495\\
6.05599999999956	47.0419044374729\\
6.05799999999955	47.0324979379955\\
6.05999999999955	47.0230933194419\\
6.06199999999955	47.013690581435\\
6.06399999999955	47.0042897235982\\
6.06599999999955	46.9948907455582\\
6.06799999999955	46.985493646936\\
6.06999999999955	46.9760984273582\\
6.07199999999955	46.9667050864472\\
6.07399999999955	46.9573136238285\\
6.07599999999955	46.9479240391257\\
6.07799999999955	46.9385363319646\\
6.07999999999955	46.9291505019695\\
6.08199999999955	46.9197665487634\\
6.08399999999955	46.9103844719725\\
6.08599999999955	46.9010042712216\\
6.08799999999955	46.8916259461346\\
6.08999999999955	46.882249496337\\
6.09199999999955	46.872874921455\\
6.09399999999955	46.8635022211118\\
6.09599999999955	46.8541313949335\\
6.09799999999955	46.8447624425453\\
6.09999999999955	46.8353953635723\\
6.10199999999955	46.8260301576401\\
6.10399999999955	46.8166668243728\\
6.10599999999955	46.8073053633987\\
6.10799999999955	46.7979457743415\\
6.10999999999955	46.7885880568271\\
6.11199999999955	46.7792322104817\\
6.11399999999955	46.7698782349315\\
6.11599999999955	46.760526129801\\
6.11799999999955	46.7511758947174\\
6.11999999999955	46.7418275293066\\
6.12199999999955	46.7324810331943\\
6.12399999999955	46.7231364060072\\
6.12599999999955	46.7137936473719\\
6.12799999999955	46.7044527569132\\
6.12999999999955	46.6951137342591\\
6.13199999999955	46.6857765790349\\
6.13399999999955	46.6764412908689\\
6.13599999999955	46.6671078693862\\
6.13799999999955	46.6577763142141\\
6.13999999999955	46.6484466249795\\
6.14199999999955	46.6391188013092\\
6.14399999999955	46.6297928428303\\
6.14599999999955	46.6204687491698\\
6.14799999999954	46.6111465199544\\
6.14999999999954	46.6018261548117\\
6.15199999999954	46.5925076533689\\
6.15399999999954	46.5831910152533\\
6.15599999999954	46.5738762400923\\
6.15799999999954	46.564563327513\\
6.15999999999954	46.5552522771445\\
6.16199999999954	46.5459430886126\\
6.16399999999954	46.5366357615454\\
6.16599999999954	46.5273302955712\\
6.16799999999954	46.5180266903174\\
6.16999999999954	46.5087249454125\\
6.17199999999954	46.4994250604837\\
6.17399999999954	46.4901270351605\\
6.17599999999954	46.4808308690689\\
6.17799999999954	46.4715365618396\\
6.17999999999954	46.4622441130995\\
6.18199999999954	46.4529535224762\\
6.18399999999954	46.4436647895999\\
6.18599999999954	46.4343779140978\\
6.18799999999954	46.425092895599\\
6.18999999999954	46.4158097337328\\
6.19199999999954	46.4065284281266\\
6.19399999999954	46.39724897841\\
6.19599999999954	46.387971384213\\
6.19799999999954	46.3786956451617\\
6.19999999999954	46.3694217608885\\
6.20199999999954	46.3601497310198\\
6.20399999999954	46.3508795551863\\
6.20599999999954	46.3416112330169\\
6.20799999999954	46.3323447641408\\
6.20999999999954	46.323080148188\\
6.21199999999954	46.3138173847871\\
6.21399999999954	46.3045564735684\\
6.21599999999954	46.2952974141601\\
6.21799999999954	46.286040206194\\
6.21999999999954	46.2767848492989\\
6.22199999999954	46.2675313431047\\
6.22399999999954	46.2582796872411\\
6.22599999999954	46.2490298813379\\
6.22799999999954	46.2397819250263\\
6.22999999999954	46.2305358179354\\
6.23199999999954	46.2212915596962\\
6.23399999999954	46.2120491499383\\
6.23599999999954	46.2028085882926\\
6.23799999999954	46.193569874389\\
6.23999999999953	46.1843330078588\\
6.24199999999953	46.1750979883317\\
6.24399999999953	46.1658648154393\\
6.24599999999953	46.1566334888112\\
6.24799999999953	46.1474040080796\\
6.24999999999953	46.1381763728744\\
6.25199999999953	46.1289505828268\\
6.25399999999953	46.1197266375684\\
6.25599999999953	46.1105045367294\\
6.25799999999953	46.1012842799415\\
6.25999999999953	46.0920658668361\\
6.26199999999953	46.0828492970435\\
6.26399999999953	46.0736345701972\\
6.26599999999953	46.0644216859267\\
6.26799999999953	46.0552106438644\\
6.26999999999953	46.0460014436419\\
6.27199999999953	46.0367940848908\\
6.27399999999953	46.0275885672429\\
6.27599999999953	46.0183848903301\\
6.27799999999953	46.009183053784\\
6.27999999999953	45.9999830572375\\
6.28199999999953	45.9907849003216\\
6.28399999999953	45.981588582669\\
6.28599999999953	45.9723941039117\\
6.28799999999953	45.9632014636827\\
6.28999999999953	45.9540106616135\\
6.29199999999953	45.9448216973372\\
6.29399999999953	45.9356345704855\\
6.29599999999953	45.926449280692\\
6.29799999999953	45.9172658275881\\
6.29999999999953	45.9080842108075\\
6.30199999999953	45.8989044299828\\
6.30399999999953	45.8897264847467\\
6.30599999999953	45.8805503747322\\
6.30799999999953	45.871376099573\\
6.30999999999953	45.8622036589014\\
6.31199999999953	45.8530330523508\\
6.31399999999953	45.8438642795548\\
6.31599999999953	45.8346973401457\\
6.31799999999953	45.825532233758\\
6.31999999999953	45.8163689600252\\
6.32199999999953	45.8072075185798\\
6.32399999999953	45.7980479090562\\
6.32599999999953	45.7888901310876\\
6.32799999999953	45.7797341843085\\
6.32999999999952	45.7705800683527\\
6.33199999999952	45.7614277828531\\
6.33399999999952	45.7522773274444\\
6.33599999999952	45.7431287017607\\
6.33799999999952	45.7339819054357\\
6.33999999999952	45.7248369381045\\
6.34199999999952	45.7156937993999\\
6.34399999999952	45.706552488958\\
6.34599999999952	45.6974130064125\\
6.34799999999952	45.6882753513967\\
6.34999999999952	45.6791395235468\\
6.35199999999952	45.6700055224969\\
6.35399999999952	45.6608733478812\\
6.35599999999952	45.6517429993354\\
6.35799999999952	45.6426144764934\\
6.35999999999952	45.6334877789912\\
6.36199999999952	45.624362906463\\
6.36399999999952	45.6152398585451\\
6.36599999999952	45.6061186348701\\
6.36799999999952	45.5969992350755\\
6.36999999999952	45.5878816587961\\
6.37199999999952	45.5787659056667\\
6.37399999999952	45.5696519753235\\
6.37599999999952	45.5605398674014\\
6.37799999999952	45.5514295815368\\
6.37999999999952	45.5423211173641\\
6.38199999999952	45.5332144745197\\
6.38399999999952	45.5241096526398\\
6.38599999999952	45.51500665136\\
6.38799999999952	45.5059054703152\\
6.38999999999952	45.4968061091435\\
6.39199999999952	45.4877085674798\\
6.39399999999952	45.4786128449608\\
6.39599999999952	45.469518941221\\
6.39799999999952	45.4604268558987\\
6.39999999999952	45.4513365886293\\
6.40199999999952	45.4422481390502\\
6.40399999999952	45.4331615067976\\
6.40599999999952	45.4240766915069\\
6.40799999999952	45.4149936928158\\
6.40999999999952	45.4059125103611\\
6.41199999999952	45.3968331437796\\
6.41399999999952	45.3877555927083\\
6.41599999999952	45.3786798567838\\
6.41799999999952	45.3696059356433\\
6.41999999999951	45.3605338289241\\
6.42199999999951	45.3514635362628\\
6.42399999999951	45.3423950572972\\
6.42599999999951	45.3333283916649\\
6.42799999999951	45.3242635390024\\
6.42999999999951	45.3152004989481\\
6.43199999999951	45.3061392711391\\
6.43399999999951	45.2970798552129\\
6.43599999999951	45.2880222508075\\
6.43799999999951	45.2789664575602\\
6.43999999999951	45.2699124751099\\
6.44199999999951	45.2608603030932\\
6.44399999999951	45.2518099411486\\
6.44599999999951	45.2427613889144\\
6.44799999999951	45.2337146460289\\
6.44999999999951	45.22466971213\\
6.45199999999951	45.2156265868552\\
6.45399999999951	45.2065852698438\\
6.45599999999951	45.1975457607344\\
6.45799999999951	45.188508059165\\
6.45999999999951	45.1794721647744\\
6.46199999999951	45.1704380772014\\
6.46399999999951	45.1614057960838\\
6.46599999999951	45.1523753210612\\
6.46799999999951	45.1433466517725\\
6.46999999999951	45.1343197878563\\
6.47199999999951	45.125294728951\\
6.47399999999951	45.1162714746972\\
6.47599999999951	45.1072500247325\\
6.47799999999951	45.098230378698\\
6.47999999999951	45.0892125362311\\
6.48199999999951	45.0801964969717\\
6.48399999999951	45.0711822605593\\
6.48599999999951	45.0621698266334\\
6.48799999999951	45.0531591948341\\
6.48999999999951	45.0441503648013\\
6.49199999999951	45.0351433361728\\
6.49399999999951	45.0261381085902\\
6.49599999999951	45.0171346816926\\
6.49799999999951	45.0081330551204\\
6.49999999999951	44.9991332285129\\
6.50199999999951	44.9901352015109\\
6.50399999999951	44.9811389737542\\
6.50599999999951	44.9721445448829\\
6.50799999999951	44.9631519145384\\
6.50999999999951	44.9541610823593\\
6.5119999999995	44.9451720479872\\
6.5139999999995	44.9361848110619\\
6.5159999999995	44.9271993712251\\
6.5179999999995	44.9182157281159\\
6.5199999999995	44.9092338813767\\
6.5219999999995	44.9002538306469\\
6.5239999999995	44.891275575568\\
6.5259999999995	44.8822991157813\\
6.5279999999995	44.8733244509276\\
6.5299999999995	44.864351580647\\
6.5319999999995	44.8553805045819\\
6.5339999999995	44.846411222373\\
6.5359999999995	44.8374437336619\\
6.5379999999995	44.8284780380896\\
6.5399999999995	44.8195141352976\\
6.5419999999995	44.810552024927\\
6.5439999999995	44.8015917066204\\
6.5459999999995	44.7926331800194\\
6.5479999999995	44.7836764447645\\
6.5499999999995	44.7747215004988\\
6.5519999999995	44.7657683468633\\
6.5539999999995	44.7568169835006\\
6.5559999999995	44.7478674100524\\
6.5579999999995	44.7389196261609\\
6.5599999999995	44.7299736314682\\
6.5619999999995	44.7210294256159\\
6.5639999999995	44.7120870082479\\
6.5659999999995	44.7031463790046\\
6.5679999999995	44.6942075375302\\
6.5699999999995	44.685270483466\\
6.5719999999995	44.6763352164554\\
6.5739999999995	44.6674017361409\\
6.5759999999995	44.6584700421644\\
6.5779999999995	44.6495401341698\\
6.5799999999995	44.6406120117996\\
6.5819999999995	44.6316856746959\\
6.5839999999995	44.6227611225036\\
6.5859999999995	44.6138383548636\\
6.5879999999995	44.6049173714207\\
6.5899999999995	44.5959981718178\\
6.5919999999995	44.587080755698\\
6.5939999999995	44.578165122704\\
6.5959999999995	44.5692512724803\\
6.5979999999995	44.5603392046704\\
6.5999999999995	44.5514289189166\\
6.60199999999949	44.542520414864\\
6.60399999999949	44.5336136921564\\
6.60599999999949	44.5247087504359\\
6.60799999999949	44.5158055893479\\
6.60999999999949	44.5069042085357\\
6.61199999999949	44.4980046076442\\
6.61399999999949	44.4891067863166\\
6.61599999999949	44.4802107441974\\
6.61799999999949	44.4713164809301\\
6.61999999999949	44.46242399616\\
6.62199999999949	44.4535332895314\\
6.62399999999949	44.4446443606878\\
6.62599999999949	44.4357572092748\\
6.62799999999949	44.4268718349365\\
6.62999999999949	44.417988237317\\
6.63199999999949	44.4091064160628\\
6.63399999999949	44.4002263708167\\
6.63599999999949	44.3913481012244\\
6.63799999999949	44.3824716069308\\
6.63999999999949	44.3735968875808\\
6.64199999999949	44.3647239428208\\
6.64399999999949	44.3558527722931\\
6.64599999999949	44.346983375646\\
6.64799999999949	44.3381157525228\\
6.64999999999949	44.3292499025693\\
6.65199999999949	44.3203858254312\\
6.65399999999949	44.3115235207541\\
6.65599999999949	44.3026629881832\\
6.65799999999949	44.2938042273646\\
6.65999999999949	44.2849472379432\\
6.66199999999949	44.2760920195661\\
6.66399999999949	44.267238571878\\
6.66599999999949	44.2583868945252\\
6.66799999999949	44.2495369871537\\
6.66999999999949	44.2406888494096\\
6.67199999999949	44.2318424809395\\
6.67399999999949	44.222997881389\\
6.67599999999949	44.2141550504045\\
6.67799999999949	44.2053139876324\\
6.67999999999949	44.1964746927191\\
6.68199999999949	44.1876371653114\\
6.68399999999949	44.1788014050556\\
6.68599999999949	44.1699674115979\\
6.68799999999949	44.1611351845863\\
6.68999999999949	44.1523047236663\\
6.69199999999949	44.1434760284852\\
6.69399999999948	44.1346490986902\\
6.69599999999948	44.1258239339281\\
6.69799999999948	44.1170005338458\\
6.69999999999948	44.1081788980899\\
6.70199999999948	44.0993590263089\\
6.70399999999948	44.0905409181495\\
6.70599999999948	44.0817245732586\\
6.70799999999948	44.0729099912839\\
6.70999999999948	44.0640971718732\\
6.71199999999948	44.0552861146742\\
6.71399999999948	44.0464768193334\\
6.71599999999948	44.0376692854998\\
6.71799999999948	44.028863512821\\
6.71999999999948	44.020059500944\\
6.72199999999948	44.0112572495166\\
6.72399999999948	44.0024567581884\\
6.72599999999948	43.9936580266055\\
6.72799999999948	43.9848610544174\\
6.72999999999948	43.9760658412716\\
6.73199999999948	43.9672723868169\\
6.73399999999948	43.9584806907016\\
6.73599999999948	43.9496907525731\\
6.73799999999948	43.9409025720809\\
6.73999999999948	43.9321161488731\\
6.74199999999948	43.9233314825986\\
6.74399999999948	43.9145485729059\\
6.74599999999948	43.9057674194445\\
6.74799999999948	43.8969880218615\\
6.74999999999948	43.8882103798067\\
6.75199999999948	43.8794344929295\\
6.75399999999948	43.8706603608787\\
6.75599999999948	43.8618879833038\\
6.75799999999948	43.853117359853\\
6.75999999999948	43.8443484901759\\
6.76199999999948	43.8355813739228\\
6.76399999999948	43.8268160107411\\
6.76599999999948	43.8180524002819\\
6.76799999999948	43.8092905421942\\
6.76999999999948	43.800530436128\\
6.77199999999948	43.791772081732\\
6.77399999999948	43.7830154786563\\
6.77599999999948	43.7742606265519\\
6.77799999999948	43.7655075250668\\
6.77999999999948	43.7567561738521\\
6.78199999999948	43.7480065725578\\
6.78399999999947	43.7392587208335\\
6.78599999999947	43.7305126183301\\
6.78799999999947	43.7217682646969\\
6.78999999999947	43.7130256595847\\
6.79199999999947	43.7042848026438\\
6.79399999999947	43.6955456935246\\
6.79599999999947	43.686808331878\\
6.79799999999947	43.6780727173538\\
6.79999999999947	43.6693388496034\\
6.80199999999947	43.6606067282767\\
6.80399999999947	43.6518763530252\\
6.80599999999947	43.6431477234997\\
6.80799999999947	43.6344208393505\\
6.80999999999947	43.6256957002294\\
6.81199999999947	43.6169723057873\\
6.81399999999947	43.6082506556749\\
6.81599999999947	43.5995307495434\\
6.81799999999947	43.5908125870444\\
6.81999999999947	43.5820961678293\\
6.82199999999947	43.5733814915493\\
6.82399999999947	43.5646685578559\\
6.82599999999947	43.5559573664006\\
6.82799999999947	43.5472479168359\\
6.82999999999947	43.538540208812\\
6.83199999999947	43.5298342419815\\
6.83399999999947	43.5211300159963\\
6.83599999999947	43.5124275305078\\
6.83799999999947	43.5037267851686\\
6.83999999999947	43.4950277796306\\
6.84199999999947	43.4863305135453\\
6.84399999999947	43.4776349865651\\
6.84599999999947	43.468941198343\\
6.84799999999947	43.4602491485306\\
6.84999999999947	43.4515588367811\\
6.85199999999947	43.4428702627454\\
6.85399999999947	43.4341834260768\\
6.85599999999947	43.4254983264285\\
6.85799999999947	43.4168149634528\\
6.85999999999947	43.4081333368029\\
6.86199999999947	43.39945344613\\
6.86399999999947	43.3907752910884\\
6.86599999999947	43.3820988713308\\
6.86799999999947	43.3734241865098\\
6.86999999999947	43.3647512362796\\
6.87199999999947	43.3560800202915\\
6.87399999999946	43.3474105382006\\
6.87599999999946	43.3387427896586\\
6.87799999999946	43.3300767743199\\
6.87999999999946	43.3214124918377\\
6.88199999999946	43.3127499418656\\
6.88399999999946	43.3040891240565\\
6.88599999999946	43.2954300380645\\
6.88799999999946	43.2867726835433\\
6.88999999999946	43.2781170601475\\
6.89199999999946	43.2694631675294\\
6.89399999999946	43.2608110053443\\
6.89599999999946	43.252160573245\\
6.89799999999946	43.2435118708866\\
6.89999999999946	43.234864897922\\
6.90199999999946	43.2262196540066\\
6.90399999999946	43.2175761387943\\
6.90599999999946	43.2089343519389\\
6.90799999999946	43.2002942930955\\
6.90999999999946	43.1916559619186\\
6.91199999999946	43.183019358062\\
6.91399999999946	43.1743844811807\\
6.91599999999946	43.1657513309293\\
6.91799999999946	43.1571199069624\\
6.91999999999946	43.1484902089354\\
6.92199999999946	43.1398622365028\\
6.92399999999946	43.1312359893193\\
6.92599999999946	43.1226114670405\\
6.92799999999946	43.1139886693213\\
6.92999999999946	43.1053675958167\\
6.93199999999946	43.0967482461811\\
6.93399999999946	43.0881306200716\\
6.93599999999946	43.0795147171427\\
6.93799999999946	43.070900537049\\
6.93999999999946	43.0622880794472\\
6.94199999999946	43.0536773439924\\
6.94399999999946	43.0450683303403\\
6.94599999999946	43.0364610381466\\
6.94799999999946	43.0278554670671\\
6.94999999999946	43.0192516167574\\
6.95199999999946	43.0106494868739\\
6.95399999999946	43.0020490770725\\
6.95599999999946	42.9934503870081\\
6.95799999999946	42.9848534163387\\
6.95999999999946	42.976258164719\\
6.96199999999946	42.9676646318061\\
6.96399999999946	42.9590728172562\\
6.96599999999945	42.9504827207251\\
6.96799999999945	42.94189434187\\
6.96999999999945	42.9333076803469\\
6.97199999999945	42.9247227358131\\
6.97399999999945	42.9161395079246\\
6.97599999999945	42.9075579963381\\
6.97799999999945	42.8989782007113\\
6.97999999999945	42.8904001207001\\
6.98199999999945	42.8818237559613\\
6.98399999999945	42.8732491061529\\
6.98599999999945	42.8646761709316\\
6.98799999999945	42.856104949954\\
6.98999999999945	42.847535442878\\
6.99199999999945	42.8389676493605\\
6.99399999999945	42.830401569059\\
6.99599999999945	42.8218372016313\\
6.99799999999945	42.8132745467345\\
6.99999999999945	42.8047136040259\\
7.00199999999945	42.7961543731635\\
7.00399999999945	42.7875968538047\\
7.00599999999945	42.7790410456078\\
7.00799999999945	42.7704869482306\\
7.00999999999945	42.7619345613299\\
7.01199999999945	42.7533838845657\\
7.01399999999945	42.7448349175937\\
7.01599999999945	42.7362876600734\\
7.01799999999945	42.7277421116631\\
7.01999999999945	42.7191982720205\\
7.02199999999945	42.7106561408041\\
7.02399999999945	42.7021157176718\\
7.02599999999945	42.6935770022834\\
7.02799999999945	42.685039994296\\
7.02999999999945	42.6765046933691\\
7.03199999999945	42.6679710991606\\
7.03399999999945	42.6594392113297\\
7.03599999999945	42.6509090295354\\
7.03799999999945	42.6423805534359\\
7.03999999999945	42.6338537826907\\
7.04199999999945	42.6253287169583\\
7.04399999999945	42.6168053558983\\
7.04599999999945	42.6082836991695\\
7.04799999999945	42.5997637464316\\
7.04999999999945	42.591245497343\\
7.05199999999945	42.5827289515634\\
7.05399999999945	42.5742141087528\\
7.05599999999944	42.5657009685701\\
7.05799999999944	42.5571895306742\\
7.05999999999944	42.5486797947267\\
7.06199999999944	42.5401717603852\\
7.06399999999944	42.5316654273104\\
7.06599999999944	42.523160795162\\
7.06799999999944	42.5146578635997\\
7.06999999999944	42.5061566322837\\
7.07199999999944	42.4976571008741\\
7.07399999999944	42.489159269031\\
7.07599999999944	42.4806631364141\\
7.07799999999944	42.4721687026839\\
7.07999999999944	42.4636759675004\\
7.08199999999944	42.4551849305246\\
7.08399999999944	42.4466955914171\\
7.08599999999944	42.438207949837\\
7.08799999999944	42.429722005446\\
7.08999999999944	42.4212377579043\\
7.09199999999944	42.4127552068732\\
7.09399999999944	42.4042743520127\\
7.09599999999944	42.3957951929845\\
7.09799999999944	42.3873177294481\\
7.09999999999944	42.3788419610659\\
7.10199999999944	42.3703678874979\\
7.10399999999944	42.361895508406\\
7.10599999999944	42.3534248234512\\
7.10799999999944	42.3449558322945\\
7.10999999999944	42.3364885345968\\
7.11199999999944	42.3280229300208\\
7.11399999999944	42.3195590182266\\
7.11599999999944	42.3110967988766\\
7.11799999999944	42.3026362716317\\
7.11999999999944	42.2941774361536\\
7.12199999999944	42.2857202921052\\
7.12399999999944	42.2772648391463\\
7.12599999999944	42.2688110769403\\
7.12799999999944	42.2603590051487\\
7.12999999999944	42.2519086234329\\
7.13199999999944	42.2434599314558\\
7.13399999999944	42.2350129288794\\
7.13599999999944	42.2265676153654\\
7.13799999999944	42.218123990576\\
7.13999999999944	42.2096820541746\\
7.14199999999944	42.2012418058223\\
7.14399999999944	42.1928032451824\\
7.14599999999944	42.1843663719171\\
7.14799999999943	42.1759311856884\\
7.14999999999943	42.1674976861601\\
7.15199999999943	42.1590658729941\\
7.15399999999943	42.1506357458538\\
7.15599999999943	42.1422073044018\\
7.15799999999943	42.1337805483004\\
7.15999999999943	42.1253554772138\\
7.16199999999943	42.1169320908046\\
7.16399999999943	42.1085103887352\\
7.16599999999943	42.10009037067\\
7.16799999999943	42.0916720362712\\
7.16999999999943	42.0832553852026\\
7.17199999999943	42.0748404171275\\
7.17399999999943	42.0664271317094\\
7.17599999999943	42.0580155286118\\
7.17799999999943	42.0496056074982\\
7.17999999999943	42.0411973680338\\
7.18199999999943	42.0327908098799\\
7.18399999999943	42.0243859327014\\
7.18599999999943	42.0159827361624\\
7.18799999999943	42.0075812199265\\
7.18999999999943	41.9991813836581\\
7.19199999999943	41.9907832270206\\
7.19399999999943	41.9823867496785\\
7.19599999999943	41.9739919512965\\
7.19799999999943	41.9655988315378\\
7.19999999999943	41.9572073900682\\
7.20199999999943	41.9488176265508\\
7.20399999999943	41.94042954065\\
7.20599999999943	41.9320431320317\\
7.20799999999943	41.9236584003593\\
7.20999999999943	41.9152753452983\\
7.21199999999943	41.9068939665123\\
7.21399999999943	41.8985142636673\\
7.21599999999943	41.8901362364278\\
7.21799999999943	41.8817598844585\\
7.21999999999943	41.8733852074249\\
7.22199999999943	41.8650122049908\\
7.22399999999943	41.8566408768233\\
7.22599999999943	41.8482712225864\\
7.22799999999943	41.8399032419454\\
7.22999999999943	41.8315369345659\\
7.23199999999943	41.8231723001131\\
7.23399999999943	41.8148093382531\\
7.23599999999943	41.8064480486505\\
7.23799999999942	41.7980884309717\\
7.23999999999942	41.789730484882\\
7.24199999999942	41.7813742100471\\
7.24399999999942	41.773019606133\\
7.24599999999942	41.764666672806\\
7.24799999999942	41.7563154097308\\
7.24999999999942	41.7479658165747\\
7.25199999999942	41.7396178930034\\
7.25399999999942	41.7312716386821\\
7.25599999999942	41.7229270532792\\
7.25799999999942	41.7145841364584\\
7.25999999999942	41.7062428878878\\
7.26199999999942	41.6979033072336\\
7.26399999999942	41.6895653941616\\
7.26599999999942	41.6812291483391\\
7.26799999999942	41.6728945694315\\
7.26999999999942	41.664561657107\\
7.27199999999942	41.6562304110316\\
7.27399999999942	41.6479008308721\\
7.27599999999942	41.6395729162957\\
7.27799999999942	41.6312466669689\\
7.27999999999942	41.622922082559\\
7.28199999999942	41.6145991627332\\
7.28399999999942	41.6062779071581\\
7.28599999999942	41.5979583155019\\
7.28799999999942	41.5896403874308\\
7.28999999999942	41.5813241226127\\
7.29199999999942	41.5730095207151\\
7.29399999999942	41.5646965814052\\
7.29599999999942	41.5563853043508\\
7.29799999999942	41.5480756892196\\
7.29999999999942	41.5397677356787\\
7.30199999999942	41.531461443396\\
7.30399999999942	41.5231568120402\\
7.30599999999942	41.5148538412784\\
7.30799999999942	41.506552530778\\
7.30999999999942	41.4982528802084\\
7.31199999999942	41.4899548892371\\
7.31399999999942	41.4816585575317\\
7.31599999999942	41.4733638847611\\
7.31799999999942	41.465070870593\\
7.31999999999942	41.4567795146961\\
7.32199999999942	41.4484898167388\\
7.32399999999942	41.4402017763896\\
7.32599999999942	41.4319153933165\\
7.32799999999941	41.4236306671895\\
7.32999999999941	41.4153475976762\\
7.33199999999941	41.4070661844448\\
7.33399999999941	41.3987864271656\\
7.33599999999941	41.3905083255061\\
7.33799999999941	41.3822318791361\\
7.33999999999941	41.3739570877245\\
7.34199999999941	41.3656839509403\\
7.34399999999941	41.3574124684521\\
7.34599999999941	41.34914263993\\
7.34799999999941	41.3408744650424\\
7.34999999999941	41.3326079434592\\
7.35199999999941	41.3243430748509\\
7.35399999999941	41.3160798588843\\
7.35599999999941	41.307818295231\\
7.35799999999941	41.2995583835601\\
7.35999999999941	41.2913001235409\\
7.36199999999941	41.2830435148432\\
7.36399999999941	41.2747885571377\\
7.36599999999941	41.2665352500934\\
7.36799999999941	41.2582835933799\\
7.36999999999941	41.2500335866679\\
7.37199999999941	41.2417852296277\\
7.37399999999941	41.2335385219285\\
7.37599999999941	41.2252934632411\\
7.37799999999941	41.2170500532359\\
7.37999999999941	41.2088082915835\\
7.38199999999941	41.2005681779528\\
7.38399999999941	41.1923297120156\\
7.38599999999941	41.1840928934424\\
7.38799999999941	41.1758577219034\\
7.38999999999941	41.1676241970689\\
7.39199999999941	41.1593923186108\\
7.39399999999941	41.1511620861984\\
7.39599999999941	41.142933499504\\
7.39799999999941	41.1347065581971\\
7.39999999999941	41.1264812619504\\
7.40199999999941	41.1182576104334\\
7.40399999999941	41.1100356033177\\
7.40599999999941	41.1018152402747\\
7.40799999999941	41.0935965209755\\
7.40999999999941	41.0853794450914\\
7.41199999999941	41.0771640122941\\
7.41399999999941	41.068950222255\\
7.41599999999941	41.0607380746448\\
7.41799999999941	41.0525275691356\\
7.4199999999994	41.0443187053997\\
7.4219999999994	41.0361114831081\\
7.4239999999994	41.0279059019323\\
7.4259999999994	41.0197019615445\\
7.4279999999994	41.0114996616177\\
7.4299999999994	41.0032990018223\\
7.4319999999994	40.99509998183\\
7.4339999999994	40.986902601315\\
7.4359999999994	40.9787068599475\\
7.4379999999994	40.9705127574011\\
7.4399999999994	40.9623202933471\\
7.4419999999994	40.9541294674584\\
7.4439999999994	40.9459402794072\\
7.4459999999994	40.9377527288657\\
7.4479999999994	40.9295668155074\\
7.4499999999994	40.9213825390043\\
7.4519999999994	40.9131998990291\\
7.4539999999994	40.9050188952543\\
7.4559999999994	40.8968395273535\\
7.4579999999994	40.888661794999\\
7.4599999999994	40.8804856978639\\
7.4619999999994	40.8723112356215\\
7.4639999999994	40.8641384079447\\
7.4659999999994	40.8559672145056\\
7.4679999999994	40.8477976549797\\
7.4699999999994	40.8396297290377\\
7.4719999999994	40.8314634363549\\
7.4739999999994	40.823298776604\\
7.4759999999994	40.8151357494587\\
7.4779999999994	40.8069743545924\\
7.4799999999994	40.7988145916777\\
7.4819999999994	40.7906564603905\\
7.4839999999994	40.7824999604026\\
7.4859999999994	40.7743450913887\\
7.4879999999994	40.7661918530225\\
7.4899999999994	40.7580402449774\\
7.4919999999994	40.7498902669284\\
7.4939999999994	40.7417419185485\\
7.4959999999994	40.7335951995134\\
7.4979999999994	40.7254501094952\\
7.4999999999994	40.7173066481699\\
7.5019999999994	40.709164815211\\
7.5039999999994	40.7010246102928\\
7.5059999999994	40.6928860330906\\
7.5079999999994	40.6847490832777\\
7.50999999999939	40.6766137605299\\
7.51199999999939	40.668480064521\\
7.51399999999939	40.6603479949258\\
7.51599999999939	40.6522175514199\\
7.51799999999939	40.6440887336772\\
7.51999999999939	40.6359615413731\\
7.52199999999939	40.6278359741819\\
7.52399999999939	40.6197120317796\\
7.52599999999939	40.6115897138405\\
7.52799999999939	40.6034690200405\\
7.52999999999939	40.5953499500542\\
7.53199999999939	40.5872325035574\\
7.53399999999939	40.5791166802256\\
7.53599999999939	40.5710024797333\\
7.53799999999939	40.5628899017566\\
7.53999999999939	40.5547789459714\\
7.54199999999939	40.546669612053\\
7.54399999999939	40.5385618996771\\
7.54599999999939	40.5304558085182\\
7.54799999999939	40.5223513382548\\
7.54999999999939	40.514248488561\\
7.55199999999939	40.5061472591127\\
7.55399999999939	40.4980476495866\\
7.55599999999939	40.4899496596586\\
7.55799999999939	40.4818532890042\\
7.55999999999939	40.4737585373004\\
7.56199999999939	40.4656654042232\\
7.56399999999939	40.4575738894491\\
7.56599999999939	40.4494839926539\\
7.56799999999939	40.4413957135151\\
7.56999999999939	40.4333090517082\\
7.57199999999939	40.4252240069106\\
7.57399999999939	40.4171405787981\\
7.57599999999939	40.409058767048\\
7.57799999999939	40.4009785713375\\
7.57999999999939	40.3928999913424\\
7.58199999999939	40.3848230267409\\
7.58399999999939	40.376747677209\\
7.58599999999939	40.3686739424241\\
7.58799999999939	40.3606018220626\\
7.58999999999939	40.3525313158031\\
7.59199999999939	40.344462423322\\
7.59399999999939	40.3363951442961\\
7.59599999999939	40.3283294784041\\
7.59799999999939	40.3202654253224\\
7.59999999999939	40.3122029847287\\
7.60199999999938	40.3041421563007\\
7.60399999999938	40.2960829397158\\
7.60599999999938	40.2880253346527\\
7.60799999999938	40.2799693407876\\
7.60999999999938	40.2719149577994\\
7.61199999999938	40.2638621853656\\
7.61399999999938	40.2558110231641\\
7.61599999999938	40.2477614708734\\
7.61799999999938	40.2397135281711\\
7.61999999999938	40.2316671947353\\
7.62199999999938	40.2236224702445\\
7.62399999999938	40.2155793543768\\
7.62599999999938	40.2075378468106\\
7.62799999999938	40.1994979472241\\
7.62999999999938	40.1914596552963\\
7.63199999999938	40.1834229707055\\
7.63399999999938	40.1753878931297\\
7.63599999999938	40.1673544222486\\
7.63799999999938	40.1593225577401\\
7.63999999999938	40.1512922992833\\
7.64199999999938	40.1432636465571\\
7.64399999999938	40.1352365992402\\
7.64599999999938	40.1272111570121\\
7.64799999999938	40.1191873195508\\
7.64999999999938	40.111165086537\\
7.65199999999938	40.1031444576483\\
7.65399999999938	40.0951254325651\\
7.65599999999938	40.087108010966\\
7.65799999999938	40.0790921925301\\
7.65999999999938	40.0710779769377\\
7.66199999999938	40.0630653638677\\
7.66399999999938	40.0550543529996\\
7.66599999999938	40.0470449440141\\
7.66799999999938	40.0390371365899\\
7.66999999999938	40.0310309304069\\
7.67199999999938	40.0230263251447\\
7.67399999999938	40.0150233204833\\
7.67599999999938	40.0070219161032\\
7.67799999999938	39.9990221116835\\
7.67999999999938	39.9910239069047\\
7.68199999999938	39.9830273014477\\
7.68399999999938	39.9750322949916\\
7.68599999999938	39.9670388872167\\
7.68799999999938	39.959047077804\\
7.68999999999938	39.9510568664336\\
7.69199999999937	39.9430682527856\\
7.69399999999937	39.9350812365408\\
7.69599999999937	39.9270958173807\\
7.69799999999937	39.9191119949835\\
7.69999999999937	39.9111297690329\\
7.70199999999937	39.9031491392078\\
7.70399999999937	39.8951701051897\\
7.70599999999937	39.8871926666589\\
7.70799999999937	39.879216823297\\
7.70999999999937	39.8712425747844\\
7.71199999999937	39.8632699208031\\
7.71399999999937	39.8552988610338\\
7.71599999999937	39.8473293951568\\
7.71799999999937	39.8393615228554\\
7.71999999999937	39.8313952438093\\
7.72199999999937	39.8234305577004\\
7.72399999999937	39.81546746421\\
7.72599999999937	39.8075059630198\\
7.72799999999937	39.7995460538118\\
7.72999999999937	39.7915877362672\\
7.73199999999937	39.7836310100676\\
7.73399999999937	39.7756758748952\\
7.73599999999937	39.7677223304313\\
7.73799999999937	39.759770376359\\
7.73999999999937	39.7518200123586\\
7.74199999999937	39.7438712381138\\
7.74399999999937	39.7359240533054\\
7.74599999999937	39.7279784576168\\
7.74799999999937	39.720034450729\\
7.74999999999937	39.7120920323247\\
7.75199999999937	39.7041512020867\\
7.75399999999937	39.6962119596971\\
7.75599999999937	39.6882743048389\\
7.75799999999937	39.6803382371938\\
7.75999999999937	39.6724037564448\\
7.76199999999937	39.6644708622752\\
7.76399999999937	39.6565395543664\\
7.76599999999937	39.6486098324025\\
7.76799999999937	39.6406816960653\\
7.76999999999937	39.6327551450389\\
7.77199999999937	39.6248301790054\\
7.77399999999937	39.6169067976484\\
7.77599999999937	39.6089850006503\\
7.77799999999937	39.6010647876948\\
7.77999999999937	39.5931461584656\\
7.78199999999936	39.5852291126455\\
7.78399999999936	39.5773136499177\\
7.78599999999936	39.5693997699656\\
7.78799999999936	39.5614874724738\\
7.78999999999936	39.5535767571246\\
7.79199999999936	39.545667623602\\
7.79399999999936	39.5377600715903\\
7.79599999999936	39.529854100772\\
7.79799999999936	39.5219497108323\\
7.79999999999936	39.514046901454\\
7.80199999999936	39.5061456723221\\
7.80399999999936	39.4982460231193\\
7.80599999999936	39.4903479535311\\
7.80799999999936	39.4824514632405\\
7.80999999999936	39.4745565519323\\
7.81199999999936	39.4666632192907\\
7.81399999999936	39.458771465\\
7.81599999999936	39.4508812887439\\
7.81799999999936	39.442992690208\\
7.81999999999936	39.4351056690767\\
7.82199999999936	39.4272202250337\\
7.82399999999936	39.4193363577641\\
7.82599999999936	39.411454066953\\
7.82799999999936	39.4035733522843\\
7.82999999999936	39.3956942134439\\
7.83199999999936	39.3878166501157\\
7.83399999999936	39.3799406619853\\
7.83599999999936	39.3720662487379\\
7.83799999999936	39.3641934100579\\
7.83999999999936	39.3563221456306\\
7.84199999999936	39.3484524551419\\
7.84399999999936	39.3405843382761\\
7.84599999999936	39.3327177947191\\
7.84799999999936	39.3248528241566\\
7.84999999999936	39.3169894262734\\
7.85199999999936	39.3091276007551\\
7.85399999999936	39.3012673472875\\
7.85599999999936	39.2934086655565\\
7.85799999999936	39.2855515552471\\
7.85999999999936	39.2776960160463\\
7.86199999999936	39.2698420476388\\
7.86399999999936	39.2619896497106\\
7.86599999999936	39.2541388219481\\
7.86799999999936	39.2462895640373\\
7.86999999999936	39.2384418756637\\
7.87199999999936	39.2305957565149\\
7.87399999999935	39.2227512062752\\
7.87599999999935	39.2149082246323\\
7.87799999999935	39.2070668112719\\
7.87999999999935	39.1992269658805\\
7.88199999999935	39.1913886881449\\
7.88399999999935	39.1835519777512\\
7.88599999999935	39.1757168343859\\
7.88799999999935	39.1678832577363\\
7.88999999999935	39.1600512474885\\
7.89199999999935	39.1522208033299\\
7.89399999999935	39.1443919249468\\
7.89599999999935	39.1365646120264\\
7.89799999999935	39.128738864255\\
7.89999999999935	39.1209146813207\\
7.90199999999935	39.1130920629098\\
7.90399999999935	39.1052710087098\\
7.90599999999935	39.0974515184082\\
7.90799999999935	39.0896335916913\\
7.90999999999935	39.0818172282476\\
7.91199999999935	39.0740024277638\\
7.91399999999935	39.0661891899275\\
7.91599999999935	39.0583775144263\\
7.91799999999935	39.0505674009479\\
7.91999999999935	39.0427588491799\\
7.92199999999935	39.0349518588097\\
7.92399999999935	39.0271464295258\\
7.92599999999935	39.0193425610152\\
7.92799999999935	39.0115402529664\\
7.92999999999935	39.003739505067\\
7.93199999999935	38.995940317005\\
7.93399999999935	38.9881426884695\\
7.93599999999935	38.980346619147\\
7.93799999999935	38.9725521087269\\
7.93999999999935	38.9647591568969\\
7.94199999999935	38.9569677633458\\
7.94399999999935	38.9491779277625\\
7.94599999999935	38.9413896498336\\
7.94799999999935	38.93360292925\\
7.94999999999935	38.9258177656981\\
7.95199999999935	38.9180341588679\\
7.95399999999935	38.9102521084482\\
7.95599999999935	38.9024716141269\\
7.95799999999935	38.8946926755934\\
7.95999999999935	38.8869152925364\\
7.96199999999935	38.8791394646452\\
7.96399999999934	38.8713651916082\\
7.96599999999934	38.8635924731151\\
7.96799999999934	38.8558213088552\\
7.96999999999934	38.8480516985169\\
7.97199999999934	38.84028364179\\
7.97399999999934	38.832517138364\\
7.97599999999934	38.824752187928\\
7.97799999999934	38.8169887901715\\
7.97999999999934	38.8092269447841\\
7.98199999999934	38.8014666514549\\
7.98399999999934	38.7937079098745\\
7.98599999999934	38.7859507197324\\
7.98799999999934	38.7781950807178\\
7.98999999999934	38.7704409925207\\
7.99199999999934	38.7626884548314\\
7.99399999999934	38.7549374673392\\
7.99599999999934	38.7471880297347\\
7.99799999999934	38.7394401417078\\
7.99999999999934	38.7316938029486\\
};
\addplot [color=mycolor1, forget plot]
  table[row sep=crcr]{%
7.99999999999934	38.7316938029486\\
8.00199999999934	38.7239490131477\\
8.00399999999934	38.7162057719948\\
8.00599999999934	38.7084640791803\\
8.00799999999934	38.7007239343949\\
8.00999999999934	38.6929853373291\\
8.01199999999934	38.6852482876728\\
8.01399999999935	38.6775127851172\\
8.01599999999935	38.6697788293528\\
8.01799999999935	38.66204642007\\
8.01999999999935	38.6543155569599\\
8.02199999999935	38.6465862397134\\
8.02399999999935	38.6388584680214\\
8.02599999999935	38.6311322415744\\
8.02799999999935	38.623407560064\\
8.02999999999935	38.6156844231807\\
8.03199999999935	38.6079628306159\\
8.03399999999935	38.6002427820606\\
8.03599999999935	38.5925242772063\\
8.03799999999935	38.5848073157449\\
8.03999999999935	38.5770918973671\\
8.04199999999935	38.5693780217636\\
8.04399999999936	38.5616656886271\\
8.04599999999936	38.5539548976493\\
8.04799999999936	38.5462456485208\\
8.04999999999936	38.538537940934\\
8.05199999999936	38.5308317745807\\
8.05399999999936	38.5231271491519\\
8.05599999999936	38.5154240643404\\
8.05799999999936	38.5077225198375\\
8.05999999999936	38.5000225153359\\
8.06199999999936	38.4923240505273\\
8.06399999999936	38.4846271251034\\
8.06599999999936	38.4769317387567\\
8.06799999999936	38.4692378911799\\
8.06999999999936	38.4615455820648\\
8.07199999999937	38.4538548111035\\
8.07399999999937	38.4461655779896\\
8.07599999999937	38.4384778824145\\
8.07799999999937	38.4307917240707\\
8.07999999999937	38.4231071026516\\
8.08199999999937	38.4154240178494\\
8.08399999999937	38.4077424693566\\
8.08599999999937	38.4000624568664\\
8.08799999999937	38.3923839800717\\
8.08999999999937	38.3847070386652\\
8.09199999999937	38.3770316323398\\
8.09399999999937	38.3693577607891\\
8.09599999999937	38.3616854237057\\
8.09799999999937	38.3540146207824\\
8.09999999999937	38.3463453517129\\
8.10199999999938	38.3386776161915\\
8.10399999999938	38.3310114139097\\
8.10599999999938	38.3233467445619\\
8.10799999999938	38.3156836078413\\
8.10999999999938	38.3080220034419\\
8.11199999999938	38.3003619310573\\
8.11399999999938	38.2927033903799\\
8.11599999999938	38.2850463811053\\
8.11799999999938	38.2773909029255\\
8.11999999999938	38.2697369555356\\
8.12199999999938	38.262084538629\\
8.12399999999938	38.2544336519\\
8.12599999999938	38.2467842950421\\
8.12799999999938	38.2391364677495\\
8.12999999999938	38.2314901697168\\
8.13199999999939	38.2238454006377\\
8.13399999999939	38.2162021602069\\
8.13599999999939	38.2085604481182\\
8.13799999999939	38.2009202640664\\
8.13999999999939	38.1932816077463\\
8.14199999999939	38.185644478852\\
8.14399999999939	38.1780088770772\\
8.14599999999939	38.1703748021179\\
8.14799999999939	38.1627422536681\\
8.14999999999939	38.155111231423\\
8.15199999999939	38.1474817350772\\
8.15399999999939	38.1398537643252\\
8.15599999999939	38.1322273188626\\
8.15799999999939	38.124602398384\\
8.15999999999939	38.1169790025847\\
8.1619999999994	38.1093571311596\\
8.1639999999994	38.1017367838039\\
8.1659999999994	38.0941179602127\\
8.1679999999994	38.0865006600817\\
8.1699999999994	38.0788848831061\\
8.1719999999994	38.0712706289818\\
8.1739999999994	38.063657897403\\
8.1759999999994	38.0560466880667\\
8.1779999999994	38.0484370006671\\
8.1799999999994	38.0408288349015\\
8.1819999999994	38.0332221904639\\
8.1839999999994	38.0256170670519\\
8.1859999999994	38.0180134643603\\
8.1879999999994	38.0104113820846\\
8.1899999999994	38.0028108199214\\
8.19199999999941	37.9952117775671\\
8.19399999999941	37.9876142547174\\
8.19599999999941	37.9800182510679\\
8.19799999999941	37.9724237663151\\
8.19999999999941	37.9648308001567\\
8.20199999999941	37.9572393522871\\
8.20399999999941	37.9496494224037\\
8.20599999999941	37.9420610102028\\
8.20799999999941	37.9344741153801\\
8.20999999999941	37.926888737634\\
8.21199999999941	37.9193048766593\\
8.21399999999941	37.9117225321538\\
8.21599999999941	37.904141703814\\
8.21799999999941	37.896562391337\\
8.21999999999941	37.888984594419\\
8.22199999999942	37.8814083127572\\
8.22399999999942	37.8738335460494\\
8.22599999999942	37.8662602939912\\
8.22799999999942	37.8586885562809\\
8.22999999999942	37.8511183326151\\
8.23199999999942	37.8435496226921\\
8.23399999999942	37.8359824262075\\
8.23599999999942	37.82841674286\\
8.23799999999942	37.8208525723464\\
8.23999999999942	37.8132899143642\\
8.24199999999942	37.8057287686111\\
8.24399999999942	37.7981691347854\\
8.24599999999942	37.7906110125838\\
8.24799999999942	37.7830544017043\\
8.24999999999942	37.7754993018448\\
8.25199999999943	37.7679457127031\\
8.25399999999943	37.7603936339774\\
8.25599999999943	37.7528430653653\\
8.25799999999943	37.7452940065647\\
8.25999999999943	37.7377464572738\\
8.26199999999943	37.7302004171915\\
8.26399999999943	37.7226558860153\\
8.26599999999943	37.7151128634431\\
8.26799999999943	37.7075713491743\\
8.26999999999943	37.7000313429067\\
8.27199999999943	37.6924928443387\\
8.27399999999943	37.6849558531688\\
8.27599999999943	37.6774203690959\\
8.27799999999943	37.6698863918183\\
8.27999999999943	37.6623539210349\\
8.28199999999944	37.6548229564444\\
8.28399999999944	37.6472934977459\\
8.28599999999944	37.6397655446382\\
8.28799999999944	37.6322390968189\\
8.28999999999944	37.6247141539893\\
8.29199999999944	37.6171907158469\\
8.29399999999944	37.6096687820912\\
8.29599999999944	37.6021483524218\\
8.29799999999944	37.5946294265372\\
8.29999999999944	37.5871120041374\\
8.30199999999944	37.5795960849211\\
8.30399999999944	37.5720816685878\\
8.30599999999944	37.5645687548376\\
8.30799999999944	37.5570573433701\\
8.30999999999944	37.549547433884\\
8.31199999999945	37.5420390260794\\
8.31399999999945	37.5345321196563\\
8.31599999999945	37.5270267143142\\
8.31799999999945	37.519522809753\\
8.31999999999945	37.5120204056726\\
8.32199999999945	37.5045195017731\\
8.32399999999945	37.4970200977544\\
8.32599999999945	37.4895221933165\\
8.32799999999945	37.4820257881594\\
8.32999999999945	37.4745308819839\\
8.33199999999945	37.46703747449\\
8.33399999999945	37.4595455653773\\
8.33599999999945	37.4520551543474\\
8.33799999999945	37.4445662410999\\
8.33999999999945	37.4370788253356\\
8.34199999999946	37.4295929067552\\
8.34399999999946	37.4221084850587\\
8.34599999999946	37.4146255599474\\
8.34799999999946	37.407144131122\\
8.34999999999946	37.3996641982828\\
8.35199999999946	37.3921857611317\\
8.35399999999946	37.3847088193685\\
8.35599999999946	37.3772333726946\\
8.35799999999946	37.369759420811\\
8.35999999999946	37.3622869634194\\
8.36199999999946	37.3548160002198\\
8.36399999999946	37.3473465309141\\
8.36599999999946	37.3398785552041\\
8.36799999999946	37.3324120727905\\
8.36999999999946	37.3249470833739\\
8.37199999999947	37.3174835866576\\
8.37399999999947	37.3100215823419\\
8.37599999999947	37.3025610701284\\
8.37799999999947	37.2951020497192\\
8.37999999999947	37.2876445208156\\
8.38199999999947	37.2801884831198\\
8.38399999999947	37.272733936333\\
8.38599999999947	37.2652808801576\\
8.38799999999947	37.2578293142955\\
8.38999999999947	37.2503792384483\\
8.39199999999947	37.2429306523186\\
8.39399999999947	37.2354835556084\\
8.39599999999947	37.2280379480197\\
8.39799999999947	37.2205938292544\\
8.39999999999947	37.2131511990151\\
8.40199999999948	37.2057100570047\\
8.40399999999948	37.1982704029247\\
8.40599999999948	37.1908322364779\\
8.40799999999948	37.1833955573671\\
8.40999999999948	37.175960365295\\
8.41199999999948	37.1685266599629\\
8.41399999999948	37.161094441076\\
8.41599999999948	37.1536637083345\\
8.41799999999948	37.146234461443\\
8.41999999999948	37.1388067001035\\
8.42199999999948	37.1313804240195\\
8.42399999999948	37.1239556328932\\
8.42599999999948	37.116532326429\\
8.42799999999948	37.1091105043287\\
8.42999999999948	37.1016901662959\\
8.43199999999949	37.0942713120339\\
8.43399999999949	37.0868539412465\\
8.43599999999949	37.0794380536363\\
8.43799999999949	37.0720236489076\\
8.43999999999949	37.0646107267626\\
8.44199999999949	37.0571992869058\\
8.44399999999949	37.0497893290408\\
8.44599999999949	37.0423808528708\\
8.44799999999949	37.0349738580995\\
8.44999999999949	37.0275683444313\\
8.45199999999949	37.02016431157\\
8.45399999999949	37.0127617592187\\
8.45599999999949	37.0053606870814\\
8.45799999999949	36.997961094863\\
8.45999999999949	36.9905629822674\\
8.4619999999995	36.9831663489983\\
8.4639999999995	36.9757711947594\\
8.4659999999995	36.9683775192563\\
8.4679999999995	36.9609853221926\\
8.4699999999995	36.9535946032725\\
8.4719999999995	36.9462053622004\\
8.4739999999995	36.9388175986814\\
8.4759999999995	36.9314313124195\\
8.4779999999995	36.9240465031195\\
8.4799999999995	36.9166631704861\\
8.4819999999995	36.9092813142237\\
8.4839999999995	36.9019009340376\\
8.4859999999995	36.8945220296322\\
8.4879999999995	36.8871446007126\\
8.4899999999995	36.8797686469837\\
8.49199999999951	36.8723941681506\\
8.49399999999951	36.8650211639184\\
8.49599999999951	36.8576496339922\\
8.49799999999951	36.8502795780772\\
8.49999999999951	36.8429109958785\\
8.50199999999951	36.8355438871014\\
8.50399999999951	36.8281782514518\\
8.50599999999951	36.8208140886347\\
8.50799999999951	36.8134513983554\\
8.50999999999951	36.8060901803202\\
8.51199999999951	36.7987304342336\\
8.51399999999951	36.7913721598021\\
8.51599999999951	36.7840153567314\\
8.51799999999951	36.7766600247269\\
8.51999999999951	36.7693061634947\\
8.52199999999952	36.7619537727413\\
8.52399999999952	36.7546028521708\\
8.52599999999952	36.7472534014912\\
8.52799999999952	36.7399054204073\\
8.52999999999952	36.7325589086264\\
8.53199999999952	36.7252138658536\\
8.53399999999952	36.717870291796\\
8.53599999999952	36.7105281861591\\
8.53799999999952	36.70318754865\\
8.53999999999952	36.6958483789746\\
8.54199999999952	36.6885106768398\\
8.54399999999952	36.6811744419518\\
8.54599999999952	36.6738396740174\\
8.54799999999952	36.6665063727434\\
8.54999999999952	36.6591745378365\\
8.55199999999953	36.6518441690032\\
8.55399999999953	36.6445152659507\\
8.55599999999953	36.6371878283852\\
8.55799999999953	36.6298618560142\\
8.55999999999953	36.6225373485453\\
8.56199999999953	36.615214305685\\
8.56399999999953	36.6078927271394\\
8.56599999999953	36.6005726126177\\
8.56799999999953	36.5932539618256\\
8.56999999999953	36.5859367744711\\
8.57199999999953	36.5786210502616\\
8.57399999999953	36.5713067889044\\
8.57599999999953	36.5639939901069\\
8.57799999999953	36.5566826535767\\
8.57999999999953	36.5493727790212\\
8.58199999999954	36.5420643661487\\
8.58399999999954	36.5347574146661\\
8.58599999999954	36.5274519242815\\
8.58799999999954	36.5201478947032\\
8.58999999999954	36.5128453256383\\
8.59199999999954	36.5055442167957\\
8.59399999999954	36.4982445678824\\
8.59599999999954	36.4909463786073\\
8.59799999999954	36.4836496486778\\
8.59999999999954	36.4763543778032\\
8.60199999999954	36.4690605656902\\
8.60399999999954	36.4617682120484\\
8.60599999999954	36.4544773165854\\
8.60799999999954	36.4471878790101\\
8.60999999999954	36.4398998990307\\
8.61199999999955	36.4326133763555\\
8.61399999999955	36.4253283106938\\
8.61599999999955	36.4180447017541\\
8.61799999999955	36.4107625492448\\
8.61999999999955	36.4034818528747\\
8.62199999999955	36.3962026123526\\
8.62399999999955	36.3889248273875\\
8.62599999999955	36.3816484976884\\
8.62799999999955	36.3743736229642\\
8.62999999999955	36.367100202924\\
8.63199999999955	36.359828237277\\
8.63399999999955	36.3525577257322\\
8.63599999999955	36.345288667999\\
8.63799999999955	36.338021063787\\
8.63999999999955	36.3307549128046\\
8.64199999999956	36.3234902147621\\
8.64399999999956	36.3162269693686\\
8.64599999999956	36.3089651763338\\
8.64799999999956	36.3017048353671\\
8.64999999999956	36.2944459461782\\
8.65199999999956	36.2871885084769\\
8.65399999999956	36.2799325219732\\
8.65599999999956	36.2726779863764\\
8.65799999999956	36.2654249013964\\
8.65999999999956	36.2581732667433\\
8.66199999999956	36.2509230821273\\
8.66399999999956	36.2436743472583\\
8.66599999999956	36.2364270618467\\
8.66799999999956	36.2291812256016\\
8.66999999999956	36.2219368382343\\
8.67199999999957	36.2146938994547\\
8.67399999999957	36.2074524089734\\
8.67599999999957	36.2002123665004\\
8.67799999999957	36.1929737717465\\
8.67999999999957	36.1857366244217\\
8.68199999999957	36.1785009242377\\
8.68399999999957	36.1712666709039\\
8.68599999999957	36.1640338641313\\
8.68799999999957	36.1568025036311\\
8.68999999999957	36.1495725891133\\
8.69199999999957	36.1423441202899\\
8.69399999999957	36.1351170968707\\
8.69599999999957	36.1278915185677\\
8.69799999999957	36.1206673850906\\
8.69999999999957	36.1134446961522\\
8.70199999999958	36.1062234514622\\
8.70399999999958	36.0990036507324\\
8.70599999999958	36.091785293674\\
8.70799999999958	36.0845683799981\\
8.70999999999958	36.0773529094168\\
8.71199999999958	36.0701388816408\\
8.71399999999958	36.062926296382\\
8.71599999999958	36.0557151533515\\
8.71799999999958	36.0485054522616\\
8.71999999999958	36.0412971928234\\
8.72199999999958	36.0340903747486\\
8.72399999999958	36.0268849977493\\
8.72599999999958	36.0196810615375\\
8.72799999999958	36.0124785658246\\
8.72999999999958	36.005277510323\\
8.73199999999959	35.9980778947446\\
8.73399999999959	35.990879718801\\
8.73599999999959	35.9836829822051\\
8.73799999999959	35.9764876846682\\
8.73999999999959	35.9692938259035\\
8.74199999999959	35.9621014056228\\
8.74399999999959	35.9549104235384\\
8.74599999999959	35.9477208793629\\
8.74799999999959	35.9405327728087\\
8.74999999999959	35.9333461035881\\
8.75199999999959	35.9261608714147\\
8.75399999999959	35.9189770759995\\
8.75599999999959	35.9117947170563\\
8.75799999999959	35.904613794298\\
8.75999999999959	35.8974343074363\\
8.7619999999996	35.8902562561858\\
8.7639999999996	35.8830796402582\\
8.7659999999996	35.8759044593665\\
8.7679999999996	35.868730713224\\
8.7699999999996	35.8615584015441\\
8.7719999999996	35.8543875240396\\
8.7739999999996	35.8472180804238\\
8.7759999999996	35.84005007041\\
8.7779999999996	35.832883493712\\
8.7799999999996	35.8257183500422\\
8.7819999999996	35.8185546391146\\
8.7839999999996	35.8113923606425\\
8.7859999999996	35.8042315143401\\
8.7879999999996	35.7970720999204\\
8.7899999999996	35.7899141170973\\
8.79199999999961	35.7827575655848\\
8.79399999999961	35.7756024450959\\
8.79599999999961	35.7684487553452\\
8.79799999999961	35.7612964960463\\
8.79999999999961	35.754145666913\\
8.80199999999961	35.7469962676599\\
8.80399999999961	35.7398482980006\\
8.80599999999961	35.7327017576495\\
8.80799999999961	35.7255566463205\\
8.80999999999961	35.718412963728\\
8.81199999999961	35.7112707095861\\
8.81399999999961	35.7041298836098\\
8.81599999999961	35.6969904855134\\
8.81799999999961	35.6898525150105\\
8.81999999999961	35.6827159718165\\
8.82199999999962	35.6755808556454\\
8.82399999999962	35.6684471662125\\
8.82599999999962	35.6613149032322\\
8.82799999999962	35.654184066419\\
8.82999999999962	35.6470546554883\\
8.83199999999962	35.6399266701546\\
8.83399999999962	35.6328001101329\\
8.83599999999962	35.6256749751382\\
8.83799999999962	35.6185512648856\\
8.83999999999962	35.61142897909\\
8.84199999999962	35.6043081174668\\
8.84399999999962	35.5971886797312\\
8.84599999999962	35.5900706655982\\
8.84799999999962	35.5829540747836\\
8.84999999999962	35.5758389070028\\
8.85199999999963	35.5687251619701\\
8.85399999999963	35.561612839403\\
8.85599999999963	35.5545019390152\\
8.85799999999963	35.5473924605235\\
8.85999999999963	35.5402844036427\\
8.86199999999963	35.5331777680897\\
8.86399999999963	35.5260725535793\\
8.86599999999963	35.5189687598276\\
8.86799999999963	35.5118663865506\\
8.86999999999963	35.5047654334642\\
8.87199999999963	35.4976659002843\\
8.87399999999963	35.490567786727\\
8.87599999999963	35.483471092509\\
8.87799999999963	35.4763758173459\\
8.87999999999963	35.4692819609538\\
8.88199999999964	35.4621895230494\\
8.88399999999964	35.4550985033489\\
8.88599999999964	35.4480089015689\\
8.88799999999964	35.440920717426\\
8.88999999999964	35.4338339506356\\
8.89199999999964	35.4267486009158\\
8.89399999999964	35.4196646679825\\
8.89599999999964	35.4125821515526\\
8.89799999999964	35.4055010513425\\
8.89999999999964	35.3984213670695\\
8.90199999999964	35.3913430984501\\
8.90399999999964	35.3842662452012\\
8.90599999999964	35.3771908070403\\
8.90799999999964	35.3701167836836\\
8.90999999999964	35.3630441748489\\
8.91199999999965	35.3559729802532\\
8.91399999999965	35.3489031996134\\
8.91599999999965	35.3418348326472\\
8.91799999999965	35.3347678790717\\
8.91999999999965	35.327702338604\\
8.92199999999965	35.3206382109623\\
8.92399999999965	35.3135754958633\\
8.92599999999965	35.3065141930249\\
8.92799999999965	35.2994543021646\\
8.92999999999965	35.2923958230003\\
8.93199999999965	35.2853387552492\\
8.93399999999965	35.2782830986299\\
8.93599999999965	35.2712288528596\\
8.93799999999965	35.2641760176565\\
8.93999999999965	35.2571245927378\\
8.94199999999966	35.2500745778227\\
8.94399999999966	35.2430259726282\\
8.94599999999966	35.2359787768732\\
8.94799999999966	35.2289329902752\\
8.94999999999966	35.221888612553\\
8.95199999999966	35.2148456434246\\
8.95399999999966	35.2078040826083\\
8.95599999999966	35.200763929823\\
8.95799999999966	35.1937251847859\\
8.95999999999966	35.1866878472168\\
8.96199999999966	35.1796519168336\\
8.96399999999966	35.1726173933551\\
8.96599999999966	35.1655842765002\\
8.96799999999966	35.1585525659873\\
8.96999999999966	35.151522261535\\
8.97199999999967	35.144493362863\\
8.97399999999967	35.1374658696888\\
8.97599999999967	35.1304397817329\\
8.97799999999967	35.1234150987133\\
8.97999999999967	35.1163918203495\\
8.98199999999967	35.1093699463604\\
8.98399999999967	35.1023494764652\\
8.98599999999967	35.0953304103831\\
8.98799999999967	35.088312747834\\
8.98999999999967	35.0812964885368\\
8.99199999999967	35.0742816322105\\
8.99399999999967	35.0672681785747\\
8.99599999999967	35.0602561273497\\
8.99799999999967	35.053245478254\\
8.99999999999967	35.0462362310083\\
9.00199999999968	35.0392283853316\\
9.00399999999968	35.0322219409435\\
9.00599999999968	35.025216897564\\
9.00799999999968	35.0182132549134\\
9.00999999999968	35.0112110127108\\
9.01199999999968	35.0042101706771\\
9.01399999999968	34.9972107285313\\
9.01599999999968	34.9902126859948\\
9.01799999999968	34.9832160427859\\
9.01999999999968	34.9762207986269\\
9.02199999999968	34.9692269532362\\
9.02399999999968	34.9622345063353\\
9.02599999999968	34.9552434576441\\
9.02799999999968	34.9482538068828\\
9.02999999999968	34.9412655537723\\
9.03199999999969	34.9342786980328\\
9.03399999999969	34.9272932393851\\
9.03599999999969	34.9203091775497\\
9.03799999999969	34.9133265122475\\
9.03999999999969	34.9063452431988\\
9.04199999999969	34.8993653701251\\
9.04399999999969	34.8923868927469\\
9.04599999999969	34.8854098107848\\
9.04799999999969	34.8784341239603\\
9.04999999999969	34.8714598319939\\
9.05199999999969	34.864486934607\\
9.05399999999969	34.8575154315212\\
9.05599999999969	34.8505453224568\\
9.05799999999969	34.8435766071357\\
9.05999999999969	34.8366092852789\\
9.0619999999997	34.8296433566078\\
9.0639999999997	34.8226788208439\\
9.0659999999997	34.8157156777085\\
9.0679999999997	34.8087539269235\\
9.0699999999997	34.8017935682102\\
9.0719999999997	34.7948346012895\\
9.0739999999997	34.7878770258848\\
9.0759999999997	34.7809208417168\\
9.0779999999997	34.773966048507\\
9.0799999999997	34.7670126459781\\
9.0819999999997	34.7600606338513\\
9.0839999999997	34.7531100118492\\
9.0859999999997	34.7461607796933\\
9.0879999999997	34.7392129371062\\
9.0899999999997	34.7322664838096\\
9.09199999999971	34.725321419526\\
9.09399999999971	34.7183777439773\\
9.09599999999971	34.7114354568862\\
9.09799999999971	34.7044945579746\\
9.09999999999971	34.6975550469658\\
9.10199999999971	34.6906169235808\\
9.10399999999971	34.6836801875437\\
9.10599999999971	34.6767448385761\\
9.10799999999971	34.6698108764011\\
9.10999999999971	34.6628783007409\\
9.11199999999971	34.655947111319\\
9.11399999999971	34.6490173078573\\
9.11599999999971	34.6420888900798\\
9.11799999999971	34.6351618577083\\
9.11999999999972	34.6282362104666\\
9.12199999999972	34.621311948077\\
9.12399999999972	34.6143890702632\\
9.12599999999972	34.607467576748\\
9.12799999999972	34.6005474672548\\
9.12999999999972	34.5936287415065\\
9.13199999999972	34.5867113992269\\
9.13399999999972	34.5797954401389\\
9.13599999999972	34.5728808639662\\
9.13799999999972	34.5659676704325\\
9.13999999999972	34.5590558592607\\
9.14199999999972	34.5521454301749\\
9.14399999999972	34.5452363828984\\
9.14599999999972	34.5383287171552\\
9.14799999999972	34.5314224326689\\
9.14999999999973	34.5245175291628\\
9.15199999999973	34.5176140063619\\
9.15399999999973	34.5107118639894\\
9.15599999999973	34.5038111017691\\
9.15799999999973	34.4969117194255\\
9.15999999999973	34.4900137166821\\
9.16199999999973	34.4831170932639\\
9.16399999999973	34.4762218488937\\
9.16599999999973	34.4693279832974\\
9.16799999999973	34.4624354961984\\
9.16999999999973	34.4555443873211\\
9.17199999999973	34.4486546563899\\
9.17399999999973	34.4417663031299\\
9.17599999999973	34.4348793272643\\
9.17799999999973	34.4279937285185\\
9.17999999999974	34.4211095066176\\
9.18199999999974	34.4142266612854\\
9.18399999999974	34.4073451922472\\
9.18599999999974	34.4004650992278\\
9.18799999999974	34.3935863819512\\
9.18999999999974	34.3867090401433\\
9.19199999999974	34.3798330735289\\
9.19399999999974	34.3729584818324\\
9.19599999999974	34.3660852647797\\
9.19799999999974	34.3592134220956\\
9.19999999999974	34.3523429535048\\
9.20199999999974	34.3454738587333\\
9.20399999999974	34.338606137506\\
9.20599999999974	34.3317397895482\\
9.20799999999974	34.3248748145852\\
9.20999999999975	34.3180112123431\\
9.21199999999975	34.3111489825468\\
9.21399999999975	34.3042881249219\\
9.21599999999975	34.297428639194\\
9.21799999999975	34.2905705250895\\
9.21999999999975	34.283713782333\\
9.22199999999975	34.2768584106506\\
9.22399999999975	34.2700044097692\\
9.22599999999975	34.2631517794132\\
9.22799999999975	34.2563005193096\\
9.22999999999975	34.2494506291839\\
9.23199999999975	34.2426021087624\\
9.23399999999975	34.2357549577709\\
9.23599999999975	34.2289091759362\\
9.23799999999975	34.2220647629837\\
9.23999999999976	34.2152217186399\\
9.24199999999976	34.2083800426317\\
9.24399999999976	34.2015397346847\\
9.24599999999976	34.194700794526\\
9.24799999999976	34.1878632218816\\
9.24999999999976	34.1810270164784\\
9.25199999999976	34.1741921780429\\
9.25399999999976	34.167358706302\\
9.25599999999976	34.1605266009818\\
9.25799999999976	34.1536958618093\\
9.25999999999976	34.1468664885121\\
9.26199999999976	34.1400384808161\\
9.26399999999976	34.1332118384482\\
9.26599999999976	34.1263865611363\\
9.26799999999976	34.1195626486067\\
9.26999999999977	34.112740100587\\
9.27199999999977	34.1059189168036\\
9.27399999999977	34.0990990969842\\
9.27599999999977	34.0922806408562\\
9.27799999999977	34.0854635481466\\
9.27999999999977	34.0786478185832\\
9.28199999999977	34.071833451893\\
9.28399999999977	34.0650204478035\\
9.28599999999977	34.0582088060423\\
9.28799999999977	34.0513985263374\\
9.28999999999977	34.0445896084156\\
9.29199999999977	34.0377820520051\\
9.29399999999977	34.0309758568342\\
9.29599999999977	34.0241710226299\\
9.29799999999977	34.0173675491199\\
9.29999999999978	34.010565436033\\
9.30199999999978	34.0037646830965\\
9.30399999999978	33.9969652900385\\
9.30599999999978	33.9901672565873\\
9.30799999999978	33.9833705824709\\
9.30999999999978	33.9765752674178\\
9.31199999999978	33.9697813111555\\
9.31399999999978	33.9629887134129\\
9.31599999999978	33.9561974739187\\
9.31799999999978	33.9494075924001\\
9.31999999999978	33.9426190685862\\
9.32199999999978	33.9358319022057\\
9.32399999999978	33.9290460929877\\
9.32599999999978	33.9222616406595\\
9.32799999999978	33.9154785449508\\
9.32999999999979	33.9086968055897\\
9.33199999999979	33.9019164223051\\
9.33399999999979	33.8951373948264\\
9.33599999999979	33.8883597228819\\
9.33799999999979	33.8815834062011\\
9.33999999999979	33.8748084445122\\
9.34199999999979	33.8680348375444\\
9.34399999999979	33.8612625850277\\
9.34599999999979	33.8544916866904\\
9.34799999999979	33.8477221422624\\
9.34999999999979	33.8409539514721\\
9.35199999999979	33.8341871140492\\
9.35399999999979	33.8274216297234\\
9.35599999999979	33.8206574982241\\
9.35799999999979	33.8138947192803\\
9.3599999999998	33.807133292622\\
9.3619999999998	33.8003732179782\\
9.3639999999998	33.7936144950794\\
9.3659999999998	33.7868571236546\\
9.3679999999998	33.7801011034344\\
9.3699999999998	33.7733464341475\\
9.3719999999998	33.7665931155245\\
9.3739999999998	33.7598411472952\\
9.3759999999998	33.7530905291893\\
9.3779999999998	33.7463412609374\\
9.3799999999998	33.7395933422689\\
9.3819999999998	33.7328467729146\\
9.3839999999998	33.7261015526044\\
9.3859999999998	33.7193576810679\\
9.3879999999998	33.7126151580368\\
9.38999999999981	33.7058739832398\\
9.39199999999981	33.6991341564089\\
9.39399999999981	33.6923956772734\\
9.39599999999981	33.6856585455644\\
9.39799999999981	33.6789227610124\\
9.39999999999981	33.6721883233478\\
9.40199999999981	33.6654552323016\\
9.40399999999981	33.6587234876042\\
9.40599999999981	33.6519930889864\\
9.40799999999981	33.6452640361792\\
9.40999999999981	33.6385363289137\\
9.41199999999981	33.6318099669203\\
9.41399999999981	33.62508494993\\
9.41599999999981	33.6183612776749\\
9.41799999999981	33.6116389498851\\
9.41999999999982	33.6049179662919\\
9.42199999999982	33.5981983266267\\
9.42399999999982	33.5914800306205\\
9.42599999999982	33.5847630780049\\
9.42799999999982	33.5780474685115\\
9.42999999999982	33.5713332018712\\
9.43199999999982	33.5646202778158\\
9.43399999999982	33.5579086960764\\
9.43599999999982	33.5511984563856\\
9.43799999999982	33.5444895584737\\
9.43999999999982	33.5377820020735\\
9.44199999999982	33.5310757869164\\
9.44399999999982	33.5243709127337\\
9.44599999999982	33.5176673792578\\
9.44799999999982	33.5109651862209\\
9.44999999999983	33.5042643333542\\
9.45199999999983	33.4975648203904\\
9.45399999999983	33.4908666470612\\
9.45599999999983	33.4841698130985\\
9.45799999999983	33.4774743182349\\
9.45999999999983	33.4707801622029\\
9.46199999999983	33.4640873447337\\
9.46399999999983	33.4573958655606\\
9.46599999999983	33.4507057244157\\
9.46799999999983	33.4440169210316\\
9.46999999999983	33.4373294551406\\
9.47199999999983	33.4306433264756\\
9.47399999999983	33.4239585347689\\
9.47599999999983	33.417275079753\\
9.47799999999983	33.4105929611607\\
9.47999999999984	33.4039121787252\\
9.48199999999984	33.3972327321787\\
9.48399999999984	33.3905546212547\\
9.48599999999984	33.3838778456855\\
9.48799999999984	33.3772024052045\\
9.48999999999984	33.370528299545\\
9.49199999999984	33.3638555284394\\
9.49399999999984	33.3571840916209\\
9.49599999999984	33.3505139888233\\
9.49799999999984	33.3438452197795\\
9.49999999999984	33.3371777842226\\
9.50199999999984	33.3305116818861\\
9.50399999999984	33.3238469125038\\
9.50599999999984	33.3171834758088\\
9.50799999999984	33.3105213715346\\
9.50999999999985	33.3038605994148\\
9.51199999999985	33.2972011591831\\
9.51399999999985	33.2905430505729\\
9.51599999999985	33.2838862733185\\
9.51799999999985	33.2772308271532\\
9.51999999999985	33.270576711811\\
9.52199999999985	33.2639239270255\\
9.52399999999985	33.257272472531\\
9.52599999999985	33.2506223480616\\
9.52799999999985	33.2439735533509\\
9.52999999999985	33.2373260881332\\
9.53199999999985	33.230679952143\\
9.53399999999985	33.2240351451142\\
9.53599999999985	33.2173916667812\\
9.53799999999985	33.2107495168775\\
9.53999999999986	33.2041086951387\\
9.54199999999986	33.1974692012982\\
9.54399999999986	33.1908310350916\\
9.54599999999986	33.1841941962525\\
9.54799999999986	33.1775586845154\\
9.54999999999986	33.1709244996157\\
9.55199999999986	33.1642916412876\\
9.55399999999986	33.1576601092656\\
9.55599999999986	33.1510299032851\\
9.55799999999986	33.1444010230803\\
9.55999999999986	33.1377734683869\\
9.56199999999986	33.1311472389392\\
9.56399999999986	33.1245223344726\\
9.56599999999986	33.1178987547216\\
9.56799999999986	33.1112764994217\\
9.56999999999987	33.1046555683081\\
9.57199999999987	33.0980359611159\\
9.57399999999987	33.0914176775806\\
9.57599999999987	33.0848007174371\\
9.57799999999987	33.0781850804213\\
9.57999999999987	33.0715707662679\\
9.58199999999987	33.064957774713\\
9.58399999999987	33.058346105492\\
9.58599999999987	33.0517357583403\\
9.58799999999987	33.0451267329942\\
9.58999999999987	33.0385190291883\\
9.59199999999987	33.031912646659\\
9.59399999999987	33.0253075851418\\
9.59599999999987	33.0187038443731\\
9.59799999999987	33.0121014240887\\
9.59999999999988	33.0055003240237\\
9.60199999999988	32.9989005439149\\
9.60399999999988	32.9923020834982\\
9.60599999999988	32.9857049425097\\
9.60799999999988	32.9791091206859\\
9.60999999999988	32.9725146177624\\
9.61199999999988	32.9659214334761\\
9.61399999999988	32.9593295675624\\
9.61599999999988	32.9527390197584\\
9.61799999999988	32.9461497898006\\
9.61999999999988	32.939561877425\\
9.62199999999988	32.9329752823688\\
9.62399999999988	32.9263900043679\\
9.62599999999988	32.9198060431593\\
9.62799999999988	32.9132233984798\\
9.62999999999989	32.9066420700658\\
9.63199999999989	32.9000620576544\\
9.63399999999989	32.8934833609822\\
9.63599999999989	32.8869059797863\\
9.63799999999989	32.8803299138035\\
9.63999999999989	32.8737551627711\\
9.64199999999989	32.867181726426\\
9.64399999999989	32.860609604505\\
9.64599999999989	32.8540387967457\\
9.64799999999989	32.8474693028852\\
9.64999999999989	32.8409011226609\\
9.65199999999989	32.8343342558099\\
9.65399999999989	32.8277687020695\\
9.65599999999989	32.8212044611772\\
9.65799999999989	32.8146415328707\\
9.6599999999999	32.8080799168874\\
9.6619999999999	32.8015196129647\\
9.6639999999999	32.7949606208409\\
9.6659999999999	32.7884029402528\\
9.6679999999999	32.7818465709386\\
9.6699999999999	32.7752915126367\\
9.6719999999999	32.7687377650832\\
9.6739999999999	32.7621853280177\\
9.6759999999999	32.7556342011774\\
9.6779999999999	32.7490843843007\\
9.6799999999999	32.7425358771255\\
9.6819999999999	32.7359886793896\\
9.6839999999999	32.7294427908312\\
9.6859999999999	32.7228982111894\\
9.6879999999999	32.7163549402012\\
9.68999999999991	32.7098129776056\\
9.69199999999991	32.7032723231413\\
9.69399999999991	32.696732976546\\
9.69599999999991	32.6901949375587\\
9.69799999999991	32.6836582059174\\
9.69999999999991	32.6771227813617\\
9.70199999999991	32.6705886636285\\
9.70399999999991	32.6640558524581\\
9.70599999999991	32.6575243475887\\
9.70799999999991	32.6509941487589\\
9.70999999999991	32.6444652557078\\
9.71199999999991	32.6379376681743\\
9.71399999999991	32.6314113858971\\
9.71599999999991	32.6248864086153\\
9.71799999999991	32.6183627360682\\
9.71999999999992	32.6118403679948\\
9.72199999999992	32.6053193041343\\
9.72399999999992	32.5987995442252\\
9.72599999999992	32.5922810880077\\
9.72799999999992	32.5857639352207\\
9.72999999999992	32.5792480856036\\
9.73199999999992	32.572733538896\\
9.73399999999992	32.5662202948372\\
9.73599999999992	32.5597083531665\\
9.73799999999992	32.5531977136237\\
9.73999999999992	32.5466883759485\\
9.74199999999992	32.5401803398808\\
9.74399999999992	32.5336736051598\\
9.74599999999992	32.5271681715256\\
9.74799999999992	32.5206640387178\\
9.74999999999993	32.5141612064765\\
9.75199999999993	32.5076596745417\\
9.75399999999993	32.5011594426532\\
9.75599999999993	32.4946605105514\\
9.75799999999993	32.4881628779755\\
9.75999999999993	32.4816665446667\\
9.76199999999993	32.4751715103649\\
9.76399999999993	32.4686777748095\\
9.76599999999993	32.4621853377424\\
9.76799999999993	32.4556941989026\\
9.76999999999993	32.4492043580311\\
9.77199999999993	32.4427158148684\\
9.77399999999993	32.4362285691543\\
9.77599999999993	32.4297426206302\\
9.77799999999993	32.4232579690365\\
9.77999999999994	32.4167746141138\\
9.78199999999994	32.4102925556028\\
9.78399999999994	32.403811793244\\
9.78599999999994	32.3973323267788\\
9.78799999999994	32.3908541559478\\
9.78999999999994	32.3843772804915\\
9.79199999999994	32.3779017001516\\
9.79399999999994	32.371427414669\\
9.79599999999994	32.3649544237843\\
9.79799999999994	32.3584827272383\\
9.79999999999994	32.3520123247738\\
9.80199999999994	32.3455432161302\\
9.80399999999994	32.3390754010507\\
9.80599999999994	32.3326088792748\\
9.80799999999994	32.3261436505447\\
9.80999999999995	32.3196797146019\\
9.81199999999995	32.3132170711874\\
9.81399999999995	32.3067557200432\\
9.81599999999995	32.3002956609113\\
9.81799999999995	32.2938368935326\\
9.81999999999995	32.2873794176496\\
9.82199999999995	32.2809232330027\\
9.82399999999995	32.2744683393351\\
9.82599999999995	32.2680147363875\\
9.82799999999995	32.2615624239033\\
9.82999999999995	32.2551114016224\\
9.83199999999995	32.248661669289\\
9.83399999999995	32.2422132266435\\
9.83599999999995	32.2357660734292\\
9.83799999999995	32.2293202093868\\
9.83999999999996	32.2228756342602\\
9.84199999999996	32.2164323477904\\
9.84399999999996	32.2099903497207\\
9.84599999999996	32.2035496397929\\
9.84799999999996	32.1971102177495\\
9.84999999999996	32.190672083333\\
9.85199999999996	32.1842352362857\\
9.85399999999996	32.1777996763508\\
9.85599999999996	32.17136540327\\
9.85799999999996	32.1649324167868\\
9.85999999999996	32.1585007166432\\
9.86199999999996	32.1520703025827\\
9.86399999999996	32.1456411743477\\
9.86599999999996	32.1392133316816\\
9.86799999999996	32.1327867743269\\
9.86999999999997	32.1263615020264\\
9.87199999999997	32.1199375145234\\
9.87399999999997	32.1135148115611\\
9.87599999999997	32.107093392883\\
9.87799999999997	32.1006732582313\\
9.87999999999997	32.09425440735\\
9.88199999999997	32.087836839982\\
9.88399999999997	32.0814205558708\\
9.88599999999997	32.0750055547598\\
9.88799999999997	32.0685918363921\\
9.88999999999997	32.0621794005122\\
9.89199999999997	32.0557682468632\\
9.89399999999997	32.0493583751881\\
9.89599999999997	32.0429497852315\\
9.89799999999997	32.0365424767359\\
9.89999999999998	32.0301364494461\\
9.90199999999998	32.0237317031053\\
9.90399999999998	32.0173282374576\\
9.90599999999998	32.0109260522473\\
9.90799999999998	32.0045251472183\\
9.90999999999998	31.9981255221137\\
9.91199999999998	31.9917271766785\\
9.91399999999998	31.9853301106565\\
9.91599999999998	31.9789343237917\\
9.91799999999998	31.9725398158284\\
9.91999999999998	31.9661465865114\\
9.92199999999998	31.9597546355841\\
9.92399999999998	31.9533639627917\\
9.92599999999998	31.946974567878\\
9.92799999999998	31.940586450588\\
9.92999999999999	31.9341996106661\\
9.93199999999999	31.9278140478563\\
9.93399999999999	31.9214297619043\\
9.93599999999999	31.9150467525535\\
9.93799999999999	31.9086650195496\\
9.93999999999999	31.9022845626371\\
9.94199999999999	31.8959053815613\\
9.94399999999999	31.8895274760657\\
9.94599999999999	31.8831508458967\\
9.94799999999999	31.8767754907984\\
9.94999999999999	31.8704014105165\\
9.95199999999999	31.8640286047956\\
9.95399999999999	31.857657073381\\
9.95599999999999	31.8512868160177\\
9.95799999999999	31.8449178324517\\
9.96	31.8385501224272\\
9.962	31.8321836856899\\
9.964	31.8258185219853\\
9.966	31.819454631059\\
9.968	31.8130920126568\\
9.97	31.8067306665239\\
9.972	31.8003705924052\\
9.974	31.7940117900473\\
9.976	31.7876542591954\\
9.978	31.7812979995953\\
9.98	31.7749430109932\\
9.982	31.7685892931351\\
9.984	31.7622368457655\\
9.986	31.7558856686318\\
9.988	31.7495357614796\\
9.99000000000001	31.7431871240548\\
9.99200000000001	31.7368397561035\\
9.99400000000001	31.730493657372\\
9.99600000000001	31.7241488276069\\
9.99800000000001	31.7178052665537\\
10	31.7114629739591\\
};
\end{axis}

\begin{axis}[%
width=2.603in,
height=1.074in,
at={(1.011in,0.751in)},
scale only axis,
xmin=0,
xmax=10,
xlabel style={font=\color{white!15!black}},
xlabel={t},
ymode=log,
ymin=37.2101583482885,
ymax=112966.215940441,
yminorticks=true,
ylabel style={font=\color{white!15!black}},
ylabel={indice stiff},
axis background/.style={fill=white},
title style={font=\bfseries},
title={N=50}
]
\addplot [color=mycolor1, forget plot]
  table[row sep=crcr]{%
0	54160.0226450488\\
0.002	112966.215940441\\
0.004	56639.3532913087\\
0.006	37859.8576177498\\
0.008	28467.1715070576\\
0.01	22829.1860640573\\
0.012	19068.5337456195\\
0.014	16380.6301947432\\
0.016	14363.1846999118\\
0.018	12792.7036822847\\
0.02	11535.0920913166\\
0.022	10505.0176761871\\
0.024	9645.59312221264\\
0.026	8917.43566337766\\
0.028	8292.4154329751\\
0.03	7749.90447453636\\
0.032	7274.43237420681\\
0.034	6854.1693174812\\
0.036	6479.91479414271\\
0.038	6144.4056599951\\
0.04	5841.83178044617\\
0.042	5567.49006203378\\
0.044	5317.5328393303\\
0.046	5088.78190037247\\
0.048	4878.58900603487\\
0.05	4684.72988503707\\
0.052	4505.32269190234\\
0.054	4338.76458561849\\
0.056	4183.68189880969\\
0.058	4038.89061695586\\
0.06	3903.36476196284\\
0.062	3776.21089523246\\
0.064	3656.64740158313\\
0.066	3543.98753989669\\
0.068	3437.62548498131\\
0.07	3337.02476240751\\
0.0720000000000001	3241.70861100972\\
0.0740000000000001	3151.25190835477\\
0.0760000000000001	3065.27437126257\\
0.0780000000000001	2983.43480251831\\
0.0800000000000001	2905.42620068714\\
0.0820000000000001	2830.97158567514\\
0.0840000000000001	2759.82042073853\\
0.0860000000000001	2691.74553384345\\
0.0880000000000001	2626.54045893409\\
0.0900000000000001	2564.01713178417\\
0.0920000000000001	2504.00388647843\\
0.0940000000000001	2446.34370773904\\
0.0960000000000001	2390.89270179156\\
0.0980000000000001	2337.51875454702\\
0.1	2286.10035087238\\
0.102	2236.52553284358\\
0.104	2188.69097826182\\
0.106	2142.50118355978\\
0.108	2097.86773755365\\
0.11	2054.70867448615\\
0.112	2012.9478964436\\
0.114	1972.51465662792\\
0.116	1933.34309613943\\
0.118	1895.37182791886\\
0.12	1858.54356234847\\
0.122	1822.80476972791\\
0.124	1788.10537546395\\
0.126	1754.39848433679\\
0.128	1721.64013066289\\
0.13	1689.78905156391\\
0.132	1658.80648089252\\
0.134	1628.65596165352\\
0.136	1599.30317501791\\
0.138	1570.71578424631\\
0.14	1542.86329202738\\
0.142	1515.71690991201\\
0.144	1489.24943866382\\
0.146	1463.43515847977\\
0.148	1438.24972814662\\
0.15	1413.67009229946\\
0.152	1389.67439603336\\
0.154	1366.24190620181\\
0.156	1343.35293880014\\
0.158	1320.98879189312\\
0.16	1299.1316836033\\
0.162	1277.76469472104\\
0.164	1256.87171554071\\
0.166	1236.43739656761\\
0.168	1216.447102771\\
0.17	1196.8868710916\\
0.172	1177.74337093873\\
0.174	1159.00386743381\\
0.176	1140.65618718431\\
0.178	1122.6886863864\\
0.18	1105.09022107541\\
0.182	1087.85011935866\\
0.184	1070.95815547921\\
0.186	1054.40452557166\\
0.188	1038.17982498517\\
0.19	1022.27502705535\\
0.192	1006.68146322131\\
0.194	991.390804388791\\
0.196	976.395043450761\\
0.198	961.686478882506\\
0.2	947.257699336485\\
0.202	933.101569166484\\
0.204	919.211214817101\\
0.206	905.580012019564\\
0.208	892.201573739072\\
0.21	879.069738822981\\
0.212	866.17856130366\\
0.214	853.522300312167\\
0.216	841.095410563546\\
0.218	828.892533375844\\
0.22	816.908488189491\\
0.222	805.138264554086\\
0.224	793.577014553955\\
0.226	782.22004564462\\
0.228	771.062813874307\\
0.23	760.100917467519\\
0.232	749.330090747535\\
0.234	738.746198378318\\
0.236	728.345229905766\\
0.238	718.123294580777\\
0.24	708.07661644742\\
0.242	698.20152968038\\
0.244	688.494474157701\\
0.246	678.951991254297\\
0.248	669.570719844301\\
0.25	660.347392499988\\
0.252	651.278831876067\\
0.254	642.36194726908\\
0.256	633.59373134201\\
0.258	624.971257004965\\
0.26	616.491674443339\\
0.262	608.152208285207\\
0.264	599.950154900737\\
0.266	591.882879826097\\
0.268	583.947815305528\\
0.27	576.142457944909\\
0.272	568.464366471312\\
0.274	560.911159592643\\
0.276	553.480513952345\\
0.278	546.170162174065\\
0.28	538.977890991906\\
0.282	531.901539461495\\
0.284	524.938997248108\\
0.286	518.088202987751\\
0.288	511.347142717638\\
0.29	504.713848372409\\
0.292	498.186396343142\\
0.294	491.762906095789\\
0.296	485.441538846158\\
0.298	479.22049628882\\
0.3	473.098019377248\\
0.302	467.072387152687\\
0.304	461.141915619417\\
0.306	455.304956664402\\
0.308	449.559897018871\\
0.31	443.905157260184\\
0.312	438.339190852026\\
0.314	432.860483220895\\
0.316	427.467550867707\\
0.318	422.158940512356\\
0.32	416.933228270191\\
0.322	411.789018858681\\
0.324	406.724944832989\\
0.326	401.73966584922\\
0.328	396.831867954035\\
0.33	392.000262899475\\
0.332	387.243587482015\\
0.334	382.560602904539\\
0.336	377.950094160525\\
0.338	373.410869439264\\
0.34	368.941759551361\\
0.342	364.541617373548\\
0.344	360.209317312041\\
0.346	355.94375478374\\
0.348	351.743845714341\\
0.35	347.608526052826\\
0.352	343.536751301624\\
0.354	339.527496061717\\
0.356	335.579753592179\\
0.358	331.692535383559\\
0.36	327.864870744499\\
0.362	324.095806400988\\
0.364	320.384406108103\\
0.366	316.72975027321\\
0.368	313.130935590674\\
0.37	309.587074687312\\
0.372	306.097295778337\\
0.374	302.660742333284\\
0.376	299.276572751685\\
0.378	295.943960047955\\
0.38	292.662091545267\\
0.382	289.430168578045\\
0.384	286.247406202774\\
0.386	283.113032916729\\
0.388	280.026290384462\\
0.39	276.986433171805\\
0.392	273.99272848684\\
0.394	271.044455927961\\
0.396	268.140907238468\\
0.398	265.281386067728\\
0.4	262.465207738359\\
0.402	259.691699019659\\
0.404	256.960197906566\\
0.406	254.27005340441\\
0.408	251.620625318959\\
0.41	249.01128405177\\
0.412	246.441410400498\\
0.414	243.910395364238\\
0.416	241.41763995343\\
0.418	238.962555004472\\
0.42	236.544560998684\\
0.422	234.163087885632\\
0.424	231.817574910524\\
0.426	229.50747044572\\
0.428	227.232231826095\\
0.43	224.991325188154\\
0.432	222.784225312888\\
0.434	220.610415472168\\
0.436	218.46938727849\\
0.438	216.360640538272\\
0.44	214.283683108148\\
0.442	212.238030754644\\
0.444	210.223207016793\\
0.446	208.238743071747\\
0.448	206.284177603298\\
0.45	204.359056673228\\
0.452	202.462933595322\\
0.454	200.595368812135\\
0.456	198.755929774247\\
0.458	196.944190822159\\
0.46	195.159733070448\\
0.462	193.402144294538\\
0.464	191.671018819572\\
0.466	189.965957411723\\
0.468	188.286567171558\\
0.47	186.632461429672\\
0.472	185.003259644326\\
0.474	183.398587301167\\
0.476	181.818075814934\\
0.478	180.261362433078\\
0.48	178.728090141304\\
0.482	177.217907570932\\
0.484	175.730468908101\\
0.486	174.265433804681\\
0.488	172.822467290978\\
0.49	171.401239689996\\
0.492	170.001426533501\\
0.494	168.622708479535\\
0.496	167.26477123162\\
0.498	165.92730545943\\
0.5	164.61000672095\\
0.502	163.312575386189\\
0.504	162.034716562236\\
0.506	160.776140019737\\
0.508	159.536560120849\\
0.51	158.31569574839\\
0.512	157.113270236457\\
0.514	155.929011302202\\
0.516	154.762650979024\\
0.518	153.613925550884\\
0.52	152.482575487923\\
0.522	151.368345383216\\
0.524	150.270983890792\\
0.526	149.190243664714\\
0.528	148.125881299357\\
0.53	147.077657270789\\
0.532	146.045335879249\\
0.534	145.028685192699\\
0.536	144.027476991439\\
0.538	143.041486713804\\
0.54	142.070493402829\\
0.542	141.11427965398\\
0.544	140.172631563847\\
0.546	139.245338679909\\
0.548	138.332193951113\\
0.55	137.432993679597\\
0.552	136.547537473217\\
0.554	135.675628199138\\
0.556	134.817071938214\\
0.558	133.971677940422\\
0.56	133.139258581139\\
0.562	132.319629318313\\
0.564	131.512608650551\\
0.566	130.718018076116\\
0.568	129.935682052771\\
0.57	129.165427958527\\
0.572	128.407086053243\\
0.574	127.660489441141\\
0.576	126.92547403414\\
0.578	126.201878516115\\
0.58	125.489544307977\\
0.582	124.788315533663\\
0.584	124.098038987\\
0.586	123.418564099413\\
0.588	122.749742908551\\
0.59	122.091430027822\\
0.592	121.443482616767\\
0.594	120.805760352418\\
0.596	120.17812540153\\
0.598	119.56044239376\\
0.6	118.952578395818\\
0.602	118.354402886582\\
0.604	117.765787733183\\
0.606	117.186607168136\\
0.608	116.616737767513\\
0.61	116.056058430176\\
0.612	115.504450358096\\
0.614	114.96179703785\\
0.616	114.427984223321\\
0.618	113.902899919527\\
0.62	113.386434367823\\
0.622	112.878480032426\\
0.624	112.378931588238\\
0.626	111.887685910256\\
0.628	111.404642064391\\
0.63	110.929701299997\\
0.632	110.462767043995\\
0.634	110.003744896854\\
0.636	109.552542630425\\
0.638	109.109070187781\\
0.64	108.673239685162\\
0.642	108.244965416189\\
0.644	107.82416385843\\
0.646	107.410753682525\\
0.648	107.004655763969\\
0.65	106.605793197774\\
0.652	106.214091316134\\
0.654	105.829477709347\\
0.656	105.451882250033\\
0.658	105.081237121065\\
0.66	104.717476847137\\
0.662	104.360538330362\\
0.664	104.010360889921\\
0.666	103.66688630598\\
0.668	103.330058867898\\
0.67	102.99982542684\\
0.672	102.676135452676\\
0.674	102.358941095095\\
0.676	102.048197248578\\
0.678	101.743861620834\\
0.68	101.445894803869\\
0.682	101.154260346843\\
0.684	100.868924829108\\
0.686	100.589857931673\\
0.688000000000001	100.317032504572\\
0.690000000000001	100.050424626812\\
0.692000000000001	99.7900136547935\\
0.694000000000001	99.535782253964\\
0.696000000000001	99.2877164071014\\
0.698000000000001	99.0458053911988\\
0.700000000000001	98.8100417131748\\
0.702000000000001	98.5804209927103\\
0.704000000000001	98.3569417784741\\
0.706000000000001	98.1396052819201\\
0.708000000000001	97.9284150108692\\
0.710000000000001	97.723376283402\\
0.712000000000001	97.5244956017161\\
0.714000000000001	97.331779865611\\
0.716000000000001	97.1452354069781\\
0.718000000000001	96.9648668305531\\
0.720000000000001	96.7906756527878\\
0.722000000000001	96.6226587405819\\
0.724000000000001	96.46080656515\\
0.726000000000001	96.3051013031458\\
0.728000000000001	96.1555148371572\\
0.730000000000001	96.0120067286207\\
0.732000000000001	95.8745222566936\\
0.734000000000001	95.7429906330636\\
0.736000000000001	95.6173235121808\\
0.738000000000001	95.4974139154504\\
0.740000000000001	95.3831356747666\\
0.742000000000001	95.2743434744577\\
0.744000000000001	95.1708735328758\\
0.746000000000001	95.0725449196918\\
0.748000000000001	94.9791614572466\\
0.750000000000001	94.8905141110794\\
0.752000000000001	94.8063837410485\\
0.754000000000001	94.7265440645756\\
0.756000000000001	94.650764679349\\
0.758000000000001	94.5788140027227\\
0.760000000000001	94.510462006866\\
0.762000000000001	94.4454826576418\\
0.764000000000001	94.3836559971251\\
0.766000000000001	94.3247698404096\\
0.768000000000001	94.2686210843528\\
0.770000000000001	94.2150166468801\\
0.772000000000001	94.1637740707112\\
0.774000000000001	94.1147218342543\\
0.776000000000001	94.0676994164731\\
0.778000000000001	94.0225571627318\\
0.780000000000001	93.9791559957332\\
0.782000000000001	93.9373670113642\\
0.784000000000001	93.8970709938708\\
0.786000000000001	93.8581578791893\\
0.788000000000001	93.8205261898527\\
0.790000000000001	93.7840824599846\\
0.792000000000001	93.748740664472\\
0.794000000000001	93.7144216627592\\
0.796000000000001	93.6810526647162\\
0.798000000000001	93.6485667234094\\
0.800000000000001	93.6169022578522\\
0.802000000000001	93.5860026071935\\
0.804000000000001	93.5558156167187\\
0.806000000000001	93.5262932552699\\
0.808000000000001	93.4973912629906\\
0.810000000000001	93.4690688280816\\
0.812000000000001	93.441288290887\\
0.814000000000001	93.4140148734911\\
0.816000000000001	93.3872164330753\\
0.818000000000001	93.3608632371577\\
0.820000000000001	93.3349277589957\\
0.822000000000001	93.3093844913654\\
0.824000000000001	93.2842097772721\\
0.826000000000001	93.2593816558826\\
0.828000000000001	93.2348797225138\\
0.830000000000001	93.2106850012008\\
0.832000000000001	93.1867798288201\\
0.834000000000001	93.1631477495758\\
0.836000000000001	93.1397734189541\\
0.838000000000001	93.1166425161864\\
0.840000000000001	93.0937416644683\\
0.842000000000001	93.0710583581989\\
0.844000000000001	93.0485808964898\\
0.846000000000001	93.0262983225156\\
0.848000000000001	93.0042003679684\\
0.850000000000001	92.9822774023308\\
0.852000000000001	92.9605203863239\\
0.854000000000001	92.9389208293139\\
0.856000000000001	92.9174707501889\\
0.858000000000001	92.8961626414962\\
0.860000000000001	92.8749894364354\\
0.862000000000001	92.8539444785171\\
0.864000000000001	92.8330214936327\\
0.866000000000001	92.8122145643462\\
0.868000000000001	92.7915181061446\\
0.870000000000001	92.7709268455464\\
0.872000000000001	92.750435799873\\
0.874000000000001	92.7300402585384\\
0.876000000000001	92.7097357657541\\
0.878000000000001	92.6895181044904\\
0.880000000000001	92.6693832816225\\
0.882000000000001	92.6493275141449\\
0.884000000000001	92.6293472163847\\
0.886000000000001	92.6094389881081\\
0.888000000000001	92.5895996034405\\
0.890000000000001	92.5698260006308\\
0.892000000000001	92.5501152724054\\
0.894000000000001	92.5304646570725\\
0.896000000000001	92.5108715301712\\
0.898000000000001	92.4913333967132\\
0.900000000000001	92.4718478838949\\
0.902000000000001	92.4524127343086\\
0.904000000000001	92.4330257995976\\
0.906000000000001	92.4136850344967\\
0.908000000000001	92.3943884912454\\
0.910000000000001	92.3751343144093\\
0.912000000000001	92.35592073592\\
0.914000000000001	92.3367460705112\\
0.916000000000001	92.3176087113892\\
0.918000000000001	92.2985071261726\\
0.920000000000001	92.2794398530663\\
0.922000000000001	92.2604054972821\\
0.924000000000001	92.241402727637\\
0.926000000000001	92.222430273384\\
0.928000000000001	92.2034869211938\\
0.930000000000001	92.1845715123229\\
0.932000000000001	92.1656829399517\\
0.934000000000001	92.1468201466449\\
0.936000000000001	92.127982121978\\
0.938000000000001	92.1091679002788\\
0.940000000000001	92.0903765584984\\
0.942000000000001	92.0716072141866\\
0.944000000000001	92.0528590235984\\
0.946000000000001	92.0341311798729\\
0.948000000000001	92.0154229113438\\
0.950000000000001	91.9967334798776\\
0.952000000000001	91.9780621793829\\
0.954000000000001	91.95940833431\\
0.956000000000001	91.9407712982943\\
0.958000000000001	91.9221504528274\\
0.960000000000001	91.9035452060103\\
0.962000000000001	91.8849549913703\\
0.964000000000001	91.8663792667276\\
0.966000000000001	91.8478175131332\\
0.968000000000001	91.8292692338417\\
0.970000000000001	91.8107339533395\\
0.972000000000001	91.7922112164121\\
0.974000000000001	91.7737005873004\\
0.976000000000001	91.7552016488094\\
0.978000000000001	91.7367140015409\\
0.980000000000001	91.7182372631581\\
0.982000000000001	91.6997710675934\\
0.984000000000001	91.6813150644141\\
0.986000000000001	91.6628689181368\\
0.988000000000001	91.6444323075995\\
0.990000000000001	91.6260049253624\\
0.992000000000001	91.607586477135\\
0.994000000000001	91.589176681226\\
0.996000000000001	91.5707752680136\\
0.998000000000001	91.5523819794583\\
1	91.5339965686169\\
1.002	91.5156187991677\\
1.004	91.4972484450065\\
1.006	91.4788852898017\\
1.008	91.4605291265993\\
1.01	91.4421797574571\\
1.012	91.4238369930459\\
1.014	91.4055006523239\\
1.016	91.3871705621924\\
1.018	91.3688465571719\\
1.02	91.3505284790859\\
1.022	91.3322161767943\\
1.024	91.3139095058706\\
1.026	91.2956083283585\\
1.028	91.2773125125002\\
1.03	91.2590219324925\\
1.032	91.2407364682364\\
1.034	91.2224560051211\\
1.036	91.2041804337974\\
1.038	91.1859096499666\\
1.04	91.1676435541758\\
1.042	91.1493820516221\\
1.044	91.1311250519852\\
1.046	91.1128724692091\\
1.048	91.0946242213693\\
1.05	91.0763802304856\\
1.052	91.058140422366\\
1.054	91.0399047264698\\
1.056	91.0216730757258\\
1.058	91.0034454064374\\
1.06	90.9852216581152\\
1.062	90.9670017733538\\
1.064	90.9487856977148\\
1.066	90.930573379595\\
1.068	90.9123647701315\\
1.07	90.8941598230552\\
1.072	90.8759584946354\\
1.074	90.8577607435242\\
1.076	90.8395665306944\\
1.078	90.8213758193285\\
1.08	90.8031885747429\\
1.082	90.7850047642772\\
1.084	90.7668243572221\\
1.086	90.7486473247548\\
1.088	90.7304736398322\\
1.09	90.7123032771359\\
1.092	90.6941362129857\\
1.094	90.6759724252857\\
1.096	90.6578118934601\\
1.098	90.6396545983605\\
1.1	90.6215005222342\\
1.102	90.6033496486513\\
1.104	90.5852019624552\\
1.106	90.5670574496977\\
1.108	90.5489160976006\\
1.11	90.5307778944919\\
1.112	90.5126428297508\\
1.114	90.4945108937895\\
1.116	90.476382077978\\
1.118	90.4582563746163\\
1.12	90.4401337768856\\
1.122	90.4220142788204\\
1.124	90.4038978752491\\
1.126	90.3857845617714\\
1.128	90.3676743347236\\
1.13	90.3495671911448\\
1.132	90.3314631287238\\
1.134	90.313362145796\\
1.136	90.2952642412863\\
1.138	90.2771694147065\\
1.14	90.2590776661072\\
1.142	90.2409889960513\\
1.144	90.2229034056055\\
1.146	90.2048208962827\\
1.148	90.1867414700597\\
1.15	90.1686651293154\\
1.152	90.1505918768353\\
1.154	90.1325217157645\\
1.156	90.1144546496252\\
1.158	90.0963906822521\\
1.16	90.078329817806\\
1.162	90.0602720607379\\
1.164	90.0422174157821\\
1.166	90.0241658879415\\
1.168	90.0061174824441\\
1.17	89.988072204769\\
1.172	89.9700300606025\\
1.174	89.9519910558258\\
1.176	89.9339551965247\\
1.178	89.915922488932\\
1.18	89.8978929394616\\
1.182	89.8798665546748\\
1.184	89.8618433412625\\
1.186	89.8438233060419\\
1.188	89.8258064559489\\
1.19	89.8077927980261\\
1.192	89.7897823394055\\
1.194	89.7717750873087\\
1.196	89.7537710490257\\
1.198	89.7357702319285\\
1.2	89.7177726434343\\
1.202	89.6997782910172\\
1.204	89.6817871822027\\
1.206	89.6637993245382\\
1.208	89.6458147256097\\
1.21	89.6278333930201\\
1.212	89.6098553344007\\
1.214	89.5918805573793\\
1.216	89.5739090696008\\
1.218	89.5559408786935\\
1.22	89.5379759922971\\
1.222	89.5200144180232\\
1.224	89.5020561634905\\
1.226	89.4841012362722\\
1.228	89.4661496439313\\
1.23	89.4482013939955\\
1.232	89.4302564939639\\
1.234	89.412314951298\\
1.236	89.394376773417\\
1.238	89.3764419677054\\
1.24	89.3585105414855\\
1.242	89.3405825020412\\
1.244	89.3226578566067\\
1.246	89.3047366123526\\
1.248	89.2868187763905\\
1.25	89.2689043557856\\
1.252	89.250993357524\\
1.254	89.2330857885432\\
1.256	89.2151816557008\\
1.258	89.1972809657982\\
1.26	89.1793837255564\\
1.262	89.1614899416298\\
1.264	89.1435996206008\\
1.266	89.1257127689772\\
1.268	89.1078293931859\\
1.27	89.0899494995901\\
1.272	89.0720730944531\\
1.274	89.0542001839812\\
1.276	89.0363307742918\\
1.278	89.0184648714181\\
1.28	89.0006024813136\\
1.282	88.9827436098564\\
1.284	88.9648882628243\\
1.286	88.9470364459382\\
1.288	88.9291881648135\\
1.29	88.9113434249891\\
1.292	88.8935022319207\\
1.294	88.8756645909674\\
1.296	88.857830507425\\
1.298	88.8399999864845\\
1.3	88.8221730332603\\
1.302	88.8043496527808\\
1.304	88.7865298499868\\
1.306	88.7687136297324\\
1.308	88.7509009967874\\
1.31	88.7330919558418\\
1.312	88.7152865114899\\
1.314	88.6974846682497\\
1.316	88.6796864305491\\
1.318	88.661891802738\\
1.32	88.6441007890719\\
1.322	88.6263133937333\\
1.324	88.6085296208081\\
1.326	88.5907494743116\\
1.328	88.5729729581719\\
1.33	88.5552000762296\\
1.332	88.537430832256\\
1.334	88.5196652299225\\
1.336	88.5019032728335\\
1.338	88.4841449645136\\
1.34	88.4663903083994\\
1.342	88.4486393078465\\
1.344	88.4308919661437\\
1.346	88.4131482864882\\
1.348	88.3954082720148\\
1.35	88.3776719257642\\
1.352	88.359939250712\\
1.354	88.342210249754\\
1.356	88.3244849257108\\
1.358	88.306763281335\\
1.36	88.2890453192965\\
1.362	88.2713310421948\\
1.364	88.2536204525574\\
1.366	88.2359135528462\\
1.368	88.2182103454374\\
1.37	88.2005108326536\\
1.372	88.1828150167386\\
1.374	88.1651228998693\\
1.376	88.1474344841521\\
1.378	88.1297497716288\\
1.38	88.1120687642792\\
1.382	88.0943914640076\\
1.384	88.076717872662\\
1.386	88.0590479920151\\
1.388	88.0413818237899\\
1.39	88.0237193696359\\
1.392	88.0060606311437\\
1.394	87.9884056098462\\
1.396	87.9707543072058\\
1.398	87.9531067246413\\
1.4	87.935462863496\\
1.402	87.917822725064\\
1.404	87.9001863105758\\
1.406	87.8825536212171\\
1.408	87.8649246580997\\
1.41	87.8472994222967\\
1.412	87.8296779148158\\
1.414	87.8120601366168\\
1.416	87.7944460885991\\
1.418	87.7768357716204\\
1.42	87.7592291864778\\
1.422	87.7416263339215\\
1.424	87.7240272146478\\
1.426	87.7064318293137\\
1.428	87.6888401785144\\
1.43	87.6712522628041\\
1.432	87.6536680826871\\
1.434	87.6360876386227\\
1.436	87.6185109310223\\
1.438	87.6009379602594\\
1.44	87.5833687266532\\
1.442	87.5658032304821\\
1.444	87.5482414719826\\
1.446	87.5306834513501\\
1.448	87.5131291687316\\
1.45	87.4955786242451\\
1.452	87.4780318179526\\
1.454	87.4604887498937\\
1.456	87.4429494200486\\
1.458	87.4254138283765\\
1.46	87.4078819747919\\
1.462	87.390353859168\\
1.464	87.3728294813488\\
1.466	87.3553088411357\\
1.468	87.3377919383071\\
1.47	87.3202787725803\\
1.472	87.3027693436651\\
1.474	87.2852636512294\\
1.476	87.2677616949037\\
1.478	87.2502634742899\\
1.48	87.2327689889557\\
1.482	87.2152782384381\\
1.484	87.1977912222466\\
1.486	87.1803079398578\\
1.488	87.1628283907152\\
1.49	87.1453525742373\\
1.492	87.1278804898151\\
1.494	87.1104121368124\\
1.496	87.0929475145605\\
1.498	87.0754866223662\\
1.5	87.0580294595091\\
1.502	87.0405760252502\\
1.504	87.0231263188095\\
1.506	87.0056803393952\\
1.508	86.9882380861899\\
1.51	86.9707995583483\\
1.512	86.9533647550034\\
1.514	86.9359336752595\\
1.516	86.9185063182093\\
1.518	86.9010826829217\\
1.52	86.8836627684326\\
1.522	86.8662465737656\\
1.524	86.848834097924\\
1.526	86.8314253398889\\
1.528	86.8140202986205\\
1.53	86.7966189730644\\
1.532	86.7792213621421\\
1.534	86.7618274647589\\
1.536	86.7444372798003\\
1.538	86.7270508061354\\
1.54	86.7096680426175\\
1.542	86.6922889880785\\
1.544	86.6749136413431\\
1.546	86.6575420012006\\
1.548	86.6401740664503\\
1.55	86.6228098358549\\
1.552	86.6054493081774\\
1.554	86.5880924821538\\
1.556	86.5707393565127\\
1.558	86.553389929968\\
1.56	86.5360442012115\\
1.562	86.5187021689409\\
1.564	86.5013638318228\\
1.566	86.4840291885183\\
1.568	86.4666982376766\\
1.57	86.4493709779329\\
1.572	86.4320474079156\\
1.574	86.4147275262348\\
1.576	86.3974113314918\\
1.578	86.3800988222824\\
1.58	86.3627899971851\\
1.582	86.3454848547745\\
1.584	86.3281833936068\\
1.586	86.3108856122407\\
1.588	86.2935915092108\\
1.59	86.2763010830581\\
1.592	86.2590143323022\\
1.594	86.2417312554662\\
1.596	86.2244518510518\\
1.598	86.2071761175624\\
1.6	86.1899040534946\\
1.602	86.1726356573299\\
1.604	86.1553709275461\\
1.606	86.1381098626172\\
1.608	86.120852461006\\
1.61	86.1035987211716\\
1.612	86.0863486415667\\
1.614	86.0691022206396\\
1.616	86.051859456827\\
1.618	86.0346203485637\\
1.62	86.0173848942831\\
1.622	86.0001530924056\\
1.624	85.9829249413528\\
1.626	85.9657004395421\\
1.628	85.9484795853794\\
1.63	85.9312623772737\\
1.632	85.914048813626\\
1.634	85.8968388928372\\
1.636	85.8796326132964\\
1.638	85.862429973398\\
1.64	85.8452309715302\\
1.642	85.8280356060786\\
1.644	85.8108438754167\\
1.646	85.79365577793\\
1.648	85.7764713119932\\
1.65	85.7592904759787\\
1.652	85.7421132682612\\
1.654	85.7249396872037\\
1.656	85.7077697311781\\
1.658	85.6906033985458\\
1.66	85.67344068767\\
1.662	85.6562815969143\\
1.664	85.6391261246388\\
1.666	85.6219742691983\\
1.668	85.6048260289586\\
1.67	85.5876814022694\\
1.672	85.5705403874909\\
1.674	85.553402982974\\
1.676	85.5362691870783\\
1.678	85.5191389981551\\
1.68	85.5020124145552\\
1.682	85.4848894346359\\
1.684	85.4677700567505\\
1.686	85.4506542792485\\
1.688	85.4335421004893\\
1.69	85.4164335188174\\
1.692	85.3993285325962\\
1.694	85.3822271401754\\
1.696	85.3651293399104\\
1.698	85.3480351301535\\
1.7	85.3309445092633\\
1.702	85.3138574756005\\
1.704	85.2967740275178\\
1.706	85.2796941633759\\
1.708	85.2626178815366\\
1.71	85.2455451803539\\
1.712	85.2284760581981\\
1.714	85.2114105134302\\
1.716	85.1943485444144\\
1.718	85.1772901495167\\
1.72	85.1602353271093\\
1.722	85.1431840755588\\
1.724	85.1261363932393\\
1.726	85.1090922785251\\
1.728	85.0920517297842\\
1.73	85.0750147454035\\
1.732	85.0579813237563\\
1.734	85.0409514632307\\
1.736	85.0239251622031\\
1.738	85.0069024190668\\
1.74	84.9898832322057\\
1.742	84.9728676000115\\
1.744	84.9558555208731\\
1.746	84.938846993195\\
1.748	84.9218420153661\\
1.75	84.9048405857938\\
1.752	84.887842702881\\
1.754	84.8708483650321\\
1.756	84.8538575706555\\
1.758	84.8368703181611\\
1.76	84.819886605969\\
1.762	84.8029064324946\\
1.764	84.7859297961566\\
1.766	84.7689566953756\\
1.768	84.7519871285847\\
1.77	84.7350210942085\\
1.772	84.7180585906786\\
1.774	84.7010996164386\\
1.776	84.6841441699191\\
1.778	84.6671922495644\\
1.78	84.650243853819\\
1.782	84.6332989811351\\
1.784	84.616357629961\\
1.786	84.5994197987546\\
1.788	84.5824854859723\\
1.79	84.5655546900762\\
1.792	84.5486274095349\\
1.794	84.5317036428131\\
1.796	84.5147833883871\\
1.798	84.4978666447301\\
1.8	84.480953410322\\
1.802	84.4640436836476\\
1.804	84.4471374631938\\
1.806	84.4302347474494\\
1.808	84.4133355349082\\
1.81	84.3964398240695\\
1.812	84.379547613431\\
1.814	84.3626589015026\\
1.816	84.3457736867899\\
1.818	84.328891967807\\
1.82	84.3120137430686\\
1.822	84.2951390110952\\
1.824	84.2782677704108\\
1.826	84.2614000195431\\
1.828	84.2445357570233\\
1.83	84.2276749813855\\
1.832	84.2108176911731\\
1.834	84.1939638849196\\
1.836	84.1771135611818\\
1.838	84.1602667185048\\
1.84	84.1434233554441\\
1.842	84.1265834705571\\
1.844	84.109747062409\\
1.846	84.0929141295631\\
1.848	84.0760846705884\\
1.85	84.0592586840628\\
1.852	84.0424361685611\\
1.854	84.0256171226666\\
1.856	84.0088015449656\\
1.858	83.991989434044\\
1.86	83.9751807884969\\
1.862	83.9583756069211\\
1.864	83.9415738879207\\
1.866	83.9247756301003\\
1.868	83.9079808320666\\
1.87	83.8911894924368\\
1.872	83.8744016098226\\
1.874	83.8576171828503\\
1.876	83.8408362101411\\
1.878	83.8240586903266\\
1.88	83.8072846220391\\
1.882	83.7905140039151\\
1.884	83.773746834593\\
1.886	83.756983112719\\
1.888	83.7402228369453\\
1.89	83.7234660059208\\
1.892	83.7067126183\\
1.894	83.6899626727468\\
1.896	83.6732161679238\\
1.898	83.6564731024992\\
1.9	83.6397334751454\\
1.902	83.6229972845389\\
1.904	83.6062645293604\\
1.906	83.5895352082895\\
1.908	83.5728093200184\\
1.91	83.5560868632339\\
1.912	83.5393678366367\\
1.914	83.522652238922\\
1.916	83.505940068795\\
1.918	83.4892313249586\\
1.92	83.4725260061283\\
1.922	83.4558241110155\\
1.924	83.4391256383416\\
1.926	83.4224305868256\\
1.928	83.405738955194\\
1.93	83.3890507421784\\
1.932	83.3723659465122\\
1.934	83.3556845669277\\
1.936	83.3390066021688\\
1.938	83.3223320509842\\
1.94	83.3056609121186\\
1.942	83.2889931843215\\
1.944	83.2723288663542\\
1.946	83.2556679569745\\
1.948	83.2390104549453\\
1.95	83.2223563590303\\
1.952	83.2057056680064\\
1.954	83.1890583806424\\
1.956	83.1724144957192\\
1.958	83.1557740120181\\
1.96	83.1391369283229\\
1.962	83.122503243423\\
1.964	83.1058729561099\\
1.966	83.0892460651832\\
1.968	83.0726225694412\\
1.97	83.0560024676805\\
1.972	83.039385758716\\
1.974	83.0227724413521\\
1.976	83.006162514409\\
1.978	82.989555976697\\
1.98	82.9729528270393\\
1.982	82.9563530642592\\
1.984	82.9397566871874\\
1.986	82.9231636946528\\
1.988	82.9065740854901\\
1.99	82.8899878585362\\
1.992	82.873405012633\\
1.994	82.8568255466237\\
1.996	82.8402494593606\\
1.998	82.8236767496905\\
2	82.8071074164706\\
2.002	82.7905414585582\\
2.004	82.7739788748158\\
2.006	82.7574196641068\\
2.008	82.7408638253008\\
2.01	82.7243113572664\\
2.012	82.7077622588813\\
2.014	82.6912165290196\\
2.016	82.6746741665645\\
2.018	82.6581351703997\\
2.02	82.6415995394143\\
2.022	82.6250672724975\\
2.024	82.6085383685447\\
2.026	82.5920128264489\\
2.028	82.575490645113\\
2.03	82.5589718234408\\
2.032	82.5424563603384\\
2.034	82.5259442547117\\
2.036	82.5094355054778\\
2.038	82.4929301115496\\
2.04	82.4764280718458\\
2.042	82.4599293852888\\
2.044	82.443434050805\\
2.046	82.4269420673195\\
2.048	82.4104534337653\\
2.05	82.3939681490724\\
2.052	82.3774862121803\\
2.054	82.3610076220283\\
2.056	82.3445323775545\\
2.05799999999999	82.3280604777127\\
2.05999999999999	82.311591921445\\
2.06199999999999	82.2951267077029\\
2.06399999999999	82.2786648354429\\
2.06599999999999	82.2622063036201\\
2.06799999999999	82.2457511111956\\
2.06999999999999	82.2292992571287\\
2.07199999999999	82.2128507403894\\
2.07399999999999	82.1964055599419\\
2.07599999999999	82.1799637147622\\
2.07799999999999	82.163525203816\\
2.07999999999999	82.1470900260878\\
2.08199999999999	82.1306581805499\\
2.08399999999999	82.1142296661921\\
2.08599999999999	82.0978044819918\\
2.08799999999999	82.0813826269374\\
2.08999999999999	82.0649641000226\\
2.09199999999999	82.0485489002378\\
2.09399999999999	82.0321370265748\\
2.09599999999999	82.0157284780392\\
2.09799999999999	81.9993232536232\\
2.09999999999999	81.9829213523393\\
2.10199999999999	81.966522773182\\
2.10399999999999	81.9501275151675\\
2.10599999999999	81.9337355773043\\
2.10799999999999	81.9173469586071\\
2.10999999999999	81.900961658088\\
2.11199999999999	81.8845796747698\\
2.11399999999999	81.8682010076679\\
2.11599999999999	81.8518256558123\\
2.11799999999999	81.835453618224\\
2.11999999999999	81.819084893931\\
2.12199999999999	81.80271948197\\
2.12399999999999	81.7863573813683\\
2.12599999999999	81.7699985911607\\
2.12799999999999	81.7536431103903\\
2.12999999999999	81.7372909380935\\
2.13199999999999	81.720942073313\\
2.13399999999999	81.7045965150963\\
2.13599999999999	81.6882542624889\\
2.13799999999999	81.6719153145405\\
2.13999999999999	81.6555796703043\\
2.14199999999999	81.6392473288341\\
2.14399999999999	81.6229182891837\\
2.14599999999999	81.6065925504163\\
2.14799999999999	81.5902701115916\\
2.14999999999998	81.5739509717701\\
2.15199999999998	81.5576351300235\\
2.15399999999998	81.5413225854142\\
2.15599999999998	81.5250133370127\\
2.15799999999998	81.5087073838934\\
2.15999999999998	81.4924047251298\\
2.16199999999998	81.4761053597977\\
2.16399999999998	81.4598092869764\\
2.16599999999998	81.4435165057484\\
2.16799999999998	81.4272270151906\\
2.16999999999998	81.4109408143949\\
2.17199999999998	81.3946579024466\\
2.17399999999998	81.3783782784353\\
2.17599999999998	81.362101941447\\
2.17799999999998	81.3458288905804\\
2.17999999999998	81.3295591249311\\
2.18199999999998	81.3132926435957\\
2.18399999999998	81.2970294456692\\
2.18599999999998	81.280769530259\\
2.18799999999998	81.2645128964663\\
2.18999999999998	81.2482595433991\\
2.19199999999998	81.2320094701583\\
2.19399999999998	81.2157626758604\\
2.19599999999998	81.1995191596105\\
2.19799999999998	81.1832789205275\\
2.19999999999998	81.167041957722\\
2.20199999999998	81.1508082703122\\
2.20399999999998	81.1345778574199\\
2.20599999999998	81.1183507181614\\
2.20799999999998	81.1021268516622\\
2.20999999999998	81.0859062570466\\
2.21199999999998	81.0696889334403\\
2.21399999999998	81.0534748799726\\
2.21599999999998	81.037264095771\\
2.21799999999998	81.021056579974\\
2.21999999999998	81.0048523317066\\
2.22199999999998	80.9886513501074\\
2.22399999999998	80.9724536343182\\
2.22599999999998	80.9562591834695\\
2.22799999999998	80.9400679967065\\
2.22999999999998	80.9238800731753\\
2.23199999999998	80.907695412017\\
2.23399999999998	80.8915140123756\\
2.23599999999998	80.8753358734003\\
2.23799999999998	80.8591609942414\\
2.23999999999997	80.8429893740492\\
2.24199999999997	80.8268210119747\\
2.24399999999997	80.8106559071743\\
2.24599999999997	80.7944940588043\\
2.24799999999997	80.778335466019\\
2.24999999999997	80.7621801279808\\
2.25199999999997	80.7460280438541\\
2.25399999999997	80.7298792127944\\
2.25599999999997	80.7137336339684\\
2.25799999999997	80.6975913065451\\
2.25999999999997	80.6814522296855\\
2.26199999999997	80.6653164025646\\
2.26399999999997	80.6491838243475\\
2.26599999999997	80.6330544942112\\
2.26799999999997	80.6169284113251\\
2.26999999999997	80.6008055748665\\
2.27199999999997	80.584685984013\\
2.27399999999997	80.5685696379411\\
2.27599999999997	80.5524565358287\\
2.27799999999997	80.5363466768598\\
2.27999999999997	80.5202400602161\\
2.28199999999997	80.5041366850812\\
2.28399999999997	80.4880365506397\\
2.28599999999997	80.4719396560798\\
2.28799999999997	80.4558460005896\\
2.28999999999997	80.4397555833584\\
2.29199999999997	80.4236684035784\\
2.29399999999997	80.4075844604406\\
2.29599999999997	80.391503753143\\
2.29799999999997	80.3754262808747\\
2.29999999999997	80.3593520428366\\
2.30199999999997	80.3432810382263\\
2.30399999999997	80.3272132662463\\
2.30599999999997	80.3111487260904\\
2.30799999999997	80.2950874169679\\
2.30999999999997	80.2790293380768\\
2.31199999999997	80.262974488624\\
2.31399999999997	80.2469228678201\\
2.31599999999997	80.2308744748636\\
2.31799999999997	80.2148293089711\\
2.31999999999997	80.1987873693539\\
2.32199999999997	80.1827486552155\\
2.32399999999997	80.1667131657744\\
2.32599999999997	80.1506809002427\\
2.32799999999997	80.1346518578354\\
2.32999999999996	80.1186260377704\\
2.33199999999996	80.1026034392644\\
2.33399999999996	80.0865840615371\\
2.33599999999996	80.0705679038094\\
2.33799999999996	80.0545549652983\\
2.33999999999996	80.0385452452311\\
2.34199999999996	80.0225387428306\\
2.34399999999996	80.0065354573197\\
2.34599999999996	79.990535387927\\
2.34799999999996	79.9745385338796\\
2.34999999999996	79.9585448944062\\
2.35199999999996	79.9425544687356\\
2.35399999999996	79.9265672560987\\
2.35599999999996	79.9105832557281\\
2.35799999999996	79.8946024668572\\
2.35999999999996	79.8786248887198\\
2.36199999999996	79.8626505205497\\
2.36399999999996	79.8466793615858\\
2.36599999999996	79.8307114110665\\
2.36799999999996	79.8147466682294\\
2.36999999999996	79.7987851323126\\
2.37199999999996	79.7828268025596\\
2.37399999999996	79.7668716782078\\
2.37599999999996	79.750919758509\\
2.37799999999996	79.7349710426984\\
2.37999999999996	79.7190255300278\\
2.38199999999996	79.7030832197383\\
2.38399999999996	79.6871441110814\\
2.38599999999996	79.671208203301\\
2.38799999999996	79.6552754956509\\
2.38999999999996	79.6393459873801\\
2.39199999999996	79.6234196777377\\
2.39399999999996	79.6074965659799\\
2.39599999999996	79.5915766513594\\
2.39799999999996	79.5756599331265\\
2.39999999999996	79.5597464105417\\
2.40199999999996	79.5438360828596\\
2.40399999999996	79.5279289493346\\
2.40599999999996	79.5120250092309\\
2.40799999999996	79.4961242618028\\
2.40999999999996	79.4802267063124\\
2.41199999999996	79.4643323420222\\
2.41399999999996	79.448441168193\\
2.41599999999996	79.4325531840881\\
2.41799999999996	79.4166683889712\\
2.41999999999996	79.4007867821079\\
2.42199999999995	79.3849083627637\\
2.42399999999995	79.3690331302062\\
2.42599999999995	79.3531610837011\\
2.42799999999995	79.3372922225194\\
2.42999999999995	79.3214265459293\\
2.43199999999995	79.3055640532023\\
2.43399999999995	79.2897047436075\\
2.43599999999995	79.273848616419\\
2.43799999999995	79.2579956709098\\
2.43999999999995	79.2421459063529\\
2.44199999999995	79.2262993220227\\
2.44399999999995	79.2104559171974\\
2.44599999999995	79.1946156911497\\
2.44799999999995	79.1787786431612\\
2.44999999999995	79.1629447725082\\
2.45199999999995	79.147114078465\\
2.45399999999995	79.1312865603202\\
2.45599999999995	79.1154622173471\\
2.45799999999995	79.0996410488315\\
2.45999999999995	79.0838230540548\\
2.46199999999995	79.0680082322976\\
2.46399999999995	79.0521965828454\\
2.46599999999995	79.0363881049835\\
2.46799999999995	79.0205827979983\\
2.46999999999995	79.0047806611737\\
2.47199999999995	78.9889816937983\\
2.47399999999995	78.9731858951591\\
2.47599999999995	78.9573932645461\\
2.47799999999995	78.9416038012451\\
2.47999999999995	78.9258175045509\\
2.48199999999995	78.9100343737508\\
2.48399999999995	78.8942544081365\\
2.48599999999995	78.8784776070034\\
2.48799999999995	78.8627039696426\\
2.48999999999995	78.8469334953468\\
2.49199999999995	78.831166183414\\
2.49399999999995	78.8154020331348\\
2.49599999999995	78.7996410438076\\
2.49799999999995	78.7838832147317\\
2.49999999999995	78.7681285451987\\
2.50199999999995	78.7523770345119\\
2.50399999999995	78.7366286819671\\
2.50599999999995	78.7208834868683\\
2.50799999999995	78.7051414485091\\
2.50999999999995	78.6894025661951\\
2.51199999999994	78.6736668392278\\
2.51399999999994	78.6579342669067\\
2.51599999999994	78.6422048485369\\
2.51799999999994	78.6264785834201\\
2.51999999999994	78.6107554708654\\
2.52199999999994	78.5950355101698\\
2.52399999999994	78.5793187006469\\
2.52599999999994	78.5636050415976\\
2.52799999999994	78.5478945323338\\
2.52999999999994	78.5321871721587\\
2.53199999999994	78.5164829603809\\
2.53399999999994	78.5007818963113\\
2.53599999999994	78.4850839792564\\
2.53799999999994	78.469389208531\\
2.53999999999994	78.4536975834441\\
2.54199999999994	78.4380091033042\\
2.54399999999994	78.4223237674235\\
2.54599999999994	78.4066415751196\\
2.54799999999994	78.3909625257\\
2.54999999999994	78.375286618484\\
2.55199999999994	78.3596138527803\\
2.55399999999994	78.343944227908\\
2.55599999999994	78.3282777431814\\
2.55799999999994	78.3126143979152\\
2.55999999999994	78.2969541914304\\
2.56199999999994	78.2812971230395\\
2.56399999999994	78.2656431920635\\
2.56599999999994	78.2499923978195\\
2.56799999999994	78.2343447396277\\
2.56999999999994	78.2187002168065\\
2.57199999999994	78.2030588286776\\
2.57399999999994	78.1874205745605\\
2.57599999999994	78.1717854537779\\
2.57799999999994	78.1561534656529\\
2.57999999999994	78.1405246095042\\
2.58199999999994	78.1248988846572\\
2.58399999999994	78.1092762904354\\
2.58599999999994	78.0936568261616\\
2.58799999999994	78.0780404911607\\
2.58999999999994	78.0624272847612\\
2.59199999999994	78.0468172062852\\
2.59399999999994	78.0312102550595\\
2.59599999999994	78.0156064304117\\
2.59799999999994	78.0000057316681\\
2.59999999999994	77.9844081581572\\
2.60199999999994	77.9688137092089\\
2.60399999999993	77.9532223841491\\
2.60599999999993	77.9376341823096\\
2.60799999999993	77.9220491030165\\
2.60999999999993	77.9064671456061\\
2.61199999999993	77.8908883094049\\
2.61399999999993	77.8753125937452\\
2.61599999999993	77.8597399979584\\
2.61799999999993	77.8441705213786\\
2.61999999999993	77.8286041633392\\
2.62199999999993	77.8130409231686\\
2.62399999999993	77.7974808002052\\
2.62599999999993	77.7819237937853\\
2.62799999999993	77.7663699032343\\
2.62999999999993	77.7508191278976\\
2.63199999999993	77.7352714671059\\
2.63399999999993	77.7197269201948\\
2.63599999999993	77.7041854865034\\
2.63799999999993	77.6886471653676\\
2.63999999999993	77.6731119561278\\
2.64199999999993	77.6575798581175\\
2.64399999999993	77.6420508706773\\
2.64599999999993	77.6265249931447\\
2.64799999999993	77.6110022248631\\
2.64999999999993	77.5954825651693\\
2.65199999999993	77.5799660134026\\
2.65399999999993	77.5644525689049\\
2.65599999999993	77.548942231018\\
2.65799999999993	77.5334349990837\\
2.65999999999993	77.5179308724435\\
2.66199999999993	77.5024298504382\\
2.66399999999993	77.4869319324128\\
2.66599999999993	77.4714371177099\\
2.66799999999993	77.4559454056744\\
2.66999999999993	77.440456795648\\
2.67199999999993	77.4249712869748\\
2.67399999999993	77.4094888790058\\
2.67599999999993	77.3940095710821\\
2.67799999999993	77.3785333625481\\
2.67999999999993	77.3630602527528\\
2.68199999999993	77.3475902410394\\
2.68399999999993	77.332123326759\\
2.68599999999993	77.316659509259\\
2.68799999999993	77.3011987878824\\
2.68999999999993	77.2857411619828\\
2.69199999999993	77.2702866309048\\
2.69399999999992	77.2548351940003\\
2.69599999999992	77.2393868506177\\
2.69799999999992	77.2239416001082\\
2.69999999999992	77.2084994418192\\
2.70199999999992	77.1930603751022\\
2.70399999999992	77.1776243993109\\
2.70599999999992	77.1621915137935\\
2.70799999999992	77.1467617179028\\
2.70999999999992	77.1313350109907\\
2.71199999999992	77.1159113924095\\
2.71399999999992	77.1004908615145\\
2.71599999999992	77.085073417656\\
2.71799999999992	77.0696590601872\\
2.71999999999992	77.0542477884642\\
2.72199999999992	77.0388396018418\\
2.72399999999992	77.023434499673\\
2.72599999999992	77.0080324813125\\
2.72799999999992	76.9926335461179\\
2.72999999999992	76.9772376934426\\
2.73199999999992	76.9618449226439\\
2.73399999999992	76.9464552330772\\
2.73599999999992	76.9310686240998\\
2.73799999999992	76.9156850950724\\
2.73999999999992	76.9003046453474\\
2.74199999999992	76.8849272742842\\
2.74399999999992	76.8695529812434\\
2.74599999999992	76.8541817655812\\
2.74799999999992	76.8388136266558\\
2.74999999999992	76.8234485638311\\
2.75199999999992	76.8080865764607\\
2.75399999999992	76.7927276639062\\
2.75599999999992	76.7773718255307\\
2.75799999999992	76.7620190606927\\
2.75999999999992	76.7466693687525\\
2.76199999999992	76.7313227490736\\
2.76399999999992	76.715979201013\\
2.76599999999992	76.7006387239387\\
2.76799999999992	76.6853013172088\\
2.76999999999992	76.6699669801879\\
2.77199999999992	76.6546357122367\\
2.77399999999992	76.6393075127211\\
2.77599999999992	76.6239823810015\\
2.77799999999992	76.6086603164439\\
2.77999999999992	76.5933413184109\\
2.78199999999992	76.5780253862691\\
2.78399999999991	76.5627125193815\\
2.78599999999991	76.547402717115\\
2.78799999999991	76.5320959788315\\
2.78999999999991	76.5167923038988\\
2.79199999999991	76.5014916916834\\
2.79399999999991	76.4861941415505\\
2.79599999999991	76.4708996528687\\
2.79799999999991	76.4556082250019\\
2.79999999999991	76.4403198573187\\
2.80199999999991	76.4250345491875\\
2.80399999999991	76.4097522999747\\
2.80599999999991	76.3944731090477\\
2.80799999999991	76.3791969757778\\
2.80999999999991	76.3639238995304\\
2.81199999999991	76.3486538796752\\
2.81399999999991	76.3333869155846\\
2.81599999999991	76.3181230066241\\
2.81799999999991	76.3028621521647\\
2.81999999999991	76.287604351577\\
2.82199999999991	76.2723496042308\\
2.82399999999991	76.2570979094965\\
2.82599999999991	76.2418492667474\\
2.82799999999991	76.226603675351\\
2.82999999999991	76.2113611346814\\
2.83199999999991	76.196121644109\\
2.83399999999991	76.1808852030076\\
2.83599999999991	76.1656518107466\\
2.83799999999991	76.1504214667011\\
2.83999999999991	76.1351941702427\\
2.84199999999991	76.1199699207465\\
2.84399999999991	76.1047487175818\\
2.84599999999991	76.0895305601261\\
2.84799999999991	76.0743154477532\\
2.84999999999991	76.0591033798341\\
2.85199999999991	76.0438943557453\\
2.85399999999991	76.0286883748624\\
2.85599999999991	76.0134854365587\\
2.85799999999991	75.9982855402104\\
2.85999999999991	75.9830886851923\\
2.86199999999991	75.9678948708822\\
2.86399999999991	75.9527040966519\\
2.86599999999991	75.9375163618807\\
2.86799999999991	75.9223316659473\\
2.86999999999991	75.9071500082217\\
2.87199999999991	75.8919713880882\\
2.87399999999991	75.8767958049195\\
2.8759999999999	75.8616232580966\\
2.8779999999999	75.8464537469932\\
2.8799999999999	75.8312872709891\\
2.8819999999999	75.8161238294646\\
2.8839999999999	75.8009634217953\\
2.8859999999999	75.7858060473592\\
2.8879999999999	75.7706517055399\\
2.8899999999999	75.755500395714\\
2.8919999999999	75.7403521172592\\
2.8939999999999	75.7252068695578\\
2.8959999999999	75.7100646519902\\
2.8979999999999	75.6949254639337\\
2.8999999999999	75.6797893047701\\
2.9019999999999	75.664656173882\\
2.9039999999999	75.6495260706483\\
2.9059999999999	75.6343989944493\\
2.9079999999999	75.6192749446665\\
2.9099999999999	75.6041539206861\\
2.9119999999999	75.589035921885\\
2.9139999999999	75.5739209476473\\
2.9159999999999	75.5588089973542\\
2.9179999999999	75.5437000703882\\
2.9199999999999	75.5285941661358\\
2.9219999999999	75.513491283972\\
2.9239999999999	75.4983914232893\\
2.9259999999999	75.4832945834673\\
2.9279999999999	75.4682007638868\\
2.9299999999999	75.4531099639342\\
2.9319999999999	75.4380221829948\\
2.9339999999999	75.4229374204491\\
2.9359999999999	75.4078556756873\\
2.9379999999999	75.3927769480892\\
2.9399999999999	75.3777012370437\\
2.9419999999999	75.3626285419326\\
2.9439999999999	75.347558862145\\
2.9459999999999	75.3324921970616\\
2.9479999999999	75.3174285460717\\
2.9499999999999	75.3023679085629\\
2.9519999999999	75.2873102839183\\
2.9539999999999	75.2722556715243\\
2.9559999999999	75.2572040707708\\
2.9579999999999	75.2421554810424\\
2.9599999999999	75.2271099017267\\
2.9619999999999	75.2120673322117\\
2.9639999999999	75.1970277718827\\
2.96599999999989	75.1819912201306\\
2.96799999999989	75.1669576763414\\
2.96999999999989	75.1519271399041\\
2.97199999999989	75.1368996102075\\
2.97399999999989	75.1218750866395\\
2.97599999999989	75.1068535685888\\
2.97799999999989	75.0918350554457\\
2.97999999999989	75.0768195465965\\
2.98199999999989	75.061807041435\\
2.98399999999989	75.0467975393475\\
2.98599999999989	75.0317910397245\\
2.98799999999989	75.0167875419567\\
2.98999999999989	75.0017870454323\\
2.99199999999989	74.9867895495456\\
2.99399999999989	74.9717950536856\\
2.99599999999989	74.9568035572416\\
2.99799999999989	74.941815059607\\
2.99999999999989	74.9268295601721\\
3.00199999999989	74.9118470583254\\
3.00399999999989	74.896867553463\\
3.00599999999989	74.8818910449748\\
3.00799999999989	74.8669175322523\\
3.00999999999989	74.8519470146891\\
3.01199999999989	74.836979491676\\
3.01399999999989	74.8220149626078\\
3.01599999999989	74.8070534268762\\
3.01799999999989	74.7920948838741\\
3.01999999999989	74.7771393329927\\
3.02199999999989	74.7621867736278\\
3.02399999999989	74.7472372051747\\
3.02599999999989	74.7322906270226\\
3.02799999999989	74.7173470385682\\
3.02999999999989	74.7024064392068\\
3.03199999999989	74.6874688283288\\
3.03399999999989	74.6725342053328\\
3.03599999999989	74.6576025696104\\
3.03799999999989	74.6426739205586\\
3.03999999999989	74.627748257571\\
3.04199999999989	74.6128255800431\\
3.04399999999989	74.5979058873722\\
3.04599999999989	74.5829891789516\\
3.04799999999989	74.5680754541785\\
3.04999999999989	74.553164712447\\
3.05199999999989	74.5382569531554\\
3.05399999999989	74.5233521757013\\
3.05599999999989	74.5084503794768\\
3.05799999999988	74.4935515638814\\
3.05999999999988	74.4786557283117\\
3.06199999999988	74.463762872166\\
3.06399999999988	74.4488729948363\\
3.06599999999988	74.4339860957279\\
3.06799999999988	74.4191021742318\\
3.06999999999988	74.404221229747\\
3.07199999999988	74.389343261677\\
3.07399999999988	74.3744682694138\\
3.07599999999988	74.359596252357\\
3.07799999999988	74.3447272099074\\
3.07999999999988	74.32986114146\\
3.08199999999988	74.3149980464175\\
3.08399999999988	74.3001379241762\\
3.08599999999988	74.2852807741353\\
3.08799999999988	74.2704265956957\\
3.08999999999988	74.2555753882573\\
3.09199999999988	74.240727151217\\
3.09399999999988	74.2258818839768\\
3.09599999999988	74.2110395859357\\
3.09799999999988	74.1962002564946\\
3.09999999999988	74.1813638950525\\
3.10199999999988	74.1665305010156\\
3.10399999999988	74.1517000737744\\
3.10599999999988	74.1368726127408\\
3.10799999999988	74.1220481173075\\
3.10999999999988	74.1072265868781\\
3.11199999999988	74.0924080208567\\
3.11399999999988	74.077592418643\\
3.11599999999988	74.0627797796373\\
3.11799999999988	74.0479701032426\\
3.11999999999988	74.0331633888624\\
3.12199999999988	74.0183596358972\\
3.12399999999988	74.0035588437495\\
3.12599999999988	73.9887610118217\\
3.12799999999988	73.9739661395176\\
3.12999999999988	73.9591742262404\\
3.13199999999988	73.9443852713902\\
3.13399999999988	73.9295992743733\\
3.13599999999988	73.9148162345914\\
3.13799999999988	73.9000361514491\\
3.13999999999988	73.8852590243494\\
3.14199999999988	73.8704848526974\\
3.14399999999988	73.855713635894\\
3.14599999999988	73.8409453733454\\
3.14799999999987	73.8261800644569\\
3.14999999999987	73.8114177086325\\
3.15199999999987	73.7966583052754\\
3.15399999999987	73.7819018537922\\
3.15599999999987	73.7671483535847\\
3.15799999999987	73.7523978040632\\
3.15999999999987	73.737650204627\\
3.16199999999987	73.7229055546879\\
3.16399999999987	73.7081638536475\\
3.16599999999987	73.6934251009108\\
3.16799999999987	73.6786892958857\\
3.16999999999987	73.6639564379791\\
3.17199999999987	73.6492265265953\\
3.17399999999987	73.6344995611394\\
3.17599999999987	73.6197755410221\\
3.17799999999987	73.6050544656473\\
3.17999999999987	73.5903363344217\\
3.18199999999987	73.5756211467525\\
3.18399999999987	73.5609089020487\\
3.18599999999987	73.5461995997153\\
3.18799999999987	73.5314932391596\\
3.18999999999987	73.5167898197929\\
3.19199999999987	73.5020893410176\\
3.19399999999987	73.487391802247\\
3.19599999999987	73.4726972028852\\
3.19799999999987	73.4580055423416\\
3.19999999999987	73.4433168200252\\
3.20199999999987	73.4286310353428\\
3.20399999999987	73.4139481877042\\
3.20599999999987	73.3992682765215\\
3.20799999999987	73.3845913011981\\
3.20999999999987	73.3699172611439\\
3.21199999999987	73.3552461557716\\
3.21399999999987	73.340577984488\\
3.21599999999987	73.3259127467047\\
3.21799999999987	73.3112504418314\\
3.21999999999987	73.2965910692728\\
3.22199999999987	73.281934628446\\
3.22399999999987	73.2672811187579\\
3.22599999999987	73.2526305396182\\
3.22799999999987	73.2379828904376\\
3.22999999999987	73.2233381706281\\
3.23199999999987	73.2086963795991\\
3.23399999999987	73.1940575167613\\
3.23599999999987	73.1794215815276\\
3.23799999999986	73.1647885733078\\
3.23999999999986	73.1501584915123\\
3.24199999999986	73.1355313355555\\
3.24399999999986	73.1209071048467\\
3.24599999999986	73.1062857987981\\
3.24799999999986	73.0916674168218\\
3.24999999999986	73.0770519583289\\
3.25199999999986	73.0624394227332\\
3.25399999999986	73.0478298094475\\
3.25599999999986	73.0332231178832\\
3.25799999999986	73.0186193474513\\
3.25999999999986	73.004018497567\\
3.26199999999986	72.9894205676415\\
3.26399999999986	72.9748255570909\\
3.26599999999986	72.9602334653257\\
3.26799999999986	72.9456442917588\\
3.26999999999986	72.9310580358052\\
3.27199999999986	72.9164746968778\\
3.27399999999986	72.9018942743914\\
3.27599999999986	72.8873167677606\\
3.27799999999986	72.8727421763946\\
3.27999999999986	72.8581704997142\\
3.28199999999986	72.8436017371283\\
3.28399999999986	72.8290358880538\\
3.28599999999986	72.8144729519068\\
3.28799999999986	72.7999129280998\\
3.28999999999986	72.7853558160487\\
3.29199999999986	72.7708016151676\\
3.29399999999986	72.7562503248722\\
3.29599999999986	72.7417019445784\\
3.29799999999986	72.7271564737012\\
3.29999999999986	72.7126139116551\\
3.30199999999986	72.6980742578595\\
3.30399999999986	72.6835375117263\\
3.30599999999986	72.6690036726721\\
3.30799999999986	72.6544727401147\\
3.30999999999986	72.6399447134684\\
3.31199999999986	72.6254195921509\\
3.31399999999986	72.6108973755796\\
3.31599999999986	72.5963780631713\\
3.31799999999986	72.5818616543412\\
3.31999999999986	72.5673481485058\\
3.32199999999986	72.5528375450825\\
3.32399999999986	72.5383298434911\\
3.32599999999986	72.5238250431487\\
3.32799999999986	72.5093231434682\\
3.32999999999985	72.4948241438714\\
3.33199999999985	72.4803280437756\\
3.33399999999985	72.4658348425977\\
3.33599999999985	72.4513445397554\\
3.33799999999985	72.4368571346689\\
3.33999999999985	72.4223726267542\\
3.34199999999985	72.407891015433\\
3.34399999999985	72.3934123001207\\
3.34599999999985	72.3789364802377\\
3.34799999999985	72.364463555202\\
3.34999999999985	72.3499935244343\\
3.35199999999985	72.3355263873515\\
3.35399999999985	72.3210621433723\\
3.35599999999985	72.3066007919184\\
3.35799999999985	72.2921423324088\\
3.35999999999985	72.2776867642639\\
3.36199999999985	72.263234086901\\
3.36399999999985	72.2487842997426\\
3.36599999999985	72.234337402206\\
3.36799999999985	72.219893393716\\
3.36999999999985	72.2054522736877\\
3.37199999999985	72.1910140415441\\
3.37399999999985	72.1765786967064\\
3.37599999999985	72.162146238596\\
3.37799999999985	72.1477166666287\\
3.37999999999985	72.1332899802319\\
3.38199999999985	72.1188661788221\\
3.38399999999985	72.1044452618234\\
3.38599999999985	72.0900272286578\\
3.38799999999985	72.0756120787421\\
3.38999999999985	72.061199811502\\
3.39199999999985	72.0467904263592\\
3.39399999999985	72.0323839227346\\
3.39599999999985	72.0179803000498\\
3.39799999999985	72.0035795577295\\
3.39999999999985	71.9891816951893\\
3.40199999999985	71.974786711861\\
3.40399999999985	71.9603946071615\\
3.40599999999985	71.9460053805132\\
3.40799999999985	71.9316190313412\\
3.40999999999985	71.9172355590675\\
3.41199999999985	71.9028549631145\\
3.41399999999985	71.8884772429058\\
3.41599999999985	71.8741023978666\\
3.41799999999985	71.8597304274171\\
3.41999999999984	71.845361330984\\
3.42199999999984	71.8309951079887\\
3.42399999999984	71.8166317578578\\
3.42599999999984	71.8022712800118\\
3.42799999999984	71.7879136738742\\
3.42999999999984	71.7735589388745\\
3.43199999999984	71.7592070744333\\
3.43399999999984	71.7448580799758\\
3.43599999999984	71.7305119549257\\
3.43799999999984	71.7161686987103\\
3.43999999999984	71.701828310752\\
3.44199999999984	71.6874907904749\\
3.44399999999984	71.6731561373075\\
3.44599999999984	71.6588243506738\\
3.44799999999984	71.6444954299979\\
3.44999999999984	71.6301693747069\\
3.45199999999984	71.6158461842244\\
3.45399999999984	71.6015258579788\\
3.45599999999984	71.587208395394\\
3.45799999999984	71.5728937958984\\
3.45999999999984	71.558582058914\\
3.46199999999984	71.5442731838702\\
3.46399999999984	71.5299671701921\\
3.46599999999984	71.5156640173071\\
3.46799999999984	71.5013637246421\\
3.46999999999984	71.4870662916233\\
3.47199999999984	71.4727717176774\\
3.47399999999984	71.4584800022322\\
3.47599999999984	71.4441911447133\\
3.47799999999984	71.4299051445484\\
3.47999999999984	71.4156220011668\\
3.48199999999984	71.4013417139932\\
3.48399999999984	71.3870642824573\\
3.48599999999984	71.3727897059861\\
3.48799999999984	71.3585179840068\\
3.48999999999984	71.3442491159494\\
3.49199999999984	71.3299831012405\\
3.49399999999984	71.3157199393074\\
3.49599999999984	71.3014596295813\\
3.49799999999984	71.2872021714889\\
3.49999999999984	71.2729475644605\\
3.50199999999984	71.2586958079201\\
3.50399999999984	71.2444469013024\\
3.50599999999984	71.2302008440331\\
3.50799999999984	71.215957635542\\
3.50999999999984	71.2017172752606\\
3.51199999999983	71.1874797626133\\
3.51399999999983	71.1732450970338\\
3.51599999999983	71.1590132779499\\
3.51799999999983	71.1447843047932\\
3.51999999999983	71.1305581769904\\
3.52199999999983	71.1163348939736\\
3.52399999999983	71.102114455174\\
3.52599999999983	71.0878968600195\\
3.52799999999983	71.0736821079415\\
3.52999999999983	71.0594701983712\\
3.53199999999983	71.0452611307379\\
3.53399999999983	71.0310549044735\\
3.53599999999983	71.0168515190059\\
3.53799999999983	71.002650973769\\
3.53999999999983	70.9884532681951\\
3.54199999999983	70.9742584017119\\
3.54399999999983	70.9600663737524\\
3.54599999999983	70.9458771837479\\
3.54799999999983	70.9316908311324\\
3.54999999999983	70.9175073153335\\
3.55199999999983	70.9033266357848\\
3.55399999999983	70.8891487919185\\
3.55599999999983	70.8749737831656\\
3.55799999999983	70.8608016089588\\
3.55999999999983	70.8466322687323\\
3.56199999999983	70.8324657619169\\
3.56399999999983	70.8183020879434\\
3.56599999999983	70.8041412462477\\
3.56799999999983	70.7899832362597\\
3.56999999999983	70.7758280574158\\
3.57199999999983	70.7616757091449\\
3.57399999999983	70.7475261908841\\
3.57599999999983	70.7333795020637\\
3.57799999999983	70.7192356421185\\
3.57999999999983	70.7050946104801\\
3.58199999999983	70.6909564065861\\
3.58399999999983	70.6768210298657\\
3.58599999999983	70.6626884797559\\
3.58799999999983	70.6485587556894\\
3.58999999999983	70.6344318571023\\
3.59199999999983	70.6203077834244\\
3.59399999999983	70.606186534093\\
3.59599999999983	70.5920681085438\\
3.59799999999983	70.5779525062105\\
3.59999999999983	70.5638397265251\\
3.60199999999982	70.549729768925\\
3.60399999999982	70.5356226328448\\
3.60599999999982	70.5215183177197\\
3.60799999999982	70.5074168229852\\
3.60999999999982	70.493318148077\\
3.61199999999982	70.4792222924273\\
3.61399999999982	70.4651292554741\\
3.61599999999982	70.4510390366543\\
3.61799999999982	70.4369516354003\\
3.61999999999982	70.422867051151\\
3.62199999999982	70.4087852833413\\
3.62399999999982	70.3947063314086\\
3.62599999999982	70.3806301947854\\
3.62799999999982	70.3665568729122\\
3.62999999999982	70.352486365222\\
3.63199999999982	70.3384186711562\\
3.63399999999982	70.3243537901466\\
3.63599999999982	70.3102917216315\\
3.63799999999982	70.2962324650501\\
3.63999999999982	70.2821760198376\\
3.64199999999982	70.2681223854294\\
3.64399999999982	70.2540715612662\\
3.64599999999982	70.2400235467835\\
3.64799999999982	70.2259783414191\\
3.64999999999982	70.2119359446117\\
3.65199999999982	70.1978963557984\\
3.65399999999982	70.1838595744172\\
3.65599999999982	70.169825599905\\
3.65799999999982	70.1557944317012\\
3.65999999999982	70.1417660692452\\
3.66199999999982	70.1277405119717\\
3.66399999999982	70.1137177593206\\
3.66599999999982	70.0996978107341\\
3.66799999999982	70.0856806656453\\
3.66999999999982	70.0716663234983\\
3.67199999999982	70.0576547837262\\
3.67399999999982	70.0436460457752\\
3.67599999999982	70.0296401090788\\
3.67799999999982	70.0156369730786\\
3.67999999999982	70.0016366372146\\
3.68199999999982	69.9876391009248\\
3.68399999999982	69.9736443636481\\
3.68599999999982	69.9596524248272\\
3.68799999999982	69.9456632838994\\
3.68999999999982	69.9316769403057\\
3.69199999999981	69.9176933934869\\
3.69399999999981	69.9037126428816\\
3.69599999999981	69.8897346879319\\
3.69799999999981	69.8757595280779\\
3.69999999999981	69.8617871627595\\
3.70199999999981	69.8478175914169\\
3.70399999999981	69.8338508134922\\
3.70599999999981	69.8198868284252\\
3.70799999999981	69.805925635658\\
3.70999999999981	69.7919672346322\\
3.71199999999981	69.7780116247874\\
3.71399999999981	69.7640588055651\\
3.71599999999981	69.7501087764082\\
3.71799999999981	69.7361615367588\\
3.71999999999981	69.7222170860557\\
3.72199999999981	69.7082754237431\\
3.72399999999981	69.6943365492637\\
3.72599999999981	69.6804004620559\\
3.72799999999981	69.6664671615651\\
3.72999999999981	69.6525366472338\\
3.73199999999981	69.6386089185026\\
3.73399999999981	69.6246839748152\\
3.73599999999981	69.6107618156132\\
3.73799999999981	69.5968424403412\\
3.73999999999981	69.5829258484404\\
3.74199999999981	69.5690120393559\\
3.74399999999981	69.5551010125279\\
3.74599999999981	69.5411927674009\\
3.74799999999981	69.52728730342\\
3.74999999999981	69.5133846200264\\
3.75199999999981	69.4994847166626\\
3.75399999999981	69.4855875927761\\
3.75599999999981	69.4716932478099\\
3.75799999999981	69.4578016812042\\
3.75999999999981	69.4439128924058\\
3.76199999999981	69.4300268808611\\
3.76399999999981	69.416143646009\\
3.76599999999981	69.4022631872979\\
3.76799999999981	69.3883855041708\\
3.76999999999981	69.3745105960737\\
3.77199999999981	69.3606384624491\\
3.77399999999981	69.3467691027447\\
3.77599999999981	69.3329025164015\\
3.77799999999981	69.3190387028693\\
3.77999999999981	69.3051776615901\\
3.78199999999981	69.2913193920094\\
3.7839999999998	69.2774638935737\\
3.7859999999998	69.2636111657275\\
3.7879999999998	69.2497612079179\\
3.7899999999998	69.235914019589\\
3.7919999999998	69.2220696001872\\
3.7939999999998	69.2082279491589\\
3.7959999999998	69.1943890659501\\
3.7979999999998	69.1805529500074\\
3.7999999999998	69.1667196007753\\
3.8019999999998	69.1528890177018\\
3.8039999999998	69.1390612002334\\
3.8059999999998	69.1252361478157\\
3.8079999999998	69.1114138598985\\
3.8099999999998	69.0975943359243\\
3.8119999999998	69.083777575343\\
3.8139999999998	69.0699635775999\\
3.8159999999998	69.0561523421439\\
3.8179999999998	69.0423438684226\\
3.8199999999998	69.0285381558835\\
3.8219999999998	69.0147352039718\\
3.8239999999998	69.0009350121349\\
3.8259999999998	68.9871375798255\\
3.8279999999998	68.9733429064868\\
3.8299999999998	68.9595509915686\\
3.8319999999998	68.9457618345175\\
3.8339999999998	68.9319754347862\\
3.8359999999998	68.9181917918182\\
3.8379999999998	68.9044109050621\\
3.8399999999998	68.8906327739687\\
3.8419999999998	68.8768573979867\\
3.8439999999998	68.863084776564\\
3.8459999999998	68.8493149091506\\
3.8479999999998	68.8355477951949\\
3.8499999999998	68.8217834341437\\
3.8519999999998	68.8080218254499\\
3.8539999999998	68.79426296856\\
3.8559999999998	68.7805068629258\\
3.8579999999998	68.7667535079955\\
3.8599999999998	68.7530029032199\\
3.8619999999998	68.7392550480483\\
3.8639999999998	68.7255099419292\\
3.8659999999998	68.7117675843148\\
3.8679999999998	68.6980279746539\\
3.8699999999998	68.6842911123965\\
3.8719999999998	68.6705569969949\\
3.87399999999979	68.6568256278993\\
3.87599999999979	68.6430970045572\\
3.87799999999979	68.629371126422\\
3.87999999999979	68.6156479929446\\
3.88199999999979	68.6019276035757\\
3.88399999999979	68.5882099577655\\
3.88599999999979	68.5744950549651\\
3.88799999999979	68.5607828946262\\
3.88999999999979	68.5470734762018\\
3.89199999999979	68.5333667991397\\
3.89399999999979	68.5196628628956\\
3.89599999999979	68.5059616669198\\
3.89799999999979	68.4922632106614\\
3.89999999999979	68.4785674935762\\
3.90199999999979	68.4648745151148\\
3.90399999999979	68.4511842747278\\
3.90599999999979	68.4374967718691\\
3.90799999999979	68.4238120059918\\
3.90999999999979	68.4101299765482\\
3.91199999999979	68.3964506829874\\
3.91399999999979	68.3827741247671\\
3.91599999999979	68.369100301338\\
3.91799999999979	68.3554292121531\\
3.91999999999979	68.3417608566635\\
3.92199999999979	68.3280952343266\\
3.92399999999979	68.3144323445905\\
3.92599999999979	68.3007721869133\\
3.92799999999979	68.2871147607466\\
3.92999999999979	68.2734600655422\\
3.93199999999979	68.2598081007572\\
3.93399999999979	68.246158865844\\
3.93599999999979	68.2325123602551\\
3.93799999999979	68.2188685834465\\
3.93999999999979	68.205227534872\\
3.94199999999979	68.1915892139848\\
3.94399999999979	68.1779536202392\\
3.94599999999979	68.164320753092\\
3.94799999999979	68.1506906119961\\
3.94999999999979	68.1370631964064\\
3.95199999999979	68.1234385057779\\
3.95399999999979	68.1098165395645\\
3.95599999999979	68.096197297223\\
3.95799999999979	68.082580778207\\
3.95999999999979	68.0689669819724\\
3.96199999999979	68.0553559079736\\
3.96399999999979	68.0417475556685\\
3.96599999999978	68.0281419245131\\
3.96799999999978	68.0145390139571\\
3.96999999999978	68.0009388234629\\
3.97199999999978	67.9873413524853\\
3.97399999999978	67.9737466004772\\
3.97599999999978	67.960154566898\\
3.97799999999978	67.9465652512015\\
3.97999999999978	67.9329786528457\\
3.98199999999978	67.9193947712859\\
3.98399999999978	67.9058136059797\\
3.98599999999978	67.8922351563834\\
3.98799999999978	67.8786594219546\\
3.98999999999978	67.8650864021476\\
3.99199999999978	67.8515160964234\\
3.99399999999978	67.8379485042356\\
3.99599999999978	67.8243836250445\\
3.99799999999978	67.8108214583045\\
3.99999999999978	67.7972620034745\\
4.00199999999978	67.7837052600119\\
4.00399999999978	67.7701512273753\\
4.00599999999978	67.7565999050202\\
4.00799999999978	67.7430512924079\\
4.00999999999978	67.7295053889937\\
4.01199999999978	67.7159621942362\\
4.01399999999978	67.7024217075937\\
4.01599999999978	67.6888839285255\\
4.01799999999978	67.6753488564895\\
4.01999999999978	67.6618164909427\\
4.02199999999978	67.6482868313463\\
4.02399999999978	67.6347598771584\\
4.02599999999978	67.6212356278364\\
4.02799999999978	67.6077140828409\\
4.02999999999978	67.5941952416305\\
4.03199999999978	67.5806791036645\\
4.03399999999978	67.5671656684025\\
4.03599999999978	67.5536549353016\\
4.03799999999978	67.5401469038247\\
4.03999999999978	67.5266415734296\\
4.04199999999978	67.5131389435782\\
4.04399999999978	67.4996390137267\\
4.04599999999978	67.4861417833364\\
4.04799999999978	67.4726472518699\\
4.04999999999978	67.4591554187846\\
4.05199999999978	67.4456662835426\\
4.05399999999978	67.432179845604\\
4.05599999999978	67.4186961044273\\
4.05799999999978	67.4052150594754\\
4.05999999999977	67.3917367102092\\
4.06199999999977	67.3782610560857\\
4.06399999999977	67.3647880965726\\
4.06599999999977	67.3513178311242\\
4.06799999999977	67.3378502592064\\
4.06999999999977	67.3243853802778\\
4.07199999999977	67.3109231938014\\
4.07399999999977	67.2974636992387\\
4.07599999999977	67.2840068960484\\
4.07799999999977	67.2705527836961\\
4.07999999999977	67.2571013616416\\
4.08199999999977	67.2436526293468\\
4.08399999999977	67.2302065862757\\
4.08599999999977	67.2167632318871\\
4.08799999999977	67.2033225656466\\
4.08999999999977	67.1898845870143\\
4.09199999999977	67.1764492954539\\
4.09399999999977	67.1630166904271\\
4.09599999999977	67.1495867713981\\
4.09799999999977	67.136159537828\\
4.09999999999977	67.1227349891807\\
4.10199999999977	67.1093131249184\\
4.10399999999977	67.0958939445035\\
4.10599999999977	67.0824774474038\\
4.10799999999977	67.0690636330783\\
4.10999999999977	67.0556525009894\\
4.11199999999977	67.0422440506055\\
4.11399999999977	67.0288382813837\\
4.11599999999977	67.015435192796\\
4.11799999999977	67.0020347843001\\
4.11999999999977	66.9886370553617\\
4.12199999999977	66.9752420054431\\
4.12399999999977	66.9618496340129\\
4.12599999999977	66.948459940534\\
4.12799999999977	66.9350729244671\\
4.12999999999977	66.9216885852799\\
4.13199999999977	66.9083069224362\\
4.13399999999977	66.8949279354023\\
4.13599999999977	66.881551623641\\
4.13799999999977	66.8681779866191\\
4.13999999999977	66.8548070237998\\
4.14199999999977	66.8414387346494\\
4.14399999999977	66.8280731186328\\
4.14599999999977	66.8147101752167\\
4.14799999999977	66.8013499038644\\
4.14999999999976	66.7879923040409\\
4.15199999999976	66.7746373752154\\
4.15399999999976	66.7612851168515\\
4.15599999999976	66.7479355284163\\
4.15799999999976	66.7345886093744\\
4.15999999999976	66.721244359193\\
4.16199999999976	66.7079027773374\\
4.16399999999976	66.6945638632755\\
4.16599999999976	66.6812276164707\\
4.16799999999976	66.6678940363947\\
4.16999999999976	66.6545631225089\\
4.17199999999976	66.6412348742844\\
4.17399999999976	66.6279092911842\\
4.17599999999976	66.6145863726778\\
4.17799999999976	66.6012661182322\\
4.17999999999976	66.5879485273128\\
4.18199999999976	66.5746335993888\\
4.18399999999976	66.5613213339273\\
4.18599999999976	66.5480117303968\\
4.18799999999976	66.5347047882615\\
4.18999999999976	66.5214005069925\\
4.19199999999976	66.5080988860581\\
4.19399999999976	66.4947999249215\\
4.19599999999976	66.4815036230564\\
4.19799999999976	66.4682099799279\\
4.19999999999976	66.454918995006\\
4.20199999999976	66.441630667758\\
4.20399999999976	66.4283449976512\\
4.20599999999976	66.4150619841556\\
4.20799999999976	66.4017816267401\\
4.20999999999976	66.388503924873\\
4.21199999999976	66.3752288780235\\
4.21399999999976	66.3619564856623\\
4.21599999999976	66.348686747256\\
4.21799999999976	66.3354196622751\\
4.21999999999976	66.3221552301876\\
4.22199999999976	66.3088934504635\\
4.22399999999976	66.2956343225763\\
4.22599999999976	66.2823778459872\\
4.22799999999976	66.2691240201748\\
4.22999999999976	66.255872844603\\
4.23199999999976	66.2426243187457\\
4.23399999999976	66.2293784420706\\
4.23599999999976	66.2161352140474\\
4.23799999999976	66.2028946341484\\
4.23999999999976	66.1896567018437\\
4.24199999999975	66.1764214166018\\
4.24399999999975	66.1631887778976\\
4.24599999999975	66.1499587851978\\
4.24799999999975	66.1367314379742\\
4.24999999999975	66.1235067356992\\
4.25199999999975	66.1102846778429\\
4.25399999999975	66.0970652638752\\
4.25599999999975	66.0838484932686\\
4.25799999999975	66.0706343654956\\
4.25999999999975	66.057422880025\\
4.26199999999975	66.0442140363307\\
4.26399999999975	66.0310078338833\\
4.26599999999975	66.0178042721549\\
4.26799999999975	66.0046033506182\\
4.26999999999975	65.9914050687447\\
4.27199999999975	65.9782094260061\\
4.27399999999975	65.9650164218734\\
4.27599999999975	65.9518260558206\\
4.27799999999975	65.9386383273215\\
4.27999999999975	65.9254532358466\\
4.28199999999975	65.9122707808694\\
4.28399999999975	65.8990909618592\\
4.28599999999975	65.8859137782945\\
4.28799999999975	65.8727392296452\\
4.28999999999975	65.8595673153837\\
4.29199999999975	65.8463980349849\\
4.29399999999975	65.8332313879231\\
4.29599999999975	65.8200673736686\\
4.29799999999975	65.806905991698\\
4.29999999999975	65.7937472414818\\
4.30199999999975	65.7805911224957\\
4.30399999999975	65.7674376342107\\
4.30599999999975	65.754286776106\\
4.30799999999975	65.7411385476514\\
4.30999999999975	65.7279929483223\\
4.31199999999975	65.7148499775927\\
4.31399999999975	65.7017096349381\\
4.31599999999975	65.6885719198323\\
4.31799999999975	65.6754368317486\\
4.31999999999975	65.6623043701624\\
4.32199999999975	65.6491745345496\\
4.32399999999975	65.6360473243836\\
4.32599999999975	65.6229227391393\\
4.32799999999975	65.6098007782942\\
4.32999999999975	65.5966814413213\\
4.33199999999974	65.5835647276944\\
4.33399999999974	65.5704506368934\\
4.33599999999974	65.5573391683895\\
4.33799999999974	65.5442303216594\\
4.33999999999974	65.5311240961815\\
4.34199999999974	65.5180204914278\\
4.34399999999974	65.5049195068768\\
4.34599999999974	65.4918211420036\\
4.34799999999974	65.4787253962846\\
4.34999999999974	65.465632269195\\
4.35199999999974	65.4525417602121\\
4.35399999999974	65.4394538688117\\
4.35599999999974	65.4263685944729\\
4.35799999999974	65.4132859366682\\
4.35999999999974	65.4002058948783\\
4.36199999999974	65.3871284685774\\
4.36399999999974	65.3740536572434\\
4.36599999999974	65.3609814603546\\
4.36799999999974	65.3479118773849\\
4.36999999999974	65.3348449078138\\
4.37199999999974	65.321780551118\\
4.37399999999974	65.308718806776\\
4.37599999999974	65.2956596742672\\
4.37799999999974	65.2826031530643\\
4.37999999999974	65.2695492426491\\
4.38199999999974	65.2564979424964\\
4.38399999999974	65.2434492520868\\
4.38599999999974	65.230403170899\\
4.38799999999974	65.2173596984086\\
4.38999999999974	65.2043188340953\\
4.39199999999974	65.1912805774372\\
4.39399999999974	65.1782449279142\\
4.39599999999974	65.1652118850029\\
4.39799999999974	65.1521814481821\\
4.39999999999974	65.1391536169339\\
4.40199999999974	65.1261283907334\\
4.40399999999974	65.1131057690618\\
4.40599999999974	65.1000857513963\\
4.40799999999974	65.0870683372196\\
4.40999999999974	65.0740535260068\\
4.41199999999974	65.0610413172405\\
4.41399999999974	65.0480317103985\\
4.41599999999974	65.0350247049641\\
4.41799999999974	65.022020300411\\
4.41999999999974	65.0090184962237\\
4.42199999999974	64.996019291882\\
4.42399999999973	64.9830226868622\\
4.42599999999973	64.9700286806507\\
4.42799999999973	64.9570372727208\\
4.42999999999973	64.9440484625589\\
4.43199999999973	64.9310622496431\\
4.43399999999973	64.9180786334533\\
4.43599999999973	64.9050976134715\\
4.43799999999973	64.8921191891763\\
4.43999999999973	64.8791433600516\\
4.44199999999973	64.8661701255773\\
4.44399999999973	64.853199485235\\
4.44599999999973	64.840231438505\\
4.44799999999973	64.8272659848683\\
4.44999999999973	64.8143031238081\\
4.45199999999973	64.8013428548043\\
4.45399999999973	64.7883851773385\\
4.45599999999973	64.7754300908934\\
4.45799999999973	64.7624775949508\\
4.45999999999973	64.7495276889935\\
4.46199999999973	64.7365803725022\\
4.46399999999973	64.7236356449593\\
4.46599999999973	64.7106935058487\\
4.46799999999973	64.697753954649\\
4.46999999999973	64.6848169908468\\
4.47199999999973	64.6718826139231\\
4.47399999999973	64.6589508233599\\
4.47599999999973	64.6460216186406\\
4.47799999999973	64.6330949992472\\
4.47999999999973	64.6201709646641\\
4.48199999999973	64.6072495143755\\
4.48399999999973	64.5943306478621\\
4.48599999999973	64.581414364607\\
4.48799999999973	64.5685006640967\\
4.48999999999973	64.5555895458112\\
4.49199999999973	64.542681009237\\
4.49399999999973	64.5297750538566\\
4.49599999999973	64.5168716791539\\
4.49799999999973	64.5039708846125\\
4.49999999999973	64.4910726697172\\
4.50199999999973	64.4781770339507\\
4.50399999999973	64.4652839767994\\
4.50599999999973	64.4523934977465\\
4.50799999999973	64.4395055962755\\
4.50999999999973	64.4266202718734\\
4.51199999999973	64.4137375240234\\
4.51399999999972	64.4008573522092\\
4.51599999999972	64.3879797559171\\
4.51799999999972	64.3751047346315\\
4.51999999999972	64.3622322878374\\
4.52199999999972	64.3493624150219\\
4.52399999999972	64.3364951156682\\
4.52599999999972	64.3236303892622\\
4.52799999999972	64.3107682352886\\
4.52999999999972	64.2979086532352\\
4.53199999999972	64.2850516425858\\
4.53399999999972	64.2721972028266\\
4.53599999999972	64.259345333443\\
4.53799999999972	64.2464960339237\\
4.53999999999972	64.2336493037515\\
4.54199999999972	64.2208051424137\\
4.54399999999972	64.2079635493979\\
4.54599999999972	64.1951245241866\\
4.54799999999972	64.1822880662714\\
4.54999999999972	64.1694541751353\\
4.55199999999972	64.1566228502681\\
4.55399999999972	64.1437940911544\\
4.55599999999972	64.1309678972811\\
4.55799999999972	64.1181442681336\\
4.55999999999972	64.1053232032062\\
4.56199999999972	64.0925047019774\\
4.56399999999972	64.0796887639406\\
4.56599999999972	64.0668753885803\\
4.56799999999972	64.0540645753852\\
4.56999999999972	64.0412563238411\\
4.57199999999972	64.0284506334366\\
4.57399999999972	64.0156475036617\\
4.57599999999972	64.0028469340027\\
4.57799999999972	63.9900489239475\\
4.57999999999972	63.9772534729837\\
4.58199999999972	63.9644605806003\\
4.58399999999972	63.9516702462871\\
4.58599999999972	63.9388824695296\\
4.58799999999972	63.9260972498203\\
4.58999999999972	63.913314586644\\
4.59199999999972	63.9005344794911\\
4.59399999999972	63.8877569278512\\
4.59599999999972	63.8749819312115\\
4.59799999999972	63.8622094890627\\
4.59999999999972	63.8494396008943\\
4.60199999999972	63.8366722661917\\
4.60399999999971	63.8239074844492\\
4.60599999999971	63.8111452551548\\
4.60799999999971	63.7983855777966\\
4.60999999999971	63.7856284518665\\
4.61199999999971	63.7728738768518\\
4.61399999999971	63.7601218522449\\
4.61599999999971	63.7473723775352\\
4.61799999999971	63.73462545221\\
4.61999999999971	63.7218810757648\\
4.62199999999971	63.7091392476839\\
4.62399999999971	63.6963999674624\\
4.62599999999971	63.6836632345888\\
4.62799999999971	63.6709290485548\\
4.62999999999971	63.6581974088498\\
4.63199999999971	63.6454683149644\\
4.63399999999971	63.6327417663902\\
4.63599999999971	63.6200177626211\\
4.63799999999971	63.6072963031437\\
4.63999999999971	63.5945773874504\\
4.64199999999971	63.5818610150347\\
4.64399999999971	63.569147185386\\
4.64599999999971	63.5564358979954\\
4.64799999999971	63.5437271523574\\
4.64999999999971	63.5310209479589\\
4.65199999999971	63.5183172842968\\
4.65399999999971	63.50561616086\\
4.65599999999971	63.4929175771412\\
4.65799999999971	63.4802215326344\\
4.65999999999971	63.4675280268277\\
4.66199999999971	63.4548370592173\\
4.66399999999971	63.4421486292949\\
4.66599999999971	63.4294627365533\\
4.66799999999971	63.4167793804829\\
4.66999999999971	63.4040985605771\\
4.67199999999971	63.3914202763317\\
4.67399999999971	63.3787445272372\\
4.67599999999971	63.3660713127855\\
4.67799999999971	63.3534006324721\\
4.67999999999971	63.3407324857896\\
4.68199999999971	63.3280668722308\\
4.68399999999971	63.3154037912901\\
4.68599999999971	63.3027432424599\\
4.68799999999971	63.2900852252343\\
4.68999999999971	63.2774297391076\\
4.69199999999971	63.2647767835737\\
4.69399999999971	63.2521263581248\\
4.6959999999997	63.2394784622566\\
4.6979999999997	63.2268330954635\\
4.6999999999997	63.2141902572403\\
4.7019999999997	63.2015499470782\\
4.7039999999997	63.1889121644763\\
4.7059999999997	63.1762769089262\\
4.7079999999997	63.1636441799227\\
4.7099999999997	63.1510139769601\\
4.7119999999997	63.1383862995359\\
4.7139999999997	63.1257611471427\\
4.7159999999997	63.1131385192767\\
4.7179999999997	63.1005184154315\\
4.7199999999997	63.0879008351034\\
4.7219999999997	63.075285777789\\
4.7239999999997	63.0626732429824\\
4.7259999999997	63.0500632301805\\
4.7279999999997	63.0374557388765\\
4.7299999999997	63.0248507685684\\
4.7319999999997	63.0122483187505\\
4.7339999999997	62.9996483889209\\
4.7359999999997	62.9870509785744\\
4.7379999999997	62.9744560872066\\
4.7399999999997	62.9618637143149\\
4.7419999999997	62.9492738593944\\
4.7439999999997	62.9366865219433\\
4.7459999999997	62.9241017014578\\
4.7479999999997	62.9115193974341\\
4.7499999999997	62.898939609368\\
4.7519999999997	62.8863623367579\\
4.7539999999997	62.8737875791017\\
4.7559999999997	62.8612153358929\\
4.7579999999997	62.8486456066318\\
4.7599999999997	62.836078390815\\
4.7619999999997	62.8235136879416\\
4.7639999999997	62.8109514975044\\
4.7659999999997	62.798391819005\\
4.7679999999997	62.785834651939\\
4.7699999999997	62.7732799958078\\
4.7719999999997	62.7607278501055\\
4.7739999999997	62.74817821433\\
4.7759999999997	62.7356310879827\\
4.7779999999997	62.7230864705582\\
4.7799999999997	62.7105443615575\\
4.7819999999997	62.6980047604764\\
4.7839999999997	62.6854676668174\\
4.78599999999969	62.6729330800755\\
4.78799999999969	62.6604009997498\\
4.78999999999969	62.64787142534\\
4.79199999999969	62.635344356347\\
4.79399999999969	62.6228197922658\\
4.79599999999969	62.6102977325969\\
4.79799999999969	62.5977781768418\\
4.79999999999969	62.5852611244969\\
4.80199999999969	62.5727465750639\\
4.80399999999969	62.5602345280397\\
4.80599999999969	62.547724982926\\
4.80799999999969	62.5352179392218\\
4.80999999999969	62.5227133964269\\
4.81199999999969	62.5102113540414\\
4.81399999999969	62.4977118115655\\
4.81599999999969	62.4852147684992\\
4.81799999999969	62.4727202243415\\
4.81999999999969	62.4602281785942\\
4.82199999999969	62.4477386307588\\
4.82399999999969	62.4352515803323\\
4.82599999999969	62.422767026818\\
4.82799999999969	62.4102849697165\\
4.82999999999969	62.3978054085264\\
4.83199999999969	62.3853283427522\\
4.83399999999969	62.3728537718914\\
4.83599999999969	62.3603816954475\\
4.83799999999969	62.3479121129201\\
4.83999999999969	62.335445023813\\
4.84199999999969	62.3229804276235\\
4.84399999999969	62.310518323857\\
4.84599999999969	62.2980587120136\\
4.84799999999969	62.2856015915949\\
4.84999999999969	62.2731469621024\\
4.85199999999969	62.2606948230383\\
4.85399999999969	62.2482451739061\\
4.85599999999969	62.2357980142054\\
4.85799999999969	62.2233533434403\\
4.85999999999969	62.210911161109\\
4.86199999999969	62.1984714667217\\
4.86399999999969	62.1860342597742\\
4.86599999999969	62.1735995397727\\
4.86799999999969	62.1611673062184\\
4.86999999999969	62.1487375586143\\
4.87199999999969	62.1363102964619\\
4.87399999999969	62.1238855192668\\
4.87599999999969	62.1114632265305\\
4.87799999999968	62.0990434177569\\
4.87999999999968	62.0866260924486\\
4.88199999999968	62.0742112501104\\
4.88399999999968	62.0617988902433\\
4.88599999999968	62.0493890123537\\
4.88799999999968	62.0369816159426\\
4.88999999999968	62.0245767005166\\
4.89199999999968	62.0121742655778\\
4.89399999999968	61.9997743106306\\
4.89599999999968	61.9873768351788\\
4.89799999999968	61.9749818387269\\
4.89999999999968	61.9625893207798\\
4.90199999999968	61.950199280842\\
4.90399999999968	61.9378117184152\\
4.90599999999968	61.9254266330058\\
4.90799999999968	61.9130440241206\\
4.90999999999968	61.9006638912622\\
4.91199999999968	61.8882862339357\\
4.91399999999968	61.8759110516461\\
4.91599999999968	61.8635383438984\\
4.91799999999968	61.8511681101991\\
4.91999999999968	61.8388003500513\\
4.92199999999968	61.8264350629619\\
4.92399999999968	61.8140722484356\\
4.92599999999968	61.8017119059789\\
4.92799999999968	61.7893540350954\\
4.92999999999968	61.7769986352935\\
4.93199999999968	61.7646457060789\\
4.93399999999968	61.7522952469558\\
4.93599999999968	61.7399472574322\\
4.93799999999968	61.7276017370109\\
4.93999999999968	61.7152586852019\\
4.94199999999968	61.70291810151\\
4.94399999999968	61.6905799854417\\
4.94599999999968	61.6782443365039\\
4.94799999999968	61.6659111542027\\
4.94999999999968	61.6535804380461\\
4.95199999999968	61.6412521875385\\
4.95399999999968	61.6289264021909\\
4.95599999999968	61.6166030815049\\
4.95799999999968	61.6042822249928\\
4.95999999999968	61.591963832159\\
4.96199999999968	61.579647902512\\
4.96399999999968	61.5673344355598\\
4.96599999999968	61.5550234308073\\
4.96799999999967	61.5427148877649\\
4.96999999999967	61.5304088059376\\
4.97199999999967	61.5181051848368\\
4.97399999999967	61.5058040239684\\
4.97599999999967	61.4935053228403\\
4.97799999999967	61.4812090809614\\
4.97999999999967	61.4689152978395\\
4.98199999999967	61.4566239729824\\
4.98399999999967	61.4443351058999\\
4.98599999999967	61.4320486960993\\
4.98799999999967	61.4197647430894\\
4.98999999999967	61.4074832463789\\
4.99199999999967	61.3952042054792\\
4.99399999999967	61.3829276198957\\
4.99599999999967	61.370653489139\\
4.99799999999967	61.3583818127186\\
4.99999999999967	61.3461125901399\\
5.00199999999967	61.3338458209188\\
5.00399999999967	61.3215815045611\\
5.00599999999967	61.3093196405757\\
5.00799999999967	61.2970602284736\\
5.00999999999967	61.2848032677635\\
5.01199999999967	61.2725487579562\\
5.01399999999967	61.2602966985611\\
5.01599999999967	61.2480470890889\\
5.01799999999967	61.2357999290495\\
5.01999999999967	61.2235552179514\\
5.02199999999967	61.2113129553067\\
5.02399999999967	61.1990731406245\\
5.02599999999967	61.1868357734167\\
5.02799999999967	61.1746008531933\\
5.02999999999967	61.162368379466\\
5.03199999999967	61.1501383517426\\
5.03399999999967	61.1379107695383\\
5.03599999999967	61.1256856323599\\
5.03799999999967	61.1134629397218\\
5.03999999999967	61.1012426911332\\
5.04199999999967	61.0890248861045\\
5.04399999999967	61.0768095241498\\
5.04599999999967	61.0645966047804\\
5.04799999999967	61.0523861275038\\
5.04999999999967	61.0401780918391\\
5.05199999999967	61.0279724972916\\
5.05399999999967	61.0157693433737\\
5.05599999999967	61.0035686296013\\
5.05799999999966	60.9913703554821\\
5.05999999999966	60.9791745205302\\
5.06199999999966	60.9669811242601\\
5.06399999999966	60.9547901661804\\
5.06599999999966	60.9426016458049\\
5.06799999999966	60.9304155626469\\
5.06999999999966	60.9182319162183\\
5.07199999999966	60.9060507060318\\
5.07399999999966	60.8938719316006\\
5.07599999999966	60.8816955924384\\
5.07799999999966	60.8695216880573\\
5.07999999999966	60.8573502179686\\
5.08199999999966	60.8451811816888\\
5.08399999999966	60.8330145787302\\
5.08599999999966	60.8208504086052\\
5.08799999999966	60.8086886708286\\
5.08999999999966	60.7965293649134\\
5.09199999999966	60.7843724903741\\
5.09399999999966	60.7722180467235\\
5.09599999999966	60.7600660334745\\
5.09799999999966	60.7479164501436\\
5.09999999999966	60.735769296243\\
5.10199999999966	60.72362457129\\
5.10399999999966	60.7114822747938\\
5.10599999999966	60.6993424062727\\
5.10799999999966	60.6872049652399\\
5.10999999999966	60.6750699512095\\
5.11199999999966	60.6629373636987\\
5.11399999999966	60.6508072022198\\
5.11599999999966	60.6386794662876\\
5.11799999999966	60.6265541554177\\
5.11999999999966	60.6144312691266\\
5.12199999999966	60.6023108069265\\
5.12399999999966	60.5901927683362\\
5.12599999999966	60.5780771528683\\
5.12799999999966	60.5659639600397\\
5.12999999999966	60.5538531893648\\
5.13199999999966	60.5417448403607\\
5.13399999999966	60.5296389125416\\
5.13599999999966	60.5175354054257\\
5.13799999999966	60.5054343185259\\
5.13999999999966	60.4933356513594\\
5.14199999999966	60.4812394034451\\
5.14399999999966	60.469145574295\\
5.14599999999966	60.457054163427\\
5.14799999999966	60.4449651703589\\
5.14999999999965	60.4328785946055\\
5.15199999999965	60.4207944356849\\
5.15399999999965	60.4087126931123\\
5.15599999999965	60.3966333664052\\
5.15799999999965	60.3845564550807\\
5.15999999999965	60.3724819586562\\
5.16199999999965	60.3604098766477\\
5.16399999999965	60.3483402085737\\
5.16599999999965	60.3362729539512\\
5.16799999999965	60.324208112297\\
5.16999999999965	60.3121456831285\\
5.17199999999965	60.3000856659628\\
5.17399999999965	60.2880280603195\\
5.17599999999965	60.2759728657152\\
5.17799999999965	60.2639200816678\\
5.17999999999965	60.2518697076961\\
5.18199999999965	60.2398217433162\\
5.18399999999965	60.2277761880496\\
5.18599999999965	60.2157330414108\\
5.18799999999965	60.2036923029196\\
5.18999999999965	60.1916539720959\\
5.19199999999965	60.1796180484565\\
5.19399999999965	60.1675845315217\\
5.19599999999965	60.1555534208074\\
5.19799999999965	60.143524715836\\
5.19999999999965	60.1314984161238\\
5.20199999999965	60.1194745211915\\
5.20399999999965	60.1074530305567\\
5.20599999999965	60.0954339437403\\
5.20799999999965	60.0834172602608\\
5.20999999999965	60.0714029796371\\
5.21199999999965	60.0593911013881\\
5.21399999999965	60.0473816250372\\
5.21599999999965	60.0353745501\\
5.21799999999965	60.0233698760975\\
5.21999999999965	60.0113676025502\\
5.22199999999965	59.999367728978\\
5.22399999999965	59.9873702549006\\
5.22599999999965	59.9753751798384\\
5.22799999999965	59.9633825033117\\
5.22999999999965	59.9513922248415\\
5.23199999999965	59.939404343947\\
5.23399999999965	59.9274188601488\\
5.23599999999965	59.9154357729687\\
5.23799999999965	59.9034550819279\\
5.23999999999964	59.891476786546\\
5.24199999999964	59.879500886345\\
5.24399999999964	59.8675273808418\\
5.24599999999964	59.8555562695647\\
5.24799999999964	59.84358755203\\
5.24999999999964	59.8316212277601\\
5.25199999999964	59.8196572962779\\
5.25399999999964	59.8076957571021\\
5.25599999999964	59.7957366097577\\
5.25799999999964	59.783779853763\\
5.25999999999964	59.7718254886419\\
5.26199999999964	59.7598735139164\\
5.26399999999964	59.747923929107\\
5.26599999999964	59.7359767337379\\
5.26799999999964	59.72403192733\\
5.26999999999964	59.7120895094046\\
5.27199999999964	59.7001494794877\\
5.27399999999964	59.6882118370987\\
5.27599999999964	59.67627658176\\
5.27799999999964	59.6643437129948\\
5.27999999999964	59.6524132303273\\
5.28199999999964	59.6404851332796\\
5.28399999999964	59.6285594213739\\
5.28599999999964	59.616636094134\\
5.28799999999964	59.6047151510834\\
5.28999999999964	59.5927965917426\\
5.29199999999964	59.5808804156394\\
5.29399999999964	59.5689666222935\\
5.29599999999964	59.5570552112314\\
5.29799999999964	59.5451461819729\\
5.29999999999964	59.5332395340479\\
5.30199999999964	59.5213352669742\\
5.30399999999964	59.5094333802787\\
5.30599999999964	59.4975338734834\\
5.30799999999964	59.4856367461143\\
5.30999999999964	59.473741997695\\
5.31199999999964	59.4618496277507\\
5.31399999999964	59.4499596358037\\
5.31599999999964	59.4380720213812\\
5.31799999999964	59.426186784005\\
5.31999999999964	59.414303923201\\
5.32199999999964	59.4024234384952\\
5.32399999999964	59.3905453294112\\
5.32599999999964	59.3786695954725\\
5.32799999999964	59.3667962362086\\
5.32999999999964	59.3549252511393\\
5.33199999999963	59.3430566397931\\
5.33399999999963	59.3311904016953\\
5.33599999999963	59.3193265363704\\
5.33799999999963	59.3074650433438\\
5.33999999999963	59.2956059221431\\
5.34199999999963	59.2837491722911\\
5.34399999999963	59.2718947933143\\
5.34599999999963	59.2600427847406\\
5.34799999999963	59.2481931460931\\
5.34999999999963	59.2363458769008\\
5.35199999999963	59.2245009766888\\
5.35399999999963	59.2126584449824\\
5.35599999999963	59.2008182813094\\
5.35799999999963	59.1889804851971\\
5.35999999999963	59.1771450561682\\
5.36199999999963	59.165311993752\\
5.36399999999963	59.1534812974787\\
5.36599999999963	59.1416529668687\\
5.36799999999963	59.1298270014516\\
5.36999999999963	59.1180034007562\\
5.37199999999963	59.1061821643078\\
5.37399999999963	59.0943632916347\\
5.37599999999963	59.0825467822629\\
5.37799999999963	59.0707326357189\\
5.37999999999963	59.0589208515356\\
5.38199999999963	59.0471114292326\\
5.38399999999963	59.035304368344\\
5.38599999999963	59.0234996683961\\
5.38799999999963	59.0116973289144\\
5.38999999999963	58.9998973494301\\
5.39199999999963	58.9880997294693\\
5.39399999999963	58.9763044685622\\
5.39599999999963	58.964511566233\\
5.39799999999963	58.9527210220136\\
5.39999999999963	58.9409328354327\\
5.40199999999963	58.9291470060173\\
5.40399999999963	58.9173635332961\\
5.40599999999963	58.9055824168001\\
5.40799999999963	58.8938036560552\\
5.40999999999963	58.8820272505904\\
5.41199999999963	58.8702531999382\\
5.41399999999963	58.8584815036218\\
5.41599999999963	58.8467121611771\\
5.41799999999963	58.8349451721292\\
5.41999999999963	58.8231805360094\\
5.42199999999962	58.8114182523446\\
5.42399999999962	58.7996583206677\\
5.42599999999962	58.7879007405065\\
5.42799999999962	58.776145511392\\
5.42999999999962	58.7643926328512\\
5.43199999999962	58.7526421044169\\
5.43399999999962	58.7408939256194\\
5.43599999999962	58.7291480959879\\
5.43799999999962	58.7174046150526\\
5.43999999999962	58.7056634823431\\
5.44199999999962	58.6939246973911\\
5.44399999999962	58.6821882597267\\
5.44599999999962	58.6704541688823\\
5.44799999999962	58.6587224243842\\
5.44999999999962	58.6469930257685\\
5.45199999999962	58.6352659725617\\
5.45399999999962	58.6235412642959\\
5.45599999999962	58.6118189005054\\
5.45799999999962	58.6000988807179\\
5.45999999999962	58.5883812044639\\
5.46199999999962	58.576665871279\\
5.46399999999962	58.5649528806906\\
5.46599999999962	58.5532422322339\\
5.46799999999962	58.5415339254365\\
5.46999999999962	58.5298279598331\\
5.47199999999962	58.5181243349559\\
5.47399999999962	58.506423050334\\
5.47599999999962	58.4947241055025\\
5.47799999999962	58.4830274999894\\
5.47999999999962	58.4713332333336\\
5.48199999999962	58.4596413050615\\
5.48399999999962	58.4479517147087\\
5.48599999999962	58.4362644618045\\
5.48799999999962	58.4245795458851\\
5.48999999999962	58.4128969664821\\
5.49199999999962	58.4012167231264\\
5.49399999999962	58.3895388153527\\
5.49599999999962	58.3778632426962\\
5.49799999999962	58.3661900046836\\
5.49999999999962	58.3545191008527\\
5.50199999999962	58.3428505307363\\
5.50399999999962	58.3311842938686\\
5.50599999999962	58.3195203897808\\
5.50799999999962	58.3078588180075\\
5.50999999999962	58.2961995780821\\
5.51199999999961	58.2845426695383\\
5.51399999999961	58.2728880919098\\
5.51599999999961	58.2612358447308\\
5.51799999999961	58.2495859275367\\
5.51999999999961	58.2379383398582\\
5.52199999999961	58.2262930812324\\
5.52399999999961	58.2146501511921\\
5.52599999999961	58.2030095492744\\
5.52799999999961	58.1913712750098\\
5.52999999999961	58.1797353279348\\
5.53199999999961	58.1681017075854\\
5.53399999999961	58.1564704134919\\
5.53599999999961	58.1448414451943\\
5.53799999999961	58.1332148022248\\
5.53999999999961	58.1215904841195\\
5.54199999999961	58.1099684904131\\
5.54399999999961	58.0983488206402\\
5.54599999999961	58.0867314743369\\
5.54799999999961	58.0751164510384\\
5.54999999999961	58.0635037502781\\
5.55199999999961	58.0518933715966\\
5.55399999999961	58.0402853145252\\
5.55599999999961	58.0286795786005\\
5.55799999999961	58.0170761633597\\
5.55999999999961	58.0054750683379\\
5.56199999999961	57.99387629307\\
5.56399999999961	57.9822798370958\\
5.56599999999961	57.9706856999475\\
5.56799999999961	57.9590938811642\\
5.56999999999961	57.9475043802797\\
5.57199999999961	57.935917196832\\
5.57399999999961	57.9243323303584\\
5.57599999999961	57.9127497803939\\
5.57799999999961	57.9011695464772\\
5.57999999999961	57.8895916281439\\
5.58199999999961	57.8780160249329\\
5.58399999999961	57.8664427363762\\
5.58599999999961	57.8548717620148\\
5.58799999999961	57.8433031013895\\
5.58999999999961	57.83173675403\\
5.59199999999961	57.8201727194794\\
5.59399999999961	57.8086109972729\\
5.59599999999961	57.7970515869466\\
5.59799999999961	57.7854944880422\\
5.59999999999961	57.7739397000952\\
5.60199999999961	57.7623872226436\\
5.6039999999996	57.7508370552251\\
5.6059999999996	57.7392891973793\\
5.6079999999996	57.7277436486428\\
5.6099999999996	57.7162004085533\\
5.6119999999996	57.7046594766512\\
5.6139999999996	57.6931208524732\\
5.6159999999996	57.6815845355602\\
5.6179999999996	57.6700505254476\\
5.6199999999996	57.6585188216763\\
5.6219999999996	57.6469894237862\\
5.6239999999996	57.6354623313124\\
5.6259999999996	57.6239375437982\\
5.6279999999996	57.6124150607786\\
5.6299999999996	57.6008948817962\\
5.6319999999996	57.5893770063885\\
5.6339999999996	57.5778614340945\\
5.6359999999996	57.5663481644551\\
5.6379999999996	57.55483719701\\
5.6399999999996	57.5433285312976\\
5.6419999999996	57.5318221668579\\
5.6439999999996	57.5203181032304\\
5.6459999999996	57.5088163399566\\
5.6479999999996	57.4973168765745\\
5.6499999999996	57.4858197126263\\
5.6519999999996	57.4743248476513\\
5.6539999999996	57.4628322811884\\
5.6559999999996	57.4513420127798\\
5.6579999999996	57.4398540419656\\
5.6599999999996	57.428368368286\\
5.6619999999996	57.4168849912811\\
5.6639999999996	57.4054039104946\\
5.6659999999996	57.3939251254634\\
5.6679999999996	57.3824486357309\\
5.6699999999996	57.370974440836\\
5.6719999999996	57.3595025403224\\
5.6739999999996	57.3480329337314\\
5.6759999999996	57.3365656206009\\
5.6779999999996	57.3251006004753\\
5.6799999999996	57.3136378728951\\
5.6819999999996	57.3021774374026\\
5.6839999999996	57.2907192935376\\
5.6859999999996	57.279263440845\\
5.6879999999996	57.2678098788634\\
5.6899999999996	57.2563586071375\\
5.6919999999996	57.2449096252064\\
5.69399999999959	57.2334629326141\\
5.69599999999959	57.2220185289031\\
5.69799999999959	57.2105764136158\\
5.69999999999959	57.199136586293\\
5.70199999999959	57.187699046478\\
5.70399999999959	57.1762637937147\\
5.70599999999959	57.1648308275442\\
5.70799999999959	57.1534001475087\\
5.70999999999959	57.1419717531543\\
5.71199999999959	57.1305456440206\\
5.71399999999959	57.1191218196526\\
5.71599999999959	57.10770027959\\
5.71799999999959	57.0962810233816\\
5.71999999999959	57.0848640505662\\
5.72199999999959	57.0734493606894\\
5.72399999999959	57.0620369532942\\
5.72599999999959	57.0506268279239\\
5.72799999999959	57.0392189841222\\
5.72999999999959	57.0278134214332\\
5.73199999999959	57.0164101394004\\
5.73399999999959	57.0050091375679\\
5.73599999999959	56.9936104154807\\
5.73799999999959	56.9822139726816\\
5.73999999999959	56.9708198087154\\
5.74199999999959	56.9594279231268\\
5.74399999999959	56.9480383154592\\
5.74599999999959	56.9366509852582\\
5.74799999999959	56.9252659320672\\
5.74999999999959	56.913883155431\\
5.75199999999959	56.9025026548966\\
5.75399999999959	56.891124430006\\
5.75599999999959	56.8797484803054\\
5.75799999999959	56.8683748053398\\
5.75999999999959	56.8570034046554\\
5.76199999999959	56.8456342777956\\
5.76399999999959	56.8342674243065\\
5.76599999999959	56.822902843733\\
5.76799999999959	56.8115405356226\\
5.76999999999959	56.8001804995181\\
5.77199999999959	56.7888227349669\\
5.77399999999959	56.7774672415144\\
5.77599999999959	56.7661140187067\\
5.77799999999959	56.7547630660893\\
5.77999999999959	56.743414383209\\
5.78199999999959	56.7320679696111\\
5.78399999999959	56.7207238248426\\
5.78599999999958	56.7093819484487\\
5.78799999999958	56.6980423399766\\
5.78999999999958	56.6867049989734\\
5.79199999999958	56.6753699249849\\
5.79399999999958	56.6640371175575\\
5.79599999999958	56.652706576239\\
5.79799999999958	56.6413783005752\\
5.79999999999958	56.6300522901125\\
5.80199999999958	56.6187285444004\\
5.80399999999958	56.6074070629848\\
5.80599999999958	56.5960878454118\\
5.80799999999958	56.5847708912306\\
5.80999999999958	56.5734561999864\\
5.81199999999958	56.5621437712284\\
5.81399999999958	56.5508336045037\\
5.81599999999958	56.5395256993611\\
5.81799999999958	56.5282200553463\\
5.81999999999958	56.5169166720091\\
5.82199999999958	56.5056155488946\\
5.82399999999958	56.4943166855551\\
5.82599999999958	56.4830200815333\\
5.82799999999958	56.4717257363833\\
5.82999999999958	56.4604336496509\\
5.83199999999958	56.4491438208833\\
5.83399999999958	56.4378562496297\\
5.83599999999958	56.4265709354393\\
5.83799999999958	56.4152878778617\\
5.83999999999958	56.4040070764423\\
5.84199999999958	56.392728530733\\
5.84399999999958	56.3814522402823\\
5.84599999999958	56.3701782046375\\
5.84799999999958	56.3589064233503\\
5.84999999999958	56.3476368959678\\
5.85199999999958	56.3363696220402\\
5.85399999999958	56.3251046011171\\
5.85599999999958	56.3138418327476\\
5.85799999999958	56.3025813164813\\
5.85999999999958	56.2913230518668\\
5.86199999999958	56.280067038457\\
5.86399999999958	56.2688132757992\\
5.86599999999958	56.2575617634431\\
5.86799999999958	56.2463125009407\\
5.86999999999958	56.2350654878398\\
5.87199999999958	56.2238207236922\\
5.87399999999958	56.2125782080479\\
5.87599999999957	56.2013379404569\\
5.87799999999957	56.1900999204697\\
5.87999999999957	56.1788641476375\\
5.88199999999957	56.1676306215106\\
5.88399999999957	56.1563993416387\\
5.88599999999957	56.1451703075741\\
5.88799999999957	56.133943518868\\
5.88999999999957	56.1227189750699\\
5.89199999999957	56.1114966757325\\
5.89399999999957	56.1002766204059\\
5.89599999999957	56.0890588086412\\
5.89799999999957	56.0778432399903\\
5.89999999999957	56.0666299140049\\
5.90199999999957	56.0554188302358\\
5.90399999999957	56.0442099882367\\
5.90599999999957	56.0330033875561\\
5.90799999999957	56.0217990277482\\
5.90999999999957	56.0105969083636\\
5.91199999999957	55.999397028955\\
5.91399999999957	55.9881993890746\\
5.91599999999957	55.9770039882747\\
5.91799999999957	55.9658108261075\\
5.91999999999957	55.9546199021236\\
5.92199999999957	55.9434312158774\\
5.92399999999957	55.9322447669231\\
5.92599999999957	55.9210605548092\\
5.92799999999957	55.9098785790904\\
5.92999999999957	55.8986988393215\\
5.93199999999957	55.8875213350512\\
5.93399999999957	55.8763460658363\\
5.93599999999957	55.8651730312274\\
5.93799999999957	55.8540022307793\\
5.93999999999957	55.8428336640433\\
5.94199999999957	55.8316673305754\\
5.94399999999957	55.8205032299269\\
5.94599999999957	55.8093413616528\\
5.94799999999957	55.7981817253047\\
5.94999999999957	55.7870243204383\\
5.95199999999957	55.7758691466076\\
5.95399999999957	55.764716203365\\
5.95599999999957	55.7535654902641\\
5.95799999999957	55.7424170068613\\
5.95999999999957	55.731270752708\\
5.96199999999957	55.7201267273602\\
5.96399999999957	55.7089849303722\\
5.96599999999956	55.6978453612968\\
5.96799999999956	55.6867080196918\\
5.96999999999956	55.6755729051083\\
5.97199999999956	55.6644400171022\\
5.97399999999956	55.6533093552295\\
5.97599999999956	55.6421809190436\\
5.97799999999956	55.6310547081005\\
5.97999999999956	55.6199307219537\\
5.98199999999956	55.6088089601597\\
5.98399999999956	55.5976894222737\\
5.98599999999956	55.5865721078496\\
5.98799999999956	55.575457016445\\
5.98999999999956	55.5643441476127\\
5.99199999999956	55.5532335009108\\
5.99399999999956	55.5421250758939\\
5.99599999999956	55.531018872118\\
5.99799999999956	55.5199148891377\\
5.99999999999956	55.5088131265105\\
6.00199999999956	55.4977135837923\\
6.00399999999956	55.4866162605374\\
6.00599999999956	55.4755211563043\\
6.00799999999956	55.4644282706495\\
6.00999999999956	55.4533376031261\\
6.01199999999956	55.4422491532931\\
6.01399999999956	55.431162920708\\
6.01599999999956	55.4200789049242\\
6.01799999999956	55.4089971055016\\
6.01999999999956	55.3979175219959\\
6.02199999999956	55.386840153963\\
6.02399999999956	55.3757650009617\\
6.02599999999956	55.3646920625469\\
6.02799999999956	55.3536213382773\\
6.02999999999956	55.3425528277103\\
6.03199999999956	55.3314865304026\\
6.03399999999956	55.3204224459116\\
6.03599999999956	55.3093605737948\\
6.03799999999956	55.2983009136115\\
6.03999999999956	55.2872434649164\\
6.04199999999956	55.2761882272695\\
6.04399999999956	55.2651352002276\\
6.04599999999956	55.2540843833496\\
6.04799999999956	55.2430357761928\\
6.04999999999956	55.231989378314\\
6.05199999999956	55.2209451892751\\
6.05399999999956	55.2099032086318\\
6.05599999999956	55.1988634359441\\
6.05799999999955	55.1878258707674\\
6.05999999999955	55.1767905126639\\
6.06199999999955	55.1657573611902\\
6.06399999999955	55.1547264159053\\
6.06599999999955	55.1436976763681\\
6.06799999999955	55.1326711421381\\
6.06999999999955	55.1216468127742\\
6.07199999999955	55.1106246878348\\
6.07399999999955	55.0996047668793\\
6.07599999999955	55.0885870494687\\
6.07799999999955	55.0775715351601\\
6.07999999999955	55.0665582235143\\
6.08199999999955	55.0555471140899\\
6.08399999999955	55.0445382064469\\
6.08599999999955	55.0335315001457\\
6.08799999999955	55.022526994745\\
6.08999999999955	55.0115246898063\\
6.09199999999955	55.0005245848879\\
6.09399999999955	54.9895266795505\\
6.09599999999955	54.9785309733544\\
6.09799999999955	54.9675374658598\\
6.09999999999955	54.9565461566268\\
6.10199999999955	54.9455570452163\\
6.10399999999955	54.9345701311882\\
6.10599999999955	54.923585414104\\
6.10799999999955	54.9126028935249\\
6.10999999999955	54.9016225690094\\
6.11199999999955	54.89064444012\\
6.11399999999955	54.8796685064176\\
6.11599999999955	54.8686947674623\\
6.11799999999955	54.8577232228172\\
6.11999999999955	54.8467538720412\\
6.12199999999955	54.8357867146967\\
6.12399999999955	54.8248217503455\\
6.12599999999955	54.8138589785484\\
6.12799999999955	54.8028983988678\\
6.12999999999955	54.7919400108637\\
6.13199999999955	54.7809838141005\\
6.13399999999955	54.7700298081361\\
6.13599999999955	54.7590779925368\\
6.13799999999955	54.7481283668633\\
6.13999999999955	54.7371809306754\\
6.14199999999955	54.7262356835374\\
6.14399999999955	54.7152926250125\\
6.14599999999955	54.7043517546599\\
6.14799999999954	54.6934130720438\\
6.14999999999954	54.6824765767269\\
6.15199999999954	54.6715422682721\\
6.15399999999954	54.6606101462415\\
6.15599999999954	54.6496802101992\\
6.15799999999954	54.6387524597052\\
6.15999999999954	54.6278268943254\\
6.16199999999954	54.616903513621\\
6.16399999999954	54.6059823171576\\
6.16599999999954	54.5950633044955\\
6.16799999999954	54.5841464751987\\
6.16999999999954	54.5732318288317\\
6.17199999999954	54.5623193649584\\
6.17399999999954	54.5514090831402\\
6.17599999999954	54.5405009829429\\
6.17799999999954	54.5295950639285\\
6.17999999999954	54.518691325663\\
6.18199999999954	54.5077897677082\\
6.18399999999954	54.4968903896285\\
6.18599999999954	54.4859931909892\\
6.18799999999954	54.4750981713547\\
6.18999999999954	54.4642053302874\\
6.19199999999954	54.4533146673528\\
6.19399999999954	54.4424261821155\\
6.19599999999954	54.4315398741399\\
6.19799999999954	54.4206557429903\\
6.19999999999954	54.4097737882315\\
6.20199999999954	54.3988940094284\\
6.20399999999954	54.3880164061471\\
6.20599999999954	54.3771409779496\\
6.20799999999954	54.3662677244041\\
6.20999999999954	54.3553966450741\\
6.21199999999954	54.344527739525\\
6.21399999999954	54.3336610073225\\
6.21599999999954	54.3227964480324\\
6.21799999999954	54.3119340612188\\
6.21999999999954	54.3010738464488\\
6.22199999999954	54.2902158032866\\
6.22399999999954	54.2793599312999\\
6.22599999999954	54.2685062300526\\
6.22799999999954	54.2576546991121\\
6.22999999999954	54.2468053380433\\
6.23199999999954	54.2359581464143\\
6.23399999999954	54.225113123788\\
6.23599999999954	54.2142702697331\\
6.23799999999954	54.2034295838151\\
6.23999999999953	54.1925910656008\\
6.24199999999953	54.1817547146582\\
6.24399999999953	54.1709205305504\\
6.24599999999953	54.1600885128472\\
6.24799999999953	54.1492586611142\\
6.24999999999953	54.1384309749181\\
6.25199999999953	54.1276054538263\\
6.25399999999953	54.1167820974069\\
6.25599999999953	54.1059609052238\\
6.25799999999953	54.0951418768482\\
6.25999999999953	54.0843250118436\\
6.26199999999953	54.0735103097825\\
6.26399999999953	54.0626977702276\\
6.26599999999953	54.0518873927481\\
6.26799999999953	54.0410791769118\\
6.26999999999953	54.0302731222855\\
6.27199999999953	54.0194692284408\\
6.27399999999953	54.0086674949404\\
6.27599999999953	53.997867921355\\
6.27799999999953	53.9870705072532\\
6.27999999999953	53.9762752522016\\
6.28199999999953	53.9654821557696\\
6.28399999999953	53.9546912175258\\
6.28599999999953	53.9439024370377\\
6.28799999999953	53.9331158138738\\
6.28999999999953	53.922331347603\\
6.29199999999953	53.9115490377952\\
6.29399999999953	53.9007688840186\\
6.29599999999953	53.8899908858403\\
6.29799999999953	53.8792150428303\\
6.29999999999953	53.8684413545584\\
6.30199999999953	53.8576698205934\\
6.30399999999953	53.8469004405053\\
6.30599999999953	53.8361332138614\\
6.30799999999953	53.8253681402334\\
6.30999999999953	53.8146052191883\\
6.31199999999953	53.8038444502974\\
6.31399999999953	53.7930858331306\\
6.31599999999953	53.7823293672563\\
6.31799999999953	53.7715750522448\\
6.31999999999953	53.7608228876677\\
6.32199999999953	53.7500728730912\\
6.32399999999953	53.7393250080896\\
6.32599999999953	53.7285792922306\\
6.32799999999953	53.7178357250859\\
6.32999999999952	53.7070943062224\\
6.33199999999952	53.6963550352147\\
6.33399999999952	53.6856179116322\\
6.33599999999952	53.6748829350447\\
6.33799999999952	53.6641501050229\\
6.33999999999952	53.6534194211374\\
6.34199999999952	53.6426908829604\\
6.34399999999952	53.6319644900626\\
6.34599999999952	53.6212402420132\\
6.34799999999952	53.6105181383846\\
6.34999999999952	53.5997981787483\\
6.35199999999952	53.5890803626743\\
6.35399999999952	53.5783646897367\\
6.35599999999952	53.5676511595036\\
6.35799999999952	53.5569397715488\\
6.35999999999952	53.5462305254437\\
6.36199999999952	53.5355234207588\\
6.36399999999952	53.524818457067\\
6.36599999999952	53.5141156339398\\
6.36799999999952	53.5034149509486\\
6.36999999999952	53.4927164076665\\
6.37199999999952	53.4820200036648\\
6.37399999999952	53.4713257385175\\
6.37599999999952	53.4606336117937\\
6.37799999999952	53.4499436230678\\
6.37999999999952	53.4392557719143\\
6.38199999999952	53.4285700579015\\
6.38399999999952	53.4178864806047\\
6.38599999999952	53.4072050395971\\
6.38799999999952	53.3965257344489\\
6.38999999999952	53.3858485647356\\
6.39199999999952	53.3751735300286\\
6.39399999999952	53.364500629902\\
6.39599999999952	53.353829863929\\
6.39799999999952	53.3431612316826\\
6.39999999999952	53.3324947327346\\
6.40199999999952	53.3218303666604\\
6.40399999999952	53.3111681330341\\
6.40599999999952	53.3005080314269\\
6.40799999999952	53.2898500614141\\
6.40999999999952	53.2791942225693\\
6.41199999999952	53.2685405144661\\
6.41399999999952	53.2578889366774\\
6.41599999999952	53.2472394887797\\
6.41799999999952	53.236592170346\\
6.41999999999951	53.2259469809487\\
6.42199999999951	53.2153039201644\\
6.42399999999951	53.2046629875663\\
6.42599999999951	53.1940241827288\\
6.42799999999951	53.1833875052276\\
6.42999999999951	53.1727529546361\\
6.43199999999951	53.1621205305297\\
6.43399999999951	53.1514902324827\\
6.43599999999951	53.1408620600703\\
6.43799999999951	53.1302360128669\\
6.43999999999951	53.1196120904479\\
6.44199999999951	53.1089902923897\\
6.44399999999951	53.0983706182654\\
6.44599999999951	53.0877530676514\\
6.44799999999951	53.0771376401231\\
6.44999999999951	53.0665243352551\\
6.45199999999951	53.0559131526243\\
6.45399999999951	53.0453040918052\\
6.45599999999951	53.0346971523752\\
6.45799999999951	53.0240923339084\\
6.45999999999951	53.0134896359812\\
6.46199999999951	53.0028890581679\\
6.46399999999951	52.9922906000467\\
6.46599999999951	52.9816942611956\\
6.46799999999951	52.9711000411863\\
6.46999999999951	52.9605079395975\\
6.47199999999951	52.9499179560058\\
6.47399999999951	52.9393300899878\\
6.47599999999951	52.9287443411194\\
6.47799999999951	52.9181607089765\\
6.47999999999951	52.9075791931377\\
6.48199999999951	52.8969997931784\\
6.48399999999951	52.8864225086765\\
6.48599999999951	52.8758473392082\\
6.48799999999951	52.8652742843507\\
6.48999999999951	52.8547033436819\\
6.49199999999951	52.8441345167784\\
6.49399999999951	52.8335678032177\\
6.49599999999951	52.8230032025763\\
6.49799999999951	52.8124407144325\\
6.49999999999951	52.8018803383655\\
6.50199999999951	52.7913220739499\\
6.50399999999951	52.7807659207651\\
6.50599999999951	52.7702118783893\\
6.50799999999951	52.7596599464001\\
6.50999999999951	52.7491101243742\\
6.5119999999995	52.7385624118916\\
6.5139999999995	52.7280168085289\\
6.5159999999995	52.7174733138664\\
6.5179999999995	52.7069319274808\\
6.5199999999995	52.6963926489507\\
6.5219999999995	52.685855477854\\
6.5239999999995	52.6753204137714\\
6.5259999999995	52.6647874562797\\
6.5279999999995	52.654256604958\\
6.5299999999995	52.6437278593861\\
6.5319999999995	52.6332012191424\\
6.5339999999995	52.622676683805\\
6.5359999999995	52.6121542529539\\
6.5379999999995	52.6016339261683\\
6.5399999999995	52.5911157030272\\
6.5419999999995	52.5805995831098\\
6.5439999999995	52.5700855659966\\
6.5459999999995	52.559573651266\\
6.5479999999995	52.5490638384979\\
6.5499999999995	52.5385561272723\\
6.5519999999995	52.5280505171687\\
6.5539999999995	52.517547007767\\
6.5559999999995	52.5070455986463\\
6.5579999999995	52.4965462893886\\
6.5599999999995	52.4860490795724\\
6.5619999999995	52.475553968778\\
6.5639999999995	52.465060956587\\
6.5659999999995	52.4545700425788\\
6.5679999999995	52.4440812263311\\
6.5699999999995	52.43359450743\\
6.5719999999995	52.4231098854533\\
6.5739999999995	52.4126273599806\\
6.5759999999995	52.4021469305947\\
6.5779999999995	52.391668596875\\
6.5799999999995	52.3811923584022\\
6.5819999999995	52.3707182147598\\
6.5839999999995	52.360246165526\\
6.5859999999995	52.3497762102827\\
6.5879999999995	52.3393083486131\\
6.5899999999995	52.3288425800971\\
6.5919999999995	52.3183789043167\\
6.5939999999995	52.3079173208514\\
6.5959999999995	52.2974578292863\\
6.5979999999995	52.2870004291996\\
6.5999999999995	52.276545120175\\
6.60199999999949	52.2660919017946\\
6.60399999999949	52.2556407736393\\
6.60599999999949	52.2451917352922\\
6.60799999999949	52.2347447863336\\
6.60999999999949	52.224299926349\\
6.61199999999949	52.213857154917\\
6.61399999999949	52.2034164716228\\
6.61599999999949	52.1929778760475\\
6.61799999999949	52.1825413677736\\
6.61999999999949	52.1721069463841\\
6.62199999999949	52.1616746114622\\
6.62399999999949	52.1512443625891\\
6.62599999999949	52.140816199349\\
6.62799999999949	52.1303901213243\\
6.62999999999949	52.1199661280988\\
6.63199999999949	52.109544219254\\
6.63399999999949	52.0991243943752\\
6.63599999999949	52.0887066530437\\
6.63799999999949	52.0782909948451\\
6.63999999999949	52.0678774193617\\
6.64199999999949	52.0574659261754\\
6.64399999999949	52.0470565148727\\
6.64599999999949	52.0366491850356\\
6.64799999999949	52.0262439362474\\
6.64999999999949	52.0158407680944\\
6.65199999999949	52.0054396801576\\
6.65399999999949	51.9950406720221\\
6.65599999999949	51.9846437432737\\
6.65799999999949	51.9742488934954\\
6.65999999999949	51.96385612227\\
6.66199999999949	51.9534654291845\\
6.66399999999949	51.9430768138216\\
6.66599999999949	51.9326902757657\\
6.66799999999949	51.9223058146026\\
6.66999999999949	51.9119234299167\\
6.67199999999949	51.9015431212922\\
6.67399999999949	51.891164888314\\
6.67599999999949	51.8807887305678\\
6.67799999999949	51.8704146476378\\
6.67999999999949	51.8600426391102\\
6.68199999999949	51.849672704569\\
6.68399999999949	51.8393048435991\\
6.68599999999949	51.8289390557886\\
6.68799999999949	51.8185753407205\\
6.68999999999949	51.8082136979806\\
6.69199999999949	51.7978541271546\\
6.69399999999948	51.7874966278287\\
6.69599999999948	51.7771411995886\\
6.69799999999948	51.7667878420199\\
6.69999999999948	51.7564365547087\\
6.70199999999948	51.7460873372414\\
6.70399999999948	51.7357401892033\\
6.70599999999948	51.725395110181\\
6.70799999999948	51.7150520997608\\
6.70999999999948	51.7047111575294\\
6.71199999999948	51.6943722830725\\
6.71399999999948	51.6840354759765\\
6.71599999999948	51.6737007358299\\
6.71799999999948	51.6633680622169\\
6.71999999999948	51.6530374547257\\
6.72199999999948	51.6427089129429\\
6.72399999999948	51.6323824364552\\
6.72599999999948	51.6220580248505\\
6.72799999999948	51.6117356777152\\
6.72999999999948	51.6014153946354\\
6.73199999999948	51.5910971752001\\
6.73399999999948	51.5807810189969\\
6.73599999999948	51.5704669256108\\
6.73799999999948	51.5601548946314\\
6.73999999999948	51.5498449256466\\
6.74199999999948	51.5395370182427\\
6.74399999999948	51.5292311720087\\
6.74599999999948	51.5189273865307\\
6.74799999999948	51.5086256613977\\
6.74999999999948	51.498325996199\\
6.75199999999948	51.4880283905201\\
6.75399999999948	51.4777328439513\\
6.75599999999948	51.46743935608\\
6.75799999999948	51.4571479264946\\
6.75999999999948	51.4468585547834\\
6.76199999999948	51.4365712405353\\
6.76399999999948	51.4262859833388\\
6.76599999999948	51.4160027827818\\
6.76799999999948	51.4057216384541\\
6.76999999999948	51.395442549944\\
6.77199999999948	51.3851655168404\\
6.77399999999948	51.3748905387322\\
6.77599999999948	51.3646176152088\\
6.77799999999948	51.3543467458598\\
6.77999999999948	51.3440779302742\\
6.78199999999948	51.3338111680398\\
6.78399999999947	51.3235464587474\\
6.78599999999947	51.313283801987\\
6.78799999999947	51.3030231973478\\
6.78999999999947	51.2927646444185\\
6.79199999999947	51.2825081427897\\
6.79399999999947	51.2722536920511\\
6.79599999999947	51.262001291793\\
6.79799999999947	51.2517509416045\\
6.79999999999947	51.2415026410756\\
6.80199999999947	51.2312563897978\\
6.80399999999947	51.2210121873603\\
6.80599999999947	51.2107700333533\\
6.80799999999947	51.2005299273682\\
6.80999999999947	51.1902918689936\\
6.81199999999947	51.180055857822\\
6.81399999999947	51.169821893443\\
6.81599999999947	51.1595899754478\\
6.81799999999947	51.1493601034268\\
6.81999999999947	51.1391322769713\\
6.82199999999947	51.1289064956724\\
6.82399999999947	51.1186827591202\\
6.82599999999947	51.1084610669069\\
6.82799999999947	51.0982414186226\\
6.82999999999947	51.08802381386\\
6.83199999999947	51.0778082522095\\
6.83399999999947	51.0675947332628\\
6.83599999999947	51.0573832566107\\
6.83799999999947	51.0471738218468\\
6.83999999999947	51.0369664285606\\
6.84199999999947	51.0267610763454\\
6.84399999999947	51.0165577647924\\
6.84599999999947	51.0063564934936\\
6.84799999999947	50.9961572620404\\
6.84999999999947	50.985960070027\\
6.85199999999947	50.9757649170437\\
6.85399999999947	50.9655718026826\\
6.85599999999947	50.9553807265373\\
6.85799999999947	50.9451916881995\\
6.85999999999947	50.9350046872621\\
6.86199999999947	50.9248197233175\\
6.86399999999947	50.9146367959577\\
6.86599999999947	50.904455904776\\
6.86799999999947	50.8942770493665\\
6.86999999999947	50.8841002293208\\
6.87199999999947	50.8739254442318\\
6.87399999999946	50.8637526936928\\
6.87599999999946	50.8535819772967\\
6.87799999999946	50.8434132946387\\
6.87999999999946	50.8332466453088\\
6.88199999999946	50.8230820289026\\
6.88399999999946	50.8129194450139\\
6.88599999999946	50.8027588932352\\
6.88799999999946	50.7926003731604\\
6.88999999999946	50.782443884384\\
6.89199999999946	50.7722894264981\\
6.89399999999946	50.7621369990985\\
6.89599999999946	50.7519866017778\\
6.89799999999946	50.74183823413\\
6.89999999999946	50.731691895752\\
6.90199999999946	50.7215475862347\\
6.90399999999946	50.7114053051733\\
6.90599999999946	50.7012650521623\\
6.90799999999946	50.691126826797\\
6.90999999999946	50.6809906286711\\
6.91199999999946	50.6708564573795\\
6.91399999999946	50.6607243125172\\
6.91599999999946	50.6505941936784\\
6.91799999999946	50.6404661004577\\
6.91999999999946	50.6303400324512\\
6.92199999999946	50.6202159892531\\
6.92399999999946	50.6100939704587\\
6.92599999999946	50.5999739756636\\
6.92799999999946	50.5898560044625\\
6.92999999999946	50.5797400564515\\
6.93199999999946	50.5696261312252\\
6.93399999999946	50.5595142283796\\
6.93599999999946	50.5494043475097\\
6.93799999999946	50.5392964882118\\
6.93999999999946	50.5291906500816\\
6.94199999999946	50.5190868327151\\
6.94399999999946	50.508985035708\\
6.94599999999946	50.4988852586562\\
6.94799999999946	50.488787501156\\
6.94999999999946	50.4786917628033\\
6.95199999999946	50.4685980431945\\
6.95399999999946	50.4585063419258\\
6.95599999999946	50.448416658594\\
6.95799999999946	50.4383289927952\\
6.95999999999946	50.4282433441266\\
6.96199999999946	50.418159712184\\
6.96399999999946	50.4080780965648\\
6.96599999999945	50.3979984968648\\
6.96799999999945	50.3879209126822\\
6.96999999999945	50.3778453436128\\
6.97199999999945	50.3677717892561\\
6.97399999999945	50.3577002492057\\
6.97599999999945	50.3476307230613\\
6.97799999999945	50.3375632104191\\
6.97999999999945	50.3274977108771\\
6.98199999999945	50.317434224032\\
6.98399999999945	50.3073727494818\\
6.98599999999945	50.2973132868246\\
6.98799999999945	50.2872558356574\\
6.98999999999945	50.2772003955779\\
6.99199999999945	50.2671469661843\\
6.99399999999945	50.2570955470751\\
6.99599999999945	50.2470461378477\\
6.99799999999945	50.2369987380994\\
6.99999999999945	50.2269533474298\\
7.00199999999945	50.2169099654369\\
7.00399999999945	50.2068685917188\\
7.00599999999945	50.1968292258735\\
7.00799999999945	50.1867918674996\\
7.00999999999945	50.1767565161966\\
7.01199999999945	50.1667231715615\\
7.01399999999945	50.1566918331952\\
7.01599999999945	50.146662500695\\
7.01799999999945	50.1366351736606\\
7.01999999999945	50.1266098516899\\
7.02199999999945	50.1165865343832\\
7.02399999999945	50.1065652213388\\
7.02599999999945	50.0965459121561\\
7.02799999999945	50.0865286064349\\
7.02999999999945	50.0765133037746\\
7.03199999999945	50.0665000037731\\
7.03399999999945	50.0564887060317\\
7.03599999999945	50.0464794101498\\
7.03799999999945	50.0364721157262\\
7.03999999999945	50.0264668223621\\
7.04199999999945	50.0164635296559\\
7.04399999999945	50.0064622372081\\
7.04599999999945	49.9964629446197\\
7.04799999999945	49.9864656514889\\
7.04999999999945	49.9764703574171\\
7.05199999999945	49.9664770620041\\
7.05399999999945	49.9564857648515\\
7.05599999999944	49.9464964655589\\
7.05799999999944	49.9365091637258\\
7.05999999999944	49.9265238589544\\
7.06199999999944	49.9165405508438\\
7.06399999999944	49.9065592389953\\
7.06599999999944	49.8965799230109\\
7.06799999999944	49.8866026024908\\
7.06999999999944	49.8766272770353\\
7.07199999999944	49.8666539462455\\
7.07399999999944	49.8566826097245\\
7.07599999999944	49.8467132670703\\
7.07799999999944	49.836745917887\\
7.07999999999944	49.8267805617742\\
7.08199999999944	49.8168171983346\\
7.08399999999944	49.8068558271688\\
7.08599999999944	49.796896447879\\
7.08799999999944	49.7869390600676\\
7.08999999999944	49.7769836633347\\
7.09199999999944	49.7670302572833\\
7.09399999999944	49.7570788415148\\
7.09599999999944	49.7471294156315\\
7.09799999999944	49.7371819792357\\
7.09999999999944	49.7272365319287\\
7.10199999999944	49.7172930733142\\
7.10399999999944	49.7073516029938\\
7.10599999999944	49.6974121205692\\
7.10799999999944	49.6874746256448\\
7.10999999999944	49.6775391178211\\
7.11199999999944	49.6676055967012\\
7.11399999999944	49.657674061889\\
7.11599999999944	49.6477445129867\\
7.11799999999944	49.6378169495962\\
7.11999999999944	49.6278913713218\\
7.12199999999944	49.6179677777666\\
7.12399999999944	49.608046168533\\
7.12599999999944	49.5981265432241\\
7.12799999999944	49.5882089014441\\
7.12999999999944	49.5782932427951\\
7.13199999999944	49.5683795668813\\
7.13399999999944	49.558467873307\\
7.13599999999944	49.5485581616746\\
7.13799999999944	49.5386504315878\\
7.13999999999944	49.5287446826505\\
7.14199999999944	49.5188409144683\\
7.14399999999944	49.5089391266431\\
7.14599999999944	49.499039318779\\
7.14799999999943	49.4891414904813\\
7.14999999999943	49.4792456413529\\
7.15199999999943	49.4693517709983\\
7.15399999999943	49.459459879023\\
7.15599999999943	49.4495699650293\\
7.15799999999943	49.4396820286241\\
7.15999999999943	49.4297960694099\\
7.16199999999943	49.419912086992\\
7.16399999999943	49.4100300809764\\
7.16599999999943	49.4001500509656\\
7.16799999999943	49.3902719965663\\
7.16999999999943	49.3803959173829\\
7.17199999999943	49.3705218130209\\
7.17399999999943	49.3606496830844\\
7.17599999999943	49.3507795271783\\
7.17799999999943	49.3409113449091\\
7.17999999999943	49.3310451358814\\
7.18199999999943	49.3211808997021\\
7.18399999999943	49.3113186359749\\
7.18599999999943	49.3014583443066\\
7.18799999999943	49.2916000243012\\
7.18999999999943	49.2817436755658\\
7.19199999999943	49.2718892977059\\
7.19399999999943	49.262036890328\\
7.19599999999943	49.2521864530376\\
7.19799999999943	49.2423379854403\\
7.19999999999943	49.232491487143\\
7.20199999999943	49.2226469577515\\
7.20399999999943	49.2128043968719\\
7.20599999999943	49.2029638041113\\
7.20799999999943	49.193125179075\\
7.20999999999943	49.1832885213705\\
7.21199999999943	49.173453830605\\
7.21399999999943	49.1636211063833\\
7.21599999999943	49.1537903483138\\
7.21799999999943	49.1439615560024\\
7.21999999999943	49.1341347290566\\
7.22199999999943	49.1243098670834\\
7.22399999999943	49.1144869696888\\
7.22599999999943	49.1046660364819\\
7.22799999999943	49.0948470670686\\
7.22999999999943	49.0850300610561\\
7.23199999999943	49.0752150180528\\
7.23399999999943	49.0654019376648\\
7.23599999999943	49.0555908195007\\
7.23799999999942	49.0457816631682\\
7.23999999999942	49.035974468274\\
7.24199999999942	49.0261692344275\\
7.24399999999942	49.0163659612352\\
7.24599999999942	49.0065646483056\\
7.24799999999942	48.9967652952465\\
7.24999999999942	48.9869679016661\\
7.25199999999942	48.977172467173\\
7.25399999999942	48.9673789913746\\
7.25599999999942	48.9575874738793\\
7.25799999999942	48.9477979142969\\
7.25999999999942	48.9380103122337\\
7.26199999999942	48.9282246673005\\
7.26399999999942	48.9184409791049\\
7.26599999999942	48.9086592472552\\
7.26799999999942	48.8988794713608\\
7.26999999999942	48.889101651031\\
7.27199999999942	48.8793257858736\\
7.27399999999942	48.8695518754982\\
7.27599999999942	48.8597799195145\\
7.27799999999942	48.8500099175309\\
7.27999999999942	48.8402418691569\\
7.28199999999942	48.8304757740019\\
7.28399999999942	48.820711631676\\
7.28599999999942	48.8109494417874\\
7.28799999999942	48.8011892039467\\
7.28999999999942	48.7914309177635\\
7.29199999999942	48.7816745828462\\
7.29399999999942	48.7719201988065\\
7.29599999999942	48.7621677652534\\
7.29799999999942	48.7524172817971\\
7.29999999999942	48.7426687480476\\
7.30199999999942	48.7329221636146\\
7.30399999999942	48.7231775281092\\
7.30599999999942	48.7134348411412\\
7.30799999999942	48.7036941023203\\
7.30999999999942	48.6939553112583\\
7.31199999999942	48.6842184675644\\
7.31399999999942	48.6744835708505\\
7.31599999999942	48.6647506207262\\
7.31799999999942	48.6550196168026\\
7.31999999999942	48.6452905586905\\
7.32199999999942	48.6355634460013\\
7.32399999999942	48.6258382783459\\
7.32599999999942	48.6161150553353\\
7.32799999999941	48.6063937765797\\
7.32999999999941	48.596674441691\\
7.33199999999941	48.5869570502811\\
7.33399999999941	48.5772416019608\\
7.33599999999941	48.5675280963414\\
7.33799999999941	48.5578165330346\\
7.33999999999941	48.5481069116527\\
7.34199999999941	48.5383992318057\\
7.34399999999941	48.5286934931074\\
7.34599999999941	48.5189896951682\\
7.34799999999941	48.5092878376008\\
7.34999999999941	48.499587920017\\
7.35199999999941	48.4898899420284\\
7.35399999999941	48.4801939032474\\
7.35599999999941	48.4704998032869\\
7.35799999999941	48.4608076417585\\
7.35999999999941	48.451117418275\\
7.36199999999941	48.4414291324482\\
7.36399999999941	48.4317427838909\\
7.36599999999941	48.4220583722169\\
7.36799999999941	48.4123758970372\\
7.36999999999941	48.4026953579659\\
7.37199999999941	48.3930167546152\\
7.37399999999941	48.3833400865976\\
7.37599999999941	48.3736653535267\\
7.37799999999941	48.3639925550159\\
7.37999999999941	48.3543216906776\\
7.38199999999941	48.3446527601253\\
7.38399999999941	48.334985762973\\
7.38599999999941	48.3253206988334\\
7.38799999999941	48.3156575673203\\
7.38999999999941	48.3059963680462\\
7.39199999999941	48.2963371006259\\
7.39399999999941	48.286679764673\\
7.39599999999941	48.2770243598011\\
7.39799999999941	48.2673708856239\\
7.39999999999941	48.2577193417552\\
7.40199999999941	48.24806972781\\
7.40399999999941	48.2384220434011\\
7.40599999999941	48.2287762881435\\
7.40799999999941	48.219132461651\\
7.40999999999941	48.2094905635386\\
7.41199999999941	48.1998505934202\\
7.41399999999941	48.1902125509096\\
7.41599999999941	48.1805764356221\\
7.41799999999941	48.1709422471732\\
7.4199999999994	48.1613099851757\\
7.4219999999994	48.1516796492461\\
7.4239999999994	48.1420512389981\\
7.4259999999994	48.1324247540476\\
7.4279999999994	48.1228001940087\\
7.4299999999994	48.1131775584967\\
7.4319999999994	48.1035568471272\\
7.4339999999994	48.0939380595147\\
7.4359999999994	48.0843211952762\\
7.4379999999994	48.0747062540256\\
7.4399999999994	48.0650932353784\\
7.4419999999994	48.0554821389502\\
7.4439999999994	48.0458729643582\\
7.4459999999994	48.0362657112152\\
7.4479999999994	48.0266603791396\\
7.4499999999994	48.0170569677458\\
7.4519999999994	48.0074554766502\\
7.4539999999994	47.9978559054694\\
7.4559999999994	47.9882582538183\\
7.4579999999994	47.9786625213144\\
7.4599999999994	47.9690687075729\\
7.4619999999994	47.9594768122103\\
7.4639999999994	47.9498868348434\\
7.4659999999994	47.9402987750882\\
7.4679999999994	47.9307126325619\\
7.4699999999994	47.9211284068807\\
7.4719999999994	47.9115460976611\\
7.4739999999994	47.90196570452\\
7.4759999999994	47.8923872270746\\
7.4779999999994	47.8828106649414\\
7.4799999999994	47.873236017738\\
7.4819999999994	47.8636632850812\\
7.4839999999994	47.8540924665878\\
7.4859999999994	47.8445235618749\\
7.4879999999994	47.8349565705607\\
7.4899999999994	47.8253914922624\\
7.4919999999994	47.8158283265971\\
7.4939999999994	47.8062670731828\\
7.4959999999994	47.7967077316365\\
7.4979999999994	47.7871503015763\\
7.4999999999994	47.7775947826195\\
7.5019999999994	47.7680411743842\\
7.5039999999994	47.7584894764893\\
7.5059999999994	47.7489396885513\\
7.5079999999994	47.7393918101893\\
7.50999999999939	47.7298458410209\\
7.51199999999939	47.7203017806652\\
7.51399999999939	47.7107596287392\\
7.51599999999939	47.7012193848621\\
7.51799999999939	47.6916810486527\\
7.51999999999939	47.6821446197285\\
7.52199999999939	47.6726100977093\\
7.52399999999939	47.6630774822124\\
7.52599999999939	47.6535467728582\\
7.52799999999939	47.644017969264\\
7.52999999999939	47.6344910710499\\
7.53199999999939	47.6249660778341\\
7.53399999999939	47.6154429892361\\
7.53599999999939	47.6059218048755\\
7.53799999999939	47.5964025243704\\
7.53999999999939	47.5868851473409\\
7.54199999999939	47.5773696734059\\
7.54399999999939	47.5678561021854\\
7.54599999999939	47.5583444332992\\
7.54799999999939	47.5488346663656\\
7.54999999999939	47.5393268010061\\
7.55199999999939	47.5298208368386\\
7.55399999999939	47.520316773484\\
7.55599999999939	47.5108146105614\\
7.55799999999939	47.5013143476918\\
7.55999999999939	47.4918159844951\\
7.56199999999939	47.4823195205906\\
7.56399999999939	47.4728249555995\\
7.56599999999939	47.4633322891422\\
7.56799999999939	47.4538415208375\\
7.56999999999939	47.4443526503072\\
7.57199999999939	47.4348656771724\\
7.57399999999939	47.4253806010522\\
7.57599999999939	47.4158974215676\\
7.57799999999939	47.4064161383397\\
7.57999999999939	47.3969367509889\\
7.58199999999939	47.3874592591383\\
7.58399999999939	47.3779836624054\\
7.58599999999939	47.3685099604132\\
7.58799999999939	47.3590381527826\\
7.58999999999939	47.3495682391346\\
7.59199999999939	47.3401002190909\\
7.59399999999939	47.3306340922729\\
7.59599999999939	47.3211698583009\\
7.59799999999939	47.3117075167974\\
7.59999999999939	47.3022470673838\\
7.60199999999938	47.2927885096824\\
7.60399999999938	47.2833318433135\\
7.60599999999938	47.2738770678998\\
7.60799999999938	47.2644241830633\\
7.60999999999938	47.2549731884261\\
7.61199999999938	47.2455240836084\\
7.61399999999938	47.2360768682354\\
7.61599999999938	47.2266315419265\\
7.61799999999938	47.2171881043059\\
7.61999999999938	47.2077465549947\\
7.62199999999938	47.1983068936161\\
7.62399999999938	47.1888691197922\\
7.62599999999938	47.1794332331453\\
7.62799999999938	47.1699992332986\\
7.62999999999938	47.1605671198748\\
7.63199999999938	47.1511368924962\\
7.63399999999938	47.1417085507862\\
7.63599999999938	47.1322820943672\\
7.63799999999938	47.1228575228632\\
7.63999999999938	47.1134348358955\\
7.64199999999938	47.104014033089\\
7.64399999999938	47.0945951140662\\
7.64599999999938	47.08517807845\\
7.64799999999938	47.0757629258657\\
7.64999999999938	47.0663496559338\\
7.65199999999938	47.0569382682806\\
7.65399999999938	47.0475287625275\\
7.65599999999938	47.0381211383\\
7.65799999999938	47.028715395221\\
7.65999999999938	47.0193115329141\\
7.66199999999938	47.0099095510041\\
7.66399999999938	47.0005094491143\\
7.66599999999938	46.9911112268688\\
7.66799999999938	46.9817148838921\\
7.66999999999938	46.9723204198081\\
7.67199999999938	46.9629278342417\\
7.67399999999938	46.953537126816\\
7.67599999999938	46.944148297156\\
7.67799999999938	46.9347613448875\\
7.67999999999938	46.9253762696337\\
7.68199999999938	46.9159930710195\\
7.68399999999938	46.9066117486697\\
7.68599999999938	46.8972323022095\\
7.68799999999938	46.8878547312629\\
7.68999999999938	46.8784790354555\\
7.69199999999937	46.8691052144128\\
7.69399999999937	46.8597332677586\\
7.69599999999937	46.8503631951198\\
7.69799999999937	46.8409949961206\\
7.69999999999937	46.831628670387\\
7.70199999999937	46.8222642175433\\
7.70399999999937	46.8129016372156\\
7.70599999999937	46.8035409290302\\
7.70799999999937	46.7941820926119\\
7.70999999999937	46.784825127586\\
7.71199999999937	46.7754700335792\\
7.71399999999937	46.7661168102173\\
7.71599999999937	46.7567654571258\\
7.71799999999937	46.7474159739309\\
7.71999999999937	46.7380683602593\\
7.72199999999937	46.7287226157364\\
7.72399999999937	46.7193787399881\\
7.72599999999937	46.7100367326415\\
7.72799999999937	46.7006965933227\\
7.72999999999937	46.6913583216582\\
7.73199999999937	46.6820219172754\\
7.73399999999937	46.6726873797988\\
7.73599999999937	46.6633547088574\\
7.73799999999937	46.6540239040761\\
7.73999999999937	46.6446949650831\\
7.74199999999937	46.6353678915049\\
7.74399999999937	46.6260426829683\\
7.74599999999937	46.6167193391002\\
7.74799999999937	46.6073978595289\\
7.74999999999937	46.5980782438797\\
7.75199999999937	46.5887604917817\\
7.75399999999937	46.5794446028606\\
7.75599999999937	46.5701305767454\\
7.75799999999937	46.5608184130622\\
7.75999999999937	46.5515081114407\\
7.76199999999937	46.5421996715061\\
7.76399999999937	46.5328930928878\\
7.76599999999937	46.5235883752126\\
7.76799999999937	46.514285518109\\
7.76999999999937	46.5049845212047\\
7.77199999999937	46.4956853841279\\
7.77399999999937	46.4863881065065\\
7.77599999999937	46.477092687969\\
7.77799999999937	46.4677991281432\\
7.77999999999937	46.4585074266581\\
7.78199999999936	46.4492175831407\\
7.78399999999936	46.4399295972216\\
7.78599999999936	46.430643468528\\
7.78799999999936	46.421359196689\\
7.78999999999936	46.4120767813326\\
7.79199999999936	46.4027962220886\\
7.79399999999936	46.3935175185845\\
7.79599999999936	46.3842406704503\\
7.79799999999936	46.3749656773147\\
7.79999999999936	46.3656925388069\\
7.80199999999936	46.3564212545562\\
7.80399999999936	46.3471518241914\\
7.80599999999936	46.3378842473413\\
7.80799999999936	46.3286185236374\\
7.80999999999936	46.3193546527065\\
7.81199999999936	46.3100926341798\\
7.81399999999936	46.300832467686\\
7.81599999999936	46.291574152856\\
7.81799999999936	46.2823176893177\\
7.81999999999936	46.2730630767024\\
7.82199999999936	46.2638103146392\\
7.82399999999936	46.2545594027587\\
7.82599999999936	46.2453103406905\\
7.82799999999936	46.236063128065\\
7.82999999999936	46.2268177645119\\
7.83199999999936	46.2175742496627\\
7.83399999999936	46.2083325831454\\
7.83599999999936	46.1990927645928\\
7.83799999999936	46.1898547936337\\
7.83999999999936	46.1806186698998\\
7.84199999999936	46.1713843930217\\
7.84399999999936	46.1621519626285\\
7.84599999999936	46.1529213783532\\
7.84799999999936	46.1436926398249\\
7.84999999999936	46.134465746676\\
7.85199999999936	46.1252406985363\\
7.85399999999936	46.1160174950373\\
7.85599999999936	46.1067961358099\\
7.85799999999936	46.0975766204856\\
7.85999999999936	46.0883589486962\\
7.86199999999936	46.0791431200719\\
7.86399999999936	46.069929134245\\
7.86599999999936	46.0607169908476\\
7.86799999999936	46.051506689509\\
7.86999999999936	46.0422982298628\\
7.87199999999936	46.0330916115408\\
7.87399999999935	46.0238868341739\\
7.87599999999935	46.0146838973938\\
7.87799999999935	46.0054828008343\\
7.87999999999935	45.9962835441259\\
7.88199999999935	45.9870861269003\\
7.88399999999935	45.9778905487903\\
7.88599999999935	45.9686968094286\\
7.88799999999935	45.9595049084466\\
7.88999999999935	45.9503148454774\\
7.89199999999935	45.9411266201535\\
7.89399999999935	45.9319402321069\\
7.89599999999935	45.9227556809712\\
7.89799999999935	45.9135729663778\\
7.89999999999935	45.9043920879605\\
7.90199999999935	45.8952130453507\\
7.90399999999935	45.8860358381836\\
7.90599999999935	45.8768604660897\\
7.90799999999935	45.8676869287047\\
7.90999999999935	45.858515225659\\
7.91199999999935	45.8493453565888\\
7.91399999999935	45.8401773211237\\
7.91599999999935	45.8310111189002\\
7.91799999999935	45.8218467495489\\
7.91999999999935	45.8126842127078\\
7.92199999999935	45.8035235080053\\
7.92399999999935	45.7943646350791\\
7.92599999999935	45.78520759356\\
7.92799999999935	45.7760523830835\\
7.92999999999935	45.7668990032829\\
7.93199999999935	45.7577474537923\\
7.93399999999935	45.7485977342453\\
7.93599999999935	45.7394498442762\\
7.93799999999935	45.7303037835197\\
7.93999999999935	45.7211595516086\\
7.94199999999935	45.7120171481798\\
7.94399999999935	45.7028765728649\\
7.94599999999935	45.6937378253003\\
7.94799999999935	45.6846009051192\\
7.94999999999935	45.6754658119566\\
7.95199999999935	45.6663325454473\\
7.95399999999935	45.6572011052264\\
7.95599999999935	45.648071490928\\
7.95799999999935	45.6389437021879\\
7.95999999999935	45.6298177386403\\
7.96199999999935	45.6206935999207\\
7.96399999999934	45.6115712856631\\
7.96599999999934	45.6024507955037\\
7.96799999999934	45.5933321290783\\
7.96999999999934	45.5842152860207\\
7.97199999999934	45.5751002659673\\
7.97399999999934	45.5659870685541\\
7.97599999999934	45.5568756934151\\
7.97799999999934	45.5477661401871\\
7.97999999999934	45.5386584085056\\
7.98199999999934	45.5295524980058\\
7.98399999999934	45.520448408324\\
7.98599999999934	45.511346139096\\
7.98799999999934	45.5022456899588\\
7.98999999999934	45.4931470605467\\
7.99199999999934	45.484050250496\\
7.99399999999934	45.4749552594442\\
7.99599999999934	45.4658620870267\\
7.99799999999934	45.4567707328802\\
7.99999999999934	45.4476811966408\\
};
\addplot [color=mycolor1, forget plot]
  table[row sep=crcr]{%
7.99999999999934	45.4476811966408\\
8.00199999999934	45.438593477945\\
8.00399999999934	45.4295075764301\\
8.00599999999934	45.4204234917316\\
8.00799999999934	45.411341223487\\
8.00999999999934	45.4022607713326\\
8.01199999999934	45.3931821349056\\
8.01399999999935	45.3841053138429\\
8.01599999999935	45.3750303077812\\
8.01799999999935	45.3659571163584\\
8.01999999999935	45.3568857392097\\
8.02199999999935	45.3478161759752\\
8.02399999999935	45.3387484262893\\
8.02599999999935	45.3296824897916\\
8.02799999999935	45.3206183661187\\
8.02999999999935	45.3115560549075\\
8.03199999999935	45.3024955557965\\
8.03399999999935	45.2934368684222\\
8.03599999999935	45.284379992424\\
8.03799999999935	45.2753249274388\\
8.03999999999935	45.2662716731045\\
8.04199999999935	45.2572202290584\\
8.04399999999936	45.2481705949397\\
8.04599999999936	45.2391227703853\\
8.04799999999936	45.2300767550342\\
8.04999999999936	45.2210325485246\\
8.05199999999936	45.2119901504946\\
8.05399999999936	45.202949560583\\
8.05599999999936	45.1939107784268\\
8.05799999999936	45.1848738036654\\
8.05999999999936	45.1758386359386\\
8.06199999999936	45.1668052748837\\
8.06399999999936	45.1577737201392\\
8.06599999999936	45.1487439713456\\
8.06799999999936	45.1397160281402\\
8.06999999999936	45.1306898901614\\
8.07199999999937	45.1216655570499\\
8.07399999999937	45.1126430284447\\
8.07599999999937	45.1036223039839\\
8.07799999999937	45.0946033833073\\
8.07999999999937	45.0855862660536\\
8.08199999999937	45.0765709518635\\
8.08399999999937	45.0675574403762\\
8.08599999999937	45.0585457312291\\
8.08799999999937	45.0495358240644\\
8.08999999999937	45.0405277185208\\
8.09199999999937	45.0315214142378\\
8.09399999999937	45.0225169108556\\
8.09599999999937	45.0135142080143\\
8.09799999999937	45.0045133053529\\
8.09999999999937	44.9955142025125\\
8.10199999999938	44.9865168991324\\
8.10399999999938	44.977521394854\\
8.10599999999938	44.9685276893152\\
8.10799999999938	44.9595357821593\\
8.10999999999938	44.9505456730255\\
8.11199999999938	44.9415573615521\\
8.11399999999938	44.932570847383\\
8.11599999999938	44.9235861301572\\
8.11799999999938	44.9146032095147\\
8.11999999999938	44.9056220850979\\
8.12199999999938	44.8966427565466\\
8.12399999999938	44.8876652235024\\
8.12599999999938	44.878689485605\\
8.12799999999938	44.8697155424962\\
8.12999999999938	44.8607433938175\\
8.13199999999939	44.8517730392095\\
8.13399999999939	44.8428044783141\\
8.13599999999939	44.8338377107722\\
8.13799999999939	44.8248727362246\\
8.13999999999939	44.8159095543139\\
8.14199999999939	44.8069481646811\\
8.14399999999939	44.7979885669679\\
8.14599999999939	44.7890307608153\\
8.14799999999939	44.7800747458657\\
8.14999999999939	44.7711205217621\\
8.15199999999939	44.7621680881447\\
8.15399999999939	44.7532174446552\\
8.15599999999939	44.744268590937\\
8.15799999999939	44.7353215266322\\
8.15999999999939	44.7263762513814\\
8.1619999999994	44.7174327648284\\
8.1639999999994	44.708491066615\\
8.1659999999994	44.6995511563833\\
8.1679999999994	44.6906130337768\\
8.1699999999994	44.6816766984372\\
8.1719999999994	44.6727421500075\\
8.1739999999994	44.6638093881301\\
8.1759999999994	44.6548784124473\\
8.1779999999994	44.6459492226028\\
8.1799999999994	44.6370218182392\\
8.1819999999994	44.6280961989994\\
8.1839999999994	44.6191723645264\\
8.1859999999994	44.6102503144638\\
8.1879999999994	44.6013300484532\\
8.1899999999994	44.5924115661405\\
8.19199999999941	44.5834948671672\\
8.19399999999941	44.5745799511767\\
8.19599999999941	44.5656668178134\\
8.19799999999941	44.5567554667199\\
8.19999999999941	44.5478458975402\\
8.20199999999941	44.538938109918\\
8.20399999999941	44.5300321034977\\
8.20599999999941	44.5211278779223\\
8.20799999999941	44.5122254328354\\
8.20999999999941	44.5033247678819\\
8.21199999999941	44.4944258827051\\
8.21399999999941	44.4855287769501\\
8.21599999999941	44.4766334502606\\
8.21799999999941	44.4677399022801\\
8.21999999999941	44.458848132653\\
8.22199999999942	44.449958141025\\
8.22399999999942	44.4410699270396\\
8.22599999999942	44.4321834903418\\
8.22799999999942	44.4232988305759\\
8.22999999999942	44.4144159473858\\
8.23199999999942	44.4055348404177\\
8.23399999999942	44.3966555093159\\
8.23599999999942	44.3877779537254\\
8.23799999999942	44.3789021732905\\
8.23999999999942	44.3700281676568\\
8.24199999999942	44.3611559364698\\
8.24399999999942	44.3522854793734\\
8.24599999999942	44.3434167960143\\
8.24799999999942	44.334549886037\\
8.24999999999942	44.3256847490873\\
8.25199999999943	44.3168213848106\\
8.25399999999943	44.3079597928519\\
8.25599999999943	44.2990999728574\\
8.25799999999943	44.2902419244727\\
8.25999999999943	44.2813856473435\\
8.26199999999943	44.2725311411148\\
8.26399999999943	44.2636784054337\\
8.26599999999943	44.254827439946\\
8.26799999999943	44.2459782442967\\
8.26999999999943	44.2371308181335\\
8.27199999999943	44.228285161101\\
8.27399999999943	44.2194412728467\\
8.27599999999943	44.2105991530158\\
8.27799999999943	44.2017588012557\\
8.27999999999943	44.1929202172117\\
8.28199999999944	44.1840834005317\\
8.28399999999944	44.1752483508619\\
8.28599999999944	44.1664150678481\\
8.28799999999944	44.1575835511375\\
8.28999999999944	44.1487538003781\\
8.29199999999944	44.1399258152149\\
8.29399999999944	44.131099595296\\
8.29599999999944	44.1222751402677\\
8.29799999999944	44.113452449778\\
8.29999999999944	44.1046315234729\\
8.30199999999944	44.095812361001\\
8.30399999999944	44.0869949620088\\
8.30599999999944	44.0781793261431\\
8.30799999999944	44.0693654530527\\
8.30999999999944	44.060553342384\\
8.31199999999945	44.0517429937855\\
8.31399999999945	44.042934406904\\
8.31599999999945	44.0341275813878\\
8.31799999999945	44.0253225168845\\
8.31999999999945	44.0165192130419\\
8.32199999999945	44.0077176695078\\
8.32399999999945	43.9989178859308\\
8.32599999999945	43.9901198619583\\
8.32799999999945	43.9813235972386\\
8.32999999999945	43.9725290914209\\
8.33199999999945	43.9637363441515\\
8.33399999999945	43.9549453550805\\
8.33599999999945	43.9461561238559\\
8.33799999999945	43.9373686501266\\
8.33999999999945	43.9285829335393\\
8.34199999999946	43.9197989737443\\
8.34399999999946	43.9110167703909\\
8.34599999999946	43.9022363231258\\
8.34799999999946	43.8934576316\\
8.34999999999946	43.8846806954603\\
8.35199999999946	43.8759055143574\\
8.35399999999946	43.8671320879396\\
8.35599999999946	43.858360415857\\
8.35799999999946	43.8495904977573\\
8.35999999999946	43.8408223332906\\
8.36199999999946	43.8320559221064\\
8.36399999999946	43.8232912638533\\
8.36599999999946	43.8145283581818\\
8.36799999999946	43.8057672047413\\
8.36999999999946	43.7970078031802\\
8.37199999999947	43.7882501531497\\
8.37399999999947	43.7794942542988\\
8.37599999999947	43.7707401062773\\
8.37799999999947	43.7619877087356\\
8.37999999999947	43.7532370613234\\
8.38199999999947	43.7444881636906\\
8.38399999999947	43.7357410154873\\
8.38599999999947	43.7269956163643\\
8.38799999999947	43.718251965971\\
8.38999999999947	43.7095100639582\\
8.39199999999947	43.7007699099761\\
8.39399999999947	43.6920315036756\\
8.39599999999947	43.6832948447065\\
8.39799999999947	43.6745599327203\\
8.39999999999947	43.6658267673668\\
8.40199999999948	43.6570953482971\\
8.40399999999948	43.648365675162\\
8.40599999999948	43.6396377476123\\
8.40799999999948	43.6309115652988\\
8.40999999999948	43.6221871278739\\
8.41199999999948	43.6134644349864\\
8.41399999999948	43.6047434862895\\
8.41599999999948	43.5960242814332\\
8.41799999999948	43.587306820069\\
8.41999999999948	43.5785911018488\\
8.42199999999948	43.5698771264235\\
8.42399999999948	43.5611648934447\\
8.42599999999948	43.5524544025645\\
8.42799999999948	43.5437456534337\\
8.42999999999948	43.5350386457046\\
8.43199999999949	43.5263333790282\\
8.43399999999949	43.5176298530578\\
8.43599999999949	43.5089280674443\\
8.43799999999949	43.5002280218399\\
8.43999999999949	43.4915297158973\\
8.44199999999949	43.4828331492671\\
8.44399999999949	43.4741383216027\\
8.44599999999949	43.4654452325564\\
8.44799999999949	43.4567538817798\\
8.44999999999949	43.448064268926\\
8.45199999999949	43.4393763936471\\
8.45399999999949	43.4306902555961\\
8.45599999999949	43.4220058544254\\
8.45799999999949	43.4133231897872\\
8.45999999999949	43.4046422613354\\
8.4619999999995	43.3959630687215\\
8.4639999999995	43.3872856115989\\
8.4659999999995	43.3786098896212\\
8.4679999999995	43.3699359024407\\
8.4699999999995	43.3612636497111\\
8.4719999999995	43.3525931310847\\
8.4739999999995	43.3439243462149\\
8.4759999999995	43.3352572947562\\
8.4779999999995	43.3265919763609\\
8.4799999999995	43.317928390683\\
8.4819999999995	43.3092665373754\\
8.4839999999995	43.3006064160923\\
8.4859999999995	43.2919480264871\\
8.4879999999995	43.283291368213\\
8.4899999999995	43.2746364409252\\
8.49199999999951	43.2659832442766\\
8.49399999999951	43.2573317779208\\
8.49599999999951	43.2486820415121\\
8.49799999999951	43.2400340347053\\
8.49999999999951	43.2313877571538\\
8.50199999999951	43.2227432085123\\
8.50399999999951	43.2141003884346\\
8.50599999999951	43.2054592965751\\
8.50799999999951	43.1968199325888\\
8.50999999999951	43.1881822961295\\
8.51199999999951	43.1795463868523\\
8.51399999999951	43.1709122044115\\
8.51599999999951	43.1622797484614\\
8.51799999999951	43.1536490186581\\
8.51999999999951	43.145020014655\\
8.52199999999952	43.1363927361075\\
8.52399999999952	43.1277671826712\\
8.52599999999952	43.1191433540002\\
8.52799999999952	43.11052124975\\
8.52999999999952	43.1019008695763\\
8.53199999999952	43.0932822131339\\
8.53399999999952	43.0846652800775\\
8.53599999999952	43.0760500700641\\
8.53799999999952	43.0674365827472\\
8.53999999999952	43.0588248177836\\
8.54199999999952	43.0502147748285\\
8.54399999999952	43.0416064535383\\
8.54599999999952	43.0329998535673\\
8.54799999999952	43.024394974572\\
8.54999999999952	43.0157918162089\\
8.55199999999953	43.0071903781334\\
8.55399999999953	42.9985906600015\\
8.55599999999953	42.9899926614695\\
8.55799999999953	42.981396382193\\
8.55999999999953	42.9728018218283\\
8.56199999999953	42.9642089800329\\
8.56399999999953	42.9556178564613\\
8.56599999999953	42.9470284507711\\
8.56799999999953	42.938440762619\\
8.56999999999953	42.92985479166\\
8.57199999999953	42.9212705375525\\
8.57399999999953	42.9126879999529\\
8.57599999999953	42.9041071785166\\
8.57799999999953	42.8955280729024\\
8.57999999999953	42.8869506827656\\
8.58199999999954	42.878375007764\\
8.58399999999954	42.8698010475545\\
8.58599999999954	42.8612288017942\\
8.58799999999954	42.8526582701407\\
8.58999999999954	42.8440894522499\\
8.59199999999954	42.8355223477804\\
8.59399999999954	42.826956956389\\
8.59599999999954	42.8183932777339\\
8.59799999999954	42.8098313114713\\
8.59999999999954	42.80127105726\\
8.60199999999954	42.792712514757\\
8.60399999999954	42.7841556836198\\
8.60599999999954	42.7756005635071\\
8.60799999999954	42.7670471540765\\
8.60999999999954	42.7584954549853\\
8.61199999999955	42.7499454658927\\
8.61399999999955	42.741397186455\\
8.61599999999955	42.7328506163324\\
8.61799999999955	42.7243057551808\\
8.61999999999955	42.7157626026605\\
8.62199999999955	42.7072211584287\\
8.62399999999955	42.6986814221444\\
8.62599999999955	42.6901433934652\\
8.62799999999955	42.6816070720513\\
8.62999999999955	42.6730724575596\\
8.63199999999955	42.6645395496497\\
8.63399999999955	42.6560083479799\\
8.63599999999955	42.6474788522101\\
8.63799999999955	42.6389510619971\\
8.63999999999955	42.6304249770016\\
8.64199999999956	42.621900596883\\
8.64399999999956	42.6133779212983\\
8.64599999999956	42.6048569499084\\
8.64799999999956	42.5963376823717\\
8.64999999999956	42.5878201183479\\
8.65199999999956	42.5793042574966\\
8.65399999999956	42.5707900994767\\
8.65599999999956	42.5622776439477\\
8.65799999999956	42.5537668905696\\
8.65999999999956	42.545257839002\\
8.66199999999956	42.5367504889043\\
8.66399999999956	42.5282448399361\\
8.66599999999956	42.5197408917582\\
8.66799999999956	42.5112386440293\\
8.66999999999956	42.5027380964098\\
8.67199999999957	42.4942392485599\\
8.67399999999957	42.4857421001398\\
8.67599999999957	42.4772466508099\\
8.67799999999957	42.4687529002297\\
8.67999999999957	42.4602608480602\\
8.68199999999957	42.4517704939614\\
8.68399999999957	42.4432818375935\\
8.68599999999957	42.4347948786191\\
8.68799999999957	42.4263096166949\\
8.68999999999957	42.4178260514844\\
8.69199999999957	42.4093441826477\\
8.69399999999957	42.400864009846\\
8.69599999999957	42.3923855327395\\
8.69799999999957	42.3839087509893\\
8.69999999999957	42.3754336642562\\
8.70199999999958	42.366960272202\\
8.70399999999958	42.3584885744871\\
8.70599999999958	42.3500185707732\\
8.70799999999958	42.3415502607208\\
8.70999999999958	42.333083643992\\
8.71199999999958	42.3246187202479\\
8.71399999999958	42.31615548915\\
8.71599999999958	42.3076939503599\\
8.71799999999958	42.2992341035392\\
8.71999999999958	42.2907759483495\\
8.72199999999958	42.2823194844528\\
8.72399999999958	42.2738647115104\\
8.72599999999958	42.2654116291847\\
8.72799999999958	42.2569602371372\\
8.72999999999958	42.24851053503\\
8.73199999999959	42.2400625225257\\
8.73399999999959	42.2316161992859\\
8.73599999999959	42.2231715649729\\
8.73799999999959	42.2147286192486\\
8.73999999999959	42.2062873617763\\
8.74199999999959	42.1978477922181\\
8.74399999999959	42.1894099102362\\
8.74599999999959	42.1809737154932\\
8.74799999999959	42.1725392076515\\
8.74999999999959	42.1641063863739\\
8.75199999999959	42.1556752513239\\
8.75399999999959	42.1472458021633\\
8.75599999999959	42.1388180385557\\
8.75799999999959	42.1303919601638\\
8.75999999999959	42.1219675666504\\
8.7619999999996	42.1135448576788\\
8.7639999999996	42.1051238329123\\
8.7659999999996	42.0967044920137\\
8.7679999999996	42.088286834647\\
8.7699999999996	42.0798708604746\\
8.7719999999996	42.0714565691613\\
8.7739999999996	42.0630439603687\\
8.7759999999996	42.0546330337623\\
8.7779999999996	42.0462237890043\\
8.7799999999996	42.0378162257593\\
8.7819999999996	42.0294103436904\\
8.7839999999996	42.0210061424623\\
8.7859999999996	42.0126036217376\\
8.7879999999996	42.0042027811816\\
8.7899999999996	41.995803620457\\
8.79199999999961	41.9874061392297\\
8.79399999999961	41.9790103371618\\
8.79599999999961	41.9706162139196\\
8.79799999999961	41.9622237691653\\
8.79999999999961	41.9538330025651\\
8.80199999999961	41.9454439137819\\
8.80399999999961	41.9370565024814\\
8.80599999999961	41.9286707683281\\
8.80799999999961	41.920286710986\\
8.80999999999961	41.9119043301198\\
8.81199999999961	41.9035236253947\\
8.81399999999961	41.8951445964757\\
8.81599999999961	41.8867672430271\\
8.81799999999961	41.8783915647141\\
8.81999999999961	41.8700175612021\\
8.82199999999962	41.8616452321549\\
8.82399999999962	41.8532745772398\\
8.82599999999962	41.8449055961203\\
8.82799999999962	41.8365382884628\\
8.82999999999962	41.8281726539319\\
8.83199999999962	41.8198086921941\\
8.83399999999962	41.811446402913\\
8.83599999999962	41.8030857857558\\
8.83799999999962	41.7947268403876\\
8.83999999999962	41.7863695664749\\
8.84199999999962	41.7780139636816\\
8.84399999999962	41.7696600316753\\
8.84599999999962	41.7613077701212\\
8.84799999999962	41.7529571786856\\
8.84999999999962	41.7446082570341\\
8.85199999999963	41.7362610048332\\
8.85399999999963	41.7279154217491\\
8.85599999999963	41.7195715074476\\
8.85799999999963	41.7112292615953\\
8.85999999999963	41.7028886838583\\
8.86199999999963	41.6945497739035\\
8.86399999999963	41.6862125313973\\
8.86599999999963	41.677876956006\\
8.86799999999963	41.6695430473968\\
8.86999999999963	41.6612108052357\\
8.87199999999963	41.6528802291897\\
8.87399999999963	41.6445513189259\\
8.87599999999963	41.6362240741118\\
8.87799999999963	41.6278984944121\\
8.87999999999963	41.6195745794961\\
8.88199999999964	41.6112523290302\\
8.88399999999964	41.6029317426819\\
8.88599999999964	41.594612820118\\
8.88799999999964	41.586295561006\\
8.88999999999964	41.5779799650126\\
8.89199999999964	41.5696660318067\\
8.89399999999964	41.5613537610543\\
8.89599999999964	41.5530431524235\\
8.89799999999964	41.5447342055825\\
8.89999999999964	41.5364269201987\\
8.90199999999964	41.5281212959398\\
8.90399999999964	41.5198173324729\\
8.90599999999964	41.511515029467\\
8.90799999999964	41.5032143865896\\
8.90999999999964	41.4949154035087\\
8.91199999999965	41.486618079893\\
8.91399999999965	41.4783224154097\\
8.91599999999965	41.4700284097282\\
8.91799999999965	41.4617360625153\\
8.91999999999965	41.4534453734408\\
8.92199999999965	41.4451563421725\\
8.92399999999965	41.436868968379\\
8.92599999999965	41.4285832517282\\
8.92799999999965	41.4202991918904\\
8.92999999999965	41.4120167885323\\
8.93199999999965	41.4037360413239\\
8.93399999999965	41.395456949934\\
8.93599999999965	41.3871795140309\\
8.93799999999965	41.3789037332845\\
8.93999999999965	41.3706296073632\\
8.94199999999966	41.3623571359359\\
8.94399999999966	41.3540863186723\\
8.94599999999966	41.3458171552408\\
8.94799999999966	41.337549645312\\
8.94999999999966	41.3292837885541\\
8.95199999999966	41.3210195846373\\
8.95399999999966	41.3127570332308\\
8.95599999999966	41.3044961340039\\
8.95799999999966	41.2962368866267\\
8.95999999999966	41.2879792907685\\
8.96199999999966	41.2797233460993\\
8.96399999999966	41.2714690522887\\
8.96599999999966	41.2632164090067\\
8.96799999999966	41.2549654159238\\
8.96999999999966	41.2467160727091\\
8.97199999999967	41.2384683790335\\
8.97399999999967	41.2302223345664\\
8.97599999999967	41.2219779389786\\
8.97799999999967	41.2137351919403\\
8.97999999999967	41.2054940931217\\
8.98199999999967	41.1972546421933\\
8.98399999999967	41.1890168388256\\
8.98599999999967	41.1807806826891\\
8.98799999999967	41.1725461734545\\
8.98999999999967	41.1643133107921\\
8.99199999999967	41.156082094373\\
8.99399999999967	41.1478525238684\\
8.99599999999967	41.1396245989487\\
8.99799999999967	41.1313983192847\\
8.99999999999967	41.123173684548\\
9.00199999999968	41.114950694409\\
9.00399999999968	41.1067293485393\\
9.00599999999968	41.0985096466099\\
9.00799999999968	41.0902915882921\\
9.00999999999968	41.0820751732575\\
9.01199999999968	41.0738604011775\\
9.01399999999968	41.0656472717233\\
9.01599999999968	41.0574357845661\\
9.01799999999968	41.0492259393782\\
9.01999999999968	41.0410177358311\\
9.02199999999968	41.0328111735965\\
9.02399999999968	41.0246062523462\\
9.02599999999968	41.0164029717516\\
9.02799999999968	41.0082013314853\\
9.02999999999968	41.0000013312189\\
9.03199999999969	40.9918029706248\\
9.03399999999969	40.983606249375\\
9.03599999999969	40.975411167142\\
9.03799999999969	40.9672177235969\\
9.03999999999969	40.9590259184132\\
9.04199999999969	40.9508357512634\\
9.04399999999969	40.9426472218187\\
9.04599999999969	40.9344603297525\\
9.04799999999969	40.9262750747376\\
9.04999999999969	40.9180914564463\\
9.05199999999969	40.9099094745518\\
9.05399999999969	40.901729128726\\
9.05599999999969	40.893550418642\\
9.05799999999969	40.8853733439729\\
9.05999999999969	40.8771979043922\\
9.0619999999997	40.8690240995726\\
9.0639999999997	40.8608519291864\\
9.0659999999997	40.852681392908\\
9.0679999999997	40.8445124904102\\
9.0699999999997	40.8363452213657\\
9.0719999999997	40.8281795854485\\
9.0739999999997	40.8200155823321\\
9.0759999999997	40.8118532116897\\
9.0779999999997	40.8036924731951\\
9.0799999999997	40.7955333665218\\
9.0819999999997	40.7873758913435\\
9.0839999999997	40.7792200473343\\
9.0859999999997	40.7710658341674\\
9.0879999999997	40.7629132515169\\
9.0899999999997	40.7547622990571\\
9.09199999999971	40.7466129764615\\
9.09399999999971	40.738465283405\\
9.09599999999971	40.730319219561\\
9.09799999999971	40.7221747846043\\
9.09999999999971	40.7140319782087\\
9.10199999999971	40.7058908000487\\
9.10399999999971	40.6977512497986\\
9.10599999999971	40.6896133271333\\
9.10799999999971	40.681477031727\\
9.10999999999971	40.6733423632544\\
9.11199999999971	40.66520932139\\
9.11399999999971	40.6570779058091\\
9.11599999999971	40.6489481161855\\
9.11799999999971	40.640819952195\\
9.11999999999972	40.6326934135124\\
9.12199999999972	40.6245684998125\\
9.12399999999972	40.6164452107706\\
9.12599999999972	40.6083235460612\\
9.12799999999972	40.6002035053606\\
9.12999999999972	40.5920850883422\\
9.13199999999972	40.5839682946836\\
9.13399999999972	40.5758531240587\\
9.13599999999972	40.5677395761435\\
9.13799999999972	40.5596276506134\\
9.13999999999972	40.5515173471439\\
9.14199999999972	40.5434086654108\\
9.14399999999972	40.5353016050899\\
9.14599999999972	40.5271961658567\\
9.14799999999972	40.5190923473872\\
9.14999999999973	40.5109901493572\\
9.15199999999973	40.5028895714431\\
9.15399999999973	40.4947906133204\\
9.15599999999973	40.4866932746659\\
9.15799999999973	40.4785975551545\\
9.15999999999973	40.4705034544633\\
9.16199999999973	40.4624109722693\\
9.16399999999973	40.4543201082473\\
9.16599999999973	40.4462308620751\\
9.16799999999973	40.4381432334285\\
9.16999999999973	40.4300572219838\\
9.17199999999973	40.4219728274183\\
9.17399999999973	40.4138900494085\\
9.17599999999973	40.4058088876311\\
9.17799999999973	40.3977293417627\\
9.17999999999974	40.3896514114802\\
9.18199999999974	40.381575096461\\
9.18399999999974	40.3735003963818\\
9.18599999999974	40.3654273109196\\
9.18799999999974	40.3573558397517\\
9.18999999999974	40.3492859825555\\
9.19199999999974	40.3412177390075\\
9.19399999999974	40.3331511087855\\
9.19599999999974	40.3250860915673\\
9.19799999999974	40.3170226870298\\
9.19999999999974	40.3089608948509\\
9.20199999999974	40.3009007147081\\
9.20399999999974	40.2928421462786\\
9.20599999999974	40.2847851892408\\
9.20799999999974	40.2767298432723\\
9.20999999999975	40.2686761080505\\
9.21199999999975	40.260623983254\\
9.21399999999975	40.2525734685603\\
9.21599999999975	40.2445245636476\\
9.21799999999975	40.2364772681939\\
9.21999999999975	40.2284315818774\\
9.22199999999975	40.2203875043768\\
9.22399999999975	40.2123450353694\\
9.22599999999975	40.2043041745345\\
9.22799999999975	40.1962649215504\\
9.22999999999975	40.1882272760951\\
9.23199999999975	40.1801912378478\\
9.23399999999975	40.1721568064863\\
9.23599999999975	40.1641239816904\\
9.23799999999975	40.1560927631374\\
9.23999999999976	40.1480631505072\\
9.24199999999976	40.1400351434787\\
9.24399999999976	40.1320087417301\\
9.24599999999976	40.123983944941\\
9.24799999999976	40.1159607527904\\
9.24999999999976	40.1079391649577\\
9.25199999999976	40.0999191811214\\
9.25399999999976	40.0919008009612\\
9.25599999999976	40.0838840241564\\
9.25799999999976	40.0758688503863\\
9.25999999999976	40.0678552793303\\
9.26199999999976	40.0598433106683\\
9.26399999999976	40.0518329440794\\
9.26599999999976	40.0438241792437\\
9.26799999999976	40.0358170158404\\
9.26999999999977	40.0278114535497\\
9.27199999999977	40.0198074920514\\
9.27399999999977	40.0118051310251\\
9.27599999999977	40.003804370151\\
9.27799999999977	39.9958052091094\\
9.27999999999977	39.98780764758\\
9.28199999999977	39.9798116852425\\
9.28399999999977	39.9718173217783\\
9.28599999999977	39.9638245568669\\
9.28799999999977	39.9558333901892\\
9.28999999999977	39.9478438214247\\
9.29199999999977	39.9398558502546\\
9.29399999999977	39.9318694763596\\
9.29599999999977	39.9238846994194\\
9.29799999999977	39.9159015191157\\
9.29999999999978	39.9079199351284\\
9.30199999999978	39.8999399471391\\
9.30399999999978	39.891961554828\\
9.30599999999978	39.8839847578764\\
9.30799999999978	39.8760095559653\\
9.30999999999978	39.8680359487759\\
9.31199999999978	39.8600639359883\\
9.31399999999978	39.8520935172848\\
9.31599999999978	39.8441246923463\\
9.31799999999978	39.8361574608544\\
9.31999999999978	39.82819182249\\
9.32199999999978	39.8202277769344\\
9.32399999999978	39.8122653238695\\
9.32599999999978	39.8043044629773\\
9.32799999999978	39.7963451939388\\
9.32999999999979	39.7883875164352\\
9.33199999999979	39.7804314301491\\
9.33399999999979	39.7724769347619\\
9.33599999999979	39.7645240299561\\
9.33799999999979	39.7565727154129\\
9.33999999999979	39.7486229908147\\
9.34199999999979	39.7406748558436\\
9.34399999999979	39.7327283101816\\
9.34599999999979	39.7247833535108\\
9.34799999999979	39.7168399855138\\
9.34999999999979	39.7088982058725\\
9.35199999999979	39.7009580142696\\
9.35399999999979	39.6930194103877\\
9.35599999999979	39.685082393909\\
9.35799999999979	39.677146964516\\
9.3599999999998	39.6692131218916\\
9.3619999999998	39.6612808657186\\
9.3639999999998	39.6533501956795\\
9.3659999999998	39.6454211114572\\
9.3679999999998	39.6374936127343\\
9.3699999999998	39.6295676991945\\
9.3719999999998	39.6216433705206\\
9.3739999999998	39.6137206263954\\
9.3759999999998	39.605799466502\\
9.3779999999998	39.5978798905243\\
9.3799999999998	39.5899618981441\\
9.3819999999998	39.5820454890463\\
9.3839999999998	39.5741306629139\\
9.3859999999998	39.5662174194299\\
9.3879999999998	39.5583057582781\\
9.38999999999981	39.5503956791425\\
9.39199999999981	39.5424871817063\\
9.39399999999981	39.534580265653\\
9.39599999999981	39.5266749306672\\
9.39799999999981	39.5187711764314\\
9.39999999999981	39.510869002631\\
9.40199999999981	39.5029684089494\\
9.40399999999981	39.4950693950701\\
9.40599999999981	39.4871719606782\\
9.40799999999981	39.479276105457\\
9.40999999999981	39.4713818290912\\
9.41199999999981	39.4634891312649\\
9.41399999999981	39.4555980116625\\
9.41599999999981	39.4477084699686\\
9.41799999999981	39.4398205058675\\
9.41999999999982	39.4319341190435\\
9.42199999999982	39.424049309182\\
9.42399999999982	39.416166075967\\
9.42599999999982	39.4082844190827\\
9.42799999999982	39.4004043382151\\
9.42999999999982	39.3925258330484\\
9.43199999999982	39.384648903268\\
9.43399999999982	39.3767735485582\\
9.43599999999982	39.3688997686046\\
9.43799999999982	39.361027563092\\
9.43999999999982	39.3531569317058\\
9.44199999999982	39.3452878741305\\
9.44399999999982	39.3374203900528\\
9.44599999999982	39.3295544791569\\
9.44799999999982	39.3216901411288\\
9.44999999999983	39.3138273756535\\
9.45199999999983	39.3059661824169\\
9.45399999999983	39.2981065611048\\
9.45599999999983	39.2902485114023\\
9.45799999999983	39.2823920329959\\
9.45999999999983	39.2745371255709\\
9.46199999999983	39.2666837888131\\
9.46399999999983	39.2588320224085\\
9.46599999999983	39.2509818260433\\
9.46799999999983	39.2431331994035\\
9.46999999999983	39.2352861421748\\
9.47199999999983	39.2274406540442\\
9.47399999999983	39.2195967346972\\
9.47599999999983	39.2117543838205\\
9.47799999999983	39.2039136011003\\
9.47999999999984	39.1960743862229\\
9.48199999999984	39.1882367388754\\
9.48399999999984	39.1804006587431\\
9.48599999999984	39.1725661455145\\
9.48799999999984	39.1647331988747\\
9.48999999999984	39.1569018185111\\
9.49199999999984	39.1490720041101\\
9.49399999999984	39.141243755359\\
9.49599999999984	39.1334170719448\\
9.49799999999984	39.125591953554\\
9.49999999999984	39.117768399874\\
9.50199999999984	39.109946410592\\
9.50399999999984	39.1021259853949\\
9.50599999999984	39.09430712397\\
9.50799999999984	39.086489826005\\
9.50999999999985	39.0786740911865\\
9.51199999999985	39.0708599192025\\
9.51399999999985	39.0630473097407\\
9.51599999999985	39.0552362624883\\
9.51799999999985	39.0474267771328\\
9.51999999999985	39.0396188533623\\
9.52199999999985	39.031812490864\\
9.52399999999985	39.024007689326\\
9.52599999999985	39.0162044484363\\
9.52799999999985	39.0084027678829\\
9.52999999999985	39.0006026473533\\
9.53199999999985	38.9928040865361\\
9.53399999999985	38.9850070851191\\
9.53599999999985	38.9772116427904\\
9.53799999999985	38.9694177592385\\
9.53999999999986	38.961625434152\\
9.54199999999986	38.9538346672182\\
9.54399999999986	38.9460454581266\\
9.54599999999986	38.9382578065653\\
9.54799999999986	38.930471712223\\
9.54999999999986	38.9226871747879\\
9.55199999999986	38.9149041939494\\
9.55399999999986	38.9071227693952\\
9.55599999999986	38.8993429008154\\
9.55799999999986	38.8915645878979\\
9.55999999999986	38.883787830332\\
9.56199999999986	38.8760126278061\\
9.56399999999986	38.86823898001\\
9.56599999999986	38.8604668866327\\
9.56799999999986	38.8526963473631\\
9.56999999999987	38.8449273618908\\
9.57199999999987	38.8371599299049\\
9.57399999999987	38.8293940510946\\
9.57599999999987	38.8216297251499\\
9.57799999999987	38.8138669517591\\
9.57999999999987	38.8061057306131\\
9.58199999999987	38.7983460614009\\
9.58399999999987	38.7905879438124\\
9.58599999999987	38.7828313775365\\
9.58799999999987	38.775076362264\\
9.58999999999987	38.7673228976847\\
9.59199999999987	38.7595709834877\\
9.59399999999987	38.751820619364\\
9.59599999999987	38.7440718050029\\
9.59799999999987	38.736324540095\\
9.59999999999988	38.7285788243301\\
9.60199999999988	38.7208346573985\\
9.60399999999988	38.7130920389907\\
9.60599999999988	38.7053509687969\\
9.60799999999988	38.6976114465085\\
9.60999999999988	38.6898734718134\\
9.61199999999988	38.6821370444044\\
9.61399999999988	38.6744021639715\\
9.61599999999988	38.6666688302056\\
9.61799999999988	38.6589370427972\\
9.61999999999988	38.6512068014369\\
9.62199999999988	38.6434781058156\\
9.62399999999988	38.6357509556247\\
9.62599999999988	38.6280253505543\\
9.62799999999988	38.6203012902964\\
9.62999999999989	38.6125787745413\\
9.63199999999989	38.604857802981\\
9.63399999999989	38.5971383753059\\
9.63599999999989	38.5894204912077\\
9.63799999999989	38.5817041503774\\
9.63999999999989	38.5739893525071\\
9.64199999999989	38.5662760972874\\
9.64399999999989	38.5585643844106\\
9.64599999999989	38.5508542135684\\
9.64799999999989	38.543145584451\\
9.64999999999989	38.5354384967521\\
9.65199999999989	38.5277329501615\\
9.65399999999989	38.5200289443728\\
9.65599999999989	38.5123264790768\\
9.65799999999989	38.5046255539661\\
9.6599999999999	38.4969261687324\\
9.6619999999999	38.4892283230678\\
9.6639999999999	38.4815320166646\\
9.6659999999999	38.4738372492146\\
9.6679999999999	38.4661440204107\\
9.6699999999999	38.4584523299443\\
9.6719999999999	38.4507621775089\\
9.6739999999999	38.4430735627973\\
9.6759999999999	38.4353864854995\\
9.6779999999999	38.4277009453106\\
9.6799999999999	38.4200169419221\\
9.6819999999999	38.4123344750271\\
9.6839999999999	38.4046535443183\\
9.6859999999999	38.3969741494882\\
9.6879999999999	38.38929629023\\
9.68999999999991	38.3816199662367\\
9.69199999999991	38.3739451772017\\
9.69399999999991	38.3662719228168\\
9.69599999999991	38.3586002027765\\
9.69799999999991	38.350930016773\\
9.69999999999991	38.3432613645003\\
9.70199999999991	38.3355942456512\\
9.70399999999991	38.3279286599192\\
9.70599999999991	38.3202646069975\\
9.70799999999991	38.31260208658\\
9.70999999999991	38.3049410983605\\
9.71199999999991	38.2972816420322\\
9.71399999999991	38.2896237172891\\
9.71599999999991	38.2819673238242\\
9.71799999999991	38.2743124613316\\
9.71999999999992	38.2666591295061\\
9.72199999999992	38.2590073280403\\
9.72399999999992	38.2513570566288\\
9.72599999999992	38.2437083149658\\
9.72799999999992	38.2360611027454\\
9.72999999999992	38.2284154196613\\
9.73199999999992	38.2207712654082\\
9.73399999999992	38.2131286396804\\
9.73599999999992	38.2054875421721\\
9.73799999999992	38.1978479725773\\
9.73999999999992	38.1902099305914\\
9.74199999999992	38.182573415908\\
9.74399999999992	38.1749384282226\\
9.74599999999992	38.1673049672291\\
9.74799999999992	38.1596730326225\\
9.74999999999993	38.1520426240979\\
9.75199999999993	38.1444137413496\\
9.75399999999993	38.1367863840727\\
9.75599999999993	38.1291605519624\\
9.75799999999993	38.1215362447133\\
9.75999999999993	38.1139134620208\\
9.76199999999993	38.1062922035801\\
9.76399999999993	38.0986724690862\\
9.76599999999993	38.0910542582345\\
9.76799999999993	38.0834375707204\\
9.76999999999993	38.0758224062395\\
9.77199999999993	38.0682087644866\\
9.77399999999993	38.0605966451575\\
9.77599999999993	38.0529860479477\\
9.77799999999993	38.0453769725534\\
9.77999999999994	38.0377694186694\\
9.78199999999994	38.0301633859924\\
9.78399999999994	38.0225588742174\\
9.78599999999994	38.0149558830409\\
9.78799999999994	38.0073544121583\\
9.78999999999994	37.9997544612662\\
9.79199999999994	37.9921560300602\\
9.79399999999994	37.9845591182368\\
9.79599999999994	37.9769637254916\\
9.79799999999994	37.9693698515216\\
9.79999999999994	37.9617774960224\\
9.80199999999994	37.9541866586903\\
9.80399999999994	37.9465973392222\\
9.80599999999994	37.9390095373148\\
9.80799999999994	37.9314232526644\\
9.80999999999995	37.9238384849673\\
9.81199999999995	37.9162552339205\\
9.81399999999995	37.9086734992211\\
9.81599999999995	37.9010932805645\\
9.81799999999995	37.8935145776493\\
9.81999999999995	37.885937390171\\
9.82199999999995	37.8783617178276\\
9.82399999999995	37.8707875603157\\
9.82599999999995	37.8632149173318\\
9.82799999999995	37.8556437885742\\
9.82999999999995	37.8480741737397\\
9.83199999999995	37.840506072525\\
9.83399999999995	37.832939484628\\
9.83599999999995	37.8253744097462\\
9.83799999999995	37.8178108475768\\
9.83999999999996	37.8102487978172\\
9.84199999999996	37.802688260165\\
9.84399999999996	37.7951292343181\\
9.84599999999996	37.7875717199748\\
9.84799999999996	37.7800157168312\\
9.84999999999996	37.7724612245858\\
9.85199999999996	37.7649082429373\\
9.85399999999996	37.757356771583\\
9.85599999999996	37.749806810221\\
9.85799999999996	37.7422583585491\\
9.85999999999996	37.7347114162662\\
9.86199999999996	37.7271659830694\\
9.86399999999996	37.7196220586576\\
9.86599999999996	37.7120796427291\\
9.86799999999996	37.7045387349821\\
9.86999999999997	37.696999335115\\
9.87199999999997	37.6894614428266\\
9.87399999999997	37.6819250578148\\
9.87599999999997	37.6743901797791\\
9.87799999999997	37.6668568084176\\
9.87999999999997	37.6593249434287\\
9.88199999999997	37.6517945845117\\
9.88399999999997	37.6442657313655\\
9.88599999999997	37.6367383836887\\
9.88799999999997	37.6292125411806\\
9.88999999999997	37.6216882035396\\
9.89199999999997	37.6141653704657\\
9.89399999999997	37.6066440416572\\
9.89599999999997	37.5991242168143\\
9.89799999999997	37.5916058956349\\
9.89999999999998	37.5840890778191\\
9.90199999999998	37.5765737630664\\
9.90399999999998	37.5690599510765\\
9.90599999999998	37.5615476415481\\
9.90799999999998	37.5540368341814\\
9.90999999999998	37.5465275286757\\
9.91199999999998	37.5390197247309\\
9.91399999999998	37.5315134220467\\
9.91599999999998	37.5240086203223\\
9.91799999999998	37.5165053192581\\
9.91999999999998	37.5090035185547\\
9.92199999999998	37.5015032179111\\
9.92399999999998	37.4940044170275\\
9.92599999999998	37.4865071156044\\
9.92799999999998	37.4790113133418\\
9.92999999999999	37.47151700994\\
9.93199999999999	37.4640242050988\\
9.93399999999999	37.456532898519\\
9.93599999999999	37.4490430899013\\
9.93799999999999	37.4415547789453\\
9.93999999999999	37.4340679653522\\
9.94199999999999	37.4265826488224\\
9.94399999999999	37.4190988290566\\
9.94599999999999	37.4116165057556\\
9.94799999999999	37.40413567862\\
9.94999999999999	37.3966563473503\\
9.95199999999999	37.389178511648\\
9.95399999999999	37.3817021712139\\
9.95599999999999	37.3742273257486\\
9.95799999999999	37.3667539749539\\
9.96	37.3592821185298\\
9.962	37.3518117561787\\
9.964	37.3443428876011\\
9.966	37.3368755124987\\
9.968	37.3294096305726\\
9.97	37.3219452415243\\
9.972	37.3144823450553\\
9.974	37.3070209408672\\
9.976	37.2995610286613\\
9.978	37.2921026081399\\
9.98	37.2846456790039\\
9.982	37.2771902409556\\
9.984	37.269736293697\\
9.986	37.2622838369295\\
9.988	37.2548328703555\\
9.99000000000001	37.2473833936767\\
9.99200000000001	37.2399354065957\\
9.99400000000001	37.2324889088135\\
9.99600000000001	37.2250439000336\\
9.99800000000001	37.2176003799575\\
10	37.2101583482885\\
};
\end{axis}

\begin{axis}[%
width=2.603in,
height=1.074in,
at={(4.436in,0.751in)},
scale only axis,
xmin=0,
xmax=10,
xlabel style={font=\color{white!15!black}},
xlabel={t},
ymode=log,
ymin=41.62434973898,
ymax=190431.907948595,
yminorticks=true,
ylabel style={font=\color{white!15!black}},
ylabel={indice stiff},
axis background/.style={fill=white},
title style={font=\bfseries},
title={N=60}
]
\addplot [color=mycolor1, forget plot]
  table[row sep=crcr]{%
0	7403.08304848238\\
0.002	190431.907948595\\
0.004	94979.2407270139\\
0.006	63159.6311672952\\
0.008	47248.2387996526\\
0.01	37700.0974533226\\
0.012	31333.5533501286\\
0.014	26785.0419731355\\
0.016	23372.7822821848\\
0.018	20718.0081176378\\
0.02	18593.460760651\\
0.022	16854.521913934\\
0.024	15404.7801911705\\
0.026	14177.4899949864\\
0.028	13124.9765892686\\
0.03	12212.2791418117\\
0.032	11413.1776252539\\
0.034	10707.6219876888\\
0.036	10080.018265811\\
0.038	9518.05592497356\\
0.04	9011.88699674972\\
0.042	8553.5397481916\\
0.044	8136.49225906237\\
0.046	7755.3572394291\\
0.048	7405.64564252709\\
0.05	7083.58701023043\\
0.052	6785.99127697344\\
0.054	6510.14128367424\\
0.056	6253.70832414536\\
0.058	6014.68516446171\\
0.06	5791.33245826266\\
0.062	5582.13553311509\\
0.064	5385.76927928771\\
0.066	5201.06942224425\\
0.068	5027.00886457738\\
0.07	4862.6780835076\\
0.0720000000000001	4707.26879537115\\
0.0740000000000001	4560.06026902873\\
0.0760000000000001	4420.40780025237\\
0.0780000000000001	4287.73295922024\\
0.0800000000000001	4161.51530084232\\
0.0820000000000001	4041.28528817348\\
0.0840000000000001	3926.61822673614\\
0.0860000000000001	3817.12904519878\\
0.0880000000000001	3712.46778776338\\
0.0900000000000001	3612.31570756978\\
0.0920000000000001	3516.38186964818\\
0.0940000000000001	3424.40018755338\\
0.0960000000000001	3336.12683042702\\
0.0980000000000001	3251.3379475847\\
0.1	3169.82766617361\\
0.102	3091.40632442948\\
0.104	3015.89890882267\\
0.106	2943.1436681618\\
0.108	2872.99088173063\\
0.11	2805.30176185208\\
0.112	2739.9474740768\\
0.114	2676.80826056329\\
0.116	2615.77265418935\\
0.118	2556.7367726386\\
0.12	2499.60368313382\\
0.122	2444.28282971003\\
0.124	2390.68951597203\\
0.126	2338.74443716935\\
0.128	2288.37325620286\\
0.13	2239.50621882544\\
0.132	2192.07780388904\\
0.134	2146.02640497261\\
0.136	2101.29404016464\\
0.138	2057.82608714354\\
0.14	2015.57104102866\\
0.142	1974.48029275532\\
0.144	1934.50792598267\\
0.146	1895.61053075321\\
0.148	1857.74703232412\\
0.15	1820.87853375236\\
0.152	1784.96817096709\\
0.154	1749.98097919546\\
0.156	1715.88376972077\\
0.158	1682.64501606058\\
0.16	1650.23474873635\\
0.162	1618.62445789632\\
0.164	1587.78700311667\\
0.166	1557.6965297789\\
0.168	1528.32839147214\\
0.17	1499.65907792533\\
0.172	1471.66614801817\\
0.174	1444.328167462\\
0.176	1417.62465077722\\
0.178	1391.53600723085\\
0.18	1366.04349042277\\
0.182	1341.12915124228\\
0.184	1316.77579393512\\
0.186	1292.96693504747\\
0.188	1269.68676503183\\
0.19	1246.92011231685\\
0.192	1224.65240966248\\
0.194	1202.86966263175\\
0.196	1181.55842003063\\
0.198	1160.70574617299\\
0.2	1140.29919484378\\
0.202	1120.3267848406\\
0.204	1100.77697698598\\
0.206	1081.63865250822\\
0.208	1062.901092698\\
0.21	1044.55395975568\\
0.212	1026.5872787486\\
0.214	1008.99142060604\\
0.216	991.757086082598\\
0.218	974.875290628014\\
0.22	958.337350104365\\
0.222	942.134867296047\\
0.224	926.259719163151\\
0.226	910.704044790435\\
0.228	895.46023398888\\
0.23	880.520916509449\\
0.232	865.87895183091\\
0.234	851.527419486814\\
0.236	837.459609899438\\
0.238	823.669015689322\\
0.24	810.149323432431\\
0.242	796.894405838813\\
0.244	783.898314326628\\
0.246	771.155271970045\\
0.248	758.65966679806\\
0.25	746.40604542448\\
0.252	734.38910699004\\
0.254	722.603697399114\\
0.256	711.044803833676\\
0.258	699.707549529999\\
0.26	688.587188802385\\
0.262	677.679102301017\\
0.264	666.978792490587\\
0.266	656.481879337847\\
0.268	646.184096196798\\
0.27	636.08128588059\\
0.272	626.169396910228\\
0.274	616.444479930967\\
0.276	606.902684286947\\
0.278	597.540254746346\\
0.28	588.353528368709\\
0.282	579.338931507353\\
0.284	570.492976939732\\
0.286	561.812261119519\\
0.288	553.293461543458\\
0.29	544.933334228226\\
0.292	536.728711290752\\
0.294	528.676498627787\\
0.296	520.773673688817\\
0.298	513.017283338599\\
0.3	505.404441804316\\
0.302	497.932328703521\\
0.304	490.598187148978\\
0.306	483.399321926698\\
0.308	476.333097743228\\
0.31	469.39693753984\\
0.312	462.588320869574\\
0.314	455.904782334686\\
0.316	449.343910081593\\
0.318	442.903344350524\\
0.32	436.580776077648\\
0.322	430.373945547042\\
0.324	424.280641090405\\
0.326	418.298697832279\\
0.328	412.425996478703\\
0.33	406.660462147819\\
0.332	401.000063239836\\
0.334	395.442810345326\\
0.336	389.986755189755\\
0.338	384.629989613025\\
0.34	379.370644582297\\
0.342	374.206889236758\\
0.344	369.136929963108\\
0.346	364.15900950025\\
0.348	359.271406072241\\
0.35	354.472432548244\\
0.352	349.760435628272\\
0.354	345.133795053921\\
0.356	340.590922842886\\
0.358	336.130262546405\\
0.36	331.750288528789\\
0.362	327.44950526811\\
0.364	323.226446677153\\
0.366	319.079675444019\\
0.368	315.007782391694\\
0.37	311.009385855399\\
0.372	307.083131077803\\
0.374	303.22768962075\\
0.376	299.441758793346\\
0.378	295.724061095549\\
0.38	292.073343676935\\
0.382	288.488377809862\\
0.384	284.967958376763\\
0.386	281.510903370729\\
0.388	278.116053409496\\
0.39	274.782271261677\\
0.392	271.508441385443\\
0.394	268.293469478748\\
0.396	265.136282041127\\
0.398	262.035825946399\\
0.4	258.991068025934\\
0.402	256.000994662419\\
0.404	253.06461139339\\
0.406	250.18094252448\\
0.408	247.349030752168\\
0.41	244.567936795255\\
0.412	241.836739035504\\
0.414	239.154533166433\\
0.416	236.520431850658\\
0.418	233.933564385069\\
0.42	231.393076373783\\
0.422	228.89812940871\\
0.424	226.447900757469\\
0.426	224.041583058185\\
0.428	221.678384021651\\
0.43	219.35752613959\\
0.432	217.078246400037\\
0.434	214.839796008719\\
0.436	212.641440116676\\
0.438	210.482457553958\\
0.44	208.362140569155\\
0.442	206.279794574537\\
0.444	204.234737896864\\
0.446	202.226301533436\\
0.448	200.253828913639\\
0.45	198.31667566532\\
0.452	196.414209386471\\
0.454	194.545809421557\\
0.456	192.710866642702\\
0.458	190.908783235558\\
0.46	189.138972489558\\
0.462	187.400858592806\\
0.464	185.693876431041\\
0.466	184.017471391102\\
0.468	182.371099168244\\
0.47	180.754225577713\\
0.472	179.166326370153\\
0.474	177.606887050879\\
0.476	176.075402702966\\
0.478	174.571377814067\\
0.48	173.094326106709\\
0.482	171.643770372375\\
0.484	170.219242308884\\
0.486	168.820282361245\\
0.488	167.446439565853\\
0.49	166.097271398073\\
0.492	164.772343622857\\
0.494	163.471230148746\\
0.496	162.19351288487\\
0.498	160.938781601089\\
0.5	159.706633791125\\
0.502	158.496674538664\\
0.504	157.308516386441\\
0.506	156.141779208253\\
0.508	154.996090083717\\
0.51	153.871083176003\\
0.512	152.766399612285\\
0.514	151.68168736702\\
0.516	150.616601147889\\
0.518	149.570802284601\\
0.52	148.543958620303\\
0.522	147.535744405735\\
0.524	146.545840196065\\
0.526	145.573932750407\\
0.528	144.619714933994\\
0.53	143.682885623057\\
0.532	142.763149612398\\
0.534	141.860217525537\\
0.536	140.973805727747\\
0.538	140.103636241662\\
0.54	139.249436665636\\
0.542	138.410940094944\\
0.544	137.587885045747\\
0.546	136.780015381807\\
0.548	135.987080244197\\
0.55	135.20883398388\\
0.552	134.44503609727\\
0.554	133.6954511648\\
0.556	132.959848792644\\
0.558	132.23800355754\\
0.56	131.529694954937\\
0.562	130.834707350411\\
0.564	130.152829934551\\
0.566	129.483856681258\\
0.568	128.827586309767\\
0.57	128.183822250295\\
0.572	127.552372613517\\
0.574	126.933050163939\\
0.576	126.325672297307\\
0.578	125.730061022071\\
0.58	125.146042945091\\
0.582	124.573449261557\\
0.584	124.012115749261\\
0.586	123.461882767254\\
0.588	122.922595258776\\
0.59	122.394102758571\\
0.592	121.876259404421\\
0.594	121.368923952669\\
0.596	120.871959797664\\
0.598	120.385234994523\\
0.6	119.908622284937\\
0.602	119.441999125286\\
0.604	118.9852477161\\
0.606	118.538255032071\\
0.608	118.100912850913\\
0.61	117.673117779579\\
0.612	117.25477127569\\
0.614	116.845779661575\\
0.616	116.446054127851\\
0.618	116.055510722784\\
0.62	115.674070323138\\
0.622	115.301658581157\\
0.624	114.938205841817\\
0.626	114.583647023356\\
0.628	114.237921453129\\
0.63	113.900972650241\\
0.632	113.572748045077\\
0.634	113.253198625433\\
0.636	112.942278498436\\
0.638	112.639944357187\\
0.64	112.346154841577\\
0.642	112.060869783653\\
0.644	111.784049330116\\
0.646	111.515652937141\\
0.648	111.255638237377\\
0.65	111.003959784233\\
0.652	110.760567685464\\
0.654	110.525406146504\\
0.656	110.298411952619\\
0.658	110.079512928649\\
0.66	109.868626424446\\
0.662	109.665657881693\\
0.664	109.470499543777\\
0.666	109.283029372754\\
0.668	109.103110235511\\
0.67	108.930589414714\\
0.672	108.765298488549\\
0.674	108.607053607156\\
0.676	108.455656174266\\
0.678	108.310893921465\\
0.68	108.172542341094\\
0.682	108.040366425141\\
0.684	107.914122642194\\
0.686	107.793561074808\\
0.688000000000001	107.678427635513\\
0.690000000000001	107.56846628152\\
0.692000000000001	107.463421155057\\
0.694000000000001	107.363038587109\\
0.696000000000001	107.267068915639\\
0.698000000000001	107.175268083878\\
0.700000000000001	107.087398998521\\
0.702000000000001	107.00323264053\\
0.704000000000001	106.922548932516\\
0.706000000000001	106.845137375199\\
0.708000000000001	106.770797471866\\
0.710000000000001	106.699338963922\\
0.712000000000001	106.630581902518\\
0.714000000000001	106.564356582013\\
0.716000000000001	106.500503359946\\
0.718000000000001	106.438872386886\\
0.720000000000001	106.379323267084\\
0.722000000000001	106.321724668501\\
0.724000000000001	106.265953898204\\
0.726000000000001	106.211896456516\\
0.728000000000001	106.159445581084\\
0.730000000000001	106.108501789653\\
0.732000000000001	106.05897242873\\
0.734000000000001	106.010771233397\\
0.736000000000001	105.963817902283\\
0.738000000000001	105.918037690587\\
0.740000000000001	105.873361022914\\
0.742000000000001	105.829723127059\\
0.744000000000001	105.787063689303\\
0.746000000000001	105.74532653108\\
0.748000000000001	105.704459306921\\
0.750000000000001	105.664413222788\\
0.752000000000001	105.625142774227\\
0.754000000000001	105.586605503349\\
0.756000000000001	105.548761773589\\
0.758000000000001	105.511574561321\\
0.760000000000001	105.47500926315\\
0.762000000000001	105.439033517961\\
0.764000000000001	105.403617042571\\
0.766000000000001	105.368731480171\\
0.768000000000001	105.334350260476\\
0.770000000000001	105.300448470726\\
0.772000000000001	105.267002736815\\
0.774000000000001	105.233991113581\\
0.776000000000001	105.2013929837\\
0.778000000000001	105.169188964373\\
0.780000000000001	105.137360821256\\
0.782000000000001	105.105891389028\\
0.784000000000001	105.07476449805\\
0.786000000000001	105.043964906636\\
0.788000000000001	105.013478238452\\
0.790000000000001	104.983290924719\\
0.792000000000001	104.953390150622\\
0.794000000000001	104.923763805894\\
0.796000000000001	104.894400438913\\
0.798000000000001	104.865289214277\\
0.800000000000001	104.836419873469\\
0.802000000000001	104.807782698319\\
0.804000000000001	104.779368477183\\
0.806000000000001	104.751168473465\\
0.808000000000001	104.723174396436\\
0.810000000000001	104.695378374069\\
0.812000000000001	104.66777292778\\
0.814000000000001	104.640350948918\\
0.816000000000001	104.613105676886\\
0.818000000000001	104.586030678709\\
0.820000000000001	104.559119830031\\
0.822000000000001	104.532367297363\\
0.824000000000001	104.505767521505\\
0.826000000000001	104.479315202064\\
0.828000000000001	104.453005282977\\
0.830000000000001	104.426832938984\\
0.832000000000001	104.400793562973\\
0.834000000000001	104.374882754084\\
0.836000000000001	104.349096306625\\
0.838000000000001	104.323430199641\\
0.840000000000001	104.297880587158\\
0.842000000000001	104.272443788948\\
0.844000000000001	104.247116281996\\
0.846000000000001	104.221894692357\\
0.848000000000001	104.19677578756\\
0.850000000000001	104.171756469421\\
0.852000000000001	104.146833767349\\
0.854000000000001	104.122004831974\\
0.856000000000001	104.09726692916\\
0.858000000000001	104.072617434395\\
0.860000000000001	104.048053827459\\
0.862000000000001	104.02357368741\\
0.864000000000001	103.999174687855\\
0.866000000000001	103.974854592469\\
0.868000000000001	103.950611250775\\
0.870000000000001	103.926442594143\\
0.872000000000001	103.902346632014\\
0.874000000000001	103.87832144832\\
0.876000000000001	103.854365198113\\
0.878000000000001	103.830476104316\\
0.880000000000001	103.806652454736\\
0.882000000000001	103.782892599147\\
0.884000000000001	103.75919494656\\
0.886000000000001	103.735557962646\\
0.888000000000001	103.711980167264\\
0.890000000000001	103.688460132125\\
0.892000000000001	103.664996478562\\
0.894000000000001	103.641587875439\\
0.896000000000001	103.618233037141\\
0.898000000000001	103.594930721637\\
0.900000000000001	103.571679728718\\
0.902000000000001	103.548478898222\\
0.904000000000001	103.525327108401\\
0.906000000000001	103.502223274384\\
0.908000000000001	103.479166346633\\
0.910000000000001	103.456155309558\\
0.912000000000001	103.433189180153\\
0.914000000000001	103.410267006702\\
0.916000000000001	103.387387867535\\
0.918000000000001	103.364550869859\\
0.920000000000001	103.341755148644\\
0.922000000000001	103.318999865518\\
0.924000000000001	103.296284207778\\
0.926000000000001	103.273607387391\\
0.928000000000001	103.250968640053\\
0.930000000000001	103.228367224308\\
0.932000000000001	103.205802420685\\
0.934000000000001	103.183273530872\\
0.936000000000001	103.160779876956\\
0.938000000000001	103.138320800633\\
0.940000000000001	103.115895662535\\
0.942000000000001	103.093503841512\\
0.944000000000001	103.071144733987\\
0.946000000000001	103.048817753316\\
0.948000000000001	103.0265223292\\
0.950000000000001	103.004257907088\\
0.952000000000001	102.982023947628\\
0.954000000000001	102.959819926145\\
0.956000000000001	102.937645332106\\
0.958000000000001	102.915499668658\\
0.960000000000001	102.893382452143\\
0.962000000000001	102.871293211639\\
0.964000000000001	102.849231488559\\
0.966000000000001	102.827196836179\\
0.968000000000001	102.805188819314\\
0.970000000000001	102.783207013858\\
0.972000000000001	102.761251006467\\
0.974000000000001	102.739320394185\\
0.976000000000001	102.717414784114\\
0.978000000000001	102.69553379307\\
0.980000000000001	102.673677047275\\
0.982000000000001	102.651844182082\\
0.984000000000001	102.630034841627\\
0.986000000000001	102.608248678614\\
0.988000000000001	102.586485353988\\
0.990000000000001	102.564744536716\\
0.992000000000001	102.543025903516\\
0.994000000000001	102.521329138633\\
0.996000000000001	102.499653933585\\
0.998000000000001	102.477999986962\\
1	102.456367004213\\
1.002	102.434754697409\\
1.004	102.413162785077\\
1.006	102.391590991989\\
1.008	102.370039048992\\
1.01	102.348506692817\\
1.012	102.326993665911\\
1.014	102.305499716263\\
1.016	102.284024597274\\
1.018	102.262568067556\\
1.02	102.241129890813\\
1.022	102.219709835719\\
1.024	102.198307675707\\
1.026	102.176923188917\\
1.028	102.155556158015\\
1.03	102.134206370093\\
1.032	102.112873616523\\
1.034	102.091557692874\\
1.036	102.070258398762\\
1.038	102.048975537774\\
1.04	102.027708917358\\
1.042	102.0064583487\\
1.044	101.985223646631\\
1.046	101.964004629555\\
1.048	101.942801119342\\
1.05	101.921612941225\\
1.052	101.900439923751\\
1.054	101.879281898652\\
1.056	101.858138700796\\
1.058	101.8370101681\\
1.06	101.815896141456\\
1.062	101.794796464667\\
1.064	101.773710984332\\
1.066	101.752639549859\\
1.068	101.731582013309\\
1.07	101.710538229397\\
1.072	101.689508055398\\
1.074	101.668491351099\\
1.076	101.647487978729\\
1.078	101.626497802918\\
1.08	101.605520690631\\
1.082	101.584556511108\\
1.084	101.563605135831\\
1.086	101.542666438459\\
1.088	101.52174029478\\
1.09	101.500826582685\\
1.092	101.479925182074\\
1.094	101.459035974864\\
1.096	101.438158844924\\
1.098	101.417293678016\\
1.1	101.396440361784\\
1.102	101.37559878568\\
1.104	101.354768840986\\
1.106	101.333950420693\\
1.108	101.313143419523\\
1.11	101.292347733886\\
1.112	101.271563261823\\
1.114	101.250789902999\\
1.116	101.230027558652\\
1.118	101.209276131561\\
1.12	101.188535526037\\
1.122	101.167805647855\\
1.124	101.14708640426\\
1.126	101.12637770394\\
1.128	101.105679456937\\
1.13	101.084991574723\\
1.132	101.064313970073\\
1.134	101.043646557094\\
1.136	101.022989251202\\
1.138	101.002341969069\\
1.14	100.981704628626\\
1.142	100.961077149011\\
1.144	100.940459450591\\
1.146	100.919851454879\\
1.148	100.899253084571\\
1.15	100.878664263489\\
1.152	100.858084916569\\
1.154	100.837514969848\\
1.156	100.816954350436\\
1.158	100.796402986509\\
1.16	100.775860807266\\
1.162	100.755327742943\\
1.164	100.734803724765\\
1.166	100.714288684951\\
1.168	100.693782556681\\
1.17	100.673285274091\\
1.172	100.652796772254\\
1.174	100.632316987158\\
1.176	100.611845855695\\
1.178	100.591383315647\\
1.18	100.570929305669\\
1.182	100.550483765277\\
1.184	100.530046634833\\
1.186	100.509617855523\\
1.188	100.48919736936\\
1.19	100.468785119147\\
1.192	100.448381048495\\
1.194	100.427985101779\\
1.196	100.407597224156\\
1.198	100.387217361514\\
1.2	100.366845460503\\
1.202	100.346481468488\\
1.204	100.326125333563\\
1.206	100.305777004528\\
1.208	100.285436430858\\
1.21	100.265103562745\\
1.212	100.244778351034\\
1.214	100.224460747237\\
1.216	100.204150703526\\
1.218	100.183848172711\\
1.22	100.163553108235\\
1.222	100.14326546417\\
1.224	100.122985195195\\
1.226	100.102712256602\\
1.228	100.082446604278\\
1.23	100.062188194688\\
1.232	100.041936984889\\
1.234	100.021692932484\\
1.236	100.001455995685\\
1.238	99.9812261331943\\
1.24	99.9610033043129\\
1.242	99.9407874688426\\
1.244	99.920578587131\\
1.246	99.9003766200427\\
1.248	99.8801815289588\\
1.25	99.8599932757665\\
1.252	99.8398118228375\\
1.254	99.819637133061\\
1.256	99.7994691697853\\
1.258	99.7793078968558\\
1.26	99.7591532785684\\
1.262	99.7390052797084\\
1.264	99.7188638654911\\
1.266	99.6987290016007\\
1.268	99.6786006541591\\
1.27	99.6584787897341\\
1.272	99.6383633753103\\
1.274	99.618254378303\\
1.276	99.5981517665668\\
1.278	99.5780555083339\\
1.28	99.5579655722807\\
1.282	99.5378819274628\\
1.284	99.517804543345\\
1.286	99.4977333897659\\
1.288	99.4776684369743\\
1.29	99.457609655576\\
1.292	99.437557016574\\
1.294	99.4175104913262\\
1.296	99.39747005156\\
1.298	99.3774356693646\\
1.3	99.3574073171781\\
1.302	99.3373849677904\\
1.304	99.3173685943395\\
1.306	99.2973581703062\\
1.308	99.277353669507\\
1.31	99.2573550660826\\
1.312	99.2373623344995\\
1.314	99.2173754495606\\
1.316	99.1973943863771\\
1.318	99.1774191203757\\
1.32	99.1574496272882\\
1.322	99.1374858831615\\
1.324	99.1175278643268\\
1.326	99.097575547434\\
1.328	99.077628909411\\
1.33	99.0576879274834\\
1.332	99.0377525791537\\
1.334	99.0178228422111\\
1.336	98.9978986947212\\
1.338	98.9779801150352\\
1.34	98.9580670817493\\
1.342	98.9381595737477\\
1.344	98.9182575701702\\
1.346	98.8983610504207\\
1.348	98.8784699941528\\
1.35	98.8585843812819\\
1.352	98.838704191955\\
1.354	98.8188294065917\\
1.356	98.7989600058349\\
1.358	98.7790959705783\\
1.36	98.7592372819286\\
1.362	98.7393839212671\\
1.364	98.71953587017\\
1.366	98.6996931104566\\
1.368	98.6798556241689\\
1.37	98.6600233935655\\
1.372	98.6401964011283\\
1.374	98.6203746295622\\
1.376	98.6005580617624\\
1.378	98.5807466808597\\
1.38	98.5609404701757\\
1.382	98.5411394132381\\
1.384	98.5213434937867\\
1.386	98.501552695747\\
1.388	98.481767003249\\
1.39	98.4619864006218\\
1.392	98.4422108723726\\
1.394	98.4224404032073\\
1.396	98.4026749780219\\
1.398	98.3829145818778\\
1.4	98.3631592000405\\
1.402	98.3434088179421\\
1.404	98.3236634212008\\
1.406	98.3039229955934\\
1.408	98.2841875270909\\
1.41	98.2644570018123\\
1.412	98.2447314060609\\
1.414	98.2250107263051\\
1.416	98.2052949491626\\
1.418	98.1855840614283\\
1.42	98.1658780500501\\
1.422	98.1461769021311\\
1.424	98.1264806049423\\
1.426	98.1067891458909\\
1.428	98.0871025125371\\
1.43	98.0674206926061\\
1.432	98.0477436739593\\
1.434	98.0280714445986\\
1.436	98.0084039926874\\
1.438	97.9887413065167\\
1.44	97.9690833745097\\
1.442	97.9494301852525\\
1.444	97.9297817274487\\
1.446	97.9101379899432\\
1.448	97.89049896171\\
1.45	97.8708646318579\\
1.452	97.8512349896271\\
1.454	97.8316100243837\\
1.456	97.8119897256091\\
1.458	97.7923740829263\\
1.46	97.772763086078\\
1.462	97.7531567249175\\
1.464	97.7335549894324\\
1.466	97.7139578697158\\
1.468	97.6943653559815\\
1.47	97.6747774385696\\
1.472	97.6551941079122\\
1.474	97.6356153545782\\
1.476	97.6160411692246\\
1.478	97.5964715426352\\
1.48	97.5769064656952\\
1.482	97.5573459293909\\
1.484	97.5377899248262\\
1.486	97.5182384431959\\
1.488	97.4986914758141\\
1.49	97.4791490140729\\
1.492	97.4596110494905\\
1.494	97.4400775736601\\
1.496	97.42054857829\\
1.498	97.4010240551788\\
1.5	97.3815039962098\\
1.502	97.3619883933801\\
1.504	97.3424772387639\\
1.506	97.3229705245243\\
1.508	97.303468242933\\
1.51	97.2839703863396\\
1.512	97.264476947172\\
1.514	97.244987917962\\
1.516	97.2255032913198\\
1.518	97.2060230599359\\
1.52	97.1865472165889\\
1.522	97.1670757541427\\
1.524	97.147608665535\\
1.526	97.1281459437962\\
1.528	97.1086875820171\\
1.53	97.0892335733816\\
1.532	97.0697839111461\\
1.534	97.0503385886483\\
1.536	97.0308975992924\\
1.538	97.0114609365625\\
1.54	96.9920285940169\\
1.542	96.972600565279\\
1.544	96.9531768440584\\
1.546	96.9337574241161\\
1.548	96.9143422993038\\
1.55	96.8949314635185\\
1.552	96.8755249107522\\
1.554	96.856122635033\\
1.556	96.8367246304885\\
1.558	96.8173308912823\\
1.56	96.7979414116645\\
1.562	96.7785561859391\\
1.564	96.7591752084627\\
1.566	96.7397984736764\\
1.568	96.7204259760664\\
1.57	96.7010577101894\\
1.572	96.6816936706467\\
1.574	96.6623338521168\\
1.576	96.642978249326\\
1.578	96.6236268570586\\
1.58	96.6042796701612\\
1.582	96.5849366835328\\
1.584	96.5655978921233\\
1.586	96.5462632909421\\
1.588	96.5269328750552\\
1.59	96.5076066395793\\
1.592	96.4882845796748\\
1.594	96.4689666905715\\
1.596	96.4496529675396\\
1.598	96.4303434058981\\
1.6	96.4110380010145\\
1.602	96.3917367483215\\
1.604	96.3724396432829\\
1.606	96.3531466814184\\
1.608	96.3338578582925\\
1.61	96.3145731695164\\
1.612	96.2952926107506\\
1.614	96.2760161777043\\
1.616	96.2567438661212\\
1.618	96.2374756717908\\
1.62	96.2182115905586\\
1.622	96.1989516183039\\
1.624	96.1796957509502\\
1.626	96.1604439844634\\
1.628	96.141196314852\\
1.63	96.1219527381631\\
1.632	96.1027132504932\\
1.634	96.0834778479609\\
1.636	96.0642465267476\\
1.638	96.0450192830547\\
1.64	96.0257961131313\\
1.642	96.0065770132638\\
1.644	95.9873619797727\\
1.646	95.9681510090225\\
1.648	95.9489440974063\\
1.65	95.92974124136\\
1.652	95.9105424373528\\
1.654	95.8913476818896\\
1.656	95.8721569715074\\
1.658	95.8529703027804\\
1.66	95.8337876723163\\
1.662	95.8146090767574\\
1.664	95.7954345127824\\
1.666	95.7762639770886\\
1.668	95.7570974664223\\
1.67	95.7379349775524\\
1.672	95.7187765072835\\
1.674	95.6996220524455\\
1.676	95.6804716099098\\
1.678	95.6613251765676\\
1.68	95.6421827493413\\
1.682	95.6230443251857\\
1.684	95.6039099010912\\
1.686	95.584779474059\\
1.688	95.5656530411388\\
1.69	95.5465305993926\\
1.692	95.5274121459226\\
1.694	95.50829767785\\
1.696	95.489187192328\\
1.698	95.4700806865261\\
1.7	95.4509781576556\\
1.702	95.4318796029475\\
1.704	95.4127850196507\\
1.706	95.3936944050482\\
1.708	95.3746077564527\\
1.71	95.3555250711852\\
1.712	95.3364463466016\\
1.714	95.3173715800879\\
1.716	95.2983007690402\\
1.718	95.2792339108822\\
1.72	95.2601710030746\\
1.722	95.2411120430752\\
1.724	95.2220570283909\\
1.726	95.2030059565334\\
1.728	95.1839588250442\\
1.73	95.1649156314782\\
1.732	95.1458763734187\\
1.734	95.1268410484745\\
1.736	95.1078096542675\\
1.738	95.0887821884403\\
1.74	95.069758648666\\
1.742	95.050739032621\\
1.744	95.031723338011\\
1.746	95.0127115625705\\
1.748	94.9937037040347\\
1.75	94.9746997601685\\
1.752	94.9556997287605\\
1.754	94.9367036076026\\
1.756	94.9177113945177\\
1.758	94.8987230873487\\
1.76	94.8797386839473\\
1.762	94.8607581821823\\
1.764	94.8417815799533\\
1.766	94.8228088751646\\
1.768	94.803840065738\\
1.77	94.7848751496184\\
1.772	94.7659141247705\\
1.774	94.7469569891552\\
1.776	94.7280037407766\\
1.778	94.7090543776394\\
1.78	94.6901088977624\\
1.782	94.6711672991852\\
1.784	94.6522295799687\\
1.786	94.6332957381707\\
1.788	94.6143657718875\\
1.79	94.5954396792074\\
1.792	94.5765174582519\\
1.794	94.557599107149\\
1.796	94.5386846240434\\
1.798	94.5197740070844\\
1.8	94.5008672544418\\
1.802	94.481964364308\\
1.804	94.4630653348812\\
1.806	94.4441701643691\\
1.808	94.4252788509894\\
1.81	94.4063913929934\\
1.812	94.387507788624\\
1.814	94.3686280361409\\
1.816	94.3497521338307\\
1.818	94.330880079977\\
1.82	94.312011872875\\
1.822	94.293147510843\\
1.824	94.2742869922048\\
1.826	94.2554303152963\\
1.828	94.2365774784687\\
1.83	94.217728480073\\
1.832	94.1988833184918\\
1.834	94.180041992097\\
1.836	94.1612044992898\\
1.838	94.1423708384704\\
1.84	94.1235410080474\\
1.842	94.1047150064659\\
1.844	94.0858928321487\\
1.846	94.0670744835403\\
1.848	94.0482599591003\\
1.85	94.0294492573012\\
1.852	94.0106423766138\\
1.854	93.9918393155248\\
1.856	93.9730400725359\\
1.858	93.9542446461535\\
1.86	93.9354530348908\\
1.862	93.916665237274\\
1.864	93.8978812518376\\
1.866	93.8791010771198\\
1.868	93.8603247116849\\
1.87	93.841552154088\\
1.872	93.8227834029061\\
1.874	93.8040184567115\\
1.876	93.7852573140925\\
1.878	93.7664999736429\\
1.88	93.7477464339736\\
1.882	93.7289966937026\\
1.884	93.710250751443\\
1.886	93.6915086058257\\
1.888	93.6727702554907\\
1.89	93.6540356990792\\
1.892	93.6353049352489\\
1.894	93.6165779626618\\
1.896	93.5978547799768\\
1.898	93.5791353858815\\
1.9	93.5604197790543\\
1.902	93.541707958184\\
1.904	93.5229999219737\\
1.906	93.5042956691223\\
1.908	93.4855951983449\\
1.91	93.466898508365\\
1.912	93.4482055979003\\
1.914	93.4295164656884\\
1.916	93.4108311104691\\
1.918	93.3921495309865\\
1.92	93.3734717259934\\
1.922	93.3547976942486\\
1.924	93.3361274345206\\
1.926	93.3174609455858\\
1.928	93.2987982262033\\
1.93	93.2801392751782\\
1.932	93.2614840912895\\
1.934	93.2428326733312\\
1.936	93.2241850201152\\
1.938	93.2055411304406\\
1.94	93.1869010031286\\
1.942	93.1682646369909\\
1.944	93.1496320308575\\
1.946	93.1310031835606\\
1.948	93.1123780939318\\
1.95	93.0937567608105\\
1.952	93.07513918305\\
1.954	93.0565253594996\\
1.956	93.0379152890155\\
1.958	93.019308970465\\
1.96	93.000706402711\\
1.962	92.9821075846297\\
1.964	92.9635125150941\\
1.966	92.9449211929889\\
1.968	92.926333617201\\
1.97	92.9077497866283\\
1.972	92.8891697001555\\
1.974	92.8705933566965\\
1.976	92.8520207551522\\
1.978	92.8334518944346\\
1.98	92.8148867734552\\
1.982	92.7963253911342\\
1.984	92.7777677464045\\
1.986	92.7592138381841\\
1.988	92.7406636654114\\
1.99	92.7221172270234\\
1.992	92.7035745219571\\
1.994	92.6850355491615\\
1.996	92.6665003075868\\
1.998	92.647968796182\\
2	92.6294410139105\\
2.002	92.6109169597282\\
2.004	92.5923966326003\\
2.006	92.5738800315006\\
2.008	92.5553671553967\\
2.01	92.5368580032674\\
2.012	92.5183525740971\\
2.014	92.4998508668657\\
2.016	92.4813528805594\\
2.018	92.4628586141722\\
2.02	92.444368066695\\
2.022	92.4258812371304\\
2.024	92.4073981244789\\
2.026	92.3889187277453\\
2.028	92.370443045933\\
2.03	92.3519710780613\\
2.032	92.3335028231399\\
2.034	92.3150382801912\\
2.036	92.2965774482327\\
2.038	92.2781203262902\\
2.04	92.259666913392\\
2.042	92.2412172085684\\
2.044	92.2227712108537\\
2.046	92.2043289192826\\
2.048	92.1858903328971\\
2.05	92.1674554507401\\
2.052	92.1490242718531\\
2.054	92.1305967952905\\
2.056	92.1121730200985\\
2.05799999999999	92.0937529453341\\
2.05999999999999	92.0753365700535\\
2.06199999999999	92.0569238933187\\
2.06399999999999	92.0385149141853\\
2.06599999999999	92.0201096317247\\
2.06799999999999	92.001708045002\\
2.06999999999999	91.9833101530875\\
2.07199999999999	91.9649159550536\\
2.07399999999999	91.9465254499763\\
2.07599999999999	91.928138636931\\
2.07799999999999	91.9097555150041\\
2.07999999999999	91.8913760832711\\
2.08199999999999	91.8730003408214\\
2.08399999999999	91.8546282867436\\
2.08599999999999	91.8362599201218\\
2.08799999999999	91.8178952400538\\
2.08999999999999	91.7995342456296\\
2.09199999999999	91.781176935953\\
2.09399999999999	91.7628233101175\\
2.09599999999999	91.7444733672231\\
2.09799999999999	91.7261271063794\\
2.09999999999999	91.7077845266902\\
2.10199999999999	91.6894456272599\\
2.10399999999999	91.6711104071993\\
2.10599999999999	91.6527788656216\\
2.10799999999999	91.6344510016437\\
2.10999999999999	91.6161268143791\\
2.11199999999999	91.597806302944\\
2.11399999999999	91.5794894664619\\
2.11599999999999	91.561176304052\\
2.11799999999999	91.5428668148471\\
2.11999999999999	91.5245609979607\\
2.12199999999999	91.5062588525298\\
2.12399999999999	91.4879603776809\\
2.12599999999999	91.4696655725455\\
2.12799999999999	91.4513744362593\\
2.12999999999999	91.433086967956\\
2.13199999999999	91.4148031667736\\
2.13399999999999	91.3965230318524\\
2.13599999999999	91.3782465623302\\
2.13799999999999	91.3599737573521\\
2.13999999999999	91.3417046160604\\
2.14199999999999	91.323439137604\\
2.14399999999999	91.3051773211287\\
2.14599999999999	91.2869191657812\\
2.14799999999999	91.2686646707177\\
2.14999999999998	91.2504138350904\\
2.15199999999998	91.2321666580489\\
2.15399999999998	91.2139231387507\\
2.15599999999998	91.1956832763565\\
2.15799999999998	91.1774470700201\\
2.15999999999998	91.1592145189075\\
2.16199999999998	91.1409856221778\\
2.16399999999998	91.1227603789922\\
2.16599999999998	91.1045387885188\\
2.16799999999998	91.0863208499251\\
2.16999999999998	91.0681065623752\\
2.17199999999998	91.049895925044\\
2.17399999999998	91.0316889370943\\
2.17599999999998	91.0134855977038\\
2.17799999999998	90.9952859060451\\
2.17999999999998	90.9770898612922\\
2.18199999999998	90.9588974626224\\
2.18399999999998	90.9407087092126\\
2.18599999999998	90.9225236002419\\
2.18799999999998	90.9043421348922\\
2.18999999999998	90.8861643123426\\
2.19199999999998	90.8679901317732\\
2.19399999999998	90.8498195923749\\
2.19599999999998	90.8316526933287\\
2.19799999999998	90.8134894338216\\
2.19999999999998	90.795329813043\\
2.20199999999998	90.7771738301785\\
2.20399999999998	90.7590214844235\\
2.20599999999998	90.7408727749654\\
2.20799999999998	90.7227277009977\\
2.20999999999998	90.7045862617104\\
2.21199999999998	90.6864484563036\\
2.21399999999998	90.6683142839755\\
2.21599999999998	90.6501837439131\\
2.21799999999998	90.6320568353256\\
2.21999999999998	90.6139335574055\\
2.22199999999998	90.5958139093568\\
2.22399999999998	90.577697890377\\
2.22599999999998	90.5595854996705\\
2.22799999999998	90.5414767364434\\
2.22999999999998	90.5233715998969\\
2.23199999999998	90.5052700892393\\
2.23399999999998	90.4871722036728\\
2.23599999999998	90.469077942409\\
2.23799999999998	90.4509873046547\\
2.23999999999997	90.4329002896226\\
2.24199999999997	90.4148168965228\\
2.24399999999997	90.3967371245641\\
2.24599999999997	90.3786609729573\\
2.24799999999997	90.3605884409199\\
2.24999999999997	90.3425195276629\\
2.25199999999997	90.3244542324064\\
2.25399999999997	90.3063925543624\\
2.25599999999997	90.2883344927506\\
2.25799999999997	90.2702800467869\\
2.25999999999997	90.2522292156941\\
2.26199999999997	90.2341819986828\\
2.26399999999997	90.2161383949876\\
2.26599999999997	90.1980984038177\\
2.26799999999997	90.1800620244001\\
2.26999999999997	90.1620292559578\\
2.27199999999997	90.1440000977173\\
2.27399999999997	90.1259745488977\\
2.27599999999997	90.1079526087302\\
2.27799999999997	90.0899342764395\\
2.27999999999997	90.0719195512478\\
2.28199999999997	90.0539084323941\\
2.28399999999997	90.0359009190967\\
2.28599999999997	90.017897010589\\
2.28799999999997	89.9998967061031\\
2.28999999999997	89.9819000048669\\
2.29199999999997	89.9639069061172\\
2.29399999999997	89.9459174090803\\
2.29599999999997	89.9279315129928\\
2.29799999999997	89.9099492170903\\
2.29999999999997	89.891970520602\\
2.30199999999997	89.8739954227692\\
2.30399999999997	89.8560239228274\\
2.30599999999997	89.8380560200119\\
2.30799999999997	89.8200917135599\\
2.30999999999997	89.8021310027068\\
2.31199999999997	89.7841738866994\\
2.31399999999997	89.7662203647735\\
2.31599999999997	89.7482704361628\\
2.31799999999997	89.7303241001216\\
2.31999999999997	89.7123813558764\\
2.32199999999997	89.6944422026857\\
2.32399999999997	89.676506639776\\
2.32599999999997	89.6585746664023\\
2.32799999999997	89.6406462818038\\
2.32999999999996	89.6227214852301\\
2.33199999999996	89.6048002759162\\
2.33399999999996	89.5868826531191\\
2.33599999999996	89.5689686160785\\
2.33799999999996	89.5510581640472\\
2.33999999999996	89.5331512962658\\
2.34199999999996	89.5152480119881\\
2.34399999999996	89.4973483104627\\
2.34599999999996	89.4794521909367\\
2.34799999999996	89.4615596526615\\
2.34999999999996	89.4436706948899\\
2.35199999999996	89.4257853168677\\
2.35399999999996	89.4079035178493\\
2.35599999999996	89.390025297089\\
2.35799999999996	89.3721506538382\\
2.35999999999996	89.3542795873472\\
2.36199999999996	89.3364120968745\\
2.36399999999996	89.3185481816758\\
2.36599999999996	89.3006878409981\\
2.36799999999996	89.2828310741071\\
2.36999999999996	89.2649778802487\\
2.37199999999996	89.2471282586847\\
2.37399999999996	89.2292822086733\\
2.37599999999996	89.2114397294707\\
2.37799999999996	89.1936008203341\\
2.37999999999996	89.1757654805202\\
2.38199999999996	89.1579337092911\\
2.38399999999996	89.1401055059075\\
2.38599999999996	89.1222808696234\\
2.38799999999996	89.1044597997056\\
2.38999999999996	89.0866422954112\\
2.39199999999996	89.0688283560036\\
2.39399999999996	89.0510179807435\\
2.39599999999996	89.0332111688918\\
2.39799999999996	89.0154079197109\\
2.39999999999996	88.9976082324675\\
2.40199999999996	88.9798121064284\\
2.40399999999996	88.9620195408471\\
2.40599999999996	88.9442305349963\\
2.40799999999996	88.9264450881365\\
2.40999999999996	88.9086631995358\\
2.41199999999996	88.890884868457\\
2.41399999999996	88.8731100941715\\
2.41599999999996	88.8553388759406\\
2.41799999999996	88.8375712130348\\
2.41999999999996	88.8198071047213\\
2.42199999999995	88.8020465502654\\
2.42399999999995	88.7842895489376\\
2.42599999999995	88.7665361000055\\
2.42799999999995	88.748786202738\\
2.42999999999995	88.7310398564083\\
2.43199999999995	88.7132970602805\\
2.43399999999995	88.6955578136298\\
2.43599999999995	88.6778221157232\\
2.43799999999995	88.6600899658317\\
2.43999999999995	88.6423613632329\\
2.44199999999995	88.624636307189\\
2.44399999999995	88.6069147969801\\
2.44599999999995	88.5891968318741\\
2.44799999999995	88.5714824111493\\
2.44999999999995	88.5537715340694\\
2.45199999999995	88.5360641999182\\
2.45399999999995	88.5183604079647\\
2.45599999999995	88.5006601574867\\
2.45799999999995	88.4829634477508\\
2.45999999999995	88.4652702780392\\
2.46199999999995	88.4475806476278\\
2.46399999999995	88.4298945557858\\
2.46599999999995	88.412212001802\\
2.46799999999995	88.3945329849427\\
2.46999999999995	88.3768575044818\\
2.47199999999995	88.3591855597038\\
2.47399999999995	88.3415171498846\\
2.47599999999995	88.3238522742997\\
2.47799999999995	88.3061909322282\\
2.47999999999995	88.2885331229495\\
2.48199999999995	88.2708788457444\\
2.48399999999995	88.2532280998882\\
2.48599999999995	88.2355808846609\\
2.48799999999995	88.2179371993439\\
2.48999999999995	88.2002970432169\\
2.49199999999995	88.1826604155604\\
2.49399999999995	88.1650273156554\\
2.49599999999995	88.1473977427785\\
2.49799999999995	88.1297716962202\\
2.49999999999995	88.1121491752544\\
2.50199999999995	88.094530179163\\
2.50399999999995	88.0769147072331\\
2.50599999999995	88.0593027587425\\
2.50799999999995	88.0416943329778\\
2.50999999999995	88.0240894292218\\
2.51199999999994	88.0064880467551\\
2.51399999999994	87.9888901848652\\
2.51599999999994	87.9712958428321\\
2.51799999999994	87.9537050199411\\
2.51999999999994	87.9361177154806\\
2.52199999999994	87.9185339287301\\
2.52399999999994	87.9009536589801\\
2.52599999999994	87.8833769055132\\
2.52799999999994	87.8658036676139\\
2.52999999999994	87.848233944567\\
2.53199999999994	87.8306677356668\\
2.53399999999994	87.8131050401918\\
2.53599999999994	87.7955458574281\\
2.53799999999994	87.7779901866701\\
2.53999999999994	87.7604380272011\\
2.54199999999994	87.7428893783082\\
2.54399999999994	87.72534423928\\
2.54599999999994	87.7078026094046\\
2.54799999999994	87.6902644879706\\
2.54999999999994	87.6727298742654\\
2.55199999999994	87.655198767579\\
2.55399999999994	87.6376711671987\\
2.55599999999994	87.6201470724204\\
2.55799999999994	87.6026264825275\\
2.55999999999994	87.5851093968102\\
2.56199999999994	87.5675958145644\\
2.56399999999994	87.5500857350722\\
2.56599999999994	87.5325791576346\\
2.56799999999994	87.515076081531\\
2.56999999999994	87.4975765060618\\
2.57199999999994	87.4800804305119\\
2.57399999999994	87.4625878541794\\
2.57599999999994	87.4450987763513\\
2.57799999999994	87.4276131963209\\
2.57999999999994	87.4101311133817\\
2.58199999999994	87.3926525268266\\
2.58399999999994	87.3751774359488\\
2.58599999999994	87.3577058400394\\
2.58799999999994	87.3402377383934\\
2.58999999999994	87.3227731303043\\
2.59199999999994	87.3053120150664\\
2.59399999999994	87.2878543919685\\
2.59599999999994	87.2704002603163\\
2.59799999999994	87.2529496193905\\
2.59999999999994	87.2355024684995\\
2.60199999999994	87.2180588069288\\
2.60399999999993	87.2006186339755\\
2.60599999999993	87.1831819489367\\
2.60799999999993	87.1657487511111\\
2.60999999999993	87.1483190397874\\
2.61199999999993	87.1308928142663\\
2.61399999999993	87.1134700738461\\
2.61599999999993	87.0960508178173\\
2.61799999999993	87.07863504548\\
2.61999999999993	87.0612227561321\\
2.62199999999993	87.0438139490724\\
2.62399999999993	87.0264086235947\\
2.62599999999993	87.0090067789955\\
2.62799999999993	86.9916084145787\\
2.62999999999993	86.9742135296347\\
2.63199999999993	86.9568221234682\\
2.63399999999993	86.9394341953753\\
2.63599999999993	86.9220497446559\\
2.63799999999993	86.9046687706084\\
2.63999999999993	86.887291272528\\
2.64199999999993	86.8699172497208\\
2.64399999999993	86.8525467014812\\
2.64599999999993	86.8351796271116\\
2.64799999999993	86.817816025913\\
2.64999999999993	86.8004558971812\\
2.65199999999993	86.7830992402211\\
2.65399999999993	86.7657460543313\\
2.65599999999993	86.7483963388105\\
2.65799999999993	86.731050092964\\
2.65999999999993	86.71370731609\\
2.66199999999993	86.6963680074917\\
2.66399999999993	86.6790321664712\\
2.66599999999993	86.661699792329\\
2.66799999999993	86.6443708843638\\
2.66999999999993	86.6270454418824\\
2.67199999999993	86.6097234641851\\
2.67399999999993	86.5924049505758\\
2.67599999999993	86.5750899003576\\
2.67799999999993	86.5577783128348\\
2.67999999999993	86.540470187304\\
2.68199999999993	86.523165523075\\
2.68399999999993	86.5058643194461\\
2.68599999999993	86.488566575727\\
2.68799999999993	86.4712722912204\\
2.68999999999993	86.4539814652242\\
2.69199999999993	86.4366940970502\\
2.69399999999992	86.4194101859993\\
2.69599999999992	86.4021297313774\\
2.69799999999992	86.3848527324867\\
2.69999999999992	86.3675791886373\\
2.70199999999992	86.3503090991326\\
2.70399999999992	86.3330424632726\\
2.70599999999992	86.3157792803695\\
2.70799999999992	86.2985195497292\\
2.70999999999992	86.2812632706504\\
2.71199999999992	86.2640104424472\\
2.71399999999992	86.2467610644224\\
2.71599999999992	86.2295151358828\\
2.71799999999992	86.2122726561352\\
2.71999999999992	86.1950336244891\\
2.72199999999992	86.177798040247\\
2.72399999999992	86.160565902716\\
2.72599999999992	86.1433372112098\\
2.72799999999992	86.1261119650316\\
2.72999999999992	86.1088901634883\\
2.73199999999992	86.0916718058885\\
2.73399999999992	86.0744568915423\\
2.73599999999992	86.0572454197579\\
2.73799999999992	86.0400373898384\\
2.73999999999992	86.022832801101\\
2.74199999999992	86.0056316528507\\
2.74399999999992	85.9884339443919\\
2.74599999999992	85.97123967504\\
2.74799999999992	85.9540488441025\\
2.74999999999992	85.9368614508886\\
2.75199999999992	85.9196774947089\\
2.75399999999992	85.9024969748708\\
2.75599999999992	85.8853198906889\\
2.75799999999992	85.8681462414711\\
2.75999999999992	85.8509760265237\\
2.76199999999992	85.8338092451616\\
2.76399999999992	85.8166458966957\\
2.76599999999992	85.7994859804384\\
2.76799999999992	85.7823294956958\\
2.76999999999992	85.7651764417803\\
2.77199999999992	85.7480268180099\\
2.77399999999992	85.7308806236864\\
2.77599999999992	85.7137378581292\\
2.77799999999992	85.6965985206443\\
2.77999999999992	85.6794626105495\\
2.78199999999992	85.6623301271534\\
2.78399999999991	85.6452010697705\\
2.78599999999991	85.6280754377099\\
2.78799999999991	85.6109532302869\\
2.78999999999991	85.5938344468161\\
2.79199999999991	85.5767190866084\\
2.79399999999991	85.5596071489762\\
2.79599999999991	85.5424986332351\\
2.79799999999991	85.5253935386962\\
2.79999999999991	85.5082918646766\\
2.80199999999991	85.4911936104864\\
2.80399999999991	85.4740987754421\\
2.80599999999991	85.4570073588568\\
2.80799999999991	85.4399193600451\\
2.80999999999991	85.4228347783233\\
2.81199999999991	85.4057536130048\\
2.81399999999991	85.3886758634041\\
2.81599999999991	85.3716015288336\\
2.81799999999991	85.3545306086159\\
2.81999999999991	85.3374631020578\\
2.82199999999991	85.3203990084815\\
2.82399999999991	85.3033383271983\\
2.82599999999991	85.2862810575263\\
2.82799999999991	85.2692271987823\\
2.82999999999991	85.252176750279\\
2.83199999999991	85.2351297113351\\
2.83399999999991	85.2180860812679\\
2.83599999999991	85.2010458593906\\
2.83799999999991	85.1840090450241\\
2.83999999999991	85.166975637483\\
2.84199999999991	85.1499456360849\\
2.84399999999991	85.1329190401454\\
2.84599999999991	85.1158958489851\\
2.84799999999991	85.0988760619191\\
2.84999999999991	85.0818596782685\\
2.85199999999991	85.0648466973463\\
2.85399999999991	85.0478371184711\\
2.85599999999991	85.0308309409662\\
2.85799999999991	85.0138281641437\\
2.85999999999991	84.9968287873256\\
2.86199999999991	84.9798328098313\\
2.86399999999991	84.9628402309771\\
2.86599999999991	84.9458510500815\\
2.86799999999991	84.928865266469\\
2.86999999999991	84.9118828794513\\
2.87199999999991	84.8949038883525\\
2.87399999999991	84.877928292491\\
2.8759999999999	84.8609560911877\\
2.8779999999999	84.8439872837617\\
2.8799999999999	84.8270218695334\\
2.8819999999999	84.8100598478184\\
2.8839999999999	84.7931012179442\\
2.8859999999999	84.7761459792284\\
2.8879999999999	84.7591941309891\\
2.8899999999999	84.7422456725493\\
2.8919999999999	84.7253006032299\\
2.8939999999999	84.7083589223518\\
2.8959999999999	84.6914206292364\\
2.8979999999999	84.6744857232056\\
2.8999999999999	84.6575542035807\\
2.9019999999999	84.6406260696804\\
2.9039999999999	84.6237013208336\\
2.9059999999999	84.6067799563544\\
2.9079999999999	84.5898619755699\\
2.9099999999999	84.5729473777991\\
2.9119999999999	84.5560361623679\\
2.9139999999999	84.5391283285959\\
2.9159999999999	84.5222238758075\\
2.9179999999999	84.5053228033256\\
2.9199999999999	84.4884251104716\\
2.9219999999999	84.4715307965695\\
2.9239999999999	84.4546398609432\\
2.9259999999999	84.4377523029163\\
2.9279999999999	84.4208681218108\\
2.9299999999999	84.4039873169531\\
2.9319999999999	84.3871098876656\\
2.9339999999999	84.3702358332708\\
2.9359999999999	84.3533651530962\\
2.9379999999999	84.3364978464632\\
2.9399999999999	84.3196339126964\\
2.9419999999999	84.3027733511215\\
2.9439999999999	84.2859161610646\\
2.9459999999999	84.2690623418504\\
2.9479999999999	84.2522118928014\\
2.9499999999999	84.2353648132449\\
2.9519999999999	84.2185211025048\\
2.9539999999999	84.2016807599058\\
2.9559999999999	84.184843784776\\
2.9579999999999	84.1680101764429\\
2.9599999999999	84.1511799342272\\
2.9619999999999	84.1343530574573\\
2.9639999999999	84.1175295454624\\
2.96599999999989	84.1007093975662\\
2.96799999999989	84.0838926130935\\
2.96999999999989	84.0670791913734\\
2.97199999999989	84.0502691317322\\
2.97399999999989	84.0334624334959\\
2.97599999999989	84.0166590959927\\
2.97799999999989	83.9998591185502\\
2.97999999999989	83.9830625004952\\
2.98199999999989	83.9662692411527\\
2.98399999999989	83.9494793398565\\
2.98599999999989	83.9326927959297\\
2.98799999999989	83.9159096086972\\
2.98999999999989	83.8991297774965\\
2.99199999999989	83.882353301647\\
2.99399999999989	83.8655801804835\\
2.99599999999989	83.8488104133291\\
2.99799999999989	83.832043999519\\
2.99999999999989	83.8152809383719\\
3.00199999999989	83.7985212292256\\
3.00399999999989	83.7817648714056\\
3.00599999999989	83.7650118642427\\
3.00799999999989	83.7482622070652\\
3.00999999999989	83.7315158992029\\
3.01199999999989	83.7147729399858\\
3.01399999999989	83.6980333287441\\
3.01599999999989	83.6812970648074\\
3.01799999999989	83.6645641475021\\
3.01999999999989	83.6478345761635\\
3.02199999999989	83.6311083501226\\
3.02399999999989	83.6143854687038\\
3.02599999999989	83.5976659312431\\
3.02799999999989	83.5809497370684\\
3.02999999999989	83.5642368855133\\
3.03199999999989	83.5475273759062\\
3.03399999999989	83.5308212075806\\
3.03599999999989	83.5141183798645\\
3.03799999999989	83.4974188920932\\
3.03999999999989	83.480722743598\\
3.04199999999989	83.4640299337091\\
3.04399999999989	83.4473404617561\\
3.04599999999989	83.4306543270735\\
3.04799999999989	83.4139715289952\\
3.04999999999989	83.3972920668524\\
3.05199999999989	83.3806159399763\\
3.05399999999989	83.3639431476988\\
3.05599999999989	83.3472736893553\\
3.05799999999988	83.3306075642797\\
3.05999999999988	83.3139447717991\\
3.06199999999988	83.2972853112528\\
3.06399999999988	83.2806291819705\\
3.06599999999988	83.2639763832856\\
3.06799999999988	83.2473269145363\\
3.06999999999988	83.230680775051\\
3.07199999999988	83.2140379641648\\
3.07399999999988	83.1973984812137\\
3.07599999999988	83.1807623255276\\
3.07799999999988	83.1641294964479\\
3.07999999999988	83.147499993302\\
3.08199999999988	83.130873815427\\
3.08399999999988	83.1142509621558\\
3.08599999999988	83.0976314328281\\
3.08799999999988	83.081015226775\\
3.08999999999988	83.0644023433326\\
3.09199999999988	83.0477927818352\\
3.09399999999988	83.0311865416199\\
3.09599999999988	83.0145836220184\\
3.09799999999988	82.9979840223724\\
3.09999999999988	82.981387742012\\
3.10199999999988	82.9647947802777\\
3.10399999999988	82.9482051365027\\
3.10599999999988	82.9316188100229\\
3.10799999999988	82.9150358001769\\
3.10999999999988	82.8984561062992\\
3.11199999999988	82.881879727726\\
3.11399999999988	82.8653066637964\\
3.11599999999988	82.8487369138462\\
3.11799999999988	82.8321704772115\\
3.11999999999988	82.815607353231\\
3.12199999999988	82.7990475412413\\
3.12399999999988	82.7824910405761\\
3.12599999999988	82.76593785058\\
3.12799999999988	82.7493879705873\\
3.12999999999988	82.7328413999308\\
3.13199999999988	82.7162981379571\\
3.13399999999988	82.6997581840002\\
3.13599999999988	82.6832215374001\\
3.13799999999988	82.6666881974912\\
3.13999999999988	82.6501581636136\\
3.14199999999988	82.6336314351095\\
3.14399999999988	82.6171080113137\\
3.14599999999988	82.6005878915659\\
3.14799999999987	82.5840710752069\\
3.14999999999987	82.5675575615741\\
3.15199999999987	82.5510473500076\\
3.15399999999987	82.534540439847\\
3.15599999999987	82.518036830429\\
3.15799999999987	82.5015365210989\\
3.15999999999987	82.4850395111899\\
3.16199999999987	82.4685458000488\\
3.16399999999987	82.4520553870116\\
3.16599999999987	82.4355682714183\\
3.16799999999987	82.4190844526099\\
3.16999999999987	82.4026039299301\\
3.17199999999987	82.3861267027148\\
3.17399999999987	82.3696527703105\\
3.17599999999987	82.3531821320516\\
3.17799999999987	82.3367147872802\\
3.17999999999987	82.3202507353415\\
3.18199999999987	82.3037899755748\\
3.18399999999987	82.2873325073209\\
3.18599999999987	82.2708783299219\\
3.18799999999987	82.2544274427201\\
3.18999999999987	82.2379798450571\\
3.19199999999987	82.2215355362765\\
3.19399999999987	82.2050945157147\\
3.19599999999987	82.1886567827202\\
3.19799999999987	82.1722223366335\\
3.19999999999987	82.155791176795\\
3.20199999999987	82.1393633025501\\
3.20399999999987	82.1229387132423\\
3.20599999999987	82.1065174082134\\
3.20799999999987	82.0900993868033\\
3.20999999999987	82.0736846483587\\
3.21199999999987	82.0572731922228\\
3.21399999999987	82.0408650177384\\
3.21599999999987	82.0244601242493\\
3.21799999999987	82.0080585110971\\
3.21999999999987	81.9916601776279\\
3.22199999999987	81.9752651231883\\
3.22399999999987	81.9588733471188\\
3.22599999999987	81.9424848487631\\
3.22799999999987	81.9260996274675\\
3.22999999999987	81.9097176825769\\
3.23199999999987	81.8933390134352\\
3.23399999999987	81.8769636193867\\
3.23599999999987	81.8605914997754\\
3.23799999999986	81.8442226539517\\
3.23999999999986	81.8278570812539\\
3.24199999999986	81.8114947810284\\
3.24399999999986	81.7951357526262\\
3.24599999999986	81.7787799953897\\
3.24799999999986	81.7624275086643\\
3.24999999999986	81.7460782917921\\
3.25199999999986	81.7297323441258\\
3.25399999999986	81.7133896650084\\
3.25599999999986	81.6970502537844\\
3.25799999999986	81.6807141098029\\
3.25999999999986	81.664381232411\\
3.26199999999986	81.6480516209541\\
3.26399999999986	81.6317252747789\\
3.26599999999986	81.6154021932334\\
3.26799999999986	81.5990823756607\\
3.26999999999986	81.5827658214121\\
3.27199999999986	81.5664525298348\\
3.27399999999986	81.5501425002744\\
3.27599999999986	81.533835732078\\
3.27799999999986	81.5175322245974\\
3.27999999999986	81.5012319771773\\
3.28199999999986	81.4849349891647\\
3.28399999999986	81.4686412599115\\
3.28599999999986	81.4523507887618\\
3.28799999999986	81.4360635750668\\
3.28999999999986	81.4197796181727\\
3.29199999999986	81.4034989174298\\
3.29399999999986	81.3872214721875\\
3.29599999999986	81.3709472817924\\
3.29799999999986	81.3546763455959\\
3.29999999999986	81.3384086629446\\
3.30199999999986	81.3221442331924\\
3.30399999999986	81.3058830556828\\
3.30599999999986	81.2896251297706\\
3.30799999999986	81.2733704548017\\
3.30999999999986	81.2571190301272\\
3.31199999999986	81.2408708550986\\
3.31399999999986	81.2246259290625\\
3.31599999999986	81.2083842513726\\
3.31799999999986	81.1921458213779\\
3.31999999999986	81.1759106384296\\
3.32199999999986	81.1596787018782\\
3.32399999999986	81.1434500110713\\
3.32599999999986	81.1272245653646\\
3.32799999999986	81.1110023641063\\
3.32999999999985	81.0947834066477\\
3.33199999999985	81.0785676923414\\
3.33399999999985	81.0623552205354\\
3.33599999999985	81.0461459905872\\
3.33799999999985	81.029940001844\\
3.33999999999985	81.0137372536583\\
3.34199999999985	80.9975377453809\\
3.34399999999985	80.9813414763652\\
3.34599999999985	80.9651484459661\\
3.34799999999985	80.9489586535311\\
3.34999999999985	80.9327720984167\\
3.35199999999985	80.9165887799739\\
3.35399999999985	80.9004086975527\\
3.35599999999985	80.8842318505108\\
3.35799999999985	80.8680582381976\\
3.35999999999985	80.8518878599659\\
3.36199999999985	80.8357207151727\\
3.36399999999985	80.8195568031667\\
3.36599999999985	80.8033961233066\\
3.36799999999985	80.7872386749419\\
3.36999999999985	80.7710844574258\\
3.37199999999985	80.7549334701171\\
3.37399999999985	80.7387857123647\\
3.37599999999985	80.7226411835252\\
3.37799999999985	80.7064998829519\\
3.37999999999985	80.6903618100006\\
3.38199999999985	80.6742269640251\\
3.38399999999985	80.6580953443788\\
3.38599999999985	80.6419669504191\\
3.38799999999985	80.6258417814987\\
3.38999999999985	80.6097198369725\\
3.39199999999985	80.5936011161967\\
3.39399999999985	80.5774856185281\\
3.39599999999985	80.5613733433201\\
3.39799999999985	80.5452642899275\\
3.39999999999985	80.5291584577071\\
3.40199999999985	80.5130558460172\\
3.40399999999985	80.4969564542085\\
3.40599999999985	80.4808602816417\\
3.40799999999985	80.4647673276702\\
3.40999999999985	80.4486775916536\\
3.41199999999985	80.4325910729432\\
3.41399999999985	80.4165077709009\\
3.41599999999985	80.4004276848798\\
3.41799999999985	80.3843508142382\\
3.41999999999984	80.3682771583329\\
3.42199999999984	80.3522067165216\\
3.42399999999984	80.3361394881613\\
3.42599999999984	80.3200754726102\\
3.42799999999984	80.3040146692247\\
3.42999999999984	80.2879570773595\\
3.43199999999984	80.2719026963774\\
3.43399999999984	80.2558515256341\\
3.43599999999984	80.2398035644872\\
3.43799999999984	80.2237588122943\\
3.43999999999984	80.2077172684187\\
3.44199999999984	80.1916789322129\\
3.44399999999984	80.1756438030369\\
3.44599999999984	80.1596118802509\\
3.44799999999984	80.1435831632109\\
3.44999999999984	80.1275576512799\\
3.45199999999984	80.1115353438136\\
3.45399999999984	80.0955162401736\\
3.45599999999984	80.0795003397169\\
3.45799999999984	80.0634876418044\\
3.45999999999984	80.0474781457944\\
3.46199999999984	80.0314718510498\\
3.46399999999984	80.0154687569254\\
3.46599999999984	79.9994688627862\\
3.46799999999984	79.9834721679887\\
3.46999999999984	79.9674786718942\\
3.47199999999984	79.951488373864\\
3.47399999999984	79.9355012732571\\
3.47599999999984	79.9195173694348\\
3.47799999999984	79.9035366617578\\
3.47999999999984	79.8875591495889\\
3.48199999999984	79.8715848322843\\
3.48399999999984	79.8556137092084\\
3.48599999999984	79.8396457797235\\
3.48799999999984	79.8236810431877\\
3.48999999999984	79.8077194989644\\
3.49199999999984	79.7917611464174\\
3.49399999999984	79.7758059849047\\
3.49599999999984	79.7598540137897\\
3.49799999999984	79.7439052324347\\
3.49999999999984	79.7279596402002\\
3.50199999999984	79.7120172364517\\
3.50399999999984	79.696078020548\\
3.50599999999984	79.6801419918542\\
3.50799999999984	79.664209149732\\
3.50999999999984	79.6482794935437\\
3.51199999999983	79.6323530226532\\
3.51399999999983	79.6164297364231\\
3.51599999999983	79.6005096342159\\
3.51799999999983	79.5845927153953\\
3.51999999999983	79.5686789793256\\
3.52199999999983	79.5527684253695\\
3.52399999999983	79.5368610528915\\
3.52599999999983	79.5209568612539\\
3.52799999999983	79.5050558498207\\
3.52999999999983	79.4891580179577\\
3.53199999999983	79.4732633650285\\
3.53399999999983	79.4573718903962\\
3.53599999999983	79.4414835934266\\
3.53799999999983	79.4255984734823\\
3.53999999999983	79.409716529931\\
3.54199999999983	79.393837762136\\
3.54399999999983	79.3779621694617\\
3.54599999999983	79.3620897512726\\
3.54799999999983	79.346220506937\\
3.54999999999983	79.3303544358168\\
3.55199999999983	79.3144915372781\\
3.55399999999983	79.298631810689\\
3.55599999999983	79.2827752554136\\
3.55799999999983	79.2669218708159\\
3.55999999999983	79.2510716562632\\
3.56199999999983	79.2352246111234\\
3.56399999999983	79.2193807347597\\
3.56599999999983	79.203540026543\\
3.56799999999983	79.1877024858331\\
3.56999999999983	79.1718681120026\\
3.57199999999983	79.1560369044149\\
3.57399999999983	79.1402088624377\\
3.57599999999983	79.1243839854379\\
3.57799999999983	79.1085622727839\\
3.57999999999983	79.0927437238418\\
3.58199999999983	79.0769283379778\\
3.58399999999983	79.0611161145624\\
3.58599999999983	79.0453070529587\\
3.58799999999983	79.0295011525393\\
3.58999999999983	79.0136984126682\\
3.59199999999983	78.9978988327167\\
3.59399999999983	78.9821024120523\\
3.59599999999983	78.9663091500402\\
3.59799999999983	78.9505190460506\\
3.59999999999983	78.9347320994524\\
3.60199999999982	78.9189483096162\\
3.60399999999982	78.9031676759073\\
3.60599999999982	78.8873901976959\\
3.60799999999982	78.8716158743486\\
3.60999999999982	78.8558447052421\\
3.61199999999982	78.8400766897353\\
3.61399999999982	78.8243118272062\\
3.61599999999982	78.8085501170191\\
3.61799999999982	78.7927915585475\\
3.61999999999982	78.7770361511587\\
3.62199999999982	78.7612838942221\\
3.62399999999982	78.7455347871073\\
3.62599999999982	78.7297888291868\\
3.62799999999982	78.7140460198316\\
3.62999999999982	78.6983063584079\\
3.63199999999982	78.6825698442902\\
3.63399999999982	78.6668364768477\\
3.63599999999982	78.6511062554487\\
3.63799999999982	78.6353791794683\\
3.63999999999982	78.6196552482754\\
3.64199999999982	78.6039344612429\\
3.64399999999982	78.5882168177387\\
3.64599999999982	78.5725023171372\\
3.64799999999982	78.5567909588063\\
3.64999999999982	78.5410827421228\\
3.65199999999982	78.5253776664556\\
3.65399999999982	78.5096757311781\\
3.65599999999982	78.4939769356576\\
3.65799999999982	78.478281279272\\
3.65999999999982	78.4625887613927\\
3.66199999999982	78.4468993813877\\
3.66399999999982	78.4312131386344\\
3.66599999999982	78.4155300325029\\
3.66799999999982	78.3998500623687\\
3.66999999999982	78.3841732276008\\
3.67199999999982	78.3684995275743\\
3.67399999999982	78.3528289616625\\
3.67599999999982	78.3371615292389\\
3.67799999999982	78.3214972296766\\
3.67999999999982	78.305836062347\\
3.68199999999982	78.2901780266294\\
3.68399999999982	78.2745231218902\\
3.68599999999982	78.2588713475103\\
3.68799999999982	78.2432227028585\\
3.68999999999982	78.227577187312\\
3.69199999999981	78.2119348002425\\
3.69399999999981	78.1962955410295\\
3.69599999999981	78.1806594090407\\
3.69799999999981	78.1650264036546\\
3.69999999999981	78.1493965242453\\
3.70199999999981	78.133769770188\\
3.70399999999981	78.1181461408591\\
3.70599999999981	78.1025256356321\\
3.70799999999981	78.0869082538813\\
3.70999999999981	78.0712939949851\\
3.71199999999981	78.0556828583158\\
3.71399999999981	78.0400748432509\\
3.71599999999981	78.024469949164\\
3.71799999999981	78.0088681754356\\
3.71999999999981	77.9932695214389\\
3.72199999999981	77.9776739865492\\
3.72399999999981	77.9620815701429\\
3.72599999999981	77.9464922715982\\
3.72799999999981	77.930906090292\\
3.72999999999981	77.9153230255972\\
3.73199999999981	77.8997430768939\\
3.73399999999981	77.8841662435595\\
3.73599999999981	77.8685925249695\\
3.73799999999981	77.8530219205006\\
3.73999999999981	77.8374544295308\\
3.74199999999981	77.8218900514384\\
3.74399999999981	77.8063287855987\\
3.74599999999981	77.7907706313912\\
3.74799999999981	77.775215588193\\
3.74999999999981	77.7596636553818\\
3.75199999999981	77.7441148323377\\
3.75399999999981	77.7285691184366\\
3.75599999999981	77.7130265130567\\
3.75799999999981	77.697487015578\\
3.75999999999981	77.6819506253771\\
3.76199999999981	77.6664173418348\\
3.76399999999981	77.6508871643275\\
3.76599999999981	77.6353600922361\\
3.76799999999981	77.6198361249386\\
3.76999999999981	77.6043152618144\\
3.77199999999981	77.5887975022439\\
3.77399999999981	77.5732828456051\\
3.77599999999981	77.5577712912765\\
3.77799999999981	77.5422628386404\\
3.77999999999981	77.5267574870763\\
3.78199999999981	77.511255235961\\
3.7839999999998	77.4957560846785\\
3.7859999999998	77.4802600326055\\
3.7879999999998	77.4647670791248\\
3.7899999999998	77.4492772236157\\
3.7919999999998	77.4337904654611\\
3.7939999999998	77.4183068040358\\
3.7959999999998	77.4028262387269\\
3.7979999999998	77.3873487689113\\
3.7999999999998	77.3718743939705\\
3.8019999999998	77.3564031132882\\
3.8039999999998	77.340934926242\\
3.8059999999998	77.3254698322181\\
3.8079999999998	77.3100078305932\\
3.8099999999998	77.29454892075\\
3.8119999999998	77.2790931020742\\
3.8139999999998	77.2636403739419\\
3.8159999999998	77.248190735739\\
3.8179999999998	77.2327441868473\\
3.8199999999998	77.2173007266468\\
3.8219999999998	77.2018603545215\\
3.8239999999998	77.186423069854\\
3.8259999999998	77.1709888720262\\
3.8279999999998	77.1555577604215\\
3.8299999999998	77.1401297344225\\
3.8319999999998	77.1247047934136\\
3.8339999999998	77.1092829367745\\
3.8359999999998	77.0938641638912\\
3.8379999999998	77.0784484741461\\
3.8399999999998	77.063035866924\\
3.8419999999998	77.0476263416064\\
3.8439999999998	77.0322198975789\\
3.8459999999998	77.016816534225\\
3.8479999999998	77.0014162509281\\
3.8499999999998	76.986019047072\\
3.8519999999998	76.9706249220403\\
3.8539999999998	76.9552338752202\\
3.8559999999998	76.939845905993\\
3.8579999999998	76.9244610137448\\
3.8599999999998	76.9090791978605\\
3.8619999999998	76.8937004577276\\
3.8639999999998	76.8783247927245\\
3.8659999999998	76.8629522022402\\
3.8679999999998	76.8475826856607\\
3.8699999999998	76.8322162423695\\
3.8719999999998	76.8168528717551\\
3.87399999999979	76.8014925731995\\
3.87599999999979	76.7861353460891\\
3.87799999999979	76.7707811898105\\
3.87999999999979	76.7554301037495\\
3.88199999999979	76.7400820872929\\
3.88399999999979	76.7247371398256\\
3.88599999999979	76.7093952607347\\
3.88799999999979	76.6940564494073\\
3.88999999999979	76.6787207052276\\
3.89199999999979	76.6633880275848\\
3.89399999999979	76.6480584158631\\
3.89599999999979	76.6327318694539\\
3.89799999999979	76.61740838774\\
3.89999999999979	76.6020879701101\\
3.90199999999979	76.5867706159512\\
3.90399999999979	76.5714563246513\\
3.90599999999979	76.5561450955969\\
3.90799999999979	76.5408369281768\\
3.90999999999979	76.5255318217798\\
3.91199999999979	76.51022977579\\
3.91399999999979	76.4949307895997\\
3.91599999999979	76.4796348625923\\
3.91799999999979	76.4643419941619\\
3.91999999999979	76.4490521836912\\
3.92199999999979	76.4337654305742\\
3.92399999999979	76.4184817341953\\
3.92599999999979	76.4032010939462\\
3.92799999999979	76.3879235092125\\
3.92999999999979	76.3726489793852\\
3.93199999999979	76.3573775038549\\
3.93399999999979	76.3421090820089\\
3.93599999999979	76.3268437132365\\
3.93799999999979	76.3115813969272\\
3.93999999999979	76.2963221324707\\
3.94199999999979	76.2810659192583\\
3.94399999999979	76.2658127566792\\
3.94599999999979	76.2505626441217\\
3.94799999999979	76.2353155809781\\
3.94999999999979	76.2200715666369\\
3.95199999999979	76.2048306004893\\
3.95399999999979	76.1895926819257\\
3.95599999999979	76.1743578103376\\
3.95799999999979	76.1591259851123\\
3.95999999999979	76.1438972056449\\
3.96199999999979	76.1286714713248\\
3.96399999999979	76.1134487815435\\
3.96599999999978	76.0982291356917\\
3.96799999999978	76.0830125331592\\
3.96999999999978	76.0677989733395\\
3.97199999999978	76.0525884556251\\
3.97399999999978	76.0373809794042\\
3.97599999999978	76.0221765440702\\
3.97799999999978	76.0069751490167\\
3.97999999999978	75.9917767936347\\
3.98199999999978	75.9765814773154\\
3.98399999999978	75.9613891994533\\
3.98599999999978	75.9461999594388\\
3.98799999999978	75.9310137566643\\
3.98999999999978	75.9158305905241\\
3.99199999999978	75.9006504604106\\
3.99399999999978	75.8854733657142\\
3.99599999999978	75.8702993058315\\
3.99799999999978	75.8551282801527\\
3.99999999999978	75.8399602880746\\
4.00199999999978	75.8247953289869\\
4.00399999999978	75.8096334022877\\
4.00599999999978	75.7944745073639\\
4.00799999999978	75.7793186436156\\
4.00999999999978	75.7641658104333\\
4.01199999999978	75.7490160072108\\
4.01399999999978	75.7338692333445\\
4.01599999999978	75.7187254882281\\
4.01799999999978	75.7035847712535\\
4.01999999999978	75.6884470818199\\
4.02199999999978	75.6733124193162\\
4.02399999999978	75.6581807831392\\
4.02599999999978	75.6430521726884\\
4.02799999999978	75.6279265873519\\
4.02999999999978	75.6128040265274\\
4.03199999999978	75.5976844896113\\
4.03399999999978	75.5825679759983\\
4.03599999999978	75.5674544850824\\
4.03799999999978	75.5523440162622\\
4.03999999999978	75.537236568932\\
4.04199999999978	75.522132142487\\
4.04399999999978	75.5070307363225\\
4.04599999999978	75.4919323498355\\
4.04799999999978	75.4768369824212\\
4.04999999999978	75.461744633478\\
4.05199999999978	75.4466553024023\\
4.05399999999978	75.4315689885872\\
4.05599999999978	75.4164856914343\\
4.05799999999978	75.4014054103355\\
4.05999999999977	75.3863281446916\\
4.06199999999977	75.3712538938983\\
4.06399999999977	75.3561826573503\\
4.06599999999977	75.3411144344475\\
4.06799999999977	75.3260492245881\\
4.06999999999977	75.3109870271671\\
4.07199999999977	75.2959278415851\\
4.07399999999977	75.2808716672378\\
4.07599999999977	75.2658185035236\\
4.07799999999977	75.2507683498388\\
4.07999999999977	75.235721205583\\
4.08199999999977	75.2206770701551\\
4.08399999999977	75.2056359429533\\
4.08599999999977	75.1905978233758\\
4.08799999999977	75.1755627108208\\
4.08999999999977	75.1605306046844\\
4.09199999999977	75.1455015043714\\
4.09399999999977	75.130475409276\\
4.09599999999977	75.1154523187993\\
4.09799999999977	75.1004322323408\\
4.09999999999977	75.0854151492963\\
4.10199999999977	75.070401069071\\
4.10399999999977	75.0553899910607\\
4.10599999999977	75.0403819146641\\
4.10799999999977	75.0253768392852\\
4.10999999999977	75.0103747643209\\
4.11199999999977	74.9953756891712\\
4.11399999999977	74.9803796132358\\
4.11599999999977	74.9653865359161\\
4.11799999999977	74.9503964566137\\
4.11999999999977	74.9354093747271\\
4.12199999999977	74.9204252896585\\
4.12399999999977	74.9054442008065\\
4.12599999999977	74.8904661075742\\
4.12799999999977	74.8754910093622\\
4.12999999999977	74.86051890557\\
4.13199999999977	74.8455497955997\\
4.13399999999977	74.8305836788539\\
4.13599999999977	74.8156205547311\\
4.13799999999977	74.8006604226364\\
4.13999999999977	74.7857032819695\\
4.14199999999977	74.7707491321327\\
4.14399999999977	74.7557979725292\\
4.14599999999977	74.7408498025583\\
4.14799999999977	74.7259046216231\\
4.14999999999976	74.7109624291306\\
4.15199999999976	74.6960232244737\\
4.15399999999976	74.6810870070621\\
4.15599999999976	74.666153776297\\
4.15799999999976	74.6512235315822\\
4.15999999999976	74.636296272318\\
4.16199999999976	74.6213719979099\\
4.16399999999976	74.6064507077585\\
4.16599999999976	74.5915324012695\\
4.16799999999976	74.5766170778437\\
4.16999999999976	74.5617047368885\\
4.17199999999976	74.5467953778059\\
4.17399999999976	74.5318889999949\\
4.17599999999976	74.5169856028671\\
4.17799999999976	74.502085185822\\
4.17999999999976	74.4871877482626\\
4.18199999999976	74.4722932895975\\
4.18399999999976	74.4574018092257\\
4.18599999999976	74.4425133065566\\
4.18799999999976	74.4276277809936\\
4.18999999999976	74.4127452319387\\
4.19199999999976	74.3978656587991\\
4.19399999999976	74.3829890609781\\
4.19599999999976	74.3681154378835\\
4.19799999999976	74.3532447889179\\
4.19999999999976	74.3383771134878\\
4.20199999999976	74.3235124109974\\
4.20399999999976	74.308650680856\\
4.20599999999976	74.2937919224624\\
4.20799999999976	74.2789361352274\\
4.20999999999976	74.2640833185582\\
4.21199999999976	74.2492334718551\\
4.21399999999976	74.2343865945298\\
4.21599999999976	74.2195426859851\\
4.21799999999976	74.2047017456305\\
4.21999999999976	74.1898637728666\\
4.22199999999976	74.1750287671079\\
4.22399999999976	74.1601967277543\\
4.22599999999976	74.1453676542181\\
4.22799999999976	74.1305415459023\\
4.22999999999976	74.1157184022159\\
4.23199999999976	74.1008982225641\\
4.23399999999976	74.086081006358\\
4.23599999999976	74.0712667529998\\
4.23799999999976	74.0564554619017\\
4.23999999999976	74.0416471324683\\
4.24199999999975	74.0268417641103\\
4.24399999999975	74.0120393562308\\
4.24599999999975	73.9972399082445\\
4.24799999999975	73.9824434195557\\
4.24999999999975	73.9676498895696\\
4.25199999999975	73.9528593177013\\
4.25399999999975	73.9380717033547\\
4.25599999999975	73.9232870459401\\
4.25799999999975	73.9085053448653\\
4.25999999999975	73.8937265995392\\
4.26199999999975	73.8789508093714\\
4.26399999999975	73.8641779737734\\
4.26599999999975	73.8494080921484\\
4.26799999999975	73.8346411639101\\
4.26999999999975	73.8198771884681\\
4.27199999999975	73.8051161652292\\
4.27399999999975	73.7903580936041\\
4.27599999999975	73.7756029730069\\
4.27799999999975	73.7608508028387\\
4.27999999999975	73.7461015825173\\
4.28199999999975	73.7313553114507\\
4.28399999999975	73.7166119890485\\
4.28599999999975	73.70187161472\\
4.28799999999975	73.687134187878\\
4.28999999999975	73.6723997079295\\
4.29199999999975	73.6576681742912\\
4.29399999999975	73.6429395863708\\
4.29599999999975	73.6282139435777\\
4.29799999999975	73.6134912453216\\
4.29999999999975	73.5987714910202\\
4.30199999999975	73.5840546800796\\
4.30399999999975	73.5693408119128\\
4.30599999999975	73.5546298859307\\
4.30799999999975	73.539921901546\\
4.30999999999975	73.5252168581698\\
4.31199999999975	73.510514755216\\
4.31399999999975	73.4958155920923\\
4.31599999999975	73.481119368214\\
4.31799999999975	73.4664260829928\\
4.31999999999975	73.4517357358419\\
4.32199999999975	73.4370483261743\\
4.32399999999975	73.4223638533983\\
4.32599999999975	73.4076823169286\\
4.32799999999975	73.3930037161853\\
4.32999999999975	73.3783280505693\\
4.33199999999974	73.3636553195021\\
4.33399999999974	73.3489855223943\\
4.33599999999974	73.3343186586591\\
4.33799999999974	73.3196547277085\\
4.33999999999974	73.304993728958\\
4.34199999999974	73.2903356618232\\
4.34399999999974	73.2756805257147\\
4.34599999999974	73.2610283200469\\
4.34799999999974	73.2463790442345\\
4.34999999999974	73.2317326976923\\
4.35199999999974	73.2170892798323\\
4.35399999999974	73.2024487900698\\
4.35599999999974	73.1878112278208\\
4.35799999999974	73.1731765924991\\
4.35999999999974	73.1585448835195\\
4.36199999999974	73.1439161002971\\
4.36399999999974	73.1292902422435\\
4.36599999999974	73.11466730878\\
4.36799999999974	73.1000472993174\\
4.36999999999974	73.0854302132684\\
4.37199999999974	73.0708160500565\\
4.37399999999974	73.0562048090906\\
4.37599999999974	73.0415964897905\\
4.37799999999974	73.0269910915708\\
4.37999999999974	73.0123886138432\\
4.38199999999974	72.9977890560292\\
4.38399999999974	72.9831924175429\\
4.38599999999974	72.9685986978012\\
4.38799999999974	72.9540078962186\\
4.38999999999974	72.9394200122119\\
4.39199999999974	72.9248350452008\\
4.39399999999974	72.9102529945997\\
4.39599999999974	72.8956738598247\\
4.39799999999974	72.8810976402939\\
4.39999999999974	72.8665243354236\\
4.40199999999974	72.8519539446322\\
4.40399999999974	72.8373864673355\\
4.40599999999974	72.8228219029524\\
4.40799999999974	72.8082602508997\\
4.40999999999974	72.7937015105955\\
4.41199999999974	72.7791456814556\\
4.41399999999974	72.7645927629009\\
4.41599999999974	72.7500427543474\\
4.41799999999974	72.7354956552137\\
4.41999999999974	72.7209514649184\\
4.42199999999974	72.7064101828781\\
4.42399999999973	72.6918718085166\\
4.42599999999973	72.6773363412439\\
4.42799999999973	72.6628037804856\\
4.42999999999973	72.6482741256584\\
4.43199999999973	72.6337473761814\\
4.43399999999973	72.6192235314724\\
4.43599999999973	72.6047025909529\\
4.43799999999973	72.5901845540376\\
4.43999999999973	72.5756694201523\\
4.44199999999973	72.561157188714\\
4.44399999999973	72.5466478591382\\
4.44599999999973	72.53214143085\\
4.44799999999973	72.5176379032686\\
4.44999999999973	72.5031372758103\\
4.45199999999973	72.4886395478993\\
4.45399999999973	72.4741447189529\\
4.45599999999973	72.4596527883931\\
4.45799999999973	72.4451637556394\\
4.45999999999973	72.4306776201129\\
4.46199999999973	72.4161943812346\\
4.46399999999973	72.4017140384233\\
4.46599999999973	72.3872365911047\\
4.46799999999973	72.3727620386944\\
4.46999999999973	72.3582903806179\\
4.47199999999973	72.3438216162925\\
4.47399999999973	72.3293557451411\\
4.47599999999973	72.3148927665868\\
4.47799999999973	72.3004326800481\\
4.47999999999973	72.2859754849491\\
4.48199999999973	72.2715211807113\\
4.48399999999973	72.2570697667559\\
4.48599999999973	72.2426212425052\\
4.48799999999973	72.2281756073823\\
4.48999999999973	72.2137328608089\\
4.49199999999973	72.1992930022066\\
4.49399999999973	72.1848560309988\\
4.49599999999973	72.1704219466082\\
4.49799999999973	72.1559907484561\\
4.49999999999973	72.1415624359678\\
4.50199999999973	72.1271370085645\\
4.50399999999973	72.1127144656689\\
4.50599999999973	72.0982948067064\\
4.50799999999973	72.0838780310984\\
4.50999999999973	72.0694641382701\\
4.51199999999973	72.0550531276424\\
4.51399999999972	72.0406449986409\\
4.51599999999972	72.0262397506895\\
4.51799999999972	72.0118373832116\\
4.51999999999972	71.9974378956296\\
4.52199999999972	71.98304128737\\
4.52399999999972	71.9686475578582\\
4.52599999999972	71.954256706516\\
4.52799999999972	71.9398687327677\\
4.52999999999972	71.9254836360388\\
4.53199999999972	71.9111014157543\\
4.53399999999972	71.8967220713395\\
4.53599999999972	71.8823456022164\\
4.53799999999972	71.8679720078147\\
4.53999999999972	71.8536012875557\\
4.54199999999972	71.8392334408655\\
4.54399999999972	71.8248684671705\\
4.54599999999972	71.8105063658952\\
4.54799999999972	71.7961471364676\\
4.54999999999972	71.7817907783108\\
4.55199999999972	71.7674372908513\\
4.55399999999972	71.7530866735148\\
4.55599999999972	71.738738925729\\
4.55799999999972	71.7243940469182\\
4.55999999999972	71.7100520365077\\
4.56199999999972	71.6957128939297\\
4.56399999999972	71.6813766186035\\
4.56599999999972	71.6670432099622\\
4.56799999999972	71.6527126674257\\
4.56999999999972	71.6383849904279\\
4.57199999999972	71.6240601783909\\
4.57399999999972	71.609738230743\\
4.57599999999972	71.5954191469123\\
4.57799999999972	71.5811029263254\\
4.57999999999972	71.5667895684104\\
4.58199999999972	71.5524790725955\\
4.58399999999972	71.5381714383068\\
4.58599999999972	71.5238666649724\\
4.58799999999972	71.5095647520229\\
4.58999999999972	71.4952656988812\\
4.59199999999972	71.4809695049785\\
4.59399999999972	71.4666761697425\\
4.59599999999972	71.4523856926038\\
4.59799999999972	71.4380980729885\\
4.59999999999972	71.423813310325\\
4.60199999999972	71.4095314040419\\
4.60399999999971	71.3952523535706\\
4.60599999999971	71.3809761583392\\
4.60799999999971	71.3667028177733\\
4.60999999999971	71.3524323313054\\
4.61199999999971	71.3381646983678\\
4.61399999999971	71.3238999183812\\
4.61599999999971	71.3096379907829\\
4.61799999999971	71.2953789149994\\
4.61999999999971	71.2811226904594\\
4.62199999999971	71.2668693165963\\
4.62399999999971	71.2526187928371\\
4.62599999999971	71.2383711186125\\
4.62799999999971	71.2241262933526\\
4.62999999999971	71.2098843164875\\
4.63199999999971	71.1956451874483\\
4.63399999999971	71.1814089056674\\
4.63599999999971	71.1671754705741\\
4.63799999999971	71.1529448815955\\
4.63999999999971	71.1387171381674\\
4.64199999999971	71.1244922397177\\
4.64399999999971	71.1102701856801\\
4.64599999999971	71.0960509754833\\
4.64799999999971	71.0818346085617\\
4.64999999999971	71.0676210843449\\
4.65199999999971	71.0534104022631\\
4.65399999999971	71.0392025617502\\
4.65599999999971	71.0249975622376\\
4.65799999999971	71.0107954031546\\
4.65999999999971	70.9965960839383\\
4.66199999999971	70.9823996040162\\
4.66399999999971	70.9682059628241\\
4.66599999999971	70.9540151597896\\
4.66799999999971	70.939827194351\\
4.66999999999971	70.9256420659363\\
4.67199999999971	70.9114597739819\\
4.67399999999971	70.897280317917\\
4.67599999999971	70.8831036971775\\
4.67799999999971	70.8689299111945\\
4.67999999999971	70.8547589594004\\
4.68199999999971	70.8405908412331\\
4.68399999999971	70.8264255561196\\
4.68599999999971	70.8122631034985\\
4.68799999999971	70.7981034828006\\
4.68999999999971	70.7839466934628\\
4.69199999999971	70.7697927349146\\
4.69399999999971	70.7556416065926\\
4.6959999999997	70.7414933079303\\
4.6979999999997	70.7273478383613\\
4.6999999999997	70.7132051973206\\
4.7019999999997	70.6990653842444\\
4.7039999999997	70.6849283985648\\
4.7059999999997	70.6707942397153\\
4.7079999999997	70.6566629071329\\
4.7099999999997	70.6425344002533\\
4.7119999999997	70.6284087185079\\
4.7139999999997	70.6142858613355\\
4.7159999999997	70.6001658281684\\
4.7179999999997	70.5860486184453\\
4.7199999999997	70.5719342315969\\
4.7219999999997	70.5578226670652\\
4.7239999999997	70.5437139242794\\
4.7259999999997	70.5296080026775\\
4.7279999999997	70.5155049016968\\
4.7299999999997	70.5014046207727\\
4.7319999999997	70.4873071593397\\
4.7339999999997	70.4732125168367\\
4.7359999999997	70.4591206926976\\
4.7379999999997	70.4450316863581\\
4.7399999999997	70.4309454972581\\
4.7419999999997	70.4168621248333\\
4.7439999999997	70.4027815685194\\
4.7459999999997	70.3887038277536\\
4.7479999999997	70.3746289019722\\
4.7499999999997	70.3605567906145\\
4.7519999999997	70.3464874931149\\
4.7539999999997	70.3324210089122\\
4.7559999999997	70.3183573374446\\
4.7579999999997	70.3042964781493\\
4.7599999999997	70.2902384304612\\
4.7619999999997	70.2761831938236\\
4.7639999999997	70.2621307676688\\
4.7659999999997	70.2480811514401\\
4.7679999999997	70.2340343445689\\
4.7699999999997	70.2199903464976\\
4.7719999999997	70.2059491566677\\
4.7739999999997	70.1919107745126\\
4.7759999999997	70.1778751994715\\
4.7779999999997	70.163842430985\\
4.7799999999997	70.1498124684903\\
4.7819999999997	70.135785311429\\
4.7839999999997	70.1217609592357\\
4.78599999999969	70.1077394113528\\
4.78799999999969	70.0937206672199\\
4.78999999999969	70.0797047262732\\
4.79199999999969	70.0656915879546\\
4.79399999999969	70.051681251705\\
4.79599999999969	70.0376737169606\\
4.79799999999969	70.0236689831638\\
4.79999999999969	70.0096670497536\\
4.80199999999969	69.9956679161682\\
4.80399999999969	69.9816715818536\\
4.80599999999969	69.9676780462445\\
4.80799999999969	69.9536873087821\\
4.80999999999969	69.9396993689083\\
4.81199999999969	69.9257142260612\\
4.81399999999969	69.911731879685\\
4.81599999999969	69.8977523292197\\
4.81799999999969	69.8837755741037\\
4.81999999999969	69.8698016137811\\
4.82199999999969	69.8558304476922\\
4.82399999999969	69.8418620752762\\
4.82599999999969	69.8278964959759\\
4.82799999999969	69.8139337092339\\
4.82999999999969	69.7999737144913\\
4.83199999999969	69.7860165111887\\
4.83399999999969	69.7720620987684\\
4.83599999999969	69.7581104766729\\
4.83799999999969	69.7441616443422\\
4.83999999999969	69.7302156012226\\
4.84199999999969	69.7162723467527\\
4.84399999999969	69.7023318803757\\
4.84599999999969	69.6883942015358\\
4.84799999999969	69.6744593096743\\
4.84999999999969	69.6605272042322\\
4.85199999999969	69.6465978846555\\
4.85399999999969	69.6326713503853\\
4.85599999999969	69.6187476008643\\
4.85799999999969	69.6048266355369\\
4.85999999999969	69.5909084538464\\
4.86199999999969	69.5769930552341\\
4.86399999999969	69.5630804391461\\
4.86599999999969	69.5491706050267\\
4.86799999999969	69.5352635523146\\
4.86999999999969	69.5213592804579\\
4.87199999999969	69.5074577889001\\
4.87399999999969	69.4935590770858\\
4.87599999999969	69.4796631444546\\
4.87799999999968	69.4657699904578\\
4.87999999999968	69.451879614533\\
4.88199999999968	69.4379920161307\\
4.88399999999968	69.4241071946925\\
4.88599999999968	69.4102251496615\\
4.88799999999968	69.3963458804852\\
4.88999999999968	69.3824693866065\\
4.89199999999968	69.3685956674735\\
4.89399999999968	69.3547247225294\\
4.89599999999968	69.3408565512175\\
4.89799999999968	69.3269911529882\\
4.89999999999968	69.3131285272808\\
4.90199999999968	69.2992686735473\\
4.90399999999968	69.2854115912279\\
4.90599999999968	69.2715572797726\\
4.90799999999968	69.2577057386239\\
4.90999999999968	69.2438569672299\\
4.91199999999968	69.2300109650381\\
4.91399999999968	69.2161677314911\\
4.91599999999968	69.202327266037\\
4.91799999999968	69.1884895681233\\
4.91999999999968	69.1746546371955\\
4.92199999999968	69.1608224727013\\
4.92399999999968	69.1469930740865\\
4.92599999999968	69.133166440798\\
4.92799999999968	69.1193425722831\\
4.92999999999968	69.1055214679884\\
4.93199999999968	69.091703127363\\
4.93399999999968	69.077887549854\\
4.93599999999968	69.0640747349078\\
4.93799999999968	69.0502646819698\\
4.93999999999968	69.0364573904909\\
4.94199999999968	69.0226528599188\\
4.94399999999968	69.0088510897017\\
4.94599999999968	68.9950520792856\\
4.94799999999968	68.9812558281195\\
4.94999999999968	68.9674623356518\\
4.95199999999968	68.9536716013305\\
4.95399999999968	68.9398836246066\\
4.95599999999968	68.926098404926\\
4.95799999999968	68.9123159417374\\
4.95999999999968	68.8985362344902\\
4.96199999999968	68.8847592826326\\
4.96399999999968	68.8709850856155\\
4.96599999999968	68.8572136428869\\
4.96799999999967	68.8434449538962\\
4.96999999999967	68.829679018092\\
4.97199999999967	68.8159158349246\\
4.97399999999967	68.8021554038437\\
4.97599999999967	68.7883977242996\\
4.97799999999967	68.7746427957404\\
4.97999999999967	68.7608906176161\\
4.98199999999967	68.7471411893768\\
4.98399999999967	68.7333945104746\\
4.98599999999967	68.7196505803602\\
4.98799999999967	68.7059093984783\\
4.98999999999967	68.6921709642864\\
4.99199999999967	68.6784352772304\\
4.99399999999967	68.6647023367621\\
4.99599999999967	68.6509721423347\\
4.99799999999967	68.6372446933954\\
4.99999999999967	68.6235199893982\\
5.00199999999967	68.6097980297923\\
5.00399999999967	68.5960788140278\\
5.00599999999967	68.5823623415619\\
5.00799999999967	68.5686486118392\\
5.00999999999967	68.5549376243108\\
5.01199999999967	68.541229378437\\
5.01399999999967	68.5275238736624\\
5.01599999999967	68.5138211094405\\
5.01799999999967	68.5001210852209\\
5.01999999999967	68.4864238004621\\
5.02199999999967	68.4727292546114\\
5.02399999999967	68.4590374471215\\
5.02599999999967	68.4453483774462\\
5.02799999999967	68.4316620450373\\
5.02999999999967	68.4179784493468\\
5.03199999999967	68.4042975898288\\
5.03399999999967	68.3906194659365\\
5.03599999999967	68.376944077121\\
5.03799999999967	68.3632714228351\\
5.03999999999967	68.3496015025363\\
5.04199999999967	68.335934315672\\
5.04399999999967	68.3222698616986\\
5.04599999999967	68.3086081400714\\
5.04799999999967	68.2949491502412\\
5.04999999999967	68.2812928916621\\
5.05199999999967	68.2676393637905\\
5.05399999999967	68.2539885660763\\
5.05599999999967	68.2403404979761\\
5.05799999999966	68.2266951589456\\
5.05999999999966	68.2130525484349\\
5.06199999999966	68.199412665902\\
5.06399999999966	68.1857755107993\\
5.06599999999966	68.172141082582\\
5.06799999999966	68.1585093807064\\
5.06999999999966	68.1448804046259\\
5.07199999999966	68.131254153794\\
5.07399999999966	68.1176306276683\\
5.07599999999966	68.1040098257047\\
5.07799999999966	68.0903917473549\\
5.07999999999966	68.0767763920759\\
5.08199999999966	68.0631637593237\\
5.08399999999966	68.0495538485545\\
5.08599999999966	68.035946659222\\
5.08799999999966	68.0223421907831\\
5.08999999999966	68.0087404426942\\
5.09199999999966	67.9951414144116\\
5.09399999999966	67.98154510539\\
5.09599999999966	67.9679515150869\\
5.09799999999966	67.9543606429591\\
5.09999999999966	67.940772488462\\
5.10199999999966	67.9271870510512\\
5.10399999999966	67.9136043301843\\
5.10599999999966	67.9000243253199\\
5.10799999999966	67.8864470359114\\
5.10999999999966	67.8728724614198\\
5.11199999999966	67.859300601299\\
5.11399999999966	67.8457314550086\\
5.11599999999966	67.8321650220037\\
5.11799999999966	67.818601301744\\
5.11999999999966	67.8050402936848\\
5.12199999999966	67.7914819972848\\
5.12399999999966	67.7779264120036\\
5.12599999999966	67.7643735372961\\
5.12799999999966	67.7508233726207\\
5.12999999999966	67.7372759174378\\
5.13199999999966	67.7237311712019\\
5.13399999999966	67.7101891333751\\
5.13599999999966	67.6966498034148\\
5.13799999999966	67.6831131807783\\
5.13999999999966	67.6695792649255\\
5.14199999999966	67.6560480553147\\
5.14399999999966	67.6425195514054\\
5.14599999999966	67.6289937526529\\
5.14799999999966	67.6154706585213\\
5.14999999999965	67.6019502684685\\
5.15199999999965	67.5884325819524\\
5.15399999999965	67.5749175984322\\
5.15599999999965	67.5614053173683\\
5.15799999999965	67.5478957382208\\
5.15999999999965	67.5343888604491\\
5.16199999999965	67.5208846835117\\
5.16399999999965	67.5073832068695\\
5.16599999999965	67.4938844299842\\
5.16799999999965	67.4803883523133\\
5.16999999999965	67.4668949733204\\
5.17199999999965	67.4534042924607\\
5.17399999999965	67.4399163091993\\
5.17599999999965	67.4264310229939\\
5.17799999999965	67.4129484333071\\
5.17999999999965	67.3994685396002\\
5.18199999999965	67.3859913413303\\
5.18399999999965	67.3725168379638\\
5.18599999999965	67.3590450289576\\
5.18799999999965	67.3455759137745\\
5.18999999999965	67.3321094918763\\
5.19199999999965	67.3186457627238\\
5.19399999999965	67.3051847257778\\
5.19599999999965	67.291726380502\\
5.19799999999965	67.2782707263576\\
5.19999999999965	67.2648177628047\\
5.20199999999965	67.2513674893062\\
5.20399999999965	67.2379199053266\\
5.20599999999965	67.2244750103223\\
5.20799999999965	67.2110328037624\\
5.20999999999965	67.1975932851046\\
5.21199999999965	67.1841564538146\\
5.21399999999965	67.1707223093541\\
5.21599999999965	67.1572908511825\\
5.21799999999965	67.1438620787671\\
5.21999999999965	67.1304359915667\\
5.22199999999965	67.1170125890509\\
5.22399999999965	67.1035918706763\\
5.22599999999965	67.0901738359092\\
5.22799999999965	67.0767584842116\\
5.22999999999965	67.0633458150485\\
5.23199999999965	67.0499358278829\\
5.23399999999965	67.0365285221784\\
5.23599999999965	67.0231238973991\\
5.23799999999965	67.0097219530097\\
5.23999999999964	66.9963226884712\\
5.24199999999964	66.9829261032506\\
5.24399999999964	66.9695321968099\\
5.24599999999964	66.9561409686172\\
5.24799999999964	66.9427524181333\\
5.24999999999964	66.9293665448251\\
5.25199999999964	66.9159833481541\\
5.25399999999964	66.9026028275889\\
5.25599999999964	66.8892249825925\\
5.25799999999964	66.8758498126295\\
5.25999999999964	66.8624773171665\\
5.26199999999964	66.849107495666\\
5.26399999999964	66.8357403475985\\
5.26599999999964	66.8223758724223\\
5.26799999999964	66.8090140696087\\
5.26999999999964	66.7956549386215\\
5.27199999999964	66.7822984789248\\
5.27399999999964	66.7689446899879\\
5.27599999999964	66.755593571274\\
5.27799999999964	66.7422451222483\\
5.27999999999964	66.7288993423803\\
5.28199999999964	66.7155562311335\\
5.28399999999964	66.7022157879767\\
5.28599999999964	66.6888780123733\\
5.28799999999964	66.6755429037921\\
5.28999999999964	66.6622104617012\\
5.29199999999964	66.6488806855638\\
5.29399999999964	66.6355535748494\\
5.29599999999964	66.6222291290224\\
5.29799999999964	66.6089073475524\\
5.29999999999964	66.5955882299064\\
5.30199999999964	66.5822717755513\\
5.30399999999964	66.5689579839552\\
5.30599999999964	66.5556468545847\\
5.30799999999964	66.5423383869072\\
5.30999999999964	66.5290325803909\\
5.31199999999964	66.5157294345037\\
5.31399999999964	66.5024289487132\\
5.31599999999964	66.48913112249\\
5.31799999999964	66.4758359552988\\
5.31999999999964	66.4625434466078\\
5.32199999999964	66.4492535958899\\
5.32399999999964	66.4359664026094\\
5.32599999999964	66.4226818662356\\
5.32799999999964	66.4093999862379\\
5.32999999999964	66.3961207620867\\
5.33199999999963	66.3828441932481\\
5.33399999999963	66.3695702791917\\
5.33599999999963	66.3562990193866\\
5.33799999999963	66.3430304133047\\
5.33999999999963	66.3297644604115\\
5.34199999999963	66.3165011601802\\
5.34399999999963	66.3032405120784\\
5.34599999999963	66.2899825155749\\
5.34799999999963	66.2767271701413\\
5.34999999999963	66.2634744752457\\
5.35199999999963	66.2502244303604\\
5.35399999999963	66.236977034954\\
5.35599999999963	66.2237322884964\\
5.35799999999963	66.2104901904577\\
5.35999999999963	66.197250740309\\
5.36199999999963	66.1840139375221\\
5.36399999999963	66.1707797815651\\
5.36599999999963	66.1575482719122\\
5.36799999999963	66.1443194080301\\
5.36999999999963	66.131093189391\\
5.37199999999963	66.1178696154684\\
5.37399999999963	66.104648685731\\
5.37599999999963	66.0914303996508\\
5.37799999999963	66.0782147566996\\
5.37999999999963	66.0650017563489\\
5.38199999999963	66.0517913980684\\
5.38399999999963	66.0385836813314\\
5.38599999999963	66.0253786056114\\
5.38799999999963	66.0121761703768\\
5.38999999999963	65.9989763751026\\
5.39199999999963	65.9857792192591\\
5.39399999999963	65.972584702318\\
5.39599999999963	65.9593928237528\\
5.39799999999963	65.9462035830347\\
5.39999999999963	65.9330169796395\\
5.40199999999963	65.9198330130371\\
5.40399999999963	65.906651682701\\
5.40599999999963	65.8934729881027\\
5.40799999999963	65.8802969287162\\
5.40999999999963	65.8671235040169\\
5.41199999999963	65.8539527134737\\
5.41399999999963	65.8407845565632\\
5.41599999999963	65.8276190327567\\
5.41799999999963	65.8144561415278\\
5.41999999999963	65.8012958823522\\
5.42199999999962	65.7881382547005\\
5.42399999999962	65.7749832580493\\
5.42599999999962	65.7618308918705\\
5.42799999999962	65.7486811556395\\
5.42999999999962	65.7355340488285\\
5.43199999999962	65.7223895709154\\
5.43399999999962	65.7092477213715\\
5.43599999999962	65.6961084996715\\
5.43799999999962	65.6829719052899\\
5.43999999999962	65.6698379377023\\
5.44199999999962	65.6567065963833\\
5.44399999999962	65.6435778808072\\
5.44599999999962	65.6304517904495\\
5.44799999999962	65.6173283247836\\
5.44999999999962	65.6042074832871\\
5.45199999999962	65.5910892654341\\
5.45399999999962	65.5779736707008\\
5.45599999999962	65.5648606985606\\
5.45799999999962	65.5517503484904\\
5.45999999999962	65.5386426199663\\
5.46199999999962	65.525537512463\\
5.46399999999962	65.5124350254596\\
5.46599999999962	65.4993351584266\\
5.46799999999962	65.4862379108445\\
5.46999999999962	65.4731432821886\\
5.47199999999962	65.4600512719347\\
5.47399999999962	65.4469618795581\\
5.47599999999962	65.433875104538\\
5.47799999999962	65.4207909463494\\
5.47999999999962	65.4077094044665\\
5.48199999999962	65.3946304783721\\
5.48399999999962	65.3815541675375\\
5.48599999999962	65.3684804714438\\
5.48799999999962	65.3554093895659\\
5.48999999999962	65.3423409213819\\
5.49199999999962	65.329275066368\\
5.49399999999962	65.3162118240031\\
5.49599999999962	65.3031511937645\\
5.49799999999962	65.29009317513\\
5.49999999999962	65.2770377675764\\
5.50199999999962	65.2639849705826\\
5.50399999999962	65.2509347836249\\
5.50599999999962	65.2378872061835\\
5.50799999999962	65.2248422377354\\
5.50999999999962	65.2117998777606\\
5.51199999999961	65.1987601257361\\
5.51399999999961	65.1857229811394\\
5.51599999999961	65.1726884434502\\
5.51799999999961	65.1596565121481\\
5.51999999999961	65.146627186711\\
5.52199999999961	65.1336004666173\\
5.52399999999961	65.1205763513477\\
5.52599999999961	65.107554840379\\
5.52799999999961	65.0945359331917\\
5.52999999999961	65.0815196292664\\
5.53199999999961	65.0685059280812\\
5.53399999999961	65.0554948291157\\
5.53599999999961	65.0424863318484\\
5.53799999999961	65.0294804357625\\
5.53999999999961	65.016477140334\\
5.54199999999961	65.0034764450447\\
5.54399999999961	64.9904783493752\\
5.54599999999961	64.9774828528037\\
5.54799999999961	64.9644899548119\\
5.54999999999961	64.9514996548822\\
5.55199999999961	64.9385119524915\\
5.55399999999961	64.9255268471217\\
5.55599999999961	64.9125443382559\\
5.55799999999961	64.8995644253696\\
5.55999999999961	64.8865871079475\\
5.56199999999961	64.8736123854712\\
5.56399999999961	64.8606402574193\\
5.56599999999961	64.8476707232744\\
5.56799999999961	64.8347037825172\\
5.56999999999961	64.821739434631\\
5.57199999999961	64.8087776790941\\
5.57399999999961	64.7958185153915\\
5.57599999999961	64.7828619430044\\
5.57799999999961	64.7699079614113\\
5.57999999999961	64.7569565700967\\
5.58199999999961	64.7440077685435\\
5.58399999999961	64.7310615562327\\
5.58599999999961	64.7181179326466\\
5.58799999999961	64.7051768972654\\
5.58999999999961	64.6922384495768\\
5.59199999999961	64.6793025890597\\
5.59399999999961	64.666369315196\\
5.59599999999961	64.6534386274713\\
5.59799999999961	64.6405105253663\\
5.59999999999961	64.6275850083627\\
5.60199999999961	64.6146620759478\\
5.6039999999996	64.601741727603\\
5.6059999999996	64.5888239628111\\
5.6079999999996	64.5759087810535\\
5.6099999999996	64.562996181818\\
5.6119999999996	64.5500861645846\\
5.6139999999996	64.537178728839\\
5.6159999999996	64.5242738740647\\
5.6179999999996	64.5113715997445\\
5.6199999999996	64.4984719053627\\
5.6219999999996	64.4855747904041\\
5.6239999999996	64.4726802543534\\
5.6259999999996	64.459788296694\\
5.6279999999996	64.4468989169113\\
5.6299999999996	64.4340121144886\\
5.6319999999996	64.4211278889103\\
5.6339999999996	64.4082462396625\\
5.6359999999996	64.3953671662297\\
5.6379999999996	64.3824906680948\\
5.6399999999996	64.3696167447469\\
5.6419999999996	64.3567453956676\\
5.6439999999996	64.3438766203436\\
5.6459999999996	64.3310104182595\\
5.6479999999996	64.3181467889017\\
5.6499999999996	64.3052857317564\\
5.6519999999996	64.292427246306\\
5.6539999999996	64.2795713320397\\
5.6559999999996	64.266717988442\\
5.6579999999996	64.253867215\\
5.6599999999996	64.2410190111971\\
5.6619999999996	64.2281733765213\\
5.6639999999996	64.2153303104603\\
5.6659999999996	64.2024898124962\\
5.6679999999996	64.1896518821214\\
5.6699999999996	64.1768165188168\\
5.6719999999996	64.1639837220727\\
5.6739999999996	64.1511534913733\\
5.6759999999996	64.1383258262102\\
5.6779999999996	64.1255007260637\\
5.6799999999996	64.1126781904254\\
5.6819999999996	64.0998582187828\\
5.6839999999996	64.0870408106206\\
5.6859999999996	64.0742259654266\\
5.6879999999996	64.0614136826896\\
5.6899999999996	64.0486039618979\\
5.6919999999996	64.0357968025377\\
5.69399999999959	64.0229922040957\\
5.69599999999959	64.0101901660634\\
5.69799999999959	63.9973906879256\\
5.69999999999959	63.9845937691723\\
5.70199999999959	63.9717994092895\\
5.70399999999959	63.9590076077677\\
5.70599999999959	63.9462183640952\\
5.70799999999959	63.9334316777598\\
5.70999999999959	63.9206475482505\\
5.71199999999959	63.9078659750552\\
5.71399999999959	63.8950869576629\\
5.71599999999959	63.8823104955648\\
5.71799999999959	63.8695365882468\\
5.71999999999959	63.8567652352002\\
5.72199999999959	63.8439964359123\\
5.72399999999959	63.8312301898746\\
5.72599999999959	63.8184664965754\\
5.72799999999959	63.8057053555051\\
5.72999999999959	63.7929467661505\\
5.73199999999959	63.7801907280059\\
5.73399999999959	63.7674372405573\\
5.73599999999959	63.7546863032965\\
5.73799999999959	63.7419379157137\\
5.73999999999959	63.7291920772989\\
5.74199999999959	63.7164487875411\\
5.74399999999959	63.703708045931\\
5.74599999999959	63.6909698519614\\
5.74799999999959	63.6782342051205\\
5.74999999999959	63.6655011048986\\
5.75199999999959	63.6527705507885\\
5.75399999999959	63.6400425422808\\
5.75599999999959	63.6273170788644\\
5.75799999999959	63.6145941600325\\
5.75999999999959	63.6018737852755\\
5.76199999999959	63.589155954085\\
5.76399999999959	63.5764406659512\\
5.76599999999959	63.5637279203673\\
5.76799999999959	63.5510177168238\\
5.76999999999959	63.5383100548131\\
5.77199999999959	63.5256049338262\\
5.77399999999959	63.5129023533558\\
5.77599999999959	63.5002023128933\\
5.77799999999959	63.4875048119315\\
5.77999999999959	63.4748098499606\\
5.78199999999959	63.4621174264767\\
5.78399999999959	63.4494275409672\\
5.78599999999958	63.4367401929286\\
5.78799999999958	63.424055381852\\
5.78999999999958	63.411373107231\\
5.79199999999958	63.3986933685563\\
5.79399999999958	63.3860161653238\\
5.79599999999958	63.3733414970243\\
5.79799999999958	63.3606693631522\\
5.79999999999958	63.3479997632011\\
5.80199999999958	63.33533269666\\
5.80399999999958	63.3226681630273\\
5.80599999999958	63.3100061617958\\
5.80799999999958	63.2973466924564\\
5.80999999999958	63.2846897545053\\
5.81199999999958	63.2720353474361\\
5.81399999999958	63.2593834707422\\
5.81599999999958	63.2467341239176\\
5.81799999999958	63.2340873064556\\
5.81999999999958	63.221443017853\\
5.82199999999958	63.2088012576002\\
5.82399999999958	63.1961620251959\\
5.82599999999958	63.1835253201318\\
5.82799999999958	63.1708911419036\\
5.82999999999958	63.1582594900063\\
5.83199999999958	63.1456303639326\\
5.83399999999958	63.1330037631799\\
5.83599999999958	63.1203796872426\\
5.83799999999958	63.1077581356147\\
5.83999999999958	63.0951391077952\\
5.84199999999958	63.0825226032729\\
5.84399999999958	63.0699086215489\\
5.84599999999958	63.0572971621163\\
5.84799999999958	63.0446882244714\\
5.84999999999958	63.0320818081099\\
5.85199999999958	63.0194779125271\\
5.85399999999958	63.0068765372192\\
5.85599999999958	62.9942776816837\\
5.85799999999958	62.9816813454144\\
5.85999999999958	62.9690875279096\\
5.86199999999958	62.9564962286628\\
5.86399999999958	62.943907447174\\
5.86599999999958	62.9313211829369\\
5.86799999999958	62.9187374354502\\
5.86999999999958	62.9061562042094\\
5.87199999999958	62.8935774887113\\
5.87399999999958	62.8810012884537\\
5.87599999999957	62.8684276029332\\
5.87799999999957	62.8558564316468\\
5.87999999999957	62.8432877740924\\
5.88199999999957	62.8307216297657\\
5.88399999999957	62.8181579981664\\
5.88599999999957	62.8055968787909\\
5.88799999999957	62.7930382711364\\
5.88999999999957	62.7804821747021\\
5.89199999999957	62.7679285889839\\
5.89399999999957	62.7553775134813\\
5.89599999999957	62.7428289476915\\
5.89799999999957	62.7302828911126\\
5.89999999999957	62.7177393432446\\
5.90199999999957	62.7051983035852\\
5.90399999999957	62.6926597716295\\
5.90599999999957	62.6801237468803\\
5.90799999999957	62.6675902288342\\
5.90999999999957	62.6550592169909\\
5.91199999999957	62.6425307108489\\
5.91399999999957	62.6300047099072\\
5.91599999999957	62.6174812136641\\
5.91799999999957	62.60496022162\\
5.91999999999957	62.5924417332733\\
5.92199999999957	62.5799257481236\\
5.92399999999957	62.5674122656702\\
5.92599999999957	62.554901285413\\
5.92799999999957	62.5423928068522\\
5.92999999999957	62.5298868294862\\
5.93199999999957	62.5173833528153\\
5.93399999999957	62.5048823763402\\
5.93599999999957	62.4923838995604\\
5.93799999999957	62.479887921977\\
5.93999999999957	62.4673944430873\\
5.94199999999957	62.4549034623944\\
5.94399999999957	62.4424149793986\\
5.94599999999957	62.4299289936011\\
5.94799999999957	62.4174455044994\\
5.94999999999957	62.404964511598\\
5.95199999999957	62.3924860143941\\
5.95399999999957	62.3800100123924\\
5.95599999999957	62.3675365050905\\
5.95799999999957	62.3550654919918\\
5.95999999999957	62.3425969725977\\
5.96199999999957	62.3301309464085\\
5.96399999999957	62.3176674129256\\
5.96599999999956	62.3052063716513\\
5.96799999999956	62.2927478220868\\
5.96999999999956	62.2802917637346\\
5.97199999999956	62.2678381960953\\
5.97399999999956	62.2553871186716\\
5.97599999999956	62.2429385309649\\
5.97799999999956	62.2304924324791\\
5.97999999999956	62.2180488227143\\
5.98199999999956	62.2056077011742\\
5.98399999999956	62.1931690673603\\
5.98599999999956	62.1807329207761\\
5.98799999999956	62.1682992609245\\
5.98999999999956	62.1558680873068\\
5.99199999999956	62.1434393994272\\
5.99399999999956	62.1310131967873\\
5.99599999999956	62.1185894788916\\
5.99799999999956	62.106168245243\\
5.99999999999956	62.0937494953435\\
6.00199999999956	62.0813332286979\\
6.00399999999956	62.0689194448088\\
6.00599999999956	62.05650814318\\
6.00799999999956	62.0440993233159\\
6.00999999999956	62.0316929847184\\
6.01199999999956	62.0192891268933\\
6.01399999999956	62.006887749343\\
6.01599999999956	61.9944888515729\\
6.01799999999956	61.9820924330868\\
6.01999999999956	61.9696984933878\\
6.02199999999956	61.9573070319813\\
6.02399999999956	61.9449180483721\\
6.02599999999956	61.9325315420633\\
6.02799999999956	61.9201475125606\\
6.02999999999956	61.9077659593689\\
6.03199999999956	61.8953868819923\\
6.03399999999956	61.8830102799368\\
6.03599999999956	61.8706361527063\\
6.03799999999956	61.8582644998057\\
6.03999999999956	61.8458953207415\\
6.04199999999956	61.8335286150187\\
6.04399999999956	61.8211643821421\\
6.04599999999956	61.808802621618\\
6.04799999999956	61.7964433329514\\
6.04999999999956	61.7840865156484\\
6.05199999999956	61.7717321692145\\
6.05399999999956	61.7593802931559\\
6.05599999999956	61.7470308869786\\
6.05799999999955	61.7346839501882\\
6.05999999999955	61.7223394822921\\
6.06199999999955	61.7099974827955\\
6.06399999999955	61.6976579512057\\
6.06599999999955	61.6853208870277\\
6.06799999999955	61.6729862897702\\
6.06999999999955	61.6606541589382\\
6.07199999999955	61.6483244940389\\
6.07399999999955	61.6359972945799\\
6.07599999999955	61.6236725600682\\
6.07799999999955	61.6113502900104\\
6.07999999999955	61.5990304839135\\
6.08199999999955	61.586713141285\\
6.08399999999955	61.5743982616328\\
6.08599999999955	61.562085844464\\
6.08799999999955	61.5497758892859\\
6.08999999999955	61.5374683956067\\
6.09199999999955	61.5251633629343\\
6.09399999999955	61.5128607907768\\
6.09599999999955	61.5005606786407\\
6.09799999999955	61.4882630260362\\
6.09999999999955	61.4759678324688\\
6.10199999999955	61.4636750974496\\
6.10399999999955	61.4513848204853\\
6.10599999999955	61.4390970010855\\
6.10799999999955	61.4268116387569\\
6.10999999999955	61.4145287330107\\
6.11199999999955	61.4022482833542\\
6.11399999999955	61.3899702892962\\
6.11599999999955	61.3776947503463\\
6.11799999999955	61.3654216660128\\
6.11999999999955	61.3531510358058\\
6.12199999999955	61.3408828592342\\
6.12399999999955	61.3286171358065\\
6.12599999999955	61.3163538650346\\
6.12799999999955	61.3040930464246\\
6.12999999999955	61.2918346794885\\
6.13199999999955	61.2795787637366\\
6.13399999999955	61.2673252986759\\
6.13599999999955	61.2550742838191\\
6.13799999999955	61.2428257186757\\
6.13999999999955	61.2305796027553\\
6.14199999999955	61.2183359355679\\
6.14399999999955	61.2060947166246\\
6.14599999999955	61.1938559454352\\
6.14799999999954	61.1816196215115\\
6.14999999999954	61.1693857443624\\
6.15199999999954	61.1571543134997\\
6.15399999999954	61.1449253284343\\
6.15599999999954	61.1326987886762\\
6.15799999999954	61.1204746937377\\
6.15999999999954	61.1082530431297\\
6.16199999999954	61.0960338363619\\
6.16399999999954	61.083817072947\\
6.16599999999954	61.0716027523967\\
6.16799999999954	61.0593908742212\\
6.16999999999954	61.047181437934\\
6.17199999999954	61.0349744430451\\
6.17399999999954	61.0227698890675\\
6.17599999999954	61.0105677755126\\
6.17799999999954	60.9983681018919\\
6.17999999999954	60.9861708677192\\
6.18199999999954	60.9739760725045\\
6.18399999999954	60.9617837157609\\
6.18599999999954	60.9495937970016\\
6.18799999999954	60.9374063157385\\
6.18999999999954	60.9252212714841\\
6.19199999999954	60.9130386637513\\
6.19399999999954	60.9008584920529\\
6.19599999999954	60.8886807559015\\
6.19799999999954	60.8765054548111\\
6.19999999999954	60.8643325882934\\
6.20199999999954	60.8521621558623\\
6.20399999999954	60.8399941570305\\
6.20599999999954	60.8278285913127\\
6.20799999999954	60.8156654582207\\
6.20999999999954	60.8035047572691\\
6.21199999999954	60.7913464879717\\
6.21399999999954	60.7791906498414\\
6.21599999999954	60.7670372423931\\
6.21799999999954	60.7548862651399\\
6.21999999999954	60.7427377175977\\
6.22199999999954	60.7305915992765\\
6.22399999999954	60.7184479096946\\
6.22599999999954	60.706306648366\\
6.22799999999954	60.6941678148021\\
6.22999999999954	60.6820314085207\\
6.23199999999954	60.6698974290349\\
6.23399999999954	60.6577658758596\\
6.23599999999954	60.6456367485096\\
6.23799999999954	60.6335100465016\\
6.23999999999953	60.6213857693467\\
6.24199999999953	60.6092639165634\\
6.24399999999953	60.5971444876655\\
6.24599999999953	60.585027482169\\
6.24799999999953	60.5729128995898\\
6.24999999999953	60.5608007394413\\
6.25199999999953	60.5486910012409\\
6.25399999999953	60.5365836845052\\
6.25599999999953	60.5244787887465\\
6.25799999999953	60.5123763134839\\
6.25999999999953	60.5002762582324\\
6.26199999999953	60.4881786225075\\
6.26399999999953	60.4760834058267\\
6.26599999999953	60.4639906077051\\
6.26799999999953	60.4519002276594\\
6.26999999999953	60.4398122652068\\
6.27199999999953	60.4277267198622\\
6.27399999999953	60.4156435911445\\
6.27599999999953	60.4035628785693\\
6.27799999999953	60.3914845816525\\
6.27999999999953	60.3794086999129\\
6.28199999999953	60.367335232866\\
6.28399999999953	60.3552641800297\\
6.28599999999953	60.3431955409217\\
6.28799999999953	60.3311293150587\\
6.28999999999953	60.3190655019581\\
6.29199999999953	60.3070041011377\\
6.29399999999953	60.2949451121154\\
6.29599999999953	60.2828885344081\\
6.29799999999953	60.270834367535\\
6.29999999999953	60.2587826110131\\
6.30199999999953	60.2467332643599\\
6.30399999999953	60.2346863270946\\
6.30599999999953	60.2226417987345\\
6.30799999999953	60.2105996787993\\
6.30999999999953	60.1985599668054\\
6.31199999999953	60.1865226622728\\
6.31399999999953	60.1744877647203\\
6.31599999999953	60.1624552736656\\
6.31799999999953	60.150425188628\\
6.31999999999953	60.1383975091257\\
6.32199999999953	60.1263722346789\\
6.32399999999953	60.1143493648058\\
6.32599999999953	60.1023288990266\\
6.32799999999953	60.0903108368589\\
6.32999999999952	60.0782951778244\\
6.33199999999952	60.0662819214391\\
6.33399999999952	60.0542710672259\\
6.33599999999952	60.0422626147033\\
6.33799999999952	60.0302565633904\\
6.33999999999952	60.0182529128076\\
6.34199999999952	60.0062516624749\\
6.34399999999952	59.9942528119124\\
6.34599999999952	59.982256360641\\
6.34799999999952	59.9702623081786\\
6.34999999999952	59.9582706540483\\
6.35199999999952	59.9462813977695\\
6.35399999999952	59.9342945388603\\
6.35599999999952	59.9223100768453\\
6.35799999999952	59.9103280112432\\
6.35999999999952	59.8983483415746\\
6.36199999999952	59.8863710673616\\
6.36399999999952	59.8743961881237\\
6.36599999999952	59.8624237033828\\
6.36799999999952	59.8504536126608\\
6.36999999999952	59.8384859154778\\
6.37199999999952	59.8265206113554\\
6.37399999999952	59.8145576998157\\
6.37599999999952	59.8025971803796\\
6.37799999999952	59.7906390525686\\
6.37999999999952	59.7786833159054\\
6.38199999999952	59.7667299699113\\
6.38399999999952	59.7547790141089\\
6.38599999999952	59.742830448019\\
6.38799999999952	59.7308842711652\\
6.38999999999952	59.7189404830679\\
6.39199999999952	59.7069990832517\\
6.39399999999952	59.6950600712368\\
6.39599999999952	59.6831234465482\\
6.39799999999952	59.671189208706\\
6.39999999999952	59.6592573572346\\
6.40199999999952	59.6473278916559\\
6.40399999999952	59.635400811494\\
6.40599999999952	59.6234761162713\\
6.40799999999952	59.6115538055097\\
6.40999999999952	59.5996338787336\\
6.41199999999952	59.5877163354669\\
6.41399999999952	59.5758011752324\\
6.41599999999952	59.5638883975528\\
6.41799999999952	59.5519780019521\\
6.41999999999951	59.5400699879545\\
6.42199999999951	59.5281643550835\\
6.42399999999951	59.5162611028627\\
6.42599999999951	59.5043602308165\\
6.42799999999951	59.4924617384688\\
6.42999999999951	59.4805656253437\\
6.43199999999951	59.4686718909658\\
6.43399999999951	59.4567805348587\\
6.43599999999951	59.4448915565476\\
6.43799999999951	59.4330049555563\\
6.43999999999951	59.4211207314109\\
6.44199999999951	59.4092388836334\\
6.44399999999951	59.3973594117507\\
6.44599999999951	59.3854823152877\\
6.44799999999951	59.3736075937692\\
6.44999999999951	59.3617352467198\\
6.45199999999951	59.3498652736649\\
6.45399999999951	59.3379976741308\\
6.45599999999951	59.3261324476406\\
6.45799999999951	59.3142695937217\\
6.45999999999951	59.3024091118995\\
6.46199999999951	59.290551001699\\
6.46399999999951	59.2786952626471\\
6.46599999999951	59.2668418942676\\
6.46799999999951	59.2549908960883\\
6.46999999999951	59.2431422676362\\
6.47199999999951	59.2312960084332\\
6.47399999999951	59.21945211801\\
6.47599999999951	59.2076105958907\\
6.47799999999951	59.1957714416023\\
6.47999999999951	59.1839346546714\\
6.48199999999951	59.172100234624\\
6.48399999999951	59.160268180988\\
6.48599999999951	59.1484384932899\\
6.48799999999951	59.1366111710555\\
6.48999999999951	59.1247862138128\\
6.49199999999951	59.1129636210884\\
6.49399999999951	59.10114339241\\
6.49599999999951	59.0893255273048\\
6.49799999999951	59.0775100252996\\
6.49999999999951	59.0656968859222\\
6.50199999999951	59.0538861087004\\
6.50399999999951	59.0420776931619\\
6.50599999999951	59.030271638834\\
6.50799999999951	59.0184679452448\\
6.50999999999951	59.0066666119229\\
6.5119999999995	58.9948676383948\\
6.5139999999995	58.9830710241899\\
6.5159999999995	58.9712767688361\\
6.5179999999995	58.959484871862\\
6.5199999999995	58.9476953327955\\
6.5219999999995	58.935908151165\\
6.5239999999995	58.9241233264998\\
6.5259999999995	58.9123408583283\\
6.5279999999995	58.90056074618\\
6.5299999999995	58.8887829895809\\
6.5319999999995	58.8770075880635\\
6.5339999999995	58.8652345411553\\
6.5359999999995	58.8534638483857\\
6.5379999999995	58.8416955092835\\
6.5399999999995	58.8299295233788\\
6.5419999999995	58.8181658902007\\
6.5439999999995	58.8064046092792\\
6.5459999999995	58.7946456801429\\
6.5479999999995	58.7828891023224\\
6.5499999999995	58.7711348753482\\
6.5519999999995	58.759382998748\\
6.5539999999995	58.7476334720533\\
6.5559999999995	58.7358862947945\\
6.5579999999995	58.7241414665013\\
6.5599999999995	58.7123989867043\\
6.5619999999995	58.7006588549331\\
6.5639999999995	58.6889210707186\\
6.5659999999995	58.6771856335919\\
6.5679999999995	58.6654525430833\\
6.5699999999995	58.6537217987235\\
6.5719999999995	58.6419934000441\\
6.5739999999995	58.6302673465741\\
6.5759999999995	58.6185436378464\\
6.5779999999995	58.6068222733919\\
6.5799999999995	58.595103252742\\
6.5819999999995	58.5833865754268\\
6.5839999999995	58.5716722409781\\
6.5859999999995	58.5599602489285\\
6.5879999999995	58.548250598809\\
6.5899999999995	58.5365432901502\\
6.5919999999995	58.524838322486\\
6.5939999999995	58.5131356953472\\
6.5959999999995	58.5014354082654\\
6.5979999999995	58.4897374607736\\
6.5999999999995	58.4780418524028\\
6.60199999999949	58.4663485826863\\
6.60399999999949	58.4546576511558\\
6.60599999999949	58.4429690573443\\
6.60799999999949	58.4312828007844\\
6.60999999999949	58.4195988810083\\
6.61199999999949	58.4079172975482\\
6.61399999999949	58.3962380499379\\
6.61599999999949	58.3845611377113\\
6.61799999999949	58.3728865603991\\
6.61999999999949	58.3612143175354\\
6.62199999999949	58.3495444086539\\
6.62399999999949	58.3378768332878\\
6.62599999999949	58.326211590969\\
6.62799999999949	58.3145486812326\\
6.62999999999949	58.3028881036126\\
6.63199999999949	58.2912298576399\\
6.63399999999949	58.2795739428516\\
6.63599999999949	58.2679203587798\\
6.63799999999949	58.2562691049602\\
6.63999999999949	58.2446201809229\\
6.64199999999949	58.2329735862061\\
6.64399999999949	58.2213293203418\\
6.64599999999949	58.2096873828653\\
6.64799999999949	58.1980477733108\\
6.64999999999949	58.186410491212\\
6.65199999999949	58.1747755361049\\
6.65399999999949	58.1631429075235\\
6.65599999999949	58.1515126050023\\
6.65799999999949	58.1398846280763\\
6.65999999999949	58.1282589762819\\
6.66199999999949	58.1166356491518\\
6.66399999999949	58.1050146462232\\
6.66599999999949	58.0933959670288\\
6.66799999999949	58.081779611107\\
6.66999999999949	58.0701655779923\\
6.67199999999949	58.0585538672179\\
6.67399999999949	58.0469444783218\\
6.67599999999949	58.0353374108408\\
6.67799999999949	58.0237326643068\\
6.67999999999949	58.0121302382591\\
6.68199999999949	58.000530132233\\
6.68399999999949	57.9889323457639\\
6.68599999999949	57.9773368783885\\
6.68799999999949	57.9657437296432\\
6.68999999999949	57.9541528990621\\
6.69199999999949	57.942564386185\\
6.69399999999948	57.9309781905469\\
6.69599999999948	57.9193943116849\\
6.69799999999948	57.9078127491348\\
6.69999999999948	57.8962335024349\\
6.70199999999948	57.8846565711202\\
6.70399999999948	57.8730819547295\\
6.70599999999948	57.8615096527995\\
6.70799999999948	57.8499396648655\\
6.70999999999948	57.8383719904674\\
6.71199999999948	57.8268066291421\\
6.71399999999948	57.8152435804256\\
6.71599999999948	57.8036828438575\\
6.71799999999948	57.7921244189729\\
6.71999999999948	57.7805683053122\\
6.72199999999948	57.7690145024121\\
6.72399999999948	57.7574630098097\\
6.72599999999948	57.7459138270443\\
6.72799999999948	57.7343669536536\\
6.72999999999948	57.7228223891759\\
6.73199999999948	57.7112801331486\\
6.73399999999948	57.6997401851125\\
6.73599999999948	57.6882025446034\\
6.73799999999948	57.6766672111608\\
6.73999999999948	57.6651341843241\\
6.74199999999948	57.653603463631\\
6.74399999999948	57.6420750486213\\
6.74599999999948	57.6305489388337\\
6.74799999999948	57.6190251338073\\
6.74999999999948	57.6075036330806\\
6.75199999999948	57.5959844361931\\
6.75399999999948	57.5844675426853\\
6.75599999999948	57.5729529520959\\
6.75799999999948	57.5614406639615\\
6.75999999999948	57.5499306778263\\
6.76199999999948	57.5384229932272\\
6.76399999999948	57.5269176097054\\
6.76599999999948	57.5154145268\\
6.76799999999948	57.5039137440514\\
6.76999999999948	57.492415260999\\
6.77199999999948	57.4809190771831\\
6.77399999999948	57.469425192145\\
6.77599999999948	57.4579336054241\\
6.77799999999948	57.4464443165603\\
6.77999999999948	57.4349573250956\\
6.78199999999948	57.4234726305697\\
6.78399999999947	57.4119902325232\\
6.78599999999947	57.4005101304972\\
6.78799999999947	57.3890323240322\\
6.78999999999947	57.3775568126693\\
6.79199999999947	57.36608359595\\
6.79399999999947	57.354612673416\\
6.79599999999947	57.3431440446068\\
6.79799999999947	57.3316777090647\\
6.79999999999947	57.3202136663316\\
6.80199999999947	57.3087519159488\\
6.80399999999947	57.2972924574566\\
6.80599999999947	57.2858352903991\\
6.80799999999947	57.2743804143161\\
6.80999999999947	57.2629278287499\\
6.81199999999947	57.2514775332442\\
6.81399999999947	57.2400295273381\\
6.81599999999947	57.2285838105758\\
6.81799999999947	57.2171403824993\\
6.81999999999947	57.2056992426528\\
6.82199999999947	57.1942603905732\\
6.82399999999947	57.1828238258076\\
6.82599999999947	57.1713895478979\\
6.82799999999947	57.1599575563872\\
6.82999999999947	57.148527850817\\
6.83199999999947	57.1371004307305\\
6.83399999999947	57.1256752956718\\
6.83599999999947	57.1142524451823\\
6.83799999999947	57.1028318788077\\
6.83999999999947	57.0914135960888\\
6.84199999999947	57.0799975965683\\
6.84399999999947	57.0685838797924\\
6.84599999999947	57.0571724453031\\
6.84799999999947	57.0457632926448\\
6.84999999999947	57.0343564213606\\
6.85199999999947	57.022951830994\\
6.85399999999947	57.0115495210902\\
6.85599999999947	57.0001494911918\\
6.85799999999947	56.9887517408436\\
6.85999999999947	56.9773562695893\\
6.86199999999947	56.9659630769739\\
6.86399999999947	56.9545721625423\\
6.86599999999947	56.9431835258367\\
6.86799999999947	56.9317971664032\\
6.86999999999947	56.9204130837862\\
6.87199999999947	56.9090312775313\\
6.87399999999946	56.8976517471814\\
6.87599999999946	56.886274492283\\
6.87799999999946	56.8748995123805\\
6.87999999999946	56.8635268070197\\
6.88199999999946	56.8521563757439\\
6.88399999999946	56.8407882181005\\
6.88599999999946	56.8294223336343\\
6.88799999999946	56.8180587218892\\
6.88999999999946	56.8066973824127\\
6.89199999999946	56.7953383147495\\
6.89399999999946	56.7839815184461\\
6.89599999999946	56.7726269930472\\
6.89799999999946	56.7612747381002\\
6.89999999999946	56.7499247531492\\
6.90199999999946	56.7385770377419\\
6.90399999999946	56.7272315914238\\
6.90599999999946	56.7158884137416\\
6.90799999999946	56.7045475042404\\
6.90999999999946	56.6932088624673\\
6.91199999999946	56.6818724879695\\
6.91399999999946	56.6705383802937\\
6.91599999999946	56.6592065389862\\
6.91799999999946	56.6478769635934\\
6.91999999999946	56.6365496536623\\
6.92199999999946	56.6252246087407\\
6.92399999999946	56.6139018283752\\
6.92599999999946	56.6025813121129\\
6.92799999999946	56.5912630595007\\
6.92999999999946	56.5799470700867\\
6.93199999999946	56.568633343418\\
6.93399999999946	56.5573218790423\\
6.93599999999946	56.5460126765077\\
6.93799999999946	56.53470573536\\
6.93999999999946	56.5234010551486\\
6.94199999999946	56.5120986354215\\
6.94399999999946	56.5007984757261\\
6.94599999999946	56.4895005756124\\
6.94799999999946	56.4782049346245\\
6.94999999999946	56.466911552314\\
6.95199999999946	56.4556204282284\\
6.95399999999946	56.4443315619165\\
6.95599999999946	56.4330449529256\\
6.95799999999946	56.4217606008055\\
6.95999999999946	56.4104785051042\\
6.96199999999946	56.3991986653712\\
6.96399999999946	56.3879210811551\\
6.96599999999945	56.3766457520046\\
6.96799999999945	56.3653726774694\\
6.96999999999945	56.3541018570974\\
6.97199999999945	56.3428332904397\\
6.97399999999945	56.3315669770441\\
6.97599999999945	56.3203029164605\\
6.97799999999945	56.3090411082395\\
6.97999999999945	56.2977815519286\\
6.98199999999945	56.2865242470795\\
6.98399999999945	56.275269193241\\
6.98599999999945	56.2640163899628\\
6.98799999999945	56.2527658367954\\
6.98999999999945	56.2415175332886\\
6.99199999999945	56.2302714789927\\
6.99399999999945	56.2190276734581\\
6.99599999999945	56.2077861162352\\
6.99799999999945	56.1965468068737\\
6.99999999999945	56.1853097449249\\
7.00199999999945	56.1740749299386\\
7.00399999999945	56.1628423614667\\
7.00599999999945	56.151612039059\\
7.00799999999945	56.1403839622662\\
7.00999999999945	56.1291581306406\\
7.01199999999945	56.1179345437311\\
7.01399999999945	56.1067132010911\\
7.01599999999945	56.0954941022704\\
7.01799999999945	56.0842772468217\\
7.01999999999945	56.0730626342942\\
7.02199999999945	56.0618502642415\\
7.02399999999945	56.050640136214\\
7.02599999999945	56.0394322497641\\
7.02799999999945	56.0282266044436\\
7.02999999999945	56.0170231998036\\
7.03199999999945	56.0058220353966\\
7.03399999999945	55.9946231107738\\
7.03599999999945	55.9834264254893\\
7.03799999999945	55.9722319790937\\
7.03999999999945	55.961039771139\\
7.04199999999945	55.9498498011787\\
7.04399999999945	55.9386620687656\\
7.04599999999945	55.9274765734507\\
7.04799999999945	55.9162933147879\\
7.04999999999945	55.9051122923289\\
7.05199999999945	55.8939335056278\\
7.05399999999945	55.8827569542368\\
7.05599999999944	55.8715826377096\\
7.05799999999944	55.8604105555981\\
7.05999999999944	55.8492407074572\\
7.06199999999944	55.838073092839\\
7.06399999999944	55.8269077112966\\
7.06599999999944	55.8157445623844\\
7.06799999999944	55.8045836456545\\
7.06999999999944	55.7934249606625\\
7.07199999999944	55.7822685069608\\
7.07399999999944	55.7711142841042\\
7.07599999999944	55.7599622916455\\
7.07799999999944	55.7488125291402\\
7.07999999999944	55.7376649961407\\
7.08199999999944	55.7265196922021\\
7.08399999999944	55.7153766168793\\
7.08599999999944	55.7042357697249\\
7.08799999999944	55.6930971502948\\
7.08999999999944	55.6819607581433\\
7.09199999999944	55.6708265928249\\
7.09399999999944	55.6596946538938\\
7.09599999999944	55.648564940906\\
7.09799999999944	55.6374374534153\\
7.09999999999944	55.6263121909771\\
7.10199999999944	55.6151891531469\\
7.10399999999944	55.6040683394784\\
7.10599999999944	55.5929497495284\\
7.10799999999944	55.5818333828516\\
7.10999999999944	55.5707192390045\\
7.11199999999944	55.5596073175409\\
7.11399999999944	55.5484976180166\\
7.11599999999944	55.5373901399893\\
7.11799999999944	55.5262848830125\\
7.11999999999944	55.5151818466436\\
7.12199999999944	55.5040810304371\\
7.12399999999944	55.4929824339501\\
7.12599999999944	55.4818860567391\\
7.12799999999944	55.4707918983595\\
7.12999999999944	55.4596999583672\\
7.13199999999944	55.4486102363202\\
7.13399999999944	55.4375227317749\\
7.13599999999944	55.4264374442846\\
7.13799999999944	55.4153543734102\\
7.13999999999944	55.4042735187064\\
7.14199999999944	55.3931948797319\\
7.14399999999944	55.3821184560397\\
7.14599999999944	55.3710442471897\\
7.14799999999943	55.3599722527391\\
7.14999999999943	55.3489024722452\\
7.15199999999943	55.3378349052643\\
7.15399999999943	55.3267695513538\\
7.15599999999943	55.3157064100716\\
7.15799999999943	55.3046454809759\\
7.15999999999943	55.2935867636228\\
7.16199999999943	55.2825302575712\\
7.16399999999943	55.2714759623788\\
7.16599999999943	55.2604238776035\\
7.16799999999943	55.2493740028031\\
7.16999999999943	55.2383263375356\\
7.17199999999943	55.2272808813589\\
7.17399999999943	55.2162376338325\\
7.17599999999943	55.2051965945139\\
7.17799999999943	55.1941577629606\\
7.17999999999943	55.1831211387328\\
7.18199999999943	55.1720867213886\\
7.18399999999943	55.1610545104865\\
7.18599999999943	55.1500245055853\\
7.18799999999943	55.1389967062441\\
7.18999999999943	55.1279711120218\\
7.19199999999943	55.1169477224771\\
7.19399999999943	55.1059265371705\\
7.19599999999943	55.0949075556592\\
7.19799999999943	55.083890777503\\
7.19999999999943	55.0728762022627\\
7.20199999999943	55.0618638294965\\
7.20399999999943	55.0508536587655\\
7.20599999999943	55.0398456896271\\
7.20799999999943	55.0288399216425\\
7.20999999999943	55.0178363543719\\
7.21199999999943	55.0068349873743\\
7.21399999999943	54.9958358202105\\
7.21599999999943	54.9848388524396\\
7.21799999999943	54.9738440836229\\
7.21999999999943	54.9628515133206\\
7.22199999999943	54.9518611410925\\
7.22399999999943	54.9408729664989\\
7.22599999999943	54.9298869891012\\
7.22799999999943	54.9189032084593\\
7.22999999999943	54.9079216241347\\
7.23199999999943	54.8969422356877\\
7.23399999999943	54.8859650426791\\
7.23599999999943	54.8749900446701\\
7.23799999999942	54.8640172412216\\
7.23999999999942	54.8530466318956\\
7.24199999999942	54.8420782162521\\
7.24399999999942	54.8311119938538\\
7.24599999999942	54.8201479642606\\
7.24799999999942	54.8091861270354\\
7.24999999999942	54.7982264817389\\
7.25199999999942	54.7872690279335\\
7.25399999999942	54.7763137651801\\
7.25599999999942	54.7653606930416\\
7.25799999999942	54.7544098110793\\
7.25999999999942	54.7434611188552\\
7.26199999999942	54.7325146159322\\
7.26399999999942	54.7215703018719\\
7.26599999999942	54.7106281762372\\
7.26799999999942	54.6996882385891\\
7.26999999999942	54.6887504884914\\
7.27199999999942	54.6778149255065\\
7.27399999999942	54.6668815491964\\
7.27599999999942	54.6559503591248\\
7.27799999999942	54.6450213548537\\
7.27999999999942	54.6340945359468\\
7.28199999999942	54.6231699019658\\
7.28399999999942	54.6122474524758\\
7.28599999999942	54.6013271870387\\
7.28799999999942	54.5904091052171\\
7.28999999999942	54.5794932065758\\
7.29199999999942	54.5685794906778\\
7.29399999999942	54.5576679570869\\
7.29599999999942	54.546758605365\\
7.29799999999942	54.5358514350782\\
7.29999999999942	54.5249464457892\\
7.30199999999942	54.5140436370618\\
7.30399999999942	54.5031430084598\\
7.30599999999942	54.4922445595481\\
7.30799999999942	54.4813482898901\\
7.30999999999942	54.4704541990502\\
7.31199999999942	54.4595622865935\\
7.31399999999942	54.4486725520823\\
7.31599999999942	54.4377849950832\\
7.31799999999942	54.4268996151606\\
7.31999999999942	54.4160164118777\\
7.32199999999942	54.405135384801\\
7.32399999999942	54.3942565334949\\
7.32599999999942	54.3833798575229\\
7.32799999999941	54.3725053564517\\
7.32999999999941	54.3616330298458\\
7.33199999999941	54.35076287727\\
7.33399999999941	54.3398948982912\\
7.33599999999941	54.3290290924721\\
7.33799999999941	54.3181654593799\\
7.33999999999941	54.3073039985802\\
7.34199999999941	54.2964447096387\\
7.34399999999941	54.2855875921202\\
7.34599999999941	54.2747326455907\\
7.34799999999941	54.2638798696169\\
7.34999999999941	54.253029263765\\
7.35199999999941	54.2421808275987\\
7.35399999999941	54.2313345606867\\
7.35599999999941	54.2204904625946\\
7.35799999999941	54.2096485328876\\
7.35999999999941	54.1988087711338\\
7.36199999999941	54.1879711768989\\
7.36399999999941	54.1771357497485\\
7.36599999999941	54.16630248925\\
7.36799999999941	54.1554713949714\\
7.36999999999941	54.1446424664778\\
7.37199999999941	54.1338157033373\\
7.37399999999941	54.122991105116\\
7.37599999999941	54.1121686713819\\
7.37799999999941	54.1013484017016\\
7.37999999999941	54.0905302956423\\
7.38199999999941	54.0797143527728\\
7.38399999999941	54.0689005726571\\
7.38599999999941	54.0580889548667\\
7.38799999999941	54.047279498967\\
7.38999999999941	54.0364722045256\\
7.39199999999941	54.0256670711112\\
7.39399999999941	54.0148640982916\\
7.39599999999941	54.0040632856344\\
7.39799999999941	53.9932646327082\\
7.39999999999941	53.9824681390804\\
7.40199999999941	53.9716738043199\\
7.40399999999941	53.9608816279935\\
7.40599999999941	53.9500916096718\\
7.40799999999941	53.9393037489229\\
7.40999999999941	53.9285180453131\\
7.41199999999941	53.9177344984133\\
7.41399999999941	53.9069531077921\\
7.41599999999941	53.8961738730184\\
7.41799999999941	53.8853967936585\\
7.4199999999994	53.8746218692851\\
7.4219999999994	53.8638490994649\\
7.4239999999994	53.8530784837678\\
7.4259999999994	53.8423100217636\\
7.4279999999994	53.8315437130211\\
7.4299999999994	53.8207795571109\\
7.4319999999994	53.8100175535988\\
7.4339999999994	53.7992577020587\\
7.4359999999994	53.7885000020588\\
7.4379999999994	53.7777444531679\\
7.4399999999994	53.7669910549561\\
7.4419999999994	53.7562398069947\\
7.4439999999994	53.7454907088538\\
7.4459999999994	53.7347437601013\\
7.4479999999994	53.7239989603095\\
7.4499999999994	53.7132563090472\\
7.4519999999994	53.7025158058859\\
7.4539999999994	53.691777450396\\
7.4559999999994	53.6810412421475\\
7.4579999999994	53.6703071807113\\
7.4599999999994	53.6595752656585\\
7.4619999999994	53.6488454965588\\
7.4639999999994	53.6381178729847\\
7.4659999999994	53.6273923945056\\
7.4679999999994	53.6166690606936\\
7.4699999999994	53.6059478711193\\
7.4719999999994	53.5952288253541\\
7.4739999999994	53.5845119229698\\
7.4759999999994	53.5737971635374\\
7.4779999999994	53.5630845466279\\
7.4799999999994	53.5523740718134\\
7.4819999999994	53.5416657386651\\
7.4839999999994	53.5309595467563\\
7.4859999999994	53.5202554956569\\
7.4879999999994	53.5095535849407\\
7.4899999999994	53.4988538141778\\
7.4919999999994	53.4881561829412\\
7.4939999999994	53.4774606908024\\
7.4959999999994	53.4667673373358\\
7.4979999999994	53.4560761221102\\
7.4999999999994	53.4453870447015\\
7.5019999999994	53.4347001046804\\
7.5039999999994	53.4240153016207\\
7.5059999999994	53.4133326350931\\
7.5079999999994	53.4026521046726\\
7.50999999999939	53.3919737099301\\
7.51199999999939	53.3812974504404\\
7.51399999999939	53.3706233257748\\
7.51599999999939	53.3599513355084\\
7.51799999999939	53.3492814792119\\
7.51999999999939	53.3386137564613\\
7.52199999999939	53.3279481668276\\
7.52399999999939	53.3172847098852\\
7.52599999999939	53.3066233852089\\
7.52799999999939	53.2959641923699\\
7.52999999999939	53.2853071309435\\
7.53199999999939	53.2746522005037\\
7.53399999999939	53.2639994006234\\
7.53599999999939	53.2533487308776\\
7.53799999999939	53.2427001908392\\
7.53999999999939	53.2320537800834\\
7.54199999999939	53.2214094981837\\
7.54399999999939	53.2107673447146\\
7.54599999999939	53.2001273192511\\
7.54799999999939	53.1894894213663\\
7.54999999999939	53.1788536506364\\
7.55199999999939	53.168220006635\\
7.55399999999939	53.1575884889375\\
7.55599999999939	53.1469590971178\\
7.55799999999939	53.1363318307514\\
7.55999999999939	53.1257066894138\\
7.56199999999939	53.1150836726791\\
7.56399999999939	53.104462780123\\
7.56599999999939	53.0938440113212\\
7.56799999999939	53.0832273658481\\
7.56999999999939	53.0726128432793\\
7.57199999999939	53.0620004431907\\
7.57399999999939	53.0513901651579\\
7.57599999999939	53.0407820087557\\
7.57799999999939	53.030175973561\\
7.57999999999939	53.0195720591496\\
7.58199999999939	53.0089702650963\\
7.58399999999939	52.9983705909777\\
7.58599999999939	52.9877730363706\\
7.58799999999939	52.9771776008508\\
7.58999999999939	52.9665842839936\\
7.59199999999939	52.9559930853767\\
7.59399999999939	52.9454040045756\\
7.59599999999939	52.9348170411674\\
7.59799999999939	52.9242321947286\\
7.59999999999939	52.9136494648354\\
7.60199999999938	52.9030688510652\\
7.60399999999938	52.8924903529945\\
7.60599999999938	52.8819139702006\\
7.60799999999938	52.8713397022603\\
7.60999999999938	52.8607675487507\\
7.61199999999938	52.8501975092485\\
7.61399999999938	52.8396295833329\\
7.61599999999938	52.8290637705779\\
7.61799999999938	52.8185000705641\\
7.61999999999938	52.8079384828675\\
7.62199999999938	52.7973790070665\\
7.62399999999938	52.7868216427371\\
7.62599999999938	52.7762663894594\\
7.62799999999938	52.7657132468099\\
7.62999999999938	52.7551622143672\\
7.63199999999938	52.7446132917087\\
7.63399999999938	52.7340664784148\\
7.63599999999938	52.7235217740583\\
7.63799999999938	52.7129791782234\\
7.63999999999938	52.7024386904846\\
7.64199999999938	52.6919003104231\\
7.64399999999938	52.6813640376149\\
7.64599999999938	52.6708298716405\\
7.64799999999938	52.6602978120788\\
7.64999999999938	52.6497678585064\\
7.65199999999938	52.6392400105046\\
7.65399999999938	52.628714267651\\
7.65599999999938	52.6181906295251\\
7.65799999999938	52.6076690957061\\
7.65999999999938	52.5971496657729\\
7.66199999999938	52.586632339305\\
7.66399999999938	52.5761171158816\\
7.66599999999938	52.5656039950825\\
7.66799999999938	52.5550929764876\\
7.66999999999938	52.5445840596756\\
7.67199999999938	52.5340772442264\\
7.67399999999938	52.5235725297205\\
7.67599999999938	52.5130699157373\\
7.67799999999938	52.5025694018577\\
7.67999999999938	52.4920709876599\\
7.68199999999938	52.481574672725\\
7.68399999999938	52.4710804566339\\
7.68599999999938	52.4605883389661\\
7.68799999999938	52.4500983193025\\
7.68999999999938	52.4396103972224\\
7.69199999999937	52.429124572309\\
7.69399999999937	52.4186408441396\\
7.69599999999937	52.4081592122967\\
7.69799999999937	52.3976796763615\\
7.69999999999937	52.3872022359142\\
7.70199999999937	52.3767268905365\\
7.70399999999937	52.3662536398084\\
7.70599999999937	52.3557824833115\\
7.70799999999937	52.3453134206274\\
7.70999999999937	52.3348464513373\\
7.71199999999937	52.3243815750223\\
7.71399999999937	52.3139187912648\\
7.71599999999937	52.3034580996443\\
7.71799999999937	52.2929994997446\\
7.71999999999937	52.2825429911459\\
7.72199999999937	52.2720885734309\\
7.72399999999937	52.2616362461821\\
7.72599999999937	52.2511860089801\\
7.72799999999937	52.2407378614077\\
7.72999999999937	52.2302918030474\\
7.73199999999937	52.2198478334806\\
7.73399999999937	52.2094059522902\\
7.73599999999937	52.1989661590581\\
7.73799999999937	52.1885284533674\\
7.73999999999937	52.1780928348007\\
7.74199999999937	52.1676593029413\\
7.74399999999937	52.1572278573686\\
7.74599999999937	52.1467984976689\\
7.74799999999937	52.136371223424\\
7.74999999999937	52.1259460342175\\
7.75199999999937	52.1155229296313\\
7.75399999999937	52.1051019092497\\
7.75599999999937	52.0946829726551\\
7.75799999999937	52.0842661194311\\
7.75999999999937	52.0738513491614\\
7.76199999999937	52.0634386614289\\
7.76399999999937	52.053028055818\\
7.76599999999937	52.0426195319119\\
7.76799999999937	52.032213089294\\
7.76999999999937	52.0218087275486\\
7.77199999999937	52.0114064462601\\
7.77399999999937	52.0010062450099\\
7.77599999999937	51.9906081233858\\
7.77799999999937	51.980212080969\\
7.77999999999937	51.9698181173465\\
7.78199999999936	51.9594262320997\\
7.78399999999936	51.9490364248146\\
7.78599999999936	51.9386486950751\\
7.78799999999936	51.9282630424667\\
7.78999999999936	51.9178794665744\\
7.79199999999936	51.9074979669803\\
7.79399999999936	51.8971185432712\\
7.79599999999936	51.8867411950325\\
7.79799999999936	51.8763659218481\\
7.79999999999936	51.8659927233031\\
7.80199999999936	51.8556215989837\\
7.80399999999936	51.8452525484741\\
7.80599999999936	51.8348855713601\\
7.80799999999936	51.8245206672262\\
7.80999999999936	51.814157835659\\
7.81199999999936	51.8037970762439\\
7.81399999999936	51.7934383885662\\
7.81599999999936	51.7830817722116\\
7.81799999999936	51.7727272267667\\
7.81999999999936	51.7623747518158\\
7.82199999999936	51.7520243469466\\
7.82399999999936	51.7416760117447\\
7.82599999999936	51.7313297457956\\
7.82799999999936	51.7209855486852\\
7.82999999999936	51.7106434200013\\
7.83199999999936	51.7003033593297\\
7.83399999999936	51.6899653662559\\
7.83599999999936	51.6796294403684\\
7.83799999999936	51.6692955812517\\
7.83999999999936	51.658963788494\\
7.84199999999936	51.6486340616821\\
7.84399999999936	51.6383064004019\\
7.84599999999936	51.6279808042409\\
7.84799999999936	51.6176572727862\\
7.84999999999936	51.6073358056251\\
7.85199999999936	51.5970164023452\\
7.85399999999936	51.5866990625321\\
7.85599999999936	51.5763837857748\\
7.85799999999936	51.5660705716606\\
7.85999999999936	51.5557594197767\\
7.86199999999936	51.5454503297111\\
7.86399999999936	51.5351433010506\\
7.86599999999936	51.5248383333841\\
7.86799999999936	51.5145354262985\\
7.86999999999936	51.5042345793833\\
7.87199999999936	51.4939357922246\\
7.87399999999935	51.4836390644115\\
7.87599999999935	51.4733443955325\\
7.87799999999935	51.4630517851755\\
7.87999999999935	51.4527612329289\\
7.88199999999935	51.4424727383812\\
7.88399999999935	51.4321863011213\\
7.88599999999935	51.4219019207366\\
7.88799999999935	51.4116195968168\\
7.88999999999935	51.4013393289515\\
7.89199999999935	51.3910611167283\\
7.89399999999935	51.3807849597361\\
7.89599999999935	51.3705108575646\\
7.89799999999935	51.3602388098031\\
7.89999999999935	51.3499688160397\\
7.90199999999935	51.3397008758657\\
7.90399999999935	51.329434988867\\
7.90599999999935	51.3191711546357\\
7.90799999999935	51.3089093727613\\
7.90999999999935	51.2986496428326\\
7.91199999999935	51.2883919644395\\
7.91399999999935	51.2781363371725\\
7.91599999999935	51.2678827606203\\
7.91799999999935	51.2576312343735\\
7.91999999999935	51.2473817580215\\
7.92199999999935	51.2371343311556\\
7.92399999999935	51.2268889533647\\
7.92599999999935	51.2166456242397\\
7.92799999999935	51.206404343371\\
7.92999999999935	51.1961651103498\\
7.93199999999935	51.1859279247644\\
7.93399999999935	51.1756927862069\\
7.93599999999935	51.1654596942684\\
7.93799999999935	51.1552286485383\\
7.93999999999935	51.1449996486089\\
7.94199999999935	51.1347726940702\\
7.94399999999935	51.1245477845129\\
7.94599999999935	51.1143249195289\\
7.94799999999935	51.1041040987098\\
7.94999999999935	51.093885321645\\
7.95199999999935	51.0836685879277\\
7.95399999999935	51.073453897148\\
7.95599999999935	51.063241248898\\
7.95799999999935	51.0530306427698\\
7.95999999999935	51.0428220783541\\
7.96199999999935	51.0326155552433\\
7.96399999999934	51.0224110730288\\
7.96599999999934	51.012208631302\\
7.96799999999934	51.0020082296571\\
7.96999999999934	50.9918098676828\\
7.97199999999934	50.9816135449739\\
7.97399999999934	50.9714192611218\\
7.97599999999934	50.9612270157184\\
7.97799999999934	50.9510368083572\\
7.97999999999934	50.9408486386288\\
7.98199999999934	50.9306625061276\\
7.98399999999934	50.9204784104456\\
7.98599999999934	50.9102963511757\\
7.98799999999934	50.9001163279099\\
7.98999999999934	50.8899383402421\\
7.99199999999934	50.8797623877647\\
7.99399999999934	50.8695884700709\\
7.99599999999934	50.8594165867529\\
7.99799999999934	50.849246737405\\
7.99999999999934	50.8390789216209\\
};
\addplot [color=mycolor1, forget plot]
  table[row sep=crcr]{%
7.99999999999934	50.8390789216209\\
8.00199999999934	50.828913138993\\
8.00399999999934	50.8187493891155\\
8.00599999999934	50.8085876715811\\
8.00799999999934	50.7984279859837\\
8.00999999999934	50.7882703319174\\
8.01199999999934	50.7781147089759\\
8.01399999999935	50.7679611167524\\
8.01599999999935	50.7578095548418\\
8.01799999999935	50.7476600228369\\
8.01999999999935	50.7375125203326\\
8.02199999999935	50.7273670469234\\
8.02399999999935	50.7172236022035\\
8.02599999999935	50.7070821857657\\
8.02799999999935	50.6969427972064\\
8.02999999999935	50.686805436119\\
8.03199999999935	50.6766701020988\\
8.03399999999935	50.6665367947395\\
8.03599999999935	50.656405513637\\
8.03799999999935	50.6462762583852\\
8.03999999999935	50.6361490285796\\
8.04199999999935	50.6260238238149\\
8.04399999999936	50.6159006436858\\
8.04599999999936	50.6057794877886\\
8.04799999999936	50.5956603557172\\
8.04999999999936	50.585543247068\\
8.05199999999936	50.5754281614357\\
8.05399999999936	50.565315098416\\
8.05599999999936	50.5552040576045\\
8.05799999999936	50.5450950385969\\
8.05999999999936	50.5349880409891\\
8.06199999999936	50.5248830643756\\
8.06399999999936	50.514780108354\\
8.06599999999936	50.5046791725194\\
8.06799999999936	50.4945802564685\\
8.06999999999936	50.4844833597959\\
8.07199999999937	50.4743884820999\\
8.07399999999937	50.4642956229755\\
8.07599999999937	50.4542047820194\\
8.07799999999937	50.4441159588271\\
8.07999999999937	50.4340291529963\\
8.08199999999937	50.4239443641236\\
8.08399999999937	50.4138615918056\\
8.08599999999937	50.4037808356381\\
8.08799999999937	50.3937020952192\\
8.08999999999937	50.3836253701448\\
8.09199999999937	50.3735506600125\\
8.09399999999937	50.3634779644205\\
8.09599999999937	50.3534072829633\\
8.09799999999937	50.3433386152409\\
8.09999999999937	50.3332719608485\\
8.10199999999938	50.3232073193845\\
8.10399999999938	50.3131446904469\\
8.10599999999938	50.3030840736316\\
8.10799999999938	50.2930254685378\\
8.10999999999938	50.282968874763\\
8.11199999999938	50.2729142919046\\
8.11399999999938	50.2628617195611\\
8.11599999999938	50.2528111573293\\
8.11799999999938	50.2427626048078\\
8.11999999999938	50.2327160615957\\
8.12199999999938	50.2226715272905\\
8.12399999999938	50.2126290014904\\
8.12599999999938	50.2025884837936\\
8.12799999999938	50.1925499737985\\
8.12999999999938	50.1825134711047\\
8.13199999999939	50.1724789753093\\
8.13399999999939	50.1624464860124\\
8.13599999999939	50.1524160028116\\
8.13799999999939	50.1423875253069\\
8.13999999999939	50.1323610530962\\
8.14199999999939	50.1223365857793\\
8.14399999999939	50.1123141229545\\
8.14599999999939	50.1022936642213\\
8.14799999999939	50.0922752091801\\
8.14999999999939	50.0822587574281\\
8.15199999999939	50.0722443085661\\
8.15399999999939	50.0622318621937\\
8.15599999999939	50.0522214179102\\
8.15799999999939	50.0422129753153\\
8.15999999999939	50.0322065340089\\
8.1619999999994	50.0222020935904\\
8.1639999999994	50.0121996536597\\
8.1659999999994	50.0021992138167\\
8.1679999999994	49.9922007736618\\
8.1699999999994	49.9822043327951\\
8.1719999999994	49.9722098908165\\
8.1739999999994	49.9622174473274\\
8.1759999999994	49.9522270019268\\
8.1779999999994	49.9422385542167\\
8.1799999999994	49.9322521037953\\
8.1819999999994	49.9222676502652\\
8.1839999999994	49.9122851932266\\
8.1859999999994	49.9023047322804\\
8.1879999999994	49.8923262670268\\
8.1899999999994	49.8823497970672\\
8.19199999999941	49.8723753220033\\
8.19399999999941	49.8624028414346\\
8.19599999999941	49.852432354964\\
8.19799999999941	49.8424638621913\\
8.19999999999941	49.8324973627185\\
8.20199999999941	49.8225328561477\\
8.20399999999941	49.8125703420789\\
8.20599999999941	49.8026098201149\\
8.20799999999941	49.7926512898572\\
8.20999999999941	49.7826947509074\\
8.21199999999941	49.7727402028662\\
8.21399999999941	49.7627876453372\\
8.21599999999941	49.752837077922\\
8.21799999999941	49.7428885002221\\
8.21999999999941	49.7329419118393\\
8.22199999999942	49.7229973123766\\
8.22399999999942	49.7130547014369\\
8.22599999999942	49.7031140786211\\
8.22799999999942	49.6931754435324\\
8.22999999999942	49.683238795773\\
8.23199999999942	49.673304134946\\
8.23399999999942	49.6633714606542\\
8.23599999999942	49.6534407724999\\
8.23799999999942	49.6435120700857\\
8.23999999999942	49.6335853530147\\
8.24199999999942	49.6236606208907\\
8.24399999999942	49.6137378733155\\
8.24599999999942	49.6038171098941\\
8.24799999999942	49.5938983302281\\
8.24999999999942	49.5839815339212\\
8.25199999999943	49.5740667205769\\
8.25399999999943	49.564153889799\\
8.25599999999943	49.5542430411906\\
8.25799999999943	49.5443341743561\\
8.25999999999943	49.534427288898\\
8.26199999999943	49.5245223844208\\
8.26399999999943	49.5146194605293\\
8.26599999999943	49.5047185168256\\
8.26799999999943	49.4948195529151\\
8.26999999999943	49.4849225684014\\
8.27199999999943	49.475027562889\\
8.27399999999943	49.4651345359816\\
8.27599999999943	49.4552434872842\\
8.27799999999943	49.4453544164013\\
8.27999999999943	49.4354673229364\\
8.28199999999944	49.4255822064948\\
8.28399999999944	49.4156990666816\\
8.28599999999944	49.4058179031013\\
8.28799999999944	49.3959387153584\\
8.28999999999944	49.386061503057\\
8.29199999999944	49.3761862658041\\
8.29399999999944	49.3663130032034\\
8.29599999999944	49.356441714861\\
8.29799999999944	49.3465724003816\\
8.29999999999944	49.3367050593686\\
8.30199999999944	49.3268396914302\\
8.30399999999944	49.3169762961713\\
8.30599999999944	49.3071148731968\\
8.30799999999944	49.297255422112\\
8.30999999999944	49.2873979425244\\
8.31199999999945	49.277542434037\\
8.31399999999945	49.2676888962574\\
8.31599999999945	49.257837328792\\
8.31799999999945	49.2479877312455\\
8.31999999999945	49.2381401032251\\
8.32199999999945	49.2282944443356\\
8.32399999999945	49.2184507541856\\
8.32599999999945	49.2086090323787\\
8.32799999999945	49.1987692785232\\
8.32999999999945	49.1889314922251\\
8.33199999999945	49.1790956730908\\
8.33399999999945	49.1692618207258\\
8.33599999999945	49.1594299347388\\
8.33799999999945	49.1496000147356\\
8.33999999999945	49.139772060324\\
8.34199999999946	49.1299460711095\\
8.34399999999946	49.1201220467003\\
8.34599999999946	49.1102999867031\\
8.34799999999946	49.1004798907249\\
8.34999999999946	49.0906617583734\\
8.35199999999946	49.0808455892552\\
8.35399999999946	49.0710313829788\\
8.35599999999946	49.061219139151\\
8.35799999999946	49.0514088573792\\
8.35999999999946	49.0416005372721\\
8.36199999999946	49.0317941784374\\
8.36399999999946	49.02198978048\\
8.36599999999946	49.0121873430118\\
8.36799999999946	49.0023868656386\\
8.36999999999946	48.992588347969\\
8.37199999999947	48.9827917896107\\
8.37399999999947	48.9729971901727\\
8.37599999999947	48.9632045492627\\
8.37799999999947	48.9534138664897\\
8.37999999999947	48.9436251414616\\
8.38199999999947	48.933838373787\\
8.38399999999947	48.924053563074\\
8.38599999999947	48.9142707089322\\
8.38799999999947	48.9044898109699\\
8.38999999999947	48.8947108687973\\
8.39199999999947	48.8849338820197\\
8.39399999999947	48.8751588502497\\
8.39599999999947	48.8653857730954\\
8.39799999999947	48.8556146501651\\
8.39999999999947	48.8458454810691\\
8.40199999999948	48.8360782654164\\
8.40399999999948	48.8263130028152\\
8.40599999999948	48.8165496928769\\
8.40799999999948	48.8067883352095\\
8.40999999999948	48.7970289294237\\
8.41199999999948	48.7872714751285\\
8.41399999999948	48.7775159719348\\
8.41599999999948	48.7677624194504\\
8.41799999999948	48.7580108172874\\
8.41999999999948	48.7482611650538\\
8.42199999999948	48.7385134623615\\
8.42399999999948	48.7287677088197\\
8.42599999999948	48.7190239040389\\
8.42799999999948	48.7092820476292\\
8.42999999999948	48.6995421392016\\
8.43199999999949	48.6898041783662\\
8.43399999999949	48.6800681647334\\
8.43599999999949	48.6703340979134\\
8.43799999999949	48.6606019775181\\
8.43999999999949	48.6508718031573\\
8.44199999999949	48.6411435744424\\
8.44399999999949	48.6314172909849\\
8.44599999999949	48.6216929523939\\
8.44799999999949	48.6119705582823\\
8.44999999999949	48.6022501082601\\
8.45199999999949	48.5925316019402\\
8.45399999999949	48.5828150389321\\
8.45599999999949	48.5731004188483\\
8.45799999999949	48.5633877413007\\
8.45999999999949	48.5536770058989\\
8.4619999999995	48.543968212257\\
8.4639999999995	48.5342613599846\\
8.4659999999995	48.5245564486948\\
8.4679999999995	48.5148534779992\\
8.4699999999995	48.5051524475097\\
8.4719999999995	48.4954533568383\\
8.4739999999995	48.4857562055975\\
8.4759999999995	48.4760609933987\\
8.4779999999995	48.4663677198547\\
8.4799999999995	48.4566763845778\\
8.4819999999995	48.4469869871804\\
8.4839999999995	48.4372995272747\\
8.4859999999995	48.4276140044741\\
8.4879999999995	48.4179304183903\\
8.4899999999995	48.4082487686364\\
8.49199999999951	48.3985690548256\\
8.49399999999951	48.3888912765703\\
8.49599999999951	48.3792154334834\\
8.49799999999951	48.3695415251786\\
8.49999999999951	48.3598695512684\\
8.50199999999951	48.3501995113661\\
8.50399999999951	48.3405314050851\\
8.50599999999951	48.3308652320386\\
8.50799999999951	48.3212009918406\\
8.50999999999951	48.3115386841037\\
8.51199999999951	48.3018783084415\\
8.51399999999951	48.2922198644686\\
8.51599999999951	48.2825633517984\\
8.51799999999951	48.2729087700446\\
8.51999999999951	48.2632561188208\\
8.52199999999952	48.253605397741\\
8.52399999999952	48.2439566064196\\
8.52599999999952	48.2343097444711\\
8.52799999999952	48.2246648115088\\
8.52999999999952	48.2150218071469\\
8.53199999999952	48.2053807310009\\
8.53399999999952	48.1957415826842\\
8.53599999999952	48.1861043618119\\
8.53799999999952	48.1764690679984\\
8.53999999999952	48.1668357008584\\
8.54199999999952	48.1572042600065\\
8.54399999999952	48.1475747450571\\
8.54599999999952	48.1379471556258\\
8.54799999999952	48.1283214913278\\
8.54999999999952	48.1186977517779\\
8.55199999999953	48.10907593659\\
8.55399999999953	48.0994560453821\\
8.55599999999953	48.0898380777654\\
8.55799999999953	48.0802220333589\\
8.55999999999953	48.0706079117768\\
8.56199999999953	48.060995712634\\
8.56399999999953	48.051385435547\\
8.56599999999953	48.041777080131\\
8.56799999999953	48.032170646002\\
8.56999999999953	48.0225661327752\\
8.57199999999953	48.0129635400669\\
8.57399999999953	48.0033628674938\\
8.57599999999953	47.9937641146709\\
8.57799999999953	47.9841672812142\\
8.57999999999953	47.9745723667413\\
8.58199999999954	47.9649793708672\\
8.58399999999954	47.9553882932085\\
8.58599999999954	47.9457991333815\\
8.58799999999954	47.9362118910036\\
8.58999999999954	47.9266265656902\\
8.59199999999954	47.9170431570586\\
8.59399999999954	47.9074616647256\\
8.59599999999954	47.8978820883083\\
8.59799999999954	47.8883044274228\\
8.59999999999954	47.8787286816865\\
8.60199999999954	47.8691548507161\\
8.60399999999954	47.8595829341293\\
8.60599999999954	47.8500129315435\\
8.60799999999954	47.8404448425745\\
8.60999999999954	47.8308786668416\\
8.61199999999955	47.8213144039606\\
8.61399999999955	47.8117520535499\\
8.61599999999955	47.8021916152269\\
8.61799999999955	47.7926330886092\\
8.61999999999955	47.7830764733148\\
8.62199999999955	47.773521768961\\
8.62399999999955	47.7639689751655\\
8.62599999999955	47.7544180915471\\
8.62799999999955	47.7448691177238\\
8.62999999999955	47.7353220533131\\
8.63199999999955	47.7257768979337\\
8.63399999999955	47.716233651203\\
8.63599999999955	47.7066923127408\\
8.63799999999955	47.6971528821641\\
8.63999999999955	47.6876153590922\\
8.64199999999956	47.6780797431436\\
8.64399999999956	47.6685460339372\\
8.64599999999956	47.6590142310907\\
8.64799999999956	47.6494843342239\\
8.64999999999956	47.6399563429554\\
8.65199999999956	47.6304302569042\\
8.65399999999956	47.6209060756889\\
8.65599999999956	47.6113837989291\\
8.65799999999956	47.6018634262437\\
8.65999999999956	47.5923449572527\\
8.66199999999956	47.582828391574\\
8.66399999999956	47.5733137288284\\
8.66599999999956	47.5638009686346\\
8.66799999999956	47.5542901106122\\
8.66999999999956	47.5447811543823\\
8.67199999999957	47.5352740995617\\
8.67399999999957	47.5257689457724\\
8.67599999999957	47.5162656926341\\
8.67799999999957	47.5067643397658\\
8.67999999999957	47.4972648867883\\
8.68199999999957	47.4877673333217\\
8.68399999999957	47.4782716789861\\
8.68599999999957	47.4687779234016\\
8.68799999999957	47.4592860661885\\
8.68999999999957	47.4497961069676\\
8.69199999999957	47.4403080453577\\
8.69399999999957	47.4308218809816\\
8.69599999999957	47.4213376134586\\
8.69799999999957	47.4118552424102\\
8.69999999999957	47.4023747674567\\
8.70199999999958	47.3928961882195\\
8.70399999999958	47.3834195043183\\
8.70599999999958	47.3739447153754\\
8.70799999999958	47.3644718210113\\
8.70999999999958	47.3550008208468\\
8.71199999999958	47.3455317145039\\
8.71399999999958	47.3360645016035\\
8.71599999999958	47.3265991817677\\
8.71799999999958	47.3171357546166\\
8.71999999999958	47.3076742197724\\
8.72199999999958	47.298214576857\\
8.72399999999958	47.2887568254922\\
8.72599999999958	47.2793009652987\\
8.72799999999958	47.2698469958997\\
8.72999999999958	47.2603949169165\\
8.73199999999959	47.250944727971\\
8.73399999999959	47.2414964286849\\
8.73599999999959	47.232050018681\\
8.73799999999959	47.2226054975819\\
8.73999999999959	47.2131628650091\\
8.74199999999959	47.2037221205847\\
8.74399999999959	47.1942832639318\\
8.74599999999959	47.1848462946727\\
8.74799999999959	47.1754112124304\\
8.74999999999959	47.1659780168268\\
8.75199999999959	47.156546707486\\
8.75399999999959	47.1471172840289\\
8.75599999999959	47.1376897460795\\
8.75799999999959	47.1282640932609\\
8.75999999999959	47.1188403251958\\
8.7619999999996	47.1094184415075\\
8.7639999999996	47.0999984418191\\
8.7659999999996	47.0905803257541\\
8.7679999999996	47.0811640929354\\
8.7699999999996	47.0717497429871\\
8.7719999999996	47.0623372755316\\
8.7739999999996	47.0529266901936\\
8.7759999999996	47.0435179865966\\
8.7779999999996	47.0341111643634\\
8.7799999999996	47.0247062231191\\
8.7819999999996	47.0153031624865\\
8.7839999999996	47.0059019820903\\
8.7859999999996	46.9965026815539\\
8.7879999999996	46.9871052605017\\
8.7899999999996	46.9777097185578\\
8.79199999999961	46.9683160553468\\
8.79399999999961	46.9589242704928\\
8.79599999999961	46.9495343636204\\
8.79799999999961	46.9401463343534\\
8.79999999999961	46.9307601823168\\
8.80199999999961	46.9213759071354\\
8.80399999999961	46.911993508434\\
8.80599999999961	46.9026129858367\\
8.80799999999961	46.8932343389686\\
8.80999999999961	46.8838575674554\\
8.81199999999961	46.8744826709211\\
8.81399999999961	46.8651096489911\\
8.81599999999961	46.8557385012912\\
8.81799999999961	46.8463692274453\\
8.81999999999961	46.8370018270805\\
8.82199999999962	46.8276362998203\\
8.82399999999962	46.8182726452907\\
8.82599999999962	46.8089108631184\\
8.82799999999962	46.7995509529279\\
8.82999999999962	46.7901929143453\\
8.83199999999962	46.7808367469958\\
8.83399999999962	46.7714824505056\\
8.83599999999962	46.7621300245003\\
8.83799999999962	46.7527794686071\\
8.83999999999962	46.7434307824504\\
8.84199999999962	46.7340839656569\\
8.84399999999962	46.7247390178534\\
8.84599999999962	46.7153959386658\\
8.84799999999962	46.7060547277202\\
8.84999999999962	46.6967153846432\\
8.85199999999963	46.6873779090613\\
8.85399999999963	46.678042300602\\
8.85599999999963	46.6687085588897\\
8.85799999999963	46.6593766835528\\
8.85999999999963	46.6500466742182\\
8.86199999999963	46.6407185305123\\
8.86399999999963	46.631392252062\\
8.86599999999963	46.6220678384938\\
8.86799999999963	46.6127452894363\\
8.86999999999963	46.6034246045148\\
8.87199999999963	46.5941057833587\\
8.87399999999963	46.5847888255932\\
8.87599999999963	46.575473730847\\
8.87799999999963	46.5661604987475\\
8.87999999999963	46.556849128922\\
8.88199999999964	46.5475396209973\\
8.88399999999964	46.5382319746029\\
8.88599999999964	46.5289261893644\\
8.88799999999964	46.5196222649114\\
8.88999999999964	46.5103202008719\\
8.89199999999964	46.5010199968721\\
8.89399999999964	46.4917216525419\\
8.89599999999964	46.4824251675082\\
8.89799999999964	46.4731305414\\
8.89999999999964	46.4638377738451\\
8.90199999999964	46.4545468644722\\
8.90399999999964	46.4452578129103\\
8.90599999999964	46.4359706187858\\
8.90799999999964	46.4266852817292\\
8.90999999999964	46.4174018013694\\
8.91199999999965	46.4081201773336\\
8.91399999999965	46.3988404092519\\
8.91599999999965	46.3895624967525\\
8.91799999999965	46.3802864394642\\
8.91999999999965	46.3710122370173\\
8.92199999999965	46.3617398890399\\
8.92399999999965	46.3524693951605\\
8.92599999999965	46.3432007550093\\
8.92799999999965	46.3339339682157\\
8.92999999999965	46.3246690344089\\
8.93199999999965	46.3154059532181\\
8.93399999999965	46.3061447242733\\
8.93599999999965	46.2968853472041\\
8.93799999999965	46.2876278216395\\
8.93999999999965	46.27837214721\\
8.94199999999966	46.2691183235454\\
8.94399999999966	46.2598663502753\\
8.94599999999966	46.2506162270297\\
8.94799999999966	46.2413679534392\\
8.94999999999966	46.2321215291335\\
8.95199999999966	46.2228769537428\\
8.95399999999966	46.2136342268972\\
8.95599999999966	46.2043933482276\\
8.95799999999966	46.1951543173642\\
8.95999999999966	46.1859171339384\\
8.96199999999966	46.1766817975778\\
8.96399999999966	46.1674483079165\\
8.96599999999966	46.1582166645832\\
8.96799999999966	46.1489868672099\\
8.96999999999966	46.1397589154269\\
8.97199999999967	46.1305328088651\\
8.97399999999967	46.1213085471557\\
8.97599999999967	46.1120861299299\\
8.97799999999967	46.1028655568187\\
8.97999999999967	46.0936468274528\\
8.98199999999967	46.0844299414646\\
8.98399999999967	46.0752148984846\\
8.98599999999967	46.0660016981452\\
8.98799999999967	46.0567903400771\\
8.98999999999967	46.0475808239126\\
8.99199999999967	46.0383731492827\\
8.99399999999967	46.0291673158197\\
8.99599999999967	46.0199633231544\\
8.99799999999967	46.0107611709208\\
8.99999999999967	46.0015608587488\\
9.00199999999968	45.9923623862716\\
9.00399999999968	45.9831657531214\\
9.00599999999968	45.9739709589291\\
9.00799999999968	45.9647780033291\\
9.00999999999968	45.9555868859512\\
9.01199999999968	45.9463976064298\\
9.01399999999968	45.9372101643974\\
9.01599999999968	45.928024559485\\
9.01799999999968	45.9188407913271\\
9.01999999999968	45.9096588595552\\
9.02199999999968	45.9004787638027\\
9.02399999999968	45.8913005037015\\
9.02599999999968	45.8821240788861\\
9.02799999999968	45.8729494889878\\
9.02999999999968	45.8637767336412\\
9.03199999999969	45.8546058124787\\
9.03399999999969	45.8454367251339\\
9.03599999999969	45.8362694712391\\
9.03799999999969	45.8271040504297\\
9.03999999999969	45.8179404623374\\
9.04199999999969	45.8087787065956\\
9.04399999999969	45.7996187828391\\
9.04599999999969	45.7904606907007\\
9.04799999999969	45.7813044298155\\
9.04999999999969	45.7721499998151\\
9.05199999999969	45.7629974003348\\
9.05399999999969	45.7538466310086\\
9.05599999999969	45.7446976914706\\
9.05799999999969	45.7355505813542\\
9.05999999999969	45.7264053002939\\
9.0619999999997	45.7172618479247\\
9.0639999999997	45.7081202238795\\
9.0659999999997	45.6989804277939\\
9.0679999999997	45.6898424593024\\
9.0699999999997	45.6807063180383\\
9.0719999999997	45.671572003638\\
9.0739999999997	45.6624395157348\\
9.0759999999997	45.6533088539643\\
9.0779999999997	45.6441800179603\\
9.0799999999997	45.635053007359\\
9.0819999999997	45.6259278217946\\
9.0839999999997	45.6168044609024\\
9.0859999999997	45.6076829243176\\
9.0879999999997	45.598563211675\\
9.0899999999997	45.5894453226108\\
9.09199999999971	45.5803292567592\\
9.09399999999971	45.5712150137564\\
9.09599999999971	45.5621025932381\\
9.09799999999971	45.552991994839\\
9.09999999999971	45.5438832181954\\
9.10199999999971	45.5347762629426\\
9.10399999999971	45.5256711287171\\
9.10599999999971	45.5165678151545\\
9.10799999999971	45.5074663218897\\
9.10999999999971	45.4983666485601\\
9.11199999999971	45.4892687948009\\
9.11399999999971	45.4801727602485\\
9.11599999999971	45.4710785445398\\
9.11799999999971	45.4619861473105\\
9.11999999999972	45.4528955681967\\
9.12199999999972	45.4438068068355\\
9.12399999999972	45.4347198628629\\
9.12599999999972	45.4256347359158\\
9.12799999999972	45.4165514256305\\
9.12999999999972	45.4074699316446\\
9.13199999999972	45.3983902535932\\
9.13399999999972	45.3893123911153\\
9.13599999999972	45.3802363438464\\
9.13799999999972	45.3711621114242\\
9.13999999999972	45.3620896934859\\
9.14199999999972	45.3530190896679\\
9.14399999999972	45.3439502996076\\
9.14599999999972	45.334883322943\\
9.14799999999972	45.3258181593111\\
9.14999999999973	45.3167548083498\\
9.15199999999973	45.3076932696953\\
9.15399999999973	45.2986335429871\\
9.15599999999973	45.2895756278613\\
9.15799999999973	45.2805195239564\\
9.15999999999973	45.2714652309104\\
9.16199999999973	45.2624127483608\\
9.16399999999973	45.2533620759457\\
9.16599999999973	45.2443132133028\\
9.16799999999973	45.2352661600705\\
9.16999999999973	45.2262209158878\\
9.17199999999973	45.2171774803916\\
9.17399999999973	45.2081358532211\\
9.17599999999973	45.1990960340144\\
9.17799999999973	45.1900580224099\\
9.17999999999974	45.1810218180461\\
9.18199999999974	45.1719874205619\\
9.18399999999974	45.1629548295962\\
9.18599999999974	45.1539240447872\\
9.18799999999974	45.1448950657745\\
9.18999999999974	45.1358678921954\\
9.19199999999974	45.1268425236906\\
9.19399999999974	45.1178189598986\\
9.19599999999974	45.1087972004591\\
9.19799999999974	45.09977724501\\
9.19999999999974	45.0907590931912\\
9.20199999999974	45.0817427446424\\
9.20399999999974	45.0727281990027\\
9.20599999999974	45.0637154559116\\
9.20799999999974	45.054704515008\\
9.20999999999975	45.0456953759331\\
9.21199999999975	45.0366880383261\\
9.21399999999975	45.0276825018249\\
9.21599999999975	45.0186787660721\\
9.21799999999975	45.0096768307058\\
9.21999999999975	45.0006766953666\\
9.22199999999975	44.9916783596946\\
9.22399999999975	44.9826818233298\\
9.22599999999975	44.9736870859126\\
9.22799999999975	44.9646941470831\\
9.22999999999975	44.9557030064824\\
9.23199999999975	44.9467136637493\\
9.23399999999975	44.9377261185254\\
9.23599999999975	44.9287403704515\\
9.23799999999975	44.9197564191675\\
9.23999999999976	44.9107742643147\\
9.24199999999976	44.9017939055338\\
9.24399999999976	44.8928153424652\\
9.24599999999976	44.8838385747504\\
9.24799999999976	44.8748636020298\\
9.24999999999976	44.8658904239451\\
9.25199999999976	44.8569190401368\\
9.25399999999976	44.8479494502469\\
9.25599999999976	44.8389816539163\\
9.25799999999976	44.8300156507859\\
9.25999999999976	44.8210514404982\\
9.26199999999976	44.8120890226935\\
9.26399999999976	44.8031283970141\\
9.26599999999976	44.7941695631013\\
9.26799999999976	44.7852125205972\\
9.26999999999977	44.7762572691434\\
9.27199999999977	44.7673038083817\\
9.27399999999977	44.758352137954\\
9.27599999999977	44.7494022575025\\
9.27799999999977	44.7404541666692\\
9.27999999999977	44.7315078650965\\
9.28199999999977	44.7225633524258\\
9.28399999999977	44.7136206283001\\
9.28599999999977	44.7046796923614\\
9.28799999999977	44.6957405442525\\
9.28999999999977	44.6868031836161\\
9.29199999999977	44.6778676100945\\
9.29399999999977	44.6689338233293\\
9.29599999999977	44.6600018229649\\
9.29799999999977	44.6510716086431\\
9.29999999999978	44.6421431800068\\
9.30199999999978	44.6332165366998\\
9.30399999999978	44.624291678364\\
9.30599999999978	44.6153686046434\\
9.30799999999978	44.6064473151804\\
9.30999999999978	44.5975278096184\\
9.31199999999978	44.5886100876007\\
9.31399999999978	44.5796941487711\\
9.31599999999978	44.5707799927723\\
9.31799999999978	44.561867619249\\
9.31999999999978	44.5529570278428\\
9.32199999999978	44.5440482181999\\
9.32399999999978	44.5351411899611\\
9.32599999999978	44.5262359427726\\
9.32799999999978	44.5173324762777\\
9.32999999999979	44.5084307901195\\
9.33199999999979	44.4995308839426\\
9.33399999999979	44.4906327573912\\
9.33599999999979	44.4817364101097\\
9.33799999999979	44.4728418417408\\
9.33999999999979	44.4639490519305\\
9.34199999999979	44.4550580403225\\
9.34399999999979	44.4461688065611\\
9.34599999999979	44.4372813502913\\
9.34799999999979	44.4283956711567\\
9.34999999999979	44.4195117688033\\
9.35199999999979	44.4106296428745\\
9.35399999999979	44.4017492930161\\
9.35599999999979	44.3928707188723\\
9.35799999999979	44.3839939200878\\
9.3599999999998	44.375118896309\\
9.3619999999998	44.3662456471798\\
9.3639999999998	44.3573741723453\\
9.3659999999998	44.3485044714506\\
9.3679999999998	44.3396365441418\\
9.3699999999998	44.3307703900635\\
9.3719999999998	44.3219060088621\\
9.3739999999998	44.3130434001821\\
9.3759999999998	44.3041825636694\\
9.3779999999998	44.2953234989691\\
9.3799999999998	44.2864662057286\\
9.3819999999998	44.2776106835914\\
9.3839999999998	44.2687569322051\\
9.3859999999998	44.2599049512148\\
9.3879999999998	44.2510547402669\\
9.38999999999981	44.2422062990077\\
9.39199999999981	44.2333596270817\\
9.39399999999981	44.2245147241365\\
9.39599999999981	44.2156715898188\\
9.39799999999981	44.2068302237739\\
9.39999999999981	44.1979906256489\\
9.40199999999981	44.1891527950898\\
9.40399999999981	44.1803167317439\\
9.40599999999981	44.1714824352563\\
9.40799999999981	44.1626499052755\\
9.40999999999981	44.153819141447\\
9.41199999999981	44.1449901434186\\
9.41399999999981	44.1361629108367\\
9.41599999999981	44.1273374433476\\
9.41799999999981	44.1185137405998\\
9.41999999999982	44.1096918022388\\
9.42199999999982	44.1008716279135\\
9.42399999999982	44.09205321727\\
9.42599999999982	44.0832365699567\\
9.42799999999982	44.0744216856188\\
9.42999999999982	44.0656085639062\\
9.43199999999982	44.0567972044651\\
9.43399999999982	44.0479876069441\\
9.43599999999982	44.0391797709895\\
9.43799999999982	44.0303736962506\\
9.43999999999982	44.0215693823738\\
9.44199999999982	44.0127668290083\\
9.44399999999982	44.003966035801\\
9.44599999999982	43.9951670024005\\
9.44799999999982	43.9863697284548\\
9.44999999999983	43.977574213612\\
9.45199999999983	43.9687804575206\\
9.45399999999983	43.9599884598286\\
9.45599999999983	43.9511982201848\\
9.45799999999983	43.942409738237\\
9.45999999999983	43.9336230136343\\
9.46199999999983	43.9248380460251\\
9.46399999999983	43.9160548350577\\
9.46599999999983	43.9072733803819\\
9.46799999999983	43.8984936816463\\
9.46999999999983	43.8897157384979\\
9.47199999999983	43.8809395505877\\
9.47399999999983	43.8721651175642\\
9.47599999999983	43.8633924390765\\
9.47799999999983	43.8546215147746\\
9.47999999999984	43.8458523443046\\
9.48199999999984	43.8370849273193\\
9.48399999999984	43.8283192634664\\
9.48599999999984	43.8195553523959\\
9.48799999999984	43.8107931937573\\
9.48999999999984	43.8020327871998\\
9.49199999999984	43.7932741323736\\
9.49399999999984	43.7845172289277\\
9.49599999999984	43.7757620765122\\
9.49799999999984	43.767008674778\\
9.49999999999984	43.7582570233726\\
9.50199999999984	43.749507121948\\
9.50399999999984	43.7407589701544\\
9.50599999999984	43.7320125676409\\
9.50799999999984	43.7232679140576\\
9.50999999999985	43.7145250090563\\
9.51199999999985	43.7057838522858\\
9.51399999999985	43.697044443397\\
9.51599999999985	43.6883067820404\\
9.51799999999985	43.6795708678672\\
9.51999999999985	43.6708367005271\\
9.52199999999985	43.6621042796709\\
9.52399999999985	43.6533736049499\\
9.52599999999985	43.6446446760148\\
9.52799999999985	43.6359174925168\\
9.52999999999985	43.6271920541054\\
9.53199999999985	43.6184683604333\\
9.53399999999985	43.6097464111511\\
9.53599999999985	43.6010262059099\\
9.53799999999985	43.5923077443612\\
9.53999999999986	43.5835910261554\\
9.54199999999986	43.5748760509458\\
9.54399999999986	43.5661628183818\\
9.54599999999986	43.5574513281167\\
9.54799999999986	43.5487415798005\\
9.54999999999986	43.5400335730858\\
9.55199999999986	43.5313273076241\\
9.55399999999986	43.5226227830675\\
9.55599999999986	43.5139199990678\\
9.55799999999986	43.5052189552775\\
9.55999999999986	43.4965196513467\\
9.56199999999986	43.4878220869288\\
9.56399999999986	43.4791262616765\\
9.56599999999986	43.4704321752416\\
9.56799999999986	43.4617398272763\\
9.56999999999987	43.4530492174331\\
9.57199999999987	43.4443603453637\\
9.57399999999987	43.4356732107216\\
9.57599999999987	43.4269878131588\\
9.57799999999987	43.4183041523286\\
9.57999999999987	43.4096222278828\\
9.58199999999987	43.4009420394751\\
9.58399999999987	43.3922635867576\\
9.58599999999987	43.3835868693837\\
9.58799999999987	43.3749118870063\\
9.58999999999987	43.3662386392784\\
9.59199999999987	43.3575671258535\\
9.59399999999987	43.3488973463838\\
9.59599999999987	43.3402293005239\\
9.59799999999987	43.3315629879261\\
9.59999999999988	43.3228984082447\\
9.60199999999988	43.3142355611328\\
9.60399999999988	43.3055744462433\\
9.60599999999988	43.2969150632306\\
9.60799999999988	43.2882574117484\\
9.60999999999988	43.2796014914502\\
9.61199999999988	43.2709473019896\\
9.61399999999988	43.2622948430208\\
9.61599999999988	43.2536441141984\\
9.61799999999988	43.2449951151751\\
9.61999999999988	43.2363478456062\\
9.62199999999988	43.2277023051448\\
9.62399999999988	43.219058493446\\
9.62599999999988	43.2104164101642\\
9.62799999999988	43.2017760549531\\
9.62999999999989	43.1931374274676\\
9.63199999999989	43.1845005273622\\
9.63399999999989	43.1758653542909\\
9.63599999999989	43.1672319079098\\
9.63799999999989	43.1586001878721\\
9.63999999999989	43.1499701938333\\
9.64199999999989	43.141341925448\\
9.64399999999989	43.1327153823716\\
9.64599999999989	43.1240905642592\\
9.64799999999989	43.115467470765\\
9.64999999999989	43.1068461015443\\
9.65199999999989	43.0982264562534\\
9.65399999999989	43.0896085345465\\
9.65599999999989	43.0809923360793\\
9.65799999999989	43.072377860507\\
9.6599999999999	43.0637651074854\\
9.6619999999999	43.0551540766701\\
9.6639999999999	43.0465447677165\\
9.6659999999999	43.0379371802803\\
9.6679999999999	43.0293313140176\\
9.6699999999999	43.020727168584\\
9.6719999999999	43.0121247436355\\
9.6739999999999	43.0035240388281\\
9.6759999999999	42.9949250538168\\
9.6779999999999	42.9863277882596\\
9.6799999999999	42.9777322418108\\
9.6819999999999	42.969138414128\\
9.6839999999999	42.9605463048675\\
9.6859999999999	42.9519559136843\\
9.6879999999999	42.9433672402375\\
9.68999999999991	42.9347802841795\\
9.69199999999991	42.9261950451705\\
9.69399999999991	42.9176115228658\\
9.69599999999991	42.9090297169224\\
9.69799999999991	42.9004496269973\\
9.69999999999991	42.8918712527465\\
9.70199999999991	42.8832945938283\\
9.70399999999991	42.8747196498978\\
9.70599999999991	42.8661464206141\\
9.70799999999991	42.8575749056326\\
9.70999999999991	42.8490051046117\\
9.71199999999991	42.8404370172086\\
9.71399999999991	42.8318706430797\\
9.71599999999991	42.8233059818831\\
9.71799999999991	42.8147430332771\\
9.71999999999992	42.8061817969171\\
9.72199999999992	42.7976222724626\\
9.72399999999992	42.7890644595706\\
9.72599999999992	42.7805083578991\\
9.72799999999992	42.7719539671057\\
9.72999999999992	42.7634012868482\\
9.73199999999992	42.7548503167849\\
9.73399999999992	42.7463010565737\\
9.73599999999992	42.7377535058724\\
9.73799999999992	42.7292076643396\\
9.73999999999992	42.7206635316335\\
9.74199999999992	42.7121211074119\\
9.74399999999992	42.7035803913334\\
9.74599999999992	42.6950413830568\\
9.74799999999992	42.6865040822406\\
9.74999999999993	42.6779684885425\\
9.75199999999993	42.6694346016226\\
9.75399999999993	42.6609024211384\\
9.75599999999993	42.6523719467488\\
9.75799999999993	42.6438431781132\\
9.75999999999993	42.6353161148902\\
9.76199999999993	42.6267907567387\\
9.76399999999993	42.6182671033183\\
9.76599999999993	42.6097451542874\\
9.76799999999993	42.6012249093055\\
9.76999999999993	42.592706368032\\
9.77199999999993	42.5841895301258\\
9.77399999999993	42.575674395247\\
9.77599999999993	42.5671609630541\\
9.77799999999993	42.5586492332075\\
9.77999999999994	42.5501392053665\\
9.78199999999994	42.5416308791911\\
9.78399999999994	42.5331242543399\\
9.78599999999994	42.5246193304738\\
9.78799999999994	42.5161161072522\\
9.78999999999994	42.5076145843355\\
9.79199999999994	42.499114761383\\
9.79399999999994	42.4906166380555\\
9.79599999999994	42.4821202140128\\
9.79799999999994	42.4736254889156\\
9.79999999999994	42.4651324624226\\
9.80199999999994	42.4566411341954\\
9.80399999999994	42.4481515038949\\
9.80599999999994	42.4396635711807\\
9.80799999999994	42.4311773357135\\
9.80999999999995	42.4226927971541\\
9.81199999999995	42.4142099551635\\
9.81399999999995	42.4057288094011\\
9.81599999999995	42.3972493595295\\
9.81799999999995	42.3887716052081\\
9.81999999999995	42.3802955460991\\
9.82199999999995	42.3718211818622\\
9.82399999999995	42.3633485121601\\
9.82599999999995	42.3548775366528\\
9.82799999999995	42.3464082550018\\
9.82999999999995	42.3379406668684\\
9.83199999999995	42.3294747719144\\
9.83399999999995	42.3210105698003\\
9.83599999999995	42.3125480601885\\
9.83799999999995	42.3040872427395\\
9.83999999999996	42.2956281171161\\
9.84199999999996	42.2871706829795\\
9.84399999999996	42.2787149399912\\
9.84599999999996	42.2702608878142\\
9.84799999999996	42.2618085261087\\
9.84999999999996	42.2533578545378\\
9.85199999999996	42.2449088727633\\
9.85399999999996	42.2364615804473\\
9.85599999999996	42.2280159772517\\
9.85799999999996	42.219572062839\\
9.85999999999996	42.2111298368719\\
9.86199999999996	42.2026892990118\\
9.86399999999996	42.1942504489222\\
9.86599999999996	42.1858132862648\\
9.86799999999996	42.1773778107028\\
9.86999999999997	42.1689440218984\\
9.87199999999997	42.160511919515\\
9.87399999999997	42.1520815032139\\
9.87599999999997	42.1436527726592\\
9.87799999999997	42.1352257275138\\
9.87999999999997	42.1268003674403\\
9.88199999999997	42.1183766921019\\
9.88399999999997	42.1099547011616\\
9.88599999999997	42.1015343942828\\
9.88799999999997	42.0931157711284\\
9.88999999999997	42.0846988313621\\
9.89199999999997	42.0762835746475\\
9.89399999999997	42.0678700006472\\
9.89599999999997	42.0594581090254\\
9.89799999999997	42.0510478994455\\
9.89999999999998	42.0426393715712\\
9.90199999999998	42.0342325250665\\
9.90399999999998	42.0258273595943\\
9.90599999999998	42.0174238748194\\
9.90799999999998	42.0090220704053\\
9.90999999999998	42.0006219460166\\
9.91199999999998	41.992223501316\\
9.91399999999998	41.9838267359688\\
9.91599999999998	41.9754316496386\\
9.91799999999998	41.9670382419902\\
9.91999999999998	41.9586465126885\\
9.92199999999998	41.9502564613953\\
9.92399999999998	41.941868087778\\
9.92599999999998	41.9334813915001\\
9.92799999999998	41.925096372225\\
9.92999999999999	41.9167130296192\\
9.93199999999999	41.9083313633464\\
9.93399999999999	41.8999513730721\\
9.93599999999999	41.8915730584605\\
9.93799999999999	41.8831964191765\\
9.93999999999999	41.8748214548853\\
9.94199999999999	41.8664481652525\\
9.94399999999999	41.8580765499423\\
9.94599999999999	41.8497066086207\\
9.94799999999999	41.8413383409523\\
9.94999999999999	41.8329717466031\\
9.95199999999999	41.8246068252382\\
9.95399999999999	41.816243576523\\
9.95599999999999	41.8078820001231\\
9.95799999999999	41.7995220957039\\
9.96	41.7911638629308\\
9.962	41.782807301471\\
9.964	41.7744524109885\\
9.966	41.7660991911504\\
9.968	41.7577476416219\\
9.97	41.74939776207\\
9.972	41.7410495521592\\
9.974	41.7327030115573\\
9.976	41.7243581399289\\
9.978	41.7160149369413\\
9.98	41.7076734022609\\
9.982	41.6993335355544\\
9.984	41.6909953364865\\
9.986	41.6826588047257\\
9.988	41.6743239399375\\
9.99000000000001	41.6659907417895\\
9.99200000000001	41.6576592099472\\
9.99400000000001	41.6493293440783\\
9.99600000000001	41.6410011438498\\
9.99800000000001	41.6326746089282\\
10	41.62434973898\\
};
\end{axis}

\begin{axis}[%
width=7.778in,
height=5.833in,
at={(0in,0in)},
scale only axis,
xmin=0,
xmax=1,
ymin=0,
ymax=1,
axis line style={draw=none},
ticks=none,
axis x line*=bottom,
axis y line*=left
]
\end{axis}
\end{tikzpicture}%
\caption{Nei grafici vengono  rappresentato l'indice di stiff (in scala logaritmica) in funzione del tempo  per la risoluzione del sistema di ode.I parametri utilizzati sono $\tau =0.3$ e $\gamma =0.1$}
 \label{stiff_soluzione}
\end{figure}
\end{document}
