\section{Un esempio completo}
In questa sezione andremo  a ricavarci le equazioni per la rete in Figura~\ref{fig::lollipop} sia usando il modello esteso che attraverso l'uso dei cut-vertex.
\begin{figure}[!ht]
\centering
\begin{tikzpicture}[scale=0.8]
\Vertex[label=1]{1}
\Vertex[label=2,y=-2]{2}
\Vertex[label=3, y=-1, x=1.73]{3}
\Vertex[label=4, y=-1, x= 3.73]{4}
\Edge(1)(2) \Edge(2)(3) \Edge(1)(3) \Edge(3)(4)
\end{tikzpicture}

\caption[Lollipop ]{Il network lollipop.}
\label{fig::lollipop}
\end{figure}
\subsection{Modello esteso}
Le equazioni per i singoli nodi sono:
 \begin{equation*}
 \begin{aligned}
 \dot{\angol{S_1}}&= -\tau \tonde{ \angol{ S_1 I_3} + \angol{S_1 I_2}},  \quad 
 &\dot{\angol{S_3}}&= -\tau \tonde{ \angol{I_1 S_3} + \angol{I_2 S_3}+ \angol{S_3 I_4}},  \\
 \dot{\angol{I_1}}&= \spa \tau \tonde{ \angol{ S_1 I_2} + \angol{S_1 I_3}} - \gamma\angol{I_1},\quad 
 &\dot{\angol{I_3}}&= \spa \tau \tonde{ \angol{I_1 S_3} + \angol{I_2 S_3}+ \angol{S_3 I_4}}-\gamma\angol{I_3},\\
 \dot{\angol{S_2}}&= -\tau \tonde{ \angol{ I_1 S_2} + \angol{S_2 I_3}}, \quad &\dot{\angol{S_4}} &= -\tau \angol{I_3S_4},  \\
 \dot{\angol{I_2}}&= \spa \tau \tonde{ \angol{ I_1 S_2} + \angol{S_2 I_3}} - \gamma\angol{I_2},\quad & \dot{\angol{I_4}}&=\spa \tau\angol{I_3S_4}-\gamma\angol{I_4}. \\\
 \end{aligned}	
 \end{equation*}
 Tali equazioni dipendono da alcune coppie: tutte le disposizioni di archi con un nodo suscettibile ed un uno infetto. Abbiamo bisogno di equazioni addizionali:
 \begin{equation*}
 \begin{aligned}
 \dot{\angol{S_1I_2}} =&\spa \tau\angol{S_1S_2I_3} - \tonde{ \tau + \gamma} \angol{S_1 I_2} -\tau\angol{S_1I_2I_3},\\
 \dot{\angol{S_1I_3}}=&\spa \tau\tonde{ \angol{S_1I_2S_3} + \angol{S_1S_3I_4}} - \tonde{ \tau +\gamma}\angol{S_1 I_3} -\tau \angol{ S_1 I_2 I_3},\\
 \dot{\angol{I_1S_2}} =&\spa\tau\angol{S_1S_2I_3} - \tonde{ \tau + \gamma} \angol{I_1 S_2} -\tau\angol{I_1S_2I_3},\\
\dot{\angol{S_2I_3}} =&\spa \tau\tonde{\angol{I_1S_2S_3} +{\angol{S_2S_3I_4}}} - \tonde{ \tau + \gamma} \angol{S_2 I_3} -\tau\angol{I_1S_2I_3},\\
 \dot{\angol{I_1S_3}} =&- \tau\tonde{\angol{I_1I_2S_3} \angol{I_1S_3I_4}} - \tonde{ \tau + \gamma} \angol{I_1 S_3} +\tau\angol{S_1I_2S_3},\\ 
\dot{\angol{I_2S_3}} =&- \tau\tonde{\angol{I_1I_2S_3} +\angol{I_2S_3I_4}} - \tonde{ \tau + \gamma} \angol{I_2 S_3} +\tau\angol{I_1S_2S_3},\\ 
 \dot{\angol{S_3I_4}} =&- \tau\tonde{\angol{I_1S_3I_4}+\angol{I_2S_3I_4}} - \tonde{ \tau + \gamma} \angol{S_3 I_4}, \\ 
    \dot{\angol{I_3S_4}} =& \spa \tau\tonde{\angol{I_1S_3S_4}+\angol{I_2S_3S_4}} - \tonde{ \tau + \gamma} \angol{I_3 S_4}.\\ 
\end{aligned}
 \end{equation*}
A questo punto, per\`o necessitiamo anche delle triple: 
\begin{equation*}
 \begin{aligned}
 		\dot{\angol{S_1S_2I_3}}=&\spa \tau \angol{S_1S_2S_3I_4} -\tonde{ 2\tau + \gamma} \angol{S_1S_2I_3},\\
 		\dot{\angol{S_1I_2I_3}} =&\spa \tau \tonde{ \angol{S_1S_2I_3}+\angol{S_1I_2S_3I_4} } -2\tonde{ \tau + \gamma} \angol{S_1I_2I_3} + \tau \angol{S_1 I_2 S_3},\\
 		\dot{\angol{S_1I_2S_3}} =&-\tau \angol{S_1I_2S_3I_4} -\tonde{ 2\tau + \gamma} \angol{S_1I_2S_3},\\
 		\dot{\angol{I_1S_2I_3}} =&\spa \tau  \angol{S_1S_2I_3} -2\tonde{ \tau + \gamma} \angol{I_1S_2I_3} + \tau \angol{I_1S_2S_3}+\tau \angol{I_1S_2S_3I_4},\\
 		\dot{\angol{I_1S_2S_3}}=&-\tau \angol{ I_1S_2S_3I_4}-\tonde{ 2 \tau + \gamma} \angol{I_1S_2S_3},\\
 		\dot{\angol{I_1I_2S_3}} =&\spa\tau \tonde{ \angol{S_1I_2S_3}+ \angol{I_1S_2S_3} - \angol{I_1I_2S_3I_4}} -2\tonde{ \tau + \gamma} \angol{I_1I_2S_3}.
 	\end{aligned}
 \end{equation*}
 Mancano:
 \begin{equation*}
 	\begin{aligned}
 	\angol{I_1S_3I_4}=&-2
 	\tonde{ \tau + \gamma}\angol{ I_1 S_3 I_4} + \tau \angol{S_1I_2S_3I_4}-\tau \angol{ I_1I_2S_3I_4},\\
 	\angol{I_1S_3 S_4} =& - \tonde{ \tau + \gamma}  \angol{I_1 S_3 S_4} + \tau\angol{S_1I_2S_3S_4} -\tau \angol{I_1I_2S_3S_4},\\
\angol{S_2S_3I_4}=& 	-\tonde{ \tau + \gamma } \angol{S_2S_3I_4}  - 2\tau \angol{ I_1S_2S_3I_4}, \\
 	\angol{I_2S_3I_4}= &-2\tonde{ \tau + \gamma}  	\angol{I_2S_3I_4} + \tau \angol{ I_1S_2S_3I_4} - \tau \angol{ I_1I_2S_3 I_4}, \\
 	\angol{I_2S_3S_4}=&- \tonde{ \tau + \gamma}\angol{ I_2 S_3S_4} +\tau \angol{ I_1S_2S_3S_4} - \tau \angol{ I_1I_2 S_3S_4},\\
 	\angol{ S_1S_3I_4} =&- 2\tau \angol{ S_1I_2S_3I_4} -\tonde{ \tau + \gamma} \angol{ S_1S_3I_4},
 	\end{aligned}
 \end{equation*}
e le quadruple:
\begin{equation*}
 	\begin{aligned}	
 	\angol{I_1I_2S_3I_4} = &- 3\tonde{ \tau +\gamma} \angol{ I_1 I_2 S_3 I_4} + \tau \tonde{\angol{ S_1 I_2 S_3 I_4 } +\angol{ I_1 S_2 S_3 I_4}}, \\
 	\angol{S_1I_2S_3I_4}=& -\tonde{ 3\tau + 2\gamma}  	\angol{S_1I_2S_3I_4},\\
 	\angol{I_1S_2S_3I_4}=&-\tonde{ 3\tau + 2\gamma}\angol{I_1S_2S_3I_4},\\
 	\angol{I_1I_2S_3S_4}=& \tau \tonde{\angol{S_1 I_2 S_3S_4} +\angol{ I_1 S_2 S_3S_4}} -2\tonde{ \tau+\gamma} \angol{I_1I_2S_3S_4},\\
 	\angol{I_1S_2S_3S_4}=&-\tonde{ 2\tau + \gamma} \angol{I_1S_2S_3S_4},\\
 	\angol{S_1I_2S_3S_4}=&-\tonde{ 2\tau + \gamma} \angol{S_1I_2S_3S_4},\\
 	\angol{S_1S_2S_3I_4}=&-\tonde{ \tau + \gamma} \angol{S_1S_2S_3I_4}.\\	
 	\end{aligned}
\end{equation*}
Abbiamo cos\`i ottenuto un sistema chiuso di $35$ equazioni.
\newpage
\subsection{Modello chiuso}
Vediamo come possiamo sfruttare il Teorema~\ref{th_cut-vertex} per ridurre il numero di equazioni.\\
La rete lollipop possiede un solo cut-vertex: il vertice $3$. Decomponiamo il grafo in un triangolo ed in un singolo arco (Figura~\ref{fig::lollipop_cut}).

\begin{figure}[!htb]
\centering
\begin{tikzpicture}[scale=0.8]
\Vertex[label=1]{1}
\Vertex[label=2,y=-2]{2}
\Vertex[label=3, y=-1, x=1.73]{3}
\Vertex[label=3, y=-1, x= 3.73]{33}
\Vertex[label=4, y=-1, x=5.73]{4}
\Edge(1)(2) \Edge(2)(3) \Edge(1)(3) \Edge(33)(4)
\end{tikzpicture}
\caption{Decomposizione della rete lollipop.}
\label{fig::lollipop_cut}
\end{figure}
Per i nodi otteniamo le stesse equazioni viste in precedenza:
\begin{equation*}
 \begin{aligned}
 \dot{\angol{S_1}}&= -\tau \tonde{ \angol{ S_1 I_3} + \angol{S_1 I_2}},  \quad 
 &\dot{\angol{S_3}}&= -\tau \tonde{ \angol{I_1 S_3} + \angol{I_2 S_3}+ \angol{S_3 I_4}},  \\
 \dot{\angol{I_1}}&= \spa \tau \tonde{ \angol{ S_1 I_2} + \angol{S_1 I_3}} - \gamma\angol{I_1},\quad 
 &\dot{\angol{I_3}}&= \spa \tau \tonde{ \angol{I_1 S_3} + \angol{I_2 S_3}+ \angol{S_3 I_4}}-\gamma\angol{I_3},\\
 \dot{\angol{S_2}}&= -\tau \tonde{ \angol{ I_1 S_2} + \angol{S_2 I_3}}, \quad &\dot{\angol{S_4}} &= -\tau \angol{I_3S_4},  \\
 \dot{\angol{I_2}}&= \spa \tau \tonde{ \angol{ I_1 S_2} + \angol{S_2 I_3}} - \gamma\angol{I_2},\quad & \dot{\angol{I_4}}&=\spa \tau\angol{I_3S_4}-\gamma\angol{I_4}. \\
 \end{aligned}	
 \end{equation*}
 Andiamo a ricavarci le equazioni per le coppie.\\ Esaminiamo l'equazione per $\angol{S_1 I_3}$. 
 \begin{equation*}
 \begin{aligned}	
 \dot{\angol{S_1I_3}}&= \tau\tonde{ \angol{S_1I_2S_3} + \angol{S_1S_3I_4}} - \tonde{ \tau +\gamma}\angol{S_1 I_3} -\tau \angol{ S_1 I_2 I_3}= \\
 &= \tau \tonde{ \angol{S_1I_2S_3} + \frac{ \angol{S_1S_3}\angol{S_3 I_4}}{\angol{S_3}}} -\tonde{ \tau + \gamma} \angol{S_1I_3} - \tau \angol{S_1I_2S_3}
 \end{aligned}
 \end{equation*}
 dove abbiamo usato il Teorema~\ref{th_cut-vertex} per sostituire $\angol{S_1S_3I_4}$.\\
Con considerazioni analoghe otteniamo
  \begin{equation*}
 \begin{aligned}
 \dot{\angol{S_1I_2}} =&\spa \tau\angol{S_1S_2I_3} - \tonde{ \tau + \gamma} \angol{S_1 I_2} -\tau\angol{S_1I_2I_3},\\
\dot{\angol{I_1S_2}} =&\spa\tau\angol{S_1S_2I_3} - \tonde{ \tau + \gamma} \angol{I_1 S_2} -\tau\angol{I_1S_2I_3},\\
\dot{\angol{S_2I_3}} =&\spa \tau\tonde{\angol{I_1S_2S_3} +\frac{\angol{S_2S_3}\angol{S_3I_4}}{\angol{S_3}}} - \tonde{ \tau + \gamma} \angol{S_2 I_3} -\tau\angol{I_1S_2I_3},\\
\dot{\angol{I_1S_3}} =&- \tau\tonde{\angol{I_1I_2S_3} +\frac{\angol{I_1S_3}\angol{S_3I_4}}{\angol{S_3}}} - \tonde{ \tau + \gamma} \angol{I_1 S_3} +\tau\angol{S_1I_2S_3},\\ 
\dot{\angol{I_2S_3}} =&- \tau\tonde{\angol{I_1I_2S_3} +\frac{\angol{I_2S_3}\angol{S_3I_4}}{\angol{S_3}}}- \tonde{ \tau + \gamma} \angol{I_2 S_3} +\tau\angol{I_1S_2S_3},\\ 
\dot{\angol{S_3I_4}} =&- \tau\tonde{\frac{\angol{I_1S_3}\angol{S_3I_4}}{\angol{S_3}}+ \frac{\angol{I_2S_3}\angol{S_3I_4}}{\angol{S_3}}} - \tonde{ \tau + \gamma} \angol{S_3 I_4}, \\ 
\dot{\angol{I_3S_4}} =& \spa \tau\tonde{\frac{\angol{I_1S_3}\angol{S_3S_4}}{\angol{S_3}}+\frac{\angol{I_2S_3}\angol{S_3S_4}}{\angol{S_3}}} - \tonde{ \tau + \gamma} \angol{I_3 S_4}.\\ 
\end{aligned}
 \end{equation*}
 Da queste equazioni emergono nuove coppie, dunque necessitiamo anche di 
 \begin{equation*}
\begin{aligned}
	\dot{\angol{S_1S_3}}=& - \tau \frac{\angol{S_1S_3}\angol{S_3I_4}}{\angol{S_3}}- 2\tau \angol{S_1I_2S_3},\\
	\dot{\angol{S_2S_3}}=& -\tau \frac{\angol{S_2S_3}\angol{S_3I_4}}{\angol{S_3}} -2\tau \angol{I_1S_2S_3},\\
	\dot{\angol{S_3S_4}}=& -\tau \frac{\angol{I_1S_3}\angol{S_3S_4}}{\angol{S_3}}-\tau \frac{\angol{I_2S_3}\angol{S_3 S_4}}{\angol{S_3}}.
\end{aligned} 	
 \end{equation*}
 Le equazioni per le coppie introducono alcune triple. Aggiungiamole.
 \begin{equation*}
 \begin{aligned}
\dot{\angol{S_1S_2I_3}}=&\spa \tau \frac{\angol{S_1S_2S_3}\angol{S_3I_4}}{\angol{S_3}} -\tonde{ 2\tau + \gamma} \angol{S_1S_2I_3},\\
\dot{\angol{S_1I_2I_3}} =&\spa \tau \tonde{ \angol{S_1S_2I_3}+\frac{\angol{S_1I_2S_3}\angol{S_3I_4}}{\angol{S_3} }} -2\tonde{ \tau + \gamma} \angol{S_1I_2I_3} + \tau \angol{S_1 I_2 S_3},\\
\dot{\angol{S_1I_2S_3}} =&-\tau \frac{\angol{S_1I_2S_3}\angol{S_3I_4}}{\angol{S_3}} -\tonde{ 2\tau + \gamma} \angol{S_1I_2S_3},\\
\dot{\angol{I_1S_2I_3}} =&\spa \tau  \angol{S_1S_2I_3} -2\tonde{ \tau + \gamma} \angol{I_1S_2I_3} + \tau \angol{I_1S_2S_3}+\tau \frac{\angol{I_1S_2S_3}\angol{S_3I_4}}{\angol{S_3}},\\
\dot{\angol{I_1S_2S_3}}=&-\tau \frac{\angol{ I_1S_2S_3}\angol{S_3I_4}}{\angol{S_3}}-\tonde{ 2 \tau + \gamma} \angol{I_1S_2S_3},\\
\dot{\angol{I_1I_2S_3}} =&\spa\tau \tonde{  \angol{S_1I_2S_3}+ \angol{I_1S_2S_3} - \frac{\angol{I_1I_2S_3}\angol{S_3I_4}}{\angol{S_3}}} -2\tonde{ \tau + \gamma} \angol{I_1I_2S_3}.
 	\end{aligned}
 	\end{equation*}
 Scrivendo queste equazioni appare una nuova tripla, completiamo il sistema aggiungendo 
$$\dot{\angol{S_1 S_2 S_3}} = -\tau\frac{\angol{S_1S_2S_3}\angol{S_3I_4}}{\angol{S_3}}.$$
 
Siamo riusciti a ridurre il numero di equazioni da $35$ a $27$.\\ \\ \\
 Andiamo a risolvere numericamente i due sistemi con condizioni iniziali a partire da uno stato puro ($1$ certamente infetto e $2,3,4$ certamente sani):
\begin{equation}
\label{statipuri_lollipop}
\begin{aligned}
\angol{S_1}(0) =& 0, \\
\angol{I_1}(0) =& 1,\\
\angol{S_2}(0) = &1, \\
\angol{I_2}(0) = &0,\\
\angol{S_3}(0) = &1, \\
\angol{I_3}(0) = &0,\\
\angol{S_4}(0) =& 1, \\
\angol{I_4}(0) = &0.
\end{aligned}
\end{equation}
Nella Figura~\ref{fig::plot_lollipop} sono riportati i grafici dello stato dei nodi in funzione del tempo sia per il modello esteso che per quello chiuso. Come ci aspettavamo dalla teoria i grafici ottenuti dai due modelli coincidono su ogni nodo.\\
Per marcare ci\`o abbiamo calcolato gli errori assoluti tra le due soluzioni (Figura~\ref{fig::lollipop_errori}) ottenendo che tali errori sono dell'ordine della tolleranza dell'integratore

\begin{figure}
	
	\centering
\subfloat[][Nodo 1]
{\resizebox{0.45\textwidth}{!}{% This file was created by matlab2tikz.
%
%The latest updates can be retrieved from
%  http://www.mathworks.com/matlabcentral/fileexchange/22022-matlab2tikz-matlab2tikz
%where you can also make suggestions and rate matlab2tikz.
%
\definecolor{mycolor1}{rgb}{0.00000,0.44700,0.74100}%
\definecolor{mycolor2}{rgb}{0.85000,0.32500,0.09800}%
\definecolor{mycolor3}{rgb}{0.92900,0.69400,0.12500}%
%
\begin{tikzpicture}

\begin{axis}[%
width=0.39\columnwidth,
height=1.7in,
at={(1.011in,0.642in)},
scale only axis,
xmin=0,
xmax=60,
xlabel style={font=\color{white!15!black}},
xlabel={T},
ymin=0,
ymax=1,
axis background/.style={fill=white},
legend style={legend cell align=left, align=left, draw=none,fill=none}
]
\addplot [color=mycolor1, line width=2.0pt]
  table[row sep=crcr]{%
0	1\\
0.00199053585276749	0.999999821758677\\
0.00398107170553497	0.999999287271421\\
0.00597160755830246	0.999998396893462\\
0.00796214341106994	0.999997150980241\\
0.0123567669950473	0.999993143127537\\
0.0167513905790247	0.999987407889205\\
0.0211460141630021	0.999979949100724\\
0.0255406377469794	0.999970770602215\\
0.0299388563931551	0.999959866624738\\
0.0343370750393307	0.999947247832025\\
0.0387352936855064	0.999932918087282\\
0.043133512331682	0.999916881258003\\
0.0475471725191068	0.999899075937224\\
0.0519608327065316	0.999879559339989\\
0.0563744928939564	0.999858335387438\\
0.0607881530813811	0.999835408004714\\
0.0652173199794049	0.999810691606811\\
0.0696464868774286	0.999784267719524\\
0.0740756537754524	0.99975614032141\\
0.0785048206734761	0.999726313394739\\
0.0829496187548012	0.999694676680215\\
0.0873944168361263	0.999661336466192\\
0.0918392149174515	0.999626296788277\\
0.0962840129987766	0.999589561685501\\
0.100744568101794	0.999550995972667\\
0.105205123204811	0.999510730952913\\
0.109665678307829	0.999468770718526\\
0.114126233410846	0.999425119364925\\
0.118602673075784	0.999379616525651\\
0.123079112740721	0.999332418774377\\
0.127555552405659	0.999283530259702\\
0.132031992070596	0.999232955133067\\
0.136524445545197	0.999180507589632\\
0.141016899019798	0.999126369731142\\
0.1455093524944	0.999070545762137\\
0.150001805969001	0.999013039889705\\
0.154510404254387	0.998953640611345\\
0.159019002539772	0.998892555816577\\
0.163527600825158	0.998829789765512\\
0.168036199110544	0.998765346720515\\
0.172561074936287	0.99869898922088\\
0.17708595076203	0.998630951204923\\
0.181610826587773	0.99856123698795\\
0.186135702413515	0.998489850887231\\
0.190676990405524	0.998416529219213\\
0.195218278397533	0.998341532236036\\
0.199759566389542	0.998264864307833\\
0.204300854381551	0.998186529806404\\
0.208858690959197	0.998106238560577\\
0.213416527536843	0.998024277401592\\
0.217974364114488	0.997940650754035\\
0.222532200692134	0.997855363043858\\
0.227106724202768	0.997768097343565\\
0.231681247713401	0.997679167332647\\
0.236255771224035	0.997588577489767\\
0.240830294734669	0.997496332294659\\
0.245421645441331	0.997402087793263\\
0.250012996147992	0.997306184784048\\
0.254604346854654	0.997208627799382\\
0.259195697561315	0.9971094213724\\
0.263804017673384	0.997008194250009\\
0.268412337785452	0.99690531462264\\
0.273020657897521	0.996800787075986\\
0.27762897800959	0.996694616196209\\
0.282254411746462	0.996586403155677\\
0.286879845483334	0.996476543812793\\
0.291505279220206	0.9963650428062\\
0.296130712957079	0.996251904774706\\
0.300773406613671	0.996136703037373\\
0.305416100270264	0.996019861399845\\
0.310058793926856	0.995901384553338\\
0.314701487583449	0.995781277188931\\
0.319361589521735	0.99565908449209\\
0.324021691460022	0.995535258496541\\
0.328681793398309	0.995409803945692\\
0.333341895336596	0.995282725582511\\
0.338019556071109	0.995153540174692\\
0.342697216805623	0.995022728268756\\
0.347374877540136	0.994890294659924\\
0.35205253827465	0.994756244142672\\
0.356747910447269	0.994620064781785\\
0.361443282619888	0.994482265922313\\
0.366138654792507	0.994342852410912\\
0.370834026965126	0.994201829093182\\
0.375547265516765	0.994058655039212\\
0.380260504068404	0.993913868684896\\
0.384973742620042	0.993767474927943\\
0.389686981171681	0.993619478664697\\
0.394418243246227	0.993469309679903\\
0.399149505320772	0.993317535791587\\
0.403880767395317	0.993164161948125\\
0.408612029469863	0.993009193096221\\
0.41336147459102	0.992852029437424\\
0.418110919712177	0.992693268470312\\
0.422860364833334	0.992532915193547\\
0.427609809954491	0.992370974603809\\
0.432377599961091	0.992206817021985\\
0.437145389967691	0.992041069925331\\
0.44191317997429	0.991873738362411\\
0.44668096998089	0.991704827379495\\
0.451467269141767	0.991533677113835\\
0.456253568302644	0.991360945324975\\
0.461039867463521	0.991186637110996\\
0.465826166624398	0.991010757567365\\
0.47063114173014	0.99083261633969\\
0.475436116835881	0.990652901778442\\
0.480241091941623	0.990471619030831\\
0.485046067047365	0.990288773241141\\
0.489869887370685	0.990103643255066\\
0.494693707694006	0.989916948322975\\
0.499517528017326	0.989728693640822\\
0.504341348340647	0.989538884401315\\
0.509184185682064	0.989346768338758\\
0.51402702302348	0.989153095915629\\
0.518869860364897	0.988957872376234\\
0.523712697706314	0.988761102961316\\
0.528574726619363	0.9885620039717\\
0.533436755532412	0.988361357404702\\
0.53829878444546	0.988159168552592\\
0.543160813358509	0.987955442703756\\
0.548042211039569	0.987749364406143\\
0.55292360872063	0.987541747512073\\
0.55780500640169	0.987332597361391\\
0.56268640408275	0.987121919289734\\
0.56758735048985	0.986908865766266\\
0.572488296896949	0.986694282824975\\
0.577389243304049	0.986478175852884\\
0.582290189711149	0.986260550232489\\
0.587210867601438	0.986040526025058\\
0.592131545491728	0.985818981776143\\
0.597052223382018	0.985595922919559\\
0.601972901272307	0.985371354884264\\
0.606913496388328	0.985144364984507\\
0.611854091504349	0.984915864617265\\
0.61679468662037	0.984685859262745\\
0.621735281736391	0.984454354395976\\
0.626695982628195	0.984220404251637\\
0.631656683519999	0.983984953411585\\
0.636617384411802	0.983748007402026\\
0.641578085303606	0.983509571743659\\
0.646559083672327	0.983268667241442\\
0.651540082041047	0.983026272013029\\
0.656521080409768	0.982782391630227\\
0.661502078778488	0.982537031659008\\
0.666503569383853	0.982289179127142\\
0.671505059989219	0.98203984603643\\
0.676506550594584	0.981789038003884\\
0.68150804119995	0.98153676064035\\
0.686530221876942	0.981281966846795\\
0.691552402553934	0.981025702859724\\
0.696574583230927	0.980767974340955\\
0.701596763907919	0.980508786945803\\
0.706639835866396	0.980247059082151\\
0.711682907824873	0.979983871588325\\
0.71672597978335	0.979719230170549\\
0.721769051741827	0.979453140528209\\
0.72683321952555	0.97918448621016\\
0.731897387309272	0.978914383023421\\
0.736961555092995	0.978642836718223\\
0.742025722876718	0.978369853037617\\
0.747111194287107	0.978094280308244\\
0.752196665697496	0.977817269670052\\
0.757282137107885	0.977538826916873\\
0.762367608518274	0.97725895783502\\
0.767474594900923	0.976976475149014\\
0.772581581283573	0.976692565712595\\
0.777688567666222	0.976407235362793\\
0.782795554048871	0.976120489928775\\
0.78792427039055	0.975831106146065\\
0.793052986732229	0.975540306970055\\
0.798181703073909	0.975248098280565\\
0.803310419415588	0.974954485949213\\
0.808461084325784	0.974658210335229\\
0.81361174923598	0.974360530884022\\
0.818762414146177	0.974061453517796\\
0.823913079056373	0.973760984150209\\
0.82908591482452	0.973457826372455\\
0.834258750592667	0.973153276512821\\
0.839431586360813	0.972847340535491\\
0.84460442212896	0.97254002439575\\
0.849799655005291	0.972229994507269\\
0.854994887881621	0.971918584491753\\
0.860190120757951	0.971605800354951\\
0.865385353634282	0.971291648093366\\
0.870603213699137	0.970974756540127\\
0.875821073763992	0.970656497014579\\
0.881038933828848	0.970336875563628\\
0.886256793893703	0.970015898224581\\
0.891497515445601	0.969692155822256\\
0.896738236997499	0.969367057802497\\
0.901978958549397	0.969040610252953\\
0.907219680101295	0.968712819251319\\
0.912483501408722	0.968382237200394\\
0.917747322716149	0.968050312087539\\
0.923011144023576	0.967717050040728\\
0.928274965331004	0.967382457177632\\
0.933562128954513	0.967045047044302\\
0.938849292578023	0.966706306605571\\
0.944136456201532	0.966366242029331\\
0.949423619825042	0.966024859472805\\
0.954734372782572	0.96568063317881\\
0.960045125740102	0.965335089537387\\
0.965355878697632	0.964988234755921\\
0.970666631655162	0.964640075030774\\
0.976001225418392	0.96428904485417\\
0.981335819181621	0.963936710490057\\
0.98667041294485	0.963583078184901\\
0.99200500670808	0.963228154173774\\
0.997363697280812	0.962870332745019\\
1.00272238785355	0.962511220491188\\
1.00808107842628	0.962150823697403\\
1.01343976899901	0.96178914863703\\
1.0188228172289	0.961424548919579\\
1.02420586545878	0.961058671942512\\
1.02958891368867	0.960691524029185\\
1.03497196191855	0.960323111490831\\
1.04037963341625	0.959951746787781\\
1.04578730491395	0.959579118594253\\
1.05119497641164	0.959205233271417\\
1.05660264790934	0.958830097167945\\
1.06203521338424	0.958451981100946\\
1.06746777885913	0.95807261551689\\
1.07290034433402	0.957692006814333\\
1.07833290980891	0.957310161378961\\
1.08379064506731	0.956925307889973\\
1.08924838032571	0.956539219062344\\
1.09470611558411	0.956151901331588\\
1.10016385084252	0.955763361119974\\
1.1056470369148	0.955371784464729\\
1.11113022298707	0.954978986855016\\
1.11661340905935	0.95458497476288\\
1.12209659513163	0.954189754646739\\
1.12760551858231	0.95379146937518\\
1.13311444203298	0.953391977739823\\
1.13862336548365	0.952991286248815\\
1.14413228893432	0.952589401396289\\
1.14966724184116	0.952184422356583\\
1.15520219474799	0.951778251751117\\
1.16073714765482	0.951370896123703\\
1.16627210056166	0.950962362003758\\
1.17183338076584	0.950550704327313\\
1.17739466097002	0.950137870091392\\
1.18295594117419	0.949723865875039\\
1.18851722137837	0.949308698242516\\
1.19410513262804	0.948890377336715\\
1.19969304387772	0.948470895086913\\
1.20528095512739	0.948050258106952\\
1.21086886637706	0.947628472995497\\
1.21648371859449	0.947203504527386\\
1.22209857081192	0.946777390140905\\
1.22771342302936	0.946350136484254\\
1.23332827524679	0.945921750190061\\
1.23897038457538	0.945490150086665\\
1.24461249390397	0.945057419701764\\
1.25025460323256	0.944623565717474\\
1.25589671256115	0.944188594799944\\
1.26156640151281	0.9437503792417\\
1.26723609046446	0.943311049251203\\
1.27290577941612	0.942870611544044\\
1.27857546836778	0.942429072819445\\
1.28427306633634	0.941984258205447\\
1.28997066430491	0.941538345221927\\
1.29566826227347	0.94109134061751\\
1.30136586024204	0.940643251124043\\
1.30709170327692	0.940191854094473\\
1.31281754631181	0.939739374972924\\
1.31854338934669	0.939285820540605\\
1.32426923238158	0.93883119756154\\
1.33002366371043	0.938373234962676\\
1.33577809503928	0.937914206765467\\
1.34153252636813	0.937454119783256\\
1.34728695769699	0.936992980811791\\
1.35307032782915	0.936528469693621\\
1.35885369796131	0.936062909688226\\
1.36463706809348	0.935596307640635\\
1.37042043822564	0.935128670377863\\
1.37622881553645	0.934657975984614\\
1.38203719284726	0.934186251151719\\
1.38784557015807	0.933713502740249\\
1.39365394746888	0.933239737592894\\
1.39948194569202	0.93276335702252\\
1.40530994391516	0.932285966581079\\
1.41113794213829	0.931807573125029\\
1.41696594036143	0.931328183492154\\
1.4228137248248	0.930846171922121\\
1.42866150928816	0.930363171135042\\
1.43450929375153	0.929879187981879\\
1.44035707821489	0.929394229294632\\
1.44622478964653	0.928906644386906\\
1.45209250107816	0.92841809100854\\
1.45796021250979	0.927928576004029\\
1.46382792394143	0.927438106198619\\
1.46971570462452	0.926945005989892\\
1.4756034853076	0.926450958146904\\
1.48149126599069	0.925955969506718\\
1.48737904667378	0.92546004688686\\
1.49328704060906	0.924961489768452\\
1.49919503454433	0.924462005939699\\
1.50510302847961	0.923961602229263\\
1.51101102241489	0.923460285445989\\
1.51693937480906	0.922956330200665\\
1.52286772720322	0.922451469254184\\
1.52879607959739	0.921945709425849\\
1.53472443199155	0.921439057514861\\
1.540673290528	0.920929763202963\\
1.54662214906446	0.920419584281743\\
1.55257100760091	0.919908527560181\\
1.55851986613736	0.919396599826883\\
1.56448937947106	0.918882025907562\\
1.57045889280476	0.918366588551248\\
1.57642840613847	0.917850294555645\\
1.58239791947216	0.917333150697807\\
1.58838823791446	0.916813356965573\\
1.59437855635675	0.91629272104683\\
1.60036887479904	0.915771249727047\\
1.60635919324133	0.915248949770777\\
1.61237046959671	0.914723996276506\\
1.61838174595208	0.914198221921806\\
1.62439302230746	0.913671633478968\\
1.63040429866284	0.913144237699095\\
1.63643668672632	0.912614184874213\\
1.6424690747898	0.912083332588473\\
1.64850146285328	0.911551687600039\\
1.65453385091676	0.911019256645618\\
1.66058750665404	0.910484165194027\\
1.66664116239132	0.909948295752223\\
1.67269481812859	0.909411655063296\\
1.67874847386587	0.908874249848618\\
1.68482355500326	0.908334180776512\\
1.69089863614064	0.907793355253628\\
1.69697371727803	0.90725178000704\\
1.70304879841541	0.906709461741847\\
1.70914546464501	0.906164476333955\\
1.71524213087462	0.905618756081197\\
1.72133879710422	0.905072307693695\\
1.72743546333383	0.904525137859337\\
1.73355387572031	0.903975297726551\\
1.73967228810678	0.90342474441914\\
1.74579070049326	0.902873484629332\\
1.75190911287974	0.902321525026874\\
1.75804943463327	0.901766892032226\\
1.76418975638681	0.901211567595197\\
1.77033007814035	0.900655558389191\\
1.77647039989388	0.900098871064878\\
1.78263279664118	0.899539507293116\\
1.78879519338848	0.898979473870868\\
1.79495759013578	0.898418777451786\\
1.80111998688309	0.89785742466654\\
1.80730462566597	0.897293392509355\\
1.81348926444885	0.896728712551129\\
1.81967390323173	0.896163391424831\\
1.82585854201462	0.895597435740203\\
1.83206559217232	0.895028797811552\\
1.83827264233002	0.894459533986568\\
1.84447969248772	0.893889650876613\\
1.85068674264543	0.893319155069582\\
1.85691637550678	0.892745974228522\\
1.86314600836813	0.89217218944891\\
1.86937564122948	0.891597807319578\\
1.87560527409083	0.891022834405654\\
1.88185766319929	0.890445173730876\\
1.88811005230775	0.889866931126182\\
1.89436244141622	0.889288113156957\\
1.90061483052468	0.888708726364643\\
1.90689015093446	0.888126649214884\\
1.91316547134424	0.887544012192636\\
1.91944079175402	0.886960821838916\\
1.92571611216381	0.886377084670567\\
1.93201454150218	0.885790654582187\\
1.93831297084055	0.885203686725287\\
1.94461140017892	0.884616187615602\\
1.9509098295173	0.884028163744463\\
1.95723154808653	0.883437444416723\\
1.96355326665576	0.882846209468744\\
1.96987498522499	0.882254465390071\\
1.97619670379422	0.881662218645617\\
1.98254189319299	0.881067274064919\\
1.98888708259176	0.8804718360546\\
1.99523227199053	0.8798759110771\\
2.0015774613893	0.879279505570005\\
2.00794630624256	0.878680399844413\\
2.01431515109582	0.878080822919905\\
2.02068399594908	0.877480781230912\\
2.02705284080234	0.87688028118679\\
2.03344552746868	0.876277078662835\\
2.03983821413503	0.875673427208739\\
2.04623090080137	0.875069333230017\\
2.05262358746771	0.874464803106895\\
2.05904030495676	0.873857568279646\\
2.06545702244581	0.873249906826959\\
2.07187373993486	0.872641825124534\\
2.07829045742391	0.872033329522565\\
2.084731396657	0.871422127101592\\
2.0911723358901	0.870810520393796\\
2.0976132751232	0.870198515744161\\
2.1040542143563	0.869586119471951\\
2.11051956957009	0.868971014244692\\
2.11698492478388	0.86835552710096\\
2.12345027999767	0.867739664354123\\
2.12991563521146	0.867123432291626\\
2.13640560238783	0.866504489268589\\
2.1428955695642	0.865885186729205\\
2.14938553674057	0.86526553095433\\
2.15587550391694	0.864645528198697\\
2.16239028199053	0.86402281249549\\
2.16890506006412	0.86339975970375\\
2.17541983813771	0.862776376070934\\
2.1819346162113	0.862152667818169\\
2.18847440630977	0.861526244724592\\
2.19501419640825	0.86089950699601\\
2.20155398650673	0.860272460845585\\
2.20809377660521	0.859645112459953\\
2.21465878275268	0.859015047369772\\
2.22122378890015	0.858384690121788\\
2.22778879504762	0.857754046893985\\
2.23435380119508	0.857123123837621\\
2.24094422983074	0.856489482291653\\
2.24753465846639	0.855855571086797\\
2.25412508710205	0.855221396364966\\
2.2607155157377	0.854586964241162\\
2.2673315759421	0.853949811902911\\
2.27394763614649	0.853312412424424\\
2.28056369635089	0.852674771910661\\
2.28717975655528	0.852036896439482\\
2.29382166035897	0.851396299062769\\
2.30046356416266	0.850755477082234\\
2.30710546796636	0.850114436565001\\
2.31374737177005	0.849473183550909\\
2.32041533418655	0.848829206974073\\
2.32708329660306	0.848185028345636\\
2.33375125901957	0.847540653694008\\
2.34041922143607	0.84689608902013\\
2.34711345989709	0.846248799218243\\
2.3538076983581	0.845601329930884\\
2.36050193681912	0.844953687146869\\
2.36719617528013	0.844305876827369\\
2.37391691043736	0.843655339832711\\
2.3806376455946	0.84300464593069\\
2.38735838075183	0.842353801069648\\
2.39407911590907	0.841702811170111\\
2.40082657131283	0.841049093101179\\
2.40757402671659	0.840395240713073\\
2.41432148212035	0.839741259912795\\
2.42106893752412	0.839087156579352\\
2.42784333963791	0.838430323637663\\
2.4346177417517	0.837773378973203\\
2.4413921438655	0.837116328450752\\
2.44816654597929	0.836459177906931\\
2.4549681243632	0.835799296357634\\
2.4617697027471	0.83513932568831\\
2.46857128113101	0.834479271720648\\
2.47537285951492	0.833819140248015\\
2.48220184698832	0.833156276402617\\
2.48903083446173	0.832493346044427\\
2.49585982193513	0.831830354951174\\
2.50268880940853	0.831167308872106\\
2.50954544177924	0.830501529113467\\
2.51640207414995	0.829835705451935\\
2.52325870652066	0.829169843620411\\
2.53011533889137	0.828503949323157\\
2.53699985562036	0.827835320039864\\
2.54388437234935	0.827166669464419\\
2.55076888907833	0.826498003284031\\
2.55765340580732	0.825829327157112\\
2.5645660494942	0.825157914792508\\
2.57147869318108	0.824486503745546\\
2.57839133686796	0.823815099656875\\
2.58530398055484	0.823143708138199\\
2.59224499685821	0.822469579197431\\
2.59918601316159	0.821795474181367\\
2.60612702946496	0.821121398683236\\
2.61306804576833	0.820447358267167\\
2.62003768421203	0.819770579238348\\
2.62700732265572	0.819093846736802\\
2.63397696109942	0.81841716630747\\
2.64094659954311	0.817740543466054\\
2.64794511298781	0.817061180871264\\
2.65494362643251	0.816381887400072\\
2.6619421398772	0.815702668548274\\
2.6689406533219	0.815023529782285\\
2.67596829852112	0.814341650123481\\
2.68299594372033	0.813659862176636\\
2.69002358891955	0.812978171387543\\
2.69705123411877	0.812296583172472\\
2.70410827073433	0.811612253027123\\
2.71116530734989	0.810928037172401\\
2.71822234396545	0.810243941003232\\
2.72527938058101	0.809559969884889\\
2.73236607296869	0.808873255733022\\
2.73945276535637	0.808186678439089\\
2.74653945774405	0.807500243346293\\
2.75362615013173	0.806813955768054\\
2.76074276611797	0.806124924111534\\
2.76785938210422	0.805436051867213\\
2.77497599809046	0.804747344325712\\
2.78209261407671	0.804058806747749\\
2.78923942490227	0.803367524116362\\
2.79638623572783	0.802676423436714\\
2.80353304655339	0.801985509945993\\
2.81067985737896	0.801294788851356\\
2.81785713902212	0.800601321675691\\
2.82503442066529	0.799908058974983\\
2.83221170230846	0.799215005932127\\
2.83938898395163	0.798522167699872\\
2.84659701614953	0.79782658240737\\
2.85380504834743	0.797131224095086\\
2.86101308054533	0.79643609789077\\
2.86822111274323	0.795741208891911\\
2.87546017926332	0.795043571881988\\
2.88269924578341	0.794346184337981\\
2.88993831230351	0.793649051331642\\
2.8971773788236	0.792952177904349\\
2.90444776780364	0.792252555517296\\
2.91171815678368	0.791553205060724\\
2.91898854576373	0.790854131549533\\
2.92625893474377	0.79015533996814\\
2.93356093845992	0.789453798509436\\
2.94086294217606	0.788752551423069\\
2.94816494589221	0.788051603666234\\
2.95546694960836	0.78735096016554\\
2.96280086487659	0.786647565869403\\
2.97013478014483	0.785944488363219\\
2.97746869541306	0.785241732545624\\
2.9848026106813	0.784539303284569\\
2.99216873909201	0.783834122294057\\
2.99953486750271	0.783129280485378\\
3.00690099591342	0.782424782697757\\
3.01426712432413	0.781720633739637\\
3.02166577186631	0.78101373214552\\
3.02906441940849	0.780307192097861\\
3.03646306695067	0.779601018375624\\
3.04386171449286	0.778895215726897\\
3.05129319192691	0.778186659534936\\
3.05872466936097	0.777478487225335\\
3.06615614679503	0.776770703515943\\
3.07358762422909	0.776063313093643\\
3.08105224696524	0.775353168239508\\
3.0885168697014	0.774643429573437\\
3.09598149243755	0.773934101751308\\
3.1034461151737	0.773225189397955\\
3.11094420406679	0.772513521675209\\
3.11844229295988	0.771802282414654\\
3.12594038185297	0.771091476209347\\
3.13343847074605	0.770381107621222\\
3.14097035080817	0.769667982807648\\
3.14850223087028	0.768955308697303\\
3.1560341109324	0.768243089819572\\
3.16356599099451	0.767531330672637\\
3.17113199342146	0.766816814342076\\
3.17869799584842	0.766102770921425\\
3.18626399827537	0.765389204875545\\
3.19383000070233	0.764676120638016\\
3.20143046110215	0.763960278332929\\
3.20903092150198	0.763244931108609\\
3.21663138190181	0.762530083364533\\
3.22423184230164	0.761815739468832\\
3.2318671021242	0.761098636569824\\
3.23950236194676	0.760382050885325\\
3.24713762176933	0.759665986748573\\
3.25477288159189	0.7589504484614\\
3.26244328804483	0.758232150200839\\
3.27011369449776	0.75751439125013\\
3.2777841009507	0.75679717587542\\
3.28545450740363	0.756080508311385\\
3.29316041266953	0.75536107985232\\
3.30086631793543	0.754642212758665\\
3.30857222320133	0.753923911228615\\
3.31627812846722	0.753206179428843\\
3.32401989110489	0.752485685739544\\
3.33176165374256	0.751765775430242\\
3.33950341638023	0.751046452630323\\
3.3472451790179	0.750327721437601\\
3.35502316313618	0.749606227375267\\
3.36280114725446	0.748885338665249\\
3.37057913137275	0.748165059367265\\
3.37835711549103	0.747445393509412\\
3.38617169154667	0.746722963752486\\
3.3939862676023	0.746001161276752\\
3.40180084365794	0.745279990071399\\
3.40961541971357	0.744559454093951\\
3.41746696420648	0.743836153177615\\
3.42531850869938	0.743113501426689\\
3.43317005319229	0.742391502758969\\
3.4410215976852	0.741670161060549\\
3.44891049346576	0.740946053355649\\
3.45679938924632	0.740222616654555\\
3.46468828502688	0.739499854802804\\
3.47257718080744	0.738777771614198\\
3.48050381756295	0.73805292129056\\
3.48843045431847	0.737328763762218\\
3.49635709107398	0.736605302801585\\
3.50428372782949	0.735882542149308\\
3.51224850159594	0.73515701322896\\
3.52021327536239	0.734432198847334\\
3.52817804912884	0.733708102702848\\
3.53614282289529	0.732984728462129\\
3.54414613607264	0.732258584825989\\
3.55214944924998	0.731533177422762\\
3.56015276242733	0.730808509876001\\
3.56815607560468	0.730084585777446\\
3.57619833889781	0.72935789099822\\
3.58424060219095	0.728631954096055\\
3.59228286548408	0.727906778618762\\
3.60032512877721	0.727182368082326\\
3.60840675922679	0.726455185613559\\
3.61648838967638	0.725728782614785\\
3.62457002012596	0.725003162557196\\
3.63265165057554	0.724278328880145\\
3.64077307248204	0.723550721981902\\
3.64889449438855	0.722823916094355\\
3.65701591629505	0.722097914611197\\
3.66513733820155	0.721372720894277\\
3.67329898353354	0.720644752607011\\
3.68146062886554	0.719917606817982\\
3.68962227419754	0.719191286842499\\
3.69778391952953	0.718465795964026\\
3.70598622860803	0.717737529059062\\
3.71418853768653	0.717010106085913\\
3.72239084676502	0.716283530280616\\
3.73059315584352	0.715557804847366\\
3.73883657626523	0.714829301934755\\
3.74707999668693	0.714101664332548\\
3.75532341710863	0.713374895196619\\
3.76356683753034	0.712648997651008\\
3.77185182516863	0.711920321102192\\
3.78013681280692	0.711192531186549\\
3.78842180044522	0.710465630978892\\
3.79670678808351	0.709739623522217\\
3.80503380745825	0.70901083545052\\
3.81336082683299	0.70828295527818\\
3.82168784620774	0.707555985998053\\
3.83001486558248	0.706829930571192\\
3.83838438958511	0.706101092869494\\
3.84675391358775	0.705373184275923\\
3.85512343759039	0.704646207700474\\
3.86349296159302	0.703920166021355\\
3.8719054722877	0.703191340305988\\
3.88031798298237	0.702463464849438\\
3.88873049367705	0.701736542477923\\
3.89714300437172	0.701010575985907\\
3.90559899284027	0.700281823625146\\
3.91405498130882	0.699554042615092\\
3.92251096977737	0.698827235697274\\
3.93096695824593	0.698101405581498\\
3.93946692479213	0.697372787692951\\
3.94796689133833	0.696645162187517\\
3.95646685788454	0.695918531721111\\
3.96496682443074	0.695192898917967\\
3.97351127881372	0.694464476364123\\
3.98205573319669	0.693737067165656\\
3.99060018757967	0.693010673891942\\
3.99914464196265	0.692285299080722\\
4.00773410425464	0.691557132412313\\
4.01632356654662	0.690830000010934\\
4.02491302883861	0.69010390435849\\
4.0335024911306	0.689378847905297\\
4.04213749081375	0.688650997455529\\
4.05077249049689	0.687924202123123\\
4.05940749018004	0.687198464301571\\
4.06804248986318	0.686473786352833\\
4.07672356764034	0.685746312101763\\
4.08540464541749	0.68501991375677\\
4.09408572319465	0.684294593621986\\
4.10276680097181	0.683570353970069\\
4.11149450824228	0.682843315608739\\
4.12022221551276	0.682117373880135\\
4.12894992278323	0.681392530998076\\
4.1376776300537	0.680668789144971\\
4.14645252883691	0.679942246100985\\
4.15522742762012	0.679216820353817\\
4.16400232640333	0.678492514026003\\
4.17277722518654	0.677769329208746\\
4.18159988926615	0.67704334057128\\
4.19042255334575	0.676318489831919\\
4.19924521742536	0.675594779020949\\
4.20806788150496	0.674872210137401\\
4.21693889629988	0.674146834689742\\
4.22580991109479	0.673422617678369\\
4.23468092588971	0.672699561040337\\
4.24355194068463	0.671977666681527\\
4.2524719033702	0.671252962912932\\
4.26139186605577	0.670529438055383\\
4.27031182874135	0.669807093951715\\
4.27923179142692	0.669085932413676\\
4.28820131218307	0.668361958440693\\
4.29717083293922	0.667639183790073\\
4.30614035369538	0.666917610209429\\
4.31510987445153	0.666197239415386\\
4.32412957614907	0.665474053032258\\
4.33314927784661	0.664752086319221\\
4.34216897954416	0.664031340927657\\
4.3511886812417	0.663311818478058\\
4.36025919971018	0.662589477159837\\
4.36932971817866	0.661868375795724\\
4.37840023664715	0.661148515939847\\
4.38747075511563	0.660429899115553\\
4.39659274002011	0.659708459973901\\
4.40571472492458	0.658988281006729\\
4.41483670982906	0.658269363669879\\
4.42395869473353	0.657551709388526\\
4.43313280988389	0.656831229174858\\
4.44230692503424	0.656112029292501\\
4.45148104018459	0.655394111097964\\
4.46065515533495	0.65467747591721\\
4.469882079136	0.653958011015836\\
4.47910900293706	0.653239846539235\\
4.48833592673811	0.652522983743521\\
4.49756285053916	0.651807423854397\\
4.50684327607396	0.651089030305455\\
4.51612370160877	0.650371957211509\\
4.52540412714357	0.649656205727214\\
4.53468455267837	0.64894177697694\\
4.54401918924822	0.648224510387655\\
4.55335382581807	0.647508584220866\\
4.56268846238792	0.646793999528674\\
4.57202309895777	0.646080757333042\\
4.58141267108196	0.645364672987697\\
4.59080224320616	0.644649948969775\\
4.60019181533036	0.643936586227725\\
4.60958138745456	0.643224585680008\\
4.61902663678989	0.642509738425537\\
4.62847188612523	0.641796271341481\\
4.63791713546056	0.641084185271521\\
4.6473623847959	0.640373481029502\\
4.65686406984621	0.639659925333397\\
4.66636575489653	0.638947769589242\\
4.67586743994685	0.638237014534813\\
4.68536912499717	0.637527660878213\\
4.69492802222378	0.636815450780957\\
4.7044869194504	0.636104660356449\\
4.71404581667701	0.635395290235407\\
4.72360471390363	0.634687341019049\\
4.73320764525141	0.633977561870872\\
4.74281057659919	0.633269217875967\\
4.75241350794698	0.632562309554615\\
4.76201643929476	0.631856837397945\\
4.77165794452804	0.63114997673488\\
4.78129944976131	0.630444564701839\\
4.79094095499459	0.629740601707849\\
4.80058246022786	0.629038088133207\\
4.81026314694842	0.628334178280249\\
4.81994383366897	0.627631730332798\\
4.82962452038952	0.62693074458904\\
4.83930520711008	0.626231221318854\\
4.84902565638322	0.625530296559525\\
4.85874610565637	0.624830846772469\\
4.86846655492951	0.624132872145415\\
4.87818700420266	0.623436372838212\\
4.88794780523678	0.622738466858768\\
4.89770860627089	0.622042048712426\\
4.90746940730501	0.621347118476831\\
4.91723020833913	0.620653676202173\\
4.92703195872998	0.619958822099275\\
4.93683370912084	0.619265468487111\\
4.9466354595117	0.618573615333589\\
4.95643720990256	0.617883262579598\\
4.96628051524823	0.617191492912449\\
4.97612382059389	0.616501236192955\\
4.98596712593956	0.615812492279625\\
4.99581043128522	0.615125261004382\\
5.00569590732134	0.614436607669413\\
5.01558138335745	0.613749479541328\\
5.02546685939356	0.613063876369559\\
5.03535233542967	0.612379797877382\\
5.04528060635073	0.611694292253228\\
5.05520887727179	0.61101032389988\\
5.06513714819285	0.61032789245799\\
5.07506541911391	0.609646997542488\\
5.08503711889176	0.608964670423268\\
5.0950088186696	0.608283892446121\\
5.10498051844744	0.607604663143204\\
5.11495221822528	0.606926982021391\\
5.12496799094167	0.60624786362098\\
5.13498376365806	0.60557030604389\\
5.14499953637444	0.60489430871405\\
5.15501530909083	0.604219871030544\\
5.16507580953502	0.60354399096966\\
5.17513630997921	0.602869683226017\\
5.18519681042339	0.602196947115565\\
5.19525731086758	0.601525781929851\\
5.2053632039849	0.600853169301093\\
5.21546909710223	0.600182140298557\\
5.22557499021955	0.599512694130444\\
5.23568088333688	0.598844829981\\
5.24583284576096	0.598175513273928\\
5.25598480818505	0.597507791319854\\
5.26613677060914	0.596841663219445\\
5.27628873303322	0.596177128049862\\
5.28648745229033	0.595511135232978\\
5.29668617154743	0.594846748116014\\
5.30688489080453	0.594183965692302\\
5.31708361006163	0.593522786932119\\
5.32732978639303	0.592860145362031\\
5.33757596272443	0.592199120261724\\
5.34782213905582	0.591539710517373\\
5.35806831538722	0.590881914992549\\
5.36836266133408	0.590222651472431\\
5.37865700728095	0.589565015017411\\
5.38895135322782	0.588909004406668\\
5.39924569917468	0.588254618397236\\
5.40958894010065	0.587598759173592\\
5.41993218102661	0.586944537438368\\
5.43027542195257	0.586291951863889\\
5.44061866287853	0.585641001100793\\
5.45101153706255	0.584988571888403\\
5.46140441124658	0.584337790418185\\
5.4717972854306	0.583688655255738\\
5.48219015961463	0.583041164945433\\
5.49263341945276	0.582392190883047\\
5.50307667929089	0.581744874649685\\
5.51351993912902	0.581099214704323\\
5.52396319896715	0.580455209485175\\
5.53445761187213	0.579809715113556\\
5.54495202477712	0.579165888493712\\
5.5554464376821	0.578523727978082\\
5.56594085058709	0.577883231898814\\
5.57648719831941	0.577241241236759\\
5.58703354605173	0.576600928087563\\
5.59757989378405	0.575962290697196\\
5.60812624151637	0.575325327291811\\
5.61872532103302	0.57468686381748\\
5.62932440054967	0.574050087457974\\
5.63992348006633	0.573414996352843\\
5.65052255958298	0.572781588622291\\
5.66117518401188	0.572146675249442\\
5.67182780844078	0.571513458436962\\
5.68248043286967	0.570881936218005\\
5.69313305729857	0.570252106606864\\
5.70384005671797	0.569620765675394\\
5.71454705613736	0.568991130596194\\
5.72525405555675	0.568363199296029\\
5.73596105497615	0.567736969683289\\
5.74672327665992	0.56710922298296\\
5.75748549834369	0.566483191275844\\
5.76824772002746	0.565858872382303\\
5.77900994171123	0.565236264104812\\
5.78982825023888	0.564612132906756\\
5.80064655876654	0.563989725694445\\
5.81146486729419	0.563369040181802\\
5.82228317582184	0.562750074065354\\
5.83315845473553	0.562129579064882\\
5.84403373364921	0.56151081689717\\
5.8549090125629	0.560893785169639\\
5.86578429147658	0.560278481472818\\
5.87671744355088	0.559661642815702\\
5.88765059562518	0.559046545695632\\
5.89858374769948	0.558433187613452\\
5.90951689977378	0.55782156605361\\
5.92050884813521	0.557208403315011\\
5.93150079649664	0.556596990677975\\
5.94249274485807	0.55598732553666\\
5.9534846932195	0.555379405269335\\
5.96453638108293	0.554769937512084\\
5.97558806894636	0.554162228283816\\
5.98663975680979	0.55355627487187\\
5.99769144467322	0.552952074548213\\
6.00880383710542	0.552346320268613\\
6.01991622953761	0.551742332811344\\
6.0310286219698	0.551140109356774\\
6.042141014402	0.550539647070414\\
6.05331509935885	0.549937624188667\\
6.0644891843157	0.549337376291671\\
6.07566326927256	0.54873890045264\\
6.08683735422941	0.548142193730456\\
6.09807414230094	0.547543919652519\\
6.10931093037248	0.546947428593664\\
6.12054771844401	0.546352717519744\\
6.13178450651555	0.545759783382808\\
6.14308503319571	0.545165274927378\\
6.15438555987588	0.544572557400604\\
6.16568608655605	0.543981627660739\\
6.17698661323621	0.543392482552768\\
6.18835193888863	0.542801756005152\\
6.19971726454106	0.542212828174083\\
6.21108259019348	0.541625695809954\\
6.2224479158459	0.541040355650427\\
6.23387912710053	0.540453426741531\\
6.24531033835517	0.539868304218671\\
6.2567415496098	0.539284984724079\\
6.26817276086443	0.538703464887806\\
6.2796709718506	0.538120348785283\\
6.29116918283678	0.537539046623281\\
6.30266739382295	0.536959554935547\\
6.31416560480912	0.536381870244199\\
6.32573195844141	0.535802581544103\\
6.33729831207371	0.535225114227549\\
6.34886466570601	0.534649464719442\\
6.36043101933831	0.534075629433613\\
6.37206668825559	0.533500182172472\\
6.38370235717287	0.532926563629986\\
6.39533802609016	0.532354770121821\\
6.40697369500744	0.531784797953136\\
6.41867988207414	0.531213205643259\\
6.43038606914084	0.53064344928266\\
6.44209225620754	0.530075525077334\\
6.45379844327423	0.529509429223351\\
6.46557638449348	0.528941704772243\\
6.47735432571273	0.528375823400569\\
6.48913226693197	0.52781178120419\\
6.50091020815122	0.527249574269626\\
6.51276117329228	0.526685730017113\\
6.52461213843333	0.526123735877627\\
6.53646310357439	0.52556358783639\\
6.54831406871544	0.525005281869872\\
6.56023936237899	0.524445329605161\\
6.57216465604253	0.523887234394405\\
6.58408994970607	0.52333099211164\\
6.59601524336962	0.522776598622751\\
6.608016207688	0.522220549527874\\
6.62001717200638	0.521666364339601\\
6.63201813632476	0.521114038820198\\
6.64401910064313	0.520563568724385\\
6.65609711612489	0.520011433410315\\
6.66817513160664	0.519461168771509\\
6.68025314708839	0.518912770457824\\
6.69233116257014	0.518366234112191\\
6.70448764976455	0.517818022619012\\
6.71664413695897	0.517271688490175\\
6.72880062415338	0.516727227262451\\
6.7409571113478	0.51618463446631\\
6.75319353351478	0.515640356222625\\
6.76542995568177	0.515097961957613\\
6.77766637784875	0.51455744709423\\
6.78990280001574	0.514018807049766\\
6.80222066442117	0.51347847089976\\
6.8145385288266	0.51294002527282\\
6.82685639323203	0.512403465477304\\
6.83917425763746	0.51186878681655\\
6.85157511856051	0.511332400975376\\
6.86397597948357	0.510797912137007\\
6.87637684040662	0.510265315494362\\
6.88877770132968	0.509734606236002\\
6.90126316148383	0.509202178321071\\
6.91374862163799	0.508671653829272\\
6.92623408179214	0.508143027837193\\
6.93871954194629	0.507616295417732\\
6.95129125486549	0.507087832441891\\
6.96386296778468	0.506561279255577\\
6.97643468070388	0.506036630818091\\
6.98900639362307	0.505513882085723\\
7.00166606764739	0.504989390390035\\
7.0143257416717	0.504466814802508\\
7.02698541569601	0.503946150164131\\
7.03964508972033	0.503427391313578\\
7.05239448925109	0.502906876622737\\
7.06514388878186	0.50238828431692\\
7.07789328831262	0.501871609117716\\
7.09064268784339	0.501356845745102\\
7.10348363709426	0.50084031312386\\
7.11632458634514	0.500325709129281\\
7.12916553559601	0.499813028362385\\
7.14200648484689	0.499302265423306\\
7.15494087138423	0.498789719251282\\
7.16787525792157	0.498279107919324\\
7.18080964445891	0.497770425906642\\
7.19374403099625	0.497263667692296\\
7.20677380774895	0.496755111707422\\
7.21980358450165	0.496248496754719\\
7.23283336125435	0.495743817190264\\
7.24586313800706	0.495241067370731\\
7.25899032889946	0.494736504576989\\
7.27211751979186	0.494233888994155\\
7.28524471068426	0.493733214853756\\
7.29837190157667	0.493234476388685\\
7.31159860422458	0.492733909090702\\
7.32482530687249	0.492235295176985\\
7.3380520095204	0.491738628753006\\
7.35127871216831	0.491243903926382\\
7.36460710304072	0.490747333677484\\
7.37793549391314	0.490252722989572\\
7.39126388478556	0.489760065840451\\
7.40459227565797	0.489269356210873\\
7.41802461355885	0.48877678383618\\
7.43145695145972	0.488286177211486\\
7.4448892893606	0.487797530185215\\
7.45832162726148	0.487310836609552\\
7.47186025959003	0.486822262136268\\
7.48539889191859	0.486335659624179\\
7.49893752424715	0.48585102279049\\
7.51247615657571	0.485368345357012\\
7.52612352432791	0.484883768005946\\
7.53977089208011	0.484401168859854\\
7.55341825983232	0.483920541502767\\
7.56706562758452	0.483441879524182\\
7.58082427117841	0.482961297684313\\
7.5945829147723	0.482482700336869\\
7.60834155836619	0.48200608093062\\
7.62210020196009	0.481531432920675\\
7.63597276754062	0.48105484412145\\
7.64984533312116	0.480580246157616\\
7.66371789870169	0.480107632340439\\
7.67759046428223	0.479636995988438\\
7.69157971106386	0.479164396852409\\
7.70556895784549	0.478693794963049\\
7.71955820462712	0.478225183491735\\
7.73354745140875	0.477758555618023\\
7.7476562591015	0.477289941829037\\
7.76176506679424	0.476823331780005\\
7.77587387448699	0.476358718499859\\
7.78998268217974	0.475896095026663\\
7.80421406003167	0.475431461263376\\
7.81844543788359	0.474968837830288\\
7.83267681573552	0.474508217611142\\
7.84690819358745	0.474049593499795\\
7.86126528820976	0.473588933430977\\
7.87562238283208	0.473130290395301\\
7.8899794774544	0.472673657128383\\
7.90433657207672	0.472219026376971\\
7.91882267896367	0.471762332558533\\
7.93330878585062	0.471307662606341\\
7.94779489273757	0.470855009104734\\
7.96228099962452	0.470404364650225\\
7.97689957365268	0.46995162848338\\
7.99151814768085	0.469500923164797\\
8.00613672170902	0.469052241124153\\
8.02075529573718	0.468605574804373\\
8.03550996276884	0.468156786485537\\
8.0502646298005	0.467710036166098\\
8.06501929683216	0.467265316117429\\
8.07977396386382	0.466822618625264\\
8.09466853533245	0.466377767035067\\
8.10956310680107	0.465934960786846\\
8.12445767826969	0.465494191989749\\
8.13935224973832	0.465055452768434\\
8.15439073744049	0.464614525391446\\
8.16942922514267	0.464175650914765\\
8.18446771284484	0.463738821281083\\
8.19950620054702	0.463304028449798\\
8.2146928332504	0.462867011278302\\
8.22987946595378	0.462432054807782\\
8.24506609865717	0.461999150809912\\
8.26025273136055	0.461568291074308\\
8.27559197362809	0.461135168490444\\
8.29093121589562	0.460704114679812\\
8.30627045816316	0.46027512123813\\
8.32160970043069	0.45984817978034\\
8.33710627308224	0.459418934439519\\
8.35260284573378	0.458991766247864\\
8.36809941838533	0.45856666661979\\
8.38359599103687	0.458143626990268\\
8.39925489573158	0.457718239643065\\
8.41491380042629	0.457294938160408\\
8.430572705121	0.456873713769596\\
8.44623160981571	0.456454557719881\\
8.46205815432056	0.456033007077208\\
8.4778846988254	0.455613551393195\\
8.49371124333024	0.455196181701716\\
8.50953778783509	0.454780889060047\\
8.52553761739547	0.454363151561012\\
8.54153744695585	0.453947518537578\\
8.55753727651623	0.453533980823309\\
8.57353710607661	0.453122529276688\\
8.58971623619016	0.452708578899271\\
8.60589536630372	0.452296742986063\\
8.62207449641727	0.451887012162801\\
8.63825362653082	0.451479377081726\\
8.65461848330403	0.451069185044272\\
8.67098334007723	0.450661117986428\\
8.68734819685043	0.450255166317864\\
8.70371305362364	0.449851320476425\\
8.72018188980809	0.449447024964239\\
8.73665072599255	0.449044842781063\\
8.75311956217701	0.448644764268001\\
8.76958839836146	0.448246779795449\\
8.78614348346302	0.447848811865672\\
8.80269856856458	0.447452940544302\\
8.81925365366614	0.447059156139383\\
8.83580873876771	0.446667448989224\\
8.85245111606455	0.446275760421256\\
8.86909349336139	0.445886151585091\\
8.88573587065824	0.445498612758089\\
8.90237824795508	0.445113134248834\\
8.91910883699713	0.444727679511298\\
8.93583942603918	0.444344287445102\\
8.95257001508123	0.443962948299509\\
8.96930060412328	0.443583652355954\\
8.98612034063892	0.443204385147681\\
9.00294007715455	0.442827163374531\\
9.01975981367019	0.442451977260227\\
9.03657955018582	0.442078817061588\\
9.05348938408815	0.441705690392878\\
9.07039921799048	0.441334591754763\\
9.08730905189281	0.440965511347948\\
9.10421888579513	0.440598439407148\\
9.12121978296026	0.440231405592967\\
9.13822068012539	0.439866382244258\\
9.15522157729052	0.439503359541193\\
9.17222247445565	0.43914232769886\\
9.18931541587169	0.438781338411493\\
9.20640835728774	0.438422341871214\\
9.22350129870378	0.438065328240123\\
9.24059424011982	0.437710287716121\\
9.25778022278737	0.437355293998283\\
9.27496620545493	0.437002275163275\\
9.29215218812249	0.436651221357553\\
9.30933817079004	0.43630212276424\\
9.32661820779546	0.435953075060328\\
9.34389824480088	0.435605984236351\\
9.3611782818063	0.435260840425513\\
9.37845831881172	0.434917633798541\\
9.39583344002363	0.434574481972756\\
9.41320856123555	0.434233268892598\\
9.43058368244747	0.43389398468038\\
9.44795880365938	0.433556619496778\\
9.46543005570071	0.433219312866005\\
9.48290130774205	0.432883926722184\\
9.50037255978338	0.432550451179065\\
9.51784381182471	0.432218876389587\\
9.53541225878365	0.431887363741214\\
9.55298070574258	0.431557753203783\\
9.57054915270152	0.431230034884777\\
9.58811759966045	0.43090419893168\\
9.60578432346384	0.4305784285484\\
9.62345104726722	0.430254541789655\\
9.64111777107061	0.429932528758923\\
9.658784494874	0.42961237960048\\
9.67655059520445	0.429292299293136\\
9.6943166955349	0.428974084020266\\
9.71208279586536	0.428657723883576\\
9.72984889619581	0.428343209026352\\
9.74771549159443	0.428028766144659\\
9.76558208699305	0.427716169610532\\
9.78344868239167	0.427405409526102\\
9.80131527779029	0.427096476035844\\
9.81928350543996	0.426787617503365\\
9.83725173308963	0.426480586541273\\
9.8552199607393	0.42617537325429\\
9.87318818838897	0.425871967790234\\
9.89125920482513	0.425568640122152\\
9.90933022126129	0.425267121163556\\
9.92740123769745	0.424967401023893\\
9.94547225413361	0.424669469856441\\
9.96364723590061	0.424371619178814\\
9.9818222176676	0.42407555827255\\
9.99999719943459	0.42378127725392\\
10.0181721812016	0.423488766283749\\
10.0364523247657	0.423196338365013\\
10.0547324683298	0.422905681208596\\
10.0730126118939	0.422616784939662\\
10.0912927554579	0.422329639728633\\
10.1096792783833	0.42204257999008\\
10.1280658013086	0.421757271940229\\
10.146452324234	0.421473705715172\\
10.1648388471593	0.421191871496949\\
10.1833329877765	0.420910125047628\\
10.2018271283938	0.42063011115492\\
10.220321269011	0.420351819967846\\
10.2388154096282	0.420075241682056\\
10.257418428294	0.419798753328155\\
10.2760214469598	0.419523978346462\\
10.2946244656257	0.419250906900897\\
10.3132274842915	0.418979529202675\\
10.3319406634311	0.418708243476107\\
10.3506538425707	0.418438651891003\\
10.3693670217104	0.418170744628118\\
10.38808020085	0.417904511916153\\
10.4069048454226	0.417638373096052\\
10.4257294899953	0.417373909146202\\
10.4445541345679	0.417111110266104\\
10.4633787791406	0.416849966703834\\
10.4823162175897	0.416588918831171\\
10.5012536560388	0.416329526522942\\
10.5201910944879	0.416071779999262\\
10.539128532937	0.415815669529443\\
10.5581801172831	0.415559656433912\\
10.5772317016292	0.415305279568075\\
10.5962832859753	0.415052529174506\\
10.6153348703214	0.414801395545576\\
10.6345019773376	0.414550360857284\\
10.6536690843538	0.414300943040726\\
10.67283619137	0.414053132362738\\
10.6920032983861	0.413806919140549\\
10.7112873293394	0.413560806321096\\
10.7305713602926	0.413316290997656\\
10.7498553912458	0.413073363463107\\
10.769139422199	0.412832014061296\\
10.7885418041401	0.412590766413866\\
10.8079441860811	0.41235109687446\\
10.8273465680222	0.412112995763741\\
10.8467489499633	0.411876453453904\\
10.8662711360792	0.411640014145958\\
10.8857933221952	0.411405133551131\\
10.9053155083112	0.411171802019583\\
10.9248376944271	0.410940009953557\\
10.9444811647161	0.410708322035114\\
10.9641246350051	0.410478173433237\\
10.9837681052941	0.410249554529263\\
11.0034115755831	0.410022455757151\\
11.02317783773	0.409795462176752\\
11.0429440998769	0.409569988519928\\
11.0627103620238	0.409346025200845\\
11.0824766241707	0.40912356268681\\
11.102367213876	0.408901206313887\\
11.1222578035813	0.408680350480175\\
11.1421483932866	0.40846098563429\\
11.1620389829919	0.40824310227849\\
11.182055464688	0.408025325920805\\
11.202071946384	0.40780903073158\\
11.22208842808	0.407594207195461\\
11.2421049097761	0.407380845851233\\
11.2622488777205	0.407167592268307\\
11.282392845665	0.406955800501642\\
11.3025368136094	0.406745461073474\\
11.3226807815539	0.406536564560655\\
11.3429428732039	0.406327889246372\\
11.363204964854	0.40612065486154\\
11.3834670565041	0.405914851982799\\
11.4037291481542	0.405710471241749\\
11.4240073959591	0.405507342047742\\
11.444285643764	0.405305618650616\\
11.464563891569	0.405105291824181\\
11.4848421393739	0.404906352396431\\
11.5051410232308	0.404708590901182\\
11.5254399070878	0.404512201401394\\
11.5457387909448	0.40431717485944\\
11.5660376748017	0.40412350229114\\
11.5863613245615	0.403930940931857\\
11.6066849743212	0.403739718989041\\
11.627008624081	0.403549827605878\\
11.6473322738407	0.403361257978285\\
11.6676846007299	0.403173738054894\\
11.6880369276191	0.402987526107524\\
11.7083892545083	0.402802613453035\\
11.7287415813975	0.402618991460345\\
11.7491263024508	0.402436362342851\\
11.7695110235041	0.402255010824245\\
11.7898957445574	0.402074928388481\\
11.8102804656107	0.401896106570923\\
11.8307011258767	0.401718224995468\\
11.8511217861426	0.401541591637467\\
11.8715424464086	0.401366198141881\\
11.8919631066745	0.401192036204458\\
11.9124230970895	0.401018765667072\\
11.9328830875046	0.400846714900237\\
11.9533430779196	0.40067587570428\\
11.9738030683347	0.400506239929714\\
11.9943056417768	0.400337450145656\\
12.014808215219	0.400169852564315\\
12.0353107886612	0.400003439136134\\
12.0558133621033	0.399838201861167\\
12.0763616495731	0.399673768267869\\
12.0969099370428	0.399510500138296\\
12.1174582245125	0.39934838956812\\
12.1380065119823	0.399187428702065\\
12.1586035338403	0.399027232042069\\
12.1792005556983	0.39886817488969\\
12.1997975775563	0.398710249481261\\
12.2203945994143	0.398553448101628\\
12.2410432774846	0.398397374031336\\
12.2616919555548	0.398242414253496\\
12.2823406336251	0.398088561140833\\
12.3029893116954	0.39793580711406\\
12.3236924799435	0.397783745854181\\
12.3443956481917	0.397632774374155\\
12.3650988164398	0.397482885179087\\
12.385801984688	0.397334070821567\\
12.406562398648	0.397185916843866\\
12.4273228126081	0.397038828800744\\
12.4480832265682	0.396892799325923\\
12.4688436405282	0.396747821100122\\
12.4896639845707	0.396603472853111\\
12.5104843286133	0.396460167330187\\
12.5313046726558	0.396317897290151\\
12.5521250166983	0.396176655538317\\
12.5730079121072	0.396036015187429\\
12.5938908075161	0.395896394954953\\
12.614773702925	0.395757787721424\\
12.6356565983338	0.395620186413427\\
12.6566046110283	0.395483159603081\\
12.6775526237227	0.395347130882659\\
12.6985006364172	0.395212093251277\\
12.7194486491116	0.39507803975365\\
12.7404642942679	0.394944535406822\\
12.7614799394242	0.394812007672877\\
12.7824955845805	0.394680449666531\\
12.8035112297368	0.394549854547653\\
12.8245969793381	0.394419784660785\\
12.8456827289394	0.394290670437487\\
12.8667684785407	0.394162505105243\\
12.887854228142	0.394035281936263\\
12.9090125143938	0.393908561414508\\
12.9301708006455	0.39378277611251\\
12.9513290868973	0.393657919367847\\
12.972487373149	0.393533984562398\\
12.9937205946389	0.393410531049265\\
13.0149538161287	0.393287992797007\\
13.0361870376186	0.39316636325074\\
13.0574202591085	0.393045635899478\\
13.0787307850171	0.39292536962997\\
13.1000413109258	0.392805999128154\\
13.1213518368345	0.39268751794427\\
13.1426623627431	0.392569919672046\\
13.1640525369264	0.392452763336247\\
13.1854427111097	0.392336483722699\\
13.206832885293	0.392221074484448\\
13.2282230594763	0.392106529317643\\
13.2496952043377	0.391992407934491\\
13.271167349199	0.391879144659067\\
13.2926394940603	0.391766733245029\\
13.3141116389217	0.391655167488746\\
13.3356680588802	0.391544008290167\\
13.3572244788388	0.391433688999945\\
13.3787808987974	0.391324203470247\\
13.400337318756	0.391215545595569\\
13.4219803036483	0.391107277918382\\
13.4436232885406	0.390999832350534\\
13.4652662734329	0.390893202840688\\
13.4869092583252	0.390787383379465\\
13.5086410871513	0.390681938562544\\
13.5303729159775	0.390577298442428\\
13.5521047448036	0.390473457062349\\
13.5738365736297	0.390370408507137\\
13.5956595174591	0.390267719799324\\
13.6174824612885	0.3901658187492\\
13.639305405118	0.390064699492728\\
13.6611283489474	0.389964356207105\\
13.6830446722324	0.389864358688032\\
13.7049609955174	0.389765132148537\\
13.7268773188025	0.389666670815543\\
13.7487936420875	0.389568968956854\\
13.770805608124	0.38947159944351\\
13.7928175741606	0.389374984581113\\
13.8148295401972	0.389279118685845\\
13.8368415062338	0.389183996114427\\
13.858951377984	0.389089193094711\\
13.8810612497342	0.388995128735779\\
13.9031711214845	0.388901797441444\\
13.9252809932347	0.388809193655711\\
13.9474910360997	0.388716897216936\\
13.9697010789646	0.388625323776858\\
13.9919111218296	0.38853446782535\\
14.0141211646945	0.388444323892139\\
14.0364336494706	0.388354475653187\\
14.0587461342466	0.388265335069122\\
14.0810586190226	0.388176896714357\\
14.1033711037986	0.388089155202834\\
14.1257883079129	0.38800169825727\\
14.1482055120272	0.387914933931689\\
14.1706227161414	0.387828856883594\\
14.1930399202557	0.387743461809686\\
14.2155641320733	0.387658340660041\\
14.2380883438909	0.387573897395632\\
14.2606125557085	0.38749012675564\\
14.283136767526	0.387407023518122\\
14.3057702854045	0.387324184034125\\
14.3284038032829	0.387242007992264\\
14.3510373211614	0.387160490212042\\
14.3736708390398	0.387079625551516\\
14.3964159769655	0.386999014905912\\
14.4191611148912	0.386919053543114\\
14.4419062528169	0.386839736361628\\
14.4646513907425	0.386761058298204\\
14.4875104777278	0.386682624929188\\
14.510369564713	0.386604826959755\\
14.5332286516983	0.386527659366147\\
14.5560877386835	0.386451117162541\\
14.5790631229979	0.386374810720994\\
14.6020385073122	0.386299126064745\\
14.6250138916265	0.38622405824654\\
14.6479892759408	0.386149602356754\\
14.6710833264109	0.386075373666774\\
14.6941773768809	0.386001753409859\\
14.7172714273509	0.385928736714066\\
14.740365477821	0.38585631874478\\
14.7635805844029	0.385784119768262\\
14.7867956909849	0.385712516127983\\
14.8100107975669	0.385641503026156\\
14.8332259041488	0.385571075702021\\
14.8565644820063	0.385500859494004\\
14.8799030598637	0.385431225774567\\
14.9032416377212	0.385362169818957\\
14.9265802155787	0.385293686939152\\
14.9500447068131	0.385225407613259\\
14.9735091980476	0.385157698171471\\
14.996973689282	0.385090553960978\\
15.0204381805165	0.38502397036541\\
15.0440310551214	0.384957583063165\\
15.0676239297264	0.384891753277956\\
15.0912168043314	0.384826476427858\\
15.1148096789363	0.384761747967096\\
15.1385334385982	0.384697208820701\\
15.16225719826	0.384633215056233\\
15.1859809579218	0.384569762161617\\
15.2097047175837	0.384506845660646\\
15.2335618956234	0.384444111768531\\
15.2574190736632	0.384381911349869\\
15.281276251703	0.384320239961436\\
15.3051334297427	0.38425909319559\\
15.3291265950906	0.384198122588184\\
15.3531197604384	0.384137673767376\\
15.3771129257863	0.384077742357816\\
15.4011060911341	0.384018324019456\\
15.4252378496272	0.383959075634014\\
15.4493696081203	0.383900337565082\\
15.4735013666134	0.383842105504232\\
15.4976331251065	0.383784375178059\\
15.5219061221034	0.383726808831365\\
15.5461791191004	0.383669741543218\\
15.5704521160973	0.383613169071181\\
15.5947251130942	0.383557087207567\\
15.619142035055	0.383501163572563\\
15.6435589570157	0.383445727945778\\
15.6679758789765	0.383390776149863\\
15.6923928009372	0.383336304041942\\
15.7169563780926	0.383281984622933\\
15.741519955248	0.383228142365152\\
15.7660835324033	0.38317477315545\\
15.7906471095587	0.383121872914882\\
15.8153601179615	0.383069120025469\\
15.8400731263644	0.383016833649481\\
15.8647861347672	0.382965009737103\\
15.88949914317	0.382913644272458\\
15.9143644081624	0.382862421011609\\
15.9392296731548	0.382811653811613\\
15.9640949381471	0.382761338685142\\
15.9889602031395	0.382711471678544\\
16.0139806003288	0.382661741913171\\
16.039000997518	0.382612457947441\\
16.0640213947072	0.382563615855687\\
16.0890417918965	0.382515211745656\\
16.1142202516767	0.382466940086837\\
16.1393987114569	0.382419104154153\\
16.1645771712372	0.382371700082789\\
16.1897556310174	0.382324724041082\\
16.2150951374624	0.382277875832482\\
16.2404346439074	0.382231453460553\\
16.2657741503524	0.382185453120536\\
16.2911136567974	0.382139871040567\\
16.3166172552097	0.382094412330957\\
16.342120853622	0.382049369749228\\
16.3676244520343	0.382004739549896\\
16.3931280504465	0.381960518020121\\
16.4187988465862	0.381916415552989\\
16.4444696427259	0.38187271968224\\
16.4701404388656	0.381829426720908\\
16.4958112350053	0.381786533014411\\
16.5216524004798	0.381743754206902\\
16.5474935659542	0.381701372638389\\
16.5733347314287	0.381659384679669\\
16.5991758969032	0.381617786733677\\
16.6251906694304	0.381576299665466\\
16.6512054419577	0.381535200649803\\
16.677220214485	0.381494486114517\\
16.7032349870123	0.381454152519321\\
16.7294266758655	0.381413925913383\\
16.7556183647187	0.38137407834149\\
16.781810053572	0.381334606287777\\
16.8080017424252	0.381295506268017\\
16.8343737319713	0.381256509475827\\
16.8607457215174	0.381217882864222\\
16.8871177110636	0.381179622972935\\
16.9134897006097	0.38114172637309\\
16.9400454511807	0.381103929364433\\
16.9666012017517	0.381066493845065\\
16.9931569523226	0.381029416409615\\
17.0197127028936	0.380992693683862\\
17.0464557582069	0.380956067027623\\
17.0731988135202	0.380919793328849\\
17.0999418688336	0.380883869236377\\
17.1266849241469	0.380848291429955\\
17.1536189111414	0.380812806286811\\
17.1805528981359	0.380777665726061\\
17.2074868851304	0.380742866450076\\
17.2344208721249	0.380708405191895\\
17.2615495082445	0.380674033297087\\
17.2886781443641	0.380639997763805\\
17.3158067804836	0.380606295347286\\
17.3429354166032	0.380572922833198\\
17.3702625121198	0.380539636487022\\
17.3975896076364	0.380506678433184\\
17.424916703153	0.380474045479126\\
17.4522437986696	0.380441734462489\\
17.4797732620512	0.380409506517375\\
17.5073027254328	0.380377598944653\\
17.5348321888143	0.380346008603327\\
17.5623616521959	0.380314732382362\\
17.5900974943972	0.380283536232092\\
17.6178333365984	0.380252652681141\\
17.6455691787997	0.380222078639433\\
17.6733050210009	0.380191811046618\\
17.7012513604556	0.380161620615354\\
17.7291976999103	0.380131735154902\\
17.757144039365	0.38010215162547\\
17.7850903788197	0.380072867016768\\
17.8132514459017	0.380043656749301\\
17.8414125129837	0.380014743966441\\
17.8695735800657	0.379986125678059\\
17.8977346471477	0.3799577989233\\
17.9261147917276	0.379929543772692\\
17.9544949363076	0.379901578760632\\
17.9828750808876	0.379873900946037\\
18.0112552254676	0.379846507416868\\
18.0398589202161	0.379819182837289\\
18.0684626149646	0.379792141188189\\
18.097066309713	0.379765379576918\\
18.1256700044615	0.379738895139652\\
18.1545018545664	0.379712477073387\\
18.1833337046712	0.379686334865492\\
18.212165554776	0.379660465671148\\
18.2409974048809	0.379634866674139\\
18.2700621491713	0.379609331547564\\
18.2991268934616	0.379584065341124\\
18.328191637752	0.379559065257233\\
18.3572563820424	0.37953432852669\\
18.3865589068877	0.379509653234945\\
18.4158614317331	0.379485240057071\\
18.4451639565785	0.379461086242124\\
18.4744664814239	0.379437189067325\\
18.5040118216459	0.379413350971557\\
18.5335571618679	0.37938976831339\\
18.56310250209	0.379366438387935\\
18.592647842312	0.379343358518251\\
18.622441195706	0.379320335431638\\
18.6522345491	0.379297561234536\\
18.682027902494	0.379275033267526\\
18.711821255888	0.379252748898925\\
18.74186798467	0.379230519084302\\
18.771914713452	0.379208531737384\\
18.801961442234	0.379186784243648\\
18.8320081710161	0.379165274016091\\
18.8623138179204	0.379143816172506\\
18.8926194648247	0.379122594499374\\
18.922925111729	0.37910160642649\\
18.9532307586333	0.379080849410965\\
18.9838010512769	0.379060142670135\\
19.0143713439205	0.379039665925278\\
19.0449416365641	0.379019416649939\\
19.0755119292076	0.37899939234477\\
19.1063527942164	0.378979416260909\\
19.1371936592252	0.378959664119629\\
19.1680345242339	0.378940133437658\\
19.1988753892427	0.378920821758626\\
19.2299929618915	0.378901556302364\\
19.2611105345403	0.378882508854708\\
19.2922281071891	0.378863676975006\\
19.3233456798379	0.378845058249306\\
19.3547463174076	0.378826483799826\\
19.3861469549774	0.378808121542781\\
19.4175475925471	0.378789969079574\\
19.4489482301169	0.378772024038111\\
19.4806385226382	0.378754121377667\\
19.5123288151595	0.378736425209672\\
19.5440191076808	0.378718933177031\\
19.5757094002021	0.37870164294895\\
19.6076961882754	0.378684393253936\\
19.6396829763487	0.378667344466048\\
19.671669764422	0.378650494269132\\
19.7036565524953	0.378633840373141\\
19.7359469398787	0.378617225209164\\
19.768237327262	0.378600805480121\\
19.8005277146453	0.378584578910242\\
19.8328181020287	0.378568543249672\\
19.8654194755566	0.378552544563404\\
19.8980208490846	0.378536735951552\\
19.9306222226126	0.378521115178174\\
19.9632235961405	0.378505680033053\\
19.9961436388709	0.378490280148502\\
20.0290636816012	0.37847506508804\\
20.0619837243316	0.378460032654996\\
20.0949037670619	0.378445180678243\\
20.1281504830269	0.378430362288047\\
20.1613971989918	0.378415723580414\\
20.1946439149567	0.37840126239739\\
20.2278906309217	0.37838697660638\\
20.2614723645969	0.378372722766883\\
20.2950540982721	0.37835864357584\\
20.3286358319473	0.378344736913459\\
20.3622175656225	0.378331000685126\\
20.3961430262838	0.378317294810136\\
20.4300684869452	0.378303758655571\\
20.4639939476066	0.37829039013924\\
20.497919408268	0.378277187203955\\
20.5321976950164	0.378264013059321\\
20.5664759817649	0.378251003811844\\
20.6007542685134	0.378238157416373\\
20.6350325552618	0.37822547185259\\
20.6696731863085	0.378212813550154\\
20.7043138173553	0.378200315425092\\
20.738954448402	0.378187975468734\\
20.7735950794487	0.378175791697069\\
20.8086080231198	0.378163633689213\\
20.8436209667909	0.378151631241191\\
20.878633910462	0.378139782380249\\
20.9136468541332	0.378128085158122\\
20.949042565344	0.378116412231642\\
20.9844382765548	0.378104890348511\\
21.0198339877656	0.378093517571314\\
21.0552296989764	0.37808229198697\\
21.0910191546516	0.378071089258705\\
21.1268086103268	0.378060033157173\\
21.162598066002	0.378049121779725\\
21.1983875216773	0.37803835324789\\
21.234582265905	0.378027606158495\\
21.2707770101326	0.378017001377966\\
21.3069717543603	0.378006537037838\\
21.343166498588	0.377996211293672\\
21.3797786897193	0.377985905602519\\
21.4163908808506	0.377975738000028\\
21.453003071982	0.37796570665133\\
21.4896152631133	0.377955809745432\\
21.5266577262528	0.377945931525565\\
21.5637001893923	0.377936187270774\\
21.6007426525319	0.377926575179184\\
21.6377851156714	0.377917093472659\\
21.6752714017348	0.377907629105709\\
21.7127576877983	0.377898294675858\\
21.7502439738617	0.37788908841362\\
21.7877302599251	0.377880008573109\\
21.8256747161126	0.377870944743255\\
21.8636191723001	0.377862006917201\\
21.9015636284876	0.377853193357229\\
21.9395080846751	0.377844502349095\\
21.9779259262067	0.377835826038977\\
22.0163437677384	0.377827271893134\\
22.0547616092701	0.377818838204982\\
22.0931794508017	0.377810523291291\\
22.1320868496398	0.377802221776166\\
22.1709942484778	0.377794038678735\\
22.2099016473158	0.377785972322898\\
22.2488090461539	0.3777780210558\\
22.2882232273236	0.377770081898746\\
22.3276374084934	0.377762257504978\\
22.3670515896632	0.377754546228216\\
22.4064657708329	0.377746946445318\\
22.4464051256478	0.377739357491759\\
22.4863444804626	0.377731879738562\\
22.5262838352775	0.377724511568581\\
22.5662231900924	0.377717251387705\\
22.6067074061995	0.377710000759939\\
22.6471916223066	0.377702857860499\\
22.6876758384138	0.377695821100657\\
22.7281600545209	0.377688888914636\\
22.7692102626757	0.377681965006885\\
22.8102604708306	0.377675145445799\\
22.8513106789854	0.377668428670326\\
22.8923608871403	0.377661813142295\\
22.9339998357608	0.377655204614902\\
22.9756387843813	0.377648697142495\\
23.0172777330018	0.377642289190928\\
23.0589166816223	0.377635979248874\\
23.1011689388352	0.377629675022646\\
23.143421196048	0.37762346864945\\
23.1856734532609	0.377617358621234\\
23.2279257104737	0.377611343452714\\
23.2707449436263	0.377605342846222\\
23.3135641767789	0.377599436660927\\
23.3563834099314	0.377593623421507\\
23.399202643084	0.377587901675226\\
23.4425351018712	0.37758220303143\\
23.4858675606585	0.377576595162821\\
23.5292000194457	0.377571076633424\\
23.5725324782329	0.377565646029514\\
23.6163923812599	0.377560237438434\\
23.6602522842869	0.377554916080051\\
23.704112187314	0.37754968055703\\
23.747972090341	0.377544529493962\\
23.7923730104802	0.377539399536455\\
23.8367739306194	0.377534353365981\\
23.8811748507586	0.377529389623267\\
23.9255757708978	0.377524506970643\\
23.9705318120247	0.377519644543263\\
24.0154878531517	0.377514862552586\\
24.0604438942787	0.377510159676826\\
24.1053999354057	0.377505534615469\\
24.1509257391711	0.377500928927648\\
24.1964515429365	0.377496400419975\\
24.2419773467019	0.377491947807571\\
24.2875031504673	0.377487569826512\\
24.3336139249026	0.377483210395044\\
24.3797246993379	0.377478924979398\\
24.4258354737732	0.377474712331034\\
24.4719462482085	0.377470571222041\\
24.518657797687	0.377466447865701\\
24.5653693471654	0.377462395451558\\
24.6120808966439	0.377458412766842\\
24.6587924461223	0.377454498619089\\
24.7061212033434	0.377450601453319\\
24.7534499605645	0.377446772245307\\
24.8007787177855	0.377443009817488\\
24.8481074750066	0.377439313012276\\
24.8960705351928	0.377435632443932\\
24.944033595379	0.377432016936593\\
24.9919966555653	0.377428465347333\\
25.0399597157515	0.377424976552887\\
25.0885748730688	0.37742150327506\\
25.1371900303861	0.377418092247683\\
25.1858051877034	0.377414742361909\\
25.2344203450208	0.37741145252823\\
25.2837061315614	0.377408177515127\\
25.332991918102	0.37740496202664\\
25.3822777046427	0.377401804987443\\
25.4315634911833	0.377398705341229\\
25.4815392186928	0.377395619843105\\
25.5315149462023	0.377392591227024\\
25.5814906737117	0.377389618450622\\
25.6314664012212	0.377386700490238\\
25.6821522057685	0.377383796028396\\
25.7328380103157	0.377380945887828\\
25.783523814863	0.377378149058583\\
25.8342096194102	0.37737540454909\\
25.8856265092156	0.377372672910929\\
25.9370433990211	0.377369993113638\\
25.9884602888265	0.377367364179127\\
26.0398771786319	0.377364785147368\\
26.0920470854784	0.377362218381477\\
26.144216992325	0.377359701054991\\
26.1963868991715	0.377357232221127\\
26.2485568060181	0.377354810950852\\
26.3015026403881	0.377352401362169\\
26.3544484747581	0.377350038888937\\
26.4073943091281	0.377347722615141\\
26.4603401434981	0.377345451642191\\
26.5140858539001	0.377343191787191\\
26.5678315643021	0.377340976799794\\
26.6215772747041	0.377338805794205\\
26.6753229851061	0.377336677901736\\
26.7298936218998	0.377334560583676\\
26.7844642586936	0.377332485960075\\
26.8390348954873	0.377330453174814\\
26.893605532281	0.377328461388567\\
26.9490273164448	0.377326479652769\\
27.0044491006085	0.377324538511598\\
27.0598708847722	0.377322637138072\\
27.1152926689359	0.377320774721683\\
27.1715930662283	0.377318921850862\\
27.2278934635207	0.377317107546765\\
27.2841938608131	0.377315331011005\\
27.3404942581055	0.377313591461359\\
27.3972810276665	0.377311873572173\\
27.4540677972275	0.377310191761109\\
27.5108545667884	0.377308545274544\\
27.5676413363494	0.377306933374247\\
27.6246835621125	0.377305348313673\\
27.6817257878755	0.377303796703927\\
27.7387680136386	0.377302277842811\\
27.7958102394016	0.377300791042542\\
27.8531117250382	0.377299329085164\\
27.9104132106748	0.377297898132375\\
27.9677146963114	0.377296497530084\\
28.025016181948	0.377295126637695\\
28.0825792753208	0.377293778767585\\
28.1401423686935	0.377292459623561\\
28.1977054620663	0.377291168596571\\
28.2552685554391	0.377289905090192\\
28.3130956457549	0.37728866291035\\
28.3709227360708	0.377287447335131\\
28.4287498263866	0.377286257797635\\
28.4865769167025	0.377285093742775\\
28.5446704260258	0.377283949435818\\
28.602763935349	0.377282829758985\\
28.6608574446723	0.377281734184809\\
28.7189509539956	0.377280662196872\\
28.7773133374	0.377279608487796\\
28.8356757208045	0.377278577571824\\
28.8940381042089	0.377277568958372\\
28.9524004876134	0.377276582167176\\
29.0110342352743	0.377275612288283\\
29.0696679829352	0.377274663494041\\
29.1283017305961	0.377273735328341\\
29.186935478257	0.37727282734472\\
29.2458431140456	0.377271935002701\\
29.3047507498342	0.377271062157062\\
29.3636583856228	0.37727020838391\\
29.4225660214114	0.377269373268363\\
29.4817501049045	0.377268552613295\\
29.5409341883977	0.377267749978636\\
29.6001182718908	0.377266964970589\\
29.659302355384	0.377266197203772\\
29.7187654815208	0.377265442800015\\
29.7782286076576	0.377264705045596\\
29.8376917337944	0.377263983574822\\
29.8971548599312	0.377263278029853\\
29.9568996607722	0.377262584828687\\
30.0166444616132	0.377261907003739\\
30.0763892624543	0.377261244215544\\
30.1361340632953	0.377260596131971\\
30.1961632086509	0.377259959445923\\
30.2561923540065	0.377259336954391\\
30.316221499362	0.37725872834239\\
30.3762506447176	0.37725813330177\\
30.436566841334	0.377257548780517\\
30.4968830379503	0.377256977357378\\
30.5571992345667	0.377256418740197\\
30.617515431183	0.377255872643191\\
30.6781214246657	0.377255336250929\\
30.7387274181483	0.377254811939931\\
30.799333411631	0.377254299439327\\
30.8599394051136	0.377253798484186\\
30.9208379817972	0.377253306478404\\
30.9817365584807	0.377252825611202\\
31.0426351351642	0.37725235563155\\
31.1035337118477	0.377251896293947\\
31.1647276970968	0.377251445205557\\
31.2259216823458	0.377251004382181\\
31.2871156675948	0.377250573591269\\
31.3483096528438	0.377250152605423\\
31.4098019130839	0.377249739220101\\
31.4712941733241	0.377249335290614\\
31.5327864335642	0.377248940601622\\
31.5942786938043	0.377248554942577\\
31.656072138367	0.377248176283295\\
31.7178655829298	0.377247806330621\\
31.7796590274925	0.377247444885235\\
31.8414524720552	0.37724709175228\\
31.903550051626	0.377246745062955\\
31.9656476311968	0.377246406386817\\
32.0277452107676	0.377246075539454\\
32.0898427903383	0.377245752340602\\
32.1522474993513	0.377245435070776\\
32.2146522083642	0.377245125172637\\
32.2770569173772	0.37724482247564\\
32.3394616263901	0.377244526813093\\
32.4021765035977	0.377244236603601\\
32.4648913808053	0.377243953172586\\
32.527606258013	0.377243676362394\\
32.5903211352206	0.377243406018953\\
32.6533492654047	0.377243140688514\\
32.7163773955889	0.377242881588235\\
32.7794055257731	0.377242628572442\\
32.8424336559572	0.377242381498789\\
32.9057781690946	0.377242139031479\\
32.969122682232	0.377241902287731\\
33.0324671953693	0.377241671132997\\
33.0958117085067	0.377241445435823\\
33.1594757818323	0.377241223969356\\
33.2231398551579	0.3772410077586\\
33.2868039284835	0.377240796679343\\
33.3504680018091	0.377240590610241\\
33.4144548595075	0.377240388425023\\
33.478441717206	0.377240191063644\\
33.5424285749044	0.377239998411484\\
33.6064154326028	0.377239810356583\\
33.6707283499096	0.377239625865482\\
33.7350412672163	0.377239445799739\\
33.7993541845231	0.377239270053633\\
33.8636671018298	0.377239098523909\\
33.9283094023156	0.377238930262722\\
33.9929517028013	0.377238766059389\\
34.057594003287	0.37723860581644\\
34.1222363037727	0.377238449438693\\
34.1872113622324	0.377238296057236\\
34.2521864206922	0.377238146394852\\
34.3171614791519	0.37723800036172\\
34.3821365376116	0.377237857870141\\
34.447447781095	0.377237718123942\\
34.5127590245783	0.377237581784658\\
34.5780702680616	0.377237448769556\\
34.643381511545	0.377237318997867\\
34.7090324201268	0.377237191740425\\
34.7746833287086	0.377237067602404\\
34.8403342372905	0.377236946507635\\
34.9059851458723	0.377236828381769\\
34.9719792533529	0.377236712557333\\
35.0379733608335	0.377236599587665\\
35.1039674683141	0.377236489402675\\
35.1699615757948	0.377236381933951\\
35.2363024710846	0.377236276570798\\
35.3026433663745	0.377236173818904\\
35.3689842616644	0.377236073613799\\
35.4353251569542	0.37723597589257\\
35.5020164861975	0.377235880096744\\
35.5687078154407	0.377235786688228\\
35.6353991446839	0.377235695607751\\
35.7020904739271	0.377235606797483\\
35.76913594038	0.37723551974697\\
35.8361814068328	0.377235434877905\\
35.9032268732857	0.377235352135827\\
35.9702723397385	0.377235271467604\\
36.0376757049608	0.377235192406906\\
36.1050790701831	0.377235115338517\\
36.1724824354054	0.377235040212419\\
36.2398858006277	0.377234966979818\\
36.3076508877251	0.377234895214916\\
36.3754159748225	0.377234825268632\\
36.4431810619199	0.37723475709505\\
36.5109461490173	0.377234690649388\\
36.5790768402747	0.377234625543054\\
36.6472075315322	0.377234562095915\\
36.7153382227897	0.377234500265843\\
36.7834689140472	0.377234440011752\\
36.8519691566203	0.377234380979196\\
36.9204693991935	0.37723432345958\\
36.9889696417667	0.377234267414269\\
37.0574698843398	0.377234212805593\\
37.1263436888854	0.37723415931042\\
37.1952174934311	0.37723410719408\\
37.2640912979767	0.377234056421162\\
37.3329651025223	0.377234006957143\\
37.4022165438372	0.377233958507599\\
37.4714679851522	0.377233911313985\\
37.5407194264671	0.377233865343862\\
37.609970867782	0.377233820565607\\
37.6796040887873	0.377233776711096\\
37.7492373097926	0.377233733999939\\
37.8188705307979	0.377233692402434\\
37.8885037518033	0.377233651889629\\
37.9585229632751	0.377233612217487\\
38.0285421747469	0.377233573585635\\
38.0985613862187	0.377233535966891\\
38.1685805976905	0.377233499334764\\
38.2389900798134	0.377233463467263\\
38.3093995619363	0.377233428545747\\
38.3798090440592	0.377233394545352\\
38.4502185261821	0.377233361441856\\
38.5209622984945	0.377233329060676\\
38.591706070807	0.377233297537791\\
38.6624498431194	0.377233266850531\\
38.7331936154318	0.377233236976809\\
38.8042656863826	0.377233207761962\\
38.8753377573333	0.377233179325263\\
38.9464098282841	0.377233151646056\\
39.0174818992348	0.377233124704223\\
39.0888909542047	0.377233098357503\\
39.1603000091745	0.37723307271603\\
39.2317090641443	0.377233047760991\\
39.3031181191142	0.377233023474062\\
39.3748725127202	0.377232999724639\\
39.4466269063262	0.37723297661416\\
39.5183812999322	0.377232954125495\\
39.5901356935382	0.37723293224196\\
39.662243568804	0.377232910843835\\
39.7343514440698	0.377232890024364\\
39.8064593193355	0.377232869767952\\
39.8785671946013	0.377232850059413\\
39.95103651206	0.377232830789162\\
40.0235058295186	0.377232812042752\\
40.0959751469772	0.37723279380599\\
40.1684444644358	0.37723277606506\\
40.2412830193083	0.377232758719786\\
40.3141215741807	0.377232741848532\\
40.3869601290532	0.377232725438385\\
40.4597986839256	0.377232709476779\\
40.5330141309314	0.377232693872258\\
40.6062295779373	0.377232678696484\\
40.6794450249431	0.377232663937716\\
40.752660471949	0.377232649584525\\
40.8262603410054	0.37723263555352\\
40.8998602100619	0.377232621910134\\
40.9734600791183	0.377232608643696\\
41.0470599481748	0.377232595743817\\
41.1210516636287	0.377232583134545\\
41.1950433790826	0.377232570875545\\
41.2690350945364	0.377232558957118\\
41.3430268099903	0.377232547369827\\
41.4174177047231	0.377232536044572\\
41.4918085994559	0.377232525035682\\
41.5661994941887	0.377232514334349\\
41.6405903889215	0.377232503932003\\
41.7153877210576	0.377232493765844\\
41.7901850531936	0.377232483885282\\
41.8649823853296	0.377232474282319\\
41.9397797174656	0.377232464949176\\
42.0149906816255	0.377232455828839\\
42.0902016457854	0.377232446966187\\
42.1654126099454	0.377232438353963\\
42.2406235741053	0.377232429985108\\
42.316255315397	0.377232421807913\\
42.3918870566888	0.377232413863095\\
42.4675187979805	0.37723240614407\\
42.5431505392722	0.377232398644437\\
42.6192101664345	0.377232391317345\\
42.6952697935968	0.377232384199689\\
42.771329420759	0.377232377285503\\
42.8473890479213	0.377232370568984\\
42.9238836429179	0.377232364007721\\
43.0003782379145	0.377232357635113\\
43.0768728329111	0.377232351445752\\
43.1533674279077	0.377232345434384\\
43.2303040564086	0.377232339562651\\
43.3072406849096	0.377232333860755\\
43.3841773134105	0.377232328323798\\
43.4611139419115	0.377232322947022\\
43.5384996649489	0.37723231769577\\
43.6158853879863	0.377232312597323\\
43.6932711110238	0.377232307647248\\
43.7706568340612	0.377232302841237\\
43.8484987166093	0.377232298148007\\
43.9263405991575	0.377232293592173\\
44.0041824817057	0.377232289169727\\
44.0820243642538	0.377232284876771\\
44.1603294838729	0.377232280685091\\
44.2386346034919	0.377232276616877\\
44.3169397231109	0.377232272668504\\
44.39524484273	0.377232268836449\\
44.4740202988846	0.377232265095288\\
44.5527957550393	0.377232261465003\\
44.631571211194	0.377232257942321\\
44.7103466673487	0.377232254524058\\
44.7895995895968	0.377232251187323\\
44.8688525118449	0.377232247950096\\
44.948105434093	0.377232244809419\\
45.0273583563411	0.377232241762419\\
45.1070959123988	0.377232238788503\\
45.1868334684566	0.377232235903833\\
45.2665710245144	0.377232233105738\\
45.3463085805721	0.377232230391626\\
45.4265379845256	0.37723222774299\\
45.5067673884792	0.377232225174341\\
45.5869967924327	0.377232222683273\\
45.6672261963862	0.377232220267446\\
45.7479547148514	0.37723221791024\\
45.8286832333165	0.377232215624675\\
45.9094117517817	0.377232213408582\\
45.9901402702469	0.377232211259852\\
46.0713752310891	0.377232209163573\\
46.1526101919314	0.377232207131416\\
46.2338451527736	0.377232205161424\\
46.3150801136159	0.377232203251701\\
46.3968289125376	0.377232201388877\\
46.4785777114593	0.377232199583403\\
46.560326510381	0.377232197833518\\
46.6420753093027	0.377232196137516\\
46.7243454163926	0.37723219448342\\
46.8066155234824	0.377232192880581\\
46.8888856305723	0.377232191327414\\
46.9711557376621	0.377232189822385\\
47.0539547042701	0.377232188354776\\
47.136753670878	0.377232186932944\\
47.2195526374859	0.377232185555465\\
47.3023516040939	0.37723218422096\\
47.3856870701744	0.377232182919844\\
47.4690225362549	0.377232181659582\\
47.5523580023354	0.377232180438893\\
47.6356934684159	0.377232179256539\\
47.7195731680798	0.377232178103957\\
47.8034528677438	0.377232176987806\\
47.8873325674077	0.377232175906938\\
47.9712122670716	0.37723217486024\\
48.0556440359198	0.37723217384007\\
48.1400758047679	0.377232172852362\\
48.2245075736161	0.377232171896086\\
48.3089393424642	0.377232170970246\\
48.3939311234603	0.377232170068023\\
48.4789229044563	0.377232169194705\\
48.5639146854524	0.377232168349367\\
48.6489064664485	0.377232167531115\\
48.7344663172648	0.377232166733875\\
48.820026168081	0.377232165962351\\
48.9055860188973	0.377232165215714\\
48.9911458697136	0.377232164493164\\
49.0772819670679	0.377232163789293\\
49.1634180644223	0.377232163108281\\
49.2495541617767	0.377232162449388\\
49.3356902591311	0.377232161811898\\
49.4224109073406	0.377232161190999\\
49.5091315555501	0.377232160590405\\
49.5958522037596	0.377232160009452\\
49.6825728519691	0.377232159447501\\
49.769886487838	0.377232158900275\\
49.857200123707	0.377232158371069\\
49.944513759576	0.377232157859292\\
50.0318273954449	0.377232157364369\\
50.1197425956162	0.377232156882506\\
50.2076577957875	0.377232156416621\\
50.2955729959588	0.377232155966186\\
50.3834881961302	0.377232155530691\\
50.4720136825001	0.377232155106766\\
50.5605391688701	0.377232154696999\\
50.64906465524	0.377232154300917\\
50.73759014161	0.377232153918064\\
50.8267347882915	0.377232153545455\\
50.915879434973	0.377232153185377\\
51.0050240816545	0.377232152837412\\
51.094168728336	0.377232152501151\\
51.1839415683769	0.377232152173951\\
51.2737144084178	0.377232151857834\\
51.3634872484587	0.377232151552427\\
51.4532600884996	0.377232151257366\\
51.5436703203087	0.377232150970314\\
51.6340805521178	0.377232150693055\\
51.7244907839268	0.377232150425256\\
51.8149010157359	0.377232150166596\\
51.9059580087448	0.377232149915006\\
51.9970150017537	0.377232149672063\\
52.0880719947626	0.377232149437469\\
52.1791289877716	0.377232149210939\\
52.270842290771	0.377232148990646\\
52.3625555937704	0.377232148777979\\
52.4542688967698	0.377232148572675\\
52.5459821997693	0.37723214837448\\
52.6383615465513	0.377232148181781\\
52.7307408933333	0.377232147995803\\
52.8231202401153	0.377232147816311\\
52.9154995868973	0.377232147643079\\
53.0085549039533	0.377232147474688\\
53.1016102210094	0.377232147312212\\
53.1946655380654	0.377232147155446\\
53.2877208551214	0.377232147004187\\
53.3814622684928	0.377232146857187\\
53.4752036818643	0.37723214671539\\
53.5689450952358	0.377232146578612\\
53.6626865086073	0.377232146446676\\
53.7571243510415	0.377232146318482\\
53.8515621934757	0.37723214619486\\
53.9460000359099	0.377232146075647\\
54.0404378783442	0.377232145960685\\
54.1355826971555	0.377232145849009\\
54.2307275159668	0.377232145741346\\
54.325872334778	0.377232145637552\\
54.4210171535893	0.377232145537487\\
54.5168797176294	0.377232145440305\\
54.6127422816695	0.37723214534664\\
54.7086048457095	0.377232145256368\\
54.8044674097496	0.377232145169364\\
54.9010587178797	0.377232145084885\\
54.9976500260098	0.377232145003488\\
55.0942413341398	0.377232144925062\\
55.1908326422699	0.377232144849497\\
55.2881639308243	0.377232144776142\\
55.3854952193786	0.377232144705484\\
55.4828265079329	0.377232144637424\\
55.5801577964873	0.377232144571867\\
55.6782405480441	0.377232144508242\\
55.776323299601	0.377232144446975\\
55.8744060511579	0.377232144387978\\
55.9724888027147	0.377232144331168\\
56.0713347542828	0.377232144276045\\
56.1701807058509	0.377232144222981\\
56.2690266574189	0.377232144171899\\
56.367872608987	0.377232144122724\\
56.4674937609263	0.377232144075023\\
56.5671149128656	0.377232144029116\\
56.6667360648049	0.377232143984938\\
56.7663572167441	0.377232143942423\\
56.8667658411891	0.377232143901192\\
56.9671744656341	0.377232143861525\\
57.0675830900791	0.377232143823363\\
57.1679917145241	0.377232143786649\\
57.2692003658787	0.377232143751052\\
57.3704090172333	0.377232143716817\\
57.4716176685879	0.377232143683891\\
57.5728263199425	0.377232143652225\\
57.6748478434472	0.37723214362153\\
57.7768693669519	0.377232143592018\\
57.8788908904565	0.377232143563645\\
57.9809124139612	0.377232143536365\\
58.0837599560178	0.37723214350993\\
58.1866074980744	0.377232143484522\\
58.2894550401309	0.377232143460102\\
58.3923025821875	0.377232143436631\\
58.4959896007833	0.377232143413892\\
58.5996766193791	0.377232143392044\\
58.7033636379749	0.377232143371053\\
58.8070506565707	0.377232143350884\\
58.911590931702	0.377232143331349\\
59.0161312068333	0.377232143312587\\
59.1206714819646	0.377232143294566\\
59.2252117570959	0.377232143277257\\
59.3306194021433	0.377232143260497\\
59.4360270471907	0.377232143244405\\
59.541434692238	0.377232143228954\\
59.6468423372854	0.377232143214118\\
59.7351317529641	0.377232143202148\\
59.8234211686427	0.377232143190579\\
59.9117105843214	0.377232143179398\\
60	0.377232143168591\\
};
\addlegendentry{Sani}

\addplot [color=mycolor2, line width=2.0pt]
  table[row sep=crcr]{%
0	0\\
0.00199053585276749	1.78229496166774e-07\\
0.00398107170553497	7.12633992228109e-07\\
0.00597160755830246	1.60278740174769e-06\\
0.00796214341106994	2.84826351452976e-06\\
0.0123567669950473	6.85404758419239e-06\\
0.0167513905790247	1.25850776329733e-05\\
0.0211460141630021	2.00367610079099e-05\\
0.0255406377469794	2.92045024365383e-05\\
0.0299388563931551	4.00933033150155e-05\\
0.0343370750393307	5.26917541574596e-05\\
0.0387352936855064	6.69952408299566e-05\\
0.043133512331682	8.29991469417208e-05\\
0.0475471725191068	0.000100763977046306\\
0.0519608327065316	0.000120231866807429\\
0.0563744928939564	0.000141398144441452\\
0.0607881530813811	0.000164258136227289\\
0.0652173199794049	0.000188896387807193\\
0.0696464868774286	0.000215230840266865\\
0.0740756537754524	0.000243256764819336\\
0.0785048206734761	0.000272969431060912\\
0.0829496187548012	0.0003044778764041\\
0.0873944168361263	0.000337675440811992\\
0.0918392149174515	0.000372557338960901\\
0.0962840129987766	0.000409118784232188\\
0.100744568101794	0.000447493511353313\\
0.105205123204811	0.000487550052866088\\
0.109665678307829	0.000529283567379222\\
0.114126233410846	0.000572689212529328\\
0.118602673075784	0.000617925628576703\\
0.123079112740721	0.000664836331890626\\
0.127555552405659	0.000713416425482893\\
0.132031992070596	0.000763661011717144\\
0.136524445545197	0.000815753845246587\\
0.141016899019798	0.000869513216840921\\
0.1455093524944	0.0009249341743874\\
0.150001805969001	0.000982011765450184\\
0.154510404254387	0.00104095506897954\\
0.159019002539772	0.00110155693967016\\
0.163527600825158	0.00116381237075881\\
0.168036199110544	0.00122771635548534\\
0.172561074936287	0.00129350350710582\\
0.17708595076203	0.0013609410336027\\
0.181610826587773	0.00143002387403821\\
0.186135702413515	0.00150074696780503\\
0.190676990405524	0.00157337067538119\\
0.195218278397533	0.00164763634457492\\
0.199759566389542	0.00172353886075141\\
0.204300854381551	0.00180107310993476\\
0.208858690959197	0.00188052541167649\\
0.213416527536843	0.00196161104109733\\
0.217974364114488	0.00204432483034484\\
0.222532200692134	0.00212866161255516\\
0.227106724202768	0.002214933880889\\
0.231681247713401	0.00230283062263748\\
0.236255771224035	0.00239234661721169\\
0.240830294734669	0.00248347664534219\\
0.245421645441331	0.00257655958859938\\
0.250012996147992	0.00267125793098603\\
0.254604346854654	0.00276756639965983\\
0.259195697561315	0.00286547972343\\
0.263804017673384	0.00296536338802291\\
0.268412337785452	0.00306685315772696\\
0.273020657897521	0.00316994370793144\\
0.27762897800959	0.00327462971601052\\
0.282254411746462	0.00338130348928685\\
0.286879845483334	0.0034895738542147\\
0.291505279220206	0.00359943543490193\\
0.296130712957079	0.00371088285777586\\
0.300773406613671	0.0038243354710918\\
0.305416100270264	0.0039393749434537\\
0.310058793926856	0.0040559958481768\\
0.314701487583449	0.00417419276123169\\
0.319361589521735	0.00429441229127567\\
0.324021691460022	0.00441620872886088\\
0.328681793398309	0.00453957659700084\\
0.333341895336596	0.00466451042170162\\
0.338019556071109	0.00479148429394043\\
0.342697216805623	0.0049200249035783\\
0.347374877540136	0.00505012672381997\\
0.35205253827465	0.00518178423120122\\
0.356747910447269	0.00531549922046413\\
0.361443282619888	0.00545077055854718\\
0.366138654792507	0.00558759266934148\\
0.370834026965126	0.00572595998040905\\
0.375547265516765	0.00586640221678161\\
0.380260504068404	0.00600839019522857\\
0.384973742620042	0.00615191829082444\\
0.389686981171681	0.00629698088265588\\
0.394418243246227	0.00644413584996011\\
0.399149505320772	0.00659282573458561\\
0.403880767395317	0.00674304486328981\\
0.408612029469863	0.00689478756718486\\
0.41336147459102	0.00704864010918091\\
0.418110919712177	0.00720401652596603\\
0.422860364833334	0.00736091109648197\\
0.427609809954491	0.00751931810436927\\
0.432377599961091	0.00767985242373971\\
0.437145389967691	0.00784189935772533\\
0.44191317997429	0.00800545313795617\\
0.44668096998089	0.00817050800110655\\
0.451467269141767	0.00833770766381715\\
0.456253568302644	0.00850640846348972\\
0.461039867463521	0.00867660458494879\\
0.465826166624398	0.00884829021841013\\
0.47063114173014	0.00902213815766839\\
0.475436116835881	0.00919747553892492\\
0.480241091941623	0.00937429650070711\\
0.485046067047365	0.0095525951872821\\
0.489869887370685	0.00973307370279398\\
0.494693707694006	0.00991502974812177\\
0.499517528017326	0.0100984574160067\\
0.504341348340647	0.0102833508052798\\
0.509184185682064	0.010470441565063\\
0.51402702302348	0.0106589977253298\\
0.518869860364897	0.0108490133335491\\
0.523712697706314	0.0110404824436309\\
0.528574726619363	0.0112341664928837\\
0.533436755532412	0.0114293035962897\\
0.53829878444546	0.0116258877565611\\
0.543160813358509	0.0118239129832041\\
0.548042211039569	0.0120241707405083\\
0.55292360872063	0.0122258689886652\\
0.55780500640169	0.0124290016861488\\
0.56268640408275	0.0126335627985819\\
0.56758735048985	0.0128403740605175\\
0.572488296896949	0.0130486130330787\\
0.577389243304049	0.0132582736310225\\
0.582290189711149	0.013469349776611\\
0.587210867601438	0.0136826937191749\\
0.592131545491728	0.0138974523752186\\
0.597052223382018	0.0141136196163057\\
0.601972901272307	0.0143311893218633\\
0.606913496388328	0.0145510445082801\\
0.611854091504349	0.0147723011941672\\
0.61679468662037	0.0149949532084216\\
0.621735281736391	0.0152189943881634\\
0.626695982628195	0.0154453387607735\\
0.631656683519999	0.0156730712018742\\
0.636617384411802	0.0159021854982251\\
0.641578085303606	0.0161326754451709\\
0.646559083672327	0.0163654863400263\\
0.651540082041047	0.0165996716554481\\
0.656521080409768	0.0168352251365909\\
0.661502078778488	0.0170721405375573\\
0.666503569383853	0.0173113946798439\\
0.671505059989219	0.0175520093777368\\
0.676506550594584	0.0177939783353207\\
0.68150804119995	0.0180372952659934\\
0.686530221876942	0.0182829687696221\\
0.691552402553934	0.0185299887467158\\
0.696574583230927	0.0187783488608277\\
0.701596763907919	0.0190280427851913\\
0.706639835866396	0.0192801111661146\\
0.711682907824873	0.0195335117211306\\
0.71672597978335	0.0197882380738021\\
0.721769051741827	0.0200442838577409\\
0.72683321952555	0.0203027220311639\\
0.731897387309272	0.0205624778619329\\
0.736961555092995	0.0208235449341646\\
0.742025722876718	0.0210859168423951\\
0.747111194287107	0.021350699117262\\
0.752196665697496	0.0216167843151022\\
0.757282137107885	0.0218841659811343\\
0.762367608518274	0.0221528376713685\\
0.767474594900923	0.0224239377630717\\
0.772581581283573	0.0226963258255768\\
0.777688567666222	0.022969995365755\\
0.782795554048871	0.0232449399016439\\
0.78792427039055	0.0235223309355833\\
0.793052986732229	0.0238009947701551\\
0.798181703073909	0.024080924874437\\
0.803310419415588	0.0243621147290501\\
0.808461084325784	0.0246457692377339\\
0.81361174923598	0.024930681158597\\
0.818762414146177	0.0252168439234816\\
0.823913079056373	0.0255042509761519\\
0.82908591482452	0.0257941408997636\\
0.834258750592667	0.0260852726284857\\
0.839431586360813	0.0263776395574862\\
0.84460442212896	0.0266712350942361\\
0.849799655005291	0.0269673317936325\\
0.854994887881621	0.0272646544721787\\
0.860190120757951	0.0275631964889335\\
0.865385353634282	0.0278629512156415\\
0.870603213699137	0.0281652254623587\\
0.875821073763992	0.0284687096429536\\
0.881038933828848	0.0287733970809441\\
0.886256793893703	0.0290792811129196\\
0.891497515445601	0.0293877031083092\\
0.896738236997499	0.0296973187726708\\
0.901978958549397	0.0300081213945536\\
0.907219680101295	0.0303201042759657\\
0.912483501408722	0.030634643633635\\
0.917747322716149	0.0309503601752138\\
0.923011144023576	0.0312672471548595\\
0.928274965331004	0.0315852978405775\\
0.933562128954513	0.0319059236013681\\
0.938849292578023	0.0322277098403878\\
0.944136456201532	0.0325506497779819\\
0.949423619825042	0.0328747366487364\\
0.954734372782572	0.0332014172869959\\
0.960045125740102	0.0335292414767053\\
0.965355878697632	0.0338582024049824\\
0.970666631655162	0.0341882932735799\\
0.976001225418392	0.0345209966929252\\
0.981335819181621	0.0348548265152663\\
0.98667041294485	0.0351897758950819\\
0.99200500670808	0.0355258380018827\\
0.997363697280812	0.0358645315353067\\
1.00272238785355	0.0362043341009619\\
1.00808107842628	0.0365452388212807\\
1.01343976899901	0.0368872388341272\\
1.0188228172289	0.037231889258631\\
1.02420586545878	0.0375776311216876\\
1.02958891368867	0.0379244575142804\\
1.03497196191855	0.0382723615432268\\
1.04037963341625	0.0386229350696265\\
1.04578730491395	0.0389745822172495\\
1.05119497641164	0.0393272960462321\\
1.05660264790934	0.0396810696329495\\
1.06203521338424	0.0400375319217707\\
1.06746777885913	0.0403950497901295\\
1.07290034433402	0.0407536162679216\\
1.07833290980891	0.041113224401689\\
1.08379064506731	0.0414755405572929\\
1.08924838032571	0.0418388940255793\\
1.09470611558411	0.0422032778068144\\
1.10016385084252	0.0425686849183207\\
1.1056470369148	0.0429368194907377\\
1.11113022298707	0.0433059728829407\\
1.11661340905935	0.0436761380661827\\
1.12209659513163	0.0440473080291863\\
1.12760551858231	0.0444212250280014\\
1.13311444203298	0.0447961421267976\\
1.13862336548365	0.0451720522684358\\
1.14413228893432	0.0455489484136621\\
1.14966724184116	0.0459286112995367\\
1.15520219474799	0.0463092553376871\\
1.16073714765482	0.0466908734432087\\
1.16627210056166	0.0470734585495009\\
1.17183338076584	0.0474588302433789\\
1.17739466097002	0.0478451639129086\\
1.18295594117419	0.0482324524460514\\
1.18851722137837	0.0486206887494949\\
1.19410513262804	0.0490117316343467\\
1.19969304387772	0.0494037170882164\\
1.20528095512739	0.0497966379725686\\
1.21086886637706	0.0501904871680194\\
1.21648371859449	0.0505871630988316\\
1.22209857081192	0.0509847619608921\\
1.22771342302936	0.0513832765898114\\
1.23332827524679	0.0517826998407798\\
1.23897038457538	0.0521849701387735\\
1.24461249390397	0.0525881434978689\\
1.25025460323256	0.0529922127284713\\
1.25589671256115	0.053397170660997\\
1.26156640151281	0.0538049961141032\\
1.26723609046446	0.0542137045244638\\
1.27290577941612	0.0546232886779336\\
1.27857546836778	0.0550337413808136\\
1.28427306633634	0.055447082270123\\
1.28997066430491	0.0558612857778388\\
1.29566826227347	0.056276344665926\\
1.30136586024204	0.0566922517172335\\
1.30709170327692	0.0571110677899407\\
1.31281754631181	0.0575307259056923\\
1.31854338934669	0.0579512188032306\\
1.32426923238158	0.0583725392426238\\
1.33002366371043	0.0587967897381723\\
1.33577809503928	0.0592218614633896\\
1.34153252636813	0.0596477471344686\\
1.34728695769699	0.0600744394893741\\
1.35307032782915	0.0605040831349702\\
1.35885369796131	0.060934526957223\\
1.36463706809348	0.061365763650457\\
1.37042043822564	0.0617977859312174\\
1.37622881553645	0.0622324596390805\\
1.38203719284726	0.062667911085466\\
1.38784557015807	0.0631041329597013\\
1.39365394746888	0.0635411179737208\\
1.39948194569202	0.0639803388396922\\
1.40530994391516	0.0644203133789042\\
1.41113794213829	0.0648610342977982\\
1.41696594036143	0.0653024943257206\\
1.4228137248248	0.0657461887088559\\
1.42866150928816	0.0661906126324521\\
1.43450929375153	0.0666357588210682\\
1.44035707821489	0.0670816200224626\\
1.44622478964653	0.067529711942288\\
1.45209250107816	0.0679785091980473\\
1.45796021250979	0.0684280045335158\\
1.46382792394143	0.0688781907159593\\
1.46971570462452	0.0693306038047524\\
1.4756034853076	0.0697836979560405\\
1.48149126599069	0.0702374659339104\\
1.48737904667378	0.0706919005262277\\
1.49328704060906	0.0711485580486969\\
1.49919503454433	0.0716058722941643\\
1.50510302847961	0.0720638360481203\\
1.51101102241489	0.0725224421201183\\
1.51693937480906	0.0729832669409283\\
1.52286772720322	0.0734447240819056\\
1.52879607959739	0.0739068063510339\\
1.53472443199155	0.0743695065806407\\
1.540673290528	0.0748344212698598\\
1.54662214906446	0.0752999438160619\\
1.55257100760091	0.0757660670508077\\
1.55851986613736	0.0762327838302791\\
1.56448937947106	0.0767017105536292\\
1.57045889280476	0.0771712206130624\\
1.57642840613847	0.0776413068647993\\
1.58239791947216	0.0781119621899552\\
1.58838823791446	0.0785848227682237\\
1.59437855635675	0.0790582421067355\\
1.60036887479904	0.0795322130874509\\
1.60635919324133	0.0800067286174952\\
1.61237046959671	0.0804834445995272\\
1.61838174595208	0.0809606947139312\\
1.62439302230746	0.0814384718694809\\
1.63040429866284	0.0819167690003819\\
1.63643668672632	0.0823972615512272\\
1.6424690747898	0.0828782635571501\\
1.64850146285328	0.0833597679548114\\
1.65453385091676	0.0838417677065666\\
1.66058750665404	0.0843259577070028\\
1.66664116239132	0.0848106324386095\\
1.67269481812859	0.085295784867003\\
1.67874847386587	0.0857814079837541\\
1.68482355500326	0.0862692160043038\\
1.69089863614064	0.0867574839882201\\
1.69697371727803	0.087246204931141\\
1.70304879841541	0.0877353718549147\\
1.70914546464501	0.0882267181784963\\
1.71524213087462	0.0887184996564751\\
1.72133879710422	0.0892107093155729\\
1.72743546333383	0.0897033402089743\\
1.73355387572031	0.0901981447891524\\
1.73967228810678	0.0906933596762237\\
1.74579070049326	0.0911889779290548\\
1.75190911287974	0.0916849926332241\\
1.75804943463327	0.0921831751622487\\
1.76418975638681	0.0926817431148583\\
1.77033007814035	0.0931806895831217\\
1.77647039989388	0.0936800076860642\\
1.78263279664118	0.0941814876231382\\
1.78879519338848	0.0946833280674213\\
1.79495759013578	0.095185522145237\\
1.80111998688309	0.0956880630101065\\
1.80730462566597	0.0961927595059342\\
1.81348926444885	0.096697791562122\\
1.81967390323173	0.0972031523402995\\
1.82585854201462	0.0977088350295318\\
1.83206559217232	0.0982166670035133\\
1.83827264233002	0.0987248095632206\\
1.84447969248772	0.0992332559066366\\
1.85068674264543	0.0997419992594138\\
1.85691637550678	0.100252885380206\\
1.86314600836813	0.100764057086937\\
1.86937564122948	0.101275507614988\\
1.87560527409083	0.10178723022764\\
1.88185766319929	0.102301088936977\\
1.88811005230775	0.102815208209952\\
1.89436244141622	0.103329581320385\\
1.90061483052468	0.103844201570222\\
1.90689015093446	0.104360951030299\\
1.91316547134424	0.104877936011763\\
1.91944079175402	0.105395149827913\\
1.92571611216381	0.1059125858204\\
1.93201454150218	0.106432144006196\\
1.93831297084055	0.10695191265382\\
1.94461140017892	0.10747188511709\\
1.9509098295173	0.107992054778392\\
1.95723154808653	0.108514339491426\\
1.96355326665576	0.109036809592043\\
1.96987498522499	0.109559458475609\\
1.97619670379422	0.110082279566271\\
1.98254189319299	0.110607208325673\\
1.98888708259176	0.11113229738624\\
1.99523227199053	0.111657540185918\\
2.0015774613893	0.112182930191648\\
2.00794630624256	0.112710420381676\\
2.01431515109582	0.113238045776889\\
2.02068399594908	0.113765799858841\\
2.02705284080234	0.114293676138286\\
2.03344552746868	0.114823644906119\\
2.03983821413503	0.115353723776117\\
2.04623090080137	0.115883906274465\\
2.05262358746771	0.116414185956758\\
2.05904030495676	0.116946550292851\\
2.06545702244581	0.117478999623602\\
2.07187373993486	0.118011527520853\\
2.07829045742391	0.118544127586053\\
2.084731396657	0.119078804267194\\
2.0911723358901	0.119613540833514\\
2.0976132751232	0.120148330903535\\
2.1040542143563	0.120683168125578\\
2.11051956957009	0.121220073835554\\
2.11698492478388	0.121757014321793\\
2.12345027999767	0.122293983250511\\
2.12991563521146	0.122830974317916\\
2.13640560238783	0.123370025521262\\
2.1428955695642	0.123909086395005\\
2.14938553674057	0.124448150654067\\
2.15587550391694	0.124987212043557\\
2.16239028199053	0.125528325089679\\
2.16890506006412	0.126069422705869\\
2.17541983813771	0.126610498656771\\
2.1819346162113	0.127151546737399\\
2.18847440630977	0.127694637801226\\
2.19501419640825	0.128237688342804\\
2.20155398650673	0.128780692177511\\
2.20809377660521	0.129323643151272\\
2.21465878275268	0.129868628295266\\
2.22122378890015	0.130413547835165\\
2.22778879504762	0.130958395638091\\
2.23435380119508	0.131503165601885\\
2.24094422983074	0.132049960739503\\
2.24753465846639	0.132596665204106\\
2.25412508710205	0.133143272915563\\
2.2607155157377	0.133689777824635\\
2.2673315759421	0.134238298742426\\
2.27394763614649	0.134786703933648\\
2.28056369635089	0.135334987371921\\
2.28717975655528	0.135883143061925\\
2.29382166035897	0.136433305448258\\
2.30046356416266	0.136983327072252\\
2.30710546796636	0.137533201962278\\
2.31374737177005	0.138082924177934\\
2.32041533418655	0.13863464362862\\
2.32708329660306	0.13918619730138\\
2.33375125901957	0.139737579280336\\
2.34041922143607	0.140288783680998\\
2.34711345989709	0.140841975655325\\
2.3538076983581	0.141394976858737\\
2.36050193681912	0.141947781432108\\
2.36719617528013	0.14250038354785\\
2.37391691043736	0.143054963437601\\
2.3806376455946	0.143609327588417\\
2.38735838075183	0.144163470198918\\
2.39407911590907	0.144717385499413\\
2.40082657131283	0.145273268604674\\
2.40757402671659	0.145828911030324\\
2.41432148212035	0.14638430703372\\
2.42106893752412	0.146939450904056\\
2.42784333963791	0.147496552436746\\
2.4346177417517	0.148053388378853\\
2.4413921438655	0.148609953047465\\
2.44816654597929	0.149166240791652\\
2.4549681243632	0.149724475892761\\
2.4617697027471	0.150282420524369\\
2.46857128113101	0.150840069064282\\
2.47537285951492	0.15139741592243\\
2.48220184698832	0.151956699677276\\
2.48903083446173	0.152515668118071\\
2.49585982193513	0.153074315684334\\
2.50268880940853	0.153632636847836\\
2.50954544177924	0.154192884265572\\
2.51640207414995	0.15475279156142\\
2.52325870652066	0.155312353237594\\
2.53011533889137	0.155871563828689\\
2.53699985562036	0.156432689898243\\
2.54388437234935	0.156993451077047\\
2.55076888907833	0.157553841930992\\
2.55765340580732	0.158113857058483\\
2.5645660494942	0.158675776707707\\
2.57147869318108	0.159237306738598\\
2.57839133686796	0.159798441781713\\
2.58530398055484	0.160359176500242\\
2.59224499685821	0.160921804590818\\
2.59918601316159	0.161484018379078\\
2.60612702946496	0.162045812561225\\
2.61306804576833	0.162607181866211\\
2.62003768421203	0.163170433260169\\
2.62700732265572	0.163733245713641\\
2.63397696109942	0.16429561398946\\
2.64094659954311	0.164857532883317\\
2.64794511298781	0.165421322400769\\
2.65494362643251	0.165984648387656\\
2.6619421398772	0.166547505674416\\
2.6689406533219	0.167109889124458\\
2.67596829852112	0.167674131589036\\
2.68299594372033	0.168237885983256\\
2.69002358891955	0.168801147206139\\
2.69705123411877	0.169363910189783\\
2.70410827073433	0.169928520349858\\
2.71116530734989	0.170492617952393\\
2.71822234396545	0.171056197965976\\
2.72527938058101	0.171619255392368\\
2.73236607296869	0.172184148064567\\
2.73945276535637	0.172748503746739\\
2.74653945774405	0.173312317478009\\
2.75362615013173	0.173875584330772\\
2.76074276611797	0.174440674301427\\
2.76785938210422	0.175005202906499\\
2.77497599809046	0.175569165256629\\
2.78209261407671	0.176132556495816\\
2.78923942490227	0.176697758516725\\
2.79638623572783	0.177262374855687\\
2.80353304655339	0.177826400695837\\
2.81067985737896	0.178389831253751\\
2.81785713902212	0.178955060146615\\
2.82503442066529	0.179519679102413\\
2.83221170230846	0.180083683377743\\
2.83938898395163	0.180647068262726\\
2.84659701614953	0.181212238840447\\
2.85380504834743	0.181776775289424\\
2.86101308054533	0.182340672940694\\
2.86822111274323	0.182903927158892\\
2.87546017926332	0.183468954246267\\
2.88269924578341	0.184033323078824\\
2.88993831230351	0.184597029063011\\
2.8971773788236	0.185160067638948\\
2.90444776780364	0.185724866098049\\
2.91171815678368	0.186288982243964\\
2.91898854576373	0.186852411559529\\
2.92625893474377	0.187415149561312\\
2.93356093845992	0.187979634271849\\
2.94086294217606	0.188543412680689\\
2.94816494589221	0.189106480348023\\
2.95546694960836	0.189668832867841\\
2.96280086487659	0.190232918756656\\
2.97013478014483	0.190796274427191\\
2.97746869541306	0.191358895517971\\
2.9848026106813	0.191920777701373\\
2.99216873909201	0.192484379758154\\
2.99953486750271	0.193047227754038\\
3.00690099591342	0.19360931740685\\
3.01426712432413	0.194170644468324\\
3.02166577186631	0.194733677714246\\
3.02906441940849	0.195295933132723\\
3.03646306695067	0.19585740652186\\
3.04386171449286	0.19641809371371\\
3.05129319192691	0.196980473227527\\
3.05872466936097	0.197542051225467\\
3.06615614679503	0.198102823586884\\
3.07358762422909	0.198662786225125\\
3.08105224696524	0.199224427131101\\
3.0885168697014	0.199785242912961\\
3.09598149243755	0.200345229532282\\
3.1034461151737	0.200904382984669\\
3.11094420406679	0.201465200509271\\
3.11844229295988	0.202025169383617\\
3.12594038185297	0.202584285652481\\
3.13343847074605	0.203142545394698\\
3.14097035080817	0.203702454766408\\
3.14850223087028	0.20426149204603\\
3.1560341109324	0.204819653362512\\
3.16356599099451	0.205376934878889\\
3.17113199342146	0.205935851476127\\
3.17869799584842	0.206493872625514\\
3.18626399827537	0.207050994541145\\
3.19383000070233	0.207607213471222\\
3.20143046110215	0.20816505268603\\
3.20903092150198	0.208721973185465\\
3.21663138190181	0.209277971269751\\
3.22423184230164	0.209833043273232\\
3.2318671021242	0.210389720613149\\
3.23950236194676	0.210945456060278\\
3.24713762176933	0.211500246001944\\
3.25477288159189	0.212054086859606\\
3.26244328804483	0.212609517935877\\
3.27011369449776	0.21316398403393\\
3.2777841009507	0.213717481629176\\
3.28545450740363	0.214270007231162\\
3.29316041266953	0.21482410769691\\
3.30086631793543	0.215377220193101\\
3.30857222320133	0.215929341284209\\
3.31627812846722	0.216480467568846\\
3.32401989110489	0.217033153215428\\
3.33176165374256	0.217584827996965\\
3.33950341638023	0.218135488567979\\
3.3472451790179	0.218685131617123\\
3.35502316313618	0.219236318308441\\
3.36280114725446	0.219786471337072\\
3.37057913137275	0.220335587448572\\
3.37835711549103	0.220883663422614\\
3.38617169154667	0.221433267146126\\
3.3939862676023	0.221981814508963\\
3.40180084365794	0.2225293023487\\
3.40961541971357	0.223075727537014\\
3.41746696420648	0.223623664375333\\
3.42531850869938	0.224170522256551\\
3.43317005319229	0.224716298111252\\
3.4410215976852	0.225260988904098\\
3.44891049346576	0.225807175049922\\
3.45679938924632	0.226352259745632\\
3.46468828502688	0.226896240015811\\
3.47257718080744	0.227439112919096\\
3.48050381756295	0.227983464701446\\
3.48843045431847	0.228526692645821\\
3.49635709107398	0.229068793871801\\
3.50428372782949	0.229609765532983\\
3.51224850159594	0.230152199375512\\
3.52021327536239	0.230693487099262\\
3.52817804912884	0.231233625919808\\
3.53614282289529	0.231772613086699\\
3.54414613607264	0.232313045501677\\
3.55214944924998	0.232852309626067\\
3.56015276242733	0.233390402772439\\
3.56815607560468	0.233927322287292\\
3.57619833889781	0.234465669998404\\
3.58424060219095	0.235002827357516\\
3.59228286548408	0.235538791775198\\
3.60032512877721	0.236073560695894\\
3.60840675922679	0.23660974049511\\
3.61648838967638	0.237144707993318\\
3.62457002012596	0.237678460700094\\
3.63265165057554	0.238210996158835\\
3.64077307248204	0.23874492496033\\
3.64889449438855	0.239277619625989\\
3.65701591629505	0.239809077765412\\
3.66513733820155	0.240339297021948\\
3.67329898353354	0.240870891879285\\
3.68146062886554	0.241401230881835\\
3.68962227419754	0.241930311740234\\
3.69778391952953	0.242458132198799\\
3.70598622860803	0.242987310338504\\
3.71418853768653	0.243515211021866\\
3.72239084676502	0.24404183206158\\
3.73059315584352	0.244567171303945\\
3.73883657626523	0.245093850044248\\
3.74707999668693	0.245619229845943\\
3.75532341710863	0.246143308624811\\
3.76356683753034	0.246666084330152\\
3.77185182516863	0.247190181135163\\
3.78013681280692	0.247712957640225\\
3.78842180044522	0.248234411865239\\
3.79670678808351	0.24875454186353\\
3.80503380745825	0.249275974354318\\
3.81336082683299	0.249796065306307\\
3.82168784620774	0.25031481284455\\
3.83001486558248	0.250832215127431\\
3.83838438958511	0.251350901051929\\
3.84675391358775	0.251868224322966\\
3.85512343759039	0.252384183171792\\
3.86349296159302	0.252898775862888\\
3.8719054722877	0.253414633132919\\
3.88031798298237	0.253929106760513\\
3.88873049367705	0.25444219508417\\
3.89714300437172	0.254953896475506\\
3.90559899284027	0.255466843142941\\
3.91405498130882	0.255978385306254\\
3.92251096977737	0.256488521412249\\
3.93096695824593	0.256997249940728\\
3.93946692479213	0.257507204195976\\
3.94796689133833	0.258015733214304\\
3.95646685788454	0.258522835551883\\
3.96496682443074	0.25902850979776\\
3.97351127881372	0.259535389969398\\
3.98205573319669	0.26004082430179\\
3.99060018757967	0.260544811461549\\
3.99914464196265	0.261047350148027\\
4.00773410425464	0.261551074738309\\
4.01632356654662	0.262053333018858\\
4.02491302883861	0.262554123767803\\
4.0335024911306	0.263053445795876\\
4.04213749081375	0.263553933411085\\
4.05077249049689	0.264052934379652\\
4.05940749018004	0.264550447592313\\
4.06804248986318	0.265046471972257\\
4.07672356764034	0.265543641412256\\
4.08540464541749	0.266039304003484\\
4.09408572319465	0.266533458750376\\
4.10276680097181	0.267026104689669\\
4.11149450824228	0.267519874898361\\
4.12022221551276	0.268012118192382\\
4.12894992278323	0.268502833690975\\
4.1376776300537	0.268992020545523\\
4.14645252883691	0.269482310588778\\
4.15522742762012	0.269971053789261\\
4.16400232640333	0.270458249382136\\
4.17277722518654	0.27094389663454\\
4.18159988926615	0.271430625746039\\
4.19042255334575	0.271915788225715\\
4.19924521742536	0.272399383425783\\
4.20806788150496	0.272881410730249\\
4.21693889629988	0.273364498283641\\
4.22580991109479	0.273845999556603\\
4.23468092588971	0.274325914019533\\
4.24355194068463	0.27480424117444\\
4.2524719033702	0.275283606669865\\
4.26139186605577	0.275761366378141\\
4.27031182874135	0.276237519888999\\
4.27923179142692	0.276712066823588\\
4.28820131218307	0.27718762993377\\
4.29717083293922	0.277661567893071\\
4.30614035369538	0.278133880411719\\
4.31510987445153	0.278604567231155\\
4.32412957614907	0.279076247761014\\
4.33314927784661	0.27954628392057\\
4.34216897954416	0.280014675541718\\
4.3511886812417	0.280481422487358\\
4.36025919971018	0.280949140365395\\
4.36932971817866	0.281415194799352\\
4.37840023664715	0.28187958574398\\
4.38747075511563	0.282342313184815\\
4.39659274002011	0.282805988484752\\
4.40571472492458	0.283267981413614\\
4.41483670982906	0.283728292050212\\
4.42395869473353	0.284186920503908\\
4.43313280988389	0.284646473434653\\
4.44230692503424	0.285104325215259\\
4.45148104018459	0.28556047604981\\
4.46065515533495	0.286014926172707\\
4.469882079136	0.286470277074459\\
4.47910900293706	0.286923908196053\\
4.48833592673811	0.287375819868075\\
4.49756285053916	0.287826012451189\\
4.50684327607396	0.288277081772373\\
4.51612370160877	0.288726412833641\\
4.52540412714357	0.289174006093337\\
4.53468455267837	0.289619862039621\\
4.54401918924822	0.290066570383689\\
4.55335382581807	0.290511522139141\\
4.56268846238792	0.290954717893339\\
4.57202309895777	0.291396158263203\\
4.58141267108196	0.291838426309799\\
4.59080224320616	0.292278919591452\\
4.60019181533036	0.292717638825831\\
4.60958138745456	0.293154584759887\\
4.61902663678989	0.293592333326316\\
4.62847188612523	0.29402828910461\\
4.63791713546056	0.294462452944047\\
4.6473623847959	0.294894825722906\\
4.65686406984621	0.295327975717169\\
4.66636575489653	0.295759315054274\\
4.67586743994685	0.296188844716435\\
4.68536912499717	0.296616565714572\\
4.69492802222378	0.297045038153209\\
4.7044869194504	0.297471682220628\\
4.71404581667701	0.297896499033327\\
4.72360471390363	0.298319489736201\\
4.73320764525141	0.298742591428255\\
4.74281057659919	0.299163852549212\\
4.75241350794698	0.299583274345707\\
4.76201643929476	0.300000858092299\\
4.77165794452804	0.300418271397703\\
4.78129944976131	0.300833834515848\\
4.79094095499459	0.301247548820932\\
4.80058246022786	0.30165941571452\\
4.81026314694842	0.302071099118523\\
4.81994383366897	0.302480922979399\\
4.82962452038952	0.302888888798427\\
4.83930520711008	0.303294998103708\\
4.84902565638322	0.303700909065379\\
4.85874610565637	0.304104951399395\\
4.86846655492951	0.304507126733653\\
4.87818700420266	0.304907436722313\\
4.88794780523678	0.305307533203506\\
4.89770860627089	0.305705752241775\\
4.90746940730501	0.30610209559118\\
4.91723020833913	0.306496565031491\\
4.92703195872998	0.306890805485332\\
4.93683370912084	0.307283159948225\\
4.9466354595117	0.307673630299966\\
4.95643720990256	0.308062218445498\\
4.96628051524823	0.308450561779924\\
4.97612382059389	0.308837010840938\\
4.98596712593956	0.309221567633657\\
4.99581043128522	0.309604234187786\\
5.00569590732134	0.3099866398291\\
5.01558138335745	0.310367143177761\\
5.02546685939356	0.310745746363818\\
5.03535233542967	0.311122451541337\\
5.04528060635073	0.31149887934466\\
5.05520887727179	0.311873397097802\\
5.06513714819285	0.312246007055365\\
5.07506541911391	0.312616711495403\\
5.08503711889176	0.312987121774924\\
5.0950088186696	0.313355614506605\\
5.10498051844744	0.313722192069247\\
5.11495221822528	0.314086856864533\\
5.12496799094167	0.314451210382936\\
5.13498376365806	0.314813639113919\\
5.14499953637444	0.315174145560149\\
5.15501530909083	0.315532732246597\\
5.16507580953502	0.315890990213408\\
5.17513630997921	0.316247316409467\\
5.18519681042339	0.316601713460987\\
5.19525731086758	0.316954184015909\\
5.2053632039849	0.317306308044239\\
5.21546909710223	0.317656493573297\\
5.22557499021955	0.318004743352544\\
5.23568088333688	0.318351060152585\\
5.24583284576096	0.318697012290307\\
5.25598480818505	0.319041019453215\\
5.26613677060914	0.319383084513737\\
5.27628873303322	0.319723210364863\\
5.28648745229033	0.320062953042536\\
5.29668617154743	0.320400744521496\\
5.30688489080453	0.320736587796878\\
5.31708361006163	0.321070485883792\\
5.32732978639303	0.321403981952345\\
5.33757596272443	0.321735520848095\\
5.34782213905582	0.322065105688642\\
5.35806831538722	0.322392739610974\\
5.36836266133408	0.322719952301306\\
5.37865700728095	0.323045202093078\\
5.38895135322782	0.32336849222614\\
5.39924569917468	0.323689825959128\\
5.40958894010065	0.324010718873815\\
5.41993218102661	0.324329643411057\\
5.43027542195257	0.324646602932749\\
5.44061866287853	0.324961600818974\\
5.45101153706255	0.325276137909873\\
5.46140441124658	0.325588701390021\\
5.4717972854306	0.325899294743179\\
5.48219015961463	0.326207921470688\\
5.49263341945276	0.326516067049995\\
5.50307667929089	0.326822234029295\\
5.51351993912902	0.327126426014053\\
5.52396319896715	0.327428646626704\\
5.53445761187213	0.3277303653652\\
5.54495202477712	0.328030100756836\\
5.5554464376821	0.328327856528642\\
5.56594085058709	0.328623636424004\\
5.57648719831941	0.328918893302825\\
5.58703354605173	0.329212162329171\\
5.59757989378405	0.329503447351518\\
5.60812624151637	0.329792752234081\\
5.61872532103302	0.330081512542118\\
5.62932440054967	0.330368280732053\\
5.63992348006633	0.330653060773713\\
5.65052255958298	0.330935856652037\\
5.66117518401188	0.331218085984256\\
5.67182780844078	0.331498319171359\\
5.68248043286967	0.331776560304449\\
5.69313305729857	0.33205281348911\\
5.70384005671797	0.33232847773698\\
5.71454705613736	0.332602142049917\\
5.72525405555675	0.332873810640247\\
5.73596105497615	0.33314348773414\\
5.74672327665992	0.33341255306081\\
5.75748549834369	0.333679614898676\\
5.76824772002746	0.333944677581259\\
5.77900994171123	0.334207745455282\\
5.78982825023888	0.334470178267998\\
5.80064655876654	0.334730604272961\\
5.81146486729419	0.334989027924879\\
5.82228317582184	0.33524545369102\\
5.83315845473553	0.335501220650941\\
5.84403373364921	0.335754977717714\\
5.8549090125629	0.336006729467256\\
5.86578429147658	0.336256480487385\\
5.87671744355088	0.336505548482283\\
5.88765059562518	0.33675260373101\\
5.89858374769948	0.336997650930735\\
5.90951689977378	0.337240694789868\\
5.92050884813521	0.337483030926078\\
5.93150079649664	0.337723351694168\\
5.94249274485807	0.337961661912629\\
5.9534846932195	0.338197966410527\\
5.96453638108293	0.338433537972472\\
5.97558806894636	0.338667091774528\\
5.98663975680979	0.3388986327566\\
5.99769144467322	0.339128165868498\\
6.00880383710542	0.339356940322845\\
6.01991622953761	0.339583694854479\\
6.0310286219698	0.339808434524849\\
6.042141014402	0.340031164404622\\
6.05331509935885	0.340253109384354\\
6.0644891843157	0.340473032506225\\
6.07566326927256	0.340690938953377\\
6.08683735422941	0.340906833917479\\
6.09807414230094	0.341121917178946\\
6.10931093037248	0.341334976874237\\
6.12054771844401	0.341546018308367\\
6.13178450651555	0.341755046794183\\
6.14308503319571	0.34196323622371\\
6.15438555987588	0.342169400604273\\
6.16568608655605	0.342373545362971\\
6.17698661323621	0.342575675934031\\
6.18835193888863	0.342776939506413\\
6.19971726454106	0.342976176771644\\
6.21108259019348	0.343173393279152\\
6.2224479158459	0.343368594584776\\
6.23387912710053	0.343562900348477\\
6.24531033835517	0.343755178770402\\
6.2567415496098	0.343945435522582\\
6.26817276086443	0.344133676282738\\
6.2796709718506	0.344320992338939\\
6.29116918283678	0.344506280241224\\
6.30266739382295	0.344689545784534\\
6.31416560480912	0.34487079476877\\
6.32573195844141	0.345051089248579\\
6.33729831207371	0.345229354983699\\
6.34886466570601	0.345405597892329\\
6.36043101933831	0.345579823896886\\
6.37206668825559	0.345753064930683\\
6.38370235717287	0.345924276849381\\
6.39533802609016	0.346093465694817\\
6.40697369500744	0.346260637512296\\
6.41867988207414	0.346426793191786\\
6.43038606914084	0.346590919605399\\
6.44209225620754	0.346753022919031\\
6.45379844327423	0.346913109301282\\
6.46557638449348	0.347072147674733\\
6.47735432571273	0.347229156849893\\
6.48913226693197	0.347384143117176\\
6.50091020815122	0.347537112768929\\
6.51276117329228	0.34768900179831\\
6.52461213843333	0.347838861914366\\
6.53646310357439	0.347986699532533\\
6.54831406871544	0.348132521069396\\
6.56023936237899	0.34827722859346\\
6.57216465604253	0.348419907705717\\
6.58408994970607	0.348560564947171\\
6.59601524336962	0.348699206859178\\
6.608016207688	0.348836700574142\\
6.62001717200638	0.348972166594166\\
6.63201813632476	0.349105611586413\\
6.64401910064313	0.349237042217592\\
6.65609711612489	0.349367289630542\\
6.66817513160664	0.349495510279849\\
6.68025314708839	0.349621710959481\\
6.69233116257014	0.349745898462128\\
6.70448764976455	0.349868866852974\\
6.71664413695897	0.349989809625034\\
6.72880062415338	0.350108733699768\\
6.7409571113478	0.350225645996529\\
6.75319353351478	0.350341302386607\\
6.76542995568177	0.350454934515177\\
6.77766637784875	0.350566549431939\\
6.78990280001574	0.350676154183636\\
6.80222066442117	0.350784465285828\\
6.8145385288266	0.350890753695308\\
6.82685639323203	0.350995026590819\\
6.83917425763746	0.351097291147287\\
6.85157511856051	0.351198223329145\\
6.86397597948357	0.351297134597414\\
6.87637684040662	0.351394032260747\\
6.88877770132968	0.351488923623095\\
6.90126316148383	0.351582442850996\\
6.91374862163799	0.351673943153843\\
6.92623408179214	0.35176343197112\\
6.93871954194629	0.35185091673672\\
6.95129125486549	0.351936988525834\\
6.96386296778468	0.352021043586922\\
6.97643468070388	0.352103089491291\\
6.98900639362307	0.352183133803746\\
7.00166606764739	0.352261723173649\\
7.0143257416717	0.352338298219662\\
7.02698541569601	0.35241286664598\\
7.03964508972033	0.352485436149371\\
7.05239448925109	0.352556507554748\\
7.06514388878186	0.352625567246572\\
7.07789328831262	0.352692623063063\\
7.09064268784339	0.35275768283407\\
7.10348363709426	0.352821200108498\\
7.11632458634514	0.352882708484721\\
7.12916553559601	0.352942215936205\\
7.14200648484689	0.352999730427078\\
7.15494087138423	0.353055656717751\\
7.16787525792157	0.353109577129254\\
7.18080964445891	0.353161499771601\\
7.19374403099625	0.353211432744486\\
7.20677380774895	0.353259730434536\\
7.21980358450165	0.353306025467216\\
7.23283336125435	0.35335032609048\\
7.24586313800706	0.353392640540957\\
7.25899032889946	0.35343327118199\\
7.27211751979186	0.353471902588641\\
7.28524471068426	0.353508543148295\\
7.29837190157667	0.353543201235983\\
7.31159860422458	0.353576125459992\\
7.32482530687249	0.35360705407256\\
7.3380520095204	0.353635995602097\\
7.35127871216831	0.353662958563605\\
7.36460710304072	0.353688135997173\\
7.37793549391314	0.35371132164063\\
7.39126388478556	0.353732524165109\\
7.40459227565797	0.353751752227266\\
7.41802461355885	0.353769141393207\\
7.43145695145972	0.353784542787452\\
7.4448892893606	0.353797965225683\\
7.45832162726148	0.353809417507998\\
7.47186025959003	0.353818975724922\\
7.48539889191859	0.35382655038389\\
7.49893752424715	0.353832150447075\\
7.51247615657571	0.35383578485994\\
7.52612352432791	0.353837468128802\\
7.53977089208011	0.353837172247032\\
7.55341825983232	0.353834906325377\\
7.56706562758452	0.353830679456719\\
7.58082427117841	0.35382444233925\\
7.5945829147723	0.353816230670218\\
7.60834155836619	0.353806053711175\\
7.62210020196009	0.353793920704617\\
7.63597276754062	0.353779716191534\\
7.64984533312116	0.353763541915752\\
7.66371789870169	0.353745407292014\\
7.67759046428223	0.353725321714787\\
7.69157971106386	0.353703101081861\\
7.70556895784549	0.353678915662686\\
7.71955820462712	0.35365277502775\\
7.73354745140875	0.353624688726011\\
7.7476562591015	0.353594401377075\\
7.76176506679424	0.353562154403583\\
7.77587387448699	0.353527957534508\\
7.78998268217974	0.353491820476003\\
7.80421406003167	0.353453413769471\\
7.81844543788359	0.353413052782574\\
7.83267681573552	0.35337074740571\\
7.84690819358745	0.353326507505128\\
7.86126528820976	0.353279926568446\\
7.87562238283208	0.353231396875465\\
7.8899794774544	0.353180928481159\\
7.90433657207672	0.353128531414989\\
7.91882267896367	0.353073718934123\\
7.93330878585062	0.353016963397582\\
7.94779489273757	0.352958275028313\\
7.96228099962452	0.352897664022334\\
7.97689957365268	0.35283456001449\\
7.99151814768085	0.352769518824679\\
8.00613672170902	0.35270255084747\\
8.02075529573718	0.352633666449041\\
8.03550996276884	0.352562208013784\\
8.0502646298005	0.352488818439665\\
8.06501929683216	0.3524135082968\\
8.07977396386382	0.352336288125407\\
8.09466853533245	0.352256409161702\\
8.10956310680107	0.352174605267227\\
8.12445767826969	0.352090887191889\\
8.13935224973832	0.352005265654125\\
8.15439073744049	0.351916896551001\\
8.16942922514267	0.351826608885354\\
8.18446771284484	0.351734413591447\\
8.19950620054702	0.351640321570461\\
8.2146928332504	0.351543388862596\\
8.22987946595378	0.351444544115211\\
8.24506609865717	0.351343798451879\\
8.26025273136055	0.351241162961403\\
8.27559197362809	0.351135588942947\\
8.29093121589562	0.351028109556632\\
8.30627045816316	0.350918736120692\\
8.32160970043069	0.350807479916851\\
8.33710627308224	0.35069318221295\\
8.35260284573378	0.350576985954729\\
8.36809941838533	0.3504589026609\\
8.38359599103687	0.350338943811842\\
8.39925489573158	0.350215834885777\\
8.41491380042629	0.350090834352854\\
8.430572705121	0.349963953938582\\
8.44623160981571	0.349835205328235\\
8.46205815432056	0.349703191942698\\
8.4778846988254	0.349569294022971\\
8.49371124333024	0.349433523508237\\
8.50953778783509	0.34929589229546\\
8.52553761739547	0.349154874881803\\
8.54153744695585	0.349011980121529\\
8.55753727651623	0.348867220175001\\
8.57353710607661	0.348720607158295\\
8.58971623619016	0.348570479122132\\
8.60589536630372	0.348418481030374\\
8.62207449641727	0.348264625272776\\
8.63825362653082	0.348108924192636\\
8.65461848330403	0.347949571096134\\
8.67098334007723	0.347788355324975\\
8.68734819685043	0.347625289507319\\
8.70371305362364	0.347460386222582\\
8.72018188980809	0.347292592831945\\
8.73665072599255	0.347122963961685\\
8.75311956217701	0.346951512283007\\
8.76958839836146	0.346778250417055\\
8.78614348346302	0.346602269421409\\
8.80269856856458	0.346424484650276\\
8.81925365366614	0.346244908771209\\
8.83580873876771	0.346063554400631\\
8.85245111606455	0.345879463883058\\
8.86909349336139	0.345693601480295\\
8.88573587065824	0.345505979853203\\
8.90237824795508	0.345316611610453\\
8.91910883699713	0.345124491786591\\
8.93583942603918	0.344930632176374\\
8.95257001508123	0.344735045430634\\
8.96930060412328	0.344537744146978\\
8.98612034063892	0.3443376759621\\
9.00294007715455	0.344135900289318\\
9.01975981367019	0.343932429766144\\
9.03657955018582	0.343727276975847\\
9.05348938408815	0.343519342098691\\
9.07039921799048	0.343309732223398\\
9.08730905189281	0.343098459970896\\
9.10421888579513	0.342885537906876\\
9.12121978296026	0.34266981868155\\
9.13822068012539	0.342452457130687\\
9.15522157729052	0.342233465855407\\
9.17222247445565	0.342012857400618\\
9.18931541587169	0.34178943683004\\
9.20640835728774	0.341564406781237\\
9.22350129870378	0.341337779832334\\
9.24059424011982	0.34110956850428\\
9.25778022278737	0.34087853021322\\
9.27496620545493	0.340645915457853\\
9.29215218812249	0.340411736790148\\
9.30933817079004	0.340176006703958\\
9.32661820779546	0.339937434912761\\
9.34389824480088	0.339697319829843\\
9.3611782818063	0.339455673977893\\
9.37845831881172	0.339212509820567\\
9.39583344002363	0.338966489311129\\
9.41320856123555	0.33871895883335\\
9.43058368244747	0.338469930877553\\
9.44795880365938	0.338219417874125\\
9.46543005570071	0.33796603396621\\
9.48290130774205	0.337711173556453\\
9.50037255978338	0.337454849099754\\
9.51784381182471	0.337197072990194\\
9.53541225878365	0.336936411508106\\
9.55298070574258	0.336674307126181\\
9.57054915270152	0.336410772260874\\
9.58811759966045	0.336145819266961\\
9.60578432346384	0.335877966511212\\
9.62345104726722	0.335608704585658\\
9.64111777107061	0.33533804586532\\
9.658784494874	0.335066002662702\\
9.67655059520445	0.334791045389354\\
9.6943166955349	0.334514712796984\\
9.71208279586536	0.334237017216225\\
9.72984889619581	0.33395797091437\\
9.74771549159443	0.333675996297084\\
9.76558208699305	0.333392680325048\\
9.78344868239167	0.333108035281588\\
9.80131527779029	0.332822073385884\\
9.81928350543996	0.332533168998522\\
9.83725173308963	0.332242957327139\\
9.8552199607393	0.331951450604866\\
9.87318818838897	0.331658660999901\\
9.89125920482513	0.331362914785642\\
9.90933022126129	0.331065895457601\\
9.92740123769745	0.330767615195854\\
9.94547225413361	0.330468086114786\\
9.96364723590061	0.330165586356243\\
9.9818222176676	0.329861847746816\\
9.99999719943459	0.329556882410705\\
10.0181721812016	0.329250702405675\\
10.0364523247657	0.328941537707875\\
10.0547324683298	0.328631168508104\\
10.0730126118939	0.328319606871895\\
10.0912927554579	0.32800686479762\\
10.1096792783833	0.327691124050744\\
10.1280658013086	0.327374213230212\\
10.146452324234	0.327056144340135\\
10.1648388471593	0.326736929316754\\
10.1833329877765	0.32641470168471\\
10.2018271283938	0.326091338480361\\
10.220321269011	0.325766851643661\\
10.2388154096282	0.325441253046017\\
10.257418428294	0.325112627928743\\
10.2760214469598	0.324782901807203\\
10.2946244656257	0.324452086554506\\
10.3132274842915	0.324120193974543\\
10.3319406634311	0.323785260991652\\
10.3506538425707	0.323449261633078\\
10.3693670217104	0.323112207702418\\
10.38808020085	0.322774110933404\\
10.4069048454226	0.322432959901501\\
10.4257294899953	0.322090777177028\\
10.4445541345679	0.321747574491434\\
10.4633787791406	0.321403363505679\\
10.4823162175897	0.321056084407515\\
10.5012536560388	0.320707808348494\\
10.5201910944879	0.320358546985316\\
10.539128532937	0.320008311903582\\
10.5581801172831	0.319654994873297\\
10.5772317016292	0.319300715656723\\
10.5962832859753	0.318945485833239\\
10.6153348703214	0.318589316910536\\
10.6345019773376	0.318230052199868\\
10.6536690843538	0.317869860114662\\
10.67283619137	0.317508752154434\\
10.6920032983861	0.317146739746438\\
10.7112873293394	0.31678161771814\\
10.7305713602926	0.316415603158784\\
10.7498553912458	0.316048707485504\\
10.769139422199	0.315680942042625\\
10.7885418041401	0.315310053135944\\
10.8079441860811	0.31493830656802\\
10.8273465680222	0.314565713671122\\
10.8467489499633	0.314192285704179\\
10.8662711360792	0.313815720418433\\
10.8857933221952	0.313438332362366\\
10.9053155083112	0.313060132780924\\
10.9248376944271	0.312681132845202\\
10.9444811647161	0.312298981718147\\
10.9641246350051	0.311916042727699\\
10.9837681052941	0.311532327029053\\
11.0034115755831	0.311147845703059\\
11.02317783773	0.310760199286064\\
11.0429440998769	0.31037179992363\\
11.0627103620238	0.309982658678798\\
11.0824766241707	0.309592786539793\\
11.102367213876	0.309199735384444\\
11.1222578035813	0.308805966207818\\
11.1421483932866	0.30841148997843\\
11.1620389829919	0.308016317589523\\
11.182055464688	0.307617952227662\\
11.202071946384	0.30721890377021\\
11.22208842808	0.306819183088806\\
11.2421049097761	0.306418800979379\\
11.2622488777205	0.306015211896388\\
11.282392845665	0.305610974640438\\
11.3025368136094	0.305206099983968\\
11.3226807815539	0.304800598623294\\
11.3429428732039	0.304392097920709\\
11.363204964854	0.303982984619067\\
11.3834670565041	0.303573269372035\\
11.4037291481542	0.303162962756927\\
11.4240073959591	0.302751747421393\\
11.444285643764	0.30233996074277\\
11.464563891569	0.301927613095161\\
11.4848421393739	0.301514714777642\\
11.5051410232308	0.301100855009043\\
11.5254399070878	0.300686463875236\\
11.5457387909448	0.300271551482049\\
11.5660376748017	0.299856127861549\\
11.5863613245615	0.299439695216006\\
11.6066849743212	0.299022769999426\\
11.627008624081	0.298605362059553\\
11.6473322738407	0.298187481171572\\
11.6676846007299	0.297768546422709\\
11.6880369276191	0.297349156792138\\
11.7083892545083	0.296929321878876\\
11.7287415813975	0.296509051210531\\
11.7491263024508	0.29608768429863\\
11.7695110235041	0.295665899160492\\
11.7898957445574	0.295243705155023\\
11.8102804656107	0.294821111570824\\
11.8307011258767	0.294397381546854\\
11.8511217861426	0.29397326898235\\
11.8715424464086	0.293548783004063\\
11.8919631066745	0.293123932669497\\
11.9124230970895	0.29269790768487\\
11.9328830875046	0.292271534933777\\
11.9533430779196	0.291844823318174\\
11.9738030683347	0.291417781671795\\
11.9943056417768	0.290989528968474\\
12.014808215219	0.290560962414\\
12.0353107886612	0.290132090692362\\
12.0558133621033	0.289702922420314\\
12.0763616495731	0.289272508285379\\
12.0969099370428	0.288841813404041\\
12.1174582245125	0.28841084624867\\
12.1380065119823	0.287979615225352\\
12.1586035338403	0.287547105020133\\
12.1792005556983	0.287114346406844\\
12.1997975775563	0.286681347652148\\
12.2203945994143	0.286248116957341\\
12.2410432774846	0.28581357510382\\
12.2616919555548	0.285378816454633\\
12.2823406336251	0.284943849076249\\
12.3029893116954	0.28450868097066\\
12.3236924799435	0.284072170945767\\
12.3443956481917	0.283635475048923\\
12.3650988164398	0.283198601151554\\
12.385801984688	0.282761557061474\\
12.406562398648	0.282323141398436\\
12.4273228126081	0.281884570132818\\
12.4480832265682	0.281445850945828\\
12.4688436405282	0.281006991455904\\
12.4896639845707	0.280566731774239\\
12.5104843286133	0.280126346136797\\
12.5313046726558	0.27968584203909\\
12.5521250166983	0.279245226914675\\
12.5730079121072	0.278803183918208\\
12.5938908075161	0.278361044019633\\
12.614773702925	0.277918814533008\\
12.6356565983338	0.277476502711232\\
12.6566046110283	0.277032736180209\\
12.6775526237227	0.276588901234942\\
12.6985006364172	0.276145005012026\\
12.7194486491116	0.275701054587669\\
12.7404642942679	0.275255623424692\\
12.7614799394242	0.274810151794982\\
12.7824955845805	0.274364646661425\\
12.8035112297368	0.273919114927277\\
12.8245969793381	0.273472077130744\\
12.8456827289394	0.273025026298366\\
12.8667684785407	0.272577969222857\\
12.887854228142	0.272130912638042\\
12.9090125143938	0.271682325353791\\
12.9301708006455	0.271233751970143\\
12.9513290868973	0.270785199112976\\
12.972487373149	0.270336673350004\\
12.9937205946389	0.269886592856412\\
13.0149538161287	0.269436552726199\\
13.0361870376186	0.268986559421558\\
13.0574202591085	0.268536619347224\\
13.0787307850171	0.268085101074123\\
13.1000413109258	0.267633649172994\\
13.1213518368345	0.267182269945318\\
13.1426623627431	0.266730969635813\\
13.1640525369264	0.266278068179419\\
13.1854427111097	0.265825258667718\\
13.206832885293	0.265372547244299\\
13.2282230594763	0.264919939996664\\
13.2496952043377	0.264465709132925\\
13.271167349199	0.264011595367981\\
13.2926394940603	0.263557604690189\\
13.3141116389217	0.263103743032486\\
13.3356680588802	0.26264823573358\\
13.3572244788388	0.262192870285215\\
13.3787808987974	0.261737652523039\\
13.400337318756	0.261282588227938\\
13.4219803036483	0.260825856675305\\
13.4436232885406	0.260369291337975\\
13.4652662734329	0.259912897901276\\
13.4869092583252	0.259456681996421\\
13.5086410871513	0.258998777582184\\
13.5303729159775	0.25854106337558\\
13.5521047448036	0.258083544913884\\
13.5738365736297	0.257626227680893\\
13.5956595174591	0.25716720102316\\
13.6174824612885	0.256708388206759\\
13.639305405118	0.256249794623064\\
13.6611283489474	0.255791425610594\\
13.6830446722324	0.255331326598683\\
13.7049609955174	0.254871464716247\\
13.7268773188025	0.254411845210797\\
13.7487936420875	0.253952473277606\\
13.770805608124	0.25349135102686\\
13.7928175741606	0.253030488860665\\
13.8148295401972	0.252569891884604\\
13.8368415062338	0.252109565152632\\
13.858951377984	0.251647468054081\\
13.8810612497342	0.251185653673952\\
13.9031711214845	0.25072412697774\\
13.9252809932347	0.250262892879915\\
13.9474910360997	0.249799868606288\\
13.9697010789646	0.249337149375078\\
13.9919111218296	0.248874740013437\\
14.0141211646945	0.248412645298086\\
14.0364336494706	0.24794874080385\\
14.0587461342466	0.247485163376988\\
14.0810586190226	0.247021917707969\\
14.1033711037986	0.246559008437414\\
14.1257883079129	0.24609426999268\\
14.1482055120272	0.245629880351585\\
14.1706227161414	0.245165844069488\\
14.1930399202557	0.244702165652479\\
14.2155641320733	0.244236638811304\\
14.2380883438909	0.243771482231322\\
14.2606125557085	0.243306700334276\\
14.283136767526	0.242842297493215\\
14.3057702854045	0.242376027169305\\
14.3284038032829	0.241910148294918\\
14.3510373211614	0.241444665159598\\
14.3736708390398	0.240979582004766\\
14.3964159769655	0.240512612409509\\
14.4191611148912	0.240046055192114\\
14.4419062528169	0.239579914511274\\
14.4646513907425	0.23911419447812\\
14.4875104777278	0.238646569186484\\
14.510369564713	0.238179376949864\\
14.5332286516983	0.237712621797379\\
14.5560877386835	0.237246307711141\\
14.5790631229979	0.236778069626931\\
14.6020385073122	0.23631028503228\\
14.6250138916265	0.235842957827942\\
14.6479892759408	0.235376091868219\\
14.6710833264109	0.234907283248964\\
14.6941773768809	0.234438948319463\\
14.7172714273509	0.23397109085325\\
14.740365477821	0.233503714577956\\
14.7635805844029	0.23303437707349\\
14.7867956909849	0.232565533232557\\
14.8100107975669	0.232097186702558\\
14.8332259041488	0.231629341085534\\
14.8565644820063	0.23115951570475\\
14.8799030598637	0.230690203742641\\
14.9032416377212	0.230221408721497\\
14.9265802155787	0.22975313411879\\
14.9500447068131	0.229282861244067\\
14.9735091980476	0.228813121332062\\
14.996973689282	0.228343917780924\\
15.0204381805165	0.227875253944519\\
15.0440310551214	0.227404573361216\\
15.0676239297264	0.226934445080867\\
15.0912168043314	0.226464872378387\\
15.1148096789363	0.225995858484948\\
15.1385334385982	0.225524809353674\\
15.16225719826	0.22505433166896\\
15.1859809579218	0.22458442858335\\
15.2097047175837	0.224115103206175\\
15.2335618956234	0.223643724112232\\
15.2574190736632	0.22317293541874\\
15.281276251703	0.222702740156674\\
15.3051334297427	0.222233141314326\\
15.3291265950906	0.221761470236792\\
15.3531197604384	0.221290408330729\\
15.3771129257863	0.220819958506295\\
15.4011060911341	0.220350123631496\\
15.4252378496272	0.219878197965946\\
15.4493696081203	0.219406900066679\\
15.4735013666134	0.218936232723742\\
15.4976331251065	0.218466198685558\\
15.5219061221034	0.217994055241025\\
15.5461791191004	0.217522557987944\\
15.5704521160973	0.217051709596907\\
15.5947251130942	0.216581512697403\\
15.619142035055	0.21610918771407\\
15.6435589570157	0.215637527184133\\
15.6679758789765	0.215166533659336\\
15.6923928009372	0.214696209650843\\
15.7169563780926	0.21422373879694\\
15.741519955248	0.213751950501495\\
15.7660835324033	0.213280847197963\\
15.7906471095587	0.212810431279739\\
15.8153601179615	0.212337849661189\\
15.8400731263644	0.211865968555518\\
15.8647861347672	0.211394790278403\\
15.88949914317	0.210924317105983\\
15.9143644081624	0.21045165924975\\
15.9392296731548	0.209979719716407\\
15.9640949381471	0.209508500704317\\
15.9889602031395	0.209038004372831\\
16.0139806003288	0.208565304258946\\
16.039000997518	0.208093340139634\\
16.0640213947072	0.207622114096371\\
16.0890417918965	0.207151628172135\\
16.1142202516767	0.206678919199192\\
16.1393987114569	0.206206963760335\\
16.1645771712372	0.205735763820526\\
16.1897556310174	0.205265321306747\\
16.2150951374624	0.204792636364864\\
16.2404346439074	0.204320722370305\\
16.2657741503524	0.203849581171848\\
16.2911136567974	0.203379214580812\\
16.3166172552097	0.202906585962857\\
16.342120853622	0.202434745585352\\
16.3676244520343	0.201963695181177\\
16.3931280504465	0.201493436446271\\
16.4187988465862	0.201020895917036\\
16.4444696427259	0.200549160807199\\
16.4701404388656	0.200078232733978\\
16.4958112350053	0.199608113278176\\
16.5216524004798	0.199135692029876\\
16.5474935659542	0.198664093271825\\
16.5733347314287	0.198193318505781\\
16.5991758969032	0.1977233691976\\
16.6251906694304	0.197251097901842\\
16.6512054419577	0.196779666065015\\
16.677220214485	0.196309075073559\\
16.7032349870123	0.195839326278541\\
16.7294266758655	0.195367235044649\\
16.7556183647187	0.194896000142285\\
16.781810053572	0.194425622842676\\
16.8080017424252	0.193956104382202\\
16.8343737319713	0.193484222753278\\
16.8607457215174	0.193013214238534\\
16.8871177110636	0.192543079994045\\
16.9134897006097	0.192073821141561\\
16.9400454511807	0.191602178140146\\
16.9666012017517	0.191131424951641\\
16.9931569523226	0.190661562616978\\
17.0197127028936	0.190192592143289\\
17.0464557582069	0.189721216201866\\
17.0731988135202	0.189250746694494\\
17.0999418688336	0.188781184546921\\
17.1266849241469	0.188312530651629\\
17.1536189111414	0.187841449681113\\
17.1805528981359	0.187371291694339\\
17.2074868851304	0.186902057501787\\
17.2344208721249	0.186433747881203\\
17.2615495082445	0.185962989209667\\
17.2886781443641	0.185493170006567\\
17.3158067804836	0.185024290966983\\
17.3429354166032	0.18455635275379\\
17.3702625121198	0.184085943155972\\
17.3975896076364	0.183616489452718\\
17.424916703153	0.183147992223521\\
17.4522437986696	0.182680452016204\\
17.4797732620512	0.182210417692499\\
17.5073027254328	0.181741355637536\\
17.5348321888143	0.181273266314979\\
17.5623616521959	0.180806150157367\\
17.5900974943972	0.180336516735021\\
17.6178333365984	0.179867871910363\\
17.6455691787997	0.179400216030943\\
17.6733050210009	0.178933549413724\\
17.7012513604556	0.178464341939359\\
17.7291976999103	0.177996139353281\\
17.757144039365	0.177528941886577\\
17.7850903788197	0.177062749740295\\
17.8132514459017	0.176593992684996\\
17.8414125129837	0.176126256777231\\
17.8695735800657	0.175659542131222\\
17.8977346471477	0.175193848831702\\
17.9261147917276	0.174725566048351\\
17.9544949363076	0.174258320647764\\
17.9828750808876	0.173792112626836\\
18.0112552254676	0.173326941953531\\
18.0398589202161	0.172859156709629\\
18.0684626149646	0.172392425067064\\
18.097066309713	0.171926746904883\\
18.1256700044615	0.171462122073762\\
18.1545018545664	0.170994856982699\\
18.1833337046712	0.17052866170274\\
18.212165554776	0.1700635359945\\
18.2409974048809	0.169599479590788\\
18.2700621491713	0.169132756688909\\
18.2991268934616	0.1686671198068\\
18.328191637752	0.168202568585993\\
18.3572563820424	0.167739102640785\\
18.3865589068877	0.167272943261795\\
18.4158614317331	0.166807886118634\\
18.4451639565785	0.166343930733037\\
18.4744664814239	0.165881076600075\\
18.5040118216459	0.16541550146561\\
18.5335571618679	0.164951044798749\\
18.56310250209	0.164487706000636\\
18.592647842312	0.164025484446332\\
18.622441195706	0.16356051354449\\
18.6522345491	0.163096677366889\\
18.682027902494	0.162633975193216\\
18.711821255888	0.162172406277669\\
18.74186798467	0.161708058955749\\
18.771914713452	0.161244862648557\\
18.801961442234	0.160782816513382\\
18.8320081710161	0.16032191968262\\
18.8623138179204	0.159858214516479\\
18.8926194648247	0.159395676699333\\
18.922925111729	0.158934305265045\\
18.9532307586333	0.15847409922319\\
18.9838010512769	0.158011054078012\\
19.0143713439205	0.157549192669835\\
19.0449416365641	0.157088513907985\\
19.0755119292076	0.15662901667811\\
19.1063527942164	0.156166648626118\\
19.1371936592252	0.155705480763607\\
19.1680345242339	0.15524551187416\\
19.1988753892427	0.154786740718303\\
19.2299929618915	0.15432506603507\\
19.2611105345403	0.153864608069417\\
19.2922281071891	0.153405365477882\\
19.3233456798379	0.152947336894575\\
19.3547463174076	0.152486371008786\\
19.3861469549774	0.152026638456125\\
19.4175475925471	0.151568137764682\\
19.4489482301169	0.151110867440754\\
19.4806385226382	0.150650624933263\\
19.5123288151595	0.150191632474196\\
19.5440191076808	0.149733888461677\\
19.5757094002021	0.149277391272688\\
19.6076961882754	0.148817885782969\\
19.6396829763487	0.148359647170015\\
19.671669764422	0.14790267370035\\
19.7036565524953	0.147446963620022\\
19.7359469398787	0.146988207842406\\
19.768237327262	0.146530735896805\\
19.8005277146453	0.146074545916391\\
19.8328181020287	0.145619636014534\\
19.8654194755566	0.145161641602166\\
19.8980208490846	0.144704948118971\\
19.9306222226126	0.144249553562878\\
19.9632235961405	0.143795455912702\\
19.9961436388709	0.143338233492925\\
20.0290636816012	0.142882329256943\\
20.0619837243316	0.142427741065415\\
20.0949037670619	0.14197446676058\\
20.1281504830269	0.141518025802194\\
20.1613971989918	0.14106292045671\\
20.1946439149567	0.140609148445331\\
20.2278906309217	0.140156707471556\\
20.2614723645969	0.139701056245672\\
20.2950540982721	0.139246758254312\\
20.3286358319473	0.138793811076874\\
20.3622175656225	0.138342212275782\\
20.3961430262838	0.137887357768135\\
20.4300684869452	0.137433874328545\\
20.4639939476066	0.136981759392079\\
20.497919408268	0.136531010377581\\
20.5321976950164	0.136076958213533\\
20.5664759817649	0.135624295183764\\
20.6007542685134	0.135173018576299\\
20.6350325552618	0.1347231256637\\
20.6696731863085	0.134269879999661\\
20.7043138173553	0.133818041791286\\
20.738954448402	0.133367608176628\\
20.7735950794487	0.132918576279064\\
20.8086080231198	0.13246613969785\\
20.8436209667909	0.13201512917309\\
20.878633910462	0.13156554168972\\
20.9136468541332	0.131117374218806\\
20.949042565344	0.130665747582995\\
20.9844382765548	0.130215565910258\\
21.0198339877656	0.129766826029021\\
21.0552296989764	0.129319524754664\\
21.0910191546516	0.128868707106104\\
21.1268086103268	0.128419353661522\\
21.162598066002	0.127971461089174\\
21.1983875216773	0.127525026045121\\
21.234582265905	0.127075014415341\\
21.2707770101326	0.126626486595951\\
21.3069717543603	0.126179439091079\\
21.343166498588	0.12573386839354\\
21.3797786897193	0.125284657631442\\
21.4163908808506	0.12483695068572\\
21.453003071982	0.124390743892108\\
21.4896152631133	0.123946033575927\\
21.5266577262528	0.123497616153184\\
21.5637001893923	0.12305072298968\\
21.6007426525319	0.122605350248122\\
21.6377851156714	0.122161494081743\\
21.6752714017348	0.121713859867902\\
21.7127576877983	0.121267770834002\\
21.7502439738617	0.12082322296471\\
21.7877302599251	0.120380212236188\\
21.8256747161126	0.119933348193819\\
21.8636191723001	0.11948805077542\\
21.9015636284876	0.119044315782171\\
21.9395080846751	0.118602139007748\\
21.9779259262067	0.118156028938286\\
22.0163437677384	0.117711507510155\\
22.0547616092701	0.117268570335117\\
22.0931794508017	0.116827213018478\\
22.1320868496398	0.116381837180327\\
22.1709942484778	0.115938072630002\\
22.2099016473158	0.115495914783389\\
22.2488090461539	0.115055359050999\\
22.2882232273236	0.114610693774996\\
22.3276374084934	0.114167663124556\\
22.3670515896632	0.113726262312625\\
22.4064657708329	0.113286486547905\\
22.4464051256478	0.112842503755477\\
22.4863444804626	0.112400179687246\\
22.5262838352775	0.111959509345484\\
22.5662231900924	0.111520487729402\\
22.6067074061995	0.111077154373666\\
22.6471916223066	0.110635504680106\\
22.6876758384138	0.110195533431821\\
22.7281600545209	0.109757235410092\\
22.7692102626757	0.109314512860207\\
22.8102604708306	0.108873499838262\\
22.8513106789854	0.108434190898826\\
22.8923608871403	0.107996580595951\\
22.9339998357608	0.107554423866854\\
22.9756387843813	0.107114003559798\\
23.0172777330018	0.106675313990476\\
23.0589166816223	0.106238349475439\\
23.1011689388352	0.10579670633816\\
23.143421196048	0.105356827660086\\
23.1856734532609	0.104918707506568\\
23.2279257104737	0.104482339945257\\
23.2707449436263	0.104041898790921\\
23.3135641767789	0.103603245321071\\
23.3563834099314	0.103166373369506\\
23.399202643084	0.102731276773833\\
23.4425351018712	0.102292766275955\\
23.4858675606585	0.101856061262585\\
23.5292000194457	0.101421155359914\\
23.5725324782329	0.100988042199476\\
23.6163923812599	0.100551476271452\\
23.6602522842869	0.100116733883224\\
23.704112187314	0.0996838084485655\\
23.747972090341	0.0992526933881746\\
23.7923730104802	0.0988180977594291\\
23.8367739306194	0.0983853438968916\\
23.8811748507586	0.0979544249973548\\
23.9255757708978	0.0975253342661745\\
23.9705318120247	0.0970927344965931\\
24.0154878531517	0.0966619949036492\\
24.0604438942787	0.0962331084623396\\
24.1053999354057	0.0958060681579141\\
24.1509257391711	0.0953754898552104\\
24.1964515429365	0.0949467903352574\\
24.2419773467019	0.0945199623461755\\
24.2875031504673	0.0940949986480839\\
24.3336139249026	0.0936664674795499\\
24.3797246993379	0.0932398339077079\\
24.4258354737732	0.0928150904484404\\
24.4719462482085	0.0923922296314368\\
24.518657797687	0.0919657713411552\\
24.5653693471654	0.091541229681944\\
24.6120808966439	0.0911185969317909\\
24.6587924461223	0.0906978653843623\\
24.7061212033434	0.0902735058106846\\
24.7534499605645	0.0898510821359996\\
24.8007787177855	0.0894305863944291\\
24.8481074750066	0.0890120106377129\\
24.8960705351928	0.0885897757308398\\
24.944033595379	0.0881694962380671\\
24.9919966555653	0.0877511639433448\\
25.0399597157515	0.0873347706502545\\
25.0885748730688	0.0869146864914475\\
25.1371900303861	0.0864965775232338\\
25.1858051877034	0.0860804352727323\\
25.2344203450208	0.0856662512887845\\
25.2837061315614	0.085248344109904\\
25.332991918102	0.0848324321742945\\
25.3822777046427	0.0844185067452075\\
25.4315634911833	0.0840065591097906\\
25.4815392186928	0.0835908553135082\\
25.5315149462023	0.0831771671047976\\
25.5814906737117	0.0827654854756056\\
25.6314664012212	0.082355801444037\\
25.6821522057685	0.0819423276250628\\
25.7328380103157	0.0815308900457019\\
25.783523814863	0.0811214794187319\\
25.8342096194102	0.0807140864854452\\
25.8856265092156	0.0803028694531365\\
25.9370433990211	0.07989370963706\\
25.9884602888265	0.0794865974625047\\
26.0398771786319	0.0790815233857315\\
26.0920470854784	0.0786725901869542\\
26.144216992325	0.0782657355232165\\
26.1963868991715	0.0778609495235137\\
26.2485568060181	0.0774582223503776\\
26.3015026403881	0.0770516002940746\\
26.3544484747581	0.0766470784522385\\
26.4073943091281	0.0762446466482464\\
26.4603401434981	0.0758442947416918\\
26.5140858539001	0.0754400114242292\\
26.5678315643021	0.0750378503805907\\
26.6215772747041	0.0746378011186559\\
26.6753229851061	0.0742398531853248\\
26.7298936218998	0.0738379365172639\\
26.7844642586936	0.0734381645825617\\
26.8390348954873	0.0730405265631278\\
26.893605532281	0.072645011682828\\
26.9490273164448	0.0722454899170071\\
27.0044491006085	0.0718481357655078\\
27.0598708847722	0.0714529380731599\\
27.1152926689359	0.0710598857298282\\
27.1715930662283	0.0706627874909405\\
27.2278934635207	0.0702678801910874\\
27.2841938608131	0.0698751523262278\\
27.3404942581055	0.0694845924405868\\
27.3972810276665	0.0690928431203327\\
27.4540677972275	0.0687032761377278\\
27.5108545667884	0.0683158798868341\\
27.5676413363494	0.0679306428118702\\
27.6246835621125	0.0675458349084529\\
27.6817257878755	0.0671631824309446\\
27.7387680136386	0.0667826738197663\\
27.7958102394016	0.0664042975664976\\
27.8531117250382	0.0660263369381029\\
27.9104132106748	0.0656505049712132\\
27.9677146963114	0.0652767901556403\\
28.025016181948	0.0649051810332829\\
28.0825792753208	0.0645339839774108\\
28.1401423686935	0.0641648888651242\\
28.1977054620663	0.063797884239309\\
28.2552685554391	0.0634329586957921\\
28.3130956457549	0.0630684414838479\\
28.3709227360708	0.062705999558175\\
28.4287498263866	0.0623456215181631\\
28.4865769167025	0.0619872960169257\\
28.5446704260258	0.0616293750069651\\
28.602763935349	0.0612735026991277\\
28.6608574446723	0.0609196677525006\\
28.7189509539956	0.0605678588806107\\
28.7773133374	0.060216450567141\\
28.8356757208045	0.059867064456519\\
28.8940381042089	0.0595196892705004\\
28.9524004876134	0.0591743137859332\\
29.0110342352743	0.0588293348349972\\
29.0696679829352	0.0584863516835368\\
29.1283017305961	0.0581453531187361\\
29.186935478257	0.0578063279834646\\
29.2458431140456	0.0574676952861799\\
29.3047507498342	0.0571310320910557\\
29.3636583856228	0.0567963272532655\\
29.4225660214114	0.0564635696842052\\
29.4817501049045	0.0561312003868851\\
29.5409341883977	0.0558007744100196\\
29.6001182718908	0.0554722806791449\\
29.659302355384	0.0551457081765024\\
29.7187654815208	0.0548195197222532\\
29.7782286076576	0.0544952485311907\\
29.8376917337944	0.0541728836014093\\
29.8971548599312	0.0538524139881413\\
29.9568996607722	0.0535323241440945\\
30.0166444616132	0.0532141256386846\\
30.0763892624543	0.0528978075445911\\
30.1361340632953	0.052583358992017\\
30.1961632086509	0.0522692858810846\\
30.2561923540065	0.0519570783246439\\
30.316221499362	0.0516467254718283\\
30.3762506447176	0.0513382165296351\\
30.436566841334	0.0510300786674595\\
30.4968830379503	0.050723780723074\\
30.5571992345667	0.0504193119237856\\
30.617515431183	0.0501166615550638\\
30.6781214246657	0.0498143778702777\\
30.7387274181483	0.0495139086206179\\
30.799333411631	0.0492152431131442\\
30.8599394051136	0.0489183707133366\\
30.9208379817972	0.0486218605671254\\
30.9817365584807	0.0483271395335823\\
31.0426351351642	0.0480341970009647\\
31.1035337118477	0.0477430224161711\\
31.1647276970968	0.0474522056377892\\
31.2259216823458	0.0471631528153783\\
31.2871156675948	0.046875853419714\\
31.3483096528438	0.0465902969803985\\
31.4098019130839	0.0463050938820847\\
31.4712941733241	0.0460216297540013\\
31.5327864335642	0.0457398941506444\\
31.5942786938043	0.0454598766854891\\
31.656072138367	0.0451802080775701\\
31.7178655829298	0.0449022536299379\\
31.7796590274925	0.0446260029819003\\
31.8414524720552	0.0443514458318645\\
31.903550051626	0.0440772330520602\\
31.9656476311968	0.0438047098027155\\
32.0277452107676	0.0435338658089366\\
32.0898427903383	0.0432646908550199\\
32.1522474993513	0.0429958557778903\\
32.2146522083642	0.0427286857855521\\
32.2770569173772	0.042463170689798\\
32.3394616263901	0.0421993003616745\\
32.4021765035977	0.0419357654154711\\
32.4648913808053	0.0416738712962329\\
32.527606258013	0.0414136079032348\\
32.5903211352206	0.0411549651950425\\
32.6533492654047	0.040896653372528\\
32.7163773955889	0.0406399583103966\\
32.7794055257731	0.0403848699961131\\
32.8424336559572	0.0401313784764463\\
32.9057781690946	0.0398782133552533\\
32.969122682232	0.0396266411221125\\
33.0324671953693	0.0393766518533053\\
33.0958117085067	0.0391282356844071\\
33.1594757818323	0.0388801414340439\\
33.2231398551579	0.0386336163964348\\
33.2868039284835	0.0383886507372277\\
33.3504680018091	0.0381452346813323\\
33.4144548595075	0.0379021360788979\\
33.478441717206	0.0376605832134043\\
33.5424285749044	0.0374205663403436\\
33.6064154326028	0.0371820757744173\\
33.6707283499096	0.0369438982024084\\
33.7350412672163	0.0367072430933637\\
33.7993541845231	0.0364721007930284\\
33.8636671018298	0.0362384617062859\\
33.9283094023156	0.0360051311758732\\
33.9929517028013	0.0357733000382193\\
34.057594003287	0.0355429587296684\\
34.1222363037727	0.0353140977456136\\
34.1872113622324	0.0350855408973816\\
34.2521864206922	0.0348584605772776\\
34.3171614791519	0.0346328473125311\\
34.3821365376116	0.0344086916893149\\
34.447447781095	0.0341848358006727\\
34.5127590245783	0.0339624337826963\\
34.5780702680616	0.0337414762537306\\
34.643381511545	0.0335219538909421\\
34.7090324201268	0.0333027268837025\\
34.7746833287086	0.033084931298219\\
34.8403342372905	0.0328685578441295\\
34.9059851458723	0.0326535972897569\\
34.9719792533529	0.03243892773715\\
35.0379733608335	0.0322256673671259\\
35.1039674683141	0.0320138069807437\\
35.1699615757948	0.0318033374375982\\
35.2363024710846	0.0315931545683351\\
35.3026433663745	0.0313843588532426\\
35.3689842616644	0.0311769411848841\\
35.4353251569542	0.0309708925141954\\
35.5020164861975	0.0307651262138568\\
35.5687078154407	0.0305607252509422\\
35.6353991446839	0.0303576806095582\\
35.7020904739271	0.0301559833320087\\
35.76913594038	0.029954564151126\\
35.8361814068328	0.0297544887032842\\
35.9032268732857	0.0295557480641325\\
35.9702723397385	0.0293583333673307\\
36.0376757049608	0.0291611925245332\\
36.1050790701831	0.0289653740233161\\
36.1724824354054	0.0287708690308321\\
36.2398858006277	0.0285776687720481\\
36.3076508877251	0.0283847381509651\\
36.3754159748225	0.0281931086934294\\
36.4431810619199	0.0280027716580229\\
36.5109461490173	0.0278137183609351\\
36.5790768402747	0.0276249305231105\\
36.6472075315322	0.0274374228844671\\
36.7153382227897	0.0272511867949097\\
36.7834689140472	0.0270662136617344\\
36.8519691566203	0.0268815018358666\\
36.9204693991935	0.0266980494587742\\
36.9889696417667	0.0265158479715465\\
37.0574698843398	0.0263348888724397\\
37.1263436888854	0.0261541869637076\\
37.1952174934311	0.0259747239673853\\
37.2640912979767	0.0257964914155799\\
37.3329651025223	0.0256194808973328\\
37.4022165438372	0.0254427234900986\\
37.4714679851522	0.025267184672904\\
37.5407194264671	0.0250928560686807\\
37.609970867782	0.024919729357055\\
37.6796040887873	0.0247468517087135\\
37.7492373097926	0.0245751725419909\\
37.8188705307979	0.0244046835704249\\
37.8885037518033	0.0242353765640005\\
37.9585229632751	0.024066314609883\\
38.0285421747469	0.0238984312427167\\
38.0985613862187	0.0237317182664035\\
38.1685805976905	0.0235661675410393\\
38.2389900798134	0.0234008578941039\\
38.3093995619363	0.0232367071529308\\
38.3798090440592	0.0230737072115226\\
38.4502185261821	0.0229118500198161\\
38.5209622984945	0.0227503671991258\\
38.591706070807	0.0225900218819899\\
38.6624498431194	0.0224308060726302\\
38.7331936154318	0.0222727118307489\\
38.8042656863826	0.0221150053597762\\
38.8753377573333	0.0219584149920867\\
38.9464098282841	0.0218029328443683\\
39.0174818992348	0.0216485510883038\\
39.0888909542047	0.0214945377290905\\
39.1603000091745	0.0213416195443194\\
39.2317090641443	0.0211897887604163\\
39.3031181191142	0.0210390376583315\\
39.3748725127202	0.0208886373081231\\
39.4466269063262	0.0207393116470363\\
39.5183812999322	0.0205910530087338\\
39.5901356935382	0.020443853780945\\
39.662243568804	0.0202969890345548\\
39.7343514440698	0.0201511789128529\\
39.8064593193355	0.020006415854414\\
39.8785671946013	0.0198626923514337\\
39.95103651206	0.019719288293001\\
40.0235058295186	0.0195769191952447\\
40.0959751469772	0.0194355775994849\\
40.1684444644358	0.0192952561002276\\
40.2412830193083	0.0191552401353303\\
40.3141215741807	0.0190162398487182\\
40.3869601290532	0.0188782478824356\\
40.4597986839256	0.018741256931288\\
40.5330141309314	0.0186045586173037\\
40.6062295779373	0.0184688570637908\\
40.6794450249431	0.018334145011627\\
40.752660471949	0.0182004152540358\\
40.8262603410054	0.0180669661611227\\
40.8998602100619	0.0179344952597792\\
40.9734600791183	0.0178029953879457\\
41.0470599481748	0.0176724594355011\\
41.1210516636287	0.0175421930117619\\
41.1950433790826	0.0174128865453105\\
41.2690350945364	0.0172845329694876\\
41.3430268099903	0.0171571252691729\\
41.4174177047231	0.017029976725388\\
41.4918085994559	0.0169037702260774\\
41.5661994941887	0.016778498798419\\
41.6405903889215	0.0166541555207375\\
41.7153877210576	0.0165300617194128\\
41.7901850531936	0.0164068923591898\\
41.8649823853296	0.0162846405596109\\
41.9397797174656	0.0161632994909792\\
42.0149906816255	0.0160421988530005\\
42.0902016457854	0.0159220053510732\\
42.1654126099454	0.0158027121957142\\
42.2406235741053	0.0156843126478218\\
42.316255315397	0.0155661450615886\\
42.3918870566888	0.0154488675944716\\
42.4675187979805	0.015332473546648\\
42.5431505392722	0.0152169562683023\\
42.6192101664345	0.0151016630072433\\
42.6952697935968	0.0149872431270977\\
42.771329420759	0.0148736900164575\\
42.8473890479213	0.0147609971135529\\
42.9238836429179	0.014648520765266\\
43.0003782379145	0.0145369013297238\\
43.0768728329111	0.0144261322827509\\
43.1533674279077	0.0143162071494459\\
43.2303040564086	0.0142064915486593\\
43.3072406849096	0.0140976166544416\\
43.3841773134105	0.0139895760287258\\
43.4611139419115	0.0138823632823593\\
43.5384996649489	0.0137753534472804\\
43.6158853879863	0.0136691683671809\\
43.6932711110238	0.0135638016890316\\
43.7706568340612	0.0134592471083613\\
43.8484987166093	0.0133548891861251\\
43.9263405991575	0.0132513403149832\\
44.0041824817057	0.0131485942259212\\
44.0820243642538	0.0130466446981303\\
44.1603294838729	0.0129448859140818\\
44.2386346034919	0.0128439207185526\\
44.3169397231109	0.0127437429255651\\
44.39524484273	0.0126443463969986\\
44.4740202988846	0.0125451350068203\\
44.5527957550393	0.0124467019779694\\
44.631571211194	0.0123490412065684\\
44.7103466673487	0.0122521466362502\\
44.7895995895968	0.0121554318830841\\
44.8688525118449	0.012059480493921\\
44.948105434093	0.0119642864460834\\
45.0273583563411	0.0118698437640616\\
45.1070959123988	0.0117755758389066\\
45.1868334684566	0.0116820565051607\\
45.2665710245144	0.011589279820481\\
45.3463085805721	0.0114972398893524\\
45.4265379845256	0.0114053698942014\\
45.5067673884792	0.0113142339378126\\
45.5869967924327	0.0112238261573441\\
45.6672261963862	0.0111341407364435\\
45.7479547148514	0.0110446206524354\\
45.8286832333165	0.0109558202699974\\
45.9094117517817	0.0108677338049836\\
45.9901402702469	0.0107803555194021\\
46.0713752310891	0.0106931381743752\\
46.1526101919314	0.0106066264050041\\
46.2338451527736	0.0105208145050601\\
46.3150801136159	0.0104356968141352\\
46.3968289125376	0.0103507358549187\\
46.4785777114593	0.0102664665527976\\
46.560326510381	0.0101828832787067\\
46.6420753093027	0.0100999804490691\\
46.7243454163926	0.0100172303157422\\
46.8066155234824	0.00993515812461587\\
46.8888856305723	0.00985375832305621\\
46.9711557376621	0.00977302540358781\\
47.0539547042701	0.00969244130522011\\
47.136753670878	0.00961252163435763\\
47.2195526374859	0.00953326091408621\\
47.3023516040939	0.00945465371232191\\
47.3856870701744	0.00937619160401383\\
47.4690225362549	0.00929838060546114\\
47.5523580023354	0.0092212153147767\\
47.6356934684159	0.0091446903745766\\
47.7195731680798	0.0090683069378295\\
47.8034528677438	0.00899256148694782\\
47.8873325674077	0.00891744869439579\\
47.9712122670716	0.0088429632768153\\
48.0556440359198	0.00876861589960235\\
48.1400758047679	0.00869489357532088\\
48.2245075736161	0.00862179105012655\\
48.3089393424642	0.00854930311402834\\
48.3939311234603	0.00847694987359231\\
48.4789229044563	0.00840520894135161\\
48.5639146854524	0.00833407513650779\\
48.6489064664485	0.00826354332179252\\
48.7344663172648	0.00819314296736795\\
48.820026168081	0.00812334236199675\\
48.9055860188973	0.00805413639729391\\
48.9911458697136	0.00798552000808233\\
49.0772819670679	0.00791703194729312\\
49.1634180644223	0.00784913125950601\\
49.2495541617767	0.00778181290812918\\
49.3356902591311	0.00771507189945753\\
49.4224109073406	0.00764845618219986\\
49.5091315555501	0.00758241564262917\\
49.5958522037596	0.0075169453153375\\
49.6825728519691	0.00745204027748329\\
49.769886487838	0.00738725758383819\\
49.857200123707	0.00732303805103374\\
49.944513759576	0.00725937678424665\\
50.0318273954449	0.00719626893090053\\
50.1197425956162	0.00713328055794548\\
50.2076577957875	0.00707084350529745\\
50.2955729959588	0.00700895294812779\\
50.3834881961302	0.006947604103536\\
50.4720136825001	0.00688637195446144\\
50.5605391688701	0.00682567945939383\\
50.64906465524	0.00676552186291759\\
50.73759014161	0.00670589445122728\\
50.8267347882915	0.00664638102403622\\
50.915879434973	0.00658739575679509\\
51.0050240816545	0.00652893396292757\\
51.094168728336	0.00647099099715006\\
51.1839415683769	0.00641315937429395\\
51.2737144084178	0.00635584458766036\\
51.3634872484587	0.00629904201894539\\
51.4532600884996	0.00624274709082098\\
51.5436703203087	0.00618656092969492\\
51.6340805521178	0.00613088044954613\\
51.7244907839268	0.00607570109978263\\
51.8149010157359	0.00602101837047195\\
51.9059580087448	0.00596644189494575\\
51.9970150017537	0.00591236011183722\\
52.0880719947626	0.00585876853771182\\
52.1791289877716	0.00580566272947868\\
52.270842290771	0.00575266072031911\\
52.3625555937704	0.00570014257998112\\
52.4542688967698	0.0056481038916383\\
52.5459821997693	0.0055965402784926\\
52.6383615465513	0.00554507806600232\\
52.7307408933333	0.00549408906201666\\
52.8231202401153	0.00544356891577305\\
52.9154995868973	0.00539351331622227\\
53.0085549039533	0.00534355677210784\\
53.1016102210094	0.00529406293783032\\
53.1946655380654	0.00524502752815098\\
53.2877208551214	0.00519644629722989\\
53.3814622684928	0.00514796182744177\\
53.4752036818643	0.00509992972888165\\
53.5689450952358	0.00505234578129876\\
53.6626865086073	0.00500520580352685\\
53.7571243510415	0.00495816034133276\\
53.8515621934757	0.0049115570702637\\
53.9460000359099	0.00486539183452456\\
54.0404378783442	0.0048196605170909\\
54.1355826971555	0.00477402151618963\\
54.2307275159668	0.00472881468330244\\
54.325872334778	0.00468403592656098\\
54.4210171535893	0.00463968119255395\\
54.5168797176294	0.00459541662100952\\
54.6127422816695	0.00455157434987212\\
54.7086048457095	0.00450815035067431\\
54.8044674097496	0.00446514063309236\\
54.9010587178797	0.00442221896687616\\
54.9976500260098	0.00437970988751067\\
55.0942413341398	0.00433760942940617\\
55.1908326422699	0.00429591366480362\\
55.2881639308243	0.00425430388215577\\
55.3854952193786	0.00421309712542096\\
55.4828265079329	0.00417228949136663\\
55.5801577964873	0.00413187711427803\\
55.6782405480441	0.00409154868977865\\
55.776323299601	0.00405161388146948\\
55.8744060511579	0.00401206884795674\\
55.9724888027147	0.00397290978505185\\
56.0713347542828	0.00393383268425837\\
56.1701807058509	0.00389513993976209\\
56.2690266574189	0.00385682777149173\\
56.367872608987	0.00381889243626879\\
56.4674937609263	0.00378103711031494\\
56.5671149128656	0.0037435570292309\\
56.6667360648049	0.00370644847375344\\
56.7663572167441	0.00366970776119992\\
56.8667658411891	0.00363304514184616\\
56.9671744656341	0.00359674880304933\\
57.0675830900791	0.00356081508584158\\
57.1679917145241	0.00352524036752361\\
57.2692003658787	0.00348974186167447\\
57.3704090172333	0.00345460081785752\\
57.4716176685879	0.00341981363688904\\
57.5728263199425	0.00338537675554203\\
57.6748478434472	0.00335101424074144\\
57.7768693669519	0.00331700051391645\\
57.8788908904565	0.00328333203515771\\
57.9809124139612	0.00325000530020089\\
58.0837599560178	0.00321675111969234\\
58.1866074980744	0.00318383719626681\\
58.2894550401309	0.00315126004878082\\
58.3923025821875	0.00311901623142431\\
58.4959896007833	0.00308684318942249\\
58.5996766193791	0.00305500201548344\\
58.7033636379749	0.00302348928672215\\
58.8070506565707	0.0029923016152756\\
58.911590931702	0.00296118297244937\\
59.0161312068333	0.00293038794925523\\
59.1206714819646	0.00289991318056028\\
59.2252117570959	0.00286975533594234\\
59.3306194021433	0.00283966480480324\\
59.4360270471907	0.00280988978418381\\
59.541434692238	0.0027804269661979\\
59.6468423372854	0.00275127307735885\\
59.7351317529641	0.00272708917638785\\
59.8234211686427	0.00270311785337353\\
59.9117105843214	0.00267935723985865\\
60	0.00265580548369465\\
};
\addlegendentry{Infetti}

\addplot [color=mycolor3, line width=2.0pt]
  table[row sep=crcr]{%
0	0\\
0.00199053585276749	1.18268676822518e-11\\
0.00398107170553497	9.45870381888635e-11\\
0.00597160755830246	3.19135881054745e-10\\
0.00796214341106994	7.56244247397302e-10\\
0.0123567669950473	2.82487874236936e-09\\
0.0167513905790247	7.03316243305228e-09\\
0.0211460141630021	1.41382679042843e-08\\
0.0255406377469794	2.4895348175285e-08\\
0.0299388563931551	4.0071947192466e-08\\
0.0343370750393307	6.04138179409832e-08\\
0.0387352936855064	8.66718883575852e-08\\
0.043133512331682	1.19595055569276e-07\\
0.0475471725191068	1.60085729869166e-07\\
0.0519608327065316	2.08793203726817e-07\\
0.0563744928939564	2.66468120257686e-07\\
0.0607881530813811	3.33859058778606e-07\\
0.0652173199794049	4.1200538178731e-07\\
0.0696464868774286	5.01440209468244e-07\\
0.0740756537754524	6.0291377108345e-07\\
0.0785048206734761	7.17174200368608e-07\\
0.0829496187548012	8.45443381288955e-07\\
0.0873944168361263	9.88092996188873e-07\\
0.0918392149174515	1.14587276190813e-06\\
0.0962840129987766	1.31953026708491e-06\\
0.100744568101794	1.51051597929865e-06\\
0.105205123204811	1.71899422092732e-06\\
0.109665678307829	1.94571409519882e-06\\
0.114126233410846	2.19142254549829e-06\\
0.118602673075784	2.45784577230395e-06\\
0.123079112740721	2.74489373242211e-06\\
0.127555552405659	3.05331481481662e-06\\
0.132031992070596	3.38385521609784e-06\\
0.136524445545197	3.73856512124059e-06\\
0.141016899019798	4.11705201686281e-06\\
0.1455093524944	4.52006347543719e-06\\
0.150001805969001	4.94834484434845e-06\\
0.154510404254387	5.40431967528308e-06\\
0.159019002539772	5.8872437526294e-06\\
0.163527600825158	6.39786372901713e-06\\
0.168036199110544	6.93692400001553e-06\\
0.172561074936287	7.5072720138803e-06\\
0.17708595076203	8.10776147478821e-06\\
0.181610826587773	8.73913801193327e-06\\
0.186135702413515	9.40214496444403e-06\\
0.190676990405524	1.01001054061685e-05\\
0.195218278397533	1.08314193895812e-05\\
0.199759566389542	1.15968314153541e-05\\
0.204300854381551	1.23970836615895e-05\\
0.208858690959197	1.32360277467236e-05\\
0.213416527536843	1.41115573110634e-05\\
0.217974364114488	1.50244156205646e-05\\
0.222532200692134	1.5975343586481e-05\\
0.227106724202768	1.69687755456906e-05\\
0.231681247713401	1.80020447155807e-05\\
0.236255771224035	1.9075893021052e-05\\
0.240830294734669	2.01910599991932e-05\\
0.245421645441331	2.13526181379454e-05\\
0.250012996147992	2.2557284965767e-05\\
0.254604346854654	2.38058009579289e-05\\
0.259195697561315	2.509890416967e-05\\
0.263804017673384	2.64423619676713e-05\\
0.268412337785452	2.78322196329488e-05\\
0.273020657897521	2.92692160824897e-05\\
0.27762897800959	3.07540877807822e-05\\
0.282254411746462	3.22933550358888e-05\\
0.286879845483334	3.3882332992138e-05\\
0.291505279220206	3.55217588982066e-05\\
0.296130712957079	3.72123675176513e-05\\
0.300773406613671	3.89614915351959e-05\\
0.305416100270264	4.07636567010808e-05\\
0.310058793926856	4.26195984847642e-05\\
0.314701487583449	4.453004983736e-05\\
0.319361589521735	4.65032166339294e-05\\
0.324021691460022	4.85327745977787e-05\\
0.328681793398309	5.06194573068192e-05\\
0.333341895336596	5.27639957878566e-05\\
0.338019556071109	5.49755313670871e-05\\
0.342697216805623	5.72468276654392e-05\\
0.347374877540136	5.95786162557385e-05\\
0.35205253827465	6.19716261270574e-05\\
0.356747910447269	6.44359977509893e-05\\
0.361443282619888	6.69635191399001e-05\\
0.366138654792507	6.95549197469352e-05\\
0.370834026965126	7.22109264088364e-05\\
0.375547265516765	7.49427440066269e-05\\
0.380260504068404	7.77411198751464e-05\\
0.384973742620042	8.06067812324099e-05\\
0.389686981171681	8.35404526473215e-05\\
0.394418243246227	8.6554470136794e-05\\
0.399149505320772	8.96384738276164e-05\\
0.403880767395317	9.27931885853417e-05\\
0.408612029469863	9.60193365939872e-05\\
0.41336147459102	9.93304533949682e-05\\
0.418110919712177	0.000102715003721942\\
0.422860364833334	0.000106173709970563\\
0.427609809954491	0.000109707291821652\\
0.432377599961091	0.000113330554275638\\
0.437145389967691	0.000117030716944083\\
0.44191317997429	0.000120808499632391\\
0.44668096998089	0.000124664619398951\\
0.451467269141767	0.000128615222347634\\
0.456253568302644	0.000132646211534851\\
0.461039867463521	0.000136758304055571\\
0.465826166624398	0.000140952214224751\\
0.47063114173014	0.000145245502641275\\
0.475436116835881	0.000149622682632739\\
0.480241091941623	0.000154084468461525\\
0.485046067047365	0.000158631571577061\\
0.489869887370685	0.000163283042140408\\
0.494693707694006	0.000168021928902846\\
0.499517528017326	0.000172848943170887\\
0.504341348340647	0.000177764793404953\\
0.509184185682064	0.000182790096178572\\
0.51402702302348	0.000187906359040702\\
0.518869860364897	0.000193114290216916\\
0.523712697706314	0.000198414595053558\\
0.528574726619363	0.000203829535416084\\
0.533436755532412	0.000209338999007842\\
0.53829878444546	0.000214943690846937\\
0.543160813358509	0.000220644313039683\\
0.548042211039569	0.000226464853348833\\
0.55292360872063	0.000232383499261329\\
0.55780500640169	0.000238400952460435\\
0.56268640408275	0.000244517911684295\\
0.56758735048985	0.000250760173216085\\
0.572488296896949	0.000257104141946433\\
0.577389243304049	0.000263550516093957\\
0.582290189711149	0.00027009999089962\\
0.587210867601438	0.000276780255767198\\
0.592131545491728	0.000283565848638045\\
0.597052223382018	0.000290457464135665\\
0.601972901272307	0.000297455793872596\\
0.606913496388328	0.000304590507212855\\
0.611854091504349	0.000311834188567691\\
0.61679468662037	0.000319187528832964\\
0.621735281736391	0.000326651215860788\\
0.626695982628195	0.000334256987589914\\
0.631656683519999	0.000341975386541179\\
0.636617384411802	0.000349807099748693\\
0.641578085303606	0.000357752811169749\\
0.646559083672327	0.000365846418531209\\
0.651540082041047	0.000374056331522935\\
0.656521080409768	0.00038238323318171\\
0.661502078778488	0.000390827803434392\\
0.666503569383853	0.000399426193013988\\
0.671505059989219	0.000408144585833606\\
0.676506550594584	0.000416983660795116\\
0.68150804119995	0.00042594409365693\\
0.686530221876942	0.000435064383582665\\
0.691552402553934	0.000444308393560324\\
0.696574583230927	0.000453676798217721\\
0.701596763907919	0.000463170269006059\\
0.706639835866396	0.000472829751734273\\
0.711682907824873	0.000482616690544425\\
0.71672597978335	0.000492531755648791\\
0.721769051741827	0.000502575614050333\\
0.72683321952555	0.000512791758676585\\
0.731897387309272	0.000523139114645837\\
0.736961555092995	0.000533618347612449\\
0.742025722876718	0.00054423011998796\\
0.747111194287107	0.0005550205744945\\
0.752196665697496	0.00056594601484564\\
0.757282137107885	0.000577007101992721\\
0.762367608518274	0.00058820449361113\\
0.767474594900923	0.000599587087914456\\
0.772581581283573	0.000611108461827721\\
0.777688567666222	0.000622769271452394\\
0.782795554048871	0.000634570169581048\\
0.78792427039055	0.000646562918351648\\
0.793052986732229	0.00065869825978947\\
0.798181703073909	0.000670976844997669\\
0.803310419415588	0.000683399321736815\\
0.808461084325784	0.000696020427037293\\
0.81361174923598	0.000708787957381274\\
0.818762414146177	0.000721702558722399\\
0.823913079056373	0.000734764873638716\\
0.82908591482452	0.000748032727781803\\
0.834258750592667	0.000761450858692864\\
0.839431586360813	0.000775019907022789\\
0.84460442212896	0.000788740510014038\\
0.849799655005291	0.000802673699098654\\
0.854994887881621	0.000816761036067998\\
0.860190120757951	0.000831003156115201\\
0.865385353634282	0.000845400690992024\\
0.870603213699137	0.000860017997514645\\
0.875821073763992	0.000874793342467206\\
0.881038933828848	0.000889727355427736\\
0.886256793893703	0.000904820662499372\\
0.891497515445601	0.000920141069434829\\
0.896738236997499	0.000935623424831927\\
0.901978958549397	0.000951268352493876\\
0.907219680101295	0.000967076472715583\\
0.912483501408722	0.000983119165970638\\
0.917747322716149	0.000999327737247601\\
0.923011144023576	0.00101570280441268\\
0.928274965331004	0.00103224498179024\\
0.933562128954513	0.00104902935432999\\
0.938849292578023	0.00106598355404102\\
0.944136456201532	0.0010831081926876\\
0.949423619825042	0.00110040387845893\\
0.954734372782572	0.00111794953419409\\
0.960045125740102	0.00113566898590811\\
0.965355878697632	0.00115356283909641\\
0.970666631655162	0.00117163169564588\\
0.976001225418392	0.00118995845290463\\
0.981335819181621	0.00120846299467622\\
0.98667041294485	0.00122714592001734\\
0.99200500670808	0.00124600782434303\\
0.997363697280812	0.00126513571967429\\
1.00272238785355	0.00128444540785023\\
1.00808107842628	0.00130393748131632\\
1.01343976899901	0.00132361252884274\\
1.0188228172289	0.00134356182178957\\
1.02420586545878	0.00136369693580067\\
1.02958891368867	0.00138401845653489\\
1.03497196191855	0.00140452696594257\\
1.04037963341625	0.00142531814259286\\
1.04578730491395	0.0014462991884976\\
1.05119497641164	0.00146747068235088\\
1.05660264790934	0.0014888331991051\\
1.06203521338424	0.00151048697728356\\
1.06746777885913	0.00153233469298081\\
1.07290034433402	0.0015543769177453\\
1.07833290980891	0.00157661421935028\\
1.08379064506731	0.0015991515527343\\
1.08924838032571	0.00162188691207705\\
1.09470611558411	0.00164482086159727\\
1.10016385084252	0.00166795396170511\\
1.1056470369148	0.00169139604453325\\
1.11113022298707	0.00171504026204362\\
1.11661340905935	0.00173888717093768\\
1.12209659513163	0.00176293732407503\\
1.12760551858231	0.00178730559681889\\
1.13311444203298	0.00181188013337916\\
1.13862336548365	0.00183666148274943\\
1.14413228893432	0.00186165019004878\\
1.14966724184116	0.00188696634388034\\
1.15520219474799	0.00191249291119588\\
1.16073714765482	0.0019382304330882\\
1.16627210056166	0.00196417944674103\\
1.17183338076584	0.00199046542930839\\
1.17739466097002	0.00201696599569975\\
1.18295594117419	0.00204368167890923\\
1.18851722137837	0.00207061300798872\\
1.19410513262804	0.00209789102893841\\
1.19969304387772	0.00212538782487044\\
1.20528095512739	0.00215310392047905\\
1.21086886637706	0.00218103983648321\\
1.21648371859449	0.00220933237378246\\
1.22209857081192	0.0022378478982028\\
1.22771342302936	0.00226658692593469\\
1.23332827524679	0.00229554996915945\\
1.23897038457538	0.00232487977456153\\
1.24461249390397	0.00235443680036725\\
1.25025460323256	0.00238422155405454\\
1.25589671256115	0.00241423453905872\\
1.26156640151281	0.00244462464419636\\
1.26723609046446	0.00247524622433333\\
1.27290577941612	0.00250609977802266\\
1.27857546836778	0.00253718579974153\\
1.28427306633634	0.00256865952443018\\
1.28997066430491	0.00260036900023369\\
1.29566826227347	0.00263231471656381\\
1.30136586024204	0.00266449715872336\\
1.30709170327692	0.0026970781155865\\
1.31281754631181	0.00272989912138352\\
1.31854338934669	0.0027629606561641\\
1.32426923238158	0.00279626319583603\\
1.33002366371043	0.00282997529915126\\
1.33577809503928	0.00286393177114344\\
1.34153252636813	0.00289813308227561\\
1.34728695769699	0.00293257969883527\\
1.35307032782915	0.00296744717140857\\
1.35885369796131	0.00300256335455072\\
1.36463706809348	0.00303792870890836\\
1.37042043822564	0.00307354369091922\\
1.37622881553645	0.00310956437630575\\
1.38203719284726	0.00314583776281463\\
1.38784557015807	0.00318236430004984\\
1.39365394746888	0.00321914443338532\\
1.39948194569202	0.00325630413778809\\
1.40530994391516	0.00329372004001648\\
1.41113794213829	0.00333139257717273\\
1.41696594036143	0.00336932218212518\\
1.4228137248248	0.00340763936902326\\
1.42866150928816	0.00344621623250614\\
1.43450929375153	0.00348505319705296\\
1.44035707821489	0.00352415068290528\\
1.44622478964653	0.0035636436708056\\
1.45209250107816	0.00360339979341288\\
1.45796021250979	0.00364341946245567\\
1.46382792394143	0.00368370308542181\\
1.46971570462452	0.00372439020535574\\
1.4756034853076	0.00376534389705535\\
1.48149126599069	0.00380656455937158\\
1.48737904667378	0.00384805258691236\\
1.49328704060906	0.00388995218285096\\
1.49919503454433	0.003932121766137\\
1.50510302847961	0.00397456172261676\\
1.51101102241489	0.00401727243389234\\
1.51693937480906	0.00406040285840711\\
1.52286772720322	0.00410380666391001\\
1.52879607959739	0.00414748422311705\\
1.53472443199155	0.00419143590449816\\
1.540673290528	0.00423581552717689\\
1.54662214906446	0.00428047190219487\\
1.55257100760091	0.00432540538901099\\
1.55851986613736	0.00437061634283833\\
1.56448937947106	0.00441626353880925\\
1.57045889280476	0.00446219083568948\\
1.57642840613847	0.00450839857955523\\
1.58239791947216	0.00455488711223737\\
1.58838823791446	0.00460182026620284\\
1.59437855635675	0.00464903684643492\\
1.60036887479904	0.0046965371855022\\
1.60635919324133	0.0047443216117283\\
1.61237046959671	0.00479255912396716\\
1.61838174595208	0.00484108336426296\\
1.62439302230746	0.00488989465155103\\
1.63040429866284	0.00493899330052336\\
1.63643668672632	0.00498855357456031\\
1.6424690747898	0.00503840385437671\\
1.64850146285328	0.00508854444514967\\
1.65453385091676	0.00513897564781561\\
1.66058750665404	0.00518987709897024\\
1.66664116239132	0.00524107180916707\\
1.67269481812859	0.00529256006970089\\
1.67874847386587	0.00534434216762808\\
1.68482355500326	0.00539660321918473\\
1.69089863614064	0.00544916075815231\\
1.69697371727803	0.00550201506181894\\
1.70304879841541	0.00555516640323793\\
1.70914546464501	0.00560880548754901\\
1.71524213087462	0.005662744262328\\
1.72133879710422	0.00571698299073248\\
1.72743546333383	0.00577152193168909\\
1.73355387572031	0.00582655748429639\\
1.73967228810678	0.00588189590463653\\
1.74579070049326	0.00593753744161279\\
1.75190911287974	0.0059934823399021\\
1.75804943463327	0.00604993280552477\\
1.76418975638681	0.00610668928994487\\
1.77033007814035	0.00616375202768757\\
1.77647039989388	0.00622112124905751\\
1.78263279664118	0.00627900508374608\\
1.78879519338848	0.00633719806171107\\
1.79495759013578	0.00639570040297693\\
1.80111998688309	0.00645451232335341\\
1.80730462566597	0.00651384798471046\\
1.81348926444885	0.00657349588674891\\
1.81967390323173	0.00663345623486944\\
1.82585854201462	0.00669372923026477\\
1.83206559217232	0.00675453518493464\\
1.83827264233002	0.00681565645021182\\
1.84447969248772	0.00687709321675042\\
1.85068674264543	0.00693884567100446\\
1.85691637550678	0.00700114039127206\\
1.86314600836813	0.00706375346415329\\
1.86937564122948	0.00712668506543371\\
1.87560527409083	0.00718993536670652\\
1.88185766319929	0.00725373733214671\\
1.88811005230775	0.00731786066386579\\
1.89436244141622	0.00738230552265794\\
1.90061483052468	0.00744707206513461\\
1.90689015093446	0.00751239975481639\\
1.91316547134424	0.0075780517956011\\
1.91944079175402	0.00764402833317059\\
1.92571611216381	0.00771032950903269\\
1.93201454150218	0.00777720141161729\\
1.93831297084055	0.00784440062089338\\
1.94461140017892	0.00791192726730794\\
1.9509098295173	0.00797978147714477\\
1.95723154808653	0.00804821609185108\\
1.96355326665576	0.0081169809392124\\
1.96987498522499	0.00818607613431983\\
1.97619670379422	0.00825550178811227\\
1.98254189319299	0.00832551760940839\\
1.98888708259176	0.00839586655916016\\
1.99523227199053	0.00846654873698177\\
2.0015774613893	0.00853756423834692\\
2.00794630624256	0.00860917977391072\\
2.01431515109582	0.00868113130320558\\
2.02068399594908	0.00875341891024753\\
2.02705284080234	0.00882604267492471\\
2.03344552746868	0.00889927643104625\\
2.03983821413503	0.00897284901514485\\
2.04623090080137	0.00904676049551728\\
2.05262358746771	0.00912101093634619\\
2.05904030495676	0.00919588142750283\\
2.06545702244581	0.00927109354943932\\
2.07187373993486	0.00934664735461334\\
2.07829045742391	0.00942254289138139\\
2.084731396657	0.0094990686312147\\
2.0911723358901	0.00957593877268979\\
2.0976132751232	0.00965315335230445\\
2.1040542143563	0.0097307124024703\\
2.11051956957009	0.00980891191975411\\
2.11698492478388	0.00988745857724684\\
2.12345027999767	0.00996635239536596\\
2.12991563521146	0.0100455933904583\\
2.13640560238783	0.0101254852101484\\
2.1428955695642	0.0102057268757906\\
2.14938553674057	0.0102863183916024\\
2.15587550391694	0.0103672597577463\\
2.16239028199053	0.010448862414831\\
2.16890506006412	0.0105308175903815\\
2.17541983813771	0.0106131252722951\\
2.1819346162113	0.0106957854444316\\
2.18847440630977	0.0107791174741812\\
2.19501419640825	0.0108628046611866\\
2.20155398650673	0.0109468469769048\\
2.20809377660521	0.0110312443887741\\
2.21465878275268	0.0111163243349622\\
2.22122378890015	0.0112017620430464\\
2.22778879504762	0.0112875574679243\\
2.23435380119508	0.0113737105604932\\
2.24094422983074	0.0114605569688445\\
2.24753465846639	0.0115477637090974\\
2.25412508710205	0.0116353307194707\\
2.2607155157377	0.0117232579342023\\
2.2673315759421	0.0118118893546629\\
2.27394763614649	0.0119008836419279\\
2.28056369635089	0.011990240717418\\
2.28717975655528	0.0120799604985929\\
2.29382166035897	0.0121703954889726\\
2.30046356416266	0.0122611958455144\\
2.30710546796636	0.0123523614727216\\
2.31374737177005	0.0124438922711574\\
2.32041533418655	0.0125361493973062\\
2.32708329660306	0.0126287743529833\\
2.33375125901957	0.0127217670256557\\
2.34041922143607	0.0128151272988719\\
2.34711345989709	0.0129092251264324\\
2.3538076983581	0.0130036932103791\\
2.36050193681912	0.0130985314210234\\
2.36719617528013	0.0131937396247812\\
2.37391691043736	0.0132896967296879\\
2.3806376455946	0.0133860264808932\\
2.38735838075183	0.0134827287314343\\
2.39407911590907	0.0135798033304762\\
2.40082657131283	0.013677638294147\\
2.40757402671659	0.0137758482566021\\
2.41432148212035	0.0138744330534852\\
2.42106893752412	0.0139733925165926\\
2.42784333963791	0.0140731239255915\\
2.4346177417517	0.0141732326479442\\
2.4413921438655	0.014273718501783\\
2.44816654597929	0.0143745813014169\\
2.4549681243632	0.0144762277496043\\
2.4617697027471	0.0145782537873214\\
2.46857128113101	0.0146806592150698\\
2.47537285951492	0.0147834438295543\\
2.48220184698832	0.014887023920107\\
2.48903083446173	0.0149909858375022\\
2.49585982193513	0.0150953293644918\\
2.50268880940853	0.0152000542800586\\
2.50954544177924	0.0153055866209608\\
2.51640207414995	0.0154115029866447\\
2.52325870652066	0.0155178031419949\\
2.53011533889137	0.0156244868481535\\
2.53699985562036	0.0157319900618928\\
2.54388437234935	0.0158398794585346\\
2.55076888907833	0.0159481547849769\\
2.55765340580732	0.0160568157844043\\
2.5645660494942	0.016166308499785\\
2.57147869318108	0.0162761895158565\\
2.57839133686796	0.0163864585614113\\
2.58530398055484	0.016497115361559\\
2.59224499685821	0.0166086162117505\\
2.59918601316159	0.0167205074395545\\
2.60612702946496	0.0168327887555398\\
2.61306804576833	0.0169454598666227\\
2.62003768421203	0.0170589875014834\\
2.62700732265572	0.0171729075495572\\
2.63397696109942	0.0172872197030706\\
2.64094659954311	0.0174019236506292\\
2.64794511298781	0.0175174967279672\\
2.65494362643251	0.0176334642122722\\
2.6619421398772	0.0177498257773098\\
2.6689406533219	0.017866581093257\\
2.67596829852112	0.017984218287483\\
2.68299594372033	0.0181022518401082\\
2.69002358891955	0.0182206814063185\\
2.69705123411877	0.0183395066377449\\
2.70410827073433	0.0184592266230197\\
2.71116530734989	0.0185793448752061\\
2.71822234396545	0.0186998610307915\\
2.72527938058101	0.0188207747227429\\
2.73236607296869	0.0189425962024104\\
2.73945276535637	0.0190648178141721\\
2.74653945774405	0.0191874391756983\\
2.75362615013173	0.0193104599011737\\
2.76074276611797	0.019434401587038\\
2.76785938210422	0.0195587452262883\\
2.77497599809046	0.019683490417659\\
2.78209261407671	0.019808636756435\\
2.78923942490227	0.0199347173669136\\
2.79638623572783	0.0200612017075988\\
2.80353304655339	0.0201880893581698\\
2.81067985737896	0.0203153798948937\\
2.81785713902212	0.0204436181776936\\
2.82503442066529	0.0205722619226042\\
2.83221170230846	0.0207013106901301\\
2.83938898395163	0.0208307640374021\\
2.84659701614953	0.0209611787521823\\
2.85380504834743	0.0210920006154897\\
2.86101308054533	0.0212232291685357\\
2.86822111274323	0.0213548639491965\\
2.87546017926332	0.0214874738717447\\
2.88269924578341	0.0216204925831955\\
2.88993831230351	0.0217539196053468\\
2.8971773788236	0.0218877544567025\\
2.90444776780364	0.0220225783846545\\
2.91171815678368	0.0221578126953115\\
2.91898854576373	0.0222934568909384\\
2.92625893474377	0.0224295104705477\\
2.93356093845992	0.0225665672187143\\
2.94086294217606	0.0227040358962417\\
2.94816494589221	0.0228419159857425\\
2.95546694960836	0.0229802069666187\\
2.96280086487659	0.0231195153739414\\
2.97013478014483	0.0232592372095901\\
2.97746869541306	0.0233993719364049\\
2.9848026106813	0.0235399190140585\\
2.99216873909201	0.0236814979477885\\
2.99953486750271	0.0238234917605843\\
3.00690099591342	0.0239658998953931\\
3.01426712432413	0.0241087217920392\\
3.02166577186631	0.0242525901402344\\
3.02906441940849	0.0243968747694157\\
3.03646306695067	0.0245415751025162\\
3.04386171449286	0.0246866905593927\\
3.05129319192691	0.0248328672375373\\
3.05872466936097	0.024979461549198\\
3.06615614679503	0.0251264728971735\\
3.07358762422909	0.0252739006812326\\
3.08105224696524	0.0254224046293906\\
3.0885168697014	0.0255713275136021\\
3.09598149243755	0.0257206687164101\\
3.1034461151737	0.0258704276173762\\
3.11094420406679	0.0260212778155197\\
3.11844229295988	0.0261725482017295\\
3.12594038185297	0.0263242381381718\\
3.13343847074605	0.0264763469840795\\
3.14097035080817	0.0266295624259439\\
3.14850223087028	0.0267831992566669\\
3.1560341109324	0.0269372568179155\\
3.16356599099451	0.0270917344484739\\
3.17113199342146	0.027247334181797\\
3.17869799584842	0.0274033564530605\\
3.18626399827537	0.0275598005833107\\
3.19383000070233	0.0277166658907617\\
3.20143046110215	0.0278746689810415\\
3.20903092150198	0.0280330957059256\\
3.21663138190181	0.0281919453657163\\
3.22423184230164	0.0283512172579366\\
3.2318671021242	0.0285116428170273\\
3.23950236194676	0.0286724930543979\\
3.24713762176933	0.0288337672494835\\
3.25477288159189	0.0289954646789939\\
3.26244328804483	0.0291583318632845\\
3.27011369449776	0.0293216247159398\\
3.2777841009507	0.0294853424954045\\
3.28545450740363	0.0296494844574528\\
3.29316041266953	0.0298148124507696\\
3.30086631793543	0.0299805670482339\\
3.30857222320133	0.0301467474871756\\
3.31627812846722	0.030313353002311\\
3.32401989110489	0.0304811610450283\\
3.33176165374256	0.0306493965727938\\
3.33950341638023	0.0308180588016982\\
3.3472451790179	0.0309871469452762\\
3.35502316313618	0.0311574543162921\\
3.36280114725446	0.0313281899976789\\
3.37057913137275	0.0314993531841631\\
3.37835711549103	0.0316709430679737\\
3.38617169154667	0.0318437691013879\\
3.3939862676023	0.0320170242142846\\
3.40180084365794	0.0321907075799006\\
3.40961541971357	0.032364818369035\\
3.41746696420648	0.032540182447052\\
3.42531850869938	0.03271597631676\\
3.43317005319229	0.032892199129779\\
3.4410215976852	0.0330688500353536\\
3.44891049346576	0.0332467715944282\\
3.45679938924632	0.0334251235998129\\
3.46468828502688	0.0336039051813845\\
3.47257718080744	0.0337831154667069\\
3.48050381756295	0.0339636140079943\\
3.48843045431847	0.0341445435919614\\
3.49635709107398	0.034325903326614\\
3.50428372782949	0.0345076923177091\\
3.51224850159594	0.0346907873955279\\
3.52021327536239	0.0348743140534043\\
3.52817804912884	0.0350582713773446\\
3.53614282289529	0.0352426584511719\\
3.54414613607264	0.0354283696723346\\
3.55214944924998	0.0356145129511712\\
3.56015276242733	0.0358010873515599\\
3.56815607560468	0.035988091935262\\
3.57619833889781	0.0361764390033761\\
3.58424060219095	0.0363652185464292\\
3.59228286548408	0.03655442960604\\
3.60032512877721	0.0367440712217798\\
3.60840675922679	0.0369350738913304\\
3.61648838967638	0.0371265093918973\\
3.62457002012596	0.0373183767427101\\
3.63265165057554	0.0375106749610201\\
3.64077307248204	0.0377043530577683\\
3.64889449438855	0.0378984642796558\\
3.65701591629505	0.0380930076233908\\
3.66513733820155	0.0382879820837747\\
3.67329898353354	0.0384843555137039\\
3.68146062886554	0.0386811623001833\\
3.68962227419754	0.038878401417267\\
3.69778391952953	0.0390760718371755\\
3.70598622860803	0.0392751606024341\\
3.71418853768653	0.0394746828922217\\
3.72239084676502	0.0396746376578042\\
3.73059315584352	0.0398750238486889\\
3.73883657626523	0.0400768480209967\\
3.74707999668693	0.0402791058215093\\
3.75532341710863	0.0404817961785703\\
3.76356683753034	0.0406849180188401\\
3.77185182516863	0.0408894977626448\\
3.78013681280692	0.0410945111732262\\
3.78842180044522	0.0412999571558691\\
3.79670678808351	0.0415058346142534\\
3.80503380745825	0.041713190195162\\
3.81336082683299	0.0419209794155133\\
3.82168784620774	0.0421292011573969\\
3.83001486558248	0.0423378543013762\\
3.83838438958511	0.0425480060785771\\
3.84675391358775	0.0427585914011111\\
3.85512343759039	0.042969609127734\\
3.86349296159302	0.0431810581157566\\
3.8719054722877	0.0433940265610931\\
3.88031798298237	0.0436074283900492\\
3.88873049367705	0.0438212624379073\\
3.89714300437172	0.0440355275385876\\
3.90559899284027	0.0442513332319133\\
3.91405498130882	0.0444675720786539\\
3.92251096977737	0.0446842428904775\\
3.93096695824593	0.0449013444777743\\
3.93946692479213	0.0451200081110725\\
3.94796689133833	0.0453391045981791\\
3.95646685788454	0.0455586327270057\\
3.96496682443074	0.0457785912842723\\
3.97351127881372	0.0460001336664787\\
3.98205573319669	0.0462221085325537\\
3.99060018757967	0.0464445146465084\\
3.99914464196265	0.0466673507712509\\
4.00773410425464	0.0468917928493782\\
4.01632356654662	0.0471166669702086\\
4.02491302883861	0.0473419718737073\\
4.0335024911306	0.0475677062988265\\
4.04213749081375	0.0477950691333865\\
4.05077249049689	0.0480228634972252\\
4.05940749018004	0.0482510881061159\\
4.06804248986318	0.0484797416749101\\
4.07672356764034	0.0487100464859807\\
4.08540464541749	0.0489407822397461\\
4.09408572319465	0.0491719476276387\\
4.10276680097181	0.0494035413402623\\
4.11149450824228	0.0496368094929005\\
4.12022221551276	0.0498705079274823\\
4.12894992278323	0.0501046353109482\\
4.1376776300537	0.0503391903095065\\
4.14645252883691	0.0505754433102368\\
4.15522742762012	0.0508121258569227\\
4.16400232640333	0.0510492365918612\\
4.17277722518654	0.0512867741567141\\
4.18159988926615	0.0515260336826812\\
4.19042255334575	0.0517657219423662\\
4.19924521742536	0.0520058375532679\\
4.20806788150496	0.0522463791323504\\
4.21693889629988	0.0524886670266166\\
4.22580991109479	0.0527313827650283\\
4.23468092588971	0.05297452494013\\
4.24355194068463	0.0532180921440331\\
4.2524719033702	0.0534634304172024\\
4.26139186605577	0.0537091955664761\\
4.27031182874135	0.0539553861592865\\
4.27923179142692	0.0542020007627358\\
4.28820131218307	0.0544504116255367\\
4.29717083293922	0.0546992483168555\\
4.30614035369538	0.0549485093788513\\
4.31510987445153	0.0551981933534592\\
4.32412957614907	0.0554496992067278\\
4.33314927784661	0.0557016297602089\\
4.34216897954416	0.0559539835306251\\
4.3511886812417	0.0562067590345836\\
4.36025919971018	0.0564613824747678\\
4.36932971817866	0.0567164294049241\\
4.37840023664715	0.0569718983161735\\
4.38747075511563	0.0572277876996319\\
4.39659274002011	0.0574855515413475\\
4.40571472492458	0.0577437375796564\\
4.41483670982906	0.0580023442799095\\
4.42395869473353	0.0582613701075658\\
4.43313280988389	0.0585222973904897\\
4.44230692503424	0.0587836454922397\\
4.45148104018459	0.0590454128522264\\
4.46065515533495	0.0593075979100832\\
4.469882079136	0.0595717119097048\\
4.47910900293706	0.059836245264713\\
4.48833592673811	0.0601011963884038\\
4.49756285053916	0.0603665636944141\\
4.50684327607396	0.0606338879221725\\
4.51612370160877	0.0609016299548492\\
4.52540412714357	0.0611697881794493\\
4.53468455267837	0.0614383609834389\\
4.54401918924822	0.0617089192286553\\
4.55335382581807	0.0619798936399932\\
4.56268846238792	0.0622512825779868\\
4.57202309895777	0.0625230844037543\\
4.58141267108196	0.0627969007025038\\
4.59080224320616	0.0630711314387738\\
4.60019181533036	0.0633457749464444\\
4.60958138745456	0.0636208295601056\\
4.61902663678989	0.0638979282481469\\
4.62847188612523	0.0641754395539089\\
4.63791713546056	0.0644533617844316\\
4.6473623847959	0.0647316932475922\\
4.65686406984621	0.0650120989494338\\
4.66636575489653	0.065292915356484\\
4.67586743994685	0.0655741407487518\\
4.68536912499717	0.0658557734072149\\
4.69492802222378	0.0661395110658345\\
4.7044869194504	0.0664236574229234\\
4.71404581667701	0.0667082107312661\\
4.72360471390363	0.0669931692447494\\
4.73320764525141	0.0672798467008731\\
4.74281057659919	0.0675669295748214\\
4.75241350794698	0.0678544160996777\\
4.76201643929476	0.0681423045097567\\
4.77165794452804	0.0684317518674175\\
4.78129944976131	0.068721600782313\\
4.79094095499459	0.0690118494712198\\
4.80058246022786	0.0693024961522729\\
4.81026314694842	0.069594722601228\\
4.81994383366897	0.0698873466878031\\
4.82962452038952	0.0701803666125327\\
4.83930520711008	0.0704737805774382\\
4.84902565638322	0.0707687943750962\\
4.85874610565637	0.0710642018281359\\
4.86846655492951	0.0713600011209321\\
4.87818700420266	0.0716561904394751\\
4.88794780523678	0.0719539999377261\\
4.89770860627089	0.0722521990457987\\
4.90746940730501	0.0725507859319888\\
4.91723020833913	0.072849758766336\\
4.92703195872998	0.0731503724153926\\
4.93683370912084	0.0734513715646641\\
4.9466354595117	0.0737527543664452\\
4.95643720990256	0.0740545189749044\\
4.96628051524823	0.0743579453076268\\
4.97612382059389	0.0746617529661076\\
4.98596712593956	0.0749659400867174\\
4.99581043128522	0.075270504807832\\
5.00569590732134	0.0755767525014873\\
5.01558138335745	0.0758833772809104\\
5.02546685939356	0.0761903772666225\\
5.03535233542967	0.0764977505812813\\
5.04528060635073	0.0768068284021117\\
5.05520887727179	0.077116279002318\\
5.06513714819285	0.0774261004866458\\
5.07506541911391	0.077736290962109\\
5.08503711889176	0.0780482078018079\\
5.0950088186696	0.0783604930472742\\
5.10498051844744	0.0786731447875494\\
5.11495221822528	0.0789861611140762\\
5.12496799094167	0.0793009259960847\\
5.13498376365806	0.0796160548421909\\
5.14499953637444	0.0799315457258014\\
5.15501530909083	0.0802473967228588\\
5.16507580953502	0.0805650188169324\\
5.17513630997921	0.0808830003645153\\
5.18519681042339	0.0812013394234477\\
5.19525731086758	0.0815200340542404\\
5.2053632039849	0.0818405226546683\\
5.21546909710223	0.0821613661281456\\
5.22557499021955	0.0824825625170125\\
5.23568088333688	0.0828041098664157\\
5.24583284576096	0.0831274744357652\\
5.25598480818505	0.083451189226931\\
5.26613677060914	0.0837752522668179\\
5.27628873303322	0.0840996615852745\\
5.28648745229033	0.0844259117244864\\
5.29668617154743	0.0847525073624905\\
5.30688489080453	0.0850794465108198\\
5.31708361006163	0.0854067271840888\\
5.32732978639303	0.0857358726856234\\
5.33757596272443	0.0860653588901811\\
5.34782213905582	0.0863951837939845\\
5.35806831538722	0.0867253453964764\\
5.36836266133408	0.0870573962262633\\
5.37865700728095	0.0873897828895112\\
5.38895135322782	0.0877225033671915\\
5.39924569917468	0.0880555556436364\\
5.40958894010065	0.0883905219525926\\
5.41993218102661	0.0887258191505748\\
5.43027542195257	0.0890614452033616\\
5.44061866287853	0.0893973980802336\\
5.45101153706255	0.0897352902017248\\
5.46140441124658	0.0900735081917938\\
5.4717972854306	0.0904120500010829\\
5.48219015961463	0.0907509135838788\\
5.49263341945276	0.0910917420669582\\
5.50307667929089	0.0914328913210201\\
5.51351993912902	0.0917743592816245\\
5.52396319896715	0.0921161438881208\\
5.53445761187213	0.0924599195212433\\
5.54495202477712	0.0928040107494511\\
5.5554464376821	0.0931484154932757\\
5.56594085058709	0.0934931316771823\\
5.57648719831941	0.0938398654604165\\
5.58703354605173	0.0941869095832666\\
5.59757989378405	0.0945342619512858\\
5.60812624151637	0.094881920474108\\
5.61872532103302	0.0952316236404023\\
5.62932440054967	0.0955816318099729\\
5.63992348006633	0.0959319428734449\\
5.65052255958298	0.096282554725672\\
5.66117518401188	0.0966352387663021\\
5.67182780844078	0.0969882223916788\\
5.68248043286967	0.0973415034775461\\
5.69313305729857	0.0976950799040265\\
5.70384005671797	0.0980507565876255\\
5.71454705613736	0.0984067273538884\\
5.72525405555675	0.098762990063724\\
5.73596105497615	0.0991195425825711\\
5.74672327665992	0.0994782239562296\\
5.75748549834369	0.0998371938254795\\
5.76824772002746	0.100196450036438\\
5.77900994171123	0.100555990439906\\
5.78982825023888	0.100917688825246\\
5.80064655876654	0.101279670032594\\
5.81146486729419	0.101641931893319\\
5.82228317582184	0.102004472243626\\
5.83315845473553	0.102369200284176\\
5.84403373364921	0.102734205385116\\
5.8549090125629	0.103099485363105\\
5.86578429147658	0.103465038039796\\
5.87671744355088	0.103832808702015\\
5.88765059562518	0.104200850573357\\
5.89858374769948	0.104569161455813\\
5.90951689977378	0.104937739156522\\
5.92050884813521	0.105308565758911\\
5.93150079649664	0.105679657627856\\
5.94249274485807	0.10605101255071\\
5.9534846932195	0.106422628320138\\
5.96453638108293	0.106796524515443\\
5.97558806894636	0.107170679941656\\
5.98663975680979	0.10754509237153\\
5.99769144467322	0.107919759583289\\
6.00880383710542	0.108296739408542\\
6.01991622953761	0.108673972334177\\
6.0310286219698	0.109051456118377\\
6.042141014402	0.109429188524964\\
6.05331509935885	0.109809266426979\\
6.0644891843157	0.110189591202104\\
6.07566326927256	0.110570160593983\\
6.08683735422941	0.110950972352064\\
6.09807414230094	0.111334163168535\\
6.10931093037248	0.111717594532099\\
6.12054771844401	0.112101264171889\\
6.13178450651555	0.11248516982301\\
6.14308503319571	0.112871488848912\\
6.15438555987588	0.113258041995124\\
6.16568608655605	0.11364482697629\\
6.17698661323621	0.1140318415132\\
6.18835193888863	0.114421304488436\\
6.19971726454106	0.114810995054273\\
6.21108259019348	0.115200910910894\\
6.2224479158459	0.115591049764797\\
6.23387912710053	0.115983672909992\\
6.24531033835517	0.116376517010927\\
6.2567415496098	0.116769579753338\\
6.26817276086443	0.117162858829456\\
6.2796709718506	0.117558658875777\\
6.29116918283678	0.117954673135495\\
6.30266739382295	0.118350899279919\\
6.31416560480912	0.118747334987031\\
6.32573195844141	0.119146329207318\\
6.33729831207371	0.119545530788751\\
6.34886466570601	0.119944937388229\\
6.36043101933831	0.120344546669502\\
6.37206668825559	0.120746752896846\\
6.38370235717287	0.121149159520633\\
6.39533802609016	0.121551764183362\\
6.40697369500744	0.121954564534568\\
6.41867988207414	0.122360001164954\\
6.43038606914084	0.122765631111941\\
6.44209225620754	0.123171452003635\\
6.45379844327423	0.123577461475367\\
6.46557638449348	0.123986147553024\\
6.47735432571273	0.124395019749538\\
6.48913226693197	0.124804075678634\\
6.50091020815122	0.125213312961445\\
6.51276117329228	0.125625268184577\\
6.52461213843333	0.126037402208007\\
6.53646310357439	0.126449712631077\\
6.54831406871544	0.126862197060732\\
6.56023936237899	0.127277441801379\\
6.57216465604253	0.127692857899879\\
6.58408994970607	0.12810844294119\\
6.59601524336962	0.128524194518071\\
6.608016207688	0.128942749897984\\
6.62001717200638	0.129361469066234\\
6.63201813632476	0.129780349593389\\
6.64401910064313	0.130199389058023\\
6.65609711612489	0.130621276959143\\
6.66817513160664	0.131043320948642\\
6.68025314708839	0.131465518582695\\
6.69233116257014	0.131887867425681\\
6.70448764976455	0.132313110528014\\
6.71664413695897	0.132738501884791\\
6.72880062415338	0.133164039037781\\
6.7409571113478	0.133589719537161\\
6.75319353351478	0.134018341390768\\
6.76542995568177	0.13444710352721\\
6.77766637784875	0.134876003473831\\
6.78990280001574	0.135305038766598\\
6.80222066442117	0.135737063814412\\
6.8145385288266	0.136169221031872\\
6.82685639323203	0.136601507931877\\
6.83917425763746	0.137033922036164\\
6.85157511856051	0.137469375695479\\
6.86397597948357	0.137904953265579\\
6.87637684040662	0.138340652244892\\
6.88877770132968	0.138776470140903\\
6.90126316148383	0.139215378827933\\
6.91374862163799	0.139654403016884\\
6.92623408179214	0.140093540191686\\
6.93871954194629	0.140532787845548\\
6.95129125486549	0.140975179032275\\
6.96386296778468	0.141417677157501\\
6.97643468070388	0.141860279690619\\
6.98900639362307	0.142302984110531\\
7.00166606764739	0.142748886436316\\
7.0143257416717	0.14319488697783\\
7.02698541569601	0.143640983189889\\
7.03964508972033	0.144087172537051\\
7.05239448925109	0.144536615822515\\
7.06514388878186	0.144986148436508\\
7.07789328831262	0.145435767819221\\
7.09064268784339	0.145885471420827\\
7.10348363709426	0.146338486767642\\
7.11632458634514	0.146791582385998\\
7.12916553559601	0.14724475570141\\
7.14200648484689	0.147698004149616\\
7.15494087138423	0.148154624030967\\
7.16787525792157	0.148611314951422\\
7.18080964445891	0.149068074321756\\
7.19374403099625	0.149524899563217\\
7.20677380774895	0.149985157858041\\
7.21980358450165	0.150445477778064\\
7.23283336125435	0.150905856719256\\
7.24586313800706	0.151366292088312\\
7.25899032889946	0.151830224241021\\
7.27211751979186	0.152294208417204\\
7.28524471068426	0.152758241997949\\
7.29837190157667	0.153222322375332\\
7.31159860422458	0.153689965449306\\
7.32482530687249	0.154157650750455\\
7.3380520095204	0.154625375644897\\
7.35127871216831	0.155093137510013\\
7.36460710304072	0.155564530325343\\
7.37793549391314	0.156035955369798\\
7.39126388478556	0.15650740999444\\
7.40459227565797	0.156978891561861\\
7.41802461355885	0.157454074770613\\
7.43145695145972	0.157929280001061\\
7.4448892893606	0.158404504589102\\
7.45832162726148	0.15887974588245\\
7.47186025959003	0.15935876213881\\
7.48539889191859	0.159837789991931\\
7.49893752424715	0.160316826762435\\
7.51247615657571	0.160795869783048\\
7.52612352432791	0.161278763865252\\
7.53977089208011	0.161761658893115\\
7.55341825983232	0.162244552171856\\
7.56706562758452	0.162727441019099\\
7.58082427117841	0.163214259976437\\
7.5945829147723	0.163701068992912\\
7.60834155836619	0.164187865358206\\
7.62210020196009	0.164674646374708\\
7.63597276754062	0.165165439687016\\
7.64984533312116	0.165656211926632\\
7.66371789870169	0.166146960367547\\
7.67759046428223	0.166637682296775\\
7.69157971106386	0.16713250206573\\
7.70556895784549	0.167627289374265\\
7.71955820462712	0.168122041480515\\
7.73354745140875	0.168616755655966\\
7.7476562591015	0.169115656793889\\
7.76176506679424	0.169614513816412\\
7.77587387448699	0.170113323965633\\
7.78998268217974	0.170612084497334\\
7.80421406003167	0.171115124967153\\
7.81844543788359	0.171618109387137\\
7.83267681573552	0.172121034983148\\
7.84690819358745	0.172623898995077\\
7.86126528820976	0.173131140000576\\
7.87562238283208	0.173638312729235\\
7.8899794774544	0.174145414390457\\
7.90433657207672	0.17465244220804\\
7.91882267896367	0.175163948507343\\
7.93330878585062	0.175675373996078\\
7.94779489273757	0.176186715866954\\
7.96228099962452	0.176697971327441\\
7.97689957365268	0.17721381150213\\
7.99151814768085	0.177729558010524\\
8.00613672170902	0.178245208028377\\
8.02075529573718	0.178760758746586\\
8.03550996276884	0.179281005500679\\
8.0502646298005	0.179801145394238\\
8.06501929683216	0.180321175585771\\
8.07977396386382	0.180841093249329\\
8.09466853533245	0.181365823803231\\
8.10956310680107	0.181890433945926\\
8.12445767826969	0.182414920818362\\
8.13935224973832	0.182939281577441\\
8.15439073744049	0.183468578057552\\
8.16942922514267	0.183997740199882\\
8.18446771284484	0.18452676512747\\
8.19950620054702	0.185055649979742\\
8.2146928332504	0.185589599859102\\
8.22987946595378	0.186123401077007\\
8.24506609865717	0.186657050738209\\
8.26025273136055	0.187190545964289\\
8.27559197362809	0.187729242566608\\
8.29093121589562	0.188267775763556\\
8.30627045816316	0.188806142641178\\
8.32160970043069	0.189344340302809\\
8.33710627308224	0.189887883347531\\
8.35260284573378	0.190431247797406\\
8.36809941838533	0.19097443071931\\
8.38359599103687	0.191517429197891\\
8.39925489573158	0.192065925471158\\
8.41491380042629	0.192614227486738\\
8.430572705121	0.193162332291822\\
8.44623160981571	0.193710236951884\\
8.46205815432056	0.194263800980095\\
8.4778846988254	0.194817154583834\\
8.49371124333024	0.195370294790047\\
8.50953778783509	0.195923218644493\\
8.52553761739547	0.196481973557185\\
8.54153744695585	0.197040501340893\\
8.55753727651623	0.19759879900169\\
8.57353710607661	0.198156863565017\\
8.58971623619016	0.198720941978597\\
8.60589536630372	0.199284775983563\\
8.62207449641727	0.199848362564423\\
8.63825362653082	0.200411698725638\\
8.65461848330403	0.200981243859593\\
8.67098334007723	0.201550526688597\\
8.68734819685043	0.202119544174818\\
8.70371305362364	0.202688293300992\\
8.72018188980809	0.203260382203816\\
8.73665072599255	0.203832193257252\\
8.75311956217701	0.204403723448993\\
8.76958839836146	0.204974969787496\\
8.78614348346302	0.205548918712919\\
8.80269856856458	0.206122574805422\\
8.81925365366614	0.206695935089408\\
8.83580873876771	0.207268996610144\\
8.85245111606455	0.207844775695686\\
8.86909349336139	0.208420246934614\\
8.88573587065824	0.208995407388708\\
8.90237824795508	0.209570254140713\\
8.91910883699713	0.21014782870211\\
8.93583942603918	0.210725080378524\\
8.95257001508123	0.211302006269857\\
8.96930060412328	0.211878603497069\\
8.98612034063892	0.212457938890219\\
9.00294007715455	0.21303693633615\\
9.01975981367019	0.213615592973629\\
9.03657955018582	0.214193905962565\\
9.05348938408815	0.214774967508431\\
9.07039921799048	0.215355676021839\\
9.08730905189281	0.215936028681156\\
9.10421888579513	0.216516022685976\\
9.12121978296026	0.217098775725482\\
9.13822068012539	0.217681160625055\\
9.15522157729052	0.218263174603399\\
9.17222247445565	0.218844814900522\\
9.18931541587169	0.219429224758466\\
9.20640835728774	0.220013251347549\\
9.22350129870378	0.220596891927543\\
9.24059424011982	0.221180143779599\\
9.25778022278737	0.221766175788497\\
9.27496620545493	0.222351809378872\\
9.29215218812249	0.2229370418523\\
9.30933817079004	0.223521870531802\\
9.32661820779546	0.224109490026911\\
9.34389824480088	0.224696695933806\\
9.3611782818063	0.225283485596594\\
9.37845831881172	0.225869856380891\\
9.39583344002363	0.226459028716115\\
9.41320856123555	0.227047772274051\\
9.43058368244747	0.227636084442067\\
9.44795880365938	0.228223962629097\\
9.46543005570071	0.228814653167785\\
9.48290130774205	0.229404899721363\\
9.50037255978338	0.229994699721182\\
9.51784381182471	0.230584050620219\\
9.53541225878365	0.23117622475068\\
9.55298070574258	0.231767939670035\\
9.57054915270152	0.232359192854348\\
9.58811759966045	0.232949981801359\\
9.60578432346384	0.233543604940388\\
9.62345104726722	0.234136753624688\\
9.64111777107061	0.234729425375757\\
9.658784494874	0.235321617736819\\
9.67655059520445	0.23591665531751\\
9.6943166955349	0.236511203182751\\
9.71208279586536	0.237105258900199\\
9.72984889619581	0.237698820059278\\
9.74771549159443	0.238295237558256\\
9.76558208699305	0.23889115006442\\
9.78344868239167	0.239486555192311\\
9.80131527779029	0.240081450578272\\
9.81928350543996	0.240679213498113\\
9.83725173308963	0.241276456131588\\
9.8552199607393	0.241873176140845\\
9.87318818838897	0.242469371209865\\
9.89125920482513	0.243068445092206\\
9.90933022126129	0.243666983378843\\
9.92740123769745	0.244264983780253\\
9.94547225413361	0.244862444028773\\
9.96364723590061	0.245462794464943\\
9.9818222176676	0.246062593980634\\
9.99999719943459	0.246661840335375\\
10.0181721812016	0.247260531310576\\
10.0364523247657	0.247862123927111\\
10.0547324683298	0.248463150283299\\
10.0730126118939	0.249063608188443\\
10.0912927554579	0.249663495473747\\
10.1096792783833	0.250266295959176\\
10.1280658013086	0.250868514829558\\
10.146452324234	0.251470149944694\\
10.1648388471593	0.252071199186297\\
10.1833329877765	0.252675173267662\\
10.2018271283938	0.25327855036472\\
10.220321269011	0.253881328388493\\
10.2388154096282	0.254483505271927\\
10.257418428294	0.255088618743103\\
10.2760214469598	0.255693119846336\\
10.2946244656257	0.256297006544597\\
10.3132274842915	0.256900276822782\\
10.3319406634311	0.257506495532241\\
10.3506538425707	0.258112086475919\\
10.3693670217104	0.258717047669464\\
10.38808020085	0.259321377150443\\
10.4069048454226	0.259928667002447\\
10.4257294899953	0.26053531367677\\
10.4445541345679	0.261141315242462\\
10.4633787791406	0.261746669790487\\
10.4823162175897	0.262354996761314\\
10.5012536560388	0.262962665128564\\
10.5201910944879	0.263569673015422\\
10.539128532937	0.264176018566974\\
10.5581801172831	0.264785348692791\\
10.5772317016292	0.265394004775202\\
10.5962832859753	0.266001984992255\\
10.6153348703214	0.266609287543888\\
10.6345019773376	0.267219586942849\\
10.6536690843538	0.267829196844612\\
10.67283619137	0.268438115482828\\
10.6920032983861	0.269046341113013\\
10.7112873293394	0.269657575960764\\
10.7305713602926	0.27026810584356\\
10.7498553912458	0.270877929051389\\
10.769139422199	0.271487043896079\\
10.7885418041401	0.27209918045019\\
10.8079441860811	0.27271059655752\\
10.8273465680222	0.273321290565137\\
10.8467489499633	0.273931260841918\\
10.8662711360792	0.274544265435608\\
10.8857933221952	0.275156534086503\\
10.9053155083112	0.275768065199493\\
10.9248376944271	0.276378857201242\\
10.9444811647161	0.276992696246739\\
10.9641246350051	0.277605783839064\\
10.9837681052941	0.278218118441684\\
11.0034115755831	0.27882969853979\\
11.02317783773	0.279444338537184\\
11.0429440998769	0.280058211556443\\
11.0627103620238	0.280671316120357\\
11.0824766241707	0.281283650773396\\
11.102367213876	0.281899058301669\\
11.1222578035813	0.282513683312006\\
11.1421483932866	0.28312752438728\\
11.1620389829919	0.283740580131987\\
11.182055464688	0.284356721851533\\
11.202071946384	0.28497206549821\\
11.22208842808	0.285586609715734\\
11.2421049097761	0.286200353169388\\
11.2622488777205	0.286817195835306\\
11.282392845665	0.28743322485792\\
11.3025368136094	0.288048438942558\\
11.3226807815539	0.288662836816051\\
11.3429428732039	0.289280012832919\\
11.363204964854	0.289896360519393\\
11.3834670565041	0.290511878645167\\
11.4037291481542	0.291126566001324\\
11.4240073959591	0.291740910530865\\
11.444285643764	0.292354420606614\\
11.464563891569	0.292967095080658\\
11.4848421393739	0.293578932825928\\
11.5051410232308	0.294190554089775\\
11.5254399070878	0.294801334723371\\
11.5457387909448	0.29541127365851\\
11.5660376748017	0.296020369847311\\
11.5863613245615	0.296629363852137\\
11.6066849743212	0.297237511011533\\
11.627008624081	0.297844810334569\\
11.6473322738407	0.298451260850143\\
11.6676846007299	0.299057715522398\\
11.6880369276191	0.299663317100338\\
11.7083892545083	0.300268064668089\\
11.7287415813975	0.300871957329124\\
11.7491263024508	0.30147595335852\\
11.7695110235041	0.302079090015263\\
11.7898957445574	0.302681366456496\\
11.8102804656107	0.303282781858253\\
11.8307011258767	0.303884393457678\\
11.8511217861426	0.304485139380183\\
11.8715424464086	0.305085018854057\\
11.8919631066745	0.305684031126044\\
11.9124230970895	0.306283326648057\\
11.9328830875046	0.306881750165986\\
11.9533430779196	0.307479300977546\\
11.9738030683347	0.308075978398491\\
11.9943056417768	0.30867302088587\\
12.014808215219	0.309269185021686\\
12.0353107886612	0.309864470171504\\
12.0558133621033	0.310458875718519\\
12.0763616495731	0.311053723446752\\
12.0969099370428	0.311647686457664\\
12.1174582245125	0.31224076418321\\
12.1380065119823	0.312832956072582\\
12.1586035338403	0.313425662937798\\
12.1792005556983	0.314017478703467\\
12.1997975775563	0.314608402866592\\
12.2203945994143	0.315198434941031\\
12.2410432774846	0.315789050864844\\
12.2616919555548	0.316378769291871\\
12.2823406336251	0.316967589782918\\
12.3029893116954	0.31755551191528\\
12.3236924799435	0.318144083200052\\
12.3443956481917	0.318731750576922\\
12.3650988164398	0.319318513669359\\
12.385801984688	0.319904372116958\\
12.406562398648	0.320490941757698\\
12.4273228126081	0.321076601066439\\
12.4480832265682	0.321661349728248\\
12.4688436405282	0.322245187443973\\
12.4896639845707	0.322829795372651\\
12.5104843286133	0.323413486533017\\
12.5313046726558	0.323996260670759\\
12.5521250166983	0.324578117547009\\
12.5730079121072	0.325160800894363\\
12.5938908075161	0.325742561025414\\
12.614773702925	0.326323397745568\\
12.6356565983338	0.326903310875342\\
12.6566046110283	0.32748410421671\\
12.6775526237227	0.328063967882399\\
12.6985006364172	0.328642901736697\\
12.7194486491116	0.329220905658681\\
12.7404642942679	0.329799841168486\\
12.7614799394242	0.330377840532141\\
12.7824955845805	0.330954903672045\\
12.8035112297368	0.33153103052507\\
12.8245969793381	0.332108138208471\\
12.8456827289394	0.332684303264147\\
12.8667684785407	0.3332595256719\\
12.887854228142	0.333833805425694\\
12.9090125143938	0.334409113231701\\
12.9301708006455	0.334983471917347\\
12.9513290868973	0.335556881519177\\
12.972487373149	0.336129342087599\\
12.9937205946389	0.336702876094323\\
13.0149538161287	0.337275454476794\\
13.0361870376186	0.337847077327701\\
13.0574202591085	0.338417744753297\\
13.0787307850171	0.338989529295907\\
13.1000413109258	0.339560351698852\\
13.1213518368345	0.340130212110413\\
13.1426623627431	0.340699110692141\\
13.1640525369264	0.341269168484334\\
13.1854427111097	0.341838257609583\\
13.206832885293	0.342406378271253\\
13.2282230594763	0.342973530685694\\
13.2496952043377	0.343541882932585\\
13.271167349199	0.344109259972952\\
13.2926394940603	0.344675662064782\\
13.3141116389217	0.345241089478767\\
13.3356680588802	0.345807755976254\\
13.3572244788388	0.346373440714839\\
13.3787808987974	0.346938144006714\\
13.400337318756	0.347501866176493\\
13.4219803036483	0.348066865406313\\
13.4436232885406	0.348630876311491\\
13.4652662734329	0.349193899258036\\
13.4869092583252	0.349755934624114\\
13.5086410871513	0.350319283855272\\
13.5303729159775	0.350881638181993\\
13.5521047448036	0.351442998023767\\
13.5738365736297	0.35200336381197\\
13.5956595174591	0.352565079177516\\
13.6174824612885	0.353125793044041\\
13.639305405118	0.353685505884208\\
13.6611283489474	0.354244218182301\\
13.6830446722324	0.354804314713285\\
13.7049609955174	0.355363403135216\\
13.7268773188025	0.35592148397366\\
13.7487936420875	0.35647855776554\\
13.770805608124	0.357037049529629\\
13.7928175741606	0.357594526558222\\
13.8148295401972	0.358150989429551\\
13.8368415062338	0.358706438732941\\
13.858951377984	0.359263338851208\\
13.8810612497342	0.359819217590269\\
13.9031711214845	0.360374075580816\\
13.9252809932347	0.360927913464373\\
13.9474910360997	0.361483234176776\\
13.9697010789646	0.362037526848064\\
13.9919111218296	0.362590792161213\\
14.0141211646945	0.363143030809775\\
14.0364336494706	0.363696783542962\\
14.0587461342466	0.364249501553891\\
14.0810586190226	0.364801185577675\\
14.1033711037986	0.365351836359753\\
14.1257883079129	0.365904031750051\\
14.1482055120272	0.366455185716726\\
14.1706227161414	0.367005299046918\\
14.1930399202557	0.367554372537835\\
14.2155641320733	0.368105020528655\\
14.2380883438909	0.368654620373046\\
14.2606125557085	0.369203172910084\\
14.283136767526	0.369750678988663\\
14.3057702854045	0.37029978879657\\
14.3284038032829	0.370847843712818\\
14.3510373211614	0.37139484462836\\
14.3736708390398	0.371940792443718\\
14.3964159769655	0.372488372684578\\
14.4191611148912	0.373034891264772\\
14.4419062528169	0.373580349127098\\
14.4646513907425	0.374124747223676\\
14.4875104777278	0.374670805884329\\
14.510369564713	0.375215796090382\\
14.5332286516983	0.375759718836474\\
14.5560877386835	0.376302575126318\\
14.5790631229979	0.376847119652075\\
14.6020385073122	0.377390588902975\\
14.6250138916265	0.377932983925517\\
14.6479892759408	0.378474305775027\\
14.6710833264109	0.379017343084262\\
14.6941773768809	0.379559298270678\\
14.7172714273509	0.380100172432684\\
14.740365477821	0.380639966677264\\
14.7635805844029	0.381181503158249\\
14.7867956909849	0.381721950639461\\
14.8100107975669	0.382261310271286\\
14.8332259041488	0.382799583212445\\
14.8565644820063	0.383339624801246\\
14.8799030598637	0.383878570482792\\
14.9032416377212	0.384416421459546\\
14.9265802155787	0.384953178942058\\
14.9500447068131	0.385491731142674\\
14.9735091980476	0.386029180496466\\
14.996973689282	0.386565528258098\\
15.0204381805165	0.387100775690071\\
15.0440310551214	0.387637843575619\\
15.0676239297264	0.388173801641177\\
15.0912168043314	0.388708651193755\\
15.1148096789363	0.389242393547956\\
15.1385334385982	0.389777981825625\\
15.16225719826	0.390312453274807\\
15.1859809579218	0.390845809255033\\
15.2097047175837	0.391378051133179\\
15.2335618956234	0.391912164119237\\
15.2574190736632	0.392445153231391\\
15.281276251703	0.39297701988189\\
15.3051334297427	0.393507765490084\\
15.3291265950906	0.394040407175023\\
15.3531197604384	0.394571917901895\\
15.3771129257863	0.395102299135889\\
15.4011060911341	0.395631552349048\\
15.4252378496272	0.39616272640004\\
15.4493696081203	0.396692762368239\\
15.4735013666134	0.397221661772026\\
15.4976331251065	0.397749426136383\\
15.5219061221034	0.39827913592761\\
15.5461791191004	0.398807700468839\\
15.5704521160973	0.399335121331912\\
15.5947251130942	0.399861400095029\\
15.619142035055	0.400389648713367\\
15.6435589570157	0.400916744870088\\
15.6679758789765	0.401442690190801\\
15.6923928009372	0.401967486307215\\
15.7169563780926	0.402494276580128\\
15.741519955248	0.403019907133353\\
15.7660835324033	0.403544379646587\\
15.7906471095587	0.404067695805379\\
15.8153601179615	0.404593030313342\\
15.8400731263644	0.405117197795001\\
15.8647861347672	0.405640199984494\\
15.88949914317	0.406162038621559\\
15.9143644081624	0.406685919738641\\
15.9392296731548	0.407208626471981\\
15.9640949381471	0.407730160610541\\
15.9889602031395	0.408250523948625\\
16.0139806003288	0.408772953827883\\
16.039000997518	0.409294201912925\\
16.0640213947072	0.409814270047942\\
16.0890417918965	0.410333160082209\\
16.1142202516767	0.410854140713971\\
16.1393987114569	0.411373932085512\\
16.1645771712372	0.411892536096685\\
16.1897556310174	0.41240995465217\\
16.2150951374624	0.412929487802655\\
16.2404346439074	0.413447824169143\\
16.2657741503524	0.413964965707616\\
16.2911136567974	0.414480914378621\\
16.3166172552097	0.414999001706186\\
16.342120853622	0.41551588466542\\
16.3676244520343	0.416031565268927\\
16.3931280504465	0.416546045533608\\
16.4187988465862	0.417062688529975\\
16.4444696427259	0.417578119510561\\
16.4701404388656	0.418092340545114\\
16.4958112350053	0.418605353707412\\
16.5216524004798	0.419120553763222\\
16.5474935659542	0.419634534089786\\
16.5733347314287	0.42014729681455\\
16.5991758969032	0.420658844068722\\
16.6251906694304	0.421172602432692\\
16.6512054419577	0.421685133285182\\
16.677220214485	0.422196438811924\\
16.7032349870123	0.422706521202138\\
16.7294266758655	0.423218839041968\\
16.7556183647187	0.423729921516225\\
16.781810053572	0.424239770869547\\
16.8080017424252	0.42474838934978\\
16.8343737319713	0.425259267770895\\
16.8607457215174	0.425768902897244\\
16.8871177110636	0.42627729703302\\
16.9134897006097	0.426784452485349\\
16.9400454511807	0.427293892495422\\
16.9666012017517	0.427802081203294\\
16.9931569523226	0.428309020973408\\
17.0197127028936	0.42881471417285\\
17.0464557582069	0.42932271677051\\
17.0731988135202	0.429829459976657\\
17.0999418688336	0.430334946216701\\
17.1266849241469	0.430839177918416\\
17.1536189111414	0.431345744032075\\
17.1805528981359	0.4318510425796\\
17.2074868851304	0.432355076048137\\
17.2344208721249	0.432857846926902\\
17.2615495082445	0.433362977493246\\
17.2886781443641	0.433866832229628\\
17.3158067804836	0.434369413685731\\
17.3429354166032	0.434870724413012\\
17.3702625121198	0.435374420357006\\
17.3975896076364	0.435876832114098\\
17.424916703153	0.436377962297353\\
17.4522437986696	0.436877813521306\\
17.4797732620512	0.437380075790126\\
17.5073027254328	0.437881045417811\\
17.5348321888143	0.438380725081693\\
17.5623616521959	0.43887911746027\\
17.5900974943972	0.439379947032887\\
17.6178333365984	0.439879475408496\\
17.6455691787997	0.440377705329625\\
17.6733050210009	0.440874639539658\\
17.7012513604556	0.441374037445287\\
17.7291976999103	0.441872125491817\\
17.757144039365	0.442368906487953\\
17.7850903788197	0.442864383242937\\
17.8132514459017	0.443362350565704\\
17.8414125129837	0.443858999256328\\
17.8695735800657	0.444354332190719\\
17.8977346471477	0.444848352244998\\
17.9261147917276	0.445344890178957\\
17.9544949363076	0.445840100591604\\
17.9828750808876	0.446333986427127\\
18.0112552254676	0.446826550629601\\
18.0398589202161	0.447321660453082\\
18.0684626149646	0.447815433744748\\
18.097066309713	0.448307873518199\\
18.1256700044615	0.448798982786586\\
18.1545018545664	0.449292665943913\\
18.1833337046712	0.449785003431768\\
18.212165554776	0.450275998334352\\
18.2409974048809	0.450765653735073\\
18.2700621491713	0.451257911763527\\
18.2991268934616	0.451748814852077\\
18.328191637752	0.452238366156774\\
18.3572563820424	0.452726568832525\\
18.3865589068877	0.453217403503261\\
18.4158614317331	0.453706873824295\\
18.4451639565785	0.454194983024839\\
18.4744664814239	0.4546817343326\\
18.5040118216459	0.455171147562833\\
18.5335571618679	0.45565918688786\\
18.56310250209	0.456145855611429\\
18.592647842312	0.456631157035417\\
18.622441195706	0.457119151023872\\
18.6522345491	0.457605761398575\\
18.682027902494	0.458090991539258\\
18.711821255888	0.458574844823406\\
18.74186798467	0.45906142195995\\
18.771914713452	0.459546605614059\\
18.801961442234	0.46003039924297\\
18.8320081710161	0.460512806301289\\
18.8623138179204	0.460997969311015\\
18.8926194648247	0.461481728801293\\
18.922925111729	0.461964088308465\\
18.9532307586333	0.462445051365845\\
18.9838010512769	0.462928803251853\\
19.0143713439205	0.463411141404887\\
19.0449416365641	0.463892069442076\\
19.0755119292076	0.46437159097712\\
19.1063527942164	0.464853935112972\\
19.1371936592252	0.465334855116764\\
19.1680345242339	0.465814354688182\\
19.1988753892427	0.466292437523071\\
19.2299929618915	0.466773377662566\\
19.2611105345403	0.467252883075875\\
19.2922281071891	0.467730957547112\\
19.3233456798379	0.468207604856119\\
19.3547463174076	0.468687145191388\\
19.3861469549774	0.469165240001094\\
19.4175475925471	0.469641893155745\\
19.4489482301169	0.470117108521135\\
19.4806385226382	0.47059525368907\\
19.5123288151595	0.471071942316132\\
19.5440191076808	0.471547178361292\\
19.5757094002021	0.472020965778362\\
19.6076961882754	0.472497720963095\\
19.6396829763487	0.472973008363937\\
19.671669764422	0.473446832030518\\
19.7036565524953	0.473919196006837\\
19.7359469398787	0.47439456694843\\
19.768237327262	0.474868458623074\\
19.8005277146453	0.475340875173366\\
19.8328181020287	0.475811820735794\\
19.8654194755566	0.47628581383443\\
19.8980208490846	0.476758315929477\\
19.9306222226126	0.477229331258948\\
19.9632235961405	0.477698864054245\\
19.9961436388709	0.478171486358573\\
20.0290636816012	0.478642605655017\\
20.0619837243316	0.479112226279589\\
20.0949037670619	0.479580352561177\\
20.1281504830269	0.480051611909759\\
20.1613971989918	0.480521355962876\\
20.1946439149567	0.480989589157279\\
20.2278906309217	0.481456315922065\\
20.2614723645969	0.481926220987445\\
20.2950540982721	0.482394598169848\\
20.3286358319473	0.482861452009667\\
20.3622175656225	0.483326787039093\\
20.3961430262838	0.483795347421728\\
20.4300684869452	0.484262367015884\\
20.4639939476066	0.48472785046868\\
20.497919408268	0.485191802418464\\
20.5321976950164	0.485659028727146\\
20.5664759817649	0.486124701004392\\
20.6007542685134	0.486588824007328\\
20.6350325552618	0.487051402483711\\
20.6696731863085	0.487517306450185\\
20.7043138173553	0.487981642783623\\
20.738954448402	0.488444416354639\\
20.7735950794487	0.488905632023866\\
20.8086080231198	0.489370226612937\\
20.8436209667909	0.489833239585718\\
20.878633910462	0.490294675930031\\
20.9136468541332	0.490754540623072\\
20.949042565344	0.491217840185363\\
20.9844382765548	0.491679543741231\\
21.0198339877656	0.492139656399665\\
21.0552296989764	0.492598183258365\\
21.0910191546516	0.493060203635191\\
21.1268086103268	0.493520613181305\\
21.162598066002	0.4939794171311\\
21.1983875216773	0.494436620706989\\
21.234582265905	0.494897379426164\\
21.2707770101326	0.495356512026084\\
21.3069717543603	0.495814023871083\\
21.343166498588	0.496269920312788\\
21.3797786897193	0.49672943676604\\
21.4163908808506	0.497187311314252\\
21.453003071982	0.497643549456562\\
21.4896152631133	0.498098156678641\\
21.5266577262528	0.49855645232125\\
21.5637001893923	0.499013089739546\\
21.6007426525319	0.499468074572694\\
21.6377851156714	0.499921412445599\\
21.6752714017348	0.500378511026388\\
21.7127576877983	0.500833934490139\\
21.7502439738617	0.50128768862167\\
21.7877302599251	0.501739779190703\\
21.8256747161126	0.502195707062926\\
21.8636191723001	0.502649942307379\\
21.9015636284876	0.5031024908606\\
21.9395080846751	0.503553358643158\\
21.9779259262067	0.504008145022736\\
22.0163437677384	0.504461220596711\\
22.0547616092701	0.504912591459901\\
22.0931794508017	0.505362263690232\\
22.1320868496398	0.505815941043507\\
22.1709942484778	0.506267888691263\\
22.2099016473158	0.506718112893712\\
22.2488090461539	0.5071666198932\\
22.2882232273236	0.507619224326258\\
22.3276374084934	0.508070079370466\\
22.3670515896632	0.508519191459159\\
22.4064657708329	0.508966567006777\\
22.4464051256478	0.509418138752765\\
22.4863444804626	0.509867940574192\\
22.5262838352775	0.510315979085935\\
22.5662231900924	0.510762260882893\\
22.6067074061995	0.511212844866395\\
22.6471916223066	0.511661637459395\\
22.6876758384138	0.512108645467522\\
22.7281600545209	0.512553875675272\\
22.7692102626757	0.513003522132908\\
22.8102604708306	0.51345135471594\\
22.8513106789854	0.513897380430848\\
22.8923608871403	0.514341606261753\\
22.9339998357608	0.514790371518243\\
22.9756387843813	0.515237299297708\\
23.0172777330018	0.515682396818596\\
23.0589166816223	0.516125671275688\\
23.1011689388352	0.516573618639194\\
23.143421196048	0.517019703690464\\
23.1856734532609	0.517463933872199\\
23.2279257104737	0.51790631660203\\
23.2707449436263	0.518352758362857\\
23.3135641767789	0.518797318018003\\
23.3563834099314	0.519240003208987\\
23.399202643084	0.519680821550942\\
23.4425351018712	0.520125030692615\\
23.4858675606585	0.520567343574594\\
23.5292000194457	0.521007768006662\\
23.5725324782329	0.52144631177101\\
23.6163923812599	0.521888286290115\\
23.6602522842869	0.522328350036725\\
23.704112187314	0.522766510994405\\
23.747972090341	0.523202777117863\\
23.7923730104802	0.523642502704116\\
23.8367739306194	0.524080302737128\\
23.8811748507586	0.524516185379378\\
23.9255757708978	0.524950158763182\\
23.9705318120247	0.525387620960144\\
24.0154878531517	0.525823142543764\\
24.0604438942787	0.526256731860835\\
24.1053999354057	0.526688397226617\\
24.1509257391711	0.527123581217141\\
24.1964515429365	0.527556809244768\\
24.2419773467019	0.527988089846253\\
24.2875031504673	0.528417431525404\\
24.3336139249026	0.528850322125406\\
24.3797246993379	0.529281241112894\\
24.4258354737732	0.529710197220526\\
24.4719462482085	0.530137199146522\\
24.518657797687	0.530567780793143\\
24.5653693471654	0.530996374866498\\
24.6120808966439	0.531422990301367\\
24.6587924461223	0.531847635996549\\
24.7061212033434	0.532275892735997\\
24.7534499605645	0.532702145618693\\
24.8007787177855	0.533126403788083\\
24.8481074750066	0.533548676350011\\
24.8960705351928	0.533974591825228\\
24.944033595379	0.53439848682534\\
24.9919966555653	0.534820370709322\\
25.0399597157515	0.535240252796858\\
25.0885748730688	0.535663810233492\\
25.1371900303861	0.536085330229083\\
25.1858051877034	0.536504822365358\\
25.2344203450208	0.536922296182985\\
25.2837061315614	0.537343478374969\\
25.332991918102	0.537762605799066\\
25.3822777046427	0.538179688267349\\
25.4315634911833	0.53859473554898\\
25.4815392186928	0.539013524843386\\
25.5315149462023	0.539430241668179\\
25.5814906737117	0.539844896073772\\
25.6314664012212	0.540257498065725\\
25.6821522057685	0.540673876346541\\
25.7328380103157	0.54108816406647\\
25.783523814863	0.541500371522685\\
25.8342096194102	0.541910508965464\\
25.8856265092156	0.542324457635935\\
25.9370433990211	0.542736297249302\\
25.9884602888265	0.543146038358368\\
26.0398771786319	0.543553691466901\\
26.0920470854784	0.543965191431568\\
26.144216992325	0.544374563421793\\
26.1963868991715	0.544781818255359\\
26.2485568060181	0.54518696669877\\
26.3015026403881	0.545595998343756\\
26.3544484747581	0.546002882658824\\
26.4073943091281	0.546407630736613\\
26.4603401434981	0.546810253616117\\
26.5140858539001	0.54721679678858\\
26.5678315643021	0.547621172819615\\
26.6215772747041	0.548023393087139\\
26.6753229851061	0.548423468912939\\
26.7298936218998	0.54882750289906\\
26.7844642586936	0.549229349457364\\
26.8390348954873	0.549629020262059\\
26.893605532281	0.550026526928605\\
26.9490273164448	0.550428030430224\\
27.0044491006085	0.550827325722894\\
27.0598708847722	0.551224424788769\\
27.1152926689359	0.551619339548489\\
27.1715930662283	0.552018290658198\\
27.2278934635207	0.552415012262148\\
27.2841938608131	0.552809516662767\\
27.3404942581055	0.553201816098054\\
27.3972810276665	0.553595283307494\\
27.4540677972275	0.553986532101163\\
27.5108545667884	0.554375574838622\\
27.5676413363494	0.554762423813882\\
27.6246835621125	0.555148816777874\\
27.6817257878755	0.555533020865128\\
27.7387680136386	0.555915048337423\\
27.7958102394016	0.55629491139096\\
27.8531117250382	0.556674333976733\\
27.9104132106748	0.557051596896412\\
27.9677146963114	0.557426712314276\\
28.025016181948	0.557799692329023\\
28.0825792753208	0.558172237255004\\
28.1401423686935	0.558542651511315\\
28.1977054620663	0.55891094716412\\
28.2552685554391	0.559277136214016\\
28.3130956457549	0.559642895605802\\
28.3709227360708	0.560006553106694\\
28.4287498263866	0.560368120684202\\
28.4865769167025	0.560727610240299\\
28.5446704260258	0.561086675557217\\
28.602763935349	0.561443667541888\\
28.6608574446723	0.561798598062691\\
28.7189509539956	0.562151478922518\\
28.7773133374	0.562503940945063\\
28.8356757208045	0.562854357971657\\
28.8940381042089	0.563202741771128\\
28.9524004876134	0.563549104046891\\
29.0110342352743	0.56389505287672\\
29.0696679829352	0.564238984822422\\
29.1283017305961	0.564580911552923\\
29.186935478257	0.564920844671816\\
29.2458431140456	0.565260369711119\\
29.3047507498342	0.565597905751882\\
29.3636583856228	0.565933464362825\\
29.4225660214114	0.566267057047431\\
29.4817501049045	0.56660024699982\\
29.5409341883977	0.566931475611344\\
29.6001182718908	0.567260754350266\\
29.659302355384	0.567588094619725\\
29.7187654815208	0.567915037477732\\
29.7782286076576	0.568240046423213\\
29.8376917337944	0.568563132823769\\
29.8971548599312	0.568884307982006\\
29.9568996607722	0.569205091027218\\
30.0166444616132	0.569523967357577\\
30.0763892624543	0.569840948239865\\
30.1361340632953	0.570156044876012\\
30.1961632086509	0.570470754672992\\
30.2561923540065	0.570783584720965\\
30.316221499362	0.571094546185781\\
30.3762506447176	0.571403650168595\\
30.436566841334	0.571712372552023\\
30.4968830379503	0.572019241919548\\
30.5571992345667	0.572324269336018\\
30.617515431183	0.572627465801745\\
30.6781214246657	0.572930285878793\\
30.7387274181483	0.573231279439452\\
30.799333411631	0.573530457447529\\
30.8599394051136	0.573827830802477\\
30.9208379817972	0.574124832954471\\
30.9817365584807	0.574420034855216\\
31.0426351351642	0.574713447367485\\
31.1035337118477	0.575005081289882\\
31.1647276970968	0.575296349156654\\
31.2259216823458	0.575585842802441\\
31.2871156675948	0.575873572989017\\
31.3483096528438	0.576159550414179\\
31.4098019130839	0.576445166897814\\
31.4712941733241	0.576729034955384\\
31.5327864335642	0.577011165247734\\
31.5942786938043	0.577291568371933\\
31.656072138367	0.577571615639135\\
31.7178655829298	0.577849940039441\\
31.7796590274925	0.578126552132864\\
31.8414524720552	0.578401462415855\\
31.903550051626	0.578676021884984\\
31.9656476311968	0.578948883810467\\
32.0277452107676	0.57922005865161\\
32.0898427903383	0.579489556804379\\
32.1522474993513	0.579758709151334\\
32.2146522083642	0.580026189041811\\
32.2770569173772	0.580292006834563\\
32.3394616263901	0.580556172825232\\
32.4021765035977	0.580819997980928\\
32.4648913808053	0.581082175531181\\
32.527606258013	0.581342715734371\\
32.5903211352206	0.581601628786004\\
32.6533492654047	0.581860205938959\\
32.7163773955889	0.582117160101368\\
32.7794055257731	0.582372501431444\\
32.8424336559572	0.582626240024764\\
32.9057781690946	0.582879647613267\\
32.969122682232	0.583131456590157\\
33.0324671953693	0.583381677013697\\
33.0958117085067	0.583630318879769\\
33.1594757818323	0.5838786345966\\
33.2231398551579	0.584125375844965\\
33.2868039284835	0.58437055258343\\
33.3504680018091	0.584614174708427\\
33.4144548595075	0.584857475496079\\
33.478441717206	0.585099225722952\\
33.5424285749044	0.585339435248172\\
33.6064154326028	0.585578113868999\\
33.6707283499096	0.58581647593211\\
33.7350412672163	0.586053311106897\\
33.7993541845231	0.586288629153339\\
33.8636671018298	0.586522439769805\\
33.9283094023156	0.586755938561405\\
33.9929517028013	0.586987933902392\\
34.057594003287	0.587218435453892\\
34.1222363037727	0.587447452815693\\
34.1872113622324	0.587676163045382\\
34.2521864206922	0.58790339302787\\
34.3171614791519	0.588129152325749\\
34.3821365376116	0.588353450440544\\
34.447447781095	0.588577446075385\\
34.5127590245783	0.588799984432645\\
34.5780702680616	0.589021074976713\\
34.643381511545	0.589240727111191\\
34.7090324201268	0.589460081375873\\
34.7746833287086	0.589678001099377\\
34.8403342372905	0.589894495648235\\
34.9059851458723	0.590109574328474\\
34.9719792533529	0.590324359705517\\
35.0379733608335	0.590537733045209\\
35.1039674683141	0.590749703616582\\
35.1699615757948	0.590960280628451\\
35.2363024710846	0.591170568860866\\
35.3026433663745	0.591379467327853\\
35.3689842616644	0.591586985201317\\
35.4353251569542	0.591793131593235\\
35.5020164861975	0.591998993689399\\
35.5687078154407	0.59220348806083\\
35.6353991446839	0.592406623782691\\
35.7020904739271	0.592608409870508\\
35.76913594038	0.592809916101904\\
35.8361814068328	0.593010076418811\\
35.9032268732857	0.593208899800041\\
35.9702723397385	0.593406395165066\\
36.0376757049608	0.593603615068561\\
36.1050790701831	0.593799510638167\\
36.1724824354054	0.593994090756749\\
36.2398858006277	0.594187364248134\\
36.3076508877251	0.594380366634119\\
36.3754159748225	0.594572066037939\\
36.4431810619199	0.594762471246927\\
36.5109461490173	0.594951590989677\\
36.5790768402747	0.595140443933835\\
36.6472075315322	0.595328015019617\\
36.7153382227897	0.595514312939248\\
36.7834689140472	0.595699346326514\\
36.8519691566203	0.595884117184937\\
36.9204693991935	0.596067627081646\\
36.9889696417667	0.596249884614184\\
37.0574698843398	0.596430898321967\\
37.1263436888854	0.596611653725873\\
37.1952174934311	0.596791168838535\\
37.2640912979767	0.596969452163258\\
37.3329651025223	0.597146512145524\\
37.4022165438372	0.597323318002303\\
37.4714679851522	0.597498904013111\\
37.5407194264671	0.597673278587457\\
37.609970867782	0.597846450077338\\
37.6796040887873	0.598019371580191\\
37.7492373097926	0.59819109345807\\
37.8188705307979	0.598361624027142\\
37.8885037518033	0.598530971546371\\
37.9585229632751	0.59870007317263\\
38.0285421747469	0.598867995171648\\
38.0985613862187	0.599034745766706\\
38.1685805976905	0.599200333124197\\
38.2389900798134	0.599365678638633\\
38.3093995619363	0.599529864301323\\
38.3798090440592	0.599692898243125\\
38.4502185261821	0.599854788538328\\
38.5209622984945	0.600016303740198\\
38.591706070807	0.600176680580219\\
38.6624498431194	0.600335927076838\\
38.7331936154318	0.600494051192442\\
38.8042656863826	0.600651786878262\\
38.8753377573333	0.600808405682651\\
38.9464098282841	0.600963915509576\\
39.0174818992348	0.601118324207473\\
39.0888909542047	0.601272363913406\\
39.1603000091745	0.60142530773965\\
39.2317090641443	0.601577163478592\\
39.3031181191142	0.601727938867606\\
39.3748725127202	0.601878362967238\\
39.4466269063262	0.602027711738804\\
39.5183812999322	0.602175992865771\\
39.5901356935382	0.602323213977095\\
39.662243568804	0.60247010012161\\
39.7343514440698	0.602615931062783\\
39.8064593193355	0.602760714377634\\
39.8785671946013	0.602904457589153\\
39.95103651206	0.603047880917837\\
40.0235058295186	0.603190268762003\\
40.0959751469772	0.603331628594525\\
40.1684444644358	0.603471967834713\\
40.2412830193083	0.603612001144884\\
40.3141215741807	0.60375101830275\\
40.3869601290532	0.603889026679179\\
40.4597986839256	0.604026033591933\\
40.5330141309314	0.604162747510438\\
40.6062295779373	0.604298464239725\\
40.6794450249431	0.604433191050657\\
40.752660471949	0.604566935161439\\
40.8262603410054	0.604700398285358\\
40.8998602100619	0.604832882830087\\
40.9734600791183	0.604964395968359\\
41.0470599481748	0.605094944820682\\
41.1210516636287	0.605225223853693\\
41.1950433790826	0.605354542579144\\
41.2690350945364	0.605482908073394\\
41.3430268099903	0.605610327361\\
41.4174177047231	0.60573748723004\\
41.4918085994559	0.60586370473824\\
41.5661994941887	0.605988986867232\\
41.6405903889215	0.606113340547259\\
41.7153877210576	0.606237444514743\\
41.7901850531936	0.606360623755528\\
41.8649823853296	0.60648288515807\\
41.9397797174656	0.606604235559845\\
42.0149906816255	0.606725345318161\\
42.0902016457854	0.60684554768274\\
42.1654126099454	0.606964849450323\\
42.2406235741053	0.60708325736707\\
42.316255315397	0.607201433130498\\
42.3918870566888	0.607318718542433\\
42.4675187979805	0.607435120309282\\
42.5431505392722	0.60755064508726\\
42.6192101664345	0.607665945675412\\
42.6952697935968	0.607780372673213\\
42.771329420759	0.60789393269804\\
42.8473890479213	0.608006632317463\\
42.9238836429179	0.608119115227013\\
43.0003782379145	0.608230741035164\\
43.0768728329111	0.608341516271497\\
43.1533674279077	0.60845144741617\\
43.2303040564086	0.60856116888869\\
43.3072406849096	0.608670049484804\\
43.3841773134105	0.608778095647476\\
43.4611139419115	0.608885313770619\\
43.5384996649489	0.60899232885695\\
43.6158853879863	0.609098519035496\\
43.6932711110238	0.609203890663721\\
43.7706568340612	0.609308450050402\\
43.8484987166093	0.609412812665868\\
43.9263405991575	0.609516366092844\\
44.0041824817057	0.609619116604352\\
44.0820243642538	0.609721070425098\\
44.1603294838729	0.609822833400827\\
44.2386346034919	0.60992380266457\\
44.3169397231109	0.610023984405931\\
44.39524484273	0.610123384766552\\
44.4740202988846	0.610222599897892\\
44.5527957550393	0.610321036557027\\
44.631571211194	0.610418700851111\\
44.7103466673487	0.610515598839692\\
44.7895995895968	0.610612316929593\\
44.8688525118449	0.610708271555983\\
44.948105434093	0.610803468744497\\
45.0273583563411	0.610897914473519\\
45.1070959123988	0.61099218537259\\
45.1868334684566	0.611085707591007\\
45.2665710245144	0.611178487073781\\
45.3463085805721	0.611270529719021\\
45.4265379845256	0.611362402362809\\
45.5067673884792	0.611453540887846\\
45.5869967924327	0.611543951159383\\
45.6672261963862	0.611633638996111\\
45.7479547148514	0.611723161437325\\
45.8286832333165	0.611811964105327\\
45.9094117517817	0.611900052786435\\
45.9901402702469	0.611987433220746\\
46.0713752310891	0.612074652662051\\
46.1526101919314	0.61216116646358\\
46.2338451527736	0.612246980333516\\
46.3150801136159	0.612332099934164\\
46.3968289125376	0.612417062756204\\
46.4785777114593	0.612501333863799\\
46.560326510381	0.612584918887775\\
46.6420753093027	0.612667823413415\\
46.7243454163926	0.612750575200838\\
46.8066155234824	0.612832648994803\\
46.8888856305723	0.612914050349529\\
46.9711557376621	0.612994784774027\\
47.0539547042701	0.613075370340004\\
47.136753670878	0.613155291432699\\
47.2195526374859	0.613234553530449\\
47.3023516040939	0.613313162066718\\
47.3856870701744	0.613391625476142\\
47.4690225362549	0.613469437734957\\
47.5523580023354	0.61354660424633\\
47.6356934684159	0.613623130368884\\
47.7195731680798	0.613699514958214\\
47.8034528677438	0.613775261525246\\
47.8873325674077	0.613850375398666\\
47.9712122670716	0.613924861862944\\
48.0556440359198	0.613999210260328\\
48.1400758047679	0.614072933572317\\
48.2245075736161	0.614146037053787\\
48.3089393424642	0.614218525915726\\
48.3939311234603	0.614290880058385\\
48.4789229044563	0.614362621863944\\
48.5639146854524	0.614433756514125\\
48.6489064664485	0.614504289147092\\
48.7344663172648	0.614574690298757\\
48.820026168081	0.614644491675653\\
48.9055860188973	0.614713698386992\\
48.9911458697136	0.614782315498754\\
49.0772819670679	0.614850804263414\\
49.1634180644223	0.614918705632213\\
49.2495541617767	0.614986024642482\\
49.3356902591311	0.615052766288644\\
49.4224109073406	0.615119382626801\\
49.5091315555501	0.615185423766966\\
49.5958522037596	0.61525089467521\\
49.6825728519691	0.615315800275016\\
49.769886487838	0.615380583515886\\
49.857200123707	0.615444803577897\\
49.944513759576	0.615508465356462\\
50.0318273954449	0.61557157370473\\
50.1197425956162	0.615634562559549\\
50.2076577957875	0.615697000078082\\
50.2955729959588	0.615758891085686\\
50.3834881961302	0.615820240365773\\
50.4720136825001	0.615881472938772\\
50.5605391688701	0.615942165843607\\
50.64906465524	0.616002323836166\\
50.73759014161	0.616061951630709\\
50.8267347882915	0.616121465430509\\
50.915879434973	0.616180451057828\\
51.0050240816545	0.616238913199661\\
51.094168728336	0.616296856501699\\
51.1839415683769	0.616354688451755\\
51.2737144084178	0.616412003554505\\
51.3634872484587	0.616468806428628\\
51.4532600884996	0.616525101651813\\
51.5436703203087	0.616581288099991\\
51.6340805521178	0.616636968857399\\
51.7244907839268	0.616692148474961\\
51.8149010157359	0.616746831462932\\
51.9059580087448	0.616801408190048\\
51.9970150017537	0.6168554902161\\
52.0880719947626	0.616909082024819\\
52.1791289877716	0.616962188059582\\
52.270842290771	0.617015190289035\\
52.3625555937704	0.61706770864204\\
52.4542688967698	0.617119747535686\\
52.5459821997693	0.617171311347028\\
52.6383615465513	0.617222773752216\\
52.7307408933333	0.617273762942181\\
52.8231202401153	0.617324283267916\\
52.9154995868973	0.617374339040699\\
53.0085549039533	0.617424295753204\\
53.1016102210094	0.617473789749957\\
53.1946655380654	0.617522825316403\\
53.2877208551214	0.617571406698583\\
53.3814622684928	0.617619891315371\\
53.4752036818643	0.617667923555728\\
53.5689450952358	0.617715507640089\\
53.6626865086073	0.617762647749797\\
53.7571243510415	0.617809693340185\\
53.8515621934757	0.617856296734876\\
53.9460000359099	0.617902462089829\\
54.0404378783442	0.617948193522224\\
54.1355826971555	0.617993832634801\\
54.2307275159668	0.618039039575351\\
54.325872334778	0.618083818435887\\
54.4210171535893	0.618128173269959\\
54.5168797176294	0.618172437938686\\
54.6127422816695	0.618216280303487\\
54.7086048457095	0.618259704392958\\
54.8044674097496	0.618302714197544\\
54.9010587178797	0.618345635948239\\
54.9976500260098	0.618388145109001\\
55.0942413341398	0.618430245645532\\
55.1908326422699	0.6184719414857\\
55.2881639308243	0.618513551341702\\
55.3854952193786	0.618554758169095\\
55.4828265079329	0.618595565871209\\
55.5801577964873	0.618635978313855\\
55.6782405480441	0.618676306801979\\
55.776323299601	0.618716241671556\\
55.8744060511579	0.618755786764065\\
55.9724888027147	0.61879494588378\\
56.0713347542828	0.618834023039696\\
56.1701807058509	0.618872715837257\\
56.2690266574189	0.61891102805661\\
56.367872608987	0.618948963441007\\
56.4674937609263	0.618986818814662\\
56.5671149128656	0.619024298941653\\
56.6667360648049	0.619061407541308\\
56.7663572167441	0.619098148296377\\
56.8667658411891	0.619134810956962\\
56.9671744656341	0.619171107335426\\
57.0675830900791	0.619207041090796\\
57.1679917145241	0.619242615845828\\
57.2692003658787	0.619278114387274\\
57.3704090172333	0.619313255465326\\
57.4716176685879	0.61934804267922\\
57.5728263199425	0.619382479592233\\
57.6748478434472	0.619416842137729\\
57.7768693669519	0.619450855894065\\
57.8788908904565	0.619484524401198\\
57.9809124139612	0.619517851163434\\
58.0837599560178	0.619551105370378\\
58.1866074980744	0.619584019319211\\
58.2894550401309	0.619616596491118\\
58.3923025821875	0.619648840331945\\
58.4959896007833	0.619681013396686\\
58.5996766193791	0.619712854592472\\
58.7033636379749	0.619744367342225\\
58.8070506565707	0.619775555033841\\
58.911590931702	0.619806673696201\\
59.0161312068333	0.619837468738158\\
59.1206714819646	0.619867943524874\\
59.2252117570959	0.619898101386801\\
59.3306194021433	0.6199281919347\\
59.4360270471907	0.619957966971412\\
59.541434692238	0.619987429804848\\
59.6468423372854	0.620016583708523\\
59.7351317529641	0.620040767621464\\
59.8234211686427	0.620064738956047\\
59.9117105843214	0.620088499580744\\
60	0.620112051347715\\
};
\addlegendentry{Rimossi}

\end{axis}
\end{tikzpicture}%}}
 \quad 
\subfloat[][Nodo 2]
{\resizebox{0.45\textwidth}{!}{ % This file was created by matlab2tikz.
%
%The latest updates can be retrieved from
%  http://www.mathworks.com/matlabcentral/fileexchange/22022-matlab2tikz-matlab2tikz
%where you can also make suggestions and rate matlab2tikz.
%
\definecolor{mycolor1}{rgb}{0.00000,0.44700,0.74100}%
\definecolor{mycolor2}{rgb}{0.85000,0.32500,0.09800}%
\definecolor{mycolor3}{rgb}{0.92900,0.69400,0.12500}%
%
\begin{tikzpicture}

\begin{axis}[%
width=0.39\columnwidth,
height=1.7in,
at={(1.011in,0.642in)},
scale only axis,
xmin=0,
xmax=60,
xlabel style={font=\color{white!15!black}},
xlabel={T},
ymin=0,
ymax=1,
axis background/.style={fill=white},
legend style={legend cell align=left, align=left, draw=none,fill=none}
]
\addplot [color=mycolor1, line width=2.0pt]
  table[row sep=crcr]{%
0	1\\
0.00199053585276749	0.999999821758677\\
0.00398107170553497	0.999999287271421\\
0.00597160755830246	0.999998396893462\\
0.00796214341106994	0.999997150980241\\
0.0123567669950473	0.999993143127537\\
0.0167513905790247	0.999987407889205\\
0.0211460141630021	0.999979949100724\\
0.0255406377469794	0.999970770602215\\
0.0299388563931551	0.999959866624738\\
0.0343370750393307	0.999947247832025\\
0.0387352936855064	0.999932918087282\\
0.043133512331682	0.999916881258003\\
0.0475471725191068	0.999899075937224\\
0.0519608327065316	0.999879559339989\\
0.0563744928939564	0.999858335387438\\
0.0607881530813811	0.999835408004714\\
0.0652173199794049	0.999810691606811\\
0.0696464868774286	0.999784267719524\\
0.0740756537754524	0.99975614032141\\
0.0785048206734761	0.999726313394739\\
0.0829496187548012	0.999694676680215\\
0.0873944168361263	0.999661336466192\\
0.0918392149174515	0.999626296788277\\
0.0962840129987766	0.999589561685501\\
0.100744568101794	0.999550995972667\\
0.105205123204811	0.999510730952913\\
0.109665678307829	0.999468770718526\\
0.114126233410846	0.999425119364925\\
0.118602673075784	0.999379616525651\\
0.123079112740721	0.999332418774377\\
0.127555552405659	0.999283530259702\\
0.132031992070596	0.999232955133067\\
0.136524445545197	0.999180507589632\\
0.141016899019798	0.999126369731142\\
0.1455093524944	0.999070545762137\\
0.150001805969001	0.999013039889705\\
0.154510404254387	0.998953640611345\\
0.159019002539772	0.998892555816577\\
0.163527600825158	0.998829789765512\\
0.168036199110544	0.998765346720515\\
0.172561074936287	0.99869898922088\\
0.17708595076203	0.998630951204923\\
0.181610826587773	0.99856123698795\\
0.186135702413515	0.998489850887231\\
0.190676990405524	0.998416529219213\\
0.195218278397533	0.998341532236036\\
0.199759566389542	0.998264864307833\\
0.204300854381551	0.998186529806404\\
0.208858690959197	0.998106238560577\\
0.213416527536843	0.998024277401592\\
0.217974364114488	0.997940650754035\\
0.222532200692134	0.997855363043858\\
0.227106724202768	0.997768097343565\\
0.231681247713401	0.997679167332647\\
0.236255771224035	0.997588577489767\\
0.240830294734669	0.997496332294659\\
0.245421645441331	0.997402087793263\\
0.250012996147992	0.997306184784048\\
0.254604346854654	0.997208627799382\\
0.259195697561315	0.9971094213724\\
0.263804017673384	0.997008194250009\\
0.268412337785452	0.99690531462264\\
0.273020657897521	0.996800787075986\\
0.27762897800959	0.996694616196209\\
0.282254411746462	0.996586403155677\\
0.286879845483334	0.996476543812793\\
0.291505279220206	0.9963650428062\\
0.296130712957079	0.996251904774706\\
0.300773406613671	0.996136703037373\\
0.305416100270264	0.996019861399845\\
0.310058793926856	0.995901384553338\\
0.314701487583449	0.995781277188931\\
0.319361589521735	0.99565908449209\\
0.324021691460022	0.995535258496541\\
0.328681793398309	0.995409803945692\\
0.333341895336596	0.995282725582511\\
0.338019556071109	0.995153540174692\\
0.342697216805623	0.995022728268756\\
0.347374877540136	0.994890294659924\\
0.35205253827465	0.994756244142672\\
0.356747910447269	0.994620064781785\\
0.361443282619888	0.994482265922313\\
0.366138654792507	0.994342852410912\\
0.370834026965126	0.994201829093182\\
0.375547265516765	0.994058655039212\\
0.380260504068404	0.993913868684896\\
0.384973742620042	0.993767474927943\\
0.389686981171681	0.993619478664697\\
0.394418243246227	0.993469309679903\\
0.399149505320772	0.993317535791587\\
0.403880767395317	0.993164161948125\\
0.408612029469863	0.993009193096221\\
0.41336147459102	0.992852029437424\\
0.418110919712177	0.992693268470312\\
0.422860364833334	0.992532915193547\\
0.427609809954491	0.992370974603809\\
0.432377599961091	0.992206817021985\\
0.437145389967691	0.992041069925331\\
0.44191317997429	0.991873738362411\\
0.44668096998089	0.991704827379495\\
0.451467269141767	0.991533677113835\\
0.456253568302644	0.991360945324975\\
0.461039867463521	0.991186637110996\\
0.465826166624398	0.991010757567365\\
0.47063114173014	0.99083261633969\\
0.475436116835881	0.990652901778442\\
0.480241091941623	0.990471619030831\\
0.485046067047365	0.990288773241141\\
0.489869887370685	0.990103643255066\\
0.494693707694006	0.989916948322975\\
0.499517528017326	0.989728693640822\\
0.504341348340647	0.989538884401315\\
0.509184185682064	0.989346768338758\\
0.51402702302348	0.989153095915629\\
0.518869860364897	0.988957872376234\\
0.523712697706314	0.988761102961316\\
0.528574726619363	0.9885620039717\\
0.533436755532412	0.988361357404702\\
0.53829878444546	0.988159168552592\\
0.543160813358509	0.987955442703756\\
0.548042211039569	0.987749364406143\\
0.55292360872063	0.987541747512073\\
0.55780500640169	0.987332597361391\\
0.56268640408275	0.987121919289734\\
0.56758735048985	0.986908865766266\\
0.572488296896949	0.986694282824975\\
0.577389243304049	0.986478175852884\\
0.582290189711149	0.986260550232489\\
0.587210867601438	0.986040526025058\\
0.592131545491728	0.985818981776143\\
0.597052223382018	0.985595922919559\\
0.601972901272307	0.985371354884264\\
0.606913496388328	0.985144364984507\\
0.611854091504349	0.984915864617265\\
0.61679468662037	0.984685859262745\\
0.621735281736391	0.984454354395976\\
0.626695982628195	0.984220404251637\\
0.631656683519999	0.983984953411585\\
0.636617384411802	0.983748007402026\\
0.641578085303606	0.983509571743659\\
0.646559083672327	0.983268667241442\\
0.651540082041047	0.983026272013029\\
0.656521080409768	0.982782391630227\\
0.661502078778488	0.982537031659008\\
0.666503569383853	0.982289179127142\\
0.671505059989219	0.98203984603643\\
0.676506550594584	0.981789038003884\\
0.68150804119995	0.98153676064035\\
0.686530221876942	0.981281966846795\\
0.691552402553934	0.981025702859724\\
0.696574583230927	0.980767974340955\\
0.701596763907919	0.980508786945803\\
0.706639835866396	0.980247059082151\\
0.711682907824873	0.979983871588325\\
0.71672597978335	0.979719230170549\\
0.721769051741827	0.979453140528209\\
0.72683321952555	0.97918448621016\\
0.731897387309272	0.978914383023421\\
0.736961555092995	0.978642836718223\\
0.742025722876718	0.978369853037617\\
0.747111194287107	0.978094280308244\\
0.752196665697496	0.977817269670052\\
0.757282137107885	0.977538826916873\\
0.762367608518274	0.97725895783502\\
0.767474594900923	0.976976475149014\\
0.772581581283573	0.976692565712595\\
0.777688567666222	0.976407235362793\\
0.782795554048871	0.976120489928775\\
0.78792427039055	0.975831106146065\\
0.793052986732229	0.975540306970055\\
0.798181703073909	0.975248098280565\\
0.803310419415588	0.974954485949213\\
0.808461084325784	0.974658210335229\\
0.81361174923598	0.974360530884022\\
0.818762414146177	0.974061453517796\\
0.823913079056373	0.973760984150209\\
0.82908591482452	0.973457826372455\\
0.834258750592667	0.973153276512821\\
0.839431586360813	0.972847340535491\\
0.84460442212896	0.97254002439575\\
0.849799655005291	0.972229994507269\\
0.854994887881621	0.971918584491753\\
0.860190120757951	0.971605800354951\\
0.865385353634282	0.971291648093366\\
0.870603213699137	0.970974756540127\\
0.875821073763992	0.970656497014579\\
0.881038933828848	0.970336875563628\\
0.886256793893703	0.970015898224581\\
0.891497515445601	0.969692155822256\\
0.896738236997499	0.969367057802497\\
0.901978958549397	0.969040610252953\\
0.907219680101295	0.968712819251319\\
0.912483501408722	0.968382237200394\\
0.917747322716149	0.968050312087539\\
0.923011144023576	0.967717050040728\\
0.928274965331004	0.967382457177632\\
0.933562128954513	0.967045047044302\\
0.938849292578023	0.966706306605571\\
0.944136456201532	0.966366242029331\\
0.949423619825042	0.966024859472805\\
0.954734372782572	0.96568063317881\\
0.960045125740102	0.965335089537387\\
0.965355878697632	0.964988234755921\\
0.970666631655162	0.964640075030774\\
0.976001225418392	0.96428904485417\\
0.981335819181621	0.963936710490057\\
0.98667041294485	0.963583078184901\\
0.99200500670808	0.963228154173774\\
0.997363697280812	0.962870332745019\\
1.00272238785355	0.962511220491188\\
1.00808107842628	0.962150823697403\\
1.01343976899901	0.96178914863703\\
1.0188228172289	0.961424548919579\\
1.02420586545878	0.961058671942512\\
1.02958891368867	0.960691524029185\\
1.03497196191855	0.960323111490831\\
1.04037963341625	0.959951746787781\\
1.04578730491395	0.959579118594253\\
1.05119497641164	0.959205233271417\\
1.05660264790934	0.958830097167945\\
1.06203521338424	0.958451981100946\\
1.06746777885913	0.95807261551689\\
1.07290034433402	0.957692006814333\\
1.07833290980891	0.957310161378961\\
1.08379064506731	0.956925307889973\\
1.08924838032571	0.956539219062344\\
1.09470611558411	0.956151901331588\\
1.10016385084252	0.955763361119974\\
1.1056470369148	0.955371784464729\\
1.11113022298707	0.954978986855016\\
1.11661340905935	0.95458497476288\\
1.12209659513163	0.954189754646739\\
1.12760551858231	0.95379146937518\\
1.13311444203298	0.953391977739823\\
1.13862336548365	0.952991286248815\\
1.14413228893432	0.952589401396289\\
1.14966724184116	0.952184422356583\\
1.15520219474799	0.951778251751117\\
1.16073714765482	0.951370896123703\\
1.16627210056166	0.950962362003758\\
1.17183338076584	0.950550704327313\\
1.17739466097002	0.950137870091392\\
1.18295594117419	0.949723865875039\\
1.18851722137837	0.949308698242516\\
1.19410513262804	0.948890377336715\\
1.19969304387772	0.948470895086913\\
1.20528095512739	0.948050258106952\\
1.21086886637706	0.947628472995497\\
1.21648371859449	0.947203504527386\\
1.22209857081192	0.946777390140905\\
1.22771342302936	0.946350136484254\\
1.23332827524679	0.945921750190061\\
1.23897038457538	0.945490150086665\\
1.24461249390397	0.945057419701764\\
1.25025460323256	0.944623565717474\\
1.25589671256115	0.944188594799944\\
1.26156640151281	0.9437503792417\\
1.26723609046446	0.943311049251203\\
1.27290577941612	0.942870611544044\\
1.27857546836778	0.942429072819445\\
1.28427306633634	0.941984258205447\\
1.28997066430491	0.941538345221927\\
1.29566826227347	0.94109134061751\\
1.30136586024204	0.940643251124043\\
1.30709170327692	0.940191854094473\\
1.31281754631181	0.939739374972924\\
1.31854338934669	0.939285820540605\\
1.32426923238158	0.93883119756154\\
1.33002366371043	0.938373234962676\\
1.33577809503928	0.937914206765467\\
1.34153252636813	0.937454119783256\\
1.34728695769699	0.936992980811791\\
1.35307032782915	0.936528469693621\\
1.35885369796131	0.936062909688226\\
1.36463706809348	0.935596307640635\\
1.37042043822564	0.935128670377863\\
1.37622881553645	0.934657975984614\\
1.38203719284726	0.934186251151719\\
1.38784557015807	0.933713502740249\\
1.39365394746888	0.933239737592894\\
1.39948194569202	0.93276335702252\\
1.40530994391516	0.932285966581079\\
1.41113794213829	0.931807573125029\\
1.41696594036143	0.931328183492154\\
1.4228137248248	0.930846171922121\\
1.42866150928816	0.930363171135042\\
1.43450929375153	0.929879187981879\\
1.44035707821489	0.929394229294632\\
1.44622478964653	0.928906644386906\\
1.45209250107816	0.92841809100854\\
1.45796021250979	0.927928576004029\\
1.46382792394143	0.927438106198619\\
1.46971570462452	0.926945005989892\\
1.4756034853076	0.926450958146904\\
1.48149126599069	0.925955969506718\\
1.48737904667378	0.92546004688686\\
1.49328704060906	0.924961489768452\\
1.49919503454433	0.924462005939699\\
1.50510302847961	0.923961602229263\\
1.51101102241489	0.923460285445989\\
1.51693937480906	0.922956330200665\\
1.52286772720322	0.922451469254184\\
1.52879607959739	0.921945709425849\\
1.53472443199155	0.921439057514861\\
1.540673290528	0.920929763202963\\
1.54662214906446	0.920419584281743\\
1.55257100760091	0.919908527560181\\
1.55851986613736	0.919396599826883\\
1.56448937947106	0.918882025907562\\
1.57045889280476	0.918366588551248\\
1.57642840613847	0.917850294555645\\
1.58239791947216	0.917333150697807\\
1.58838823791446	0.916813356965573\\
1.59437855635675	0.91629272104683\\
1.60036887479904	0.915771249727047\\
1.60635919324133	0.915248949770777\\
1.61237046959671	0.914723996276506\\
1.61838174595208	0.914198221921806\\
1.62439302230746	0.913671633478968\\
1.63040429866284	0.913144237699095\\
1.63643668672632	0.912614184874213\\
1.6424690747898	0.912083332588473\\
1.64850146285328	0.911551687600039\\
1.65453385091676	0.911019256645618\\
1.66058750665404	0.910484165194027\\
1.66664116239132	0.909948295752223\\
1.67269481812859	0.909411655063296\\
1.67874847386587	0.908874249848618\\
1.68482355500326	0.908334180776512\\
1.69089863614064	0.907793355253628\\
1.69697371727803	0.90725178000704\\
1.70304879841541	0.906709461741847\\
1.70914546464501	0.906164476333955\\
1.71524213087462	0.905618756081197\\
1.72133879710422	0.905072307693695\\
1.72743546333383	0.904525137859337\\
1.73355387572031	0.903975297726551\\
1.73967228810678	0.90342474441914\\
1.74579070049326	0.902873484629332\\
1.75190911287974	0.902321525026874\\
1.75804943463327	0.901766892032226\\
1.76418975638681	0.901211567595197\\
1.77033007814035	0.900655558389191\\
1.77647039989388	0.900098871064878\\
1.78263279664118	0.899539507293116\\
1.78879519338848	0.898979473870868\\
1.79495759013578	0.898418777451786\\
1.80111998688309	0.89785742466654\\
1.80730462566597	0.897293392509355\\
1.81348926444885	0.896728712551129\\
1.81967390323173	0.896163391424831\\
1.82585854201462	0.895597435740203\\
1.83206559217232	0.895028797811552\\
1.83827264233002	0.894459533986568\\
1.84447969248772	0.893889650876613\\
1.85068674264543	0.893319155069582\\
1.85691637550678	0.892745974228522\\
1.86314600836813	0.89217218944891\\
1.86937564122948	0.891597807319578\\
1.87560527409083	0.891022834405654\\
1.88185766319929	0.890445173730876\\
1.88811005230775	0.889866931126182\\
1.89436244141622	0.889288113156957\\
1.90061483052468	0.888708726364643\\
1.90689015093446	0.888126649214884\\
1.91316547134424	0.887544012192636\\
1.91944079175402	0.886960821838916\\
1.92571611216381	0.886377084670567\\
1.93201454150218	0.885790654582187\\
1.93831297084055	0.885203686725287\\
1.94461140017892	0.884616187615602\\
1.9509098295173	0.884028163744463\\
1.95723154808653	0.883437444416723\\
1.96355326665576	0.882846209468744\\
1.96987498522499	0.882254465390071\\
1.97619670379422	0.881662218645617\\
1.98254189319299	0.881067274064919\\
1.98888708259176	0.8804718360546\\
1.99523227199053	0.8798759110771\\
2.0015774613893	0.879279505570005\\
2.00794630624256	0.878680399844413\\
2.01431515109582	0.878080822919905\\
2.02068399594908	0.877480781230912\\
2.02705284080234	0.87688028118679\\
2.03344552746868	0.876277078662835\\
2.03983821413503	0.875673427208739\\
2.04623090080137	0.875069333230017\\
2.05262358746771	0.874464803106895\\
2.05904030495676	0.873857568279646\\
2.06545702244581	0.873249906826959\\
2.07187373993486	0.872641825124534\\
2.07829045742391	0.872033329522565\\
2.084731396657	0.871422127101592\\
2.0911723358901	0.870810520393796\\
2.0976132751232	0.870198515744161\\
2.1040542143563	0.869586119471951\\
2.11051956957009	0.868971014244692\\
2.11698492478388	0.86835552710096\\
2.12345027999767	0.867739664354123\\
2.12991563521146	0.867123432291626\\
2.13640560238783	0.866504489268589\\
2.1428955695642	0.865885186729205\\
2.14938553674057	0.86526553095433\\
2.15587550391694	0.864645528198697\\
2.16239028199053	0.86402281249549\\
2.16890506006412	0.86339975970375\\
2.17541983813771	0.862776376070934\\
2.1819346162113	0.862152667818169\\
2.18847440630977	0.861526244724592\\
2.19501419640825	0.86089950699601\\
2.20155398650673	0.860272460845585\\
2.20809377660521	0.859645112459953\\
2.21465878275268	0.859015047369772\\
2.22122378890015	0.858384690121788\\
2.22778879504762	0.857754046893985\\
2.23435380119508	0.857123123837621\\
2.24094422983074	0.856489482291653\\
2.24753465846639	0.855855571086797\\
2.25412508710205	0.855221396364966\\
2.2607155157377	0.854586964241162\\
2.2673315759421	0.853949811902911\\
2.27394763614649	0.853312412424424\\
2.28056369635089	0.852674771910661\\
2.28717975655528	0.852036896439482\\
2.29382166035897	0.851396299062769\\
2.30046356416266	0.850755477082234\\
2.30710546796636	0.850114436565001\\
2.31374737177005	0.849473183550909\\
2.32041533418655	0.848829206974073\\
2.32708329660306	0.848185028345636\\
2.33375125901957	0.847540653694008\\
2.34041922143607	0.84689608902013\\
2.34711345989709	0.846248799218243\\
2.3538076983581	0.845601329930884\\
2.36050193681912	0.844953687146869\\
2.36719617528013	0.844305876827369\\
2.37391691043736	0.843655339832711\\
2.3806376455946	0.84300464593069\\
2.38735838075183	0.842353801069648\\
2.39407911590907	0.841702811170111\\
2.40082657131283	0.841049093101179\\
2.40757402671659	0.840395240713073\\
2.41432148212035	0.839741259912795\\
2.42106893752412	0.839087156579352\\
2.42784333963791	0.838430323637663\\
2.4346177417517	0.837773378973203\\
2.4413921438655	0.837116328450752\\
2.44816654597929	0.836459177906931\\
2.4549681243632	0.835799296357634\\
2.4617697027471	0.83513932568831\\
2.46857128113101	0.834479271720648\\
2.47537285951492	0.833819140248015\\
2.48220184698832	0.833156276402617\\
2.48903083446173	0.832493346044427\\
2.49585982193513	0.831830354951174\\
2.50268880940853	0.831167308872106\\
2.50954544177924	0.830501529113467\\
2.51640207414995	0.829835705451935\\
2.52325870652066	0.829169843620411\\
2.53011533889137	0.828503949323157\\
2.53699985562036	0.827835320039864\\
2.54388437234935	0.827166669464419\\
2.55076888907833	0.826498003284031\\
2.55765340580732	0.825829327157112\\
2.5645660494942	0.825157914792508\\
2.57147869318108	0.824486503745546\\
2.57839133686796	0.823815099656875\\
2.58530398055484	0.823143708138199\\
2.59224499685821	0.822469579197431\\
2.59918601316159	0.821795474181367\\
2.60612702946496	0.821121398683236\\
2.61306804576833	0.820447358267167\\
2.62003768421203	0.819770579238348\\
2.62700732265572	0.819093846736802\\
2.63397696109942	0.81841716630747\\
2.64094659954311	0.817740543466054\\
2.64794511298781	0.817061180871264\\
2.65494362643251	0.816381887400072\\
2.6619421398772	0.815702668548274\\
2.6689406533219	0.815023529782285\\
2.67596829852112	0.814341650123481\\
2.68299594372033	0.813659862176636\\
2.69002358891955	0.812978171387543\\
2.69705123411877	0.812296583172472\\
2.70410827073433	0.811612253027123\\
2.71116530734989	0.810928037172401\\
2.71822234396545	0.810243941003232\\
2.72527938058101	0.809559969884889\\
2.73236607296869	0.808873255733022\\
2.73945276535637	0.808186678439089\\
2.74653945774405	0.807500243346293\\
2.75362615013173	0.806813955768054\\
2.76074276611797	0.806124924111534\\
2.76785938210422	0.805436051867213\\
2.77497599809046	0.804747344325712\\
2.78209261407671	0.804058806747749\\
2.78923942490227	0.803367524116362\\
2.79638623572783	0.802676423436714\\
2.80353304655339	0.801985509945993\\
2.81067985737896	0.801294788851356\\
2.81785713902212	0.800601321675691\\
2.82503442066529	0.799908058974983\\
2.83221170230846	0.799215005932127\\
2.83938898395163	0.798522167699872\\
2.84659701614953	0.79782658240737\\
2.85380504834743	0.797131224095086\\
2.86101308054533	0.79643609789077\\
2.86822111274323	0.795741208891911\\
2.87546017926332	0.795043571881988\\
2.88269924578341	0.794346184337981\\
2.88993831230351	0.793649051331642\\
2.8971773788236	0.792952177904349\\
2.90444776780364	0.792252555517296\\
2.91171815678368	0.791553205060724\\
2.91898854576373	0.790854131549533\\
2.92625893474377	0.79015533996814\\
2.93356093845992	0.789453798509436\\
2.94086294217606	0.788752551423069\\
2.94816494589221	0.788051603666234\\
2.95546694960836	0.78735096016554\\
2.96280086487659	0.786647565869403\\
2.97013478014483	0.785944488363219\\
2.97746869541306	0.785241732545624\\
2.9848026106813	0.784539303284569\\
2.99216873909201	0.783834122294057\\
2.99953486750271	0.783129280485378\\
3.00690099591342	0.782424782697757\\
3.01426712432413	0.781720633739637\\
3.02166577186631	0.78101373214552\\
3.02906441940849	0.780307192097861\\
3.03646306695067	0.779601018375624\\
3.04386171449286	0.778895215726897\\
3.05129319192691	0.778186659534936\\
3.05872466936097	0.777478487225335\\
3.06615614679503	0.776770703515943\\
3.07358762422909	0.776063313093643\\
3.08105224696524	0.775353168239508\\
3.0885168697014	0.774643429573437\\
3.09598149243755	0.773934101751308\\
3.1034461151737	0.773225189397955\\
3.11094420406679	0.772513521675209\\
3.11844229295988	0.771802282414654\\
3.12594038185297	0.771091476209347\\
3.13343847074605	0.770381107621222\\
3.14097035080817	0.769667982807648\\
3.14850223087028	0.768955308697303\\
3.1560341109324	0.768243089819572\\
3.16356599099451	0.767531330672637\\
3.17113199342146	0.766816814342076\\
3.17869799584842	0.766102770921425\\
3.18626399827537	0.765389204875545\\
3.19383000070233	0.764676120638016\\
3.20143046110215	0.763960278332929\\
3.20903092150198	0.763244931108609\\
3.21663138190181	0.762530083364533\\
3.22423184230164	0.761815739468832\\
3.2318671021242	0.761098636569824\\
3.23950236194676	0.760382050885325\\
3.24713762176933	0.759665986748573\\
3.25477288159189	0.7589504484614\\
3.26244328804483	0.758232150200839\\
3.27011369449776	0.75751439125013\\
3.2777841009507	0.75679717587542\\
3.28545450740363	0.756080508311385\\
3.29316041266953	0.75536107985232\\
3.30086631793543	0.754642212758665\\
3.30857222320133	0.753923911228615\\
3.31627812846722	0.753206179428843\\
3.32401989110489	0.752485685739544\\
3.33176165374256	0.751765775430242\\
3.33950341638023	0.751046452630323\\
3.3472451790179	0.750327721437601\\
3.35502316313618	0.749606227375267\\
3.36280114725446	0.748885338665249\\
3.37057913137275	0.748165059367265\\
3.37835711549103	0.747445393509412\\
3.38617169154667	0.746722963752486\\
3.3939862676023	0.746001161276752\\
3.40180084365794	0.745279990071399\\
3.40961541971357	0.744559454093951\\
3.41746696420648	0.743836153177615\\
3.42531850869938	0.743113501426689\\
3.43317005319229	0.742391502758969\\
3.4410215976852	0.741670161060549\\
3.44891049346576	0.740946053355649\\
3.45679938924632	0.740222616654555\\
3.46468828502688	0.739499854802804\\
3.47257718080744	0.738777771614198\\
3.48050381756295	0.73805292129056\\
3.48843045431847	0.737328763762218\\
3.49635709107398	0.736605302801585\\
3.50428372782949	0.735882542149308\\
3.51224850159594	0.73515701322896\\
3.52021327536239	0.734432198847334\\
3.52817804912884	0.733708102702848\\
3.53614282289529	0.732984728462129\\
3.54414613607264	0.732258584825989\\
3.55214944924998	0.731533177422762\\
3.56015276242733	0.730808509876001\\
3.56815607560468	0.730084585777446\\
3.57619833889781	0.72935789099822\\
3.58424060219095	0.728631954096055\\
3.59228286548408	0.727906778618762\\
3.60032512877721	0.727182368082326\\
3.60840675922679	0.726455185613559\\
3.61648838967638	0.725728782614785\\
3.62457002012596	0.725003162557196\\
3.63265165057554	0.724278328880145\\
3.64077307248204	0.723550721981902\\
3.64889449438855	0.722823916094355\\
3.65701591629505	0.722097914611197\\
3.66513733820155	0.721372720894277\\
3.67329898353354	0.720644752607011\\
3.68146062886554	0.719917606817982\\
3.68962227419754	0.719191286842499\\
3.69778391952953	0.718465795964026\\
3.70598622860803	0.717737529059062\\
3.71418853768653	0.717010106085913\\
3.72239084676502	0.716283530280616\\
3.73059315584352	0.715557804847366\\
3.73883657626523	0.714829301934755\\
3.74707999668693	0.714101664332548\\
3.75532341710863	0.713374895196619\\
3.76356683753034	0.712648997651008\\
3.77185182516863	0.711920321102192\\
3.78013681280692	0.711192531186549\\
3.78842180044522	0.710465630978892\\
3.79670678808351	0.709739623522217\\
3.80503380745825	0.70901083545052\\
3.81336082683299	0.70828295527818\\
3.82168784620774	0.707555985998053\\
3.83001486558248	0.706829930571192\\
3.83838438958511	0.706101092869494\\
3.84675391358775	0.705373184275923\\
3.85512343759039	0.704646207700474\\
3.86349296159302	0.703920166021355\\
3.8719054722877	0.703191340305988\\
3.88031798298237	0.702463464849438\\
3.88873049367705	0.701736542477923\\
3.89714300437172	0.701010575985907\\
3.90559899284027	0.700281823625146\\
3.91405498130882	0.699554042615092\\
3.92251096977737	0.698827235697274\\
3.93096695824593	0.698101405581498\\
3.93946692479213	0.697372787692951\\
3.94796689133833	0.696645162187517\\
3.95646685788454	0.695918531721111\\
3.96496682443074	0.695192898917967\\
3.97351127881372	0.694464476364123\\
3.98205573319669	0.693737067165656\\
3.99060018757967	0.693010673891942\\
3.99914464196265	0.692285299080722\\
4.00773410425464	0.691557132412313\\
4.01632356654662	0.690830000010934\\
4.02491302883861	0.69010390435849\\
4.0335024911306	0.689378847905297\\
4.04213749081375	0.688650997455529\\
4.05077249049689	0.687924202123123\\
4.05940749018004	0.687198464301571\\
4.06804248986318	0.686473786352833\\
4.07672356764034	0.685746312101763\\
4.08540464541749	0.68501991375677\\
4.09408572319465	0.684294593621986\\
4.10276680097181	0.683570353970069\\
4.11149450824228	0.682843315608739\\
4.12022221551276	0.682117373880135\\
4.12894992278323	0.681392530998076\\
4.1376776300537	0.680668789144971\\
4.14645252883691	0.679942246100985\\
4.15522742762012	0.679216820353817\\
4.16400232640333	0.678492514026003\\
4.17277722518654	0.677769329208746\\
4.18159988926615	0.67704334057128\\
4.19042255334575	0.676318489831919\\
4.19924521742536	0.675594779020949\\
4.20806788150496	0.674872210137401\\
4.21693889629988	0.674146834689742\\
4.22580991109479	0.673422617678369\\
4.23468092588971	0.672699561040337\\
4.24355194068463	0.671977666681527\\
4.2524719033702	0.671252962912932\\
4.26139186605577	0.670529438055383\\
4.27031182874135	0.669807093951715\\
4.27923179142692	0.669085932413676\\
4.28820131218307	0.668361958440693\\
4.29717083293922	0.667639183790073\\
4.30614035369538	0.666917610209429\\
4.31510987445153	0.666197239415386\\
4.32412957614907	0.665474053032258\\
4.33314927784661	0.664752086319221\\
4.34216897954416	0.664031340927657\\
4.3511886812417	0.663311818478058\\
4.36025919971018	0.662589477159837\\
4.36932971817866	0.661868375795724\\
4.37840023664715	0.661148515939847\\
4.38747075511563	0.660429899115553\\
4.39659274002011	0.659708459973901\\
4.40571472492458	0.658988281006729\\
4.41483670982906	0.658269363669879\\
4.42395869473353	0.657551709388526\\
4.43313280988389	0.656831229174858\\
4.44230692503424	0.656112029292501\\
4.45148104018459	0.655394111097964\\
4.46065515533495	0.65467747591721\\
4.469882079136	0.653958011015836\\
4.47910900293706	0.653239846539235\\
4.48833592673811	0.652522983743521\\
4.49756285053916	0.651807423854397\\
4.50684327607396	0.651089030305455\\
4.51612370160877	0.650371957211509\\
4.52540412714357	0.649656205727214\\
4.53468455267837	0.64894177697694\\
4.54401918924822	0.648224510387655\\
4.55335382581807	0.647508584220866\\
4.56268846238792	0.646793999528674\\
4.57202309895777	0.646080757333042\\
4.58141267108196	0.645364672987697\\
4.59080224320616	0.644649948969775\\
4.60019181533036	0.643936586227725\\
4.60958138745456	0.643224585680008\\
4.61902663678989	0.642509738425537\\
4.62847188612523	0.641796271341481\\
4.63791713546056	0.641084185271521\\
4.6473623847959	0.640373481029502\\
4.65686406984621	0.639659925333397\\
4.66636575489653	0.638947769589242\\
4.67586743994685	0.638237014534813\\
4.68536912499717	0.637527660878213\\
4.69492802222378	0.636815450780957\\
4.7044869194504	0.636104660356449\\
4.71404581667701	0.635395290235407\\
4.72360471390363	0.634687341019049\\
4.73320764525141	0.633977561870872\\
4.74281057659919	0.633269217875967\\
4.75241350794698	0.632562309554615\\
4.76201643929476	0.631856837397945\\
4.77165794452804	0.63114997673488\\
4.78129944976131	0.630444564701839\\
4.79094095499459	0.629740601707849\\
4.80058246022786	0.629038088133207\\
4.81026314694842	0.628334178280249\\
4.81994383366897	0.627631730332798\\
4.82962452038952	0.62693074458904\\
4.83930520711008	0.626231221318854\\
4.84902565638322	0.625530296559525\\
4.85874610565637	0.624830846772469\\
4.86846655492951	0.624132872145415\\
4.87818700420266	0.623436372838212\\
4.88794780523678	0.622738466858768\\
4.89770860627089	0.622042048712426\\
4.90746940730501	0.621347118476831\\
4.91723020833913	0.620653676202173\\
4.92703195872998	0.619958822099275\\
4.93683370912084	0.619265468487111\\
4.9466354595117	0.618573615333589\\
4.95643720990256	0.617883262579598\\
4.96628051524823	0.617191492912449\\
4.97612382059389	0.616501236192955\\
4.98596712593956	0.615812492279625\\
4.99581043128522	0.615125261004382\\
5.00569590732134	0.614436607669413\\
5.01558138335745	0.613749479541328\\
5.02546685939356	0.613063876369559\\
5.03535233542967	0.612379797877382\\
5.04528060635073	0.611694292253228\\
5.05520887727179	0.61101032389988\\
5.06513714819285	0.61032789245799\\
5.07506541911391	0.609646997542488\\
5.08503711889176	0.608964670423268\\
5.0950088186696	0.608283892446121\\
5.10498051844744	0.607604663143204\\
5.11495221822528	0.606926982021391\\
5.12496799094167	0.60624786362098\\
5.13498376365806	0.60557030604389\\
5.14499953637444	0.60489430871405\\
5.15501530909083	0.604219871030544\\
5.16507580953502	0.60354399096966\\
5.17513630997921	0.602869683226017\\
5.18519681042339	0.602196947115565\\
5.19525731086758	0.601525781929851\\
5.2053632039849	0.600853169301093\\
5.21546909710223	0.600182140298557\\
5.22557499021955	0.599512694130444\\
5.23568088333688	0.598844829981\\
5.24583284576096	0.598175513273928\\
5.25598480818505	0.597507791319854\\
5.26613677060914	0.596841663219445\\
5.27628873303322	0.596177128049862\\
5.28648745229033	0.595511135232978\\
5.29668617154743	0.594846748116014\\
5.30688489080453	0.594183965692302\\
5.31708361006163	0.593522786932119\\
5.32732978639303	0.592860145362031\\
5.33757596272443	0.592199120261724\\
5.34782213905582	0.591539710517373\\
5.35806831538722	0.590881914992549\\
5.36836266133408	0.590222651472431\\
5.37865700728095	0.589565015017411\\
5.38895135322782	0.588909004406668\\
5.39924569917468	0.588254618397236\\
5.40958894010065	0.587598759173592\\
5.41993218102661	0.586944537438368\\
5.43027542195257	0.586291951863889\\
5.44061866287853	0.585641001100793\\
5.45101153706255	0.584988571888403\\
5.46140441124658	0.584337790418185\\
5.4717972854306	0.583688655255738\\
5.48219015961463	0.583041164945433\\
5.49263341945276	0.582392190883047\\
5.50307667929089	0.581744874649685\\
5.51351993912902	0.581099214704323\\
5.52396319896715	0.580455209485175\\
5.53445761187213	0.579809715113556\\
5.54495202477712	0.579165888493712\\
5.5554464376821	0.578523727978082\\
5.56594085058709	0.577883231898814\\
5.57648719831941	0.577241241236759\\
5.58703354605173	0.576600928087563\\
5.59757989378405	0.575962290697196\\
5.60812624151637	0.575325327291811\\
5.61872532103302	0.57468686381748\\
5.62932440054967	0.574050087457974\\
5.63992348006633	0.573414996352843\\
5.65052255958298	0.572781588622291\\
5.66117518401188	0.572146675249442\\
5.67182780844078	0.571513458436962\\
5.68248043286967	0.570881936218005\\
5.69313305729857	0.570252106606864\\
5.70384005671797	0.569620765675394\\
5.71454705613736	0.568991130596194\\
5.72525405555675	0.568363199296029\\
5.73596105497615	0.567736969683289\\
5.74672327665992	0.56710922298296\\
5.75748549834369	0.566483191275844\\
5.76824772002746	0.565858872382303\\
5.77900994171123	0.565236264104812\\
5.78982825023888	0.564612132906756\\
5.80064655876654	0.563989725694445\\
5.81146486729419	0.563369040181802\\
5.82228317582184	0.562750074065354\\
5.83315845473553	0.562129579064882\\
5.84403373364921	0.56151081689717\\
5.8549090125629	0.560893785169639\\
5.86578429147658	0.560278481472818\\
5.87671744355088	0.559661642815702\\
5.88765059562518	0.559046545695632\\
5.89858374769948	0.558433187613452\\
5.90951689977378	0.55782156605361\\
5.92050884813521	0.557208403315011\\
5.93150079649664	0.556596990677975\\
5.94249274485807	0.55598732553666\\
5.9534846932195	0.555379405269335\\
5.96453638108293	0.554769937512084\\
5.97558806894636	0.554162228283816\\
5.98663975680979	0.55355627487187\\
5.99769144467322	0.552952074548213\\
6.00880383710542	0.552346320268613\\
6.01991622953761	0.551742332811344\\
6.0310286219698	0.551140109356774\\
6.042141014402	0.550539647070414\\
6.05331509935885	0.549937624188667\\
6.0644891843157	0.549337376291671\\
6.07566326927256	0.54873890045264\\
6.08683735422941	0.548142193730456\\
6.09807414230094	0.547543919652519\\
6.10931093037248	0.546947428593664\\
6.12054771844401	0.546352717519744\\
6.13178450651555	0.545759783382808\\
6.14308503319571	0.545165274927378\\
6.15438555987588	0.544572557400604\\
6.16568608655605	0.543981627660739\\
6.17698661323621	0.543392482552768\\
6.18835193888863	0.542801756005152\\
6.19971726454106	0.542212828174083\\
6.21108259019348	0.541625695809954\\
6.2224479158459	0.541040355650427\\
6.23387912710053	0.540453426741531\\
6.24531033835517	0.539868304218671\\
6.2567415496098	0.539284984724079\\
6.26817276086443	0.538703464887806\\
6.2796709718506	0.538120348785283\\
6.29116918283678	0.537539046623281\\
6.30266739382295	0.536959554935547\\
6.31416560480912	0.536381870244199\\
6.32573195844141	0.535802581544103\\
6.33729831207371	0.535225114227549\\
6.34886466570601	0.534649464719442\\
6.36043101933831	0.534075629433613\\
6.37206668825559	0.533500182172472\\
6.38370235717287	0.532926563629986\\
6.39533802609016	0.532354770121821\\
6.40697369500744	0.531784797953136\\
6.41867988207414	0.531213205643259\\
6.43038606914084	0.53064344928266\\
6.44209225620754	0.530075525077334\\
6.45379844327423	0.529509429223351\\
6.46557638449348	0.528941704772243\\
6.47735432571273	0.528375823400569\\
6.48913226693197	0.52781178120419\\
6.50091020815122	0.527249574269626\\
6.51276117329228	0.526685730017113\\
6.52461213843333	0.526123735877627\\
6.53646310357439	0.52556358783639\\
6.54831406871544	0.525005281869872\\
6.56023936237899	0.524445329605161\\
6.57216465604253	0.523887234394405\\
6.58408994970607	0.52333099211164\\
6.59601524336962	0.522776598622751\\
6.608016207688	0.522220549527874\\
6.62001717200638	0.521666364339601\\
6.63201813632476	0.521114038820198\\
6.64401910064313	0.520563568724385\\
6.65609711612489	0.520011433410315\\
6.66817513160664	0.519461168771509\\
6.68025314708839	0.518912770457824\\
6.69233116257014	0.518366234112191\\
6.70448764976455	0.517818022619012\\
6.71664413695897	0.517271688490175\\
6.72880062415338	0.516727227262451\\
6.7409571113478	0.51618463446631\\
6.75319353351478	0.515640356222625\\
6.76542995568177	0.515097961957613\\
6.77766637784875	0.51455744709423\\
6.78990280001574	0.514018807049766\\
6.80222066442117	0.51347847089976\\
6.8145385288266	0.51294002527282\\
6.82685639323203	0.512403465477304\\
6.83917425763746	0.51186878681655\\
6.85157511856051	0.511332400975376\\
6.86397597948357	0.510797912137007\\
6.87637684040662	0.510265315494362\\
6.88877770132968	0.509734606236002\\
6.90126316148383	0.509202178321071\\
6.91374862163799	0.508671653829272\\
6.92623408179214	0.508143027837193\\
6.93871954194629	0.507616295417732\\
6.95129125486549	0.507087832441891\\
6.96386296778468	0.506561279255577\\
6.97643468070388	0.506036630818091\\
6.98900639362307	0.505513882085723\\
7.00166606764739	0.504989390390035\\
7.0143257416717	0.504466814802508\\
7.02698541569601	0.503946150164131\\
7.03964508972033	0.503427391313578\\
7.05239448925109	0.502906876622737\\
7.06514388878186	0.50238828431692\\
7.07789328831262	0.501871609117716\\
7.09064268784339	0.501356845745102\\
7.10348363709426	0.50084031312386\\
7.11632458634514	0.500325709129281\\
7.12916553559601	0.499813028362385\\
7.14200648484689	0.499302265423306\\
7.15494087138423	0.498789719251282\\
7.16787525792157	0.498279107919324\\
7.18080964445891	0.497770425906642\\
7.19374403099625	0.497263667692296\\
7.20677380774895	0.496755111707422\\
7.21980358450165	0.496248496754719\\
7.23283336125435	0.495743817190264\\
7.24586313800706	0.495241067370731\\
7.25899032889946	0.494736504576989\\
7.27211751979186	0.494233888994155\\
7.28524471068426	0.493733214853756\\
7.29837190157667	0.493234476388685\\
7.31159860422458	0.492733909090702\\
7.32482530687249	0.492235295176985\\
7.3380520095204	0.491738628753006\\
7.35127871216831	0.491243903926382\\
7.36460710304072	0.490747333677484\\
7.37793549391314	0.490252722989572\\
7.39126388478556	0.489760065840451\\
7.40459227565797	0.489269356210873\\
7.41802461355885	0.48877678383618\\
7.43145695145972	0.488286177211486\\
7.4448892893606	0.487797530185215\\
7.45832162726148	0.487310836609552\\
7.47186025959003	0.486822262136268\\
7.48539889191859	0.486335659624179\\
7.49893752424715	0.48585102279049\\
7.51247615657571	0.485368345357012\\
7.52612352432791	0.484883768005946\\
7.53977089208011	0.484401168859854\\
7.55341825983232	0.483920541502767\\
7.56706562758452	0.483441879524182\\
7.58082427117841	0.482961297684313\\
7.5945829147723	0.482482700336869\\
7.60834155836619	0.48200608093062\\
7.62210020196009	0.481531432920675\\
7.63597276754062	0.48105484412145\\
7.64984533312116	0.480580246157616\\
7.66371789870169	0.480107632340439\\
7.67759046428223	0.479636995988438\\
7.69157971106386	0.479164396852409\\
7.70556895784549	0.478693794963049\\
7.71955820462712	0.478225183491735\\
7.73354745140875	0.477758555618023\\
7.7476562591015	0.477289941829037\\
7.76176506679424	0.476823331780005\\
7.77587387448699	0.476358718499859\\
7.78998268217974	0.475896095026663\\
7.80421406003167	0.475431461263376\\
7.81844543788359	0.474968837830288\\
7.83267681573552	0.474508217611142\\
7.84690819358745	0.474049593499795\\
7.86126528820976	0.473588933430977\\
7.87562238283208	0.473130290395301\\
7.8899794774544	0.472673657128383\\
7.90433657207672	0.472219026376971\\
7.91882267896367	0.471762332558533\\
7.93330878585062	0.471307662606341\\
7.94779489273757	0.470855009104734\\
7.96228099962452	0.470404364650225\\
7.97689957365268	0.46995162848338\\
7.99151814768085	0.469500923164797\\
8.00613672170902	0.469052241124153\\
8.02075529573718	0.468605574804373\\
8.03550996276884	0.468156786485537\\
8.0502646298005	0.467710036166098\\
8.06501929683216	0.467265316117429\\
8.07977396386382	0.466822618625264\\
8.09466853533245	0.466377767035067\\
8.10956310680107	0.465934960786846\\
8.12445767826969	0.465494191989749\\
8.13935224973832	0.465055452768434\\
8.15439073744049	0.464614525391446\\
8.16942922514267	0.464175650914765\\
8.18446771284484	0.463738821281083\\
8.19950620054702	0.463304028449798\\
8.2146928332504	0.462867011278302\\
8.22987946595378	0.462432054807782\\
8.24506609865717	0.461999150809912\\
8.26025273136055	0.461568291074308\\
8.27559197362809	0.461135168490444\\
8.29093121589562	0.460704114679812\\
8.30627045816316	0.46027512123813\\
8.32160970043069	0.45984817978034\\
8.33710627308224	0.459418934439519\\
8.35260284573378	0.458991766247864\\
8.36809941838533	0.45856666661979\\
8.38359599103687	0.458143626990268\\
8.39925489573158	0.457718239643065\\
8.41491380042629	0.457294938160408\\
8.430572705121	0.456873713769596\\
8.44623160981571	0.456454557719881\\
8.46205815432056	0.456033007077208\\
8.4778846988254	0.455613551393195\\
8.49371124333024	0.455196181701716\\
8.50953778783509	0.454780889060047\\
8.52553761739547	0.454363151561012\\
8.54153744695585	0.453947518537578\\
8.55753727651623	0.453533980823309\\
8.57353710607661	0.453122529276688\\
8.58971623619016	0.452708578899271\\
8.60589536630372	0.452296742986063\\
8.62207449641727	0.451887012162801\\
8.63825362653082	0.451479377081726\\
8.65461848330403	0.451069185044272\\
8.67098334007723	0.450661117986428\\
8.68734819685043	0.450255166317864\\
8.70371305362364	0.449851320476425\\
8.72018188980809	0.449447024964239\\
8.73665072599255	0.449044842781063\\
8.75311956217701	0.448644764268001\\
8.76958839836146	0.448246779795449\\
8.78614348346302	0.447848811865672\\
8.80269856856458	0.447452940544302\\
8.81925365366614	0.447059156139383\\
8.83580873876771	0.446667448989224\\
8.85245111606455	0.446275760421256\\
8.86909349336139	0.445886151585091\\
8.88573587065824	0.445498612758089\\
8.90237824795508	0.445113134248834\\
8.91910883699713	0.444727679511298\\
8.93583942603918	0.444344287445102\\
8.95257001508123	0.443962948299509\\
8.96930060412328	0.443583652355954\\
8.98612034063892	0.443204385147681\\
9.00294007715455	0.442827163374531\\
9.01975981367019	0.442451977260227\\
9.03657955018582	0.442078817061588\\
9.05348938408815	0.441705690392878\\
9.07039921799048	0.441334591754763\\
9.08730905189281	0.440965511347948\\
9.10421888579513	0.440598439407148\\
9.12121978296026	0.440231405592967\\
9.13822068012539	0.439866382244258\\
9.15522157729052	0.439503359541193\\
9.17222247445565	0.43914232769886\\
9.18931541587169	0.438781338411493\\
9.20640835728774	0.438422341871214\\
9.22350129870378	0.438065328240123\\
9.24059424011982	0.437710287716121\\
9.25778022278737	0.437355293998283\\
9.27496620545493	0.437002275163275\\
9.29215218812249	0.436651221357553\\
9.30933817079004	0.43630212276424\\
9.32661820779546	0.435953075060328\\
9.34389824480088	0.435605984236351\\
9.3611782818063	0.435260840425513\\
9.37845831881172	0.434917633798541\\
9.39583344002363	0.434574481972756\\
9.41320856123555	0.434233268892598\\
9.43058368244747	0.43389398468038\\
9.44795880365938	0.433556619496778\\
9.46543005570071	0.433219312866005\\
9.48290130774205	0.432883926722184\\
9.50037255978338	0.432550451179065\\
9.51784381182471	0.432218876389587\\
9.53541225878365	0.431887363741214\\
9.55298070574258	0.431557753203783\\
9.57054915270152	0.431230034884777\\
9.58811759966045	0.43090419893168\\
9.60578432346384	0.4305784285484\\
9.62345104726722	0.430254541789655\\
9.64111777107061	0.429932528758923\\
9.658784494874	0.42961237960048\\
9.67655059520445	0.429292299293136\\
9.6943166955349	0.428974084020266\\
9.71208279586536	0.428657723883576\\
9.72984889619581	0.428343209026352\\
9.74771549159443	0.428028766144659\\
9.76558208699305	0.427716169610532\\
9.78344868239167	0.427405409526102\\
9.80131527779029	0.427096476035844\\
9.81928350543996	0.426787617503365\\
9.83725173308963	0.426480586541273\\
9.8552199607393	0.42617537325429\\
9.87318818838897	0.425871967790234\\
9.89125920482513	0.425568640122152\\
9.90933022126129	0.425267121163556\\
9.92740123769745	0.424967401023893\\
9.94547225413361	0.424669469856441\\
9.96364723590061	0.424371619178814\\
9.9818222176676	0.42407555827255\\
9.99999719943459	0.42378127725392\\
10.0181721812016	0.423488766283749\\
10.0364523247657	0.423196338365013\\
10.0547324683298	0.422905681208596\\
10.0730126118939	0.422616784939662\\
10.0912927554579	0.422329639728633\\
10.1096792783833	0.42204257999008\\
10.1280658013086	0.421757271940229\\
10.146452324234	0.421473705715172\\
10.1648388471593	0.421191871496949\\
10.1833329877765	0.420910125047628\\
10.2018271283938	0.42063011115492\\
10.220321269011	0.420351819967846\\
10.2388154096282	0.420075241682056\\
10.257418428294	0.419798753328155\\
10.2760214469598	0.419523978346462\\
10.2946244656257	0.419250906900897\\
10.3132274842915	0.418979529202675\\
10.3319406634311	0.418708243476107\\
10.3506538425707	0.418438651891003\\
10.3693670217104	0.418170744628118\\
10.38808020085	0.417904511916153\\
10.4069048454226	0.417638373096052\\
10.4257294899953	0.417373909146202\\
10.4445541345679	0.417111110266104\\
10.4633787791406	0.416849966703834\\
10.4823162175897	0.416588918831171\\
10.5012536560388	0.416329526522942\\
10.5201910944879	0.416071779999262\\
10.539128532937	0.415815669529443\\
10.5581801172831	0.415559656433912\\
10.5772317016292	0.415305279568075\\
10.5962832859753	0.415052529174506\\
10.6153348703214	0.414801395545576\\
10.6345019773376	0.414550360857284\\
10.6536690843538	0.414300943040726\\
10.67283619137	0.414053132362738\\
10.6920032983861	0.413806919140549\\
10.7112873293394	0.413560806321096\\
10.7305713602926	0.413316290997656\\
10.7498553912458	0.413073363463107\\
10.769139422199	0.412832014061296\\
10.7885418041401	0.412590766413866\\
10.8079441860811	0.41235109687446\\
10.8273465680222	0.412112995763741\\
10.8467489499633	0.411876453453904\\
10.8662711360792	0.411640014145958\\
10.8857933221952	0.411405133551131\\
10.9053155083112	0.411171802019583\\
10.9248376944271	0.410940009953557\\
10.9444811647161	0.410708322035114\\
10.9641246350051	0.410478173433237\\
10.9837681052941	0.410249554529263\\
11.0034115755831	0.410022455757151\\
11.02317783773	0.409795462176752\\
11.0429440998769	0.409569988519928\\
11.0627103620238	0.409346025200845\\
11.0824766241707	0.40912356268681\\
11.102367213876	0.408901206313887\\
11.1222578035813	0.408680350480175\\
11.1421483932866	0.40846098563429\\
11.1620389829919	0.40824310227849\\
11.182055464688	0.408025325920805\\
11.202071946384	0.40780903073158\\
11.22208842808	0.407594207195461\\
11.2421049097761	0.407380845851233\\
11.2622488777205	0.407167592268307\\
11.282392845665	0.406955800501642\\
11.3025368136094	0.406745461073474\\
11.3226807815539	0.406536564560655\\
11.3429428732039	0.406327889246372\\
11.363204964854	0.40612065486154\\
11.3834670565041	0.405914851982799\\
11.4037291481542	0.405710471241749\\
11.4240073959591	0.405507342047742\\
11.444285643764	0.405305618650616\\
11.464563891569	0.405105291824181\\
11.4848421393739	0.404906352396431\\
11.5051410232308	0.404708590901182\\
11.5254399070878	0.404512201401394\\
11.5457387909448	0.40431717485944\\
11.5660376748017	0.40412350229114\\
11.5863613245615	0.403930940931857\\
11.6066849743212	0.403739718989041\\
11.627008624081	0.403549827605878\\
11.6473322738407	0.403361257978285\\
11.6676846007299	0.403173738054894\\
11.6880369276191	0.402987526107524\\
11.7083892545083	0.402802613453035\\
11.7287415813975	0.402618991460345\\
11.7491263024508	0.402436362342851\\
11.7695110235041	0.402255010824245\\
11.7898957445574	0.402074928388481\\
11.8102804656107	0.401896106570923\\
11.8307011258767	0.401718224995468\\
11.8511217861426	0.401541591637467\\
11.8715424464086	0.401366198141881\\
11.8919631066745	0.401192036204458\\
11.9124230970895	0.401018765667072\\
11.9328830875046	0.400846714900237\\
11.9533430779196	0.40067587570428\\
11.9738030683347	0.400506239929714\\
11.9943056417768	0.400337450145656\\
12.014808215219	0.400169852564315\\
12.0353107886612	0.400003439136134\\
12.0558133621033	0.399838201861167\\
12.0763616495731	0.399673768267869\\
12.0969099370428	0.399510500138296\\
12.1174582245125	0.39934838956812\\
12.1380065119823	0.399187428702065\\
12.1586035338403	0.399027232042069\\
12.1792005556983	0.39886817488969\\
12.1997975775563	0.398710249481261\\
12.2203945994143	0.398553448101628\\
12.2410432774846	0.398397374031336\\
12.2616919555548	0.398242414253496\\
12.2823406336251	0.398088561140833\\
12.3029893116954	0.39793580711406\\
12.3236924799435	0.397783745854181\\
12.3443956481917	0.397632774374155\\
12.3650988164398	0.397482885179087\\
12.385801984688	0.397334070821567\\
12.406562398648	0.397185916843866\\
12.4273228126081	0.397038828800744\\
12.4480832265682	0.396892799325923\\
12.4688436405282	0.396747821100122\\
12.4896639845707	0.396603472853111\\
12.5104843286133	0.396460167330187\\
12.5313046726558	0.396317897290151\\
12.5521250166983	0.396176655538317\\
12.5730079121072	0.396036015187429\\
12.5938908075161	0.395896394954953\\
12.614773702925	0.395757787721424\\
12.6356565983338	0.395620186413427\\
12.6566046110283	0.395483159603081\\
12.6775526237227	0.395347130882659\\
12.6985006364172	0.395212093251277\\
12.7194486491116	0.39507803975365\\
12.7404642942679	0.394944535406822\\
12.7614799394242	0.394812007672877\\
12.7824955845805	0.394680449666531\\
12.8035112297368	0.394549854547653\\
12.8245969793381	0.394419784660785\\
12.8456827289394	0.394290670437487\\
12.8667684785407	0.394162505105243\\
12.887854228142	0.394035281936263\\
12.9090125143938	0.393908561414508\\
12.9301708006455	0.39378277611251\\
12.9513290868973	0.393657919367847\\
12.972487373149	0.393533984562398\\
12.9937205946389	0.393410531049265\\
13.0149538161287	0.393287992797007\\
13.0361870376186	0.39316636325074\\
13.0574202591085	0.393045635899478\\
13.0787307850171	0.39292536962997\\
13.1000413109258	0.392805999128154\\
13.1213518368345	0.39268751794427\\
13.1426623627431	0.392569919672046\\
13.1640525369264	0.392452763336247\\
13.1854427111097	0.392336483722699\\
13.206832885293	0.392221074484448\\
13.2282230594763	0.392106529317643\\
13.2496952043377	0.391992407934491\\
13.271167349199	0.391879144659067\\
13.2926394940603	0.391766733245029\\
13.3141116389217	0.391655167488746\\
13.3356680588802	0.391544008290167\\
13.3572244788388	0.391433688999945\\
13.3787808987974	0.391324203470247\\
13.400337318756	0.391215545595569\\
13.4219803036483	0.391107277918382\\
13.4436232885406	0.390999832350534\\
13.4652662734329	0.390893202840688\\
13.4869092583252	0.390787383379465\\
13.5086410871513	0.390681938562544\\
13.5303729159775	0.390577298442428\\
13.5521047448036	0.390473457062349\\
13.5738365736297	0.390370408507137\\
13.5956595174591	0.390267719799324\\
13.6174824612885	0.3901658187492\\
13.639305405118	0.390064699492728\\
13.6611283489474	0.389964356207105\\
13.6830446722324	0.389864358688032\\
13.7049609955174	0.389765132148537\\
13.7268773188025	0.389666670815543\\
13.7487936420875	0.389568968956854\\
13.770805608124	0.38947159944351\\
13.7928175741606	0.389374984581113\\
13.8148295401972	0.389279118685845\\
13.8368415062338	0.389183996114427\\
13.858951377984	0.389089193094711\\
13.8810612497342	0.388995128735779\\
13.9031711214845	0.388901797441444\\
13.9252809932347	0.388809193655711\\
13.9474910360997	0.388716897216936\\
13.9697010789646	0.388625323776858\\
13.9919111218296	0.38853446782535\\
14.0141211646945	0.388444323892139\\
14.0364336494706	0.388354475653187\\
14.0587461342466	0.388265335069122\\
14.0810586190226	0.388176896714357\\
14.1033711037986	0.388089155202834\\
14.1257883079129	0.38800169825727\\
14.1482055120272	0.387914933931689\\
14.1706227161414	0.387828856883594\\
14.1930399202557	0.387743461809686\\
14.2155641320733	0.387658340660041\\
14.2380883438909	0.387573897395632\\
14.2606125557085	0.38749012675564\\
14.283136767526	0.387407023518122\\
14.3057702854045	0.387324184034125\\
14.3284038032829	0.387242007992264\\
14.3510373211614	0.387160490212042\\
14.3736708390398	0.387079625551516\\
14.3964159769655	0.386999014905912\\
14.4191611148912	0.386919053543114\\
14.4419062528169	0.386839736361628\\
14.4646513907425	0.386761058298204\\
14.4875104777278	0.386682624929188\\
14.510369564713	0.386604826959755\\
14.5332286516983	0.386527659366147\\
14.5560877386835	0.386451117162541\\
14.5790631229979	0.386374810720994\\
14.6020385073122	0.386299126064745\\
14.6250138916265	0.38622405824654\\
14.6479892759408	0.386149602356754\\
14.6710833264109	0.386075373666774\\
14.6941773768809	0.386001753409859\\
14.7172714273509	0.385928736714066\\
14.740365477821	0.38585631874478\\
14.7635805844029	0.385784119768262\\
14.7867956909849	0.385712516127983\\
14.8100107975669	0.385641503026156\\
14.8332259041488	0.385571075702021\\
14.8565644820063	0.385500859494004\\
14.8799030598637	0.385431225774567\\
14.9032416377212	0.385362169818957\\
14.9265802155787	0.385293686939152\\
14.9500447068131	0.385225407613259\\
14.9735091980476	0.385157698171471\\
14.996973689282	0.385090553960978\\
15.0204381805165	0.38502397036541\\
15.0440310551214	0.384957583063165\\
15.0676239297264	0.384891753277956\\
15.0912168043314	0.384826476427858\\
15.1148096789363	0.384761747967096\\
15.1385334385982	0.384697208820701\\
15.16225719826	0.384633215056233\\
15.1859809579218	0.384569762161617\\
15.2097047175837	0.384506845660646\\
15.2335618956234	0.384444111768531\\
15.2574190736632	0.384381911349869\\
15.281276251703	0.384320239961436\\
15.3051334297427	0.38425909319559\\
15.3291265950906	0.384198122588184\\
15.3531197604384	0.384137673767376\\
15.3771129257863	0.384077742357816\\
15.4011060911341	0.384018324019456\\
15.4252378496272	0.383959075634014\\
15.4493696081203	0.383900337565082\\
15.4735013666134	0.383842105504232\\
15.4976331251065	0.383784375178059\\
15.5219061221034	0.383726808831365\\
15.5461791191004	0.383669741543218\\
15.5704521160973	0.383613169071181\\
15.5947251130942	0.383557087207567\\
15.619142035055	0.383501163572563\\
15.6435589570157	0.383445727945778\\
15.6679758789765	0.383390776149863\\
15.6923928009372	0.383336304041942\\
15.7169563780926	0.383281984622933\\
15.741519955248	0.383228142365152\\
15.7660835324033	0.38317477315545\\
15.7906471095587	0.383121872914882\\
15.8153601179615	0.383069120025469\\
15.8400731263644	0.383016833649481\\
15.8647861347672	0.382965009737103\\
15.88949914317	0.382913644272458\\
15.9143644081624	0.382862421011609\\
15.9392296731548	0.382811653811613\\
15.9640949381471	0.382761338685142\\
15.9889602031395	0.382711471678544\\
16.0139806003288	0.382661741913171\\
16.039000997518	0.382612457947441\\
16.0640213947072	0.382563615855687\\
16.0890417918965	0.382515211745656\\
16.1142202516767	0.382466940086837\\
16.1393987114569	0.382419104154153\\
16.1645771712372	0.382371700082789\\
16.1897556310174	0.382324724041082\\
16.2150951374624	0.382277875832482\\
16.2404346439074	0.382231453460553\\
16.2657741503524	0.382185453120536\\
16.2911136567974	0.382139871040567\\
16.3166172552097	0.382094412330957\\
16.342120853622	0.382049369749228\\
16.3676244520343	0.382004739549896\\
16.3931280504465	0.381960518020121\\
16.4187988465862	0.381916415552989\\
16.4444696427259	0.38187271968224\\
16.4701404388656	0.381829426720908\\
16.4958112350053	0.381786533014411\\
16.5216524004798	0.381743754206902\\
16.5474935659542	0.381701372638389\\
16.5733347314287	0.381659384679669\\
16.5991758969032	0.381617786733677\\
16.6251906694304	0.381576299665466\\
16.6512054419577	0.381535200649803\\
16.677220214485	0.381494486114517\\
16.7032349870123	0.381454152519321\\
16.7294266758655	0.381413925913383\\
16.7556183647187	0.38137407834149\\
16.781810053572	0.381334606287777\\
16.8080017424252	0.381295506268017\\
16.8343737319713	0.381256509475827\\
16.8607457215174	0.381217882864222\\
16.8871177110636	0.381179622972935\\
16.9134897006097	0.38114172637309\\
16.9400454511807	0.381103929364433\\
16.9666012017517	0.381066493845065\\
16.9931569523226	0.381029416409615\\
17.0197127028936	0.380992693683862\\
17.0464557582069	0.380956067027623\\
17.0731988135202	0.380919793328849\\
17.0999418688336	0.380883869236377\\
17.1266849241469	0.380848291429955\\
17.1536189111414	0.380812806286811\\
17.1805528981359	0.380777665726061\\
17.2074868851304	0.380742866450076\\
17.2344208721249	0.380708405191895\\
17.2615495082445	0.380674033297087\\
17.2886781443641	0.380639997763805\\
17.3158067804836	0.380606295347286\\
17.3429354166032	0.380572922833198\\
17.3702625121198	0.380539636487022\\
17.3975896076364	0.380506678433184\\
17.424916703153	0.380474045479126\\
17.4522437986696	0.380441734462489\\
17.4797732620512	0.380409506517375\\
17.5073027254328	0.380377598944653\\
17.5348321888143	0.380346008603327\\
17.5623616521959	0.380314732382362\\
17.5900974943972	0.380283536232092\\
17.6178333365984	0.380252652681141\\
17.6455691787997	0.380222078639433\\
17.6733050210009	0.380191811046618\\
17.7012513604556	0.380161620615354\\
17.7291976999103	0.380131735154902\\
17.757144039365	0.38010215162547\\
17.7850903788197	0.380072867016768\\
17.8132514459017	0.380043656749301\\
17.8414125129837	0.380014743966441\\
17.8695735800657	0.379986125678059\\
17.8977346471477	0.3799577989233\\
17.9261147917276	0.379929543772692\\
17.9544949363076	0.379901578760632\\
17.9828750808876	0.379873900946037\\
18.0112552254676	0.379846507416868\\
18.0398589202161	0.379819182837289\\
18.0684626149646	0.379792141188189\\
18.097066309713	0.379765379576918\\
18.1256700044615	0.379738895139652\\
18.1545018545664	0.379712477073387\\
18.1833337046712	0.379686334865492\\
18.212165554776	0.379660465671148\\
18.2409974048809	0.379634866674139\\
18.2700621491713	0.379609331547564\\
18.2991268934616	0.379584065341124\\
18.328191637752	0.379559065257233\\
18.3572563820424	0.37953432852669\\
18.3865589068877	0.379509653234945\\
18.4158614317331	0.379485240057071\\
18.4451639565785	0.379461086242124\\
18.4744664814239	0.379437189067325\\
18.5040118216459	0.379413350971557\\
18.5335571618679	0.37938976831339\\
18.56310250209	0.379366438387935\\
18.592647842312	0.379343358518251\\
18.622441195706	0.379320335431638\\
18.6522345491	0.379297561234536\\
18.682027902494	0.379275033267526\\
18.711821255888	0.379252748898925\\
18.74186798467	0.379230519084302\\
18.771914713452	0.379208531737384\\
18.801961442234	0.379186784243648\\
18.8320081710161	0.379165274016091\\
18.8623138179204	0.379143816172506\\
18.8926194648247	0.379122594499374\\
18.922925111729	0.37910160642649\\
18.9532307586333	0.379080849410965\\
18.9838010512769	0.379060142670135\\
19.0143713439205	0.379039665925278\\
19.0449416365641	0.379019416649939\\
19.0755119292076	0.37899939234477\\
19.1063527942164	0.378979416260909\\
19.1371936592252	0.378959664119629\\
19.1680345242339	0.378940133437658\\
19.1988753892427	0.378920821758626\\
19.2299929618915	0.378901556302364\\
19.2611105345403	0.378882508854708\\
19.2922281071891	0.378863676975006\\
19.3233456798379	0.378845058249306\\
19.3547463174076	0.378826483799826\\
19.3861469549774	0.378808121542781\\
19.4175475925471	0.378789969079574\\
19.4489482301169	0.378772024038111\\
19.4806385226382	0.378754121377667\\
19.5123288151595	0.378736425209672\\
19.5440191076808	0.378718933177031\\
19.5757094002021	0.37870164294895\\
19.6076961882754	0.378684393253936\\
19.6396829763487	0.378667344466048\\
19.671669764422	0.378650494269132\\
19.7036565524953	0.378633840373141\\
19.7359469398787	0.378617225209164\\
19.768237327262	0.378600805480121\\
19.8005277146453	0.378584578910242\\
19.8328181020287	0.378568543249672\\
19.8654194755566	0.378552544563404\\
19.8980208490846	0.378536735951552\\
19.9306222226126	0.378521115178174\\
19.9632235961405	0.378505680033053\\
19.9961436388709	0.378490280148502\\
20.0290636816012	0.37847506508804\\
20.0619837243316	0.378460032654996\\
20.0949037670619	0.378445180678243\\
20.1281504830269	0.378430362288047\\
20.1613971989918	0.378415723580414\\
20.1946439149567	0.37840126239739\\
20.2278906309217	0.37838697660638\\
20.2614723645969	0.378372722766883\\
20.2950540982721	0.37835864357584\\
20.3286358319473	0.378344736913459\\
20.3622175656225	0.378331000685126\\
20.3961430262838	0.378317294810136\\
20.4300684869452	0.378303758655571\\
20.4639939476066	0.37829039013924\\
20.497919408268	0.378277187203955\\
20.5321976950164	0.378264013059321\\
20.5664759817649	0.378251003811844\\
20.6007542685134	0.378238157416373\\
20.6350325552618	0.37822547185259\\
20.6696731863085	0.378212813550154\\
20.7043138173553	0.378200315425092\\
20.738954448402	0.378187975468734\\
20.7735950794487	0.378175791697069\\
20.8086080231198	0.378163633689213\\
20.8436209667909	0.378151631241191\\
20.878633910462	0.378139782380249\\
20.9136468541332	0.378128085158122\\
20.949042565344	0.378116412231642\\
20.9844382765548	0.378104890348511\\
21.0198339877656	0.378093517571314\\
21.0552296989764	0.37808229198697\\
21.0910191546516	0.378071089258705\\
21.1268086103268	0.378060033157173\\
21.162598066002	0.378049121779725\\
21.1983875216773	0.37803835324789\\
21.234582265905	0.378027606158495\\
21.2707770101326	0.378017001377966\\
21.3069717543603	0.378006537037838\\
21.343166498588	0.377996211293672\\
21.3797786897193	0.377985905602519\\
21.4163908808506	0.377975738000028\\
21.453003071982	0.37796570665133\\
21.4896152631133	0.377955809745432\\
21.5266577262528	0.377945931525565\\
21.5637001893923	0.377936187270774\\
21.6007426525319	0.377926575179184\\
21.6377851156714	0.377917093472659\\
21.6752714017348	0.377907629105709\\
21.7127576877983	0.377898294675858\\
21.7502439738617	0.37788908841362\\
21.7877302599251	0.377880008573109\\
21.8256747161126	0.377870944743255\\
21.8636191723001	0.377862006917201\\
21.9015636284876	0.377853193357229\\
21.9395080846751	0.377844502349095\\
21.9779259262067	0.377835826038977\\
22.0163437677384	0.377827271893134\\
22.0547616092701	0.377818838204982\\
22.0931794508017	0.377810523291291\\
22.1320868496398	0.377802221776166\\
22.1709942484778	0.377794038678735\\
22.2099016473158	0.377785972322898\\
22.2488090461539	0.3777780210558\\
22.2882232273236	0.377770081898746\\
22.3276374084934	0.377762257504978\\
22.3670515896632	0.377754546228216\\
22.4064657708329	0.377746946445318\\
22.4464051256478	0.377739357491759\\
22.4863444804626	0.377731879738562\\
22.5262838352775	0.377724511568581\\
22.5662231900924	0.377717251387705\\
22.6067074061995	0.377710000759939\\
22.6471916223066	0.377702857860499\\
22.6876758384138	0.377695821100657\\
22.7281600545209	0.377688888914636\\
22.7692102626757	0.377681965006885\\
22.8102604708306	0.377675145445799\\
22.8513106789854	0.377668428670326\\
22.8923608871403	0.377661813142295\\
22.9339998357608	0.377655204614902\\
22.9756387843813	0.377648697142495\\
23.0172777330018	0.377642289190928\\
23.0589166816223	0.377635979248874\\
23.1011689388352	0.377629675022646\\
23.143421196048	0.37762346864945\\
23.1856734532609	0.377617358621234\\
23.2279257104737	0.377611343452714\\
23.2707449436263	0.377605342846222\\
23.3135641767789	0.377599436660927\\
23.3563834099314	0.377593623421507\\
23.399202643084	0.377587901675226\\
23.4425351018712	0.37758220303143\\
23.4858675606585	0.377576595162821\\
23.5292000194457	0.377571076633424\\
23.5725324782329	0.377565646029514\\
23.6163923812599	0.377560237438434\\
23.6602522842869	0.377554916080051\\
23.704112187314	0.37754968055703\\
23.747972090341	0.377544529493962\\
23.7923730104802	0.377539399536455\\
23.8367739306194	0.377534353365981\\
23.8811748507586	0.377529389623267\\
23.9255757708978	0.377524506970643\\
23.9705318120247	0.377519644543263\\
24.0154878531517	0.377514862552586\\
24.0604438942787	0.377510159676826\\
24.1053999354057	0.377505534615469\\
24.1509257391711	0.377500928927648\\
24.1964515429365	0.377496400419975\\
24.2419773467019	0.377491947807571\\
24.2875031504673	0.377487569826512\\
24.3336139249026	0.377483210395044\\
24.3797246993379	0.377478924979398\\
24.4258354737732	0.377474712331034\\
24.4719462482085	0.377470571222041\\
24.518657797687	0.377466447865701\\
24.5653693471654	0.377462395451558\\
24.6120808966439	0.377458412766842\\
24.6587924461223	0.377454498619089\\
24.7061212033434	0.377450601453319\\
24.7534499605645	0.377446772245307\\
24.8007787177855	0.377443009817488\\
24.8481074750066	0.377439313012276\\
24.8960705351928	0.377435632443932\\
24.944033595379	0.377432016936593\\
24.9919966555653	0.377428465347333\\
25.0399597157515	0.377424976552887\\
25.0885748730688	0.37742150327506\\
25.1371900303861	0.377418092247683\\
25.1858051877034	0.377414742361909\\
25.2344203450208	0.37741145252823\\
25.2837061315614	0.377408177515127\\
25.332991918102	0.37740496202664\\
25.3822777046427	0.377401804987443\\
25.4315634911833	0.377398705341229\\
25.4815392186928	0.377395619843105\\
25.5315149462023	0.377392591227024\\
25.5814906737117	0.377389618450622\\
25.6314664012212	0.377386700490238\\
25.6821522057685	0.377383796028396\\
25.7328380103157	0.377380945887828\\
25.783523814863	0.377378149058583\\
25.8342096194102	0.37737540454909\\
25.8856265092156	0.377372672910929\\
25.9370433990211	0.377369993113638\\
25.9884602888265	0.377367364179127\\
26.0398771786319	0.377364785147368\\
26.0920470854784	0.377362218381477\\
26.144216992325	0.377359701054991\\
26.1963868991715	0.377357232221127\\
26.2485568060181	0.377354810950852\\
26.3015026403881	0.377352401362169\\
26.3544484747581	0.377350038888937\\
26.4073943091281	0.377347722615141\\
26.4603401434981	0.377345451642191\\
26.5140858539001	0.377343191787191\\
26.5678315643021	0.377340976799794\\
26.6215772747041	0.377338805794205\\
26.6753229851061	0.377336677901736\\
26.7298936218998	0.377334560583676\\
26.7844642586936	0.377332485960075\\
26.8390348954873	0.377330453174814\\
26.893605532281	0.377328461388567\\
26.9490273164448	0.377326479652769\\
27.0044491006085	0.377324538511598\\
27.0598708847722	0.377322637138072\\
27.1152926689359	0.377320774721683\\
27.1715930662283	0.377318921850862\\
27.2278934635207	0.377317107546765\\
27.2841938608131	0.377315331011005\\
27.3404942581055	0.377313591461359\\
27.3972810276665	0.377311873572173\\
27.4540677972275	0.377310191761109\\
27.5108545667884	0.377308545274544\\
27.5676413363494	0.377306933374247\\
27.6246835621125	0.377305348313673\\
27.6817257878755	0.377303796703927\\
27.7387680136386	0.377302277842811\\
27.7958102394016	0.377300791042542\\
27.8531117250382	0.377299329085164\\
27.9104132106748	0.377297898132375\\
27.9677146963114	0.377296497530084\\
28.025016181948	0.377295126637695\\
28.0825792753208	0.377293778767585\\
28.1401423686935	0.377292459623561\\
28.1977054620663	0.377291168596571\\
28.2552685554391	0.377289905090192\\
28.3130956457549	0.37728866291035\\
28.3709227360708	0.377287447335131\\
28.4287498263866	0.377286257797635\\
28.4865769167025	0.377285093742775\\
28.5446704260258	0.377283949435818\\
28.602763935349	0.377282829758985\\
28.6608574446723	0.377281734184809\\
28.7189509539956	0.377280662196872\\
28.7773133374	0.377279608487796\\
28.8356757208045	0.377278577571824\\
28.8940381042089	0.377277568958372\\
28.9524004876134	0.377276582167176\\
29.0110342352743	0.377275612288283\\
29.0696679829352	0.377274663494041\\
29.1283017305961	0.377273735328341\\
29.186935478257	0.37727282734472\\
29.2458431140456	0.377271935002701\\
29.3047507498342	0.377271062157062\\
29.3636583856228	0.37727020838391\\
29.4225660214114	0.377269373268363\\
29.4817501049045	0.377268552613295\\
29.5409341883977	0.377267749978636\\
29.6001182718908	0.377266964970589\\
29.659302355384	0.377266197203772\\
29.7187654815208	0.377265442800015\\
29.7782286076576	0.377264705045596\\
29.8376917337944	0.377263983574822\\
29.8971548599312	0.377263278029853\\
29.9568996607722	0.377262584828687\\
30.0166444616132	0.377261907003739\\
30.0763892624543	0.377261244215544\\
30.1361340632953	0.377260596131971\\
30.1961632086509	0.377259959445923\\
30.2561923540065	0.377259336954391\\
30.316221499362	0.37725872834239\\
30.3762506447176	0.37725813330177\\
30.436566841334	0.377257548780517\\
30.4968830379503	0.377256977357378\\
30.5571992345667	0.377256418740197\\
30.617515431183	0.377255872643191\\
30.6781214246657	0.377255336250929\\
30.7387274181483	0.377254811939931\\
30.799333411631	0.377254299439327\\
30.8599394051136	0.377253798484186\\
30.9208379817972	0.377253306478404\\
30.9817365584807	0.377252825611202\\
31.0426351351642	0.37725235563155\\
31.1035337118477	0.377251896293947\\
31.1647276970968	0.377251445205557\\
31.2259216823458	0.377251004382181\\
31.2871156675948	0.377250573591269\\
31.3483096528438	0.377250152605423\\
31.4098019130839	0.377249739220101\\
31.4712941733241	0.377249335290614\\
31.5327864335642	0.377248940601622\\
31.5942786938043	0.377248554942577\\
31.656072138367	0.377248176283295\\
31.7178655829298	0.377247806330621\\
31.7796590274925	0.377247444885235\\
31.8414524720552	0.37724709175228\\
31.903550051626	0.377246745062955\\
31.9656476311968	0.377246406386817\\
32.0277452107676	0.377246075539454\\
32.0898427903383	0.377245752340602\\
32.1522474993513	0.377245435070776\\
32.2146522083642	0.377245125172637\\
32.2770569173772	0.37724482247564\\
32.3394616263901	0.377244526813093\\
32.4021765035977	0.377244236603601\\
32.4648913808053	0.377243953172586\\
32.527606258013	0.377243676362394\\
32.5903211352206	0.377243406018953\\
32.6533492654047	0.377243140688514\\
32.7163773955889	0.377242881588235\\
32.7794055257731	0.377242628572442\\
32.8424336559572	0.377242381498789\\
32.9057781690946	0.377242139031479\\
32.969122682232	0.377241902287731\\
33.0324671953693	0.377241671132997\\
33.0958117085067	0.377241445435823\\
33.1594757818323	0.377241223969356\\
33.2231398551579	0.3772410077586\\
33.2868039284835	0.377240796679343\\
33.3504680018091	0.377240590610241\\
33.4144548595075	0.377240388425023\\
33.478441717206	0.377240191063644\\
33.5424285749044	0.377239998411484\\
33.6064154326028	0.377239810356583\\
33.6707283499096	0.377239625865482\\
33.7350412672163	0.377239445799739\\
33.7993541845231	0.377239270053633\\
33.8636671018298	0.377239098523909\\
33.9283094023156	0.377238930262722\\
33.9929517028013	0.377238766059389\\
34.057594003287	0.37723860581644\\
34.1222363037727	0.377238449438693\\
34.1872113622324	0.377238296057236\\
34.2521864206922	0.377238146394852\\
34.3171614791519	0.37723800036172\\
34.3821365376116	0.377237857870141\\
34.447447781095	0.377237718123942\\
34.5127590245783	0.377237581784658\\
34.5780702680616	0.377237448769556\\
34.643381511545	0.377237318997867\\
34.7090324201268	0.377237191740425\\
34.7746833287086	0.377237067602404\\
34.8403342372905	0.377236946507635\\
34.9059851458723	0.377236828381769\\
34.9719792533529	0.377236712557333\\
35.0379733608335	0.377236599587665\\
35.1039674683141	0.377236489402675\\
35.1699615757948	0.377236381933951\\
35.2363024710846	0.377236276570798\\
35.3026433663745	0.377236173818904\\
35.3689842616644	0.377236073613799\\
35.4353251569542	0.37723597589257\\
35.5020164861975	0.377235880096744\\
35.5687078154407	0.377235786688228\\
35.6353991446839	0.377235695607751\\
35.7020904739271	0.377235606797483\\
35.76913594038	0.37723551974697\\
35.8361814068328	0.377235434877905\\
35.9032268732857	0.377235352135827\\
35.9702723397385	0.377235271467604\\
36.0376757049608	0.377235192406906\\
36.1050790701831	0.377235115338517\\
36.1724824354054	0.377235040212419\\
36.2398858006277	0.377234966979818\\
36.3076508877251	0.377234895214916\\
36.3754159748225	0.377234825268632\\
36.4431810619199	0.37723475709505\\
36.5109461490173	0.377234690649388\\
36.5790768402747	0.377234625543054\\
36.6472075315322	0.377234562095915\\
36.7153382227897	0.377234500265843\\
36.7834689140472	0.377234440011752\\
36.8519691566203	0.377234380979196\\
36.9204693991935	0.37723432345958\\
36.9889696417667	0.377234267414269\\
37.0574698843398	0.377234212805593\\
37.1263436888854	0.37723415931042\\
37.1952174934311	0.37723410719408\\
37.2640912979767	0.377234056421162\\
37.3329651025223	0.377234006957143\\
37.4022165438372	0.377233958507599\\
37.4714679851522	0.377233911313985\\
37.5407194264671	0.377233865343862\\
37.609970867782	0.377233820565607\\
37.6796040887873	0.377233776711096\\
37.7492373097926	0.377233733999939\\
37.8188705307979	0.377233692402434\\
37.8885037518033	0.377233651889629\\
37.9585229632751	0.377233612217487\\
38.0285421747469	0.377233573585635\\
38.0985613862187	0.377233535966891\\
38.1685805976905	0.377233499334764\\
38.2389900798134	0.377233463467263\\
38.3093995619363	0.377233428545747\\
38.3798090440592	0.377233394545352\\
38.4502185261821	0.377233361441856\\
38.5209622984945	0.377233329060676\\
38.591706070807	0.377233297537791\\
38.6624498431194	0.377233266850531\\
38.7331936154318	0.377233236976809\\
38.8042656863826	0.377233207761962\\
38.8753377573333	0.377233179325263\\
38.9464098282841	0.377233151646056\\
39.0174818992348	0.377233124704223\\
39.0888909542047	0.377233098357503\\
39.1603000091745	0.37723307271603\\
39.2317090641443	0.377233047760991\\
39.3031181191142	0.377233023474062\\
39.3748725127202	0.377232999724639\\
39.4466269063262	0.37723297661416\\
39.5183812999322	0.377232954125495\\
39.5901356935382	0.37723293224196\\
39.662243568804	0.377232910843835\\
39.7343514440698	0.377232890024364\\
39.8064593193355	0.377232869767952\\
39.8785671946013	0.377232850059413\\
39.95103651206	0.377232830789162\\
40.0235058295186	0.377232812042752\\
40.0959751469772	0.37723279380599\\
40.1684444644358	0.37723277606506\\
40.2412830193083	0.377232758719786\\
40.3141215741807	0.377232741848532\\
40.3869601290532	0.377232725438385\\
40.4597986839256	0.377232709476779\\
40.5330141309314	0.377232693872258\\
40.6062295779373	0.377232678696484\\
40.6794450249431	0.377232663937716\\
40.752660471949	0.377232649584525\\
40.8262603410054	0.37723263555352\\
40.8998602100619	0.377232621910134\\
40.9734600791183	0.377232608643696\\
41.0470599481748	0.377232595743817\\
41.1210516636287	0.377232583134545\\
41.1950433790826	0.377232570875545\\
41.2690350945364	0.377232558957118\\
41.3430268099903	0.377232547369827\\
41.4174177047231	0.377232536044572\\
41.4918085994559	0.377232525035682\\
41.5661994941887	0.377232514334349\\
41.6405903889215	0.377232503932003\\
41.7153877210576	0.377232493765844\\
41.7901850531936	0.377232483885282\\
41.8649823853296	0.377232474282319\\
41.9397797174656	0.377232464949176\\
42.0149906816255	0.377232455828839\\
42.0902016457854	0.377232446966187\\
42.1654126099454	0.377232438353963\\
42.2406235741053	0.377232429985108\\
42.316255315397	0.377232421807913\\
42.3918870566888	0.377232413863095\\
42.4675187979805	0.37723240614407\\
42.5431505392722	0.377232398644437\\
42.6192101664345	0.377232391317345\\
42.6952697935968	0.377232384199689\\
42.771329420759	0.377232377285503\\
42.8473890479213	0.377232370568984\\
42.9238836429179	0.377232364007721\\
43.0003782379145	0.377232357635113\\
43.0768728329111	0.377232351445752\\
43.1533674279077	0.377232345434384\\
43.2303040564086	0.377232339562651\\
43.3072406849096	0.377232333860755\\
43.3841773134105	0.377232328323798\\
43.4611139419115	0.377232322947022\\
43.5384996649489	0.37723231769577\\
43.6158853879863	0.377232312597323\\
43.6932711110238	0.377232307647248\\
43.7706568340612	0.377232302841237\\
43.8484987166093	0.377232298148007\\
43.9263405991575	0.377232293592173\\
44.0041824817057	0.377232289169727\\
44.0820243642538	0.377232284876771\\
44.1603294838729	0.377232280685091\\
44.2386346034919	0.377232276616877\\
44.3169397231109	0.377232272668504\\
44.39524484273	0.377232268836449\\
44.4740202988846	0.377232265095288\\
44.5527957550393	0.377232261465003\\
44.631571211194	0.377232257942321\\
44.7103466673487	0.377232254524058\\
44.7895995895968	0.377232251187323\\
44.8688525118449	0.377232247950096\\
44.948105434093	0.377232244809419\\
45.0273583563411	0.377232241762419\\
45.1070959123988	0.377232238788503\\
45.1868334684566	0.377232235903833\\
45.2665710245144	0.377232233105738\\
45.3463085805721	0.377232230391626\\
45.4265379845256	0.37723222774299\\
45.5067673884792	0.377232225174341\\
45.5869967924327	0.377232222683273\\
45.6672261963862	0.377232220267446\\
45.7479547148514	0.37723221791024\\
45.8286832333165	0.377232215624675\\
45.9094117517817	0.377232213408582\\
45.9901402702469	0.377232211259852\\
46.0713752310891	0.377232209163573\\
46.1526101919314	0.377232207131416\\
46.2338451527736	0.377232205161424\\
46.3150801136159	0.377232203251701\\
46.3968289125376	0.377232201388877\\
46.4785777114593	0.377232199583403\\
46.560326510381	0.377232197833518\\
46.6420753093027	0.377232196137516\\
46.7243454163926	0.37723219448342\\
46.8066155234824	0.377232192880581\\
46.8888856305723	0.377232191327414\\
46.9711557376621	0.377232189822385\\
47.0539547042701	0.377232188354776\\
47.136753670878	0.377232186932944\\
47.2195526374859	0.377232185555465\\
47.3023516040939	0.37723218422096\\
47.3856870701744	0.377232182919844\\
47.4690225362549	0.377232181659582\\
47.5523580023354	0.377232180438893\\
47.6356934684159	0.377232179256539\\
47.7195731680798	0.377232178103957\\
47.8034528677438	0.377232176987806\\
47.8873325674077	0.377232175906938\\
47.9712122670716	0.37723217486024\\
48.0556440359198	0.37723217384007\\
48.1400758047679	0.377232172852362\\
48.2245075736161	0.377232171896086\\
48.3089393424642	0.377232170970246\\
48.3939311234603	0.377232170068023\\
48.4789229044563	0.377232169194705\\
48.5639146854524	0.377232168349367\\
48.6489064664485	0.377232167531115\\
48.7344663172648	0.377232166733875\\
48.820026168081	0.377232165962351\\
48.9055860188973	0.377232165215714\\
48.9911458697136	0.377232164493164\\
49.0772819670679	0.377232163789293\\
49.1634180644223	0.377232163108281\\
49.2495541617767	0.377232162449388\\
49.3356902591311	0.377232161811898\\
49.4224109073406	0.377232161190999\\
49.5091315555501	0.377232160590405\\
49.5958522037596	0.377232160009452\\
49.6825728519691	0.377232159447501\\
49.769886487838	0.377232158900275\\
49.857200123707	0.377232158371069\\
49.944513759576	0.377232157859292\\
50.0318273954449	0.377232157364369\\
50.1197425956162	0.377232156882506\\
50.2076577957875	0.377232156416621\\
50.2955729959588	0.377232155966186\\
50.3834881961302	0.377232155530691\\
50.4720136825001	0.377232155106766\\
50.5605391688701	0.377232154696999\\
50.64906465524	0.377232154300917\\
50.73759014161	0.377232153918064\\
50.8267347882915	0.377232153545455\\
50.915879434973	0.377232153185377\\
51.0050240816545	0.377232152837412\\
51.094168728336	0.377232152501151\\
51.1839415683769	0.377232152173951\\
51.2737144084178	0.377232151857834\\
51.3634872484587	0.377232151552427\\
51.4532600884996	0.377232151257366\\
51.5436703203087	0.377232150970314\\
51.6340805521178	0.377232150693055\\
51.7244907839268	0.377232150425256\\
51.8149010157359	0.377232150166596\\
51.9059580087448	0.377232149915006\\
51.9970150017537	0.377232149672063\\
52.0880719947626	0.377232149437469\\
52.1791289877716	0.377232149210939\\
52.270842290771	0.377232148990646\\
52.3625555937704	0.377232148777979\\
52.4542688967698	0.377232148572675\\
52.5459821997693	0.37723214837448\\
52.6383615465513	0.377232148181781\\
52.7307408933333	0.377232147995803\\
52.8231202401153	0.377232147816311\\
52.9154995868973	0.377232147643079\\
53.0085549039533	0.377232147474688\\
53.1016102210094	0.377232147312212\\
53.1946655380654	0.377232147155446\\
53.2877208551214	0.377232147004187\\
53.3814622684928	0.377232146857187\\
53.4752036818643	0.37723214671539\\
53.5689450952358	0.377232146578612\\
53.6626865086073	0.377232146446676\\
53.7571243510415	0.377232146318482\\
53.8515621934757	0.37723214619486\\
53.9460000359099	0.377232146075647\\
54.0404378783442	0.377232145960685\\
54.1355826971555	0.377232145849009\\
54.2307275159668	0.377232145741346\\
54.325872334778	0.377232145637552\\
54.4210171535893	0.377232145537487\\
54.5168797176294	0.377232145440305\\
54.6127422816695	0.37723214534664\\
54.7086048457095	0.377232145256368\\
54.8044674097496	0.377232145169364\\
54.9010587178797	0.377232145084885\\
54.9976500260098	0.377232145003488\\
55.0942413341398	0.377232144925062\\
55.1908326422699	0.377232144849497\\
55.2881639308243	0.377232144776142\\
55.3854952193786	0.377232144705484\\
55.4828265079329	0.377232144637424\\
55.5801577964873	0.377232144571867\\
55.6782405480441	0.377232144508242\\
55.776323299601	0.377232144446975\\
55.8744060511579	0.377232144387978\\
55.9724888027147	0.377232144331168\\
56.0713347542828	0.377232144276045\\
56.1701807058509	0.377232144222981\\
56.2690266574189	0.377232144171899\\
56.367872608987	0.377232144122724\\
56.4674937609263	0.377232144075023\\
56.5671149128656	0.377232144029116\\
56.6667360648049	0.377232143984938\\
56.7663572167441	0.377232143942423\\
56.8667658411891	0.377232143901192\\
56.9671744656341	0.377232143861525\\
57.0675830900791	0.377232143823363\\
57.1679917145241	0.377232143786649\\
57.2692003658787	0.377232143751052\\
57.3704090172333	0.377232143716817\\
57.4716176685879	0.377232143683891\\
57.5728263199425	0.377232143652225\\
57.6748478434472	0.37723214362153\\
57.7768693669519	0.377232143592018\\
57.8788908904565	0.377232143563645\\
57.9809124139612	0.377232143536365\\
58.0837599560178	0.37723214350993\\
58.1866074980744	0.377232143484522\\
58.2894550401309	0.377232143460102\\
58.3923025821875	0.377232143436631\\
58.4959896007833	0.377232143413892\\
58.5996766193791	0.377232143392044\\
58.7033636379749	0.377232143371053\\
58.8070506565707	0.377232143350884\\
58.911590931702	0.377232143331349\\
59.0161312068333	0.377232143312587\\
59.1206714819646	0.377232143294566\\
59.2252117570959	0.377232143277257\\
59.3306194021433	0.377232143260497\\
59.4360270471907	0.377232143244405\\
59.541434692238	0.377232143228954\\
59.6468423372854	0.377232143214118\\
59.7351317529641	0.377232143202148\\
59.8234211686427	0.377232143190579\\
59.9117105843214	0.377232143179398\\
60	0.377232143168591\\
};
\addlegendentry{Sani}

\addplot [color=mycolor2, line width=2.0pt]
  table[row sep=crcr]{%
0	0\\
0.00199053585276749	1.78229496166774e-07\\
0.00398107170553497	7.12633992228109e-07\\
0.00597160755830246	1.60278740174769e-06\\
0.00796214341106994	2.84826351452976e-06\\
0.0123567669950473	6.85404758419239e-06\\
0.0167513905790247	1.25850776329733e-05\\
0.0211460141630021	2.00367610079099e-05\\
0.0255406377469794	2.92045024365383e-05\\
0.0299388563931551	4.00933033150155e-05\\
0.0343370750393307	5.26917541574596e-05\\
0.0387352936855064	6.69952408299566e-05\\
0.043133512331682	8.29991469417208e-05\\
0.0475471725191068	0.000100763977046306\\
0.0519608327065316	0.000120231866807429\\
0.0563744928939564	0.000141398144441452\\
0.0607881530813811	0.000164258136227289\\
0.0652173199794049	0.000188896387807193\\
0.0696464868774286	0.000215230840266865\\
0.0740756537754524	0.000243256764819336\\
0.0785048206734761	0.000272969431060912\\
0.0829496187548012	0.0003044778764041\\
0.0873944168361263	0.000337675440811992\\
0.0918392149174515	0.000372557338960901\\
0.0962840129987766	0.000409118784232188\\
0.100744568101794	0.000447493511353313\\
0.105205123204811	0.000487550052866088\\
0.109665678307829	0.000529283567379222\\
0.114126233410846	0.000572689212529328\\
0.118602673075784	0.000617925628576703\\
0.123079112740721	0.000664836331890626\\
0.127555552405659	0.000713416425482893\\
0.132031992070596	0.000763661011717144\\
0.136524445545197	0.000815753845246587\\
0.141016899019798	0.000869513216840921\\
0.1455093524944	0.0009249341743874\\
0.150001805969001	0.000982011765450184\\
0.154510404254387	0.00104095506897954\\
0.159019002539772	0.00110155693967016\\
0.163527600825158	0.00116381237075881\\
0.168036199110544	0.00122771635548534\\
0.172561074936287	0.00129350350710582\\
0.17708595076203	0.0013609410336027\\
0.181610826587773	0.00143002387403821\\
0.186135702413515	0.00150074696780503\\
0.190676990405524	0.00157337067538119\\
0.195218278397533	0.00164763634457492\\
0.199759566389542	0.00172353886075141\\
0.204300854381551	0.00180107310993476\\
0.208858690959197	0.00188052541167649\\
0.213416527536843	0.00196161104109733\\
0.217974364114488	0.00204432483034484\\
0.222532200692134	0.00212866161255516\\
0.227106724202768	0.002214933880889\\
0.231681247713401	0.00230283062263748\\
0.236255771224035	0.00239234661721169\\
0.240830294734669	0.00248347664534219\\
0.245421645441331	0.00257655958859938\\
0.250012996147992	0.00267125793098603\\
0.254604346854654	0.00276756639965983\\
0.259195697561315	0.00286547972343\\
0.263804017673384	0.00296536338802291\\
0.268412337785452	0.00306685315772696\\
0.273020657897521	0.00316994370793144\\
0.27762897800959	0.00327462971601052\\
0.282254411746462	0.00338130348928685\\
0.286879845483334	0.0034895738542147\\
0.291505279220206	0.00359943543490193\\
0.296130712957079	0.00371088285777586\\
0.300773406613671	0.0038243354710918\\
0.305416100270264	0.0039393749434537\\
0.310058793926856	0.0040559958481768\\
0.314701487583449	0.00417419276123169\\
0.319361589521735	0.00429441229127567\\
0.324021691460022	0.00441620872886088\\
0.328681793398309	0.00453957659700084\\
0.333341895336596	0.00466451042170162\\
0.338019556071109	0.00479148429394043\\
0.342697216805623	0.0049200249035783\\
0.347374877540136	0.00505012672381997\\
0.35205253827465	0.00518178423120122\\
0.356747910447269	0.00531549922046413\\
0.361443282619888	0.00545077055854718\\
0.366138654792507	0.00558759266934148\\
0.370834026965126	0.00572595998040905\\
0.375547265516765	0.00586640221678161\\
0.380260504068404	0.00600839019522857\\
0.384973742620042	0.00615191829082444\\
0.389686981171681	0.00629698088265588\\
0.394418243246227	0.00644413584996011\\
0.399149505320772	0.00659282573458561\\
0.403880767395317	0.00674304486328981\\
0.408612029469863	0.00689478756718486\\
0.41336147459102	0.00704864010918091\\
0.418110919712177	0.00720401652596603\\
0.422860364833334	0.00736091109648197\\
0.427609809954491	0.00751931810436927\\
0.432377599961091	0.00767985242373971\\
0.437145389967691	0.00784189935772533\\
0.44191317997429	0.00800545313795617\\
0.44668096998089	0.00817050800110655\\
0.451467269141767	0.00833770766381715\\
0.456253568302644	0.00850640846348972\\
0.461039867463521	0.00867660458494879\\
0.465826166624398	0.00884829021841013\\
0.47063114173014	0.00902213815766839\\
0.475436116835881	0.00919747553892492\\
0.480241091941623	0.00937429650070711\\
0.485046067047365	0.0095525951872821\\
0.489869887370685	0.00973307370279398\\
0.494693707694006	0.00991502974812177\\
0.499517528017326	0.0100984574160067\\
0.504341348340647	0.0102833508052798\\
0.509184185682064	0.010470441565063\\
0.51402702302348	0.0106589977253298\\
0.518869860364897	0.0108490133335491\\
0.523712697706314	0.0110404824436309\\
0.528574726619363	0.0112341664928837\\
0.533436755532412	0.0114293035962897\\
0.53829878444546	0.0116258877565611\\
0.543160813358509	0.0118239129832041\\
0.548042211039569	0.0120241707405083\\
0.55292360872063	0.0122258689886652\\
0.55780500640169	0.0124290016861488\\
0.56268640408275	0.0126335627985819\\
0.56758735048985	0.0128403740605175\\
0.572488296896949	0.0130486130330787\\
0.577389243304049	0.0132582736310225\\
0.582290189711149	0.013469349776611\\
0.587210867601438	0.0136826937191749\\
0.592131545491728	0.0138974523752186\\
0.597052223382018	0.0141136196163057\\
0.601972901272307	0.0143311893218633\\
0.606913496388328	0.0145510445082801\\
0.611854091504349	0.0147723011941672\\
0.61679468662037	0.0149949532084216\\
0.621735281736391	0.0152189943881634\\
0.626695982628195	0.0154453387607735\\
0.631656683519999	0.0156730712018742\\
0.636617384411802	0.0159021854982251\\
0.641578085303606	0.0161326754451709\\
0.646559083672327	0.0163654863400263\\
0.651540082041047	0.0165996716554481\\
0.656521080409768	0.0168352251365909\\
0.661502078778488	0.0170721405375573\\
0.666503569383853	0.0173113946798439\\
0.671505059989219	0.0175520093777368\\
0.676506550594584	0.0177939783353207\\
0.68150804119995	0.0180372952659934\\
0.686530221876942	0.0182829687696221\\
0.691552402553934	0.0185299887467158\\
0.696574583230927	0.0187783488608277\\
0.701596763907919	0.0190280427851913\\
0.706639835866396	0.0192801111661146\\
0.711682907824873	0.0195335117211306\\
0.71672597978335	0.0197882380738021\\
0.721769051741827	0.0200442838577409\\
0.72683321952555	0.0203027220311639\\
0.731897387309272	0.0205624778619329\\
0.736961555092995	0.0208235449341646\\
0.742025722876718	0.0210859168423951\\
0.747111194287107	0.021350699117262\\
0.752196665697496	0.0216167843151022\\
0.757282137107885	0.0218841659811343\\
0.762367608518274	0.0221528376713685\\
0.767474594900923	0.0224239377630717\\
0.772581581283573	0.0226963258255768\\
0.777688567666222	0.022969995365755\\
0.782795554048871	0.0232449399016439\\
0.78792427039055	0.0235223309355833\\
0.793052986732229	0.0238009947701551\\
0.798181703073909	0.024080924874437\\
0.803310419415588	0.0243621147290501\\
0.808461084325784	0.0246457692377339\\
0.81361174923598	0.024930681158597\\
0.818762414146177	0.0252168439234816\\
0.823913079056373	0.0255042509761519\\
0.82908591482452	0.0257941408997636\\
0.834258750592667	0.0260852726284857\\
0.839431586360813	0.0263776395574862\\
0.84460442212896	0.0266712350942361\\
0.849799655005291	0.0269673317936325\\
0.854994887881621	0.0272646544721787\\
0.860190120757951	0.0275631964889335\\
0.865385353634282	0.0278629512156415\\
0.870603213699137	0.0281652254623587\\
0.875821073763992	0.0284687096429536\\
0.881038933828848	0.0287733970809441\\
0.886256793893703	0.0290792811129196\\
0.891497515445601	0.0293877031083092\\
0.896738236997499	0.0296973187726708\\
0.901978958549397	0.0300081213945536\\
0.907219680101295	0.0303201042759657\\
0.912483501408722	0.030634643633635\\
0.917747322716149	0.0309503601752138\\
0.923011144023576	0.0312672471548595\\
0.928274965331004	0.0315852978405775\\
0.933562128954513	0.0319059236013681\\
0.938849292578023	0.0322277098403878\\
0.944136456201532	0.0325506497779819\\
0.949423619825042	0.0328747366487364\\
0.954734372782572	0.0332014172869959\\
0.960045125740102	0.0335292414767053\\
0.965355878697632	0.0338582024049824\\
0.970666631655162	0.0341882932735799\\
0.976001225418392	0.0345209966929252\\
0.981335819181621	0.0348548265152663\\
0.98667041294485	0.0351897758950819\\
0.99200500670808	0.0355258380018827\\
0.997363697280812	0.0358645315353067\\
1.00272238785355	0.0362043341009619\\
1.00808107842628	0.0365452388212807\\
1.01343976899901	0.0368872388341272\\
1.0188228172289	0.037231889258631\\
1.02420586545878	0.0375776311216876\\
1.02958891368867	0.0379244575142804\\
1.03497196191855	0.0382723615432268\\
1.04037963341625	0.0386229350696265\\
1.04578730491395	0.0389745822172495\\
1.05119497641164	0.0393272960462321\\
1.05660264790934	0.0396810696329495\\
1.06203521338424	0.0400375319217707\\
1.06746777885913	0.0403950497901295\\
1.07290034433402	0.0407536162679216\\
1.07833290980891	0.041113224401689\\
1.08379064506731	0.0414755405572929\\
1.08924838032571	0.0418388940255793\\
1.09470611558411	0.0422032778068144\\
1.10016385084252	0.0425686849183207\\
1.1056470369148	0.0429368194907377\\
1.11113022298707	0.0433059728829407\\
1.11661340905935	0.0436761380661827\\
1.12209659513163	0.0440473080291863\\
1.12760551858231	0.0444212250280014\\
1.13311444203298	0.0447961421267976\\
1.13862336548365	0.0451720522684358\\
1.14413228893432	0.0455489484136621\\
1.14966724184116	0.0459286112995367\\
1.15520219474799	0.0463092553376871\\
1.16073714765482	0.0466908734432087\\
1.16627210056166	0.0470734585495009\\
1.17183338076584	0.0474588302433789\\
1.17739466097002	0.0478451639129086\\
1.18295594117419	0.0482324524460514\\
1.18851722137837	0.0486206887494949\\
1.19410513262804	0.0490117316343467\\
1.19969304387772	0.0494037170882164\\
1.20528095512739	0.0497966379725686\\
1.21086886637706	0.0501904871680194\\
1.21648371859449	0.0505871630988316\\
1.22209857081192	0.0509847619608921\\
1.22771342302936	0.0513832765898114\\
1.23332827524679	0.0517826998407798\\
1.23897038457538	0.0521849701387735\\
1.24461249390397	0.0525881434978689\\
1.25025460323256	0.0529922127284713\\
1.25589671256115	0.053397170660997\\
1.26156640151281	0.0538049961141032\\
1.26723609046446	0.0542137045244638\\
1.27290577941612	0.0546232886779336\\
1.27857546836778	0.0550337413808136\\
1.28427306633634	0.055447082270123\\
1.28997066430491	0.0558612857778388\\
1.29566826227347	0.056276344665926\\
1.30136586024204	0.0566922517172335\\
1.30709170327692	0.0571110677899407\\
1.31281754631181	0.0575307259056923\\
1.31854338934669	0.0579512188032306\\
1.32426923238158	0.0583725392426238\\
1.33002366371043	0.0587967897381723\\
1.33577809503928	0.0592218614633896\\
1.34153252636813	0.0596477471344686\\
1.34728695769699	0.0600744394893741\\
1.35307032782915	0.0605040831349702\\
1.35885369796131	0.060934526957223\\
1.36463706809348	0.061365763650457\\
1.37042043822564	0.0617977859312174\\
1.37622881553645	0.0622324596390805\\
1.38203719284726	0.062667911085466\\
1.38784557015807	0.0631041329597013\\
1.39365394746888	0.0635411179737208\\
1.39948194569202	0.0639803388396922\\
1.40530994391516	0.0644203133789042\\
1.41113794213829	0.0648610342977982\\
1.41696594036143	0.0653024943257206\\
1.4228137248248	0.0657461887088559\\
1.42866150928816	0.0661906126324521\\
1.43450929375153	0.0666357588210682\\
1.44035707821489	0.0670816200224626\\
1.44622478964653	0.067529711942288\\
1.45209250107816	0.0679785091980473\\
1.45796021250979	0.0684280045335158\\
1.46382792394143	0.0688781907159593\\
1.46971570462452	0.0693306038047524\\
1.4756034853076	0.0697836979560405\\
1.48149126599069	0.0702374659339104\\
1.48737904667378	0.0706919005262277\\
1.49328704060906	0.0711485580486969\\
1.49919503454433	0.0716058722941643\\
1.50510302847961	0.0720638360481203\\
1.51101102241489	0.0725224421201183\\
1.51693937480906	0.0729832669409283\\
1.52286772720322	0.0734447240819056\\
1.52879607959739	0.0739068063510339\\
1.53472443199155	0.0743695065806407\\
1.540673290528	0.0748344212698598\\
1.54662214906446	0.0752999438160619\\
1.55257100760091	0.0757660670508077\\
1.55851986613736	0.0762327838302791\\
1.56448937947106	0.0767017105536292\\
1.57045889280476	0.0771712206130624\\
1.57642840613847	0.0776413068647993\\
1.58239791947216	0.0781119621899552\\
1.58838823791446	0.0785848227682237\\
1.59437855635675	0.0790582421067355\\
1.60036887479904	0.0795322130874509\\
1.60635919324133	0.0800067286174952\\
1.61237046959671	0.0804834445995272\\
1.61838174595208	0.0809606947139312\\
1.62439302230746	0.0814384718694809\\
1.63040429866284	0.0819167690003819\\
1.63643668672632	0.0823972615512272\\
1.6424690747898	0.0828782635571501\\
1.64850146285328	0.0833597679548114\\
1.65453385091676	0.0838417677065666\\
1.66058750665404	0.0843259577070028\\
1.66664116239132	0.0848106324386095\\
1.67269481812859	0.085295784867003\\
1.67874847386587	0.0857814079837541\\
1.68482355500326	0.0862692160043038\\
1.69089863614064	0.0867574839882201\\
1.69697371727803	0.087246204931141\\
1.70304879841541	0.0877353718549147\\
1.70914546464501	0.0882267181784963\\
1.71524213087462	0.0887184996564751\\
1.72133879710422	0.0892107093155729\\
1.72743546333383	0.0897033402089743\\
1.73355387572031	0.0901981447891524\\
1.73967228810678	0.0906933596762237\\
1.74579070049326	0.0911889779290548\\
1.75190911287974	0.0916849926332241\\
1.75804943463327	0.0921831751622487\\
1.76418975638681	0.0926817431148583\\
1.77033007814035	0.0931806895831217\\
1.77647039989388	0.0936800076860642\\
1.78263279664118	0.0941814876231382\\
1.78879519338848	0.0946833280674213\\
1.79495759013578	0.095185522145237\\
1.80111998688309	0.0956880630101065\\
1.80730462566597	0.0961927595059342\\
1.81348926444885	0.096697791562122\\
1.81967390323173	0.0972031523402995\\
1.82585854201462	0.0977088350295318\\
1.83206559217232	0.0982166670035133\\
1.83827264233002	0.0987248095632206\\
1.84447969248772	0.0992332559066366\\
1.85068674264543	0.0997419992594138\\
1.85691637550678	0.100252885380206\\
1.86314600836813	0.100764057086937\\
1.86937564122948	0.101275507614988\\
1.87560527409083	0.10178723022764\\
1.88185766319929	0.102301088936977\\
1.88811005230775	0.102815208209952\\
1.89436244141622	0.103329581320385\\
1.90061483052468	0.103844201570222\\
1.90689015093446	0.104360951030299\\
1.91316547134424	0.104877936011763\\
1.91944079175402	0.105395149827913\\
1.92571611216381	0.1059125858204\\
1.93201454150218	0.106432144006196\\
1.93831297084055	0.10695191265382\\
1.94461140017892	0.10747188511709\\
1.9509098295173	0.107992054778392\\
1.95723154808653	0.108514339491426\\
1.96355326665576	0.109036809592043\\
1.96987498522499	0.109559458475609\\
1.97619670379422	0.110082279566271\\
1.98254189319299	0.110607208325673\\
1.98888708259176	0.11113229738624\\
1.99523227199053	0.111657540185918\\
2.0015774613893	0.112182930191648\\
2.00794630624256	0.112710420381676\\
2.01431515109582	0.113238045776889\\
2.02068399594908	0.113765799858841\\
2.02705284080234	0.114293676138286\\
2.03344552746868	0.114823644906119\\
2.03983821413503	0.115353723776117\\
2.04623090080137	0.115883906274465\\
2.05262358746771	0.116414185956758\\
2.05904030495676	0.116946550292851\\
2.06545702244581	0.117478999623602\\
2.07187373993486	0.118011527520853\\
2.07829045742391	0.118544127586053\\
2.084731396657	0.119078804267194\\
2.0911723358901	0.119613540833514\\
2.0976132751232	0.120148330903535\\
2.1040542143563	0.120683168125578\\
2.11051956957009	0.121220073835554\\
2.11698492478388	0.121757014321793\\
2.12345027999767	0.122293983250511\\
2.12991563521146	0.122830974317916\\
2.13640560238783	0.123370025521262\\
2.1428955695642	0.123909086395005\\
2.14938553674057	0.124448150654067\\
2.15587550391694	0.124987212043557\\
2.16239028199053	0.125528325089679\\
2.16890506006412	0.126069422705869\\
2.17541983813771	0.126610498656771\\
2.1819346162113	0.127151546737399\\
2.18847440630977	0.127694637801226\\
2.19501419640825	0.128237688342804\\
2.20155398650673	0.128780692177511\\
2.20809377660521	0.129323643151272\\
2.21465878275268	0.129868628295266\\
2.22122378890015	0.130413547835165\\
2.22778879504762	0.130958395638091\\
2.23435380119508	0.131503165601885\\
2.24094422983074	0.132049960739503\\
2.24753465846639	0.132596665204106\\
2.25412508710205	0.133143272915563\\
2.2607155157377	0.133689777824635\\
2.2673315759421	0.134238298742426\\
2.27394763614649	0.134786703933648\\
2.28056369635089	0.135334987371921\\
2.28717975655528	0.135883143061925\\
2.29382166035897	0.136433305448258\\
2.30046356416266	0.136983327072252\\
2.30710546796636	0.137533201962278\\
2.31374737177005	0.138082924177934\\
2.32041533418655	0.13863464362862\\
2.32708329660306	0.13918619730138\\
2.33375125901957	0.139737579280336\\
2.34041922143607	0.140288783680998\\
2.34711345989709	0.140841975655325\\
2.3538076983581	0.141394976858737\\
2.36050193681912	0.141947781432108\\
2.36719617528013	0.14250038354785\\
2.37391691043736	0.143054963437601\\
2.3806376455946	0.143609327588417\\
2.38735838075183	0.144163470198918\\
2.39407911590907	0.144717385499413\\
2.40082657131283	0.145273268604674\\
2.40757402671659	0.145828911030324\\
2.41432148212035	0.14638430703372\\
2.42106893752412	0.146939450904056\\
2.42784333963791	0.147496552436746\\
2.4346177417517	0.148053388378853\\
2.4413921438655	0.148609953047465\\
2.44816654597929	0.149166240791652\\
2.4549681243632	0.149724475892761\\
2.4617697027471	0.150282420524369\\
2.46857128113101	0.150840069064282\\
2.47537285951492	0.15139741592243\\
2.48220184698832	0.151956699677276\\
2.48903083446173	0.152515668118071\\
2.49585982193513	0.153074315684334\\
2.50268880940853	0.153632636847836\\
2.50954544177924	0.154192884265572\\
2.51640207414995	0.15475279156142\\
2.52325870652066	0.155312353237594\\
2.53011533889137	0.155871563828689\\
2.53699985562036	0.156432689898243\\
2.54388437234935	0.156993451077047\\
2.55076888907833	0.157553841930992\\
2.55765340580732	0.158113857058483\\
2.5645660494942	0.158675776707707\\
2.57147869318108	0.159237306738598\\
2.57839133686796	0.159798441781713\\
2.58530398055484	0.160359176500242\\
2.59224499685821	0.160921804590818\\
2.59918601316159	0.161484018379078\\
2.60612702946496	0.162045812561225\\
2.61306804576833	0.162607181866211\\
2.62003768421203	0.163170433260169\\
2.62700732265572	0.163733245713641\\
2.63397696109942	0.16429561398946\\
2.64094659954311	0.164857532883317\\
2.64794511298781	0.165421322400769\\
2.65494362643251	0.165984648387656\\
2.6619421398772	0.166547505674416\\
2.6689406533219	0.167109889124458\\
2.67596829852112	0.167674131589036\\
2.68299594372033	0.168237885983256\\
2.69002358891955	0.168801147206139\\
2.69705123411877	0.169363910189783\\
2.70410827073433	0.169928520349858\\
2.71116530734989	0.170492617952393\\
2.71822234396545	0.171056197965976\\
2.72527938058101	0.171619255392368\\
2.73236607296869	0.172184148064567\\
2.73945276535637	0.172748503746739\\
2.74653945774405	0.173312317478009\\
2.75362615013173	0.173875584330772\\
2.76074276611797	0.174440674301427\\
2.76785938210422	0.175005202906499\\
2.77497599809046	0.175569165256629\\
2.78209261407671	0.176132556495816\\
2.78923942490227	0.176697758516725\\
2.79638623572783	0.177262374855687\\
2.80353304655339	0.177826400695837\\
2.81067985737896	0.178389831253751\\
2.81785713902212	0.178955060146615\\
2.82503442066529	0.179519679102413\\
2.83221170230846	0.180083683377743\\
2.83938898395163	0.180647068262726\\
2.84659701614953	0.181212238840447\\
2.85380504834743	0.181776775289424\\
2.86101308054533	0.182340672940694\\
2.86822111274323	0.182903927158892\\
2.87546017926332	0.183468954246267\\
2.88269924578341	0.184033323078824\\
2.88993831230351	0.184597029063011\\
2.8971773788236	0.185160067638948\\
2.90444776780364	0.185724866098049\\
2.91171815678368	0.186288982243964\\
2.91898854576373	0.186852411559529\\
2.92625893474377	0.187415149561312\\
2.93356093845992	0.187979634271849\\
2.94086294217606	0.188543412680689\\
2.94816494589221	0.189106480348023\\
2.95546694960836	0.189668832867841\\
2.96280086487659	0.190232918756656\\
2.97013478014483	0.190796274427191\\
2.97746869541306	0.191358895517971\\
2.9848026106813	0.191920777701373\\
2.99216873909201	0.192484379758154\\
2.99953486750271	0.193047227754038\\
3.00690099591342	0.19360931740685\\
3.01426712432413	0.194170644468324\\
3.02166577186631	0.194733677714246\\
3.02906441940849	0.195295933132723\\
3.03646306695067	0.19585740652186\\
3.04386171449286	0.19641809371371\\
3.05129319192691	0.196980473227527\\
3.05872466936097	0.197542051225467\\
3.06615614679503	0.198102823586884\\
3.07358762422909	0.198662786225125\\
3.08105224696524	0.199224427131101\\
3.0885168697014	0.199785242912961\\
3.09598149243755	0.200345229532282\\
3.1034461151737	0.200904382984669\\
3.11094420406679	0.201465200509271\\
3.11844229295988	0.202025169383617\\
3.12594038185297	0.202584285652481\\
3.13343847074605	0.203142545394698\\
3.14097035080817	0.203702454766408\\
3.14850223087028	0.20426149204603\\
3.1560341109324	0.204819653362512\\
3.16356599099451	0.205376934878889\\
3.17113199342146	0.205935851476127\\
3.17869799584842	0.206493872625514\\
3.18626399827537	0.207050994541145\\
3.19383000070233	0.207607213471222\\
3.20143046110215	0.20816505268603\\
3.20903092150198	0.208721973185465\\
3.21663138190181	0.209277971269751\\
3.22423184230164	0.209833043273232\\
3.2318671021242	0.210389720613149\\
3.23950236194676	0.210945456060278\\
3.24713762176933	0.211500246001944\\
3.25477288159189	0.212054086859606\\
3.26244328804483	0.212609517935877\\
3.27011369449776	0.21316398403393\\
3.2777841009507	0.213717481629176\\
3.28545450740363	0.214270007231162\\
3.29316041266953	0.21482410769691\\
3.30086631793543	0.215377220193101\\
3.30857222320133	0.215929341284209\\
3.31627812846722	0.216480467568846\\
3.32401989110489	0.217033153215428\\
3.33176165374256	0.217584827996965\\
3.33950341638023	0.218135488567979\\
3.3472451790179	0.218685131617123\\
3.35502316313618	0.219236318308441\\
3.36280114725446	0.219786471337072\\
3.37057913137275	0.220335587448572\\
3.37835711549103	0.220883663422614\\
3.38617169154667	0.221433267146126\\
3.3939862676023	0.221981814508963\\
3.40180084365794	0.2225293023487\\
3.40961541971357	0.223075727537014\\
3.41746696420648	0.223623664375333\\
3.42531850869938	0.224170522256551\\
3.43317005319229	0.224716298111252\\
3.4410215976852	0.225260988904098\\
3.44891049346576	0.225807175049922\\
3.45679938924632	0.226352259745632\\
3.46468828502688	0.226896240015811\\
3.47257718080744	0.227439112919096\\
3.48050381756295	0.227983464701446\\
3.48843045431847	0.228526692645821\\
3.49635709107398	0.229068793871801\\
3.50428372782949	0.229609765532983\\
3.51224850159594	0.230152199375512\\
3.52021327536239	0.230693487099262\\
3.52817804912884	0.231233625919808\\
3.53614282289529	0.231772613086699\\
3.54414613607264	0.232313045501677\\
3.55214944924998	0.232852309626067\\
3.56015276242733	0.233390402772439\\
3.56815607560468	0.233927322287292\\
3.57619833889781	0.234465669998404\\
3.58424060219095	0.235002827357516\\
3.59228286548408	0.235538791775198\\
3.60032512877721	0.236073560695894\\
3.60840675922679	0.23660974049511\\
3.61648838967638	0.237144707993318\\
3.62457002012596	0.237678460700094\\
3.63265165057554	0.238210996158835\\
3.64077307248204	0.23874492496033\\
3.64889449438855	0.239277619625989\\
3.65701591629505	0.239809077765412\\
3.66513733820155	0.240339297021948\\
3.67329898353354	0.240870891879285\\
3.68146062886554	0.241401230881835\\
3.68962227419754	0.241930311740234\\
3.69778391952953	0.242458132198799\\
3.70598622860803	0.242987310338504\\
3.71418853768653	0.243515211021866\\
3.72239084676502	0.24404183206158\\
3.73059315584352	0.244567171303945\\
3.73883657626523	0.245093850044248\\
3.74707999668693	0.245619229845943\\
3.75532341710863	0.246143308624811\\
3.76356683753034	0.246666084330152\\
3.77185182516863	0.247190181135163\\
3.78013681280692	0.247712957640225\\
3.78842180044522	0.248234411865239\\
3.79670678808351	0.24875454186353\\
3.80503380745825	0.249275974354318\\
3.81336082683299	0.249796065306307\\
3.82168784620774	0.25031481284455\\
3.83001486558248	0.250832215127431\\
3.83838438958511	0.251350901051929\\
3.84675391358775	0.251868224322966\\
3.85512343759039	0.252384183171792\\
3.86349296159302	0.252898775862888\\
3.8719054722877	0.253414633132919\\
3.88031798298237	0.253929106760513\\
3.88873049367705	0.25444219508417\\
3.89714300437172	0.254953896475506\\
3.90559899284027	0.255466843142941\\
3.91405498130882	0.255978385306254\\
3.92251096977737	0.256488521412249\\
3.93096695824593	0.256997249940728\\
3.93946692479213	0.257507204195976\\
3.94796689133833	0.258015733214304\\
3.95646685788454	0.258522835551883\\
3.96496682443074	0.25902850979776\\
3.97351127881372	0.259535389969398\\
3.98205573319669	0.26004082430179\\
3.99060018757967	0.260544811461549\\
3.99914464196265	0.261047350148027\\
4.00773410425464	0.261551074738309\\
4.01632356654662	0.262053333018858\\
4.02491302883861	0.262554123767803\\
4.0335024911306	0.263053445795876\\
4.04213749081375	0.263553933411085\\
4.05077249049689	0.264052934379652\\
4.05940749018004	0.264550447592313\\
4.06804248986318	0.265046471972257\\
4.07672356764034	0.265543641412256\\
4.08540464541749	0.266039304003484\\
4.09408572319465	0.266533458750376\\
4.10276680097181	0.267026104689669\\
4.11149450824228	0.267519874898361\\
4.12022221551276	0.268012118192382\\
4.12894992278323	0.268502833690975\\
4.1376776300537	0.268992020545523\\
4.14645252883691	0.269482310588778\\
4.15522742762012	0.269971053789261\\
4.16400232640333	0.270458249382136\\
4.17277722518654	0.27094389663454\\
4.18159988926615	0.271430625746039\\
4.19042255334575	0.271915788225715\\
4.19924521742536	0.272399383425783\\
4.20806788150496	0.272881410730249\\
4.21693889629988	0.273364498283641\\
4.22580991109479	0.273845999556603\\
4.23468092588971	0.274325914019533\\
4.24355194068463	0.27480424117444\\
4.2524719033702	0.275283606669865\\
4.26139186605577	0.275761366378141\\
4.27031182874135	0.276237519888999\\
4.27923179142692	0.276712066823588\\
4.28820131218307	0.27718762993377\\
4.29717083293922	0.277661567893071\\
4.30614035369538	0.278133880411719\\
4.31510987445153	0.278604567231155\\
4.32412957614907	0.279076247761014\\
4.33314927784661	0.27954628392057\\
4.34216897954416	0.280014675541718\\
4.3511886812417	0.280481422487358\\
4.36025919971018	0.280949140365395\\
4.36932971817866	0.281415194799352\\
4.37840023664715	0.28187958574398\\
4.38747075511563	0.282342313184815\\
4.39659274002011	0.282805988484752\\
4.40571472492458	0.283267981413614\\
4.41483670982906	0.283728292050212\\
4.42395869473353	0.284186920503908\\
4.43313280988389	0.284646473434653\\
4.44230692503424	0.285104325215259\\
4.45148104018459	0.28556047604981\\
4.46065515533495	0.286014926172707\\
4.469882079136	0.286470277074459\\
4.47910900293706	0.286923908196053\\
4.48833592673811	0.287375819868075\\
4.49756285053916	0.287826012451189\\
4.50684327607396	0.288277081772373\\
4.51612370160877	0.288726412833641\\
4.52540412714357	0.289174006093337\\
4.53468455267837	0.289619862039621\\
4.54401918924822	0.290066570383689\\
4.55335382581807	0.290511522139141\\
4.56268846238792	0.290954717893339\\
4.57202309895777	0.291396158263203\\
4.58141267108196	0.291838426309799\\
4.59080224320616	0.292278919591452\\
4.60019181533036	0.292717638825831\\
4.60958138745456	0.293154584759887\\
4.61902663678989	0.293592333326316\\
4.62847188612523	0.29402828910461\\
4.63791713546056	0.294462452944047\\
4.6473623847959	0.294894825722906\\
4.65686406984621	0.295327975717169\\
4.66636575489653	0.295759315054274\\
4.67586743994685	0.296188844716435\\
4.68536912499717	0.296616565714572\\
4.69492802222378	0.297045038153209\\
4.7044869194504	0.297471682220628\\
4.71404581667701	0.297896499033327\\
4.72360471390363	0.298319489736201\\
4.73320764525141	0.298742591428255\\
4.74281057659919	0.299163852549212\\
4.75241350794698	0.299583274345707\\
4.76201643929476	0.300000858092299\\
4.77165794452804	0.300418271397703\\
4.78129944976131	0.300833834515848\\
4.79094095499459	0.301247548820932\\
4.80058246022786	0.30165941571452\\
4.81026314694842	0.302071099118523\\
4.81994383366897	0.302480922979399\\
4.82962452038952	0.302888888798427\\
4.83930520711008	0.303294998103708\\
4.84902565638322	0.303700909065379\\
4.85874610565637	0.304104951399395\\
4.86846655492951	0.304507126733653\\
4.87818700420266	0.304907436722313\\
4.88794780523678	0.305307533203506\\
4.89770860627089	0.305705752241775\\
4.90746940730501	0.30610209559118\\
4.91723020833913	0.306496565031491\\
4.92703195872998	0.306890805485332\\
4.93683370912084	0.307283159948225\\
4.9466354595117	0.307673630299966\\
4.95643720990256	0.308062218445498\\
4.96628051524823	0.308450561779924\\
4.97612382059389	0.308837010840938\\
4.98596712593956	0.309221567633657\\
4.99581043128522	0.309604234187786\\
5.00569590732134	0.3099866398291\\
5.01558138335745	0.310367143177761\\
5.02546685939356	0.310745746363818\\
5.03535233542967	0.311122451541337\\
5.04528060635073	0.31149887934466\\
5.05520887727179	0.311873397097802\\
5.06513714819285	0.312246007055365\\
5.07506541911391	0.312616711495403\\
5.08503711889176	0.312987121774924\\
5.0950088186696	0.313355614506605\\
5.10498051844744	0.313722192069247\\
5.11495221822528	0.314086856864533\\
5.12496799094167	0.314451210382936\\
5.13498376365806	0.314813639113919\\
5.14499953637444	0.315174145560149\\
5.15501530909083	0.315532732246597\\
5.16507580953502	0.315890990213408\\
5.17513630997921	0.316247316409467\\
5.18519681042339	0.316601713460987\\
5.19525731086758	0.316954184015909\\
5.2053632039849	0.317306308044239\\
5.21546909710223	0.317656493573297\\
5.22557499021955	0.318004743352544\\
5.23568088333688	0.318351060152585\\
5.24583284576096	0.318697012290307\\
5.25598480818505	0.319041019453215\\
5.26613677060914	0.319383084513737\\
5.27628873303322	0.319723210364863\\
5.28648745229033	0.320062953042536\\
5.29668617154743	0.320400744521496\\
5.30688489080453	0.320736587796878\\
5.31708361006163	0.321070485883792\\
5.32732978639303	0.321403981952345\\
5.33757596272443	0.321735520848095\\
5.34782213905582	0.322065105688642\\
5.35806831538722	0.322392739610974\\
5.36836266133408	0.322719952301306\\
5.37865700728095	0.323045202093078\\
5.38895135322782	0.32336849222614\\
5.39924569917468	0.323689825959128\\
5.40958894010065	0.324010718873815\\
5.41993218102661	0.324329643411057\\
5.43027542195257	0.324646602932749\\
5.44061866287853	0.324961600818974\\
5.45101153706255	0.325276137909873\\
5.46140441124658	0.325588701390021\\
5.4717972854306	0.325899294743179\\
5.48219015961463	0.326207921470688\\
5.49263341945276	0.326516067049995\\
5.50307667929089	0.326822234029295\\
5.51351993912902	0.327126426014053\\
5.52396319896715	0.327428646626704\\
5.53445761187213	0.3277303653652\\
5.54495202477712	0.328030100756836\\
5.5554464376821	0.328327856528642\\
5.56594085058709	0.328623636424004\\
5.57648719831941	0.328918893302825\\
5.58703354605173	0.329212162329171\\
5.59757989378405	0.329503447351518\\
5.60812624151637	0.329792752234081\\
5.61872532103302	0.330081512542118\\
5.62932440054967	0.330368280732053\\
5.63992348006633	0.330653060773713\\
5.65052255958298	0.330935856652037\\
5.66117518401188	0.331218085984256\\
5.67182780844078	0.331498319171359\\
5.68248043286967	0.331776560304449\\
5.69313305729857	0.33205281348911\\
5.70384005671797	0.33232847773698\\
5.71454705613736	0.332602142049917\\
5.72525405555675	0.332873810640247\\
5.73596105497615	0.33314348773414\\
5.74672327665992	0.33341255306081\\
5.75748549834369	0.333679614898676\\
5.76824772002746	0.333944677581259\\
5.77900994171123	0.334207745455282\\
5.78982825023888	0.334470178267998\\
5.80064655876654	0.334730604272961\\
5.81146486729419	0.334989027924879\\
5.82228317582184	0.33524545369102\\
5.83315845473553	0.335501220650941\\
5.84403373364921	0.335754977717714\\
5.8549090125629	0.336006729467256\\
5.86578429147658	0.336256480487385\\
5.87671744355088	0.336505548482283\\
5.88765059562518	0.33675260373101\\
5.89858374769948	0.336997650930735\\
5.90951689977378	0.337240694789868\\
5.92050884813521	0.337483030926078\\
5.93150079649664	0.337723351694168\\
5.94249274485807	0.337961661912629\\
5.9534846932195	0.338197966410527\\
5.96453638108293	0.338433537972472\\
5.97558806894636	0.338667091774528\\
5.98663975680979	0.3388986327566\\
5.99769144467322	0.339128165868498\\
6.00880383710542	0.339356940322845\\
6.01991622953761	0.339583694854479\\
6.0310286219698	0.339808434524849\\
6.042141014402	0.340031164404622\\
6.05331509935885	0.340253109384354\\
6.0644891843157	0.340473032506225\\
6.07566326927256	0.340690938953377\\
6.08683735422941	0.340906833917479\\
6.09807414230094	0.341121917178946\\
6.10931093037248	0.341334976874237\\
6.12054771844401	0.341546018308367\\
6.13178450651555	0.341755046794183\\
6.14308503319571	0.34196323622371\\
6.15438555987588	0.342169400604273\\
6.16568608655605	0.342373545362971\\
6.17698661323621	0.342575675934031\\
6.18835193888863	0.342776939506413\\
6.19971726454106	0.342976176771644\\
6.21108259019348	0.343173393279152\\
6.2224479158459	0.343368594584776\\
6.23387912710053	0.343562900348477\\
6.24531033835517	0.343755178770402\\
6.2567415496098	0.343945435522582\\
6.26817276086443	0.344133676282738\\
6.2796709718506	0.344320992338939\\
6.29116918283678	0.344506280241224\\
6.30266739382295	0.344689545784534\\
6.31416560480912	0.34487079476877\\
6.32573195844141	0.345051089248579\\
6.33729831207371	0.345229354983699\\
6.34886466570601	0.345405597892329\\
6.36043101933831	0.345579823896886\\
6.37206668825559	0.345753064930683\\
6.38370235717287	0.345924276849381\\
6.39533802609016	0.346093465694817\\
6.40697369500744	0.346260637512296\\
6.41867988207414	0.346426793191786\\
6.43038606914084	0.346590919605399\\
6.44209225620754	0.346753022919031\\
6.45379844327423	0.346913109301282\\
6.46557638449348	0.347072147674733\\
6.47735432571273	0.347229156849893\\
6.48913226693197	0.347384143117176\\
6.50091020815122	0.347537112768929\\
6.51276117329228	0.34768900179831\\
6.52461213843333	0.347838861914366\\
6.53646310357439	0.347986699532533\\
6.54831406871544	0.348132521069396\\
6.56023936237899	0.34827722859346\\
6.57216465604253	0.348419907705717\\
6.58408994970607	0.348560564947171\\
6.59601524336962	0.348699206859178\\
6.608016207688	0.348836700574142\\
6.62001717200638	0.348972166594166\\
6.63201813632476	0.349105611586413\\
6.64401910064313	0.349237042217592\\
6.65609711612489	0.349367289630542\\
6.66817513160664	0.349495510279849\\
6.68025314708839	0.349621710959481\\
6.69233116257014	0.349745898462128\\
6.70448764976455	0.349868866852974\\
6.71664413695897	0.349989809625034\\
6.72880062415338	0.350108733699768\\
6.7409571113478	0.350225645996529\\
6.75319353351478	0.350341302386607\\
6.76542995568177	0.350454934515177\\
6.77766637784875	0.350566549431939\\
6.78990280001574	0.350676154183636\\
6.80222066442117	0.350784465285828\\
6.8145385288266	0.350890753695308\\
6.82685639323203	0.350995026590819\\
6.83917425763746	0.351097291147287\\
6.85157511856051	0.351198223329145\\
6.86397597948357	0.351297134597414\\
6.87637684040662	0.351394032260747\\
6.88877770132968	0.351488923623095\\
6.90126316148383	0.351582442850996\\
6.91374862163799	0.351673943153843\\
6.92623408179214	0.35176343197112\\
6.93871954194629	0.35185091673672\\
6.95129125486549	0.351936988525834\\
6.96386296778468	0.352021043586922\\
6.97643468070388	0.352103089491291\\
6.98900639362307	0.352183133803746\\
7.00166606764739	0.352261723173649\\
7.0143257416717	0.352338298219662\\
7.02698541569601	0.35241286664598\\
7.03964508972033	0.352485436149371\\
7.05239448925109	0.352556507554748\\
7.06514388878186	0.352625567246572\\
7.07789328831262	0.352692623063063\\
7.09064268784339	0.35275768283407\\
7.10348363709426	0.352821200108498\\
7.11632458634514	0.352882708484721\\
7.12916553559601	0.352942215936205\\
7.14200648484689	0.352999730427078\\
7.15494087138423	0.353055656717751\\
7.16787525792157	0.353109577129254\\
7.18080964445891	0.353161499771601\\
7.19374403099625	0.353211432744486\\
7.20677380774895	0.353259730434536\\
7.21980358450165	0.353306025467216\\
7.23283336125435	0.35335032609048\\
7.24586313800706	0.353392640540957\\
7.25899032889946	0.35343327118199\\
7.27211751979186	0.353471902588641\\
7.28524471068426	0.353508543148295\\
7.29837190157667	0.353543201235983\\
7.31159860422458	0.353576125459992\\
7.32482530687249	0.35360705407256\\
7.3380520095204	0.353635995602097\\
7.35127871216831	0.353662958563605\\
7.36460710304072	0.353688135997173\\
7.37793549391314	0.35371132164063\\
7.39126388478556	0.353732524165109\\
7.40459227565797	0.353751752227266\\
7.41802461355885	0.353769141393207\\
7.43145695145972	0.353784542787452\\
7.4448892893606	0.353797965225683\\
7.45832162726148	0.353809417507998\\
7.47186025959003	0.353818975724922\\
7.48539889191859	0.35382655038389\\
7.49893752424715	0.353832150447075\\
7.51247615657571	0.35383578485994\\
7.52612352432791	0.353837468128802\\
7.53977089208011	0.353837172247032\\
7.55341825983232	0.353834906325377\\
7.56706562758452	0.353830679456719\\
7.58082427117841	0.35382444233925\\
7.5945829147723	0.353816230670218\\
7.60834155836619	0.353806053711175\\
7.62210020196009	0.353793920704617\\
7.63597276754062	0.353779716191534\\
7.64984533312116	0.353763541915752\\
7.66371789870169	0.353745407292014\\
7.67759046428223	0.353725321714787\\
7.69157971106386	0.353703101081861\\
7.70556895784549	0.353678915662686\\
7.71955820462712	0.35365277502775\\
7.73354745140875	0.353624688726011\\
7.7476562591015	0.353594401377075\\
7.76176506679424	0.353562154403583\\
7.77587387448699	0.353527957534508\\
7.78998268217974	0.353491820476003\\
7.80421406003167	0.353453413769471\\
7.81844543788359	0.353413052782574\\
7.83267681573552	0.35337074740571\\
7.84690819358745	0.353326507505128\\
7.86126528820976	0.353279926568446\\
7.87562238283208	0.353231396875465\\
7.8899794774544	0.353180928481159\\
7.90433657207672	0.353128531414989\\
7.91882267896367	0.353073718934123\\
7.93330878585062	0.353016963397582\\
7.94779489273757	0.352958275028313\\
7.96228099962452	0.352897664022334\\
7.97689957365268	0.35283456001449\\
7.99151814768085	0.352769518824679\\
8.00613672170902	0.35270255084747\\
8.02075529573718	0.352633666449041\\
8.03550996276884	0.352562208013784\\
8.0502646298005	0.352488818439665\\
8.06501929683216	0.3524135082968\\
8.07977396386382	0.352336288125407\\
8.09466853533245	0.352256409161702\\
8.10956310680107	0.352174605267227\\
8.12445767826969	0.352090887191889\\
8.13935224973832	0.352005265654125\\
8.15439073744049	0.351916896551001\\
8.16942922514267	0.351826608885354\\
8.18446771284484	0.351734413591447\\
8.19950620054702	0.351640321570461\\
8.2146928332504	0.351543388862596\\
8.22987946595378	0.351444544115211\\
8.24506609865717	0.351343798451879\\
8.26025273136055	0.351241162961403\\
8.27559197362809	0.351135588942947\\
8.29093121589562	0.351028109556632\\
8.30627045816316	0.350918736120692\\
8.32160970043069	0.350807479916851\\
8.33710627308224	0.35069318221295\\
8.35260284573378	0.350576985954729\\
8.36809941838533	0.3504589026609\\
8.38359599103687	0.350338943811842\\
8.39925489573158	0.350215834885777\\
8.41491380042629	0.350090834352854\\
8.430572705121	0.349963953938582\\
8.44623160981571	0.349835205328235\\
8.46205815432056	0.349703191942698\\
8.4778846988254	0.349569294022971\\
8.49371124333024	0.349433523508237\\
8.50953778783509	0.34929589229546\\
8.52553761739547	0.349154874881803\\
8.54153744695585	0.349011980121529\\
8.55753727651623	0.348867220175001\\
8.57353710607661	0.348720607158295\\
8.58971623619016	0.348570479122132\\
8.60589536630372	0.348418481030374\\
8.62207449641727	0.348264625272776\\
8.63825362653082	0.348108924192636\\
8.65461848330403	0.347949571096134\\
8.67098334007723	0.347788355324975\\
8.68734819685043	0.347625289507319\\
8.70371305362364	0.347460386222582\\
8.72018188980809	0.347292592831945\\
8.73665072599255	0.347122963961685\\
8.75311956217701	0.346951512283007\\
8.76958839836146	0.346778250417055\\
8.78614348346302	0.346602269421409\\
8.80269856856458	0.346424484650276\\
8.81925365366614	0.346244908771209\\
8.83580873876771	0.346063554400631\\
8.85245111606455	0.345879463883058\\
8.86909349336139	0.345693601480295\\
8.88573587065824	0.345505979853203\\
8.90237824795508	0.345316611610453\\
8.91910883699713	0.345124491786591\\
8.93583942603918	0.344930632176374\\
8.95257001508123	0.344735045430634\\
8.96930060412328	0.344537744146978\\
8.98612034063892	0.3443376759621\\
9.00294007715455	0.344135900289318\\
9.01975981367019	0.343932429766144\\
9.03657955018582	0.343727276975847\\
9.05348938408815	0.343519342098691\\
9.07039921799048	0.343309732223398\\
9.08730905189281	0.343098459970896\\
9.10421888579513	0.342885537906876\\
9.12121978296026	0.34266981868155\\
9.13822068012539	0.342452457130687\\
9.15522157729052	0.342233465855407\\
9.17222247445565	0.342012857400618\\
9.18931541587169	0.34178943683004\\
9.20640835728774	0.341564406781237\\
9.22350129870378	0.341337779832334\\
9.24059424011982	0.34110956850428\\
9.25778022278737	0.34087853021322\\
9.27496620545493	0.340645915457853\\
9.29215218812249	0.340411736790148\\
9.30933817079004	0.340176006703958\\
9.32661820779546	0.339937434912761\\
9.34389824480088	0.339697319829843\\
9.3611782818063	0.339455673977893\\
9.37845831881172	0.339212509820567\\
9.39583344002363	0.338966489311129\\
9.41320856123555	0.33871895883335\\
9.43058368244747	0.338469930877553\\
9.44795880365938	0.338219417874125\\
9.46543005570071	0.33796603396621\\
9.48290130774205	0.337711173556453\\
9.50037255978338	0.337454849099754\\
9.51784381182471	0.337197072990194\\
9.53541225878365	0.336936411508106\\
9.55298070574258	0.336674307126181\\
9.57054915270152	0.336410772260874\\
9.58811759966045	0.336145819266961\\
9.60578432346384	0.335877966511212\\
9.62345104726722	0.335608704585658\\
9.64111777107061	0.33533804586532\\
9.658784494874	0.335066002662702\\
9.67655059520445	0.334791045389354\\
9.6943166955349	0.334514712796984\\
9.71208279586536	0.334237017216225\\
9.72984889619581	0.33395797091437\\
9.74771549159443	0.333675996297084\\
9.76558208699305	0.333392680325048\\
9.78344868239167	0.333108035281588\\
9.80131527779029	0.332822073385884\\
9.81928350543996	0.332533168998522\\
9.83725173308963	0.332242957327139\\
9.8552199607393	0.331951450604866\\
9.87318818838897	0.331658660999901\\
9.89125920482513	0.331362914785642\\
9.90933022126129	0.331065895457601\\
9.92740123769745	0.330767615195854\\
9.94547225413361	0.330468086114786\\
9.96364723590061	0.330165586356243\\
9.9818222176676	0.329861847746816\\
9.99999719943459	0.329556882410705\\
10.0181721812016	0.329250702405675\\
10.0364523247657	0.328941537707875\\
10.0547324683298	0.328631168508104\\
10.0730126118939	0.328319606871895\\
10.0912927554579	0.32800686479762\\
10.1096792783833	0.327691124050744\\
10.1280658013086	0.327374213230212\\
10.146452324234	0.327056144340135\\
10.1648388471593	0.326736929316754\\
10.1833329877765	0.32641470168471\\
10.2018271283938	0.326091338480361\\
10.220321269011	0.325766851643661\\
10.2388154096282	0.325441253046017\\
10.257418428294	0.325112627928743\\
10.2760214469598	0.324782901807203\\
10.2946244656257	0.324452086554506\\
10.3132274842915	0.324120193974543\\
10.3319406634311	0.323785260991652\\
10.3506538425707	0.323449261633078\\
10.3693670217104	0.323112207702418\\
10.38808020085	0.322774110933404\\
10.4069048454226	0.322432959901501\\
10.4257294899953	0.322090777177028\\
10.4445541345679	0.321747574491434\\
10.4633787791406	0.321403363505679\\
10.4823162175897	0.321056084407515\\
10.5012536560388	0.320707808348494\\
10.5201910944879	0.320358546985316\\
10.539128532937	0.320008311903582\\
10.5581801172831	0.319654994873297\\
10.5772317016292	0.319300715656723\\
10.5962832859753	0.318945485833239\\
10.6153348703214	0.318589316910536\\
10.6345019773376	0.318230052199868\\
10.6536690843538	0.317869860114662\\
10.67283619137	0.317508752154434\\
10.6920032983861	0.317146739746438\\
10.7112873293394	0.31678161771814\\
10.7305713602926	0.316415603158784\\
10.7498553912458	0.316048707485504\\
10.769139422199	0.315680942042625\\
10.7885418041401	0.315310053135944\\
10.8079441860811	0.31493830656802\\
10.8273465680222	0.314565713671122\\
10.8467489499633	0.314192285704179\\
10.8662711360792	0.313815720418433\\
10.8857933221952	0.313438332362366\\
10.9053155083112	0.313060132780924\\
10.9248376944271	0.312681132845202\\
10.9444811647161	0.312298981718147\\
10.9641246350051	0.311916042727699\\
10.9837681052941	0.311532327029053\\
11.0034115755831	0.311147845703059\\
11.02317783773	0.310760199286064\\
11.0429440998769	0.31037179992363\\
11.0627103620238	0.309982658678798\\
11.0824766241707	0.309592786539793\\
11.102367213876	0.309199735384444\\
11.1222578035813	0.308805966207818\\
11.1421483932866	0.30841148997843\\
11.1620389829919	0.308016317589523\\
11.182055464688	0.307617952227662\\
11.202071946384	0.30721890377021\\
11.22208842808	0.306819183088806\\
11.2421049097761	0.306418800979379\\
11.2622488777205	0.306015211896388\\
11.282392845665	0.305610974640438\\
11.3025368136094	0.305206099983968\\
11.3226807815539	0.304800598623294\\
11.3429428732039	0.304392097920709\\
11.363204964854	0.303982984619067\\
11.3834670565041	0.303573269372035\\
11.4037291481542	0.303162962756927\\
11.4240073959591	0.302751747421393\\
11.444285643764	0.30233996074277\\
11.464563891569	0.301927613095161\\
11.4848421393739	0.301514714777642\\
11.5051410232308	0.301100855009043\\
11.5254399070878	0.300686463875236\\
11.5457387909448	0.300271551482049\\
11.5660376748017	0.299856127861549\\
11.5863613245615	0.299439695216006\\
11.6066849743212	0.299022769999426\\
11.627008624081	0.298605362059553\\
11.6473322738407	0.298187481171572\\
11.6676846007299	0.297768546422709\\
11.6880369276191	0.297349156792138\\
11.7083892545083	0.296929321878876\\
11.7287415813975	0.296509051210531\\
11.7491263024508	0.29608768429863\\
11.7695110235041	0.295665899160492\\
11.7898957445574	0.295243705155023\\
11.8102804656107	0.294821111570824\\
11.8307011258767	0.294397381546854\\
11.8511217861426	0.29397326898235\\
11.8715424464086	0.293548783004063\\
11.8919631066745	0.293123932669497\\
11.9124230970895	0.29269790768487\\
11.9328830875046	0.292271534933777\\
11.9533430779196	0.291844823318174\\
11.9738030683347	0.291417781671795\\
11.9943056417768	0.290989528968474\\
12.014808215219	0.290560962414\\
12.0353107886612	0.290132090692362\\
12.0558133621033	0.289702922420314\\
12.0763616495731	0.289272508285379\\
12.0969099370428	0.288841813404041\\
12.1174582245125	0.28841084624867\\
12.1380065119823	0.287979615225352\\
12.1586035338403	0.287547105020133\\
12.1792005556983	0.287114346406844\\
12.1997975775563	0.286681347652148\\
12.2203945994143	0.286248116957341\\
12.2410432774846	0.28581357510382\\
12.2616919555548	0.285378816454633\\
12.2823406336251	0.284943849076249\\
12.3029893116954	0.28450868097066\\
12.3236924799435	0.284072170945767\\
12.3443956481917	0.283635475048923\\
12.3650988164398	0.283198601151554\\
12.385801984688	0.282761557061474\\
12.406562398648	0.282323141398436\\
12.4273228126081	0.281884570132818\\
12.4480832265682	0.281445850945828\\
12.4688436405282	0.281006991455904\\
12.4896639845707	0.280566731774239\\
12.5104843286133	0.280126346136797\\
12.5313046726558	0.27968584203909\\
12.5521250166983	0.279245226914675\\
12.5730079121072	0.278803183918208\\
12.5938908075161	0.278361044019633\\
12.614773702925	0.277918814533008\\
12.6356565983338	0.277476502711232\\
12.6566046110283	0.277032736180209\\
12.6775526237227	0.276588901234942\\
12.6985006364172	0.276145005012026\\
12.7194486491116	0.275701054587669\\
12.7404642942679	0.275255623424692\\
12.7614799394242	0.274810151794982\\
12.7824955845805	0.274364646661425\\
12.8035112297368	0.273919114927277\\
12.8245969793381	0.273472077130744\\
12.8456827289394	0.273025026298366\\
12.8667684785407	0.272577969222857\\
12.887854228142	0.272130912638042\\
12.9090125143938	0.271682325353791\\
12.9301708006455	0.271233751970143\\
12.9513290868973	0.270785199112976\\
12.972487373149	0.270336673350004\\
12.9937205946389	0.269886592856412\\
13.0149538161287	0.269436552726199\\
13.0361870376186	0.268986559421558\\
13.0574202591085	0.268536619347224\\
13.0787307850171	0.268085101074123\\
13.1000413109258	0.267633649172994\\
13.1213518368345	0.267182269945318\\
13.1426623627431	0.266730969635813\\
13.1640525369264	0.266278068179419\\
13.1854427111097	0.265825258667718\\
13.206832885293	0.265372547244299\\
13.2282230594763	0.264919939996664\\
13.2496952043377	0.264465709132925\\
13.271167349199	0.264011595367981\\
13.2926394940603	0.263557604690189\\
13.3141116389217	0.263103743032486\\
13.3356680588802	0.26264823573358\\
13.3572244788388	0.262192870285215\\
13.3787808987974	0.261737652523039\\
13.400337318756	0.261282588227938\\
13.4219803036483	0.260825856675305\\
13.4436232885406	0.260369291337975\\
13.4652662734329	0.259912897901276\\
13.4869092583252	0.259456681996421\\
13.5086410871513	0.258998777582184\\
13.5303729159775	0.25854106337558\\
13.5521047448036	0.258083544913884\\
13.5738365736297	0.257626227680893\\
13.5956595174591	0.25716720102316\\
13.6174824612885	0.256708388206759\\
13.639305405118	0.256249794623064\\
13.6611283489474	0.255791425610594\\
13.6830446722324	0.255331326598683\\
13.7049609955174	0.254871464716247\\
13.7268773188025	0.254411845210797\\
13.7487936420875	0.253952473277606\\
13.770805608124	0.25349135102686\\
13.7928175741606	0.253030488860665\\
13.8148295401972	0.252569891884604\\
13.8368415062338	0.252109565152632\\
13.858951377984	0.251647468054081\\
13.8810612497342	0.251185653673952\\
13.9031711214845	0.25072412697774\\
13.9252809932347	0.250262892879915\\
13.9474910360997	0.249799868606288\\
13.9697010789646	0.249337149375078\\
13.9919111218296	0.248874740013437\\
14.0141211646945	0.248412645298086\\
14.0364336494706	0.24794874080385\\
14.0587461342466	0.247485163376988\\
14.0810586190226	0.247021917707969\\
14.1033711037986	0.246559008437414\\
14.1257883079129	0.24609426999268\\
14.1482055120272	0.245629880351585\\
14.1706227161414	0.245165844069488\\
14.1930399202557	0.244702165652479\\
14.2155641320733	0.244236638811304\\
14.2380883438909	0.243771482231322\\
14.2606125557085	0.243306700334276\\
14.283136767526	0.242842297493215\\
14.3057702854045	0.242376027169305\\
14.3284038032829	0.241910148294918\\
14.3510373211614	0.241444665159598\\
14.3736708390398	0.240979582004766\\
14.3964159769655	0.240512612409509\\
14.4191611148912	0.240046055192114\\
14.4419062528169	0.239579914511274\\
14.4646513907425	0.23911419447812\\
14.4875104777278	0.238646569186484\\
14.510369564713	0.238179376949864\\
14.5332286516983	0.237712621797379\\
14.5560877386835	0.237246307711141\\
14.5790631229979	0.236778069626931\\
14.6020385073122	0.23631028503228\\
14.6250138916265	0.235842957827942\\
14.6479892759408	0.235376091868219\\
14.6710833264109	0.234907283248964\\
14.6941773768809	0.234438948319463\\
14.7172714273509	0.23397109085325\\
14.740365477821	0.233503714577956\\
14.7635805844029	0.23303437707349\\
14.7867956909849	0.232565533232557\\
14.8100107975669	0.232097186702558\\
14.8332259041488	0.231629341085534\\
14.8565644820063	0.23115951570475\\
14.8799030598637	0.230690203742641\\
14.9032416377212	0.230221408721497\\
14.9265802155787	0.22975313411879\\
14.9500447068131	0.229282861244067\\
14.9735091980476	0.228813121332062\\
14.996973689282	0.228343917780924\\
15.0204381805165	0.227875253944519\\
15.0440310551214	0.227404573361216\\
15.0676239297264	0.226934445080867\\
15.0912168043314	0.226464872378387\\
15.1148096789363	0.225995858484948\\
15.1385334385982	0.225524809353674\\
15.16225719826	0.22505433166896\\
15.1859809579218	0.22458442858335\\
15.2097047175837	0.224115103206175\\
15.2335618956234	0.223643724112232\\
15.2574190736632	0.22317293541874\\
15.281276251703	0.222702740156674\\
15.3051334297427	0.222233141314326\\
15.3291265950906	0.221761470236792\\
15.3531197604384	0.221290408330729\\
15.3771129257863	0.220819958506295\\
15.4011060911341	0.220350123631496\\
15.4252378496272	0.219878197965946\\
15.4493696081203	0.219406900066679\\
15.4735013666134	0.218936232723742\\
15.4976331251065	0.218466198685558\\
15.5219061221034	0.217994055241025\\
15.5461791191004	0.217522557987944\\
15.5704521160973	0.217051709596907\\
15.5947251130942	0.216581512697403\\
15.619142035055	0.21610918771407\\
15.6435589570157	0.215637527184133\\
15.6679758789765	0.215166533659336\\
15.6923928009372	0.214696209650843\\
15.7169563780926	0.21422373879694\\
15.741519955248	0.213751950501495\\
15.7660835324033	0.213280847197963\\
15.7906471095587	0.212810431279739\\
15.8153601179615	0.212337849661189\\
15.8400731263644	0.211865968555518\\
15.8647861347672	0.211394790278403\\
15.88949914317	0.210924317105983\\
15.9143644081624	0.21045165924975\\
15.9392296731548	0.209979719716407\\
15.9640949381471	0.209508500704317\\
15.9889602031395	0.209038004372831\\
16.0139806003288	0.208565304258946\\
16.039000997518	0.208093340139634\\
16.0640213947072	0.207622114096371\\
16.0890417918965	0.207151628172135\\
16.1142202516767	0.206678919199192\\
16.1393987114569	0.206206963760335\\
16.1645771712372	0.205735763820526\\
16.1897556310174	0.205265321306747\\
16.2150951374624	0.204792636364864\\
16.2404346439074	0.204320722370305\\
16.2657741503524	0.203849581171848\\
16.2911136567974	0.203379214580812\\
16.3166172552097	0.202906585962857\\
16.342120853622	0.202434745585352\\
16.3676244520343	0.201963695181177\\
16.3931280504465	0.201493436446271\\
16.4187988465862	0.201020895917036\\
16.4444696427259	0.200549160807199\\
16.4701404388656	0.200078232733978\\
16.4958112350053	0.199608113278176\\
16.5216524004798	0.199135692029876\\
16.5474935659542	0.198664093271825\\
16.5733347314287	0.198193318505781\\
16.5991758969032	0.1977233691976\\
16.6251906694304	0.197251097901842\\
16.6512054419577	0.196779666065015\\
16.677220214485	0.196309075073559\\
16.7032349870123	0.195839326278541\\
16.7294266758655	0.195367235044649\\
16.7556183647187	0.194896000142285\\
16.781810053572	0.194425622842676\\
16.8080017424252	0.193956104382202\\
16.8343737319713	0.193484222753278\\
16.8607457215174	0.193013214238534\\
16.8871177110636	0.192543079994045\\
16.9134897006097	0.192073821141561\\
16.9400454511807	0.191602178140146\\
16.9666012017517	0.191131424951641\\
16.9931569523226	0.190661562616978\\
17.0197127028936	0.190192592143289\\
17.0464557582069	0.189721216201866\\
17.0731988135202	0.189250746694494\\
17.0999418688336	0.188781184546921\\
17.1266849241469	0.188312530651629\\
17.1536189111414	0.187841449681113\\
17.1805528981359	0.187371291694339\\
17.2074868851304	0.186902057501787\\
17.2344208721249	0.186433747881203\\
17.2615495082445	0.185962989209667\\
17.2886781443641	0.185493170006567\\
17.3158067804836	0.185024290966983\\
17.3429354166032	0.18455635275379\\
17.3702625121198	0.184085943155972\\
17.3975896076364	0.183616489452718\\
17.424916703153	0.183147992223521\\
17.4522437986696	0.182680452016204\\
17.4797732620512	0.182210417692499\\
17.5073027254328	0.181741355637536\\
17.5348321888143	0.181273266314979\\
17.5623616521959	0.180806150157367\\
17.5900974943972	0.180336516735021\\
17.6178333365984	0.179867871910363\\
17.6455691787997	0.179400216030943\\
17.6733050210009	0.178933549413724\\
17.7012513604556	0.178464341939359\\
17.7291976999103	0.177996139353281\\
17.757144039365	0.177528941886577\\
17.7850903788197	0.177062749740295\\
17.8132514459017	0.176593992684996\\
17.8414125129837	0.176126256777231\\
17.8695735800657	0.175659542131222\\
17.8977346471477	0.175193848831702\\
17.9261147917276	0.174725566048351\\
17.9544949363076	0.174258320647764\\
17.9828750808876	0.173792112626836\\
18.0112552254676	0.173326941953531\\
18.0398589202161	0.172859156709629\\
18.0684626149646	0.172392425067064\\
18.097066309713	0.171926746904883\\
18.1256700044615	0.171462122073762\\
18.1545018545664	0.170994856982699\\
18.1833337046712	0.17052866170274\\
18.212165554776	0.1700635359945\\
18.2409974048809	0.169599479590788\\
18.2700621491713	0.169132756688909\\
18.2991268934616	0.1686671198068\\
18.328191637752	0.168202568585993\\
18.3572563820424	0.167739102640785\\
18.3865589068877	0.167272943261795\\
18.4158614317331	0.166807886118634\\
18.4451639565785	0.166343930733037\\
18.4744664814239	0.165881076600075\\
18.5040118216459	0.16541550146561\\
18.5335571618679	0.164951044798749\\
18.56310250209	0.164487706000636\\
18.592647842312	0.164025484446332\\
18.622441195706	0.16356051354449\\
18.6522345491	0.163096677366889\\
18.682027902494	0.162633975193216\\
18.711821255888	0.162172406277669\\
18.74186798467	0.161708058955749\\
18.771914713452	0.161244862648557\\
18.801961442234	0.160782816513382\\
18.8320081710161	0.16032191968262\\
18.8623138179204	0.159858214516479\\
18.8926194648247	0.159395676699333\\
18.922925111729	0.158934305265045\\
18.9532307586333	0.15847409922319\\
18.9838010512769	0.158011054078012\\
19.0143713439205	0.157549192669835\\
19.0449416365641	0.157088513907985\\
19.0755119292076	0.15662901667811\\
19.1063527942164	0.156166648626118\\
19.1371936592252	0.155705480763607\\
19.1680345242339	0.15524551187416\\
19.1988753892427	0.154786740718303\\
19.2299929618915	0.15432506603507\\
19.2611105345403	0.153864608069417\\
19.2922281071891	0.153405365477882\\
19.3233456798379	0.152947336894575\\
19.3547463174076	0.152486371008786\\
19.3861469549774	0.152026638456125\\
19.4175475925471	0.151568137764682\\
19.4489482301169	0.151110867440754\\
19.4806385226382	0.150650624933263\\
19.5123288151595	0.150191632474196\\
19.5440191076808	0.149733888461677\\
19.5757094002021	0.149277391272688\\
19.6076961882754	0.148817885782969\\
19.6396829763487	0.148359647170015\\
19.671669764422	0.14790267370035\\
19.7036565524953	0.147446963620022\\
19.7359469398787	0.146988207842406\\
19.768237327262	0.146530735896805\\
19.8005277146453	0.146074545916391\\
19.8328181020287	0.145619636014534\\
19.8654194755566	0.145161641602166\\
19.8980208490846	0.144704948118971\\
19.9306222226126	0.144249553562878\\
19.9632235961405	0.143795455912702\\
19.9961436388709	0.143338233492925\\
20.0290636816012	0.142882329256943\\
20.0619837243316	0.142427741065415\\
20.0949037670619	0.14197446676058\\
20.1281504830269	0.141518025802194\\
20.1613971989918	0.14106292045671\\
20.1946439149567	0.140609148445331\\
20.2278906309217	0.140156707471556\\
20.2614723645969	0.139701056245672\\
20.2950540982721	0.139246758254312\\
20.3286358319473	0.138793811076874\\
20.3622175656225	0.138342212275782\\
20.3961430262838	0.137887357768135\\
20.4300684869452	0.137433874328545\\
20.4639939476066	0.136981759392079\\
20.497919408268	0.136531010377581\\
20.5321976950164	0.136076958213533\\
20.5664759817649	0.135624295183764\\
20.6007542685134	0.135173018576299\\
20.6350325552618	0.1347231256637\\
20.6696731863085	0.134269879999661\\
20.7043138173553	0.133818041791286\\
20.738954448402	0.133367608176628\\
20.7735950794487	0.132918576279064\\
20.8086080231198	0.13246613969785\\
20.8436209667909	0.13201512917309\\
20.878633910462	0.13156554168972\\
20.9136468541332	0.131117374218806\\
20.949042565344	0.130665747582995\\
20.9844382765548	0.130215565910258\\
21.0198339877656	0.129766826029021\\
21.0552296989764	0.129319524754664\\
21.0910191546516	0.128868707106104\\
21.1268086103268	0.128419353661522\\
21.162598066002	0.127971461089174\\
21.1983875216773	0.127525026045121\\
21.234582265905	0.127075014415341\\
21.2707770101326	0.126626486595951\\
21.3069717543603	0.126179439091079\\
21.343166498588	0.12573386839354\\
21.3797786897193	0.125284657631442\\
21.4163908808506	0.12483695068572\\
21.453003071982	0.124390743892108\\
21.4896152631133	0.123946033575927\\
21.5266577262528	0.123497616153184\\
21.5637001893923	0.12305072298968\\
21.6007426525319	0.122605350248122\\
21.6377851156714	0.122161494081743\\
21.6752714017348	0.121713859867902\\
21.7127576877983	0.121267770834002\\
21.7502439738617	0.12082322296471\\
21.7877302599251	0.120380212236188\\
21.8256747161126	0.119933348193819\\
21.8636191723001	0.11948805077542\\
21.9015636284876	0.119044315782171\\
21.9395080846751	0.118602139007748\\
21.9779259262067	0.118156028938286\\
22.0163437677384	0.117711507510155\\
22.0547616092701	0.117268570335117\\
22.0931794508017	0.116827213018478\\
22.1320868496398	0.116381837180327\\
22.1709942484778	0.115938072630002\\
22.2099016473158	0.115495914783389\\
22.2488090461539	0.115055359050999\\
22.2882232273236	0.114610693774996\\
22.3276374084934	0.114167663124556\\
22.3670515896632	0.113726262312625\\
22.4064657708329	0.113286486547905\\
22.4464051256478	0.112842503755477\\
22.4863444804626	0.112400179687246\\
22.5262838352775	0.111959509345484\\
22.5662231900924	0.111520487729402\\
22.6067074061995	0.111077154373666\\
22.6471916223066	0.110635504680106\\
22.6876758384138	0.110195533431821\\
22.7281600545209	0.109757235410092\\
22.7692102626757	0.109314512860207\\
22.8102604708306	0.108873499838262\\
22.8513106789854	0.108434190898826\\
22.8923608871403	0.107996580595951\\
22.9339998357608	0.107554423866854\\
22.9756387843813	0.107114003559798\\
23.0172777330018	0.106675313990476\\
23.0589166816223	0.106238349475439\\
23.1011689388352	0.10579670633816\\
23.143421196048	0.105356827660086\\
23.1856734532609	0.104918707506568\\
23.2279257104737	0.104482339945257\\
23.2707449436263	0.104041898790921\\
23.3135641767789	0.103603245321071\\
23.3563834099314	0.103166373369506\\
23.399202643084	0.102731276773833\\
23.4425351018712	0.102292766275955\\
23.4858675606585	0.101856061262585\\
23.5292000194457	0.101421155359914\\
23.5725324782329	0.100988042199476\\
23.6163923812599	0.100551476271452\\
23.6602522842869	0.100116733883224\\
23.704112187314	0.0996838084485655\\
23.747972090341	0.0992526933881746\\
23.7923730104802	0.0988180977594291\\
23.8367739306194	0.0983853438968916\\
23.8811748507586	0.0979544249973548\\
23.9255757708978	0.0975253342661745\\
23.9705318120247	0.0970927344965931\\
24.0154878531517	0.0966619949036492\\
24.0604438942787	0.0962331084623396\\
24.1053999354057	0.0958060681579141\\
24.1509257391711	0.0953754898552104\\
24.1964515429365	0.0949467903352574\\
24.2419773467019	0.0945199623461755\\
24.2875031504673	0.0940949986480839\\
24.3336139249026	0.0936664674795499\\
24.3797246993379	0.0932398339077079\\
24.4258354737732	0.0928150904484404\\
24.4719462482085	0.0923922296314368\\
24.518657797687	0.0919657713411552\\
24.5653693471654	0.091541229681944\\
24.6120808966439	0.0911185969317909\\
24.6587924461223	0.0906978653843623\\
24.7061212033434	0.0902735058106846\\
24.7534499605645	0.0898510821359996\\
24.8007787177855	0.0894305863944291\\
24.8481074750066	0.0890120106377129\\
24.8960705351928	0.0885897757308398\\
24.944033595379	0.0881694962380671\\
24.9919966555653	0.0877511639433448\\
25.0399597157515	0.0873347706502545\\
25.0885748730688	0.0869146864914475\\
25.1371900303861	0.0864965775232338\\
25.1858051877034	0.0860804352727323\\
25.2344203450208	0.0856662512887845\\
25.2837061315614	0.085248344109904\\
25.332991918102	0.0848324321742945\\
25.3822777046427	0.0844185067452075\\
25.4315634911833	0.0840065591097906\\
25.4815392186928	0.0835908553135082\\
25.5315149462023	0.0831771671047976\\
25.5814906737117	0.0827654854756056\\
25.6314664012212	0.082355801444037\\
25.6821522057685	0.0819423276250628\\
25.7328380103157	0.0815308900457019\\
25.783523814863	0.0811214794187319\\
25.8342096194102	0.0807140864854452\\
25.8856265092156	0.0803028694531365\\
25.9370433990211	0.07989370963706\\
25.9884602888265	0.0794865974625047\\
26.0398771786319	0.0790815233857315\\
26.0920470854784	0.0786725901869542\\
26.144216992325	0.0782657355232165\\
26.1963868991715	0.0778609495235137\\
26.2485568060181	0.0774582223503776\\
26.3015026403881	0.0770516002940746\\
26.3544484747581	0.0766470784522385\\
26.4073943091281	0.0762446466482464\\
26.4603401434981	0.0758442947416918\\
26.5140858539001	0.0754400114242292\\
26.5678315643021	0.0750378503805907\\
26.6215772747041	0.0746378011186559\\
26.6753229851061	0.0742398531853248\\
26.7298936218998	0.0738379365172639\\
26.7844642586936	0.0734381645825617\\
26.8390348954873	0.0730405265631278\\
26.893605532281	0.072645011682828\\
26.9490273164448	0.0722454899170071\\
27.0044491006085	0.0718481357655078\\
27.0598708847722	0.0714529380731599\\
27.1152926689359	0.0710598857298282\\
27.1715930662283	0.0706627874909405\\
27.2278934635207	0.0702678801910874\\
27.2841938608131	0.0698751523262278\\
27.3404942581055	0.0694845924405868\\
27.3972810276665	0.0690928431203327\\
27.4540677972275	0.0687032761377278\\
27.5108545667884	0.0683158798868341\\
27.5676413363494	0.0679306428118702\\
27.6246835621125	0.0675458349084529\\
27.6817257878755	0.0671631824309446\\
27.7387680136386	0.0667826738197663\\
27.7958102394016	0.0664042975664976\\
27.8531117250382	0.0660263369381029\\
27.9104132106748	0.0656505049712132\\
27.9677146963114	0.0652767901556403\\
28.025016181948	0.0649051810332829\\
28.0825792753208	0.0645339839774108\\
28.1401423686935	0.0641648888651242\\
28.1977054620663	0.063797884239309\\
28.2552685554391	0.0634329586957921\\
28.3130956457549	0.0630684414838479\\
28.3709227360708	0.062705999558175\\
28.4287498263866	0.0623456215181631\\
28.4865769167025	0.0619872960169257\\
28.5446704260258	0.0616293750069651\\
28.602763935349	0.0612735026991277\\
28.6608574446723	0.0609196677525006\\
28.7189509539956	0.0605678588806107\\
28.7773133374	0.060216450567141\\
28.8356757208045	0.059867064456519\\
28.8940381042089	0.0595196892705004\\
28.9524004876134	0.0591743137859332\\
29.0110342352743	0.0588293348349972\\
29.0696679829352	0.0584863516835368\\
29.1283017305961	0.0581453531187361\\
29.186935478257	0.0578063279834646\\
29.2458431140456	0.0574676952861799\\
29.3047507498342	0.0571310320910557\\
29.3636583856228	0.0567963272532655\\
29.4225660214114	0.0564635696842052\\
29.4817501049045	0.0561312003868851\\
29.5409341883977	0.0558007744100196\\
29.6001182718908	0.0554722806791449\\
29.659302355384	0.0551457081765024\\
29.7187654815208	0.0548195197222532\\
29.7782286076576	0.0544952485311907\\
29.8376917337944	0.0541728836014093\\
29.8971548599312	0.0538524139881413\\
29.9568996607722	0.0535323241440945\\
30.0166444616132	0.0532141256386846\\
30.0763892624543	0.0528978075445911\\
30.1361340632953	0.052583358992017\\
30.1961632086509	0.0522692858810846\\
30.2561923540065	0.0519570783246439\\
30.316221499362	0.0516467254718283\\
30.3762506447176	0.0513382165296351\\
30.436566841334	0.0510300786674595\\
30.4968830379503	0.050723780723074\\
30.5571992345667	0.0504193119237856\\
30.617515431183	0.0501166615550638\\
30.6781214246657	0.0498143778702777\\
30.7387274181483	0.0495139086206179\\
30.799333411631	0.0492152431131442\\
30.8599394051136	0.0489183707133366\\
30.9208379817972	0.0486218605671254\\
30.9817365584807	0.0483271395335823\\
31.0426351351642	0.0480341970009647\\
31.1035337118477	0.0477430224161711\\
31.1647276970968	0.0474522056377892\\
31.2259216823458	0.0471631528153783\\
31.2871156675948	0.046875853419714\\
31.3483096528438	0.0465902969803985\\
31.4098019130839	0.0463050938820847\\
31.4712941733241	0.0460216297540013\\
31.5327864335642	0.0457398941506444\\
31.5942786938043	0.0454598766854891\\
31.656072138367	0.0451802080775701\\
31.7178655829298	0.0449022536299379\\
31.7796590274925	0.0446260029819003\\
31.8414524720552	0.0443514458318645\\
31.903550051626	0.0440772330520602\\
31.9656476311968	0.0438047098027155\\
32.0277452107676	0.0435338658089366\\
32.0898427903383	0.0432646908550199\\
32.1522474993513	0.0429958557778903\\
32.2146522083642	0.0427286857855521\\
32.2770569173772	0.042463170689798\\
32.3394616263901	0.0421993003616745\\
32.4021765035977	0.0419357654154711\\
32.4648913808053	0.0416738712962329\\
32.527606258013	0.0414136079032348\\
32.5903211352206	0.0411549651950425\\
32.6533492654047	0.040896653372528\\
32.7163773955889	0.0406399583103966\\
32.7794055257731	0.0403848699961131\\
32.8424336559572	0.0401313784764463\\
32.9057781690946	0.0398782133552533\\
32.969122682232	0.0396266411221125\\
33.0324671953693	0.0393766518533053\\
33.0958117085067	0.0391282356844071\\
33.1594757818323	0.0388801414340439\\
33.2231398551579	0.0386336163964348\\
33.2868039284835	0.0383886507372277\\
33.3504680018091	0.0381452346813323\\
33.4144548595075	0.0379021360788979\\
33.478441717206	0.0376605832134043\\
33.5424285749044	0.0374205663403436\\
33.6064154326028	0.0371820757744173\\
33.6707283499096	0.0369438982024084\\
33.7350412672163	0.0367072430933637\\
33.7993541845231	0.0364721007930284\\
33.8636671018298	0.0362384617062859\\
33.9283094023156	0.0360051311758732\\
33.9929517028013	0.0357733000382193\\
34.057594003287	0.0355429587296684\\
34.1222363037727	0.0353140977456136\\
34.1872113622324	0.0350855408973816\\
34.2521864206922	0.0348584605772776\\
34.3171614791519	0.0346328473125311\\
34.3821365376116	0.0344086916893149\\
34.447447781095	0.0341848358006727\\
34.5127590245783	0.0339624337826963\\
34.5780702680616	0.0337414762537306\\
34.643381511545	0.0335219538909421\\
34.7090324201268	0.0333027268837025\\
34.7746833287086	0.033084931298219\\
34.8403342372905	0.0328685578441295\\
34.9059851458723	0.0326535972897569\\
34.9719792533529	0.03243892773715\\
35.0379733608335	0.0322256673671259\\
35.1039674683141	0.0320138069807437\\
35.1699615757948	0.0318033374375982\\
35.2363024710846	0.0315931545683351\\
35.3026433663745	0.0313843588532426\\
35.3689842616644	0.0311769411848841\\
35.4353251569542	0.0309708925141954\\
35.5020164861975	0.0307651262138568\\
35.5687078154407	0.0305607252509422\\
35.6353991446839	0.0303576806095582\\
35.7020904739271	0.0301559833320087\\
35.76913594038	0.029954564151126\\
35.8361814068328	0.0297544887032842\\
35.9032268732857	0.0295557480641325\\
35.9702723397385	0.0293583333673307\\
36.0376757049608	0.0291611925245332\\
36.1050790701831	0.0289653740233161\\
36.1724824354054	0.0287708690308321\\
36.2398858006277	0.0285776687720481\\
36.3076508877251	0.0283847381509651\\
36.3754159748225	0.0281931086934294\\
36.4431810619199	0.0280027716580229\\
36.5109461490173	0.0278137183609351\\
36.5790768402747	0.0276249305231105\\
36.6472075315322	0.0274374228844671\\
36.7153382227897	0.0272511867949097\\
36.7834689140472	0.0270662136617344\\
36.8519691566203	0.0268815018358666\\
36.9204693991935	0.0266980494587742\\
36.9889696417667	0.0265158479715465\\
37.0574698843398	0.0263348888724397\\
37.1263436888854	0.0261541869637076\\
37.1952174934311	0.0259747239673853\\
37.2640912979767	0.0257964914155799\\
37.3329651025223	0.0256194808973328\\
37.4022165438372	0.0254427234900986\\
37.4714679851522	0.025267184672904\\
37.5407194264671	0.0250928560686807\\
37.609970867782	0.024919729357055\\
37.6796040887873	0.0247468517087135\\
37.7492373097926	0.0245751725419909\\
37.8188705307979	0.0244046835704249\\
37.8885037518033	0.0242353765640005\\
37.9585229632751	0.024066314609883\\
38.0285421747469	0.0238984312427167\\
38.0985613862187	0.0237317182664035\\
38.1685805976905	0.0235661675410393\\
38.2389900798134	0.0234008578941039\\
38.3093995619363	0.0232367071529308\\
38.3798090440592	0.0230737072115226\\
38.4502185261821	0.0229118500198161\\
38.5209622984945	0.0227503671991258\\
38.591706070807	0.0225900218819899\\
38.6624498431194	0.0224308060726302\\
38.7331936154318	0.0222727118307489\\
38.8042656863826	0.0221150053597762\\
38.8753377573333	0.0219584149920867\\
38.9464098282841	0.0218029328443683\\
39.0174818992348	0.0216485510883038\\
39.0888909542047	0.0214945377290905\\
39.1603000091745	0.0213416195443194\\
39.2317090641443	0.0211897887604163\\
39.3031181191142	0.0210390376583315\\
39.3748725127202	0.0208886373081231\\
39.4466269063262	0.0207393116470363\\
39.5183812999322	0.0205910530087338\\
39.5901356935382	0.020443853780945\\
39.662243568804	0.0202969890345548\\
39.7343514440698	0.0201511789128529\\
39.8064593193355	0.020006415854414\\
39.8785671946013	0.0198626923514337\\
39.95103651206	0.019719288293001\\
40.0235058295186	0.0195769191952447\\
40.0959751469772	0.0194355775994849\\
40.1684444644358	0.0192952561002276\\
40.2412830193083	0.0191552401353303\\
40.3141215741807	0.0190162398487182\\
40.3869601290532	0.0188782478824356\\
40.4597986839256	0.018741256931288\\
40.5330141309314	0.0186045586173037\\
40.6062295779373	0.0184688570637908\\
40.6794450249431	0.018334145011627\\
40.752660471949	0.0182004152540358\\
40.8262603410054	0.0180669661611227\\
40.8998602100619	0.0179344952597792\\
40.9734600791183	0.0178029953879457\\
41.0470599481748	0.0176724594355011\\
41.1210516636287	0.0175421930117619\\
41.1950433790826	0.0174128865453105\\
41.2690350945364	0.0172845329694876\\
41.3430268099903	0.0171571252691729\\
41.4174177047231	0.017029976725388\\
41.4918085994559	0.0169037702260774\\
41.5661994941887	0.016778498798419\\
41.6405903889215	0.0166541555207375\\
41.7153877210576	0.0165300617194128\\
41.7901850531936	0.0164068923591898\\
41.8649823853296	0.0162846405596109\\
41.9397797174656	0.0161632994909792\\
42.0149906816255	0.0160421988530005\\
42.0902016457854	0.0159220053510732\\
42.1654126099454	0.0158027121957142\\
42.2406235741053	0.0156843126478218\\
42.316255315397	0.0155661450615886\\
42.3918870566888	0.0154488675944716\\
42.4675187979805	0.015332473546648\\
42.5431505392722	0.0152169562683023\\
42.6192101664345	0.0151016630072433\\
42.6952697935968	0.0149872431270977\\
42.771329420759	0.0148736900164575\\
42.8473890479213	0.0147609971135529\\
42.9238836429179	0.014648520765266\\
43.0003782379145	0.0145369013297238\\
43.0768728329111	0.0144261322827509\\
43.1533674279077	0.0143162071494459\\
43.2303040564086	0.0142064915486593\\
43.3072406849096	0.0140976166544416\\
43.3841773134105	0.0139895760287258\\
43.4611139419115	0.0138823632823593\\
43.5384996649489	0.0137753534472804\\
43.6158853879863	0.0136691683671809\\
43.6932711110238	0.0135638016890316\\
43.7706568340612	0.0134592471083613\\
43.8484987166093	0.0133548891861251\\
43.9263405991575	0.0132513403149832\\
44.0041824817057	0.0131485942259212\\
44.0820243642538	0.0130466446981303\\
44.1603294838729	0.0129448859140818\\
44.2386346034919	0.0128439207185526\\
44.3169397231109	0.0127437429255651\\
44.39524484273	0.0126443463969986\\
44.4740202988846	0.0125451350068203\\
44.5527957550393	0.0124467019779694\\
44.631571211194	0.0123490412065684\\
44.7103466673487	0.0122521466362502\\
44.7895995895968	0.0121554318830841\\
44.8688525118449	0.012059480493921\\
44.948105434093	0.0119642864460834\\
45.0273583563411	0.0118698437640616\\
45.1070959123988	0.0117755758389066\\
45.1868334684566	0.0116820565051607\\
45.2665710245144	0.011589279820481\\
45.3463085805721	0.0114972398893524\\
45.4265379845256	0.0114053698942014\\
45.5067673884792	0.0113142339378126\\
45.5869967924327	0.0112238261573441\\
45.6672261963862	0.0111341407364435\\
45.7479547148514	0.0110446206524354\\
45.8286832333165	0.0109558202699974\\
45.9094117517817	0.0108677338049836\\
45.9901402702469	0.0107803555194021\\
46.0713752310891	0.0106931381743752\\
46.1526101919314	0.0106066264050041\\
46.2338451527736	0.0105208145050601\\
46.3150801136159	0.0104356968141352\\
46.3968289125376	0.0103507358549187\\
46.4785777114593	0.0102664665527976\\
46.560326510381	0.0101828832787067\\
46.6420753093027	0.0100999804490691\\
46.7243454163926	0.0100172303157422\\
46.8066155234824	0.00993515812461587\\
46.8888856305723	0.00985375832305621\\
46.9711557376621	0.00977302540358781\\
47.0539547042701	0.00969244130522011\\
47.136753670878	0.00961252163435763\\
47.2195526374859	0.00953326091408621\\
47.3023516040939	0.00945465371232191\\
47.3856870701744	0.00937619160401383\\
47.4690225362549	0.00929838060546114\\
47.5523580023354	0.0092212153147767\\
47.6356934684159	0.0091446903745766\\
47.7195731680798	0.0090683069378295\\
47.8034528677438	0.00899256148694782\\
47.8873325674077	0.00891744869439579\\
47.9712122670716	0.0088429632768153\\
48.0556440359198	0.00876861589960235\\
48.1400758047679	0.00869489357532088\\
48.2245075736161	0.00862179105012655\\
48.3089393424642	0.00854930311402834\\
48.3939311234603	0.00847694987359231\\
48.4789229044563	0.00840520894135161\\
48.5639146854524	0.00833407513650779\\
48.6489064664485	0.00826354332179252\\
48.7344663172648	0.00819314296736795\\
48.820026168081	0.00812334236199675\\
48.9055860188973	0.00805413639729391\\
48.9911458697136	0.00798552000808233\\
49.0772819670679	0.00791703194729312\\
49.1634180644223	0.00784913125950601\\
49.2495541617767	0.00778181290812918\\
49.3356902591311	0.00771507189945753\\
49.4224109073406	0.00764845618219986\\
49.5091315555501	0.00758241564262917\\
49.5958522037596	0.0075169453153375\\
49.6825728519691	0.00745204027748329\\
49.769886487838	0.00738725758383819\\
49.857200123707	0.00732303805103374\\
49.944513759576	0.00725937678424665\\
50.0318273954449	0.00719626893090053\\
50.1197425956162	0.00713328055794548\\
50.2076577957875	0.00707084350529745\\
50.2955729959588	0.00700895294812779\\
50.3834881961302	0.006947604103536\\
50.4720136825001	0.00688637195446144\\
50.5605391688701	0.00682567945939383\\
50.64906465524	0.00676552186291759\\
50.73759014161	0.00670589445122728\\
50.8267347882915	0.00664638102403622\\
50.915879434973	0.00658739575679509\\
51.0050240816545	0.00652893396292757\\
51.094168728336	0.00647099099715006\\
51.1839415683769	0.00641315937429395\\
51.2737144084178	0.00635584458766036\\
51.3634872484587	0.00629904201894539\\
51.4532600884996	0.00624274709082098\\
51.5436703203087	0.00618656092969492\\
51.6340805521178	0.00613088044954613\\
51.7244907839268	0.00607570109978263\\
51.8149010157359	0.00602101837047195\\
51.9059580087448	0.00596644189494575\\
51.9970150017537	0.00591236011183722\\
52.0880719947626	0.00585876853771182\\
52.1791289877716	0.00580566272947868\\
52.270842290771	0.00575266072031911\\
52.3625555937704	0.00570014257998112\\
52.4542688967698	0.0056481038916383\\
52.5459821997693	0.0055965402784926\\
52.6383615465513	0.00554507806600232\\
52.7307408933333	0.00549408906201666\\
52.8231202401153	0.00544356891577305\\
52.9154995868973	0.00539351331622227\\
53.0085549039533	0.00534355677210784\\
53.1016102210094	0.00529406293783032\\
53.1946655380654	0.00524502752815098\\
53.2877208551214	0.00519644629722989\\
53.3814622684928	0.00514796182744177\\
53.4752036818643	0.00509992972888165\\
53.5689450952358	0.00505234578129876\\
53.6626865086073	0.00500520580352685\\
53.7571243510415	0.00495816034133276\\
53.8515621934757	0.0049115570702637\\
53.9460000359099	0.00486539183452456\\
54.0404378783442	0.0048196605170909\\
54.1355826971555	0.00477402151618963\\
54.2307275159668	0.00472881468330244\\
54.325872334778	0.00468403592656098\\
54.4210171535893	0.00463968119255395\\
54.5168797176294	0.00459541662100952\\
54.6127422816695	0.00455157434987212\\
54.7086048457095	0.00450815035067431\\
54.8044674097496	0.00446514063309236\\
54.9010587178797	0.00442221896687616\\
54.9976500260098	0.00437970988751067\\
55.0942413341398	0.00433760942940617\\
55.1908326422699	0.00429591366480362\\
55.2881639308243	0.00425430388215577\\
55.3854952193786	0.00421309712542096\\
55.4828265079329	0.00417228949136663\\
55.5801577964873	0.00413187711427803\\
55.6782405480441	0.00409154868977865\\
55.776323299601	0.00405161388146948\\
55.8744060511579	0.00401206884795674\\
55.9724888027147	0.00397290978505185\\
56.0713347542828	0.00393383268425837\\
56.1701807058509	0.00389513993976209\\
56.2690266574189	0.00385682777149173\\
56.367872608987	0.00381889243626879\\
56.4674937609263	0.00378103711031494\\
56.5671149128656	0.0037435570292309\\
56.6667360648049	0.00370644847375344\\
56.7663572167441	0.00366970776119992\\
56.8667658411891	0.00363304514184616\\
56.9671744656341	0.00359674880304933\\
57.0675830900791	0.00356081508584158\\
57.1679917145241	0.00352524036752361\\
57.2692003658787	0.00348974186167447\\
57.3704090172333	0.00345460081785752\\
57.4716176685879	0.00341981363688904\\
57.5728263199425	0.00338537675554203\\
57.6748478434472	0.00335101424074144\\
57.7768693669519	0.00331700051391645\\
57.8788908904565	0.00328333203515771\\
57.9809124139612	0.00325000530020089\\
58.0837599560178	0.00321675111969234\\
58.1866074980744	0.00318383719626681\\
58.2894550401309	0.00315126004878082\\
58.3923025821875	0.00311901623142431\\
58.4959896007833	0.00308684318942249\\
58.5996766193791	0.00305500201548344\\
58.7033636379749	0.00302348928672215\\
58.8070506565707	0.0029923016152756\\
58.911590931702	0.00296118297244937\\
59.0161312068333	0.00293038794925523\\
59.1206714819646	0.00289991318056028\\
59.2252117570959	0.00286975533594234\\
59.3306194021433	0.00283966480480324\\
59.4360270471907	0.00280988978418381\\
59.541434692238	0.0027804269661979\\
59.6468423372854	0.00275127307735885\\
59.7351317529641	0.00272708917638785\\
59.8234211686427	0.00270311785337353\\
59.9117105843214	0.00267935723985865\\
60	0.00265580548369465\\
};
\addlegendentry{Infetti}

\addplot [color=mycolor3, line width=2.0pt]
  table[row sep=crcr]{%
0	0\\
0.00199053585276749	1.18268676822518e-11\\
0.00398107170553497	9.45870381888635e-11\\
0.00597160755830246	3.19135881054745e-10\\
0.00796214341106994	7.56244247397302e-10\\
0.0123567669950473	2.82487874236936e-09\\
0.0167513905790247	7.03316243305228e-09\\
0.0211460141630021	1.41382679042843e-08\\
0.0255406377469794	2.4895348175285e-08\\
0.0299388563931551	4.0071947192466e-08\\
0.0343370750393307	6.04138179409832e-08\\
0.0387352936855064	8.66718883575852e-08\\
0.043133512331682	1.19595055569276e-07\\
0.0475471725191068	1.60085729869166e-07\\
0.0519608327065316	2.08793203726817e-07\\
0.0563744928939564	2.66468120257686e-07\\
0.0607881530813811	3.33859058778606e-07\\
0.0652173199794049	4.1200538178731e-07\\
0.0696464868774286	5.01440209468244e-07\\
0.0740756537754524	6.0291377108345e-07\\
0.0785048206734761	7.17174200368608e-07\\
0.0829496187548012	8.45443381288955e-07\\
0.0873944168361263	9.88092996188873e-07\\
0.0918392149174515	1.14587276190813e-06\\
0.0962840129987766	1.31953026708491e-06\\
0.100744568101794	1.51051597929865e-06\\
0.105205123204811	1.71899422092732e-06\\
0.109665678307829	1.94571409519882e-06\\
0.114126233410846	2.19142254549829e-06\\
0.118602673075784	2.45784577230395e-06\\
0.123079112740721	2.74489373242211e-06\\
0.127555552405659	3.05331481481662e-06\\
0.132031992070596	3.38385521609784e-06\\
0.136524445545197	3.73856512124059e-06\\
0.141016899019798	4.11705201686281e-06\\
0.1455093524944	4.52006347543719e-06\\
0.150001805969001	4.94834484434845e-06\\
0.154510404254387	5.40431967528308e-06\\
0.159019002539772	5.8872437526294e-06\\
0.163527600825158	6.39786372901713e-06\\
0.168036199110544	6.93692400001553e-06\\
0.172561074936287	7.5072720138803e-06\\
0.17708595076203	8.10776147478821e-06\\
0.181610826587773	8.73913801193327e-06\\
0.186135702413515	9.40214496444403e-06\\
0.190676990405524	1.01001054061685e-05\\
0.195218278397533	1.08314193895812e-05\\
0.199759566389542	1.15968314153541e-05\\
0.204300854381551	1.23970836615895e-05\\
0.208858690959197	1.32360277467236e-05\\
0.213416527536843	1.41115573110634e-05\\
0.217974364114488	1.50244156205646e-05\\
0.222532200692134	1.5975343586481e-05\\
0.227106724202768	1.69687755456906e-05\\
0.231681247713401	1.80020447155807e-05\\
0.236255771224035	1.9075893021052e-05\\
0.240830294734669	2.01910599991932e-05\\
0.245421645441331	2.13526181379454e-05\\
0.250012996147992	2.2557284965767e-05\\
0.254604346854654	2.38058009579289e-05\\
0.259195697561315	2.509890416967e-05\\
0.263804017673384	2.64423619676713e-05\\
0.268412337785452	2.78322196329488e-05\\
0.273020657897521	2.92692160824897e-05\\
0.27762897800959	3.07540877807822e-05\\
0.282254411746462	3.22933550358888e-05\\
0.286879845483334	3.3882332992138e-05\\
0.291505279220206	3.55217588982066e-05\\
0.296130712957079	3.72123675176513e-05\\
0.300773406613671	3.89614915351959e-05\\
0.305416100270264	4.07636567010808e-05\\
0.310058793926856	4.26195984847642e-05\\
0.314701487583449	4.453004983736e-05\\
0.319361589521735	4.65032166339294e-05\\
0.324021691460022	4.85327745977787e-05\\
0.328681793398309	5.06194573068192e-05\\
0.333341895336596	5.27639957878566e-05\\
0.338019556071109	5.49755313670871e-05\\
0.342697216805623	5.72468276654392e-05\\
0.347374877540136	5.95786162557385e-05\\
0.35205253827465	6.19716261270574e-05\\
0.356747910447269	6.44359977509893e-05\\
0.361443282619888	6.69635191399001e-05\\
0.366138654792507	6.95549197469352e-05\\
0.370834026965126	7.22109264088364e-05\\
0.375547265516765	7.49427440066269e-05\\
0.380260504068404	7.77411198751464e-05\\
0.384973742620042	8.06067812324099e-05\\
0.389686981171681	8.35404526473215e-05\\
0.394418243246227	8.6554470136794e-05\\
0.399149505320772	8.96384738276164e-05\\
0.403880767395317	9.27931885853417e-05\\
0.408612029469863	9.60193365939872e-05\\
0.41336147459102	9.93304533949682e-05\\
0.418110919712177	0.000102715003721942\\
0.422860364833334	0.000106173709970563\\
0.427609809954491	0.000109707291821652\\
0.432377599961091	0.000113330554275638\\
0.437145389967691	0.000117030716944083\\
0.44191317997429	0.000120808499632391\\
0.44668096998089	0.000124664619398951\\
0.451467269141767	0.000128615222347634\\
0.456253568302644	0.000132646211534851\\
0.461039867463521	0.000136758304055571\\
0.465826166624398	0.000140952214224751\\
0.47063114173014	0.000145245502641275\\
0.475436116835881	0.000149622682632739\\
0.480241091941623	0.000154084468461525\\
0.485046067047365	0.000158631571577061\\
0.489869887370685	0.000163283042140408\\
0.494693707694006	0.000168021928902846\\
0.499517528017326	0.000172848943170887\\
0.504341348340647	0.000177764793404953\\
0.509184185682064	0.000182790096178572\\
0.51402702302348	0.000187906359040702\\
0.518869860364897	0.000193114290216916\\
0.523712697706314	0.000198414595053558\\
0.528574726619363	0.000203829535416084\\
0.533436755532412	0.000209338999007842\\
0.53829878444546	0.000214943690846937\\
0.543160813358509	0.000220644313039683\\
0.548042211039569	0.000226464853348833\\
0.55292360872063	0.000232383499261329\\
0.55780500640169	0.000238400952460435\\
0.56268640408275	0.000244517911684295\\
0.56758735048985	0.000250760173216085\\
0.572488296896949	0.000257104141946433\\
0.577389243304049	0.000263550516093957\\
0.582290189711149	0.00027009999089962\\
0.587210867601438	0.000276780255767198\\
0.592131545491728	0.000283565848638045\\
0.597052223382018	0.000290457464135665\\
0.601972901272307	0.000297455793872596\\
0.606913496388328	0.000304590507212855\\
0.611854091504349	0.000311834188567691\\
0.61679468662037	0.000319187528832964\\
0.621735281736391	0.000326651215860788\\
0.626695982628195	0.000334256987589914\\
0.631656683519999	0.000341975386541179\\
0.636617384411802	0.000349807099748693\\
0.641578085303606	0.000357752811169749\\
0.646559083672327	0.000365846418531209\\
0.651540082041047	0.000374056331522935\\
0.656521080409768	0.00038238323318171\\
0.661502078778488	0.000390827803434392\\
0.666503569383853	0.000399426193013988\\
0.671505059989219	0.000408144585833606\\
0.676506550594584	0.000416983660795116\\
0.68150804119995	0.00042594409365693\\
0.686530221876942	0.000435064383582665\\
0.691552402553934	0.000444308393560324\\
0.696574583230927	0.000453676798217721\\
0.701596763907919	0.000463170269006059\\
0.706639835866396	0.000472829751734273\\
0.711682907824873	0.000482616690544425\\
0.71672597978335	0.000492531755648791\\
0.721769051741827	0.000502575614050333\\
0.72683321952555	0.000512791758676585\\
0.731897387309272	0.000523139114645837\\
0.736961555092995	0.000533618347612449\\
0.742025722876718	0.00054423011998796\\
0.747111194287107	0.0005550205744945\\
0.752196665697496	0.00056594601484564\\
0.757282137107885	0.000577007101992721\\
0.762367608518274	0.00058820449361113\\
0.767474594900923	0.000599587087914456\\
0.772581581283573	0.000611108461827721\\
0.777688567666222	0.000622769271452394\\
0.782795554048871	0.000634570169581048\\
0.78792427039055	0.000646562918351648\\
0.793052986732229	0.00065869825978947\\
0.798181703073909	0.000670976844997669\\
0.803310419415588	0.000683399321736815\\
0.808461084325784	0.000696020427037293\\
0.81361174923598	0.000708787957381274\\
0.818762414146177	0.000721702558722399\\
0.823913079056373	0.000734764873638716\\
0.82908591482452	0.000748032727781803\\
0.834258750592667	0.000761450858692864\\
0.839431586360813	0.000775019907022789\\
0.84460442212896	0.000788740510014038\\
0.849799655005291	0.000802673699098654\\
0.854994887881621	0.000816761036067998\\
0.860190120757951	0.000831003156115201\\
0.865385353634282	0.000845400690992024\\
0.870603213699137	0.000860017997514645\\
0.875821073763992	0.000874793342467206\\
0.881038933828848	0.000889727355427736\\
0.886256793893703	0.000904820662499372\\
0.891497515445601	0.000920141069434829\\
0.896738236997499	0.000935623424831927\\
0.901978958549397	0.000951268352493876\\
0.907219680101295	0.000967076472715583\\
0.912483501408722	0.000983119165970638\\
0.917747322716149	0.000999327737247601\\
0.923011144023576	0.00101570280441268\\
0.928274965331004	0.00103224498179024\\
0.933562128954513	0.00104902935432999\\
0.938849292578023	0.00106598355404102\\
0.944136456201532	0.0010831081926876\\
0.949423619825042	0.00110040387845893\\
0.954734372782572	0.00111794953419409\\
0.960045125740102	0.00113566898590811\\
0.965355878697632	0.00115356283909641\\
0.970666631655162	0.00117163169564588\\
0.976001225418392	0.00118995845290463\\
0.981335819181621	0.00120846299467622\\
0.98667041294485	0.00122714592001734\\
0.99200500670808	0.00124600782434303\\
0.997363697280812	0.00126513571967429\\
1.00272238785355	0.00128444540785023\\
1.00808107842628	0.00130393748131632\\
1.01343976899901	0.00132361252884274\\
1.0188228172289	0.00134356182178957\\
1.02420586545878	0.00136369693580067\\
1.02958891368867	0.00138401845653489\\
1.03497196191855	0.00140452696594257\\
1.04037963341625	0.00142531814259286\\
1.04578730491395	0.0014462991884976\\
1.05119497641164	0.00146747068235088\\
1.05660264790934	0.0014888331991051\\
1.06203521338424	0.00151048697728356\\
1.06746777885913	0.00153233469298081\\
1.07290034433402	0.0015543769177453\\
1.07833290980891	0.00157661421935028\\
1.08379064506731	0.0015991515527343\\
1.08924838032571	0.00162188691207705\\
1.09470611558411	0.00164482086159727\\
1.10016385084252	0.00166795396170511\\
1.1056470369148	0.00169139604453325\\
1.11113022298707	0.00171504026204362\\
1.11661340905935	0.00173888717093768\\
1.12209659513163	0.00176293732407503\\
1.12760551858231	0.00178730559681889\\
1.13311444203298	0.00181188013337916\\
1.13862336548365	0.00183666148274943\\
1.14413228893432	0.00186165019004878\\
1.14966724184116	0.00188696634388034\\
1.15520219474799	0.00191249291119588\\
1.16073714765482	0.0019382304330882\\
1.16627210056166	0.00196417944674103\\
1.17183338076584	0.00199046542930839\\
1.17739466097002	0.00201696599569975\\
1.18295594117419	0.00204368167890923\\
1.18851722137837	0.00207061300798872\\
1.19410513262804	0.00209789102893841\\
1.19969304387772	0.00212538782487044\\
1.20528095512739	0.00215310392047905\\
1.21086886637706	0.00218103983648321\\
1.21648371859449	0.00220933237378246\\
1.22209857081192	0.0022378478982028\\
1.22771342302936	0.00226658692593469\\
1.23332827524679	0.00229554996915945\\
1.23897038457538	0.00232487977456153\\
1.24461249390397	0.00235443680036725\\
1.25025460323256	0.00238422155405454\\
1.25589671256115	0.00241423453905872\\
1.26156640151281	0.00244462464419636\\
1.26723609046446	0.00247524622433333\\
1.27290577941612	0.00250609977802266\\
1.27857546836778	0.00253718579974153\\
1.28427306633634	0.00256865952443018\\
1.28997066430491	0.00260036900023369\\
1.29566826227347	0.00263231471656381\\
1.30136586024204	0.00266449715872336\\
1.30709170327692	0.0026970781155865\\
1.31281754631181	0.00272989912138352\\
1.31854338934669	0.0027629606561641\\
1.32426923238158	0.00279626319583603\\
1.33002366371043	0.00282997529915126\\
1.33577809503928	0.00286393177114344\\
1.34153252636813	0.00289813308227561\\
1.34728695769699	0.00293257969883527\\
1.35307032782915	0.00296744717140857\\
1.35885369796131	0.00300256335455072\\
1.36463706809348	0.00303792870890836\\
1.37042043822564	0.00307354369091922\\
1.37622881553645	0.00310956437630575\\
1.38203719284726	0.00314583776281463\\
1.38784557015807	0.00318236430004984\\
1.39365394746888	0.00321914443338532\\
1.39948194569202	0.00325630413778809\\
1.40530994391516	0.00329372004001648\\
1.41113794213829	0.00333139257717273\\
1.41696594036143	0.00336932218212518\\
1.4228137248248	0.00340763936902326\\
1.42866150928816	0.00344621623250614\\
1.43450929375153	0.00348505319705296\\
1.44035707821489	0.00352415068290528\\
1.44622478964653	0.0035636436708056\\
1.45209250107816	0.00360339979341288\\
1.45796021250979	0.00364341946245567\\
1.46382792394143	0.00368370308542181\\
1.46971570462452	0.00372439020535574\\
1.4756034853076	0.00376534389705535\\
1.48149126599069	0.00380656455937158\\
1.48737904667378	0.00384805258691236\\
1.49328704060906	0.00388995218285096\\
1.49919503454433	0.003932121766137\\
1.50510302847961	0.00397456172261676\\
1.51101102241489	0.00401727243389234\\
1.51693937480906	0.00406040285840711\\
1.52286772720322	0.00410380666391001\\
1.52879607959739	0.00414748422311705\\
1.53472443199155	0.00419143590449816\\
1.540673290528	0.00423581552717689\\
1.54662214906446	0.00428047190219487\\
1.55257100760091	0.00432540538901099\\
1.55851986613736	0.00437061634283833\\
1.56448937947106	0.00441626353880925\\
1.57045889280476	0.00446219083568948\\
1.57642840613847	0.00450839857955523\\
1.58239791947216	0.00455488711223737\\
1.58838823791446	0.00460182026620284\\
1.59437855635675	0.00464903684643492\\
1.60036887479904	0.0046965371855022\\
1.60635919324133	0.0047443216117283\\
1.61237046959671	0.00479255912396716\\
1.61838174595208	0.00484108336426296\\
1.62439302230746	0.00488989465155103\\
1.63040429866284	0.00493899330052336\\
1.63643668672632	0.00498855357456031\\
1.6424690747898	0.00503840385437671\\
1.64850146285328	0.00508854444514967\\
1.65453385091676	0.00513897564781561\\
1.66058750665404	0.00518987709897024\\
1.66664116239132	0.00524107180916707\\
1.67269481812859	0.00529256006970089\\
1.67874847386587	0.00534434216762808\\
1.68482355500326	0.00539660321918473\\
1.69089863614064	0.00544916075815231\\
1.69697371727803	0.00550201506181894\\
1.70304879841541	0.00555516640323793\\
1.70914546464501	0.00560880548754901\\
1.71524213087462	0.005662744262328\\
1.72133879710422	0.00571698299073248\\
1.72743546333383	0.00577152193168909\\
1.73355387572031	0.00582655748429639\\
1.73967228810678	0.00588189590463653\\
1.74579070049326	0.00593753744161279\\
1.75190911287974	0.0059934823399021\\
1.75804943463327	0.00604993280552477\\
1.76418975638681	0.00610668928994487\\
1.77033007814035	0.00616375202768757\\
1.77647039989388	0.00622112124905751\\
1.78263279664118	0.00627900508374608\\
1.78879519338848	0.00633719806171107\\
1.79495759013578	0.00639570040297693\\
1.80111998688309	0.00645451232335341\\
1.80730462566597	0.00651384798471046\\
1.81348926444885	0.00657349588674891\\
1.81967390323173	0.00663345623486944\\
1.82585854201462	0.00669372923026477\\
1.83206559217232	0.00675453518493464\\
1.83827264233002	0.00681565645021182\\
1.84447969248772	0.00687709321675042\\
1.85068674264543	0.00693884567100446\\
1.85691637550678	0.00700114039127206\\
1.86314600836813	0.00706375346415329\\
1.86937564122948	0.00712668506543371\\
1.87560527409083	0.00718993536670652\\
1.88185766319929	0.00725373733214671\\
1.88811005230775	0.00731786066386579\\
1.89436244141622	0.00738230552265794\\
1.90061483052468	0.00744707206513461\\
1.90689015093446	0.00751239975481639\\
1.91316547134424	0.0075780517956011\\
1.91944079175402	0.00764402833317059\\
1.92571611216381	0.00771032950903269\\
1.93201454150218	0.00777720141161729\\
1.93831297084055	0.00784440062089338\\
1.94461140017892	0.00791192726730794\\
1.9509098295173	0.00797978147714477\\
1.95723154808653	0.00804821609185108\\
1.96355326665576	0.0081169809392124\\
1.96987498522499	0.00818607613431983\\
1.97619670379422	0.00825550178811227\\
1.98254189319299	0.00832551760940839\\
1.98888708259176	0.00839586655916016\\
1.99523227199053	0.00846654873698177\\
2.0015774613893	0.00853756423834692\\
2.00794630624256	0.00860917977391072\\
2.01431515109582	0.00868113130320558\\
2.02068399594908	0.00875341891024753\\
2.02705284080234	0.00882604267492471\\
2.03344552746868	0.00889927643104625\\
2.03983821413503	0.00897284901514485\\
2.04623090080137	0.00904676049551728\\
2.05262358746771	0.00912101093634619\\
2.05904030495676	0.00919588142750283\\
2.06545702244581	0.00927109354943932\\
2.07187373993486	0.00934664735461334\\
2.07829045742391	0.00942254289138139\\
2.084731396657	0.0094990686312147\\
2.0911723358901	0.00957593877268979\\
2.0976132751232	0.00965315335230445\\
2.1040542143563	0.0097307124024703\\
2.11051956957009	0.00980891191975411\\
2.11698492478388	0.00988745857724684\\
2.12345027999767	0.00996635239536596\\
2.12991563521146	0.0100455933904583\\
2.13640560238783	0.0101254852101484\\
2.1428955695642	0.0102057268757906\\
2.14938553674057	0.0102863183916024\\
2.15587550391694	0.0103672597577463\\
2.16239028199053	0.010448862414831\\
2.16890506006412	0.0105308175903815\\
2.17541983813771	0.0106131252722951\\
2.1819346162113	0.0106957854444316\\
2.18847440630977	0.0107791174741812\\
2.19501419640825	0.0108628046611866\\
2.20155398650673	0.0109468469769048\\
2.20809377660521	0.0110312443887741\\
2.21465878275268	0.0111163243349622\\
2.22122378890015	0.0112017620430464\\
2.22778879504762	0.0112875574679243\\
2.23435380119508	0.0113737105604932\\
2.24094422983074	0.0114605569688445\\
2.24753465846639	0.0115477637090974\\
2.25412508710205	0.0116353307194707\\
2.2607155157377	0.0117232579342023\\
2.2673315759421	0.0118118893546629\\
2.27394763614649	0.0119008836419279\\
2.28056369635089	0.011990240717418\\
2.28717975655528	0.0120799604985929\\
2.29382166035897	0.0121703954889726\\
2.30046356416266	0.0122611958455144\\
2.30710546796636	0.0123523614727216\\
2.31374737177005	0.0124438922711574\\
2.32041533418655	0.0125361493973062\\
2.32708329660306	0.0126287743529833\\
2.33375125901957	0.0127217670256557\\
2.34041922143607	0.0128151272988719\\
2.34711345989709	0.0129092251264324\\
2.3538076983581	0.0130036932103791\\
2.36050193681912	0.0130985314210234\\
2.36719617528013	0.0131937396247812\\
2.37391691043736	0.0132896967296879\\
2.3806376455946	0.0133860264808932\\
2.38735838075183	0.0134827287314343\\
2.39407911590907	0.0135798033304762\\
2.40082657131283	0.013677638294147\\
2.40757402671659	0.0137758482566021\\
2.41432148212035	0.0138744330534852\\
2.42106893752412	0.0139733925165926\\
2.42784333963791	0.0140731239255915\\
2.4346177417517	0.0141732326479442\\
2.4413921438655	0.014273718501783\\
2.44816654597929	0.0143745813014169\\
2.4549681243632	0.0144762277496043\\
2.4617697027471	0.0145782537873214\\
2.46857128113101	0.0146806592150698\\
2.47537285951492	0.0147834438295543\\
2.48220184698832	0.014887023920107\\
2.48903083446173	0.0149909858375022\\
2.49585982193513	0.0150953293644918\\
2.50268880940853	0.0152000542800586\\
2.50954544177924	0.0153055866209608\\
2.51640207414995	0.0154115029866447\\
2.52325870652066	0.0155178031419949\\
2.53011533889137	0.0156244868481535\\
2.53699985562036	0.0157319900618928\\
2.54388437234935	0.0158398794585346\\
2.55076888907833	0.0159481547849769\\
2.55765340580732	0.0160568157844043\\
2.5645660494942	0.016166308499785\\
2.57147869318108	0.0162761895158565\\
2.57839133686796	0.0163864585614113\\
2.58530398055484	0.016497115361559\\
2.59224499685821	0.0166086162117505\\
2.59918601316159	0.0167205074395545\\
2.60612702946496	0.0168327887555398\\
2.61306804576833	0.0169454598666227\\
2.62003768421203	0.0170589875014834\\
2.62700732265572	0.0171729075495572\\
2.63397696109942	0.0172872197030706\\
2.64094659954311	0.0174019236506292\\
2.64794511298781	0.0175174967279672\\
2.65494362643251	0.0176334642122722\\
2.6619421398772	0.0177498257773098\\
2.6689406533219	0.017866581093257\\
2.67596829852112	0.017984218287483\\
2.68299594372033	0.0181022518401082\\
2.69002358891955	0.0182206814063185\\
2.69705123411877	0.0183395066377449\\
2.70410827073433	0.0184592266230197\\
2.71116530734989	0.0185793448752061\\
2.71822234396545	0.0186998610307915\\
2.72527938058101	0.0188207747227429\\
2.73236607296869	0.0189425962024104\\
2.73945276535637	0.0190648178141721\\
2.74653945774405	0.0191874391756983\\
2.75362615013173	0.0193104599011737\\
2.76074276611797	0.019434401587038\\
2.76785938210422	0.0195587452262883\\
2.77497599809046	0.019683490417659\\
2.78209261407671	0.019808636756435\\
2.78923942490227	0.0199347173669136\\
2.79638623572783	0.0200612017075988\\
2.80353304655339	0.0201880893581698\\
2.81067985737896	0.0203153798948937\\
2.81785713902212	0.0204436181776936\\
2.82503442066529	0.0205722619226042\\
2.83221170230846	0.0207013106901301\\
2.83938898395163	0.0208307640374021\\
2.84659701614953	0.0209611787521823\\
2.85380504834743	0.0210920006154897\\
2.86101308054533	0.0212232291685357\\
2.86822111274323	0.0213548639491965\\
2.87546017926332	0.0214874738717447\\
2.88269924578341	0.0216204925831955\\
2.88993831230351	0.0217539196053468\\
2.8971773788236	0.0218877544567025\\
2.90444776780364	0.0220225783846545\\
2.91171815678368	0.0221578126953115\\
2.91898854576373	0.0222934568909384\\
2.92625893474377	0.0224295104705477\\
2.93356093845992	0.0225665672187143\\
2.94086294217606	0.0227040358962417\\
2.94816494589221	0.0228419159857425\\
2.95546694960836	0.0229802069666187\\
2.96280086487659	0.0231195153739414\\
2.97013478014483	0.0232592372095901\\
2.97746869541306	0.0233993719364049\\
2.9848026106813	0.0235399190140585\\
2.99216873909201	0.0236814979477885\\
2.99953486750271	0.0238234917605843\\
3.00690099591342	0.0239658998953931\\
3.01426712432413	0.0241087217920392\\
3.02166577186631	0.0242525901402344\\
3.02906441940849	0.0243968747694157\\
3.03646306695067	0.0245415751025162\\
3.04386171449286	0.0246866905593927\\
3.05129319192691	0.0248328672375373\\
3.05872466936097	0.024979461549198\\
3.06615614679503	0.0251264728971735\\
3.07358762422909	0.0252739006812326\\
3.08105224696524	0.0254224046293906\\
3.0885168697014	0.0255713275136021\\
3.09598149243755	0.0257206687164101\\
3.1034461151737	0.0258704276173762\\
3.11094420406679	0.0260212778155197\\
3.11844229295988	0.0261725482017295\\
3.12594038185297	0.0263242381381718\\
3.13343847074605	0.0264763469840795\\
3.14097035080817	0.0266295624259439\\
3.14850223087028	0.0267831992566669\\
3.1560341109324	0.0269372568179155\\
3.16356599099451	0.0270917344484739\\
3.17113199342146	0.027247334181797\\
3.17869799584842	0.0274033564530605\\
3.18626399827537	0.0275598005833107\\
3.19383000070233	0.0277166658907617\\
3.20143046110215	0.0278746689810415\\
3.20903092150198	0.0280330957059256\\
3.21663138190181	0.0281919453657163\\
3.22423184230164	0.0283512172579366\\
3.2318671021242	0.0285116428170273\\
3.23950236194676	0.0286724930543979\\
3.24713762176933	0.0288337672494835\\
3.25477288159189	0.0289954646789939\\
3.26244328804483	0.0291583318632845\\
3.27011369449776	0.0293216247159398\\
3.2777841009507	0.0294853424954045\\
3.28545450740363	0.0296494844574528\\
3.29316041266953	0.0298148124507696\\
3.30086631793543	0.0299805670482339\\
3.30857222320133	0.0301467474871756\\
3.31627812846722	0.030313353002311\\
3.32401989110489	0.0304811610450283\\
3.33176165374256	0.0306493965727938\\
3.33950341638023	0.0308180588016982\\
3.3472451790179	0.0309871469452762\\
3.35502316313618	0.0311574543162921\\
3.36280114725446	0.0313281899976789\\
3.37057913137275	0.0314993531841631\\
3.37835711549103	0.0316709430679737\\
3.38617169154667	0.0318437691013879\\
3.3939862676023	0.0320170242142846\\
3.40180084365794	0.0321907075799006\\
3.40961541971357	0.032364818369035\\
3.41746696420648	0.032540182447052\\
3.42531850869938	0.03271597631676\\
3.43317005319229	0.032892199129779\\
3.4410215976852	0.0330688500353536\\
3.44891049346576	0.0332467715944282\\
3.45679938924632	0.0334251235998129\\
3.46468828502688	0.0336039051813845\\
3.47257718080744	0.0337831154667069\\
3.48050381756295	0.0339636140079943\\
3.48843045431847	0.0341445435919614\\
3.49635709107398	0.034325903326614\\
3.50428372782949	0.0345076923177091\\
3.51224850159594	0.0346907873955279\\
3.52021327536239	0.0348743140534043\\
3.52817804912884	0.0350582713773446\\
3.53614282289529	0.0352426584511719\\
3.54414613607264	0.0354283696723346\\
3.55214944924998	0.0356145129511712\\
3.56015276242733	0.0358010873515599\\
3.56815607560468	0.035988091935262\\
3.57619833889781	0.0361764390033761\\
3.58424060219095	0.0363652185464292\\
3.59228286548408	0.03655442960604\\
3.60032512877721	0.0367440712217798\\
3.60840675922679	0.0369350738913304\\
3.61648838967638	0.0371265093918973\\
3.62457002012596	0.0373183767427101\\
3.63265165057554	0.0375106749610201\\
3.64077307248204	0.0377043530577683\\
3.64889449438855	0.0378984642796558\\
3.65701591629505	0.0380930076233908\\
3.66513733820155	0.0382879820837747\\
3.67329898353354	0.0384843555137039\\
3.68146062886554	0.0386811623001833\\
3.68962227419754	0.038878401417267\\
3.69778391952953	0.0390760718371755\\
3.70598622860803	0.0392751606024341\\
3.71418853768653	0.0394746828922217\\
3.72239084676502	0.0396746376578042\\
3.73059315584352	0.0398750238486889\\
3.73883657626523	0.0400768480209967\\
3.74707999668693	0.0402791058215093\\
3.75532341710863	0.0404817961785703\\
3.76356683753034	0.0406849180188401\\
3.77185182516863	0.0408894977626448\\
3.78013681280692	0.0410945111732262\\
3.78842180044522	0.0412999571558691\\
3.79670678808351	0.0415058346142534\\
3.80503380745825	0.041713190195162\\
3.81336082683299	0.0419209794155133\\
3.82168784620774	0.0421292011573969\\
3.83001486558248	0.0423378543013762\\
3.83838438958511	0.0425480060785771\\
3.84675391358775	0.0427585914011111\\
3.85512343759039	0.042969609127734\\
3.86349296159302	0.0431810581157566\\
3.8719054722877	0.0433940265610931\\
3.88031798298237	0.0436074283900492\\
3.88873049367705	0.0438212624379073\\
3.89714300437172	0.0440355275385876\\
3.90559899284027	0.0442513332319133\\
3.91405498130882	0.0444675720786539\\
3.92251096977737	0.0446842428904775\\
3.93096695824593	0.0449013444777743\\
3.93946692479213	0.0451200081110725\\
3.94796689133833	0.0453391045981791\\
3.95646685788454	0.0455586327270057\\
3.96496682443074	0.0457785912842723\\
3.97351127881372	0.0460001336664787\\
3.98205573319669	0.0462221085325537\\
3.99060018757967	0.0464445146465084\\
3.99914464196265	0.0466673507712509\\
4.00773410425464	0.0468917928493782\\
4.01632356654662	0.0471166669702086\\
4.02491302883861	0.0473419718737073\\
4.0335024911306	0.0475677062988265\\
4.04213749081375	0.0477950691333865\\
4.05077249049689	0.0480228634972252\\
4.05940749018004	0.0482510881061159\\
4.06804248986318	0.0484797416749101\\
4.07672356764034	0.0487100464859807\\
4.08540464541749	0.0489407822397461\\
4.09408572319465	0.0491719476276387\\
4.10276680097181	0.0494035413402623\\
4.11149450824228	0.0496368094929005\\
4.12022221551276	0.0498705079274823\\
4.12894992278323	0.0501046353109482\\
4.1376776300537	0.0503391903095065\\
4.14645252883691	0.0505754433102368\\
4.15522742762012	0.0508121258569227\\
4.16400232640333	0.0510492365918612\\
4.17277722518654	0.0512867741567141\\
4.18159988926615	0.0515260336826812\\
4.19042255334575	0.0517657219423662\\
4.19924521742536	0.0520058375532679\\
4.20806788150496	0.0522463791323504\\
4.21693889629988	0.0524886670266166\\
4.22580991109479	0.0527313827650283\\
4.23468092588971	0.05297452494013\\
4.24355194068463	0.0532180921440331\\
4.2524719033702	0.0534634304172024\\
4.26139186605577	0.0537091955664761\\
4.27031182874135	0.0539553861592865\\
4.27923179142692	0.0542020007627358\\
4.28820131218307	0.0544504116255367\\
4.29717083293922	0.0546992483168555\\
4.30614035369538	0.0549485093788513\\
4.31510987445153	0.0551981933534592\\
4.32412957614907	0.0554496992067278\\
4.33314927784661	0.0557016297602089\\
4.34216897954416	0.0559539835306251\\
4.3511886812417	0.0562067590345836\\
4.36025919971018	0.0564613824747678\\
4.36932971817866	0.0567164294049241\\
4.37840023664715	0.0569718983161735\\
4.38747075511563	0.0572277876996319\\
4.39659274002011	0.0574855515413475\\
4.40571472492458	0.0577437375796564\\
4.41483670982906	0.0580023442799095\\
4.42395869473353	0.0582613701075658\\
4.43313280988389	0.0585222973904897\\
4.44230692503424	0.0587836454922397\\
4.45148104018459	0.0590454128522264\\
4.46065515533495	0.0593075979100832\\
4.469882079136	0.0595717119097048\\
4.47910900293706	0.059836245264713\\
4.48833592673811	0.0601011963884038\\
4.49756285053916	0.0603665636944141\\
4.50684327607396	0.0606338879221725\\
4.51612370160877	0.0609016299548492\\
4.52540412714357	0.0611697881794493\\
4.53468455267837	0.0614383609834389\\
4.54401918924822	0.0617089192286553\\
4.55335382581807	0.0619798936399932\\
4.56268846238792	0.0622512825779868\\
4.57202309895777	0.0625230844037543\\
4.58141267108196	0.0627969007025038\\
4.59080224320616	0.0630711314387738\\
4.60019181533036	0.0633457749464444\\
4.60958138745456	0.0636208295601056\\
4.61902663678989	0.0638979282481469\\
4.62847188612523	0.0641754395539089\\
4.63791713546056	0.0644533617844316\\
4.6473623847959	0.0647316932475922\\
4.65686406984621	0.0650120989494338\\
4.66636575489653	0.065292915356484\\
4.67586743994685	0.0655741407487518\\
4.68536912499717	0.0658557734072149\\
4.69492802222378	0.0661395110658345\\
4.7044869194504	0.0664236574229234\\
4.71404581667701	0.0667082107312661\\
4.72360471390363	0.0669931692447494\\
4.73320764525141	0.0672798467008731\\
4.74281057659919	0.0675669295748214\\
4.75241350794698	0.0678544160996777\\
4.76201643929476	0.0681423045097567\\
4.77165794452804	0.0684317518674175\\
4.78129944976131	0.068721600782313\\
4.79094095499459	0.0690118494712198\\
4.80058246022786	0.0693024961522729\\
4.81026314694842	0.069594722601228\\
4.81994383366897	0.0698873466878031\\
4.82962452038952	0.0701803666125327\\
4.83930520711008	0.0704737805774382\\
4.84902565638322	0.0707687943750962\\
4.85874610565637	0.0710642018281359\\
4.86846655492951	0.0713600011209321\\
4.87818700420266	0.0716561904394751\\
4.88794780523678	0.0719539999377261\\
4.89770860627089	0.0722521990457987\\
4.90746940730501	0.0725507859319888\\
4.91723020833913	0.072849758766336\\
4.92703195872998	0.0731503724153926\\
4.93683370912084	0.0734513715646641\\
4.9466354595117	0.0737527543664452\\
4.95643720990256	0.0740545189749044\\
4.96628051524823	0.0743579453076268\\
4.97612382059389	0.0746617529661076\\
4.98596712593956	0.0749659400867174\\
4.99581043128522	0.075270504807832\\
5.00569590732134	0.0755767525014873\\
5.01558138335745	0.0758833772809104\\
5.02546685939356	0.0761903772666225\\
5.03535233542967	0.0764977505812813\\
5.04528060635073	0.0768068284021117\\
5.05520887727179	0.077116279002318\\
5.06513714819285	0.0774261004866458\\
5.07506541911391	0.077736290962109\\
5.08503711889176	0.0780482078018079\\
5.0950088186696	0.0783604930472742\\
5.10498051844744	0.0786731447875494\\
5.11495221822528	0.0789861611140762\\
5.12496799094167	0.0793009259960847\\
5.13498376365806	0.0796160548421909\\
5.14499953637444	0.0799315457258014\\
5.15501530909083	0.0802473967228588\\
5.16507580953502	0.0805650188169324\\
5.17513630997921	0.0808830003645153\\
5.18519681042339	0.0812013394234477\\
5.19525731086758	0.0815200340542404\\
5.2053632039849	0.0818405226546683\\
5.21546909710223	0.0821613661281456\\
5.22557499021955	0.0824825625170125\\
5.23568088333688	0.0828041098664157\\
5.24583284576096	0.0831274744357652\\
5.25598480818505	0.083451189226931\\
5.26613677060914	0.0837752522668179\\
5.27628873303322	0.0840996615852745\\
5.28648745229033	0.0844259117244864\\
5.29668617154743	0.0847525073624905\\
5.30688489080453	0.0850794465108198\\
5.31708361006163	0.0854067271840888\\
5.32732978639303	0.0857358726856234\\
5.33757596272443	0.0860653588901811\\
5.34782213905582	0.0863951837939845\\
5.35806831538722	0.0867253453964764\\
5.36836266133408	0.0870573962262633\\
5.37865700728095	0.0873897828895112\\
5.38895135322782	0.0877225033671915\\
5.39924569917468	0.0880555556436364\\
5.40958894010065	0.0883905219525926\\
5.41993218102661	0.0887258191505748\\
5.43027542195257	0.0890614452033616\\
5.44061866287853	0.0893973980802336\\
5.45101153706255	0.0897352902017248\\
5.46140441124658	0.0900735081917938\\
5.4717972854306	0.0904120500010829\\
5.48219015961463	0.0907509135838788\\
5.49263341945276	0.0910917420669582\\
5.50307667929089	0.0914328913210201\\
5.51351993912902	0.0917743592816245\\
5.52396319896715	0.0921161438881208\\
5.53445761187213	0.0924599195212433\\
5.54495202477712	0.0928040107494511\\
5.5554464376821	0.0931484154932757\\
5.56594085058709	0.0934931316771823\\
5.57648719831941	0.0938398654604165\\
5.58703354605173	0.0941869095832666\\
5.59757989378405	0.0945342619512858\\
5.60812624151637	0.094881920474108\\
5.61872532103302	0.0952316236404023\\
5.62932440054967	0.0955816318099729\\
5.63992348006633	0.0959319428734449\\
5.65052255958298	0.096282554725672\\
5.66117518401188	0.0966352387663021\\
5.67182780844078	0.0969882223916788\\
5.68248043286967	0.0973415034775461\\
5.69313305729857	0.0976950799040265\\
5.70384005671797	0.0980507565876255\\
5.71454705613736	0.0984067273538884\\
5.72525405555675	0.098762990063724\\
5.73596105497615	0.0991195425825711\\
5.74672327665992	0.0994782239562296\\
5.75748549834369	0.0998371938254795\\
5.76824772002746	0.100196450036438\\
5.77900994171123	0.100555990439906\\
5.78982825023888	0.100917688825246\\
5.80064655876654	0.101279670032594\\
5.81146486729419	0.101641931893319\\
5.82228317582184	0.102004472243626\\
5.83315845473553	0.102369200284176\\
5.84403373364921	0.102734205385116\\
5.8549090125629	0.103099485363105\\
5.86578429147658	0.103465038039796\\
5.87671744355088	0.103832808702015\\
5.88765059562518	0.104200850573357\\
5.89858374769948	0.104569161455813\\
5.90951689977378	0.104937739156522\\
5.92050884813521	0.105308565758911\\
5.93150079649664	0.105679657627856\\
5.94249274485807	0.10605101255071\\
5.9534846932195	0.106422628320138\\
5.96453638108293	0.106796524515443\\
5.97558806894636	0.107170679941656\\
5.98663975680979	0.10754509237153\\
5.99769144467322	0.107919759583289\\
6.00880383710542	0.108296739408542\\
6.01991622953761	0.108673972334177\\
6.0310286219698	0.109051456118377\\
6.042141014402	0.109429188524964\\
6.05331509935885	0.109809266426979\\
6.0644891843157	0.110189591202104\\
6.07566326927256	0.110570160593983\\
6.08683735422941	0.110950972352064\\
6.09807414230094	0.111334163168535\\
6.10931093037248	0.111717594532099\\
6.12054771844401	0.112101264171889\\
6.13178450651555	0.11248516982301\\
6.14308503319571	0.112871488848912\\
6.15438555987588	0.113258041995124\\
6.16568608655605	0.11364482697629\\
6.17698661323621	0.1140318415132\\
6.18835193888863	0.114421304488436\\
6.19971726454106	0.114810995054273\\
6.21108259019348	0.115200910910894\\
6.2224479158459	0.115591049764797\\
6.23387912710053	0.115983672909992\\
6.24531033835517	0.116376517010927\\
6.2567415496098	0.116769579753338\\
6.26817276086443	0.117162858829456\\
6.2796709718506	0.117558658875777\\
6.29116918283678	0.117954673135495\\
6.30266739382295	0.118350899279919\\
6.31416560480912	0.118747334987031\\
6.32573195844141	0.119146329207318\\
6.33729831207371	0.119545530788751\\
6.34886466570601	0.119944937388229\\
6.36043101933831	0.120344546669502\\
6.37206668825559	0.120746752896846\\
6.38370235717287	0.121149159520633\\
6.39533802609016	0.121551764183362\\
6.40697369500744	0.121954564534568\\
6.41867988207414	0.122360001164954\\
6.43038606914084	0.122765631111941\\
6.44209225620754	0.123171452003635\\
6.45379844327423	0.123577461475367\\
6.46557638449348	0.123986147553024\\
6.47735432571273	0.124395019749538\\
6.48913226693197	0.124804075678634\\
6.50091020815122	0.125213312961445\\
6.51276117329228	0.125625268184577\\
6.52461213843333	0.126037402208007\\
6.53646310357439	0.126449712631077\\
6.54831406871544	0.126862197060732\\
6.56023936237899	0.127277441801379\\
6.57216465604253	0.127692857899879\\
6.58408994970607	0.12810844294119\\
6.59601524336962	0.128524194518071\\
6.608016207688	0.128942749897984\\
6.62001717200638	0.129361469066234\\
6.63201813632476	0.129780349593389\\
6.64401910064313	0.130199389058023\\
6.65609711612489	0.130621276959143\\
6.66817513160664	0.131043320948642\\
6.68025314708839	0.131465518582695\\
6.69233116257014	0.131887867425681\\
6.70448764976455	0.132313110528014\\
6.71664413695897	0.132738501884791\\
6.72880062415338	0.133164039037781\\
6.7409571113478	0.133589719537161\\
6.75319353351478	0.134018341390768\\
6.76542995568177	0.13444710352721\\
6.77766637784875	0.134876003473831\\
6.78990280001574	0.135305038766598\\
6.80222066442117	0.135737063814412\\
6.8145385288266	0.136169221031872\\
6.82685639323203	0.136601507931877\\
6.83917425763746	0.137033922036164\\
6.85157511856051	0.137469375695479\\
6.86397597948357	0.137904953265579\\
6.87637684040662	0.138340652244892\\
6.88877770132968	0.138776470140903\\
6.90126316148383	0.139215378827933\\
6.91374862163799	0.139654403016884\\
6.92623408179214	0.140093540191686\\
6.93871954194629	0.140532787845548\\
6.95129125486549	0.140975179032275\\
6.96386296778468	0.141417677157501\\
6.97643468070388	0.141860279690619\\
6.98900639362307	0.142302984110531\\
7.00166606764739	0.142748886436316\\
7.0143257416717	0.14319488697783\\
7.02698541569601	0.143640983189889\\
7.03964508972033	0.144087172537051\\
7.05239448925109	0.144536615822515\\
7.06514388878186	0.144986148436508\\
7.07789328831262	0.145435767819221\\
7.09064268784339	0.145885471420827\\
7.10348363709426	0.146338486767642\\
7.11632458634514	0.146791582385998\\
7.12916553559601	0.14724475570141\\
7.14200648484689	0.147698004149616\\
7.15494087138423	0.148154624030967\\
7.16787525792157	0.148611314951422\\
7.18080964445891	0.149068074321756\\
7.19374403099625	0.149524899563217\\
7.20677380774895	0.149985157858041\\
7.21980358450165	0.150445477778064\\
7.23283336125435	0.150905856719256\\
7.24586313800706	0.151366292088312\\
7.25899032889946	0.151830224241021\\
7.27211751979186	0.152294208417204\\
7.28524471068426	0.152758241997949\\
7.29837190157667	0.153222322375332\\
7.31159860422458	0.153689965449306\\
7.32482530687249	0.154157650750455\\
7.3380520095204	0.154625375644897\\
7.35127871216831	0.155093137510013\\
7.36460710304072	0.155564530325343\\
7.37793549391314	0.156035955369798\\
7.39126388478556	0.15650740999444\\
7.40459227565797	0.156978891561861\\
7.41802461355885	0.157454074770613\\
7.43145695145972	0.157929280001061\\
7.4448892893606	0.158404504589102\\
7.45832162726148	0.15887974588245\\
7.47186025959003	0.15935876213881\\
7.48539889191859	0.159837789991931\\
7.49893752424715	0.160316826762435\\
7.51247615657571	0.160795869783048\\
7.52612352432791	0.161278763865252\\
7.53977089208011	0.161761658893115\\
7.55341825983232	0.162244552171856\\
7.56706562758452	0.162727441019099\\
7.58082427117841	0.163214259976437\\
7.5945829147723	0.163701068992912\\
7.60834155836619	0.164187865358206\\
7.62210020196009	0.164674646374708\\
7.63597276754062	0.165165439687016\\
7.64984533312116	0.165656211926632\\
7.66371789870169	0.166146960367547\\
7.67759046428223	0.166637682296775\\
7.69157971106386	0.16713250206573\\
7.70556895784549	0.167627289374265\\
7.71955820462712	0.168122041480515\\
7.73354745140875	0.168616755655966\\
7.7476562591015	0.169115656793889\\
7.76176506679424	0.169614513816412\\
7.77587387448699	0.170113323965633\\
7.78998268217974	0.170612084497334\\
7.80421406003167	0.171115124967153\\
7.81844543788359	0.171618109387137\\
7.83267681573552	0.172121034983148\\
7.84690819358745	0.172623898995077\\
7.86126528820976	0.173131140000576\\
7.87562238283208	0.173638312729235\\
7.8899794774544	0.174145414390457\\
7.90433657207672	0.17465244220804\\
7.91882267896367	0.175163948507343\\
7.93330878585062	0.175675373996078\\
7.94779489273757	0.176186715866954\\
7.96228099962452	0.176697971327441\\
7.97689957365268	0.17721381150213\\
7.99151814768085	0.177729558010524\\
8.00613672170902	0.178245208028377\\
8.02075529573718	0.178760758746586\\
8.03550996276884	0.179281005500679\\
8.0502646298005	0.179801145394238\\
8.06501929683216	0.180321175585771\\
8.07977396386382	0.180841093249329\\
8.09466853533245	0.181365823803231\\
8.10956310680107	0.181890433945926\\
8.12445767826969	0.182414920818362\\
8.13935224973832	0.182939281577441\\
8.15439073744049	0.183468578057552\\
8.16942922514267	0.183997740199882\\
8.18446771284484	0.18452676512747\\
8.19950620054702	0.185055649979742\\
8.2146928332504	0.185589599859102\\
8.22987946595378	0.186123401077007\\
8.24506609865717	0.186657050738209\\
8.26025273136055	0.187190545964289\\
8.27559197362809	0.187729242566608\\
8.29093121589562	0.188267775763556\\
8.30627045816316	0.188806142641178\\
8.32160970043069	0.189344340302809\\
8.33710627308224	0.189887883347531\\
8.35260284573378	0.190431247797406\\
8.36809941838533	0.19097443071931\\
8.38359599103687	0.191517429197891\\
8.39925489573158	0.192065925471158\\
8.41491380042629	0.192614227486738\\
8.430572705121	0.193162332291822\\
8.44623160981571	0.193710236951884\\
8.46205815432056	0.194263800980095\\
8.4778846988254	0.194817154583834\\
8.49371124333024	0.195370294790047\\
8.50953778783509	0.195923218644493\\
8.52553761739547	0.196481973557185\\
8.54153744695585	0.197040501340893\\
8.55753727651623	0.19759879900169\\
8.57353710607661	0.198156863565017\\
8.58971623619016	0.198720941978597\\
8.60589536630372	0.199284775983563\\
8.62207449641727	0.199848362564423\\
8.63825362653082	0.200411698725638\\
8.65461848330403	0.200981243859593\\
8.67098334007723	0.201550526688597\\
8.68734819685043	0.202119544174818\\
8.70371305362364	0.202688293300992\\
8.72018188980809	0.203260382203816\\
8.73665072599255	0.203832193257252\\
8.75311956217701	0.204403723448993\\
8.76958839836146	0.204974969787496\\
8.78614348346302	0.205548918712919\\
8.80269856856458	0.206122574805422\\
8.81925365366614	0.206695935089408\\
8.83580873876771	0.207268996610144\\
8.85245111606455	0.207844775695686\\
8.86909349336139	0.208420246934614\\
8.88573587065824	0.208995407388708\\
8.90237824795508	0.209570254140713\\
8.91910883699713	0.21014782870211\\
8.93583942603918	0.210725080378524\\
8.95257001508123	0.211302006269857\\
8.96930060412328	0.211878603497069\\
8.98612034063892	0.212457938890219\\
9.00294007715455	0.21303693633615\\
9.01975981367019	0.213615592973629\\
9.03657955018582	0.214193905962565\\
9.05348938408815	0.214774967508431\\
9.07039921799048	0.215355676021839\\
9.08730905189281	0.215936028681156\\
9.10421888579513	0.216516022685976\\
9.12121978296026	0.217098775725482\\
9.13822068012539	0.217681160625055\\
9.15522157729052	0.218263174603399\\
9.17222247445565	0.218844814900522\\
9.18931541587169	0.219429224758466\\
9.20640835728774	0.220013251347549\\
9.22350129870378	0.220596891927543\\
9.24059424011982	0.221180143779599\\
9.25778022278737	0.221766175788497\\
9.27496620545493	0.222351809378872\\
9.29215218812249	0.2229370418523\\
9.30933817079004	0.223521870531802\\
9.32661820779546	0.224109490026911\\
9.34389824480088	0.224696695933806\\
9.3611782818063	0.225283485596594\\
9.37845831881172	0.225869856380891\\
9.39583344002363	0.226459028716115\\
9.41320856123555	0.227047772274051\\
9.43058368244747	0.227636084442067\\
9.44795880365938	0.228223962629097\\
9.46543005570071	0.228814653167785\\
9.48290130774205	0.229404899721363\\
9.50037255978338	0.229994699721182\\
9.51784381182471	0.230584050620219\\
9.53541225878365	0.23117622475068\\
9.55298070574258	0.231767939670035\\
9.57054915270152	0.232359192854348\\
9.58811759966045	0.232949981801359\\
9.60578432346384	0.233543604940388\\
9.62345104726722	0.234136753624688\\
9.64111777107061	0.234729425375757\\
9.658784494874	0.235321617736819\\
9.67655059520445	0.23591665531751\\
9.6943166955349	0.236511203182751\\
9.71208279586536	0.237105258900199\\
9.72984889619581	0.237698820059278\\
9.74771549159443	0.238295237558256\\
9.76558208699305	0.23889115006442\\
9.78344868239167	0.239486555192311\\
9.80131527779029	0.240081450578272\\
9.81928350543996	0.240679213498113\\
9.83725173308963	0.241276456131588\\
9.8552199607393	0.241873176140845\\
9.87318818838897	0.242469371209865\\
9.89125920482513	0.243068445092206\\
9.90933022126129	0.243666983378843\\
9.92740123769745	0.244264983780253\\
9.94547225413361	0.244862444028773\\
9.96364723590061	0.245462794464943\\
9.9818222176676	0.246062593980634\\
9.99999719943459	0.246661840335375\\
10.0181721812016	0.247260531310576\\
10.0364523247657	0.247862123927111\\
10.0547324683298	0.248463150283299\\
10.0730126118939	0.249063608188443\\
10.0912927554579	0.249663495473747\\
10.1096792783833	0.250266295959176\\
10.1280658013086	0.250868514829558\\
10.146452324234	0.251470149944694\\
10.1648388471593	0.252071199186297\\
10.1833329877765	0.252675173267662\\
10.2018271283938	0.25327855036472\\
10.220321269011	0.253881328388493\\
10.2388154096282	0.254483505271927\\
10.257418428294	0.255088618743103\\
10.2760214469598	0.255693119846336\\
10.2946244656257	0.256297006544597\\
10.3132274842915	0.256900276822782\\
10.3319406634311	0.257506495532241\\
10.3506538425707	0.258112086475919\\
10.3693670217104	0.258717047669464\\
10.38808020085	0.259321377150443\\
10.4069048454226	0.259928667002447\\
10.4257294899953	0.26053531367677\\
10.4445541345679	0.261141315242462\\
10.4633787791406	0.261746669790487\\
10.4823162175897	0.262354996761314\\
10.5012536560388	0.262962665128564\\
10.5201910944879	0.263569673015422\\
10.539128532937	0.264176018566974\\
10.5581801172831	0.264785348692791\\
10.5772317016292	0.265394004775202\\
10.5962832859753	0.266001984992255\\
10.6153348703214	0.266609287543888\\
10.6345019773376	0.267219586942849\\
10.6536690843538	0.267829196844612\\
10.67283619137	0.268438115482828\\
10.6920032983861	0.269046341113013\\
10.7112873293394	0.269657575960764\\
10.7305713602926	0.27026810584356\\
10.7498553912458	0.270877929051389\\
10.769139422199	0.271487043896079\\
10.7885418041401	0.27209918045019\\
10.8079441860811	0.27271059655752\\
10.8273465680222	0.273321290565137\\
10.8467489499633	0.273931260841918\\
10.8662711360792	0.274544265435608\\
10.8857933221952	0.275156534086503\\
10.9053155083112	0.275768065199493\\
10.9248376944271	0.276378857201242\\
10.9444811647161	0.276992696246739\\
10.9641246350051	0.277605783839064\\
10.9837681052941	0.278218118441684\\
11.0034115755831	0.27882969853979\\
11.02317783773	0.279444338537184\\
11.0429440998769	0.280058211556443\\
11.0627103620238	0.280671316120357\\
11.0824766241707	0.281283650773396\\
11.102367213876	0.281899058301669\\
11.1222578035813	0.282513683312006\\
11.1421483932866	0.28312752438728\\
11.1620389829919	0.283740580131987\\
11.182055464688	0.284356721851533\\
11.202071946384	0.28497206549821\\
11.22208842808	0.285586609715734\\
11.2421049097761	0.286200353169388\\
11.2622488777205	0.286817195835306\\
11.282392845665	0.28743322485792\\
11.3025368136094	0.288048438942558\\
11.3226807815539	0.288662836816051\\
11.3429428732039	0.289280012832919\\
11.363204964854	0.289896360519393\\
11.3834670565041	0.290511878645167\\
11.4037291481542	0.291126566001324\\
11.4240073959591	0.291740910530865\\
11.444285643764	0.292354420606614\\
11.464563891569	0.292967095080658\\
11.4848421393739	0.293578932825928\\
11.5051410232308	0.294190554089775\\
11.5254399070878	0.294801334723371\\
11.5457387909448	0.29541127365851\\
11.5660376748017	0.296020369847311\\
11.5863613245615	0.296629363852137\\
11.6066849743212	0.297237511011533\\
11.627008624081	0.297844810334569\\
11.6473322738407	0.298451260850143\\
11.6676846007299	0.299057715522398\\
11.6880369276191	0.299663317100338\\
11.7083892545083	0.300268064668089\\
11.7287415813975	0.300871957329124\\
11.7491263024508	0.30147595335852\\
11.7695110235041	0.302079090015263\\
11.7898957445574	0.302681366456496\\
11.8102804656107	0.303282781858253\\
11.8307011258767	0.303884393457678\\
11.8511217861426	0.304485139380183\\
11.8715424464086	0.305085018854057\\
11.8919631066745	0.305684031126044\\
11.9124230970895	0.306283326648057\\
11.9328830875046	0.306881750165986\\
11.9533430779196	0.307479300977546\\
11.9738030683347	0.308075978398491\\
11.9943056417768	0.30867302088587\\
12.014808215219	0.309269185021686\\
12.0353107886612	0.309864470171504\\
12.0558133621033	0.310458875718519\\
12.0763616495731	0.311053723446752\\
12.0969099370428	0.311647686457664\\
12.1174582245125	0.31224076418321\\
12.1380065119823	0.312832956072582\\
12.1586035338403	0.313425662937798\\
12.1792005556983	0.314017478703467\\
12.1997975775563	0.314608402866592\\
12.2203945994143	0.315198434941031\\
12.2410432774846	0.315789050864844\\
12.2616919555548	0.316378769291871\\
12.2823406336251	0.316967589782918\\
12.3029893116954	0.31755551191528\\
12.3236924799435	0.318144083200052\\
12.3443956481917	0.318731750576922\\
12.3650988164398	0.319318513669359\\
12.385801984688	0.319904372116958\\
12.406562398648	0.320490941757698\\
12.4273228126081	0.321076601066439\\
12.4480832265682	0.321661349728248\\
12.4688436405282	0.322245187443973\\
12.4896639845707	0.322829795372651\\
12.5104843286133	0.323413486533017\\
12.5313046726558	0.323996260670759\\
12.5521250166983	0.324578117547009\\
12.5730079121072	0.325160800894363\\
12.5938908075161	0.325742561025414\\
12.614773702925	0.326323397745568\\
12.6356565983338	0.326903310875342\\
12.6566046110283	0.32748410421671\\
12.6775526237227	0.328063967882399\\
12.6985006364172	0.328642901736697\\
12.7194486491116	0.329220905658681\\
12.7404642942679	0.329799841168486\\
12.7614799394242	0.330377840532141\\
12.7824955845805	0.330954903672045\\
12.8035112297368	0.33153103052507\\
12.8245969793381	0.332108138208471\\
12.8456827289394	0.332684303264147\\
12.8667684785407	0.3332595256719\\
12.887854228142	0.333833805425694\\
12.9090125143938	0.334409113231701\\
12.9301708006455	0.334983471917347\\
12.9513290868973	0.335556881519177\\
12.972487373149	0.336129342087599\\
12.9937205946389	0.336702876094323\\
13.0149538161287	0.337275454476794\\
13.0361870376186	0.337847077327701\\
13.0574202591085	0.338417744753297\\
13.0787307850171	0.338989529295907\\
13.1000413109258	0.339560351698852\\
13.1213518368345	0.340130212110413\\
13.1426623627431	0.340699110692141\\
13.1640525369264	0.341269168484334\\
13.1854427111097	0.341838257609583\\
13.206832885293	0.342406378271253\\
13.2282230594763	0.342973530685694\\
13.2496952043377	0.343541882932585\\
13.271167349199	0.344109259972952\\
13.2926394940603	0.344675662064782\\
13.3141116389217	0.345241089478767\\
13.3356680588802	0.345807755976254\\
13.3572244788388	0.346373440714839\\
13.3787808987974	0.346938144006714\\
13.400337318756	0.347501866176493\\
13.4219803036483	0.348066865406313\\
13.4436232885406	0.348630876311491\\
13.4652662734329	0.349193899258036\\
13.4869092583252	0.349755934624114\\
13.5086410871513	0.350319283855272\\
13.5303729159775	0.350881638181993\\
13.5521047448036	0.351442998023767\\
13.5738365736297	0.35200336381197\\
13.5956595174591	0.352565079177516\\
13.6174824612885	0.353125793044041\\
13.639305405118	0.353685505884208\\
13.6611283489474	0.354244218182301\\
13.6830446722324	0.354804314713285\\
13.7049609955174	0.355363403135216\\
13.7268773188025	0.35592148397366\\
13.7487936420875	0.35647855776554\\
13.770805608124	0.357037049529629\\
13.7928175741606	0.357594526558222\\
13.8148295401972	0.358150989429551\\
13.8368415062338	0.358706438732941\\
13.858951377984	0.359263338851208\\
13.8810612497342	0.359819217590269\\
13.9031711214845	0.360374075580816\\
13.9252809932347	0.360927913464373\\
13.9474910360997	0.361483234176776\\
13.9697010789646	0.362037526848064\\
13.9919111218296	0.362590792161213\\
14.0141211646945	0.363143030809775\\
14.0364336494706	0.363696783542962\\
14.0587461342466	0.364249501553891\\
14.0810586190226	0.364801185577675\\
14.1033711037986	0.365351836359753\\
14.1257883079129	0.365904031750051\\
14.1482055120272	0.366455185716726\\
14.1706227161414	0.367005299046918\\
14.1930399202557	0.367554372537835\\
14.2155641320733	0.368105020528655\\
14.2380883438909	0.368654620373046\\
14.2606125557085	0.369203172910084\\
14.283136767526	0.369750678988663\\
14.3057702854045	0.37029978879657\\
14.3284038032829	0.370847843712818\\
14.3510373211614	0.37139484462836\\
14.3736708390398	0.371940792443718\\
14.3964159769655	0.372488372684578\\
14.4191611148912	0.373034891264772\\
14.4419062528169	0.373580349127098\\
14.4646513907425	0.374124747223676\\
14.4875104777278	0.374670805884329\\
14.510369564713	0.375215796090382\\
14.5332286516983	0.375759718836474\\
14.5560877386835	0.376302575126318\\
14.5790631229979	0.376847119652075\\
14.6020385073122	0.377390588902975\\
14.6250138916265	0.377932983925517\\
14.6479892759408	0.378474305775027\\
14.6710833264109	0.379017343084262\\
14.6941773768809	0.379559298270678\\
14.7172714273509	0.380100172432684\\
14.740365477821	0.380639966677264\\
14.7635805844029	0.381181503158249\\
14.7867956909849	0.381721950639461\\
14.8100107975669	0.382261310271286\\
14.8332259041488	0.382799583212445\\
14.8565644820063	0.383339624801246\\
14.8799030598637	0.383878570482792\\
14.9032416377212	0.384416421459546\\
14.9265802155787	0.384953178942058\\
14.9500447068131	0.385491731142674\\
14.9735091980476	0.386029180496466\\
14.996973689282	0.386565528258098\\
15.0204381805165	0.387100775690071\\
15.0440310551214	0.387637843575619\\
15.0676239297264	0.388173801641177\\
15.0912168043314	0.388708651193755\\
15.1148096789363	0.389242393547956\\
15.1385334385982	0.389777981825625\\
15.16225719826	0.390312453274807\\
15.1859809579218	0.390845809255033\\
15.2097047175837	0.391378051133179\\
15.2335618956234	0.391912164119237\\
15.2574190736632	0.392445153231391\\
15.281276251703	0.39297701988189\\
15.3051334297427	0.393507765490084\\
15.3291265950906	0.394040407175023\\
15.3531197604384	0.394571917901895\\
15.3771129257863	0.395102299135889\\
15.4011060911341	0.395631552349048\\
15.4252378496272	0.39616272640004\\
15.4493696081203	0.396692762368239\\
15.4735013666134	0.397221661772026\\
15.4976331251065	0.397749426136383\\
15.5219061221034	0.39827913592761\\
15.5461791191004	0.398807700468839\\
15.5704521160973	0.399335121331912\\
15.5947251130942	0.399861400095029\\
15.619142035055	0.400389648713367\\
15.6435589570157	0.400916744870088\\
15.6679758789765	0.401442690190801\\
15.6923928009372	0.401967486307215\\
15.7169563780926	0.402494276580128\\
15.741519955248	0.403019907133353\\
15.7660835324033	0.403544379646587\\
15.7906471095587	0.404067695805379\\
15.8153601179615	0.404593030313342\\
15.8400731263644	0.405117197795001\\
15.8647861347672	0.405640199984494\\
15.88949914317	0.406162038621559\\
15.9143644081624	0.406685919738641\\
15.9392296731548	0.407208626471981\\
15.9640949381471	0.407730160610541\\
15.9889602031395	0.408250523948625\\
16.0139806003288	0.408772953827883\\
16.039000997518	0.409294201912925\\
16.0640213947072	0.409814270047942\\
16.0890417918965	0.410333160082209\\
16.1142202516767	0.410854140713971\\
16.1393987114569	0.411373932085512\\
16.1645771712372	0.411892536096685\\
16.1897556310174	0.41240995465217\\
16.2150951374624	0.412929487802655\\
16.2404346439074	0.413447824169143\\
16.2657741503524	0.413964965707616\\
16.2911136567974	0.414480914378621\\
16.3166172552097	0.414999001706186\\
16.342120853622	0.41551588466542\\
16.3676244520343	0.416031565268927\\
16.3931280504465	0.416546045533608\\
16.4187988465862	0.417062688529975\\
16.4444696427259	0.417578119510561\\
16.4701404388656	0.418092340545114\\
16.4958112350053	0.418605353707412\\
16.5216524004798	0.419120553763222\\
16.5474935659542	0.419634534089786\\
16.5733347314287	0.42014729681455\\
16.5991758969032	0.420658844068722\\
16.6251906694304	0.421172602432692\\
16.6512054419577	0.421685133285182\\
16.677220214485	0.422196438811924\\
16.7032349870123	0.422706521202138\\
16.7294266758655	0.423218839041968\\
16.7556183647187	0.423729921516225\\
16.781810053572	0.424239770869547\\
16.8080017424252	0.42474838934978\\
16.8343737319713	0.425259267770895\\
16.8607457215174	0.425768902897244\\
16.8871177110636	0.42627729703302\\
16.9134897006097	0.426784452485349\\
16.9400454511807	0.427293892495422\\
16.9666012017517	0.427802081203294\\
16.9931569523226	0.428309020973408\\
17.0197127028936	0.42881471417285\\
17.0464557582069	0.42932271677051\\
17.0731988135202	0.429829459976657\\
17.0999418688336	0.430334946216701\\
17.1266849241469	0.430839177918416\\
17.1536189111414	0.431345744032075\\
17.1805528981359	0.4318510425796\\
17.2074868851304	0.432355076048137\\
17.2344208721249	0.432857846926902\\
17.2615495082445	0.433362977493246\\
17.2886781443641	0.433866832229628\\
17.3158067804836	0.434369413685731\\
17.3429354166032	0.434870724413012\\
17.3702625121198	0.435374420357006\\
17.3975896076364	0.435876832114098\\
17.424916703153	0.436377962297353\\
17.4522437986696	0.436877813521306\\
17.4797732620512	0.437380075790126\\
17.5073027254328	0.437881045417811\\
17.5348321888143	0.438380725081693\\
17.5623616521959	0.43887911746027\\
17.5900974943972	0.439379947032887\\
17.6178333365984	0.439879475408496\\
17.6455691787997	0.440377705329625\\
17.6733050210009	0.440874639539658\\
17.7012513604556	0.441374037445287\\
17.7291976999103	0.441872125491817\\
17.757144039365	0.442368906487953\\
17.7850903788197	0.442864383242937\\
17.8132514459017	0.443362350565704\\
17.8414125129837	0.443858999256328\\
17.8695735800657	0.444354332190719\\
17.8977346471477	0.444848352244998\\
17.9261147917276	0.445344890178957\\
17.9544949363076	0.445840100591604\\
17.9828750808876	0.446333986427127\\
18.0112552254676	0.446826550629601\\
18.0398589202161	0.447321660453082\\
18.0684626149646	0.447815433744748\\
18.097066309713	0.448307873518199\\
18.1256700044615	0.448798982786586\\
18.1545018545664	0.449292665943913\\
18.1833337046712	0.449785003431768\\
18.212165554776	0.450275998334352\\
18.2409974048809	0.450765653735073\\
18.2700621491713	0.451257911763527\\
18.2991268934616	0.451748814852077\\
18.328191637752	0.452238366156774\\
18.3572563820424	0.452726568832525\\
18.3865589068877	0.453217403503261\\
18.4158614317331	0.453706873824295\\
18.4451639565785	0.454194983024839\\
18.4744664814239	0.4546817343326\\
18.5040118216459	0.455171147562833\\
18.5335571618679	0.45565918688786\\
18.56310250209	0.456145855611429\\
18.592647842312	0.456631157035417\\
18.622441195706	0.457119151023872\\
18.6522345491	0.457605761398575\\
18.682027902494	0.458090991539258\\
18.711821255888	0.458574844823406\\
18.74186798467	0.45906142195995\\
18.771914713452	0.459546605614059\\
18.801961442234	0.46003039924297\\
18.8320081710161	0.460512806301289\\
18.8623138179204	0.460997969311015\\
18.8926194648247	0.461481728801293\\
18.922925111729	0.461964088308465\\
18.9532307586333	0.462445051365845\\
18.9838010512769	0.462928803251853\\
19.0143713439205	0.463411141404887\\
19.0449416365641	0.463892069442076\\
19.0755119292076	0.46437159097712\\
19.1063527942164	0.464853935112972\\
19.1371936592252	0.465334855116764\\
19.1680345242339	0.465814354688182\\
19.1988753892427	0.466292437523071\\
19.2299929618915	0.466773377662566\\
19.2611105345403	0.467252883075875\\
19.2922281071891	0.467730957547112\\
19.3233456798379	0.468207604856119\\
19.3547463174076	0.468687145191388\\
19.3861469549774	0.469165240001094\\
19.4175475925471	0.469641893155745\\
19.4489482301169	0.470117108521135\\
19.4806385226382	0.47059525368907\\
19.5123288151595	0.471071942316132\\
19.5440191076808	0.471547178361292\\
19.5757094002021	0.472020965778362\\
19.6076961882754	0.472497720963095\\
19.6396829763487	0.472973008363937\\
19.671669764422	0.473446832030518\\
19.7036565524953	0.473919196006837\\
19.7359469398787	0.47439456694843\\
19.768237327262	0.474868458623074\\
19.8005277146453	0.475340875173366\\
19.8328181020287	0.475811820735794\\
19.8654194755566	0.47628581383443\\
19.8980208490846	0.476758315929477\\
19.9306222226126	0.477229331258948\\
19.9632235961405	0.477698864054245\\
19.9961436388709	0.478171486358573\\
20.0290636816012	0.478642605655017\\
20.0619837243316	0.479112226279589\\
20.0949037670619	0.479580352561177\\
20.1281504830269	0.480051611909759\\
20.1613971989918	0.480521355962876\\
20.1946439149567	0.480989589157279\\
20.2278906309217	0.481456315922065\\
20.2614723645969	0.481926220987445\\
20.2950540982721	0.482394598169848\\
20.3286358319473	0.482861452009667\\
20.3622175656225	0.483326787039093\\
20.3961430262838	0.483795347421728\\
20.4300684869452	0.484262367015884\\
20.4639939476066	0.48472785046868\\
20.497919408268	0.485191802418464\\
20.5321976950164	0.485659028727146\\
20.5664759817649	0.486124701004392\\
20.6007542685134	0.486588824007328\\
20.6350325552618	0.487051402483711\\
20.6696731863085	0.487517306450185\\
20.7043138173553	0.487981642783623\\
20.738954448402	0.488444416354639\\
20.7735950794487	0.488905632023866\\
20.8086080231198	0.489370226612937\\
20.8436209667909	0.489833239585718\\
20.878633910462	0.490294675930031\\
20.9136468541332	0.490754540623072\\
20.949042565344	0.491217840185363\\
20.9844382765548	0.491679543741231\\
21.0198339877656	0.492139656399665\\
21.0552296989764	0.492598183258365\\
21.0910191546516	0.493060203635191\\
21.1268086103268	0.493520613181305\\
21.162598066002	0.4939794171311\\
21.1983875216773	0.494436620706989\\
21.234582265905	0.494897379426164\\
21.2707770101326	0.495356512026084\\
21.3069717543603	0.495814023871083\\
21.343166498588	0.496269920312788\\
21.3797786897193	0.49672943676604\\
21.4163908808506	0.497187311314252\\
21.453003071982	0.497643549456562\\
21.4896152631133	0.498098156678641\\
21.5266577262528	0.49855645232125\\
21.5637001893923	0.499013089739546\\
21.6007426525319	0.499468074572694\\
21.6377851156714	0.499921412445599\\
21.6752714017348	0.500378511026388\\
21.7127576877983	0.500833934490139\\
21.7502439738617	0.50128768862167\\
21.7877302599251	0.501739779190703\\
21.8256747161126	0.502195707062926\\
21.8636191723001	0.502649942307379\\
21.9015636284876	0.5031024908606\\
21.9395080846751	0.503553358643158\\
21.9779259262067	0.504008145022736\\
22.0163437677384	0.504461220596711\\
22.0547616092701	0.504912591459901\\
22.0931794508017	0.505362263690232\\
22.1320868496398	0.505815941043507\\
22.1709942484778	0.506267888691263\\
22.2099016473158	0.506718112893712\\
22.2488090461539	0.5071666198932\\
22.2882232273236	0.507619224326258\\
22.3276374084934	0.508070079370466\\
22.3670515896632	0.508519191459159\\
22.4064657708329	0.508966567006777\\
22.4464051256478	0.509418138752765\\
22.4863444804626	0.509867940574192\\
22.5262838352775	0.510315979085935\\
22.5662231900924	0.510762260882893\\
22.6067074061995	0.511212844866395\\
22.6471916223066	0.511661637459395\\
22.6876758384138	0.512108645467522\\
22.7281600545209	0.512553875675272\\
22.7692102626757	0.513003522132908\\
22.8102604708306	0.51345135471594\\
22.8513106789854	0.513897380430848\\
22.8923608871403	0.514341606261753\\
22.9339998357608	0.514790371518243\\
22.9756387843813	0.515237299297708\\
23.0172777330018	0.515682396818596\\
23.0589166816223	0.516125671275688\\
23.1011689388352	0.516573618639194\\
23.143421196048	0.517019703690464\\
23.1856734532609	0.517463933872199\\
23.2279257104737	0.51790631660203\\
23.2707449436263	0.518352758362857\\
23.3135641767789	0.518797318018003\\
23.3563834099314	0.519240003208987\\
23.399202643084	0.519680821550942\\
23.4425351018712	0.520125030692615\\
23.4858675606585	0.520567343574594\\
23.5292000194457	0.521007768006662\\
23.5725324782329	0.52144631177101\\
23.6163923812599	0.521888286290115\\
23.6602522842869	0.522328350036725\\
23.704112187314	0.522766510994405\\
23.747972090341	0.523202777117863\\
23.7923730104802	0.523642502704116\\
23.8367739306194	0.524080302737128\\
23.8811748507586	0.524516185379378\\
23.9255757708978	0.524950158763182\\
23.9705318120247	0.525387620960144\\
24.0154878531517	0.525823142543764\\
24.0604438942787	0.526256731860835\\
24.1053999354057	0.526688397226617\\
24.1509257391711	0.527123581217141\\
24.1964515429365	0.527556809244768\\
24.2419773467019	0.527988089846253\\
24.2875031504673	0.528417431525404\\
24.3336139249026	0.528850322125406\\
24.3797246993379	0.529281241112894\\
24.4258354737732	0.529710197220526\\
24.4719462482085	0.530137199146522\\
24.518657797687	0.530567780793143\\
24.5653693471654	0.530996374866498\\
24.6120808966439	0.531422990301367\\
24.6587924461223	0.531847635996549\\
24.7061212033434	0.532275892735997\\
24.7534499605645	0.532702145618693\\
24.8007787177855	0.533126403788083\\
24.8481074750066	0.533548676350011\\
24.8960705351928	0.533974591825228\\
24.944033595379	0.53439848682534\\
24.9919966555653	0.534820370709322\\
25.0399597157515	0.535240252796858\\
25.0885748730688	0.535663810233492\\
25.1371900303861	0.536085330229083\\
25.1858051877034	0.536504822365358\\
25.2344203450208	0.536922296182985\\
25.2837061315614	0.537343478374969\\
25.332991918102	0.537762605799066\\
25.3822777046427	0.538179688267349\\
25.4315634911833	0.53859473554898\\
25.4815392186928	0.539013524843386\\
25.5315149462023	0.539430241668179\\
25.5814906737117	0.539844896073772\\
25.6314664012212	0.540257498065725\\
25.6821522057685	0.540673876346541\\
25.7328380103157	0.54108816406647\\
25.783523814863	0.541500371522685\\
25.8342096194102	0.541910508965464\\
25.8856265092156	0.542324457635935\\
25.9370433990211	0.542736297249302\\
25.9884602888265	0.543146038358368\\
26.0398771786319	0.543553691466901\\
26.0920470854784	0.543965191431568\\
26.144216992325	0.544374563421793\\
26.1963868991715	0.544781818255359\\
26.2485568060181	0.54518696669877\\
26.3015026403881	0.545595998343756\\
26.3544484747581	0.546002882658824\\
26.4073943091281	0.546407630736613\\
26.4603401434981	0.546810253616117\\
26.5140858539001	0.54721679678858\\
26.5678315643021	0.547621172819615\\
26.6215772747041	0.548023393087139\\
26.6753229851061	0.548423468912939\\
26.7298936218998	0.54882750289906\\
26.7844642586936	0.549229349457364\\
26.8390348954873	0.549629020262059\\
26.893605532281	0.550026526928605\\
26.9490273164448	0.550428030430224\\
27.0044491006085	0.550827325722894\\
27.0598708847722	0.551224424788769\\
27.1152926689359	0.551619339548489\\
27.1715930662283	0.552018290658198\\
27.2278934635207	0.552415012262148\\
27.2841938608131	0.552809516662767\\
27.3404942581055	0.553201816098054\\
27.3972810276665	0.553595283307494\\
27.4540677972275	0.553986532101163\\
27.5108545667884	0.554375574838622\\
27.5676413363494	0.554762423813882\\
27.6246835621125	0.555148816777874\\
27.6817257878755	0.555533020865128\\
27.7387680136386	0.555915048337423\\
27.7958102394016	0.55629491139096\\
27.8531117250382	0.556674333976733\\
27.9104132106748	0.557051596896412\\
27.9677146963114	0.557426712314276\\
28.025016181948	0.557799692329023\\
28.0825792753208	0.558172237255004\\
28.1401423686935	0.558542651511315\\
28.1977054620663	0.55891094716412\\
28.2552685554391	0.559277136214016\\
28.3130956457549	0.559642895605802\\
28.3709227360708	0.560006553106694\\
28.4287498263866	0.560368120684202\\
28.4865769167025	0.560727610240299\\
28.5446704260258	0.561086675557217\\
28.602763935349	0.561443667541888\\
28.6608574446723	0.561798598062691\\
28.7189509539956	0.562151478922518\\
28.7773133374	0.562503940945063\\
28.8356757208045	0.562854357971657\\
28.8940381042089	0.563202741771128\\
28.9524004876134	0.563549104046891\\
29.0110342352743	0.56389505287672\\
29.0696679829352	0.564238984822422\\
29.1283017305961	0.564580911552923\\
29.186935478257	0.564920844671816\\
29.2458431140456	0.565260369711119\\
29.3047507498342	0.565597905751882\\
29.3636583856228	0.565933464362825\\
29.4225660214114	0.566267057047431\\
29.4817501049045	0.56660024699982\\
29.5409341883977	0.566931475611344\\
29.6001182718908	0.567260754350266\\
29.659302355384	0.567588094619725\\
29.7187654815208	0.567915037477732\\
29.7782286076576	0.568240046423213\\
29.8376917337944	0.568563132823769\\
29.8971548599312	0.568884307982006\\
29.9568996607722	0.569205091027218\\
30.0166444616132	0.569523967357577\\
30.0763892624543	0.569840948239865\\
30.1361340632953	0.570156044876012\\
30.1961632086509	0.570470754672992\\
30.2561923540065	0.570783584720965\\
30.316221499362	0.571094546185781\\
30.3762506447176	0.571403650168595\\
30.436566841334	0.571712372552023\\
30.4968830379503	0.572019241919548\\
30.5571992345667	0.572324269336018\\
30.617515431183	0.572627465801745\\
30.6781214246657	0.572930285878793\\
30.7387274181483	0.573231279439452\\
30.799333411631	0.573530457447529\\
30.8599394051136	0.573827830802477\\
30.9208379817972	0.574124832954471\\
30.9817365584807	0.574420034855216\\
31.0426351351642	0.574713447367485\\
31.1035337118477	0.575005081289882\\
31.1647276970968	0.575296349156654\\
31.2259216823458	0.575585842802441\\
31.2871156675948	0.575873572989017\\
31.3483096528438	0.576159550414179\\
31.4098019130839	0.576445166897814\\
31.4712941733241	0.576729034955384\\
31.5327864335642	0.577011165247734\\
31.5942786938043	0.577291568371933\\
31.656072138367	0.577571615639135\\
31.7178655829298	0.577849940039441\\
31.7796590274925	0.578126552132864\\
31.8414524720552	0.578401462415855\\
31.903550051626	0.578676021884984\\
31.9656476311968	0.578948883810467\\
32.0277452107676	0.57922005865161\\
32.0898427903383	0.579489556804379\\
32.1522474993513	0.579758709151334\\
32.2146522083642	0.580026189041811\\
32.2770569173772	0.580292006834563\\
32.3394616263901	0.580556172825232\\
32.4021765035977	0.580819997980928\\
32.4648913808053	0.581082175531181\\
32.527606258013	0.581342715734371\\
32.5903211352206	0.581601628786004\\
32.6533492654047	0.581860205938959\\
32.7163773955889	0.582117160101368\\
32.7794055257731	0.582372501431444\\
32.8424336559572	0.582626240024764\\
32.9057781690946	0.582879647613267\\
32.969122682232	0.583131456590157\\
33.0324671953693	0.583381677013697\\
33.0958117085067	0.583630318879769\\
33.1594757818323	0.5838786345966\\
33.2231398551579	0.584125375844965\\
33.2868039284835	0.58437055258343\\
33.3504680018091	0.584614174708427\\
33.4144548595075	0.584857475496079\\
33.478441717206	0.585099225722952\\
33.5424285749044	0.585339435248172\\
33.6064154326028	0.585578113868999\\
33.6707283499096	0.58581647593211\\
33.7350412672163	0.586053311106897\\
33.7993541845231	0.586288629153339\\
33.8636671018298	0.586522439769805\\
33.9283094023156	0.586755938561405\\
33.9929517028013	0.586987933902392\\
34.057594003287	0.587218435453892\\
34.1222363037727	0.587447452815693\\
34.1872113622324	0.587676163045382\\
34.2521864206922	0.58790339302787\\
34.3171614791519	0.588129152325749\\
34.3821365376116	0.588353450440544\\
34.447447781095	0.588577446075385\\
34.5127590245783	0.588799984432645\\
34.5780702680616	0.589021074976713\\
34.643381511545	0.589240727111191\\
34.7090324201268	0.589460081375873\\
34.7746833287086	0.589678001099377\\
34.8403342372905	0.589894495648235\\
34.9059851458723	0.590109574328474\\
34.9719792533529	0.590324359705517\\
35.0379733608335	0.590537733045209\\
35.1039674683141	0.590749703616582\\
35.1699615757948	0.590960280628451\\
35.2363024710846	0.591170568860866\\
35.3026433663745	0.591379467327853\\
35.3689842616644	0.591586985201317\\
35.4353251569542	0.591793131593235\\
35.5020164861975	0.591998993689399\\
35.5687078154407	0.59220348806083\\
35.6353991446839	0.592406623782691\\
35.7020904739271	0.592608409870508\\
35.76913594038	0.592809916101904\\
35.8361814068328	0.593010076418811\\
35.9032268732857	0.593208899800041\\
35.9702723397385	0.593406395165066\\
36.0376757049608	0.593603615068561\\
36.1050790701831	0.593799510638167\\
36.1724824354054	0.593994090756749\\
36.2398858006277	0.594187364248134\\
36.3076508877251	0.594380366634119\\
36.3754159748225	0.594572066037939\\
36.4431810619199	0.594762471246927\\
36.5109461490173	0.594951590989677\\
36.5790768402747	0.595140443933835\\
36.6472075315322	0.595328015019617\\
36.7153382227897	0.595514312939248\\
36.7834689140472	0.595699346326514\\
36.8519691566203	0.595884117184937\\
36.9204693991935	0.596067627081646\\
36.9889696417667	0.596249884614184\\
37.0574698843398	0.596430898321967\\
37.1263436888854	0.596611653725873\\
37.1952174934311	0.596791168838535\\
37.2640912979767	0.596969452163258\\
37.3329651025223	0.597146512145524\\
37.4022165438372	0.597323318002303\\
37.4714679851522	0.597498904013111\\
37.5407194264671	0.597673278587457\\
37.609970867782	0.597846450077338\\
37.6796040887873	0.598019371580191\\
37.7492373097926	0.59819109345807\\
37.8188705307979	0.598361624027142\\
37.8885037518033	0.598530971546371\\
37.9585229632751	0.59870007317263\\
38.0285421747469	0.598867995171648\\
38.0985613862187	0.599034745766706\\
38.1685805976905	0.599200333124197\\
38.2389900798134	0.599365678638633\\
38.3093995619363	0.599529864301323\\
38.3798090440592	0.599692898243125\\
38.4502185261821	0.599854788538328\\
38.5209622984945	0.600016303740198\\
38.591706070807	0.600176680580219\\
38.6624498431194	0.600335927076838\\
38.7331936154318	0.600494051192442\\
38.8042656863826	0.600651786878262\\
38.8753377573333	0.600808405682651\\
38.9464098282841	0.600963915509576\\
39.0174818992348	0.601118324207473\\
39.0888909542047	0.601272363913406\\
39.1603000091745	0.60142530773965\\
39.2317090641443	0.601577163478592\\
39.3031181191142	0.601727938867606\\
39.3748725127202	0.601878362967238\\
39.4466269063262	0.602027711738804\\
39.5183812999322	0.602175992865771\\
39.5901356935382	0.602323213977095\\
39.662243568804	0.60247010012161\\
39.7343514440698	0.602615931062783\\
39.8064593193355	0.602760714377634\\
39.8785671946013	0.602904457589153\\
39.95103651206	0.603047880917837\\
40.0235058295186	0.603190268762003\\
40.0959751469772	0.603331628594525\\
40.1684444644358	0.603471967834713\\
40.2412830193083	0.603612001144884\\
40.3141215741807	0.60375101830275\\
40.3869601290532	0.603889026679179\\
40.4597986839256	0.604026033591933\\
40.5330141309314	0.604162747510438\\
40.6062295779373	0.604298464239725\\
40.6794450249431	0.604433191050657\\
40.752660471949	0.604566935161439\\
40.8262603410054	0.604700398285358\\
40.8998602100619	0.604832882830087\\
40.9734600791183	0.604964395968359\\
41.0470599481748	0.605094944820682\\
41.1210516636287	0.605225223853693\\
41.1950433790826	0.605354542579144\\
41.2690350945364	0.605482908073394\\
41.3430268099903	0.605610327361\\
41.4174177047231	0.60573748723004\\
41.4918085994559	0.60586370473824\\
41.5661994941887	0.605988986867232\\
41.6405903889215	0.606113340547259\\
41.7153877210576	0.606237444514743\\
41.7901850531936	0.606360623755528\\
41.8649823853296	0.60648288515807\\
41.9397797174656	0.606604235559845\\
42.0149906816255	0.606725345318161\\
42.0902016457854	0.60684554768274\\
42.1654126099454	0.606964849450323\\
42.2406235741053	0.60708325736707\\
42.316255315397	0.607201433130498\\
42.3918870566888	0.607318718542433\\
42.4675187979805	0.607435120309282\\
42.5431505392722	0.60755064508726\\
42.6192101664345	0.607665945675412\\
42.6952697935968	0.607780372673213\\
42.771329420759	0.60789393269804\\
42.8473890479213	0.608006632317463\\
42.9238836429179	0.608119115227013\\
43.0003782379145	0.608230741035164\\
43.0768728329111	0.608341516271497\\
43.1533674279077	0.60845144741617\\
43.2303040564086	0.60856116888869\\
43.3072406849096	0.608670049484804\\
43.3841773134105	0.608778095647476\\
43.4611139419115	0.608885313770619\\
43.5384996649489	0.60899232885695\\
43.6158853879863	0.609098519035496\\
43.6932711110238	0.609203890663721\\
43.7706568340612	0.609308450050402\\
43.8484987166093	0.609412812665868\\
43.9263405991575	0.609516366092844\\
44.0041824817057	0.609619116604352\\
44.0820243642538	0.609721070425098\\
44.1603294838729	0.609822833400827\\
44.2386346034919	0.60992380266457\\
44.3169397231109	0.610023984405931\\
44.39524484273	0.610123384766552\\
44.4740202988846	0.610222599897892\\
44.5527957550393	0.610321036557027\\
44.631571211194	0.610418700851111\\
44.7103466673487	0.610515598839692\\
44.7895995895968	0.610612316929593\\
44.8688525118449	0.610708271555983\\
44.948105434093	0.610803468744497\\
45.0273583563411	0.610897914473519\\
45.1070959123988	0.61099218537259\\
45.1868334684566	0.611085707591007\\
45.2665710245144	0.611178487073781\\
45.3463085805721	0.611270529719021\\
45.4265379845256	0.611362402362809\\
45.5067673884792	0.611453540887846\\
45.5869967924327	0.611543951159383\\
45.6672261963862	0.611633638996111\\
45.7479547148514	0.611723161437325\\
45.8286832333165	0.611811964105327\\
45.9094117517817	0.611900052786435\\
45.9901402702469	0.611987433220746\\
46.0713752310891	0.612074652662051\\
46.1526101919314	0.61216116646358\\
46.2338451527736	0.612246980333516\\
46.3150801136159	0.612332099934164\\
46.3968289125376	0.612417062756204\\
46.4785777114593	0.612501333863799\\
46.560326510381	0.612584918887775\\
46.6420753093027	0.612667823413415\\
46.7243454163926	0.612750575200838\\
46.8066155234824	0.612832648994803\\
46.8888856305723	0.612914050349529\\
46.9711557376621	0.612994784774027\\
47.0539547042701	0.613075370340004\\
47.136753670878	0.613155291432699\\
47.2195526374859	0.613234553530449\\
47.3023516040939	0.613313162066718\\
47.3856870701744	0.613391625476142\\
47.4690225362549	0.613469437734957\\
47.5523580023354	0.61354660424633\\
47.6356934684159	0.613623130368884\\
47.7195731680798	0.613699514958214\\
47.8034528677438	0.613775261525246\\
47.8873325674077	0.613850375398666\\
47.9712122670716	0.613924861862944\\
48.0556440359198	0.613999210260328\\
48.1400758047679	0.614072933572317\\
48.2245075736161	0.614146037053787\\
48.3089393424642	0.614218525915726\\
48.3939311234603	0.614290880058385\\
48.4789229044563	0.614362621863944\\
48.5639146854524	0.614433756514125\\
48.6489064664485	0.614504289147092\\
48.7344663172648	0.614574690298757\\
48.820026168081	0.614644491675653\\
48.9055860188973	0.614713698386992\\
48.9911458697136	0.614782315498754\\
49.0772819670679	0.614850804263414\\
49.1634180644223	0.614918705632213\\
49.2495541617767	0.614986024642482\\
49.3356902591311	0.615052766288644\\
49.4224109073406	0.615119382626801\\
49.5091315555501	0.615185423766966\\
49.5958522037596	0.61525089467521\\
49.6825728519691	0.615315800275016\\
49.769886487838	0.615380583515886\\
49.857200123707	0.615444803577897\\
49.944513759576	0.615508465356462\\
50.0318273954449	0.61557157370473\\
50.1197425956162	0.615634562559549\\
50.2076577957875	0.615697000078082\\
50.2955729959588	0.615758891085686\\
50.3834881961302	0.615820240365773\\
50.4720136825001	0.615881472938772\\
50.5605391688701	0.615942165843607\\
50.64906465524	0.616002323836166\\
50.73759014161	0.616061951630709\\
50.8267347882915	0.616121465430509\\
50.915879434973	0.616180451057828\\
51.0050240816545	0.616238913199661\\
51.094168728336	0.616296856501699\\
51.1839415683769	0.616354688451755\\
51.2737144084178	0.616412003554505\\
51.3634872484587	0.616468806428628\\
51.4532600884996	0.616525101651813\\
51.5436703203087	0.616581288099991\\
51.6340805521178	0.616636968857399\\
51.7244907839268	0.616692148474961\\
51.8149010157359	0.616746831462932\\
51.9059580087448	0.616801408190048\\
51.9970150017537	0.6168554902161\\
52.0880719947626	0.616909082024819\\
52.1791289877716	0.616962188059582\\
52.270842290771	0.617015190289035\\
52.3625555937704	0.61706770864204\\
52.4542688967698	0.617119747535686\\
52.5459821997693	0.617171311347028\\
52.6383615465513	0.617222773752216\\
52.7307408933333	0.617273762942181\\
52.8231202401153	0.617324283267916\\
52.9154995868973	0.617374339040699\\
53.0085549039533	0.617424295753204\\
53.1016102210094	0.617473789749957\\
53.1946655380654	0.617522825316403\\
53.2877208551214	0.617571406698583\\
53.3814622684928	0.617619891315371\\
53.4752036818643	0.617667923555728\\
53.5689450952358	0.617715507640089\\
53.6626865086073	0.617762647749797\\
53.7571243510415	0.617809693340185\\
53.8515621934757	0.617856296734876\\
53.9460000359099	0.617902462089829\\
54.0404378783442	0.617948193522224\\
54.1355826971555	0.617993832634801\\
54.2307275159668	0.618039039575351\\
54.325872334778	0.618083818435887\\
54.4210171535893	0.618128173269959\\
54.5168797176294	0.618172437938686\\
54.6127422816695	0.618216280303487\\
54.7086048457095	0.618259704392958\\
54.8044674097496	0.618302714197544\\
54.9010587178797	0.618345635948239\\
54.9976500260098	0.618388145109001\\
55.0942413341398	0.618430245645532\\
55.1908326422699	0.6184719414857\\
55.2881639308243	0.618513551341702\\
55.3854952193786	0.618554758169095\\
55.4828265079329	0.618595565871209\\
55.5801577964873	0.618635978313855\\
55.6782405480441	0.618676306801979\\
55.776323299601	0.618716241671556\\
55.8744060511579	0.618755786764065\\
55.9724888027147	0.61879494588378\\
56.0713347542828	0.618834023039696\\
56.1701807058509	0.618872715837257\\
56.2690266574189	0.61891102805661\\
56.367872608987	0.618948963441007\\
56.4674937609263	0.618986818814662\\
56.5671149128656	0.619024298941653\\
56.6667360648049	0.619061407541308\\
56.7663572167441	0.619098148296377\\
56.8667658411891	0.619134810956962\\
56.9671744656341	0.619171107335426\\
57.0675830900791	0.619207041090796\\
57.1679917145241	0.619242615845828\\
57.2692003658787	0.619278114387274\\
57.3704090172333	0.619313255465326\\
57.4716176685879	0.61934804267922\\
57.5728263199425	0.619382479592233\\
57.6748478434472	0.619416842137729\\
57.7768693669519	0.619450855894065\\
57.8788908904565	0.619484524401198\\
57.9809124139612	0.619517851163434\\
58.0837599560178	0.619551105370378\\
58.1866074980744	0.619584019319211\\
58.2894550401309	0.619616596491118\\
58.3923025821875	0.619648840331945\\
58.4959896007833	0.619681013396686\\
58.5996766193791	0.619712854592472\\
58.7033636379749	0.619744367342225\\
58.8070506565707	0.619775555033841\\
58.911590931702	0.619806673696201\\
59.0161312068333	0.619837468738158\\
59.1206714819646	0.619867943524874\\
59.2252117570959	0.619898101386801\\
59.3306194021433	0.6199281919347\\
59.4360270471907	0.619957966971412\\
59.541434692238	0.619987429804848\\
59.6468423372854	0.620016583708523\\
59.7351317529641	0.620040767621464\\
59.8234211686427	0.620064738956047\\
59.9117105843214	0.620088499580744\\
60	0.620112051347715\\
};
\addlegendentry{Rimossi}

\end{axis}
\end{tikzpicture}%}}
\\
\subfloat[][Nodo 3]
{\resizebox{0.45\textwidth}{!}
{% This file was created by matlab2tikz.
%
%The latest updates can be retrieved from
%  http://www.mathworks.com/matlabcentral/fileexchange/22022-matlab2tikz-matlab2tikz
%where you can also make suggestions and rate matlab2tikz.
%
\definecolor{mycolor1}{rgb}{0.00000,0.44700,0.74100}%
\definecolor{mycolor2}{rgb}{0.85000,0.32500,0.09800}%
\definecolor{mycolor3}{rgb}{0.92900,0.69400,0.12500}%
%
\begin{tikzpicture}

\begin{axis}[%
width=6.028in,
height=4.754in,
at={(1.011in,0.642in)},
scale only axis,
xmin=0,
xmax=60,
xlabel style={font=\color{white!15!black}},
xlabel={T},
ymin=0,
ymax=1,
axis background/.style={fill=white},
legend style={legend cell align=left, align=left, draw=white!15!black}
]
\addplot [color=mycolor1, line width=2.0pt]
  table[row sep=crcr]{%
0	1\\
0.000265404780368998	0.999920379622812\\
0.000530809560737996	0.999840761360777\\
0.000796214341106994	0.999761145215858\\
0.00106161912147599	0.999681531190014\\
0.00238864302332098	0.999283492915512\\
0.00371566692516597	0.998885507911571\\
0.00504269082701096	0.998487576422374\\
0.00636971472885596	0.998089698691587\\
0.0130048342380809	0.996101124907442\\
0.0196399537473059	0.994113931339345\\
0.0262750732565308	0.99212814786611\\
0.0329101927657558	0.990143804053816\\
0.0481555938880245	0.985590023476951\\
0.0634009950102933	0.981044345893974\\
0.0786463961325621	0.976507111232378\\
0.0938917972548308	0.971978651186694\\
0.112426119267152	0.96648554211062\\
0.130960441279472	0.961006444737743\\
0.149494763291793	0.955541906369534\\
0.168029085304114	0.950092457745236\\
0.18883016427751	0.943995146515351\\
0.209631243250907	0.937918191475811\\
0.230432322224303	0.931862269120601\\
0.2512334011977	0.925828032042185\\
0.273784345742275	0.919311400350685\\
0.29633529028685	0.912821764555487\\
0.318886234831425	0.906359862045494\\
0.341437179376	0.899926400477043\\
0.365441427620669	0.893110342234843\\
0.389445675865339	0.88632805818202\\
0.413449924110009	0.879580292065648\\
0.437454172354678	0.872867753563623\\
0.462718995465789	0.865841502062934\\
0.487983818576899	0.858855771478212\\
0.513248641688009	0.851911268479349\\
0.538513464799119	0.84500866272885\\
0.564903949168048	0.83784393950633\\
0.591294433536978	0.830726299306524\\
0.617684917905907	0.823656377658425\\
0.644075402274836	0.81663477141201\\
0.671492335525317	0.80939183196384\\
0.698909268775797	0.802202224607483\\
0.726326202026278	0.795066487703988\\
0.753743135276759	0.787985120383959\\
0.782111039573039	0.780715853055172\\
0.810478943869319	0.773505751151274\\
0.838846848165599	0.766355236759803\\
0.867214752461879	0.759264693156574\\
0.896474834868797	0.752014344672475\\
0.925734917275715	0.744828500921624\\
0.954994999682632	0.737707454558892\\
0.98425508208955	0.730651460668923\\
1.01436081133407	0.723459670585272\\
1.04446654057859	0.716337174404839\\
1.0745722698231	0.709284127265409\\
1.10467799906762	0.702300648658131\\
1.13559219349808	0.695202118723548\\
1.16650638792855	0.688177085226914\\
1.19742058235901	0.681225561993248\\
1.22833477678948	0.674347529675068\\
1.26002761487183	0.667372494862188\\
1.29172045295419	0.660474552899474\\
1.32341329103654	0.653653576082489\\
1.35510612911889	0.646909406438085\\
1.38755375131887	0.640083996978042\\
1.42000137351885	0.633338665658757\\
1.45244899571883	0.626673145969478\\
1.48489661791881	0.620087144354785\\
1.51811813384306	0.613426114141793\\
1.5513396497673	0.606847731928686\\
1.58456116569155	0.600351596063373\\
1.61778268161579	0.593937281193426\\
1.65178046131382	0.587457330586247\\
1.68577824101184	0.581062093168831\\
1.71977602070987	0.574751037392435\\
1.7537738004079	0.568523611491979\\
1.78853445564131	0.562242318946228\\
1.82329511087471	0.55604722137516\\
1.85805576610812	0.549937664769262\\
1.89281642134153	0.543912978510487\\
1.92833097625931	0.537844563946501\\
1.96384553117709	0.531863273773617\\
1.99936008609487	0.525968340225666\\
2.03487464101265	0.520158982585322\\
2.07113752249011	0.514314691487631\\
2.10740040396758	0.508557935154472\\
2.14366328544504	0.502887841682759\\
2.1799261669225	0.497303529863589\\
2.21693501810426	0.491691864934012\\
2.25394386928602	0.486167663817515\\
2.29095272046778	0.480729960678294\\
2.32796157164954	0.47537778395626\\
2.3657171313366	0.470004737946266\\
2.40347269102366	0.464718644157364\\
2.44122825071071	0.459518453324148\\
2.47898381039777	0.454403113932197\\
2.51748987308115	0.449272398097323\\
2.55599593576452	0.444227724775493\\
2.5945019984479	0.439267971829513\\
2.63300806113127	0.434392018208363\\
2.67227151262563	0.42950528530762\\
2.71153496411999	0.424703330499022\\
2.75079841561434	0.41998496917984\\
2.7900618671087	0.415349021002782\\
2.83009279630781	0.410706081422708\\
2.87012372550692	0.406146344280337\\
2.91015465470604	0.401668572590164\\
2.95018558390515	0.397271536606841\\
2.99099747582647	0.392870566185433\\
3.03180936774779	0.388550954214473\\
3.07262125966911	0.384311420943539\\
3.11343315159043	0.380150696650499\\
3.15504317204847	0.375988434865911\\
3.1966531925065	0.371905461016809\\
3.23826321296453	0.367900461626084\\
3.27987323342256	0.363972135830292\\
3.32230262192904	0.360044073159492\\
3.36473201043552	0.356193041702198\\
3.407161398942	0.352417702621128\\
3.44959078744848	0.348716732074459\\
3.49286538309579	0.345017285478447\\
3.5361399787431	0.341392465735002\\
3.57941457439041	0.337840916267954\\
3.62268917003772	0.334361297678769\\
3.66684010640848	0.330883973298156\\
3.71099104277924	0.32747876043342\\
3.75514197915	0.324144291586173\\
3.79929291552076	0.320879218426636\\
3.84435753680729	0.317616760654588\\
3.88942215809383	0.314423822807541\\
3.93448677938037	0.311299032426936\\
3.9795514006669	0.30824103803513\\
4.02557447530949	0.305185564123122\\
4.07159754995208	0.302196975156314\\
4.11762062459467	0.299273898777079\\
4.16364369923726	0.296414985259025\\
4.21041243310513	0.293574115841727\\
4.257181166973	0.290796765771676\\
4.30394990084087	0.288081589832912\\
4.35071863470874	0.285427266410189\\
4.39747966483664	0.282832919990706\\
4.44424069496454	0.2802968342534\\
4.49100172509244	0.27781775833243\\
4.53776275522034	0.275394464167488\\
4.58452386492069	0.273025742423808\\
4.63128497462103	0.270710414490959\\
4.67804608432138	0.268447320063311\\
4.72480719402173	0.266235320689741\\
4.77156830290013	0.264073299703239\\
4.81832941177852	0.261960162014457\\
4.86509052065691	0.25989483410002\\
4.91185162953531	0.257876263225551\\
4.95861273842228	0.255903417329157\\
5.00537384730925	0.25397528496543\\
5.05213495619622	0.252090875156618\\
5.09889606508318	0.250249216578415\\
5.14565717397096	0.248449357439398\\
5.19241828285873	0.246690365451497\\
5.2391793917465	0.244971327669425\\
5.28594050063428	0.243291349630532\\
5.33270160952073	0.241649555232897\\
5.37946271840718	0.240045086726923\\
5.42622382729363	0.238477104547841\\
5.47298493618008	0.236944786434456\\
5.51974604506734	0.235447327308005\\
5.56650715395461	0.233983939280354\\
5.61326826284187	0.232553851483472\\
5.66002937172913	0.231156309186229\\
5.70679048061676	0.22979057367583\\
5.7535515895044	0.228455922278267\\
5.80031269839203	0.22715164818843\\
5.84707380727967	0.225877059599311\\
5.89383491616663	0.224631479587297\\
5.9405960250536	0.223414246141564\\
5.98735713394057	0.222224711997401\\
6.03411824282754	0.221062243788614\\
6.08087935171415	0.21992622193753\\
6.12764046060076	0.218816040690815\\
6.17440156948738	0.217731107957776\\
6.22116267837399	0.216670844493845\\
6.26792378726144	0.215634683795874\\
6.3146848961489	0.214622072142329\\
6.36144600503635	0.213632468437871\\
6.4082071139238	0.212665343433379\\
6.45496822281113	0.211720179626983\\
6.50172933169847	0.210796471306796\\
6.5484904405858	0.209893724402858\\
6.59525154947314	0.209011455747217\\
6.64201265836067	0.208149192980847\\
6.68877376724821	0.207306474597704\\
6.73553487613575	0.20648284980454\\
6.78229598502329	0.20567787782312\\
6.82916482319815	0.204889335982544\\
6.87603366137301	0.204118676379985\\
6.92290249954786	0.20336548516626\\
6.96977133772272	0.202629357759794\\
7.01788255937152	0.201891050790031\\
7.06599378102033	0.201169890092204\\
7.11410500266913	0.200465468210331\\
7.16221622431793	0.199777387094444\\
7.21159336850751	0.199087784577821\\
7.26097051269708	0.198414574671164\\
7.31034765688666	0.19775735764743\\
7.35972480107623	0.197115743281315\\
7.41043619055615	0.196472632389825\\
7.46114758003607	0.195845174983313\\
7.511858969516	0.195232979298448\\
7.56257035899592	0.194635663160314\\
7.61468820401402	0.194036893180763\\
7.66680604903213	0.193453051890863\\
7.71892389405023	0.1928837557376\\
7.77104173906833	0.19232863083302\\
7.8246442387161	0.191772091318111\\
7.87824673836387	0.191229770859208\\
7.93184923801164	0.190701294332935\\
7.98545173765941	0.190186296347902\\
8.04062368192306	0.189669920756369\\
8.09579562618671	0.189167070470016\\
8.15096757045037	0.188677378988652\\
8.20613951471402	0.188200489601581\\
8.26297303935407	0.187722257815148\\
8.31980656399412	0.187256874116386\\
8.37664008863416	0.186803980796159\\
8.43347361327421	0.186363229983252\\
8.49206906178453	0.185921170615016\\
8.55066451029485	0.185491299228534\\
8.60925995880517	0.185073267048098\\
8.66785540731549	0.18466673517563\\
8.7283223218246	0.184258927637025\\
8.7887892363337	0.183862665591848\\
8.84925615084281	0.183477609314923\\
8.90972306535191	0.183103428990109\\
8.97139693657501	0.182732651726419\\
9.03307080779811	0.182372527034763\\
9.09474467902121	0.182022736431701\\
9.15641855024431	0.18168297089185\\
9.21815611731334	0.181352594789413\\
9.27989368438236	0.181031673016921\\
9.34163125145139	0.18071992372891\\
9.40336881852041	0.180417073443978\\
9.46537482585172	0.180121596130978\\
9.52738083318303	0.179834567625701\\
9.58938684051434	0.179555736046405\\
9.65139284784564	0.179284857000562\\
9.71505322212012	0.17901477561658\\
9.77871359639459	0.178752578715394\\
9.84237397066907	0.178498026168253\\
9.90603434494354	0.178250885148823\\
9.97140380506578	0.178004584969863\\
10.036773265188	0.177765625452221\\
10.1021427253103	0.177533778259694\\
10.1675121854325	0.177308822152785\\
10.2347010682591	0.17708456102111\\
10.3018899510858	0.176867128774367\\
10.3690788339124	0.176656308339152\\
10.436267716739	0.176451889539478\\
10.505392885882	0.176248046463589\\
10.574518055025	0.176050550210767\\
10.643643224168	0.175859194488648\\
10.712768393311	0.175673779708257\\
10.7839560546811	0.175488838505601\\
10.8551437160513	0.175309790057681\\
10.9263313774214	0.175136438395844\\
10.9975190387916	0.174968594065861\\
11.07090572606	0.174801136546676\\
11.1442924133284	0.174639144087281\\
11.2176791005968	0.174482430608949\\
11.2910657878652	0.174330816363258\\
11.3667995476522	0.174179515797712\\
11.4425333074392	0.174033277455374\\
11.5182670672261	0.173891924735859\\
11.5940008270131	0.173755287189635\\
11.6722425439244	0.173618902295892\\
11.7504842608358	0.173487200225933\\
11.8287259777471	0.173360013467346\\
11.9069676946584	0.173237180483605\\
11.9878925800102	0.173114549860981\\
12.068817465362	0.172996244763857\\
12.1497423507139	0.172882106397926\\
12.2306672360657	0.172771981774091\\
12.3144665036616	0.172662018728584\\
12.3982657712575	0.172556044753712\\
12.4820650388535	0.172453909423417\\
12.5658643064494	0.172355467950273\\
12.652747062149	0.172257155668898\\
12.7396298178485	0.172162515661939\\
12.826512573548	0.172071405541485\\
12.9133953292476	0.171983688395539\\
13.0035907048582	0.171896075434045\\
13.0937860804688	0.171811836493653\\
13.1839814560794	0.171730836914144\\
13.27417683169	0.171652947352106\\
13.3679363981715	0.171575143407781\\
13.461695964653	0.171500432724795\\
13.5554555311345	0.171428688079876\\
13.649215097616	0.171359787410741\\
13.7468155979555	0.171290959358322\\
13.8444160982949	0.171224960316652\\
13.9420165986344	0.17116167022852\\
14.0396170989738	0.1711009740448\\
14.1413635477079	0.171040342219282\\
14.2431099964419	0.170982290743427\\
14.344856445176	0.170926706475257\\
14.44660289391	0.170873481130447\\
14.5528320987134	0.170820315846984\\
14.6590613035167	0.170769496992646\\
14.7652905083201	0.170720918101486\\
14.8715197131234	0.170674477420963\\
14.9753724014636	0.170631049461392\\
15.0792250898038	0.170589485923885\\
15.183077778144	0.17054970467506\\
15.2869304664842	0.170511627108107\\
15.3888306269013	0.170475848776339\\
15.4907307873185	0.170441570679938\\
15.5926309477356	0.170408728385756\\
15.6945311081528	0.170377260157929\\
15.7977993001492	0.17034671077947\\
15.9010674921457	0.170317453291173\\
16.0043356841422	0.170289431842804\\
16.1076038761386	0.170262592934218\\
16.2124869885142	0.170236492072486\\
16.3173701008898	0.170211507096487\\
16.4222532132654	0.170187589329757\\
16.527136325641	0.170164692157418\\
16.6339006004935	0.170142386457477\\
16.7406648753461	0.170121047272233\\
16.8474291501986	0.170100631952817\\
16.9541934250511	0.170081099672673\\
17.0630876639797	0.170062046886225\\
17.1719819029082	0.170043832668741\\
17.2808761418368	0.170026419504215\\
17.3897703807653	0.170009771497202\\
17.5010359523856	0.16999351560365\\
17.6123015240059	0.169977987914482\\
17.7235670956262	0.169963155331227\\
17.8348326672466	0.169948986203599\\
17.9487100003084	0.169935139939025\\
18.0625873333702	0.169921926160561\\
18.1764646664321	0.169909315604903\\
18.2903419994939	0.169897280307986\\
18.4070757021241	0.169885512287295\\
18.5238094047543	0.169874293388258\\
18.6405431073846	0.16986359770135\\
18.7572768100148	0.169853400486347\\
18.8771200075905	0.169843425452272\\
18.9969632051661	0.169833926703911\\
19.1168064027418	0.169824881282418\\
19.2366496003175	0.169816267283936\\
19.3598678839536	0.169807838554629\\
19.4830861675897	0.169799822326585\\
19.6063044512258	0.16979219825028\\
19.7295227348619	0.169784946929851\\
19.8563979903555	0.169777850436707\\
19.9832732458491	0.169771110499881\\
20.1101485013428	0.169764709087207\\
20.2370237568364	0.169758629029753\\
20.3678580796439	0.169752678533031\\
20.4986924024514	0.169747035480456\\
20.6295267252589	0.169741683905268\\
20.7603610480664	0.169736608622748\\
20.8954809190363	0.169731641817224\\
21.0306007900061	0.169726939331065\\
21.1657206609759	0.169722487043716\\
21.3008405319458	0.169718271543358\\
21.4406014192669	0.169714146848229\\
21.5803623065881	0.169710248618147\\
21.7201231939092	0.169706564386628\\
21.8598840812304	0.169703082329405\\
22.0046756408933	0.169699676212059\\
22.1494672005562	0.169696463362408\\
22.2942587602192	0.16969343279847\\
22.4390503198821	0.169690574119828\\
22.5893023876591	0.169687778841525\\
22.7395544554361	0.169685147761148\\
22.8898065232131	0.169682671230574\\
23.0400585909901	0.169680340127733\\
23.1962481938828	0.169678061826513\\
23.3524377967755	0.169675922320624\\
23.5086273996682	0.169673913161183\\
23.6648170025609	0.169672026374356\\
23.8274766689118	0.169670183420147\\
23.9901363352628	0.169668457122489\\
24.1527960016138	0.169666840110759\\
24.3154556679648	0.169665325442367\\
24.4851833872328	0.169663847013823\\
24.6549111065008	0.169662466011027\\
24.8246388257688	0.169661176032328\\
24.9943665450368	0.16965997106065\\
25.1718378270833	0.169658795904094\\
25.3493091091298	0.169657701534499\\
25.5267803911763	0.169656682419993\\
25.7042516732229	0.16965573337302\\
25.8902342929274	0.169654808712133\\
26.0762169126319	0.169653950509183\\
26.2621995323364	0.169653154011658\\
26.4481821520409	0.169652414774012\\
26.6435543528394	0.169651695345289\\
26.838926553638	0.169651030101681\\
27.0342987544366	0.169650414987322\\
27.2296709552352	0.169649846218607\\
27.4354441469161	0.169649293412556\\
27.641217338597	0.169648784346083\\
27.8469905302779	0.16964831558404\\
28.0527637219589	0.1696478839313\\
28.2701110770545	0.169647465023311\\
28.4874584321501	0.169647081029035\\
28.7048057872458	0.169646729064196\\
28.9221531423414	0.169646406454576\\
29.1524460226538	0.16964609391248\\
29.3827389029661	0.169645808888932\\
29.6130317832784	0.16964554898614\\
29.8433246635908	0.169645311988563\\
30.088179573096	0.169645082846468\\
30.3330344826013	0.169644875085923\\
30.5778893921066	0.169644686736205\\
30.8227443016119	0.169644515983098\\
31.0840841898221	0.169644351273095\\
31.3454240780323	0.169644202907952\\
31.6067639662425	0.169644069289195\\
31.8681038544528	0.169643948951076\\
32.1482392887021	0.169643833186095\\
32.4283747229513	0.169643729685004\\
32.7085101572006	0.169643637171026\\
32.9886455914499	0.169643554478241\\
33.2903839143714	0.169643475180709\\
33.5921222372929	0.169643404889372\\
33.8938605602144	0.16964334260262\\
34.195598883136	0.169643287409717\\
34.5223935450538	0.169643234681886\\
34.8491882069717	0.169643188405061\\
35.1759828688896	0.169643147809425\\
35.5027775308075	0.169643112198198\\
35.8301107995782	0.169643080895984\\
36.1574440683489	0.169643053428213\\
36.4847773371196	0.169643029337458\\
36.8121106058903	0.169643008208994\\
37.1394383112246	0.16964298966989\\
37.4667660165588	0.169642973401646\\
37.7940937218931	0.169642959133378\\
38.1214214272273	0.169642946619492\\
38.448749190017	0.169642935638969\\
38.7760769528068	0.169642926003413\\
39.1034047155965	0.169642917552392\\
39.4307324783862	0.169642910140454\\
39.7580602405942	0.169642903636702\\
40.0853880028023	0.169642897929557\\
40.4127157650103	0.169642892923998\\
40.7400435272184	0.169642888533879\\
41.0673712894278	0.169642884681674\\
41.3946990516372	0.169642881301296\\
41.7220268138466	0.169642878336467\\
42.049354576056	0.169642875736162\\
42.3766823382676	0.169642873454465\\
42.7040101004792	0.169642871452233\\
43.0313378626908	0.169642869696133\\
43.3586656249024	0.169642868155942\\
43.6859933871135	0.169642866804466\\
44.0133211493247	0.169642865618518\\
44.3406489115359	0.169642864578356\\
44.667976673747	0.169642863666081\\
44.9953044359533	0.169642862865582\\
45.3226321981596	0.169642862163128\\
45.649959960366	0.169642861547025\\
45.9772877225723	0.16964286100667\\
46.3046154847852	0.169642860532522\\
46.6319432469981	0.169642860116448\\
46.9592710092111	0.16964285975152\\
47.286598771424	0.169642859431459\\
47.6139265336322	0.169642859150614\\
47.9412542958405	0.169642858904166\\
48.2685820580487	0.169642858688013\\
48.595909820257	0.169642858498436\\
48.926850153611	0.169642858330369\\
49.257790486965	0.169642858183098\\
49.5887308203189	0.169642858054118\\
49.9196711536729	0.169642857941163\\
50.2593448649822	0.169642857839749\\
50.5990185762914	0.169642857751191\\
50.9386922876007	0.169642857673905\\
51.2783659989099	0.16964285760646\\
51.6270508037435	0.16964285754611\\
51.9757356085771	0.169642857493598\\
52.3244204134106	0.169642857447936\\
52.6731052182442	0.169642857408233\\
53.0312891287569	0.169642857372813\\
53.3894730392697	0.169642857342108\\
53.7476569497824	0.169642857315511\\
54.1058408602952	0.169642857292474\\
54.4740437546715	0.169642857271986\\
54.8422466490478	0.169642857254295\\
55.2104495434241	0.169642857239034\\
55.5786524378004	0.169642857225868\\
55.95743714694	0.169642857214199\\
56.3362218560795	0.169642857204164\\
56.7150065652191	0.169642857195545\\
57.0937912743587	0.169642857188141\\
57.4837682235858	0.169642857181601\\
57.8737451728129	0.169642857176003\\
58.26372212204	0.169642857171215\\
58.653699071267	0.169642857167122\\
58.9902743034503	0.169642857164066\\
59.3268495356335	0.169642857161394\\
59.6634247678168	0.16964285715906\\
60	0.169642857157019\\
};
\addlegendentry{Sani}

\addplot [color=mycolor2, line width=2.0pt]
  table[row sep=crcr]{%
0	0\\
0.000265404780368998	7.96193206113447e-05\\
0.000530809560737996	0.000159234412990547\\
0.000796214341106994	0.000238845275288115\\
0.00106161912147599	0.000318451905655436\\
0.00238864302332098	0.000716421513897922\\
0.00371566692516597	0.00111428504709583\\
0.00504269082701096	0.00151204227515959\\
0.00636971472885596	0.00190969296854476\\
0.0130048342380809	0.00389634041792623\\
0.0196399537473059	0.00588029038017132\\
0.0262750732565308	0.00786151477595251\\
0.0329101927657558	0.00983998585682818\\
0.0481555938880245	0.0143753059541866\\
0.0634009950102933	0.0188956201725025\\
0.0786463961325621	0.0234006117172357\\
0.0938917972548308	0.0278899724876022\\
0.112426119267152	0.0333263477349753\\
0.130960441279472	0.0387386575339915\\
0.149494763291793	0.0441263996884948\\
0.168029085304114	0.0494890894404082\\
0.18883016427751	0.0554772244121025\\
0.209631243250907	0.06143258118428\\
0.230432322224303	0.0673545521366771\\
0.2512334011977	0.0732425547251004\\
0.273784345742275	0.0795868563467941\\
0.29633529028685	0.0858899014314326\\
0.318886234831425	0.0921510464344595\\
0.341437179376	0.0983696788740851\\
0.365441427620669	0.104941710428724\\
0.389445675865339	0.111464251351923\\
0.413449924110009	0.117936677544424\\
0.437454172354678	0.124358400326086\\
0.462718995465789	0.131061982268842\\
0.487983818576899	0.137708179130721\\
0.513248641688009	0.144296430048338\\
0.538513464799119	0.150826212412867\\
0.564903949168048	0.157583966733224\\
0.591294433536978	0.16427688937188\\
0.617684917905907	0.170904516632869\\
0.644075402274836	0.177466424537201\\
0.671492335525317	0.184213539869432\\
0.698909268775797	0.190888922777673\\
0.726326202026278	0.197492232171808\\
0.753743135276759	0.20402316693137\\
0.782111039573039	0.210704168581778\\
0.810478943869319	0.217307162734025\\
0.838846848165599	0.223831949014888\\
0.867214752461879	0.230278366230621\\
0.896474834868797	0.236845290509706\\
0.925734917275715	0.243328617402045\\
0.954994999682632	0.249728299069283\\
0.98425508208955	0.25604432518299\\
1.01436081133407	0.262455602145825\\
1.04446654057859	0.268778416707964\\
1.0745722698231	0.275012880024244\\
1.10467799906762	0.281159138366247\\
1.13559219349808	0.287378849728093\\
1.16650638792855	0.29350598002137\\
1.19742058235901	0.299540801339563\\
1.22833477678948	0.305483617921375\\
1.26002761487183	0.311480959422317\\
1.29172045295419	0.317382352888909\\
1.32341329103654	0.32318822955323\\
1.35510612911889	0.328899049375918\\
1.38755375131887	0.334647906065237\\
1.42000137351885	0.340298190960686\\
1.45244899571883	0.34585048957972\\
1.48489661791881	0.351305412426831\\
1.51811813384306	0.356790214160127\\
1.5513396497673	0.362174313973544\\
1.58456116569155	0.367458446948796\\
1.61778268161579	0.372643369285467\\
1.65178046131382	0.377847539582545\\
1.68577824101184	0.382949477650117\\
1.71977602070987	0.38795006120303\\
1.7537738004079	0.392850185067347\\
1.78853445564131	0.397757345825956\\
1.82329511087471	0.402561433393168\\
1.85805576610812	0.407263458388646\\
1.89281642134153	0.411864444427794\\
1.92833097625931	0.416461946113508\\
1.96384553117709	0.420956179262987\\
1.99936008609487	0.425348276447477\\
2.03487464101265	0.429639379090051\\
2.07113752249011	0.433917885185415\\
2.10740040396758	0.438093528274411\\
2.14366328544504	0.442167551064745\\
2.1799261669225	0.446141201010285\\
2.21693501810426	0.450094401807613\\
2.25394386928602	0.453945697414069\\
2.29095272046778	0.457696428354991\\
2.32796157164954	0.461347935890118\\
2.3657171313366	0.46497226319865\\
2.40347269102366	0.468496144493092\\
2.44122825071071	0.471921005600901\\
2.47898381039777	0.475248269211322\\
2.51748987308115	0.478542616608486\\
2.55599593576452	0.481738426450887\\
2.5945019984479	0.48483719741713\\
2.63300806113127	0.48784042134973\\
2.67227151262563	0.490805870950144\\
2.71153496411999	0.493675088515757\\
2.75079841561434	0.496449633409078\\
2.7900618671087	0.499131054660124\\
2.83009279630781	0.501770611569377\\
2.87012372550692	0.504316587506397\\
2.91015465470604	0.506770590815342\\
2.95018558390515	0.50913421623161\\
2.99099747582647	0.511452551706088\\
3.03180936774779	0.51368025269445\\
3.07262125966911	0.515818965432711\\
3.11343315159043	0.5178703195034\\
3.15504317204847	0.519873523551073\\
3.1966531925065	0.5217892868578\\
3.23826321296453	0.523619283192223\\
3.27987323342256	0.525365166861141\\
3.32230262192904	0.527060510618511\\
3.36473201043552	0.52867180885806\\
3.407161398942	0.530200753345482\\
3.44959078744848	0.531649013811561\\
3.49286538309579	0.533044721528169\\
3.5361399787431	0.534359937344547\\
3.57941457439041	0.53559636238418\\
3.62268917003772	0.536755673391454\\
3.66684010640848	0.537860703229373\\
3.71099104277924	0.5388889130037\\
3.75514197915	0.539842005509871\\
3.79929291552076	0.540721657037549\\
3.84435753680729	0.541545489094816\\
3.88942215809383	0.542296253993051\\
3.93448677938037	0.542975649514004\\
3.9795514006669	0.543585345008461\\
4.02557447530949	0.544137773912871\\
4.07159754995208	0.544620935532212\\
4.11762062459467	0.545036516991745\\
4.16364369923726	0.545386175241368\\
4.21041243310513	0.545675640125071\\
4.257181166973	0.545900383928461\\
4.30394990084087	0.546062050567971\\
4.35071863470874	0.546162252903246\\
4.39747966483664	0.546202571075644\\
4.44424069496454	0.546184577032739\\
4.49100172509244	0.546109790616896\\
4.53776275522034	0.545979701982606\\
4.58452386492069	0.545795771390092\\
4.63128497462103	0.545559430218996\\
4.67804608432138	0.545272080436278\\
4.72480719402173	0.544935095813513\\
4.77156830290013	0.544549822095754\\
4.81832941177852	0.544117577051542\\
4.86509052065691	0.543639650699098\\
4.91185162953531	0.543117306439483\\
4.95861273842228	0.542551781223272\\
5.00537384730925	0.541944285584908\\
5.05213495619622	0.54129600386617\\
5.09889606508318	0.540608095409983\\
5.14565717397096	0.539881694728435\\
5.19241828285873	0.539117911509018\\
5.2391793917465	0.538317830846842\\
5.28594050063428	0.537482514461363\\
5.33270160952073	0.536613000862731\\
5.37946271840718	0.535710305336657\\
5.42622382729363	0.534775420179512\\
5.47298493618008	0.533809315913283\\
5.51974604506734	0.532812941447954\\
5.56650715395461	0.531787224050559\\
5.61326826284187	0.530733069578577\\
5.66002937172913	0.529651363674392\\
5.70679048061676	0.528542971922137\\
5.7535515895044	0.52740873980511\\
5.80031269839203	0.526249492934282\\
5.84707380727967	0.525066038208289\\
5.89383491616663	0.5238591639636\\
5.9405960250536	0.522629639923702\\
5.98735713394057	0.521378217420353\\
6.03411824282754	0.520105630509025\\
6.08087935171415	0.5188125961116\\
6.12764046060076	0.517499813960196\\
6.17440156948738	0.516167966809323\\
6.22116267837399	0.514817721499884\\
6.26792378726144	0.513449729093914\\
6.3146848961489	0.512064624815301\\
6.36144600503635	0.51066302825161\\
6.4082071139238	0.509245544362206\\
6.45496822281113	0.507812763604884\\
6.50172933169847	0.50636526187501\\
6.5484904405858	0.504903600696462\\
6.59525154947314	0.503428328171341\\
6.64201265836067	0.501939979098395\\
6.68877376724821	0.50043907491203\\
6.73553487613575	0.498926123861879\\
6.78229598502329	0.497401621903164\\
6.82916482319815	0.495862502717243\\
6.87603366137301	0.494312740118908\\
6.92290249954786	0.49275279669739\\
6.96977133772272	0.491183123696659\\
7.01788255937152	0.48956218409036\\
7.06599378102033	0.487931919394372\\
7.11410500266913	0.486292780780921\\
7.16221622431793	0.484645207939397\\
7.21159336850751	0.482945962641166\\
7.26097051269708	0.481238735040489\\
7.31034765688666	0.479523963132277\\
7.35972480107623	0.477802073344369\\
7.41043619055615	0.476026682263871\\
7.46114758003607	0.474244657984931\\
7.511858969516	0.472456424765289\\
7.56257035899592	0.470662395223797\\
7.61468820401402	0.468812991190295\\
7.66680604903213	0.466958311124058\\
7.71892389405023	0.465098764937043\\
7.77104173906833	0.463234750844493\\
7.8246442387161	0.461313383968036\\
7.87824673836387	0.459388107732552\\
7.93184923801164	0.457459317084857\\
7.98545173765941	0.455527395231312\\
8.04062368192306	0.453536029737548\\
8.09579562618671	0.451542132884355\\
8.15096757045037	0.449546084015402\\
8.20613951471402	0.447548250704425\\
8.26297303935407	0.445488759814244\\
8.31980656399412	0.443428129099261\\
8.37664008863416	0.441366721635848\\
8.43347361327421	0.439304888715489\\
8.49206906178453	0.437179049652099\\
8.55066451029485	0.435053478481305\\
8.60925995880517	0.432928521312928\\
8.66785540731549	0.430804512471809\\
8.7283223218246	0.428614001487089\\
8.7887892363337	0.426425185436627\\
8.84925615084281	0.424238392718652\\
8.90972306535191	0.422053939961626\\
8.97139693657501	0.419828610916175\\
9.03307080779811	0.417606344515369\\
9.09474467902121	0.41538743935066\\
9.15641855024431	0.413172182834932\\
9.21815611731334	0.410958569646973\\
9.27989368438236	0.408749155644525\\
9.34163125145139	0.406544195887508\\
9.40336881852041	0.404343935604274\\
9.46537482585172	0.402139075970495\\
9.52738083318303	0.399939422987885\\
9.58938684051434	0.397745195508923\\
9.65139284784564	0.395556603636121\\
9.71505322212012	0.393315693055108\\
9.77871359639459	0.391081143603174\\
9.84237397066907	0.38885315421472\\
9.90603434494354	0.386631915350906\\
9.97140380506578	0.384358258988989\\
10.036773265188	0.382092100359132\\
10.1021427253103	0.379833618135456\\
10.1675121854325	0.3775829828225\\
10.2347010682591	0.375278054137498\\
10.3018899510858	0.372981754259926\\
10.3690788339124	0.370694241689034\\
10.436267716739	0.368415667056321\\
10.505392885882	0.366080905718559\\
10.574518055025	0.363755903019528\\
10.643643224168	0.36144079725129\\
10.712768393311	0.359135719140677\\
10.7839560546811	0.356772475260186\\
10.8551437160513	0.354420123340048\\
10.9263313774214	0.35207878132634\\
10.9975190387916	0.349748559904645\\
11.07090572606	0.347358107300461\\
11.1442924133284	0.344979688313676\\
11.2176791005968	0.342613400289811\\
11.2910657878652	0.340259333622639\\
11.3667995476522	0.337842880972138\\
11.4425333074392	0.335439616943107\\
11.5182670672261	0.333049617899304\\
11.5940008270131	0.330672953567525\\
11.6722425439244	0.328231656405513\\
11.7504842608358	0.325804721430683\\
11.8287259777471	0.323392203503813\\
11.9069676946584	0.320994151171926\\
11.9878925800102	0.318529124882893\\
12.068817465362	0.316079658361816\\
12.1497423507139	0.313645784289115\\
12.2306672360657	0.311227529365848\\
12.3144665036616	0.308739863454223\\
12.3982657712575	0.30626898463244\\
12.4820650388535	0.303814902547669\\
12.5658643064494	0.301377621216609\\
12.652747062149	0.298868395167789\\
12.7396298178485	0.296377219301633\\
12.826512573548	0.293904079180658\\
12.9133953292476	0.291448955104236\\
13.0035907048582	0.288919256184115\\
13.0937860804688	0.286408912671253\\
13.1839814560794	0.283917884766382\\
13.27417683169	0.281446127797595\\
13.3679363981715	0.278897071173874\\
13.461695964653	0.276368724027238\\
13.5554555311345	0.273861019661688\\
13.649215097616	0.271373886928018\\
13.7468155979555	0.268806638527678\\
13.8444160982949	0.266261509599191\\
13.9420165986344	0.26373840471936\\
14.0396170989738	0.26123722446712\\
14.1413635477079	0.25865302700502\\
14.2431099964419	0.256092422383121\\
14.344856445176	0.253555284280836\\
14.44660289391	0.251041482879466\\
14.5528320987134	0.248441685492433\\
14.6590613035167	0.245867025487101\\
14.7652905083201	0.243317343093898\\
14.8715197131234	0.240792475594827\\
14.9753724014636	0.238347922947531\\
15.0792250898038	0.235926771581\\
15.183077778144	0.233528861396839\\
15.2869304664842	0.231154030476649\\
15.3888306269013	0.228846123002591\\
15.4907307873185	0.226560121176269\\
15.5926309477356	0.224295866994774\\
15.6945311081528	0.22205320139511\\
15.7977993001492	0.219802288759369\\
15.9010674921457	0.217573216096225\\
16.0043356841422	0.215365814574595\\
16.1076038761386	0.21317991477478\\
16.2124869885142	0.210981665454971\\
16.3173701008898	0.208805241565975\\
16.4222532132654	0.206650463802466\\
16.527136325641	0.204517152688328\\
16.6339006004935	0.20236744099754\\
16.7406648753461	0.200239597132695\\
16.8474291501986	0.198133431280434\\
16.9541934250511	0.196048753834623\\
17.0630876639797	0.193944426109672\\
17.1719819029082	0.191862055004276\\
17.2808761418368	0.189801440030545\\
17.3897703807653	0.18776238125939\\
17.5010359523856	0.185700977025756\\
17.6123015240059	0.183661657857383\\
17.7235670956262	0.181644212302376\\
17.8348326672466	0.179648429802812\\
17.9487100003084	0.17762801286308\\
18.0625873333702	0.175629844511332\\
18.1764646664321	0.1736537019272\\
18.2903419994939	0.171699363511084\\
18.4070757021241	0.169718405462914\\
18.5238094047543	0.167759891516815\\
18.6405431073846	0.165823586967249\\
18.7572768100148	0.163909258654671\\
18.8771200075905	0.161966556467113\\
18.9969632051661	0.160046523863357\\
19.1168064027418	0.15814891362245\\
19.2366496003175	0.156273480399185\\
19.3598678839536	0.154368096702994\\
19.4830861675897	0.152485636806717\\
19.6063044512258	0.150625840225332\\
19.7295227348619	0.148788448689612\\
19.8563979903555	0.146919667722775\\
19.9832732458491	0.14507409288779\\
20.1101485013428	0.143251449560267\\
20.2370237568364	0.141451465687711\\
20.3678580796439	0.139618760348262\\
20.4986924024514	0.137809571470173\\
20.6295267252589	0.136023609276098\\
20.7603610480664	0.134260586938742\\
20.8954809190363	0.132463593932882\\
21.0306007900061	0.13069045602407\\
21.1657206609759	0.128940867114839\\
21.3008405319458	0.127214524464437\\
21.4406014192669	0.125453025511515\\
21.5803623065881	0.123715749288163\\
21.7201231939092	0.122002372036564\\
21.8598840812304	0.120312573798048\\
22.0046756408933	0.11858648148955\\
22.1494672005562	0.116885009588693\\
22.2942587602192	0.115207815139767\\
22.4390503198821	0.113554559472666\\
22.5893023876591	0.111863906811155\\
22.7395544554361	0.110198303679884\\
22.8898065232131	0.10855738616032\\
23.0400585909901	0.106940795159728\\
23.1962481938828	0.105285728100806\\
23.3524377967755	0.103656172893924\\
23.5086273996682	0.102051742628342\\
23.6648170025609	0.100472055824588\\
23.8274766689118	0.0988528283585092\\
23.9901363352628	0.0972596103929272\\
24.1527960016138	0.0956919896852907\\
24.3154556679648	0.0941495601090019\\
24.4851833872328	0.092566531577325\\
24.6549111065008	0.0910100480495009\\
24.8246388257688	0.0894796692446875\\
24.9943665450368	0.0879749617789839\\
25.1718378270833	0.0864285962462848\\
25.3493091091298	0.0849093520435786\\
25.5267803911763	0.0834167577079514\\
25.7042516732229	0.0819503495718885\\
25.8902342929274	0.0804412171752062\\
26.0762169126319	0.0789598267151353\\
26.2621995323364	0.0775056718775356\\
26.4481821520409	0.0760782551861341\\
26.6435543528394	0.074607035476423\\
26.838926553638	0.0731642266002272\\
27.0342987544366	0.071749283086771\\
27.2296709552352	0.070361669523269\\
27.4354441469161	0.0689291569007654\\
27.641217338597	0.0675257769364429\\
27.8469905302779	0.0661509399205312\\
28.0527637219589	0.0648040676421743\\
28.2701110770545	0.0634111790780992\\
28.4874584321501	0.0620482032797823\\
28.7048057872458	0.0607145002590764\\
28.9221531423414	0.0594094432444491\\
29.1524460226538	0.0580572285816927\\
29.3827389029661	0.0567357712534136\\
29.6130317832784	0.0554443737640059\\
29.8433246635908	0.054182353902474\\
30.088179573096	0.0528720091443956\\
30.3330344826013	0.0515933378729189\\
30.5778893921066	0.0503455763624138\\
30.8227443016119	0.0491279786879864\\
31.0840841898221	0.0478608628255104\\
31.3454240780323	0.0466264163861174\\
31.6067639662425	0.0454237987970064\\
31.8681038544528	0.0442521903823925\\
32.1482392887021	0.0430298464841119\\
32.4283747229513	0.0418412571518388\\
32.7085101572006	0.0406854919178096\\
32.9886455914499	0.0395616450685915\\
33.2903839143714	0.0383858265773305\\
33.5921222372929	0.0372449477808292\\
33.8938605602144	0.0361379721120593\\
34.195598883136	0.0350638926285885\\
34.5223935450538	0.0339365960865236\\
34.8491882069717	0.0328455366151846\\
35.1759828688896	0.0317895511488351\\
35.5027775308075	0.030767512509376\\
35.8301107995782	0.0297767251423153\\
36.1574440683489	0.0288178401548459\\
36.4847773371196	0.0278898317647339\\
36.8121106058903	0.0269917058865364\\
37.1394383112246	0.02612251426706\\
37.4667660165588	0.0252813103701553\\
37.7940937218931	0.0244671941723621\\
38.1214214272273	0.0236792934717006\\
38.448749190017	0.0229167634067557\\
38.7760769528068	0.022178787233021\\
39.1034047155965	0.0214645752761595\\
39.4307324783862	0.0207733622786386\\
39.7580602405942	0.0201044070942416\\
40.0853880028023	0.0194569930146475\\
40.4127157650103	0.0188304272148054\\
40.7400435272184	0.0182240382956871\\
41.0673712894278	0.0176371760150987\\
41.3946990516372	0.0170692115791435\\
41.7220268138466	0.0165195371513932\\
42.049354576056	0.0159875636953735\\
42.3766823382676	0.0154727207383149\\
42.7040101004792	0.0149744566279204\\
43.0313378626908	0.0144922381011502\\
43.3586656249024	0.0140255483897505\\
43.6859933871135	0.0135738870129456\\
44.0133211493247	0.013136770003317\\
44.3406489115359	0.012713729528152\\
44.667976673747	0.0123043122264372\\
44.9953044359533	0.011908079026963\\
45.3226321981596	0.0115246053468423\\
45.649959960366	0.0111534807591267\\
45.9772877225723	0.010794307533275\\
46.3046154847852	0.0104467004755479\\
46.6319432469981	0.0101102871034238\\
46.9592710092111	0.00978470735383904\\
47.286598771424	0.00946961230242632\\
47.6139265336322	0.00916466402348239\\
47.9412542958405	0.00886953574304957\\
48.2685820580487	0.00858391158312934\\
48.595909820257	0.00830748543764265\\
48.926850153611	0.00803705687956992\\
49.257790486965	0.0077754312446477\\
49.5887308203189	0.00752232230471196\\
49.9196711536729	0.00727745277666186\\
50.2593448649822	0.00703440810659438\\
50.5990185762914	0.00679948021791447\\
50.9386922876007	0.00657239836072517\\
51.2783659989099	0.00635290045700388\\
51.6270508037435	0.00613520200104912\\
51.9757356085771	0.00592496337833594\\
52.3244204134106	0.0057219292831018\\
52.6731052182442	0.00552585278924831\\
53.0312891287569	0.00533142844426998\\
53.3894730392697	0.00514384466063594\\
53.7476569497824	0.00496286108018382\\
54.1058408602952	0.0047882454339452\\
54.4740437546715	0.00461514719723964\\
54.8422466490478	0.00444830641651244\\
55.2104495434241	0.00428749720318605\\
55.5786524378004	0.00413250146841838\\
55.95743714694	0.00397889618179677\\
56.3362218560795	0.00383100025272828\\
56.7150065652191	0.00368860178579705\\
57.0937912743587	0.00355149639692103\\
57.4837682235858	0.00341566206440921\\
57.8737451728129	0.00328502283921547\\
58.26372212204	0.00315938034450855\\
58.653699071267	0.00303854342748146\\
58.9902743034503	0.00293797554850899\\
59.3268495356335	0.00284073614437124\\
59.6634247678168	0.00274671514774453\\
60	0.00265580600854047\\
};
\addlegendentry{Infetti}

\addplot [color=mycolor3, line width=2.0pt]
  table[row sep=crcr]{%
0	0\\
0.000265404780368998	1.05657677182056e-09\\
0.000530809560737996	4.22623230064267e-09\\
0.000796214341106994	9.50885420229752e-09\\
0.00106161912147599	1.69043304363084e-08\\
0.00238864302332098	8.55705903049682e-08\\
0.00371566692516597	2.07041332898374e-07\\
0.00504269082701096	3.81302466234169e-07\\
0.00636971472885596	6.08339868311044e-07\\
0.0130048342380809	2.53467463148024e-06\\
0.0196399537473059	5.77828048401244e-06\\
0.0262750732565308	1.03373579370142e-05\\
0.0329101927657558	1.62100893554059e-05\\
0.0481555938880245	3.4670568862416e-05\\
0.0634009950102933	6.00339335236864e-05\\
0.0786463961325621	9.22770503867659e-05\\
0.0938917972548308	0.000131376325703779\\
0.112426119267152	0.000188110154404314\\
0.130960441279472	0.000254897728265202\\
0.149494763291793	0.000331693941971203\\
0.168029085304114	0.000418452814355791\\
0.18883016427751	0.000527629072546748\\
0.209631243250907	0.000649227339909436\\
0.230432322224303	0.000783178742722171\\
0.2512334011977	0.00092941323271474\\
0.273784345742275	0.00110174330252133\\
0.29633529028685	0.00128833401308059\\
0.318886234831425	0.0014890915200467\\
0.341437179376	0.00170392064887168\\
0.365441427620669	0.00194794733643344\\
0.389445675865339	0.00220769046605715\\
0.413449924110009	0.00248303038992827\\
0.437454172354678	0.00277384611029083\\
0.462718995465789	0.00309651566822428\\
0.487983818576899	0.00343604939106698\\
0.513248641688009	0.00379230147231319\\
0.538513464799119	0.00416512485828352\\
0.564903949168048	0.00457209376044573\\
0.591294433536978	0.00499681132159646\\
0.617684917905907	0.00543910570870562\\
0.644075402274836	0.00589880405078916\\
0.671492335525317	0.00639462816672884\\
0.698909268775797	0.00690885261484436\\
0.726326202026278	0.00744128012420436\\
0.753743135276759	0.00799171268467175\\
0.782111039573039	0.00857997836304988\\
0.810478943869319	0.0091870861147014\\
0.838846848165599	0.00981281422530858\\
0.867214752461879	0.0104569406128047\\
0.896474834868797	0.0111403648178191\\
0.925734917275715	0.0118428816763308\\
0.954994999682632	0.0125642463718249\\
0.98425508208955	0.0133042141480864\\
1.01436081133407	0.0140847272689033\\
1.04446654057859	0.0148844088871976\\
1.0745722698231	0.0157029927103467\\
1.10467799906762	0.0165402129756224\\
1.13559219349808	0.0174190315483592\\
1.16650638792855	0.0183169347517166\\
1.19742058235901	0.0192336366671889\\
1.22833477678948	0.0201688524035575\\
1.26002761487183	0.0211465457154948\\
1.29172045295419	0.0221430942116163\\
1.32341329103654	0.0231581943642808\\
1.35510612911889	0.0241915441859971\\
1.38755375131887	0.0252680969567211\\
1.42000137351885	0.026363143380557\\
1.45244899571883	0.027476364450802\\
1.48489661791881	0.0286074432183838\\
1.51811813384306	0.0297836716980796\\
1.5513396497673	0.0309779540977703\\
1.58456116569155	0.0321899569878306\\
1.61778268161579	0.0334193495211074\\
1.65178046131382	0.034695129831208\\
1.68577824101184	0.0359884291810524\\
1.71977602070987	0.0372989014045348\\
1.7537738004079	0.0386262034406735\\
1.78853445564131	0.0400003352278154\\
1.82329511087471	0.0413913452316714\\
1.85805576610812	0.0427988768420917\\
1.89281642134153	0.0442225770617189\\
1.92833097625931	0.0456934899399917\\
1.96384553117709	0.0471805469633965\\
1.99936008609487	0.0486833833268573\\
2.03487464101265	0.0502016383246265\\
2.07113752249011	0.0517674233269546\\
2.10740040396758	0.0533485365711167\\
2.14366328544504	0.0549446072524953\\
2.1799261669225	0.0565552691261263\\
2.21693501810426	0.058213733258374\\
2.25394386928602	0.0598866387684157\\
2.29095272046778	0.0615736109667155\\
2.32796157164954	0.0632742801536216\\
2.3657171313366	0.0650229988550834\\
2.40347269102366	0.0667852113495447\\
2.44122825071071	0.0685605410749514\\
2.47898381039777	0.0703486168564811\\
2.51748987308115	0.0721849852941919\\
2.55599593576452	0.0740338487736203\\
2.5945019984479	0.0758948307533564\\
2.63300806113127	0.0777675604419065\\
2.67227151262563	0.0796888437422359\\
2.71153496411999	0.0816215809852215\\
2.75079841561434	0.0835653974110818\\
2.7900618671087	0.0855199243370943\\
2.83009279630781	0.0875233070079156\\
2.87012372550692	0.0895370682132653\\
2.91015465470604	0.0915608365944941\\
2.95018558390515	0.0935942471615495\\
2.99099747582647	0.0956768821084787\\
3.03180936774779	0.0977687930910769\\
3.07262125966911	0.0998696136237498\\
3.11343315159043	0.101978983846102\\
3.15504317204847	0.104138041583016\\
3.1966531925065	0.106305252125391\\
3.23826321296453	0.108480255181693\\
3.27987323342256	0.110662697308567\\
3.32230262192904	0.112895416221997\\
3.36473201043552	0.115135149439742\\
3.407161398942	0.11738154403339\\
3.44959078744848	0.11963425411398\\
3.49286538309579	0.121937992993384\\
3.5361399787431	0.12424759692045\\
3.57941457439041	0.126562721347867\\
3.62268917003772	0.128883028929777\\
3.66684010640848	0.131255323472471\\
3.71099104277924	0.13363232656288\\
3.75514197915	0.136013702903956\\
3.79929291552076	0.138399124535815\\
3.84435753680729	0.140837750250596\\
3.88942215809383	0.143279923199408\\
3.93448677938037	0.14572531805906\\
3.9795514006669	0.148173616956409\\
4.02557447530949	0.150676661964007\\
4.07159754995208	0.153182089311474\\
4.11762062459467	0.155689584231177\\
4.16364369923726	0.158198839499607\\
4.21041243310513	0.160750244033202\\
4.257181166973	0.163302850299863\\
4.30394990084087	0.165856359599116\\
4.35071863470874	0.168410480686565\\
4.39747966483664	0.170964508933651\\
4.44424069496454	0.173518588713861\\
4.49100172509244	0.176072451050674\\
4.53776275522034	0.178625833849906\\
4.58452386492069	0.1811784861861\\
4.63128497462103	0.183730155290045\\
4.67804608432138	0.186280599500411\\
4.72480719402173	0.188829583496746\\
4.77156830290013	0.191376878201007\\
4.81832941177852	0.193922260934001\\
4.86509052065691	0.196465515200881\\
4.91185162953531	0.199006430334966\\
4.95861273842228	0.20154480144757\\
5.00537384730925	0.204080429449661\\
5.05213495619622	0.206613120977212\\
5.09889606508318	0.209142688011602\\
5.14565717397096	0.211668947832168\\
5.19241828285873	0.214191723039485\\
5.2391793917465	0.216710841483733\\
5.28594050063428	0.219226135908105\\
5.33270160952073	0.221737443904372\\
5.37946271840718	0.22424460793642\\
5.42622382729363	0.226747475272647\\
5.47298493618008	0.229245897652261\\
5.51974604506734	0.231739731244041\\
5.56650715395461	0.234228836669087\\
5.61326826284187	0.236713078937951\\
5.66002937172913	0.239192327139379\\
5.70679048061676	0.241666454402033\\
5.7535515895044	0.244135337916623\\
5.80031269839203	0.246598858877288\\
5.84707380727967	0.2490569021924\\
5.89383491616663	0.251509356449103\\
5.9405960250536	0.253956113934734\\
5.98735713394057	0.256397070582246\\
6.03411824282754	0.258832125702362\\
6.08087935171415	0.26126118195087\\
6.12764046060076	0.26368414534899\\
6.17440156948738	0.266100925232901\\
6.22116267837399	0.268511434006271\\
6.26792378726144	0.270915587110211\\
6.3146848961489	0.273313303042369\\
6.36144600503635	0.27570450331052\\
6.4082071139238	0.278089112204415\\
6.45496822281113	0.280467056768132\\
6.50172933169847	0.282838266818194\\
6.5484904405858	0.28520267490068\\
6.59525154947314	0.287560216081442\\
6.64201265836067	0.289910827920759\\
6.68877376724821	0.292254450490265\\
6.73553487613575	0.294591026333581\\
6.78229598502329	0.296920500273716\\
6.82916482319815	0.299248161300213\\
6.87603366137301	0.301568583501107\\
6.92290249954786	0.30388171813635\\
6.96977133772272	0.306187518543546\\
7.01788255937152	0.308546765119608\\
7.06599378102033	0.310898190513424\\
7.11410500266913	0.313241751008748\\
7.16221622431793	0.315577404966159\\
7.21159336850751	0.317966252781014\\
7.26097051269708	0.320346690288347\\
7.31034765688666	0.322718679220293\\
7.35972480107623	0.325082183374316\\
7.41043619055615	0.327500685346304\\
7.46114758003607	0.329910167031756\\
7.511858969516	0.332310595936263\\
7.56257035899592	0.334701941615889\\
7.61468820401402	0.337150115628942\\
7.66680604903213	0.339588636985078\\
7.71892389405023	0.342017479325357\\
7.77104173906833	0.344436618322487\\
7.8246442387161	0.346914524713853\\
7.87824673836387	0.34938212140824\\
7.93184923801164	0.351839388582208\\
7.98545173765941	0.354286308420786\\
8.04062368192306	0.356794049506083\\
8.09579562618671	0.359290796645629\\
8.15096757045037	0.361776536995946\\
8.20613951471402	0.364251259693994\\
8.26297303935407	0.366788982370608\\
8.31980656399412	0.369314996784353\\
8.37664008863416	0.371829297567993\\
8.43347361327421	0.374331881301259\\
8.49206906178453	0.376899779732885\\
8.55066451029485	0.379455222290161\\
8.60925995880517	0.381998211638974\\
8.66785540731549	0.38452875235256\\
8.7283223218246	0.387127070875886\\
8.7887892363337	0.389712148971524\\
8.84925615084281	0.392283997966425\\
8.90972306535191	0.394842631048265\\
8.97139693657501	0.397438737357407\\
9.03307080779811	0.400021128449867\\
9.09474467902121	0.402589824217639\\
9.15641855024431	0.405144846273218\\
9.21815611731334	0.407688835563614\\
9.27989368438236	0.410219171338555\\
9.34163125145139	0.412735880383582\\
9.40336881852041	0.415238990951748\\
9.46537482585172	0.417739327898527\\
9.52738083318303	0.420226009386414\\
9.58938684051434	0.422699068444672\\
9.65139284784564	0.425158539363317\\
9.71505322212012	0.427669531328312\\
9.77871359639459	0.430166277681433\\
9.84237397066907	0.432648819617027\\
9.90603434494354	0.435117199500271\\
9.97140380506578	0.437637156041148\\
10.036773265188	0.440142274188647\\
10.1021427253103	0.44263260360485\\
10.1675121854325	0.445108195024715\\
10.2347010682591	0.447637384841392\\
10.3018899510858	0.450151116965707\\
10.3690788339124	0.452649449971815\\
10.436267716739	0.455132443404201\\
10.505392885882	0.457671047817852\\
10.574518055025	0.460193546769705\\
10.643643224168	0.462700008260062\\
10.712768393311	0.465190501151067\\
10.7839560546811	0.467738686234212\\
10.8551437160513	0.470270086602271\\
10.9263313774214	0.472784780277815\\
10.9975190387916	0.475282846029494\\
11.07090572606	0.477840756152863\\
11.1442924133284	0.480381167599043\\
11.2176791005968	0.48290416910124\\
11.2910657878652	0.485409850014103\\
11.3667995476522	0.48797760323015\\
11.4425333074392	0.490527105601518\\
11.5182670672261	0.493058457364837\\
11.5940008270131	0.49557175924284\\
11.6722425439244	0.498149441298596\\
11.7504842608358	0.500708078343385\\
11.8287259777471	0.503247783028841\\
11.9069676946584	0.505768668344469\\
11.9878925800102	0.508356325256126\\
12.068817465362	0.510924096874326\\
12.1497423507139	0.513472109312959\\
12.2306672360657	0.516000488860061\\
12.3144665036616	0.518598117817193\\
12.3982657712575	0.521174970613848\\
12.4820650388535	0.523731188028914\\
12.5658643064494	0.526266910833119\\
12.652747062149	0.528874449163313\\
12.7396298178485	0.531460265036428\\
12.826512573548	0.534024515277856\\
12.9133953292476	0.536567356500225\\
13.0035907048582	0.53918466838184\\
13.0937860804688	0.541779250835095\\
13.1839814560794	0.544351278319474\\
13.27417683169	0.5469009248503\\
13.3679363981715	0.549527785418344\\
13.461695964653	0.552130843247967\\
13.5554555311345	0.554710292258436\\
13.649215097616	0.557266325661241\\
13.7468155979555	0.559902402114\\
13.8444160982949	0.562513530084157\\
13.9420165986344	0.56509992505212\\
14.0396170989738	0.567661801488081\\
14.1413635477079	0.570306630775698\\
14.2431099964419	0.572925286873452\\
14.344856445176	0.575518009243907\\
14.44660289391	0.578085035990088\\
14.5528320987134	0.580737998660584\\
14.6590613035167	0.583363477520253\\
14.7652905083201	0.585961738804615\\
14.8715197131234	0.588533046984209\\
14.9753724014636	0.591021027591076\\
15.0792250898038	0.593483742495115\\
15.183077778144	0.595921433928101\\
15.2869304664842	0.598334342415244\\
15.3888306269013	0.60067802822107\\
15.4907307873185	0.602998308143793\\
15.5926309477356	0.60529540461947\\
15.6945311081528	0.60756953844696\\
15.7977993001492	0.609851000461161\\
15.9010674921457	0.612109330612602\\
16.0043356841422	0.614344753582601\\
16.1076038761386	0.616557492291002\\
16.2124869885142	0.618781842472543\\
16.3173701008898	0.620983251337538\\
16.4222532132654	0.623161946867777\\
16.527136325641	0.625318155154254\\
16.6339006004935	0.627490172544983\\
16.7406648753461	0.629639355595072\\
16.8474291501986	0.631765936766749\\
16.9541934250511	0.633870146492704\\
17.0630876639797	0.635993527004103\\
17.1719819029082	0.638094112326983\\
17.2808761418368	0.64017214046524\\
17.3897703807653	0.642227847243408\\
17.5010359523856	0.644305507370594\\
17.6123015240059	0.646360354228135\\
17.7235670956262	0.648392632366397\\
17.8348326672466	0.650402583993589\\
17.9487100003084	0.652436847197895\\
18.0625873333702	0.654448229328107\\
18.1764646664321	0.656436982467898\\
18.2903419994939	0.65840335618093\\
18.4070757021241	0.660396082249791\\
18.5238094047543	0.662365815094926\\
18.6405431073846	0.664312815331401\\
18.7572768100148	0.666237340858982\\
18.8771200075905	0.668190018080615\\
18.9969632051661	0.670119549432732\\
19.1168064027418	0.672026205095132\\
19.2366496003175	0.673910252316878\\
19.3598678839536	0.675824064742377\\
19.4830861675897	0.677714540866698\\
19.6063044512258	0.679581961524387\\
19.7295227348619	0.681426604380537\\
19.8563979903555	0.683302481840519\\
19.9832732458491	0.685154796612329\\
20.1101485013428	0.686983841352526\\
20.2370237568364	0.688789905282536\\
20.3678580796439	0.690628561118707\\
20.4986924024514	0.692443393049371\\
20.6295267252589	0.694234706818634\\
20.7603610480664	0.696002804438511\\
20.8954809190363	0.697804764249895\\
21.0306007900061	0.699582604644865\\
21.1657206609759	0.701336645841445\\
21.3008405319458	0.703067203992206\\
21.4406014192669	0.704832827640256\\
21.5803623065881	0.706574002093691\\
21.7201231939092	0.708291063576807\\
21.8598840812304	0.709984343872547\\
22.0046756408933	0.711713842298391\\
22.1494672005562	0.713418527048898\\
22.2942587602192	0.715098752061764\\
22.4390503198821	0.716754866407506\\
22.5893023876591	0.71844831434732\\
22.7395544554361	0.720116548558967\\
22.8898065232131	0.721759942609106\\
23.0400585909901	0.723378864712539\\
23.1962481938828	0.725036210072681\\
23.3524377967755	0.726667904785453\\
23.5086273996682	0.728274344210475\\
23.6648170025609	0.729855917801056\\
23.8274766689118	0.731476988221344\\
23.9901363352628	0.733071932484584\\
24.1527960016138	0.73464117020395\\
24.3154556679648	0.736185114448631\\
24.4851833872328	0.737769621408852\\
24.6549111065008	0.739327485939472\\
24.8246388257688	0.740859154722984\\
24.9943665450368	0.742365067160366\\
25.1718378270833	0.743912607849621\\
25.3493091091298	0.745432946421922\\
25.5267803911763	0.746926559872055\\
25.7042516732229	0.748393917055091\\
25.8902342929274	0.74990397411266\\
26.0762169126319	0.751386222775682\\
26.2621995323364	0.752841174110806\\
26.4481821520409	0.754269330039854\\
26.6435543528394	0.755741269178288\\
26.838926553638	0.757184743298092\\
27.0342987544366	0.758600301925907\\
27.2296709552352	0.759988484258124\\
27.4354441469161	0.761421549686678\\
27.641217338597	0.762825438717475\\
27.8469905302779	0.764200744495429\\
28.0527637219589	0.765548048426526\\
28.2701110770545	0.76694135589859\\
28.4874584321501	0.768304715691183\\
28.7048057872458	0.769638770676728\\
28.9221531423414	0.770944150300975\\
29.1524460226538	0.772296677505827\\
29.3827389029661	0.773618419857654\\
29.6130317832784	0.774910077249855\\
29.8433246635908	0.776172334108963\\
30.088179573096	0.777482908009136\\
30.3330344826013	0.778761787041158\\
30.5778893921066	0.780009736901381\\
30.8227443016119	0.781227505328916\\
31.0840841898221	0.782494785901395\\
31.3454240780323	0.783729380705931\\
31.6067639662425	0.784932131913799\\
31.8681038544528	0.786103860666532\\
32.1482392887021	0.787326320329794\\
32.4283747229513	0.788515013163157\\
32.7085101572006	0.789670870911165\\
32.9886455914499	0.790794800453168\\
33.2903839143714	0.79197069824196\\
33.5921222372929	0.793111647329799\\
33.8938605602144	0.794218685285321\\
34.195598883136	0.795292819961694\\
34.5223935450538	0.796420169231591\\
34.8491882069717	0.797511274979754\\
35.1759828688896	0.79856730104174\\
35.5027775308075	0.799589375292426\\
35.8301107995782	0.8005801939617\\
36.1574440683489	0.801539106416941\\
36.4847773371196	0.802467138897808\\
36.8121106058903	0.803365285904469\\
37.1394383112246	0.80423449606305\\
37.4667660165588	0.805075716228198\\
37.7940937218931	0.80588984669426\\
38.1214214272273	0.806677759908807\\
38.448749190017	0.807440300954275\\
38.7760769528068	0.808178286763566\\
39.1034047155965	0.808892507171449\\
39.4307324783862	0.809583727580908\\
39.7580602405942	0.810252689269056\\
40.0853880028023	0.810900109055796\\
40.4127157650103	0.811526679861197\\
40.7400435272184	0.812133073170434\\
41.0673712894278	0.812719939303227\\
41.3946990516372	0.81328790711956\\
41.7220268138466	0.81383758451214\\
42.049354576056	0.814369560568464\\
42.3766823382676	0.81488440580722\\
42.7040101004792	0.815382671919846\\
43.0313378626908	0.815864892202717\\
43.3586656249024	0.816331583454307\\
43.6859933871135	0.816783246182589\\
44.0133211493247	0.817220364378165\\
44.3406489115359	0.817643405893491\\
44.667976673747	0.818052824107482\\
44.9953044359533	0.818449058107455\\
45.3226321981596	0.81883253249003\\
45.649959960366	0.819203657693848\\
45.9772877225723	0.819562831460055\\
46.3046154847852	0.81991043899193\\
46.6319432469981	0.820246852780128\\
46.9592710092111	0.820572432894641\\
47.286598771424	0.820887528266114\\
47.6139265336322	0.821192476825904\\
47.9412542958405	0.821487605352784\\
48.2685820580487	0.821773229728857\\
48.595909820257	0.822049656063921\\
48.926850153611	0.822320084790061\\
49.257790486965	0.822581710572255\\
49.5887308203189	0.82283481964117\\
49.9196711536729	0.823079689282176\\
50.2593448649822	0.823322734053657\\
50.5990185762914	0.823557662030895\\
50.9386922876007	0.823784743965369\\
51.2783659989099	0.824004241936537\\
51.6270508037435	0.824221940452841\\
51.9757356085771	0.824432179128066\\
52.3244204134106	0.824635213268962\\
52.6731052182442	0.824831289802518\\
53.0312891287569	0.825025714182917\\
53.3894730392697	0.825213297997256\\
53.7476569497824	0.825394281604305\\
54.1058408602952	0.825568897273581\\
54.4740437546715	0.825741995530775\\
54.8422466490478	0.825908836329192\\
55.2104495434241	0.82606964555778\\
55.5786524378004	0.826224641305713\\
55.95743714694	0.826378246604005\\
56.3362218560795	0.826526142543107\\
56.7150065652191	0.826668541018658\\
57.0937912743587	0.826805646414938\\
57.4837682235858	0.82694148075399\\
57.8737451728129	0.827072119984782\\
58.26372212204	0.827197762484276\\
58.653699071267	0.827318599405397\\
58.9902743034503	0.827419167287425\\
59.3268495356335	0.827516406694234\\
59.6634247678168	0.827610427693196\\
60	0.827701336834441\\
};
\addlegendentry{Rimossi}

\end{axis}
\end{tikzpicture}%}
}
\quad
\subfloat[][Nodo 4]
{\resizebox{0.45\textwidth}{!}
{% This file was created by matlab2tikz.
%
%The latest updates can be retrieved from
%  http://www.mathworks.com/matlabcentral/fileexchange/22022-matlab2tikz-matlab2tikz
%where you can also make suggestions and rate matlab2tikz.
%
\definecolor{mycolor1}{rgb}{0.00000,0.44700,0.74100}%
\definecolor{mycolor2}{rgb}{0.85000,0.32500,0.09800}%
\definecolor{mycolor3}{rgb}{0.92900,0.69400,0.12500}%
%
\begin{tikzpicture}

\begin{axis}[%
width=6.028in,
height=4.754in,
at={(1.011in,0.642in)},
scale only axis,
xmin=0,
xmax=60,
xlabel style={font=\color{white!15!black}},
xlabel={T},
ymin=0,
ymax=1,
axis background/.style={fill=white},
legend style={legend cell align=left, align=left, draw=white!15!black}
]
\addplot [color=mycolor1, line width=2.0pt]
  table[row sep=crcr]{%
0	0\\
0.00199053585276749	0\\
0.00398107170553497	0\\
0.00597160755830246	0\\
0.00796214341106994	0\\
0.0123567669950473	0\\
0.0167513905790247	0\\
0.0211460141630021	0\\
0.0255406377469794	0\\
0.0299388563931551	0\\
0.0343370750393307	0\\
0.0387352936855064	0\\
0.043133512331682	0\\
0.0475471725191068	0\\
0.0519608327065316	0\\
0.0563744928939564	0\\
0.0607881530813811	0\\
0.0652173199794049	0\\
0.0696464868774286	0\\
0.0740756537754524	0\\
0.0785048206734761	0\\
0.0829496187548012	0\\
0.0873944168361263	0\\
0.0918392149174515	0\\
0.0962840129987766	0\\
0.100744568101794	0\\
0.105205123204811	0\\
0.109665678307829	0\\
0.114126233410846	0\\
0.118602673075784	0\\
0.123079112740721	0\\
0.127555552405659	0\\
0.132031992070596	0\\
0.136524445545197	0\\
0.141016899019798	0\\
0.1455093524944	0\\
0.150001805969001	0\\
0.154510404254387	0\\
0.159019002539772	0\\
0.163527600825158	0\\
0.168036199110544	0\\
0.172561074936287	0\\
0.17708595076203	0\\
0.181610826587773	0\\
0.186135702413515	0\\
0.190676990405524	0\\
0.195218278397533	0\\
0.199759566389542	0\\
0.204300854381551	0\\
0.208858690959197	0\\
0.213416527536843	0\\
0.217974364114488	0\\
0.222532200692134	0\\
0.227106724202768	0\\
0.231681247713401	0\\
0.236255771224035	0\\
0.240830294734669	0\\
0.245421645441331	0\\
0.250012996147992	0\\
0.254604346854654	0\\
0.259195697561315	0\\
0.263804017673384	0\\
0.268412337785452	0\\
0.273020657897521	0\\
0.27762897800959	0\\
0.282254411746462	0\\
0.286879845483334	0\\
0.291505279220206	0\\
0.296130712957079	0\\
0.300773406613671	0\\
0.305416100270264	0\\
0.310058793926856	0\\
0.314701487583449	0\\
0.319361589521735	0\\
0.324021691460022	0\\
0.328681793398309	0\\
0.333341895336596	0\\
0.338019556071109	0\\
0.342697216805623	0\\
0.347374877540136	0\\
0.35205253827465	0\\
0.356747910447269	0\\
0.361443282619888	0\\
0.366138654792507	0\\
0.370834026965126	0\\
0.375547265516765	0\\
0.380260504068404	0\\
0.384973742620042	0\\
0.389686981171681	0\\
0.394418243246227	0\\
0.399149505320772	0\\
0.403880767395317	0\\
0.408612029469863	0\\
0.41336147459102	0\\
0.418110919712177	0\\
0.422860364833334	0\\
0.427609809954491	0\\
0.432377599961091	0\\
0.437145389967691	0\\
0.44191317997429	0\\
0.44668096998089	0\\
0.451467269141767	0\\
0.456253568302644	0\\
0.461039867463521	0\\
0.465826166624398	0\\
0.47063114173014	0\\
0.475436116835881	0\\
0.480241091941623	0\\
0.485046067047365	0\\
0.489869887370685	0\\
0.494693707694006	0\\
0.499517528017326	0\\
0.504341348340647	0\\
0.509184185682064	0\\
0.51402702302348	0\\
0.518869860364897	0\\
0.523712697706314	0\\
0.528574726619363	0\\
0.533436755532412	0\\
0.53829878444546	0\\
0.543160813358509	0\\
0.548042211039569	0\\
0.55292360872063	0\\
0.55780500640169	0\\
0.56268640408275	0\\
0.56758735048985	0\\
0.572488296896949	0\\
0.577389243304049	0\\
0.582290189711149	0\\
0.587210867601438	0\\
0.592131545491728	0\\
0.597052223382018	0\\
0.601972901272307	0\\
0.606913496388328	0\\
0.611854091504349	0\\
0.61679468662037	0\\
0.621735281736391	0\\
0.626695982628195	0\\
0.631656683519999	0\\
0.636617384411802	0\\
0.641578085303606	0\\
0.646559083672327	0\\
0.651540082041047	0\\
0.656521080409768	0\\
0.661502078778488	0\\
0.666503569383853	0\\
0.671505059989219	0\\
0.676506550594584	0\\
0.68150804119995	0\\
0.686530221876942	0\\
0.691552402553934	0\\
0.696574583230927	0\\
0.701596763907919	0\\
0.706639835866396	0\\
0.711682907824873	0\\
0.71672597978335	0\\
0.721769051741827	0\\
0.72683321952555	0\\
0.731897387309272	0\\
0.736961555092995	0\\
0.742025722876718	0\\
0.747111194287107	0\\
0.752196665697496	0\\
0.757282137107885	0\\
0.762367608518274	0\\
0.767474594900923	0\\
0.772581581283573	0\\
0.777688567666222	0\\
0.782795554048871	0\\
0.78792427039055	0\\
0.793052986732229	0\\
0.798181703073909	0\\
0.803310419415588	0\\
0.808461084325784	0\\
0.81361174923598	0\\
0.818762414146177	0\\
0.823913079056373	0\\
0.82908591482452	0\\
0.834258750592667	0\\
0.839431586360813	0\\
0.84460442212896	0\\
0.849799655005291	0\\
0.854994887881621	0\\
0.860190120757951	0\\
0.865385353634282	0\\
0.870603213699137	0\\
0.875821073763992	0\\
0.881038933828848	0\\
0.886256793893703	0\\
0.891497515445601	0\\
0.896738236997499	0\\
0.901978958549397	0\\
0.907219680101295	0\\
0.912483501408722	0\\
0.917747322716149	0\\
0.923011144023576	0\\
0.928274965331004	0\\
0.933562128954513	0\\
0.938849292578023	0\\
0.944136456201532	0\\
0.949423619825042	0\\
0.954734372782572	0\\
0.960045125740102	0\\
0.965355878697632	0\\
0.970666631655162	0\\
0.976001225418392	0\\
0.981335819181621	0\\
0.98667041294485	0\\
0.99200500670808	0\\
0.997363697280812	0\\
1.00272238785355	0\\
1.00808107842628	0\\
1.01343976899901	0\\
1.0188228172289	0\\
1.02420586545878	0\\
1.02958891368867	0\\
1.03497196191855	0\\
1.04037963341625	0\\
1.04578730491395	0\\
1.05119497641164	0\\
1.05660264790934	0\\
1.06203521338424	0\\
1.06746777885913	0\\
1.07290034433402	0\\
1.07833290980891	0\\
1.08379064506731	0\\
1.08924838032571	0\\
1.09470611558411	0\\
1.10016385084252	0\\
1.1056470369148	0\\
1.11113022298707	0\\
1.11661340905935	0\\
1.12209659513163	0\\
1.12760551858231	0\\
1.13311444203298	0\\
1.13862336548365	0\\
1.14413228893432	0\\
1.14966724184116	0\\
1.15520219474799	0\\
1.16073714765482	0\\
1.16627210056166	0\\
1.17183338076584	0\\
1.17739466097002	0\\
1.18295594117419	0\\
1.18851722137837	0\\
1.19410513262804	0\\
1.19969304387772	0\\
1.20528095512739	0\\
1.21086886637706	0\\
1.21648371859449	0\\
1.22209857081192	0\\
1.22771342302936	0\\
1.23332827524679	0\\
1.23897038457538	0\\
1.24461249390397	0\\
1.25025460323256	0\\
1.25589671256115	0\\
1.26156640151281	0\\
1.26723609046446	0\\
1.27290577941612	0\\
1.27857546836778	0\\
1.28427306633634	0\\
1.28997066430491	0\\
1.29566826227347	0\\
1.30136586024204	0\\
1.30709170327692	0\\
1.31281754631181	0\\
1.31854338934669	0\\
1.32426923238158	0\\
1.33002366371043	0\\
1.33577809503928	0\\
1.34153252636813	0\\
1.34728695769699	0\\
1.35307032782915	0\\
1.35885369796131	0\\
1.36463706809348	0\\
1.37042043822564	0\\
1.37622881553645	0\\
1.38203719284726	0\\
1.38784557015807	0\\
1.39365394746888	0\\
1.39948194569202	0\\
1.40530994391516	0\\
1.41113794213829	0\\
1.41696594036143	0\\
1.4228137248248	0\\
1.42866150928816	0\\
1.43450929375153	0\\
1.44035707821489	0\\
1.44622478964653	0\\
1.45209250107816	0\\
1.45796021250979	0\\
1.46382792394143	0\\
1.46971570462452	0\\
1.4756034853076	0\\
1.48149126599069	0\\
1.48737904667378	0\\
1.49328704060906	0\\
1.49919503454433	0\\
1.50510302847961	0\\
1.51101102241489	0\\
1.51693937480906	0\\
1.52286772720322	0\\
1.52879607959739	0\\
1.53472443199155	0\\
1.540673290528	0\\
1.54662214906446	0\\
1.55257100760091	0\\
1.55851986613736	0\\
1.56448937947106	0\\
1.57045889280476	0\\
1.57642840613847	0\\
1.58239791947216	0\\
1.58838823791446	0\\
1.59437855635675	0\\
1.60036887479904	0\\
1.60635919324133	0\\
1.61237046959671	0\\
1.61838174595208	0\\
1.62439302230746	0\\
1.63040429866284	0\\
1.63643668672632	0\\
1.6424690747898	0\\
1.64850146285328	0\\
1.65453385091676	0\\
1.66058750665404	0\\
1.66664116239132	0\\
1.67269481812859	0\\
1.67874847386587	0\\
1.68482355500326	0\\
1.69089863614064	0\\
1.69697371727803	0\\
1.70304879841541	0\\
1.70914546464501	0\\
1.71524213087462	0\\
1.72133879710422	0\\
1.72743546333383	0\\
1.73355387572031	0\\
1.73967228810678	0\\
1.74579070049326	0\\
1.75190911287974	0\\
1.75804943463327	0\\
1.76418975638681	0\\
1.77033007814035	0\\
1.77647039989388	0\\
1.78263279664118	0\\
1.78879519338848	0\\
1.79495759013578	0\\
1.80111998688309	0\\
1.80730462566597	0\\
1.81348926444885	0\\
1.81967390323173	0\\
1.82585854201462	0\\
1.83206559217232	0\\
1.83827264233002	0\\
1.84447969248772	0\\
1.85068674264543	0\\
1.85691637550678	0\\
1.86314600836813	0\\
1.86937564122948	0\\
1.87560527409083	0\\
1.88185766319929	0\\
1.88811005230775	0\\
1.89436244141622	0\\
1.90061483052468	0\\
1.90689015093446	0\\
1.91316547134424	0\\
1.91944079175402	0\\
1.92571611216381	0\\
1.93201454150218	0\\
1.93831297084055	0\\
1.94461140017892	0\\
1.9509098295173	0\\
1.95723154808653	0\\
1.96355326665576	0\\
1.96987498522499	0\\
1.97619670379422	0\\
1.98254189319299	0\\
1.98888708259176	0\\
1.99523227199053	0\\
2.0015774613893	0\\
2.00794630624256	0\\
2.01431515109582	0\\
2.02068399594908	0\\
2.02705284080234	0\\
2.03344552746868	0\\
2.03983821413503	0\\
2.04623090080137	0\\
2.05262358746771	0\\
2.05904030495676	0\\
2.06545702244581	0\\
2.07187373993486	0\\
2.07829045742391	0\\
2.084731396657	0\\
2.0911723358901	0\\
2.0976132751232	0\\
2.1040542143563	0\\
2.11051956957009	0\\
2.11698492478388	0\\
2.12345027999767	0\\
2.12991563521146	0\\
2.13640560238783	0\\
2.1428955695642	0\\
2.14938553674057	0\\
2.15587550391694	0\\
2.16239028199053	0\\
2.16890506006412	0\\
2.17541983813771	0\\
2.1819346162113	0\\
2.18847440630977	0\\
2.19501419640825	0\\
2.20155398650673	0\\
2.20809377660521	0\\
2.21465878275268	0\\
2.22122378890015	0\\
2.22778879504762	0\\
2.23435380119508	0\\
2.24094422983074	0\\
2.24753465846639	0\\
2.25412508710205	0\\
2.2607155157377	0\\
2.2673315759421	0\\
2.27394763614649	0\\
2.28056369635089	0\\
2.28717975655528	0\\
2.29382166035897	0\\
2.30046356416266	0\\
2.30710546796636	0\\
2.31374737177005	0\\
2.32041533418655	0\\
2.32708329660306	0\\
2.33375125901957	0\\
2.34041922143607	0\\
2.34711345989709	0\\
2.3538076983581	0\\
2.36050193681912	0\\
2.36719617528013	0\\
2.37391691043736	0\\
2.3806376455946	0\\
2.38735838075183	0\\
2.39407911590907	0\\
2.40082657131283	0\\
2.40757402671659	0\\
2.41432148212035	0\\
2.42106893752412	0\\
2.42784333963791	0\\
2.4346177417517	0\\
2.4413921438655	0\\
2.44816654597929	0\\
2.4549681243632	0\\
2.4617697027471	0\\
2.46857128113101	0\\
2.47537285951492	0\\
2.48220184698832	0\\
2.48903083446173	0\\
2.49585982193513	0\\
2.50268880940853	0\\
2.50954544177924	0\\
2.51640207414995	0\\
2.52325870652066	0\\
2.53011533889137	0\\
2.53699985562036	0\\
2.54388437234935	0\\
2.55076888907833	0\\
2.55765340580732	0\\
2.5645660494942	0\\
2.57147869318108	0\\
2.57839133686796	0\\
2.58530398055484	0\\
2.59224499685821	0\\
2.59918601316159	0\\
2.60612702946496	0\\
2.61306804576833	0\\
2.62003768421203	0\\
2.62700732265572	0\\
2.63397696109942	0\\
2.64094659954311	0\\
2.64794511298781	0\\
2.65494362643251	0\\
2.6619421398772	0\\
2.6689406533219	0\\
2.67596829852112	0\\
2.68299594372033	0\\
2.69002358891955	0\\
2.69705123411877	0\\
2.70410827073433	0\\
2.71116530734989	0\\
2.71822234396545	0\\
2.72527938058101	0\\
2.73236607296869	0\\
2.73945276535637	0\\
2.74653945774405	0\\
2.75362615013173	0\\
2.76074276611797	0\\
2.76785938210422	0\\
2.77497599809046	0\\
2.78209261407671	0\\
2.78923942490227	0\\
2.79638623572783	0\\
2.80353304655339	0\\
2.81067985737896	0\\
2.81785713902212	0\\
2.82503442066529	0\\
2.83221170230846	0\\
2.83938898395163	0\\
2.84659701614953	0\\
2.85380504834743	0\\
2.86101308054533	0\\
2.86822111274323	0\\
2.87546017926332	0\\
2.88269924578341	0\\
2.88993831230351	0\\
2.8971773788236	0\\
2.90444776780364	0\\
2.91171815678368	0\\
2.91898854576373	0\\
2.92625893474377	0\\
2.93356093845992	0\\
2.94086294217606	0\\
2.94816494589221	0\\
2.95546694960836	0\\
2.96280086487659	0\\
2.97013478014483	0\\
2.97746869541306	0\\
2.9848026106813	0\\
2.99216873909201	0\\
2.99953486750271	0\\
3.00690099591342	0\\
3.01426712432413	0\\
3.02166577186631	0\\
3.02906441940849	0\\
3.03646306695067	0\\
3.04386171449286	0\\
3.05129319192691	0\\
3.05872466936097	0\\
3.06615614679503	0\\
3.07358762422909	0\\
3.08105224696524	0\\
3.0885168697014	0\\
3.09598149243755	0\\
3.1034461151737	0\\
3.11094420406679	0\\
3.11844229295988	0\\
3.12594038185297	0\\
3.13343847074605	0\\
3.14097035080817	0\\
3.14850223087028	0\\
3.1560341109324	0\\
3.16356599099451	0\\
3.17113199342146	0\\
3.17869799584842	0\\
3.18626399827537	0\\
3.19383000070233	0\\
3.20143046110215	0\\
3.20903092150198	0\\
3.21663138190181	0\\
3.22423184230164	0\\
3.2318671021242	0\\
3.23950236194676	0\\
3.24713762176933	0\\
3.25477288159189	0\\
3.26244328804483	0\\
3.27011369449776	0\\
3.2777841009507	0\\
3.28545450740363	0\\
3.29316041266953	0\\
3.30086631793543	0\\
3.30857222320133	0\\
3.31627812846722	0\\
3.32401989110489	0\\
3.33176165374256	0\\
3.33950341638023	0\\
3.3472451790179	0\\
3.35502316313618	0\\
3.36280114725446	0\\
3.37057913137275	0\\
3.37835711549103	0\\
3.38617169154667	0\\
3.3939862676023	0\\
3.40180084365794	0\\
3.40961541971357	0\\
3.41746696420648	0\\
3.42531850869938	0\\
3.43317005319229	0\\
3.4410215976852	0\\
3.44891049346576	0\\
3.45679938924632	0\\
3.46468828502688	0\\
3.47257718080744	0\\
3.48050381756295	0\\
3.48843045431847	0\\
3.49635709107398	0\\
3.50428372782949	0\\
3.51224850159594	0\\
3.52021327536239	0\\
3.52817804912884	0\\
3.53614282289529	0\\
3.54414613607264	0\\
3.55214944924998	0\\
3.56015276242733	0\\
3.56815607560468	0\\
3.57619833889781	0\\
3.58424060219095	0\\
3.59228286548408	0\\
3.60032512877721	0\\
3.60840675922679	0\\
3.61648838967638	0\\
3.62457002012596	0\\
3.63265165057554	0\\
3.64077307248204	0\\
3.64889449438855	0\\
3.65701591629505	0\\
3.66513733820155	0\\
3.67329898353354	0\\
3.68146062886554	0\\
3.68962227419754	0\\
3.69778391952953	0\\
3.70598622860803	0\\
3.71418853768653	0\\
3.72239084676502	0\\
3.73059315584352	0\\
3.73883657626523	0\\
3.74707999668693	0\\
3.75532341710863	0\\
3.76356683753034	0\\
3.77185182516863	0\\
3.78013681280692	0\\
3.78842180044522	0\\
3.79670678808351	0\\
3.80503380745825	0\\
3.81336082683299	0\\
3.82168784620774	0\\
3.83001486558248	0\\
3.83838438958511	0\\
3.84675391358775	0\\
3.85512343759039	0\\
3.86349296159302	0\\
3.8719054722877	0\\
3.88031798298237	0\\
3.88873049367705	0\\
3.89714300437172	0\\
3.90559899284027	0\\
3.91405498130882	0\\
3.92251096977737	0\\
3.93096695824593	0\\
3.93946692479213	0\\
3.94796689133833	0\\
3.95646685788454	0\\
3.96496682443074	0\\
3.97351127881372	0\\
3.98205573319669	0\\
3.99060018757967	0\\
3.99914464196265	0\\
4.00773410425464	0\\
4.01632356654662	0\\
4.02491302883861	0\\
4.0335024911306	0\\
4.04213749081375	0\\
4.05077249049689	0\\
4.05940749018004	0\\
4.06804248986318	0\\
4.07672356764034	0\\
4.08540464541749	0\\
4.09408572319465	0\\
4.10276680097181	0\\
4.11149450824228	0\\
4.12022221551276	0\\
4.12894992278323	0\\
4.1376776300537	0\\
4.14645252883691	0\\
4.15522742762012	0\\
4.16400232640333	0\\
4.17277722518654	0\\
4.18159988926615	0\\
4.19042255334575	0\\
4.19924521742536	0\\
4.20806788150496	0\\
4.21693889629988	0\\
4.22580991109479	0\\
4.23468092588971	0\\
4.24355194068463	0\\
4.2524719033702	0\\
4.26139186605577	0\\
4.27031182874135	0\\
4.27923179142692	0\\
4.28820131218307	0\\
4.29717083293922	0\\
4.30614035369538	0\\
4.31510987445153	0\\
4.32412957614907	0\\
4.33314927784661	0\\
4.34216897954416	0\\
4.3511886812417	0\\
4.36025919971018	0\\
4.36932971817866	0\\
4.37840023664715	0\\
4.38747075511563	0\\
4.39659274002011	0\\
4.40571472492458	0\\
4.41483670982906	0\\
4.42395869473353	0\\
4.43313280988389	0\\
4.44230692503424	0\\
4.45148104018459	0\\
4.46065515533495	0\\
4.469882079136	0\\
4.47910900293706	0\\
4.48833592673811	0\\
4.49756285053916	0\\
4.50684327607396	0\\
4.51612370160877	0\\
4.52540412714357	0\\
4.53468455267837	0\\
4.54401918924822	0\\
4.55335382581807	0\\
4.56268846238792	0\\
4.57202309895777	0\\
4.58141267108196	0\\
4.59080224320616	0\\
4.60019181533036	0\\
4.60958138745456	0\\
4.61902663678989	0\\
4.62847188612523	0\\
4.63791713546056	0\\
4.6473623847959	0\\
4.65686406984621	0\\
4.66636575489653	0\\
4.67586743994685	0\\
4.68536912499717	0\\
4.69492802222378	0\\
4.7044869194504	0\\
4.71404581667701	0\\
4.72360471390363	0\\
4.73320764525141	0\\
4.74281057659919	0\\
4.75241350794698	0\\
4.76201643929476	0\\
4.77165794452804	0\\
4.78129944976131	0\\
4.79094095499459	0\\
4.80058246022786	0\\
4.81026314694842	0\\
4.81994383366897	0\\
4.82962452038952	0\\
4.83930520711008	0\\
4.84902565638322	0\\
4.85874610565637	0\\
4.86846655492951	0\\
4.87818700420266	0\\
4.88794780523678	0\\
4.89770860627089	0\\
4.90746940730501	0\\
4.91723020833913	0\\
4.92703195872998	0\\
4.93683370912084	0\\
4.9466354595117	0\\
4.95643720990256	0\\
4.96628051524823	0\\
4.97612382059389	0\\
4.98596712593956	0\\
4.99581043128522	0\\
5.00569590732134	0\\
5.01558138335745	0\\
5.02546685939356	0\\
5.03535233542967	0\\
5.04528060635073	0\\
5.05520887727179	0\\
5.06513714819285	0\\
5.07506541911391	0\\
5.08503711889176	0\\
5.0950088186696	0\\
5.10498051844744	0\\
5.11495221822528	0\\
5.12496799094167	0\\
5.13498376365806	0\\
5.14499953637444	0\\
5.15501530909083	0\\
5.16507580953502	0\\
5.17513630997921	0\\
5.18519681042339	0\\
5.19525731086758	0\\
5.2053632039849	0\\
5.21546909710223	0\\
5.22557499021955	0\\
5.23568088333688	0\\
5.24583284576096	0\\
5.25598480818505	0\\
5.26613677060914	0\\
5.27628873303322	0\\
5.28648745229033	0\\
5.29668617154743	0\\
5.30688489080453	0\\
5.31708361006163	0\\
5.32732978639303	0\\
5.33757596272443	0\\
5.34782213905582	0\\
5.35806831538722	0\\
5.36836266133408	0\\
5.37865700728095	0\\
5.38895135322782	0\\
5.39924569917468	0\\
5.40958894010065	0\\
5.41993218102661	0\\
5.43027542195257	0\\
5.44061866287853	0\\
5.45101153706255	0\\
5.46140441124658	0\\
5.4717972854306	0\\
5.48219015961463	0\\
5.49263341945276	0\\
5.50307667929089	0\\
5.51351993912902	0\\
5.52396319896715	0\\
5.53445761187213	0\\
5.54495202477712	0\\
5.5554464376821	0\\
5.56594085058709	0\\
5.57648719831941	0\\
5.58703354605173	0\\
5.59757989378405	0\\
5.60812624151637	0\\
5.61872532103302	0\\
5.62932440054967	0\\
5.63992348006633	0\\
5.65052255958298	0\\
5.66117518401188	0\\
5.67182780844078	0\\
5.68248043286967	0\\
5.69313305729857	0\\
5.70384005671797	0\\
5.71454705613736	0\\
5.72525405555675	0\\
5.73596105497615	0\\
5.74672327665992	0\\
5.75748549834369	0\\
5.76824772002746	0\\
5.77900994171123	0\\
5.78982825023888	0\\
5.80064655876654	0\\
5.81146486729419	0\\
5.82228317582184	0\\
5.83315845473553	0\\
5.84403373364921	0\\
5.8549090125629	0\\
5.86578429147658	0\\
5.87671744355088	0\\
5.88765059562518	0\\
5.89858374769948	0\\
5.90951689977378	0\\
5.92050884813521	0\\
5.93150079649664	0\\
5.94249274485807	0\\
5.9534846932195	0\\
5.96453638108293	0\\
5.97558806894636	0\\
5.98663975680979	0\\
5.99769144467322	0\\
6.00880383710542	0\\
6.01991622953761	0\\
6.0310286219698	0\\
6.042141014402	0\\
6.05331509935885	0\\
6.0644891843157	0\\
6.07566326927256	0\\
6.08683735422941	0\\
6.09807414230094	0\\
6.10931093037248	0\\
6.12054771844401	0\\
6.13178450651555	0\\
6.14308503319571	0\\
6.15438555987588	0\\
6.16568608655605	0\\
6.17698661323621	0\\
6.18835193888863	0\\
6.19971726454106	0\\
6.21108259019348	0\\
6.2224479158459	0\\
6.23387912710053	0\\
6.24531033835517	0\\
6.2567415496098	0\\
6.26817276086443	0\\
6.2796709718506	0\\
6.29116918283678	0\\
6.30266739382295	0\\
6.31416560480912	0\\
6.32573195844141	0\\
6.33729831207371	0\\
6.34886466570601	0\\
6.36043101933831	0\\
6.37206668825559	0\\
6.38370235717287	0\\
6.39533802609016	0\\
6.40697369500744	0\\
6.41867988207414	0\\
6.43038606914084	0\\
6.44209225620754	0\\
6.45379844327423	0\\
6.46557638449348	0\\
6.47735432571273	0\\
6.48913226693197	0\\
6.50091020815122	0\\
6.51276117329228	0\\
6.52461213843333	0\\
6.53646310357439	0\\
6.54831406871544	0\\
6.56023936237899	0\\
6.57216465604253	0\\
6.58408994970607	0\\
6.59601524336962	0\\
6.608016207688	0\\
6.62001717200638	0\\
6.63201813632476	0\\
6.64401910064313	0\\
6.65609711612489	0\\
6.66817513160664	0\\
6.68025314708839	0\\
6.69233116257014	0\\
6.70448764976455	0\\
6.71664413695897	0\\
6.72880062415338	0\\
6.7409571113478	0\\
6.75319353351478	0\\
6.76542995568177	0\\
6.77766637784875	0\\
6.78990280001574	0\\
6.80222066442117	0\\
6.8145385288266	0\\
6.82685639323203	0\\
6.83917425763746	0\\
6.85157511856051	0\\
6.86397597948357	0\\
6.87637684040662	0\\
6.88877770132968	0\\
6.90126316148383	0\\
6.91374862163799	0\\
6.92623408179214	0\\
6.93871954194629	0\\
6.95129125486549	0\\
6.96386296778468	0\\
6.97643468070388	0\\
6.98900639362307	0\\
7.00166606764739	0\\
7.0143257416717	0\\
7.02698541569601	0\\
7.03964508972033	0\\
7.05239448925109	0\\
7.06514388878186	0\\
7.07789328831262	0\\
7.09064268784339	0\\
7.10348363709426	0\\
7.11632458634514	0\\
7.12916553559601	0\\
7.14200648484689	0\\
7.15494087138423	0\\
7.16787525792157	0\\
7.18080964445891	0\\
7.19374403099625	0\\
7.20677380774895	0\\
7.21980358450165	0\\
7.23283336125435	0\\
7.24586313800706	0\\
7.25899032889946	0\\
7.27211751979186	0\\
7.28524471068426	0\\
7.29837190157667	0\\
7.31159860422458	0\\
7.32482530687249	0\\
7.3380520095204	0\\
7.35127871216831	0\\
7.36460710304072	0\\
7.37793549391314	0\\
7.39126388478556	0\\
7.40459227565797	0\\
7.41802461355885	0\\
7.43145695145972	0\\
7.4448892893606	0\\
7.45832162726148	0\\
7.47186025959003	0\\
7.48539889191859	0\\
7.49893752424715	0\\
7.51247615657571	0\\
7.52612352432791	0\\
7.53977089208011	0\\
7.55341825983232	0\\
7.56706562758452	0\\
7.58082427117841	0\\
7.5945829147723	0\\
7.60834155836619	0\\
7.62210020196009	0\\
7.63597276754062	0\\
7.64984533312116	0\\
7.66371789870169	0\\
7.67759046428223	0\\
7.69157971106386	0\\
7.70556895784549	0\\
7.71955820462712	0\\
7.73354745140875	0\\
7.7476562591015	0\\
7.76176506679424	0\\
7.77587387448699	0\\
7.78998268217974	0\\
7.80421406003167	0\\
7.81844543788359	0\\
7.83267681573552	0\\
7.84690819358745	0\\
7.86126528820976	0\\
7.87562238283208	0\\
7.8899794774544	0\\
7.90433657207672	0\\
7.91882267896367	0\\
7.93330878585062	0\\
7.94779489273757	0\\
7.96228099962452	0\\
7.97689957365268	0\\
7.99151814768085	0\\
8.00613672170902	0\\
8.02075529573718	0\\
8.03550996276884	0\\
8.0502646298005	0\\
8.06501929683216	0\\
8.07977396386382	0\\
8.09466853533245	0\\
8.10956310680107	0\\
8.12445767826969	0\\
8.13935224973832	0\\
8.15439073744049	0\\
8.16942922514267	0\\
8.18446771284484	0\\
8.19950620054702	0\\
8.2146928332504	0\\
8.22987946595378	0\\
8.24506609865717	0\\
8.26025273136055	0\\
8.27559197362809	0\\
8.29093121589562	0\\
8.30627045816316	0\\
8.32160970043069	0\\
8.33710627308224	0\\
8.35260284573378	0\\
8.36809941838533	0\\
8.38359599103687	0\\
8.39925489573158	0\\
8.41491380042629	0\\
8.430572705121	0\\
8.44623160981571	0\\
8.46205815432056	0\\
8.4778846988254	0\\
8.49371124333024	0\\
8.50953778783509	0\\
8.52553761739547	0\\
8.54153744695585	0\\
8.55753727651623	0\\
8.57353710607661	0\\
8.58971623619016	0\\
8.60589536630372	0\\
8.62207449641727	0\\
8.63825362653082	0\\
8.65461848330403	0\\
8.67098334007723	0\\
8.68734819685043	0\\
8.70371305362364	0\\
8.72018188980809	0\\
8.73665072599255	0\\
8.75311956217701	0\\
8.76958839836146	0\\
8.78614348346302	0\\
8.80269856856458	0\\
8.81925365366614	0\\
8.83580873876771	0\\
8.85245111606455	0\\
8.86909349336139	0\\
8.88573587065824	0\\
8.90237824795508	0\\
8.91910883699713	0\\
8.93583942603918	0\\
8.95257001508123	0\\
8.96930060412328	0\\
8.98612034063892	0\\
9.00294007715455	0\\
9.01975981367019	0\\
9.03657955018582	0\\
9.05348938408815	0\\
9.07039921799048	0\\
9.08730905189281	0\\
9.10421888579513	0\\
9.12121978296026	0\\
9.13822068012539	0\\
9.15522157729052	0\\
9.17222247445565	0\\
9.18931541587169	0\\
9.20640835728774	0\\
9.22350129870378	0\\
9.24059424011982	0\\
9.25778022278737	0\\
9.27496620545493	0\\
9.29215218812249	0\\
9.30933817079004	0\\
9.32661820779546	0\\
9.34389824480088	0\\
9.3611782818063	0\\
9.37845831881172	0\\
9.39583344002363	0\\
9.41320856123555	0\\
9.43058368244747	0\\
9.44795880365938	0\\
9.46543005570071	0\\
9.48290130774205	0\\
9.50037255978338	0\\
9.51784381182471	0\\
9.53541225878365	0\\
9.55298070574258	0\\
9.57054915270152	0\\
9.58811759966045	0\\
9.60578432346384	0\\
9.62345104726722	0\\
9.64111777107061	0\\
9.658784494874	0\\
9.67655059520445	0\\
9.6943166955349	0\\
9.71208279586536	0\\
9.72984889619581	0\\
9.74771549159443	0\\
9.76558208699305	0\\
9.78344868239167	0\\
9.80131527779029	0\\
9.81928350543996	0\\
9.83725173308963	0\\
9.8552199607393	0\\
9.87318818838897	0\\
9.89125920482513	0\\
9.90933022126129	0\\
9.92740123769745	0\\
9.94547225413361	0\\
9.96364723590061	0\\
9.9818222176676	0\\
9.99999719943459	0\\
10.0181721812016	0\\
10.0364523247657	0\\
10.0547324683298	0\\
10.0730126118939	0\\
10.0912927554579	0\\
10.1096792783833	0\\
10.1280658013086	0\\
10.146452324234	0\\
10.1648388471593	0\\
10.1833329877765	0\\
10.2018271283938	0\\
10.220321269011	0\\
10.2388154096282	0\\
10.257418428294	0\\
10.2760214469598	0\\
10.2946244656257	0\\
10.3132274842915	0\\
10.3319406634311	0\\
10.3506538425707	0\\
10.3693670217104	0\\
10.38808020085	0\\
10.4069048454226	0\\
10.4257294899953	0\\
10.4445541345679	0\\
10.4633787791406	0\\
10.4823162175897	0\\
10.5012536560388	0\\
10.5201910944879	0\\
10.539128532937	0\\
10.5581801172831	0\\
10.5772317016292	0\\
10.5962832859753	0\\
10.6153348703214	0\\
10.6345019773376	0\\
10.6536690843538	0\\
10.67283619137	0\\
10.6920032983861	0\\
10.7112873293394	0\\
10.7305713602926	0\\
10.7498553912458	0\\
10.769139422199	0\\
10.7885418041401	0\\
10.8079441860811	0\\
10.8273465680222	0\\
10.8467489499633	0\\
10.8662711360792	0\\
10.8857933221952	0\\
10.9053155083112	0\\
10.9248376944271	0\\
10.9444811647161	0\\
10.9641246350051	0\\
10.9837681052941	0\\
11.0034115755831	0\\
11.02317783773	0\\
11.0429440998769	0\\
11.0627103620238	0\\
11.0824766241707	0\\
11.102367213876	0\\
11.1222578035813	0\\
11.1421483932866	0\\
11.1620389829919	0\\
11.182055464688	0\\
11.202071946384	0\\
11.22208842808	0\\
11.2421049097761	0\\
11.2622488777205	0\\
11.282392845665	0\\
11.3025368136094	0\\
11.3226807815539	0\\
11.3429428732039	0\\
11.363204964854	0\\
11.3834670565041	0\\
11.4037291481542	0\\
11.4240073959591	0\\
11.444285643764	0\\
11.464563891569	0\\
11.4848421393739	0\\
11.5051410232308	0\\
11.5254399070878	0\\
11.5457387909448	0\\
11.5660376748017	0\\
11.5863613245615	0\\
11.6066849743212	0\\
11.627008624081	0\\
11.6473322738407	0\\
11.6676846007299	0\\
11.6880369276191	0\\
11.7083892545083	0\\
11.7287415813975	0\\
11.7491263024508	0\\
11.7695110235041	0\\
11.7898957445574	0\\
11.8102804656107	0\\
11.8307011258767	0\\
11.8511217861426	0\\
11.8715424464086	0\\
11.8919631066745	0\\
11.9124230970895	0\\
11.9328830875046	0\\
11.9533430779196	0\\
11.9738030683347	0\\
11.9943056417768	0\\
12.014808215219	0\\
12.0353107886612	0\\
12.0558133621033	0\\
12.0763616495731	0\\
12.0969099370428	0\\
12.1174582245125	0\\
12.1380065119823	0\\
12.1586035338403	0\\
12.1792005556983	0\\
12.1997975775563	0\\
12.2203945994143	0\\
12.2410432774846	0\\
12.2616919555548	0\\
12.2823406336251	0\\
12.3029893116954	0\\
12.3236924799435	0\\
12.3443956481917	0\\
12.3650988164398	0\\
12.385801984688	0\\
12.406562398648	0\\
12.4273228126081	0\\
12.4480832265682	0\\
12.4688436405282	0\\
12.4896639845707	0\\
12.5104843286133	0\\
12.5313046726558	0\\
12.5521250166983	0\\
12.5730079121072	0\\
12.5938908075161	0\\
12.614773702925	0\\
12.6356565983338	0\\
12.6566046110283	0\\
12.6775526237227	0\\
12.6985006364172	0\\
12.7194486491116	0\\
12.7404642942679	0\\
12.7614799394242	0\\
12.7824955845805	0\\
12.8035112297368	0\\
12.8245969793381	0\\
12.8456827289394	0\\
12.8667684785407	0\\
12.887854228142	0\\
12.9090125143938	0\\
12.9301708006455	0\\
12.9513290868973	0\\
12.972487373149	0\\
12.9937205946389	0\\
13.0149538161287	0\\
13.0361870376186	0\\
13.0574202591085	0\\
13.0787307850171	0\\
13.1000413109258	0\\
13.1213518368345	0\\
13.1426623627431	0\\
13.1640525369264	0\\
13.1854427111097	0\\
13.206832885293	0\\
13.2282230594763	0\\
13.2496952043377	0\\
13.271167349199	0\\
13.2926394940603	0\\
13.3141116389217	0\\
13.3356680588802	0\\
13.3572244788388	0\\
13.3787808987974	0\\
13.400337318756	0\\
13.4219803036483	0\\
13.4436232885406	0\\
13.4652662734329	0\\
13.4869092583252	0\\
13.5086410871513	0\\
13.5303729159775	0\\
13.5521047448036	0\\
13.5738365736297	0\\
13.5956595174591	0\\
13.6174824612885	0\\
13.639305405118	0\\
13.6611283489474	0\\
13.6830446722324	0\\
13.7049609955174	0\\
13.7268773188025	0\\
13.7487936420875	0\\
13.770805608124	0\\
13.7928175741606	0\\
13.8148295401972	0\\
13.8368415062338	0\\
13.858951377984	0\\
13.8810612497342	0\\
13.9031711214845	0\\
13.9252809932347	0\\
13.9474910360997	0\\
13.9697010789646	0\\
13.9919111218296	0\\
14.0141211646945	0\\
14.0364336494706	0\\
14.0587461342466	0\\
14.0810586190226	0\\
14.1033711037986	0\\
14.1257883079129	0\\
14.1482055120272	0\\
14.1706227161414	0\\
14.1930399202557	0\\
14.2155641320733	0\\
14.2380883438909	0\\
14.2606125557085	0\\
14.283136767526	0\\
14.3057702854045	0\\
14.3284038032829	0\\
14.3510373211614	0\\
14.3736708390398	0\\
14.3964159769655	0\\
14.4191611148912	0\\
14.4419062528169	0\\
14.4646513907425	0\\
14.4875104777278	0\\
14.510369564713	0\\
14.5332286516983	0\\
14.5560877386835	0\\
14.5790631229979	0\\
14.6020385073122	0\\
14.6250138916265	0\\
14.6479892759408	0\\
14.6710833264109	0\\
14.6941773768809	0\\
14.7172714273509	0\\
14.740365477821	0\\
14.7635805844029	0\\
14.7867956909849	0\\
14.8100107975669	0\\
14.8332259041488	0\\
14.8565644820063	0\\
14.8799030598637	0\\
14.9032416377212	0\\
14.9265802155787	0\\
14.9500447068131	0\\
14.9735091980476	0\\
14.996973689282	0\\
15.0204381805165	0\\
15.0440310551214	0\\
15.0676239297264	0\\
15.0912168043314	0\\
15.1148096789363	0\\
15.1385334385982	0\\
15.16225719826	0\\
15.1859809579218	0\\
15.2097047175837	0\\
15.2335618956234	0\\
15.2574190736632	0\\
15.281276251703	0\\
15.3051334297427	0\\
15.3291265950906	0\\
15.3531197604384	0\\
15.3771129257863	0\\
15.4011060911341	0\\
15.4252378496272	0\\
15.4493696081203	0\\
15.4735013666134	0\\
15.4976331251065	0\\
15.5219061221034	0\\
15.5461791191004	0\\
15.5704521160973	0\\
15.5947251130942	0\\
15.619142035055	0\\
15.6435589570157	0\\
15.6679758789765	0\\
15.6923928009372	0\\
15.7169563780926	0\\
15.741519955248	0\\
15.7660835324033	0\\
15.7906471095587	0\\
15.8153601179615	0\\
15.8400731263644	0\\
15.8647861347672	0\\
15.88949914317	0\\
15.9143644081624	0\\
15.9392296731548	0\\
15.9640949381471	0\\
15.9889602031395	0\\
16.0139806003288	0\\
16.039000997518	0\\
16.0640213947072	0\\
16.0890417918965	0\\
16.1142202516767	0\\
16.1393987114569	0\\
16.1645771712372	0\\
16.1897556310174	0\\
16.2150951374624	0\\
16.2404346439074	0\\
16.2657741503524	0\\
16.2911136567974	0\\
16.3166172552097	0\\
16.342120853622	0\\
16.3676244520343	0\\
16.3931280504465	0\\
16.4187988465862	0\\
16.4444696427259	0\\
16.4701404388656	0\\
16.4958112350053	0\\
16.5216524004798	0\\
16.5474935659542	0\\
16.5733347314287	0\\
16.5991758969032	0\\
16.6251906694304	0\\
16.6512054419577	0\\
16.677220214485	0\\
16.7032349870123	0\\
16.7294266758655	0\\
16.7556183647187	0\\
16.781810053572	0\\
16.8080017424252	0\\
16.8343737319713	0\\
16.8607457215174	0\\
16.8871177110636	0\\
16.9134897006097	0\\
16.9400454511807	0\\
16.9666012017517	0\\
16.9931569523226	0\\
17.0197127028936	0\\
17.0464557582069	0\\
17.0731988135202	0\\
17.0999418688336	0\\
17.1266849241469	0\\
17.1536189111414	0\\
17.1805528981359	0\\
17.2074868851304	0\\
17.2344208721249	0\\
17.2615495082445	0\\
17.2886781443641	0\\
17.3158067804836	0\\
17.3429354166032	0\\
17.3702625121198	0\\
17.3975896076364	0\\
17.424916703153	0\\
17.4522437986696	0\\
17.4797732620512	0\\
17.5073027254328	0\\
17.5348321888143	0\\
17.5623616521959	0\\
17.5900974943972	0\\
17.6178333365984	0\\
17.6455691787997	0\\
17.6733050210009	0\\
17.7012513604556	0\\
17.7291976999103	0\\
17.757144039365	0\\
17.7850903788197	0\\
17.8132514459017	0\\
17.8414125129837	0\\
17.8695735800657	0\\
17.8977346471477	0\\
17.9261147917276	0\\
17.9544949363076	0\\
17.9828750808876	0\\
18.0112552254676	0\\
18.0398589202161	0\\
18.0684626149646	0\\
18.097066309713	0\\
18.1256700044615	0\\
18.1545018545664	0\\
18.1833337046712	0\\
18.212165554776	0\\
18.2409974048809	0\\
18.2700621491713	0\\
18.2991268934616	0\\
18.328191637752	0\\
18.3572563820424	0\\
18.3865589068877	0\\
18.4158614317331	0\\
18.4451639565785	0\\
18.4744664814239	0\\
18.5040118216459	0\\
18.5335571618679	0\\
18.56310250209	0\\
18.592647842312	0\\
18.622441195706	0\\
18.6522345491	0\\
18.682027902494	0\\
18.711821255888	0\\
18.74186798467	0\\
18.771914713452	0\\
18.801961442234	0\\
18.8320081710161	0\\
18.8623138179204	0\\
18.8926194648247	0\\
18.922925111729	0\\
18.9532307586333	0\\
18.9838010512769	0\\
19.0143713439205	0\\
19.0449416365641	0\\
19.0755119292076	0\\
19.1063527942164	0\\
19.1371936592252	0\\
19.1680345242339	0\\
19.1988753892427	0\\
19.2299929618915	0\\
19.2611105345403	0\\
19.2922281071891	0\\
19.3233456798379	0\\
19.3547463174076	0\\
19.3861469549774	0\\
19.4175475925471	0\\
19.4489482301169	0\\
19.4806385226382	0\\
19.5123288151595	0\\
19.5440191076808	0\\
19.5757094002021	0\\
19.6076961882754	0\\
19.6396829763487	0\\
19.671669764422	0\\
19.7036565524953	0\\
19.7359469398787	0\\
19.768237327262	0\\
19.8005277146453	0\\
19.8328181020287	0\\
19.8654194755566	0\\
19.8980208490846	0\\
19.9306222226126	0\\
19.9632235961405	0\\
19.9961436388709	0\\
20.0290636816012	0\\
20.0619837243316	0\\
20.0949037670619	0\\
20.1281504830269	0\\
20.1613971989918	0\\
20.1946439149567	0\\
20.2278906309217	0\\
20.2614723645969	0\\
20.2950540982721	0\\
20.3286358319473	0\\
20.3622175656225	0\\
20.3961430262838	0\\
20.4300684869452	0\\
20.4639939476066	0\\
20.497919408268	0\\
20.5321976950164	0\\
20.5664759817649	0\\
20.6007542685134	0\\
20.6350325552618	0\\
20.6696731863085	0\\
20.7043138173553	0\\
20.738954448402	0\\
20.7735950794487	0\\
20.8086080231198	0\\
20.8436209667909	0\\
20.878633910462	0\\
20.9136468541332	0\\
20.949042565344	0\\
20.9844382765548	0\\
21.0198339877656	0\\
21.0552296989764	0\\
21.0910191546516	0\\
21.1268086103268	0\\
21.162598066002	0\\
21.1983875216773	0\\
21.234582265905	0\\
21.2707770101326	0\\
21.3069717543603	0\\
21.343166498588	0\\
21.3797786897193	0\\
21.4163908808506	0\\
21.453003071982	0\\
21.4896152631133	0\\
21.5266577262528	0\\
21.5637001893923	0\\
21.6007426525319	0\\
21.6377851156714	0\\
21.6752714017348	0\\
21.7127576877983	0\\
21.7502439738617	0\\
21.7877302599251	0\\
21.8256747161126	0\\
21.8636191723001	0\\
21.9015636284876	0\\
21.9395080846751	0\\
21.9779259262067	0\\
22.0163437677384	0\\
22.0547616092701	0\\
22.0931794508017	0\\
22.1320868496398	0\\
22.1709942484778	0\\
22.2099016473158	0\\
22.2488090461539	0\\
22.2882232273236	0\\
22.3276374084934	0\\
22.3670515896632	0\\
22.4064657708329	0\\
22.4464051256478	0\\
22.4863444804626	0\\
22.5262838352775	0\\
22.5662231900924	0\\
22.6067074061995	0\\
22.6471916223066	0\\
22.6876758384138	0\\
22.7281600545209	0\\
22.7692102626757	0\\
22.8102604708306	0\\
22.8513106789854	0\\
22.8923608871403	0\\
22.9339998357608	0\\
22.9756387843813	0\\
23.0172777330018	0\\
23.0589166816223	0\\
23.1011689388352	0\\
23.143421196048	0\\
23.1856734532609	0\\
23.2279257104737	0\\
23.2707449436263	0\\
23.3135641767789	0\\
23.3563834099314	0\\
23.399202643084	0\\
23.4425351018712	0\\
23.4858675606585	0\\
23.5292000194457	0\\
23.5725324782329	0\\
23.6163923812599	0\\
23.6602522842869	0\\
23.704112187314	0\\
23.747972090341	0\\
23.7923730104802	0\\
23.8367739306194	0\\
23.8811748507586	0\\
23.9255757708978	0\\
23.9705318120247	0\\
24.0154878531517	0\\
24.0604438942787	0\\
24.1053999354057	0\\
24.1509257391711	0\\
24.1964515429365	0\\
24.2419773467019	0\\
24.2875031504673	0\\
24.3336139249026	0\\
24.3797246993379	0\\
24.4258354737732	0\\
24.4719462482085	0\\
24.518657797687	0\\
24.5653693471654	0\\
24.6120808966439	0\\
24.6587924461223	0\\
24.7061212033434	0\\
24.7534499605645	0\\
24.8007787177855	0\\
24.8481074750066	0\\
24.8960705351928	0\\
24.944033595379	0\\
24.9919966555653	0\\
25.0399597157515	0\\
25.0885748730688	0\\
25.1371900303861	0\\
25.1858051877034	0\\
25.2344203450208	0\\
25.2837061315614	0\\
25.332991918102	0\\
25.3822777046427	0\\
25.4315634911833	0\\
25.4815392186928	0\\
25.5315149462023	0\\
25.5814906737117	0\\
25.6314664012212	0\\
25.6821522057685	0\\
25.7328380103157	0\\
25.783523814863	0\\
25.8342096194102	0\\
25.8856265092156	0\\
25.9370433990211	0\\
25.9884602888265	0\\
26.0398771786319	0\\
26.0920470854784	0\\
26.144216992325	0\\
26.1963868991715	0\\
26.2485568060181	0\\
26.3015026403881	0\\
26.3544484747581	0\\
26.4073943091281	0\\
26.4603401434981	0\\
26.5140858539001	0\\
26.5678315643021	0\\
26.6215772747041	0\\
26.6753229851061	0\\
26.7298936218998	0\\
26.7844642586936	0\\
26.8390348954873	0\\
26.893605532281	0\\
26.9490273164448	0\\
27.0044491006085	0\\
27.0598708847722	0\\
27.1152926689359	0\\
27.1715930662283	0\\
27.2278934635207	0\\
27.2841938608131	0\\
27.3404942581055	0\\
27.3972810276665	0\\
27.4540677972275	0\\
27.5108545667884	0\\
27.5676413363494	0\\
27.6246835621125	0\\
27.6817257878755	0\\
27.7387680136386	0\\
27.7958102394016	0\\
27.8531117250382	0\\
27.9104132106748	0\\
27.9677146963114	0\\
28.025016181948	0\\
28.0825792753208	0\\
28.1401423686935	0\\
28.1977054620663	0\\
28.2552685554391	0\\
28.3130956457549	0\\
28.3709227360708	0\\
28.4287498263866	0\\
28.4865769167025	0\\
28.5446704260258	0\\
28.602763935349	0\\
28.6608574446723	0\\
28.7189509539956	0\\
28.7773133374	0\\
28.8356757208045	0\\
28.8940381042089	0\\
28.9524004876134	0\\
29.0110342352743	0\\
29.0696679829352	0\\
29.1283017305961	0\\
29.186935478257	0\\
29.2458431140456	0\\
29.3047507498342	0\\
29.3636583856228	0\\
29.4225660214114	0\\
29.4817501049045	0\\
29.5409341883977	0\\
29.6001182718908	0\\
29.659302355384	0\\
29.7187654815208	0\\
29.7782286076576	0\\
29.8376917337944	0\\
29.8971548599312	0\\
29.9568996607722	0\\
30.0166444616132	0\\
30.0763892624543	0\\
30.1361340632953	0\\
30.1961632086509	0\\
30.2561923540065	0\\
30.316221499362	0\\
30.3762506447176	0\\
30.436566841334	0\\
30.4968830379503	0\\
30.5571992345667	0\\
30.617515431183	0\\
30.6781214246657	0\\
30.7387274181483	0\\
30.799333411631	0\\
30.8599394051136	0\\
30.9208379817972	0\\
30.9817365584807	0\\
31.0426351351642	0\\
31.1035337118477	0\\
31.1647276970968	0\\
31.2259216823458	0\\
31.2871156675948	0\\
31.3483096528438	0\\
31.4098019130839	0\\
31.4712941733241	0\\
31.5327864335642	0\\
31.5942786938043	0\\
31.656072138367	0\\
31.7178655829298	0\\
31.7796590274925	0\\
31.8414524720552	0\\
31.903550051626	0\\
31.9656476311968	0\\
32.0277452107676	0\\
32.0898427903383	0\\
32.1522474993513	0\\
32.2146522083642	0\\
32.2770569173772	0\\
32.3394616263901	0\\
32.4021765035977	0\\
32.4648913808053	0\\
32.527606258013	0\\
32.5903211352206	0\\
32.6533492654047	0\\
32.7163773955889	0\\
32.7794055257731	0\\
32.8424336559572	0\\
32.9057781690946	0\\
32.969122682232	0\\
33.0324671953693	0\\
33.0958117085067	0\\
33.1594757818323	0\\
33.2231398551579	0\\
33.2868039284835	0\\
33.3504680018091	0\\
33.4144548595075	0\\
33.478441717206	0\\
33.5424285749044	0\\
33.6064154326028	0\\
33.6707283499096	0\\
33.7350412672163	0\\
33.7993541845231	0\\
33.8636671018298	0\\
33.9283094023156	0\\
33.9929517028013	0\\
34.057594003287	0\\
34.1222363037727	0\\
34.1872113622324	0\\
34.2521864206922	0\\
34.3171614791519	0\\
34.3821365376116	0\\
34.447447781095	0\\
34.5127590245783	0\\
34.5780702680616	0\\
34.643381511545	0\\
34.7090324201268	0\\
34.7746833287086	0\\
34.8403342372905	0\\
34.9059851458723	0\\
34.9719792533529	0\\
35.0379733608335	0\\
35.1039674683141	0\\
35.1699615757948	0\\
35.2363024710846	0\\
35.3026433663745	0\\
35.3689842616644	0\\
35.4353251569542	0\\
35.5020164861975	0\\
35.5687078154407	0\\
35.6353991446839	0\\
35.7020904739271	0\\
35.76913594038	0\\
35.8361814068328	0\\
35.9032268732857	0\\
35.9702723397385	0\\
36.0376757049608	0\\
36.1050790701831	0\\
36.1724824354054	0\\
36.2398858006277	0\\
36.3076508877251	0\\
36.3754159748225	0\\
36.4431810619199	0\\
36.5109461490173	0\\
36.5790768402747	0\\
36.6472075315322	0\\
36.7153382227897	0\\
36.7834689140472	0\\
36.8519691566203	0\\
36.9204693991935	0\\
36.9889696417667	0\\
37.0574698843398	0\\
37.1263436888854	0\\
37.1952174934311	0\\
37.2640912979767	0\\
37.3329651025223	0\\
37.4022165438372	0\\
37.4714679851522	0\\
37.5407194264671	0\\
37.609970867782	0\\
37.6796040887873	0\\
37.7492373097926	0\\
37.8188705307979	0\\
37.8885037518033	0\\
37.9585229632751	0\\
38.0285421747469	0\\
38.0985613862187	0\\
38.1685805976905	0\\
38.2389900798134	0\\
38.3093995619363	0\\
38.3798090440592	0\\
38.4502185261821	0\\
38.5209622984945	0\\
38.591706070807	0\\
38.6624498431194	0\\
38.7331936154318	0\\
38.8042656863826	0\\
38.8753377573333	0\\
38.9464098282841	0\\
39.0174818992348	0\\
39.0888909542047	0\\
39.1603000091745	0\\
39.2317090641443	0\\
39.3031181191142	0\\
39.3748725127202	0\\
39.4466269063262	0\\
39.5183812999322	0\\
39.5901356935382	0\\
39.662243568804	0\\
39.7343514440698	0\\
39.8064593193355	0\\
39.8785671946013	0\\
39.95103651206	0\\
40.0235058295186	0\\
40.0959751469772	0\\
40.1684444644358	0\\
40.2412830193083	0\\
40.3141215741807	0\\
40.3869601290532	0\\
40.4597986839256	0\\
40.5330141309314	0\\
40.6062295779373	0\\
40.6794450249431	0\\
40.752660471949	0\\
40.8262603410054	0\\
40.8998602100619	0\\
40.9734600791183	0\\
41.0470599481748	0\\
41.1210516636287	0\\
41.1950433790826	0\\
41.2690350945364	0\\
41.3430268099903	0\\
41.4174177047231	0\\
41.4918085994559	0\\
41.5661994941887	0\\
41.6405903889215	0\\
41.7153877210576	0\\
41.7901850531936	0\\
41.8649823853296	0\\
41.9397797174656	0\\
42.0149906816255	0\\
42.0902016457854	0\\
42.1654126099454	0\\
42.2406235741053	0\\
42.316255315397	0\\
42.3918870566888	0\\
42.4675187979805	0\\
42.5431505392722	0\\
42.6192101664345	0\\
42.6952697935968	0\\
42.771329420759	0\\
42.8473890479213	0\\
42.9238836429179	0\\
43.0003782379145	0\\
43.0768728329111	0\\
43.1533674279077	0\\
43.2303040564086	0\\
43.3072406849096	0\\
43.3841773134105	0\\
43.4611139419115	0\\
43.5384996649489	0\\
43.6158853879863	0\\
43.6932711110238	0\\
43.7706568340612	0\\
43.8484987166093	0\\
43.9263405991575	0\\
44.0041824817057	0\\
44.0820243642538	0\\
44.1603294838729	0\\
44.2386346034919	0\\
44.3169397231109	0\\
44.39524484273	0\\
44.4740202988846	0\\
44.5527957550393	0\\
44.631571211194	0\\
44.7103466673487	0\\
44.7895995895968	0\\
44.8688525118449	0\\
44.948105434093	0\\
45.0273583563411	0\\
45.1070959123988	0\\
45.1868334684566	0\\
45.2665710245144	0\\
45.3463085805721	0\\
45.4265379845256	0\\
45.5067673884792	0\\
45.5869967924327	0\\
45.6672261963862	0\\
45.7479547148514	0\\
45.8286832333165	0\\
45.9094117517817	0\\
45.9901402702469	0\\
46.0713752310891	0\\
46.1526101919314	0\\
46.2338451527736	0\\
46.3150801136159	0\\
46.3968289125376	0\\
46.4785777114593	0\\
46.560326510381	0\\
46.6420753093027	0\\
46.7243454163926	0\\
46.8066155234824	0\\
46.8888856305723	0\\
46.9711557376621	0\\
47.0539547042701	0\\
47.136753670878	0\\
47.2195526374859	0\\
47.3023516040939	0\\
47.3856870701744	0\\
47.4690225362549	0\\
47.5523580023354	0\\
47.6356934684159	0\\
47.7195731680798	0\\
47.8034528677438	0\\
47.8873325674077	0\\
47.9712122670716	0\\
48.0556440359198	0\\
48.1400758047679	0\\
48.2245075736161	0\\
48.3089393424642	0\\
48.3939311234603	0\\
48.4789229044563	0\\
48.5639146854524	0\\
48.6489064664485	0\\
48.7344663172648	0\\
48.820026168081	0\\
48.9055860188973	0\\
48.9911458697136	0\\
49.0772819670679	0\\
49.1634180644223	0\\
49.2495541617767	0\\
49.3356902591311	0\\
49.4224109073406	0\\
49.5091315555501	0\\
49.5958522037596	0\\
49.6825728519691	0\\
49.769886487838	0\\
49.857200123707	0\\
49.944513759576	0\\
50.0318273954449	0\\
50.1197425956162	0\\
50.2076577957875	0\\
50.2955729959588	0\\
50.3834881961302	0\\
50.4720136825001	0\\
50.5605391688701	0\\
50.64906465524	0\\
50.73759014161	0\\
50.8267347882915	0\\
50.915879434973	0\\
51.0050240816545	0\\
51.094168728336	0\\
51.1839415683769	0\\
51.2737144084178	0\\
51.3634872484587	0\\
51.4532600884996	0\\
51.5436703203087	0\\
51.6340805521178	0\\
51.7244907839268	0\\
51.8149010157359	0\\
51.9059580087448	0\\
51.9970150017537	0\\
52.0880719947626	0\\
52.1791289877716	0\\
52.270842290771	0\\
52.3625555937704	0\\
52.4542688967698	0\\
52.5459821997693	0\\
52.6383615465513	0\\
52.7307408933333	0\\
52.8231202401153	0\\
52.9154995868973	0\\
53.0085549039533	0\\
53.1016102210094	0\\
53.1946655380654	0\\
53.2877208551214	0\\
53.3814622684928	0\\
53.4752036818643	0\\
53.5689450952358	0\\
53.6626865086073	0\\
53.7571243510415	0\\
53.8515621934757	0\\
53.9460000359099	0\\
54.0404378783442	0\\
54.1355826971555	0\\
54.2307275159668	0\\
54.325872334778	0\\
54.4210171535893	0\\
54.5168797176294	0\\
54.6127422816695	0\\
54.7086048457095	0\\
54.8044674097496	0\\
54.9010587178797	0\\
54.9976500260098	0\\
55.0942413341398	0\\
55.1908326422699	0\\
55.2881639308243	0\\
55.3854952193786	0\\
55.4828265079329	0\\
55.5801577964873	0\\
55.6782405480441	0\\
55.776323299601	0\\
55.8744060511579	0\\
55.9724888027147	0\\
56.0713347542828	0\\
56.1701807058509	0\\
56.2690266574189	0\\
56.367872608987	0\\
56.4674937609263	0\\
56.5671149128656	0\\
56.6667360648049	0\\
56.7663572167441	0\\
56.8667658411891	0\\
56.9671744656341	0\\
57.0675830900791	0\\
57.1679917145241	0\\
57.2692003658787	0\\
57.3704090172333	0\\
57.4716176685879	0\\
57.5728263199425	0\\
57.6748478434472	0\\
57.7768693669519	0\\
57.8788908904565	0\\
57.9809124139612	0\\
58.0837599560178	0\\
58.1866074980744	0\\
58.2894550401309	0\\
58.3923025821875	0\\
58.4959896007833	0\\
58.5996766193791	0\\
58.7033636379749	0\\
58.8070506565707	0\\
58.911590931702	0\\
59.0161312068333	0\\
59.1206714819646	0\\
59.2252117570959	0\\
59.3306194021433	0\\
59.4360270471907	0\\
59.541434692238	0\\
59.6468423372854	0\\
59.7351317529641	0\\
59.8234211686427	0\\
59.9117105843214	0\\
60	0\\
};
\addlegendentry{Sani}

\addplot [color=mycolor2, line width=2.0pt]
  table[row sep=crcr]{%
0	1\\
0.00199053585276749	0.999800966224574\\
0.00398107170553497	0.999601972063591\\
0.00597160755830246	0.999403017509168\\
0.00796214341106994	0.999204102553421\\
0.0123567669950473	0.998765086434587\\
0.0167513905790247	0.998326263204425\\
0.0211460141630021	0.997887632778187\\
0.0255406377469794	0.997449195071161\\
0.0299388563931551	0.997010591567099\\
0.0343370750393307	0.996572180928032\\
0.0387352936855064	0.996133963069153\\
0.043133512331682	0.99569593790569\\
0.0475471725191068	0.995256568522189\\
0.0519608327065316	0.994817393018614\\
0.0563744928939564	0.994378411309413\\
0.0607881530813811	0.993939623309069\\
0.0652173199794049	0.993499488340007\\
0.0696464868774286	0.993059548270902\\
0.0740756537754524	0.992619803015451\\
0.0785048206734761	0.992180252487385\\
0.0829496187548012	0.991739346393314\\
0.0873944168361263	0.991298636229551\\
0.0918392149174515	0.990858121909031\\
0.0962840129987766	0.990417803344723\\
0.100744568101794	0.989976120540888\\
0.105205123204811	0.989534634708167\\
0.109665678307829	0.989093345758722\\
0.114126233410846	0.98865225360475\\
0.118602673075784	0.988209788429285\\
0.123079112740721	0.987767521276361\\
0.127555552405659	0.987325452057355\\
0.132031992070596	0.986883580683681\\
0.136524445545197	0.986440327398767\\
0.141016899019798	0.985997273198607\\
0.1455093524944	0.985554417993782\\
0.150001805969001	0.985111761694916\\
0.154510404254387	0.984667714483994\\
0.159019002539772	0.984223867430996\\
0.163527600825158	0.983780220445699\\
0.168036199110544	0.983336773437922\\
0.172561074936287	0.982891926409921\\
0.17708595076203	0.982447280624143\\
0.181610826587773	0.982002835989549\\
0.186135702413515	0.981558592415141\\
0.190676990405524	0.981112939589773\\
0.195218278397533	0.980667489102239\\
0.199759566389542	0.980222240860675\\
0.204300854381551	0.979777194773254\\
0.208858690959197	0.979330730093007\\
0.213416527536843	0.978884468857699\\
0.217974364114488	0.978438410974622\\
0.222532200692134	0.977992556351115\\
0.227106724202768	0.977545273669947\\
0.231681247713401	0.9770981955525\\
0.236255771224035	0.976651321905218\\
0.240830294734669	0.976204652634586\\
0.245421645441331	0.975756545721086\\
0.250012996147992	0.975308644501961\\
0.254604346854654	0.974860948882791\\
0.259195697561315	0.974413458769199\\
0.263804017673384	0.97396452130557\\
0.268412337785452	0.973515790679032\\
0.273020657897521	0.973067266794291\\
0.27762897800959	0.972618949556096\\
0.282254411746462	0.972169175133965\\
0.286879845483334	0.971719608703905\\
0.291505279220206	0.971270250169735\\
0.296130712957079	0.970821099435315\\
0.300773406613671	0.970370481551446\\
0.305416100270264	0.969920072827098\\
0.310058793926856	0.969469873165189\\
0.314701487583449	0.969019882468679\\
0.319361589521735	0.968568414527928\\
0.324021691460022	0.968117156926825\\
0.328681793398309	0.967666109567373\\
0.333341895336596	0.96721527235162\\
0.338019556071109	0.966762947660809\\
0.342697216805623	0.966310834502666\\
0.347374877540136	0.965858932778265\\
0.35205253827465	0.965407242388727\\
0.356747910447269	0.964954054161307\\
0.361443282619888	0.964501078672678\\
0.366138654792507	0.964048315822974\\
0.370834026965126	0.963595765512378\\
0.375547265516765	0.963141706854101\\
0.380260504068404	0.962687862154076\\
0.384973742620042	0.962234231311482\\
0.389686981171681	0.961780814225548\\
0.394418243246227	0.961325878146088\\
0.399149505320772	0.9608711572579\\
0.403880767395317	0.960416651459195\\
0.408612029469863	0.959962360648233\\
0.41336147459102	0.95950654004652\\
0.418110919712177	0.959050935882899\\
0.422860364833334	0.958595548054597\\
0.427609809954491	0.958140376458892\\
0.432377599961091	0.957683664131787\\
0.437145389967691	0.957227169503627\\
0.44191317997429	0.956770892470644\\
0.44668096998089	0.956314832929117\\
0.451467269141767	0.95585722156286\\
0.456253568302644	0.955399829170684\\
0.461039867463521	0.954942655647808\\
0.465826166624398	0.954485700889499\\
0.47063114173014	0.954027183053504\\
0.475436116835881	0.953568885481266\\
0.480241091941623	0.953110808066973\\
0.485046067047365	0.952652950704866\\
0.489869887370685	0.952193518858141\\
0.494693707694006	0.951734308579639\\
0.499517528017326	0.951275319762505\\
0.504341348340647	0.950816552299938\\
0.509184185682064	0.950356198789362\\
0.51402702302348	0.949896068166528\\
0.518869860364897	0.949436160323523\\
0.523712697706314	0.948976475152482\\
0.528574726619363	0.948515192194123\\
0.533436755532412	0.94805413345836\\
0.53829878444546	0.9475932988362\\
0.543160813358509	0.947132688218705\\
0.548042211039569	0.946670467911148\\
0.55292360872063	0.946208473176644\\
0.55780500640169	0.945746703905109\\
0.56268640408275	0.945285159986513\\
0.56758735048985	0.944821994302442\\
0.572488296896949	0.944359055557774\\
0.577389243304049	0.943896343641316\\
0.582290189711149	0.943433858441928\\
0.587210867601438	0.942969739227532\\
0.592131545491728	0.942505848335072\\
0.597052223382018	0.942042185652226\\
0.601972901272307	0.941578751066727\\
0.606913496388328	0.941113670027156\\
0.611854091504349	0.940648818708544\\
0.61679468662037	0.940184196997423\\
0.621735281736391	0.93971980478038\\
0.626695982628195	0.939253753499626\\
0.631656683519999	0.938787933355637\\
0.636617384411802	0.938322344233782\\
0.641578085303606	0.937856986019487\\
0.646559083672327	0.937389955931203\\
0.651540082041047	0.936923158412603\\
0.656521080409768	0.936456593347874\\
0.661502078778488	0.935990260621258\\
0.666503569383853	0.935522243020775\\
0.671505059989219	0.935054459440328\\
0.676506550594584	0.9345869097629\\
0.68150804119995	0.934119593871536\\
0.686530221876942	0.933650579917601\\
0.691552402553934	0.933181801451809\\
0.696574583230927	0.932713258355924\\
0.701596763907919	0.932244950511769\\
0.706639835866396	0.931774931201955\\
0.711682907824873	0.931305148866502\\
0.71672597978335	0.930835603385932\\
0.721769051741827	0.930366294640828\\
0.72683321952555	0.929895260818972\\
0.731897387309272	0.929424465476157\\
0.736961555092995	0.928953908491642\\
0.742025722876718	0.928483589744751\\
0.747111194287107	0.928011532111639\\
0.752196665697496	0.927539714481054\\
0.757282137107885	0.927068136730976\\
0.762367608518274	0.926596798739446\\
0.767474594900923	0.926123707829833\\
0.772581581283573	0.92565085846542\\
0.777688567666222	0.925178250522881\\
0.782795554048871	0.924705883878954\\
0.78792427039055	0.924231750056451\\
0.793052986732229	0.92375785934139\\
0.798181703073909	0.92328421160912\\
0.803310419415588	0.922810806735053\\
0.808461084325784	0.92233562019779\\
0.81361174923598	0.921860678350167\\
0.818762414146177	0.921385981066187\\
0.823913079056373	0.920911528219915\\
0.82908591482452	0.920435278999237\\
0.834258750592667	0.919959276070753\\
0.839431586360813	0.919483519307092\\
0.84460442212896	0.919008008580951\\
0.849799655005291	0.918530686519671\\
0.854994887881621	0.918053612373913\\
0.860190120757951	0.917576786014913\\
0.865385353634282	0.917100207313973\\
0.870603213699137	0.916621802082683\\
0.875821073763992	0.916143646411434\\
0.881038933828848	0.915665740170044\\
0.886256793893703	0.915188083228396\\
0.891497515445601	0.914708584294209\\
0.896738236997499	0.914229336586226\\
0.901978958549397	0.91375033997282\\
0.907219680101295	0.913271594322435\\
0.912483501408722	0.912790990976273\\
0.917747322716149	0.912310640544553\\
0.923011144023576	0.911830542894181\\
0.928274965331004	0.911350697892133\\
0.933562128954513	0.91086897922381\\
0.938849292578023	0.910387515180711\\
0.944136456201532	0.909906305628247\\
0.949423619825042	0.9094253504319\\
0.954734372782572	0.908942505319858\\
0.960045125740102	0.908459916566847\\
0.965355878697632	0.907977584036759\\
0.970666631655162	0.907495507593555\\
0.976001225418392	0.907011524710139\\
0.981335819181621	0.906527799943077\\
0.98667041294485	0.906044333154711\\
0.99200500670808	0.905561124207457\\
0.997363697280812	0.905075992016798\\
1.00272238785355	0.904591119723828\\
1.00808107842628	0.904106507189311\\
1.01343976899901	0.903622154274089\\
1.0188228172289	0.903135861009032\\
1.02420586545878	0.902649829447541\\
1.02958891368867	0.902164059448776\\
1.03497196191855	0.901678550871976\\
1.04037963341625	0.901191084546779\\
1.04578730491395	0.900703881756096\\
1.05119497641164	0.900216942357454\\
1.05660264790934	0.899730266208457\\
1.06203521338424	0.899241614593978\\
1.06746777885913	0.898753228370593\\
1.07290034433402	0.898265107394167\\
1.07833290980891	0.897777251520641\\
1.08379064506731	0.897287402150227\\
1.08924838032571	0.89679782005369\\
1.09470611558411	0.896308505085197\\
1.10016385084252	0.895819457098998\\
1.1056470369148	0.89532839726288\\
1.11113022298707	0.894837606610201\\
1.11661340905935	0.894347084993404\\
1.12209659513163	0.893856832265011\\
1.12760551858231	0.893364548988525\\
1.13311444203298	0.892872536832441\\
1.13862336548365	0.892380795647443\\
1.14413228893432	0.891889325284295\\
1.14966724184116	0.891395805336023\\
1.15520219474799	0.890902558473135\\
1.16073714765482	0.890409584544521\\
1.16627210056166	0.889916883399157\\
1.17183338076584	0.889422113274972\\
1.17739466097002	0.888927618229821\\
1.18295594117419	0.888433398110766\\
1.18851722137837	0.887939452764956\\
1.19410513262804	0.887443418681871\\
1.19969304387772	0.886947661700802\\
1.20528095512739	0.886452181666949\\
1.21086886637706	0.885956978425599\\
1.21648371859449	0.885459666304881\\
1.22209857081192	0.884962633339192\\
1.22771342302936	0.884465879371833\\
1.23332827524679	0.883969404246196\\
1.23897038457538	0.883470799716298\\
1.24461249390397	0.88297247642518\\
1.25025460323256	0.88247443421421\\
1.25589671256115	0.881976672924843\\
1.26156640151281	0.881476761315591\\
1.26723609046446	0.880977133060337\\
1.27290577941612	0.880477787998475\\
1.27857546836778	0.879978725969489\\
1.28427306633634	0.879477492274306\\
1.28997066430491	0.87897654408059\\
1.29566826227347	0.87847588122572\\
1.30136586024204	0.877975503547168\\
1.30709170327692	0.877472932450881\\
1.31281754631181	0.876970649036546\\
1.31854338934669	0.876468653139488\\
1.32426923238158	0.875966944595127\\
1.33002366371043	0.875463020435979\\
1.33577809503928	0.874959386173111\\
1.34153252636813	0.874456041639751\\
1.34728695769699	0.873952986669225\\
1.35307032782915	0.873447693438196\\
1.35885369796131	0.872942692352458\\
1.36463706809348	0.872437983243101\\
1.37042043822564	0.871933565941313\\
1.37622881553645	0.871427261081936\\
1.38203719284726	0.870921250218134\\
1.38784557015807	0.870415533179194\\
1.39365394746888	0.8699101097945\\
1.39948194569202	0.869403274043322\\
1.40530994391516	0.868896733589871\\
1.41113794213829	0.868390488262098\\
1.41696594036143	0.867884537888055\\
1.4228137248248	0.867377166080896\\
1.42866150928816	0.8668700908871\\
1.43450929375153	0.866363312133263\\
1.44035707821489	0.865856829646085\\
1.44622478964653	0.865348918872603\\
1.45209250107816	0.864841306039087\\
1.45796021250979	0.864333990970764\\
1.46382792394143	0.863826973492967\\
1.46971570462452	0.863318520813743\\
1.4756034853076	0.862810367412191\\
1.48149126599069	0.862302513112158\\
1.48737904667378	0.861794957737589\\
1.49328704060906	0.861285960171744\\
1.49919503454433	0.860777263232538\\
1.50510302847961	0.860268866742415\\
1.51101102241489	0.85976077052392\\
1.51693937480906	0.859251225094853\\
1.52286772720322	0.858741981652748\\
1.52879607959739	0.858233040018632\\
1.53472443199155	0.857724400013634\\
1.540673290528	0.85721430364134\\
1.54662214906446	0.856704510627922\\
1.55257100760091	0.856195020792968\\
1.55851986613736	0.855685833956176\\
1.56448937947106	0.855175183588523\\
1.57045889280476	0.854664837963321\\
1.57642840613847	0.854154796898707\\
1.58239791947216	0.853645060212928\\
1.58838823791446	0.853133852768248\\
1.59437855635675	0.852622951461403\\
1.60036887479904	0.852112356109064\\
1.60635919324133	0.851602066528007\\
1.61237046959671	0.851090298825601\\
1.61838174595208	0.850578838668458\\
1.62439302230746	0.850067685871759\\
1.63040429866284	0.849556840250797\\
1.63643668672632	0.849044509141099\\
1.6424690747898	0.848532486996209\\
1.64850146285328	0.848020773629804\\
1.65453385091676	0.847509368855672\\
1.66058750665404	0.846996471121334\\
1.66664116239132	0.846483883783666\\
1.67269481812859	0.845971606654821\\
1.67874847386587	0.845459639547067\\
1.68482355500326	0.844946171939865\\
1.69089863614064	0.844433016173667\\
1.69697371727803	0.843920172059086\\
1.70304879841541	0.843407639406848\\
1.70914546464501	0.842893598632226\\
1.71524213087462	0.842379871155593\\
1.72133879710422	0.841866456786001\\
1.72743546333383	0.841353355332617\\
1.73355387572031	0.840838738101625\\
1.73967228810678	0.840324435638374\\
1.74579070049326	0.839810447750335\\
1.75190911287974	0.839296774245095\\
1.75804943463327	0.838781577211003\\
1.76418975638681	0.838266696427363\\
1.77033007814035	0.837752131700046\\
1.77647039989388	0.837237882835044\\
1.78263279664118	0.836722102572907\\
1.78879519338848	0.836206640057116\\
1.79495759013578	0.835691495091926\\
1.80111998688309	0.835176667481707\\
1.80730462566597	0.83466030057448\\
1.81348926444885	0.834144252922798\\
1.81967390323173	0.833628524329275\\
1.82585854201462	0.833113114596647\\
1.83206559217232	0.832596157563222\\
1.83827264233002	0.832079521308056\\
1.84447969248772	0.831563205632102\\
1.85068674264543	0.831047210336437\\
1.85691637550678	0.830529659659625\\
1.86314600836813	0.830012431297478\\
1.86937564122948	0.829495525049269\\
1.87560527409083	0.828978940714394\\
1.88185766319929	0.828460792824378\\
1.88811005230775	0.827942968799327\\
1.89436244141622	0.827425468436813\\
1.90061483052468	0.826908291534531\\
1.90689015093446	0.82638954286937\\
1.91316547134424	0.825871119633498\\
1.91944079175402	0.825353021622763\\
1.92571611216381	0.824835248633138\\
1.93201454150218	0.82431589555281\\
1.93831297084055	0.823796869480356\\
1.94461140017892	0.82327817020988\\
1.9509098295173	0.82275979753561\\
1.95723154808653	0.822239836316342\\
1.96355326665576	0.821720203698045\\
1.96987498522499	0.821200899473055\\
1.97619670379422	0.820681923433835\\
1.98254189319299	0.820161350383997\\
1.98888708259176	0.819641107542846\\
1.99523227199053	0.819121194700925\\
2.0015774613893	0.818601611648909\\
2.00794630624256	0.818080422968896\\
2.01431515109582	0.817559566120187\\
2.02068399594908	0.817039040891511\\
2.02705284080234	0.816518847071732\\
2.03344552746868	0.81599703896264\\
2.03983821413503	0.815475564322523\\
2.04623090080137	0.814954422938272\\
2.05262358746771	0.814433614596916\\
2.05904030495676	0.813911183187734\\
2.06545702244581	0.813389086900498\\
2.07187373993486	0.812867325520238\\
2.07829045742391	0.812345898832122\\
2.084731396657	0.811822840242571\\
2.0911723358901	0.811300118443405\\
2.0976132751232	0.81077773321777\\
2.1040542143563	0.810255684348949\\
2.11051956957009	0.80973199457788\\
2.11698492478388	0.809208643281421\\
2.12345027999767	0.808685630240805\\
2.12991563521146	0.808162955237411\\
2.13640560238783	0.807638630293133\\
2.1428955695642	0.807114645523626\\
2.14938553674057	0.806591000708187\\
2.15587550391694	0.80606769562626\\
2.16239028199053	0.805542731431159\\
2.16890506006412	0.805018109127202\\
2.17541983813771	0.804493828491726\\
2.1819346162113	0.803969889302215\\
2.18847440630977	0.803444281756945\\
2.19501419640825	0.802919017835604\\
2.20155398650673	0.802394097313542\\
2.20809377660521	0.801869519966257\\
2.21465878275268	0.801343264895742\\
2.22122378890015	0.800817355198623\\
2.22778879504762	0.800291790648236\\
2.23435380119508	0.799766571018068\\
2.24094422983074	0.799239664213076\\
2.24753465846639	0.798713104547852\\
2.25412508710205	0.79818689179369\\
2.2607155157377	0.797661025722036\\
2.2673315759421	0.797133462923754\\
2.27394763614649	0.796606249048758\\
2.28056369635089	0.796079383866274\\
2.28717975655528	0.795552867145682\\
2.29382166035897	0.795024644024026\\
2.30046356416266	0.794496771626599\\
2.30710546796636	0.793969249720531\\
2.31374737177005	0.793442078073106\\
2.32041533418655	0.792913190227289\\
2.32708329660306	0.79238465492435\\
2.33375125901957	0.791856471929293\\
2.34041922143607	0.791328641007278\\
2.34711345989709	0.790799084013701\\
2.3538076983581	0.790269881399574\\
2.36050193681912	0.789741032927747\\
2.36719617528013	0.78921253836123\\
2.37391691043736	0.788682307712866\\
2.3806376455946	0.788152433298758\\
2.38735838075183	0.787622914879568\\
2.39407911590907	0.787093752216125\\
2.40082657131283	0.786562843351296\\
2.40757402671659	0.786032292594027\\
2.41432148212035	0.785502099702767\\
2.42106893752412	0.78497226443613\\
2.42784333963791	0.784440672740509\\
2.4346177417517	0.783909441044526\\
2.4413921438655	0.783378569104385\\
2.44816654597929	0.782848056676455\\
2.4549681243632	0.782315777471894\\
2.4617697027471	0.781783860178114\\
2.46857128113101	0.781252304549043\\
2.47537285951492	0.780721110338774\\
2.48220184698832	0.780188138873987\\
2.48903083446173	0.779655531250498\\
2.49585982193513	0.779123287219926\\
2.50268880940853	0.778591406534057\\
2.50954544177924	0.778057738009249\\
2.51640207414995	0.77752443527591\\
2.52325870652066	0.776991498083317\\
2.53011533889137	0.776458926180916\\
2.53699985562036	0.775924555699482\\
2.54388437234935	0.775390552979692\\
2.55076888907833	0.774856917768448\\
2.55765340580732	0.774323649812824\\
2.5645660494942	0.773788572425149\\
2.57147869318108	0.773253864789594\\
2.57839133686796	0.77271952665065\\
2.58530398055484	0.772185557752987\\
2.59224499685821	0.771649768466047\\
2.59918601316159	0.771114350942289\\
2.60612702946496	0.770579304923762\\
2.61306804576833	0.770044630152691\\
2.62003768421203	0.769508123871382\\
2.62700732265572	0.768971991385277\\
2.63397696109942	0.768436232433945\\
2.64094659954311	0.767900846757137\\
2.64794511298781	0.767363618329061\\
2.65494362643251	0.766826765749488\\
2.6619421398772	0.766290288755473\\
2.6689406533219	0.765754187084253\\
2.67596829852112	0.765216231260862\\
2.68299594372033	0.764678653360925\\
2.69002358891955	0.764141453118946\\
2.69705123411877	0.763604630269612\\
2.70410827073433	0.763065941785624\\
2.71116530734989	0.762527633321965\\
2.71822234396545	0.761989704610547\\
2.72527938058101	0.761452155383474\\
2.73236607296869	0.76091272882424\\
2.73945276535637	0.760373684404594\\
2.74653945774405	0.759835021853821\\
2.75362615013173	0.759296740901399\\
2.76074276611797	0.758756570800896\\
2.76785938210422	0.758216784981954\\
2.77497599809046	0.757677383171192\\
2.78209261407671	0.757138365095425\\
2.78923942490227	0.75659744594438\\
2.79638623572783	0.75605691323991\\
2.80353304655339	0.755516766705927\\
2.81067985737896	0.754977006066542\\
2.81785713902212	0.754435332216418\\
2.82503442066529	0.753894047001389\\
2.83221170230846	0.753353150142619\\
2.83938898395163	0.752812641361476\\
2.84659701614953	0.752270207103349\\
2.85380504834743	0.751728163692703\\
2.86101308054533	0.751186510847916\\
2.86822111274323	0.750645248287567\\
2.87546017926332	0.750102047835964\\
2.88269924578341	0.749559240468486\\
2.88993831230351	0.74901682590068\\
2.8971773788236	0.748474803848298\\
2.90444776780364	0.74793083132028\\
2.91171815678368	0.747387254137715\\
2.91898854576373	0.746844072013277\\
2.92625893474377	0.746301284659848\\
2.93356093845992	0.74575653409719\\
2.94086294217606	0.745212181166401\\
2.94816494589221	0.744668225577238\\
2.95546694960836	0.744124667039666\\
2.96280086487659	0.743579132383658\\
2.97013478014483	0.743033997671468\\
2.97746869541306	0.74248926260989\\
2.9848026106813	0.741944926905929\\
2.99216873909201	0.741398601985082\\
2.99953486750271	0.740852679346006\\
3.00690099591342	0.740307158692483\\
3.01426712432413	0.739762039728517\\
3.02166577186631	0.739214918291705\\
3.02906441940849	0.73866820150105\\
3.03646306695067	0.738121889057279\\
3.04386171449286	0.737575980661342\\
3.05129319192691	0.737028056355296\\
3.05872466936097	0.736480539086698\\
3.06615614679503	0.735933428553172\\
3.07358762422909	0.735386724452565\\
3.08105224696524	0.734837990836195\\
3.0885168697014	0.734289666675927\\
3.09598149243755	0.733741751666232\\
3.1034461151737	0.733194245501807\\
3.11094420406679	0.73264469599324\\
3.11844229295988	0.732095558387337\\
3.12594038185297	0.731546832375364\\
3.13343847074605	0.73099851764882\\
3.14097035080817	0.730448145625591\\
3.14850223087028	0.729898187979896\\
3.1560341109324	0.72934864439975\\
3.16356599099451	0.728799514573401\\
3.17113199342146	0.728248313229628\\
3.17869799584842	0.7276975287672\\
3.18626399827537	0.727147160870822\\
3.19383000070233	0.726597209225441\\
3.20143046110215	0.726045171707446\\
3.20903092150198	0.725493553603973\\
3.21663138190181	0.724942354596369\\
3.22423184230164	0.724391574366223\\
3.2318671021242	0.723838693674113\\
3.23950236194676	0.72328623495966\\
3.24713762176933	0.722734197900798\\
3.25477288159189	0.722182582175702\\
3.26244328804483	0.721628851176091\\
3.27011369449776	0.721075544747811\\
3.2777841009507	0.720522662565322\\
3.28545450740363	0.719970204303337\\
3.29316041266953	0.719415615792244\\
3.30086631793543	0.718861454477186\\
3.30857222320133	0.718307720029096\\
3.31627812846722	0.717754412119162\\
3.32401989110489	0.717198958727244\\
3.33176165374256	0.716643935187745\\
3.33950341638023	0.716089341168013\\
3.3472451790179	0.715535176335652\\
3.35502316313618	0.714978850594577\\
3.36280114725446	0.714422957394542\\
3.37057913137275	0.71386749639925\\
3.37835711549103	0.713312467272663\\
3.38617169154667	0.712755261564645\\
3.3939862676023	0.712198491119173\\
3.40180084365794	0.711642155596242\\
3.40961541971357	0.71108625465611\\
3.41746696420648	0.710528161242871\\
3.42531850869938	0.70997050584718\\
3.43317005319229	0.709413288125262\\
3.4410215976852	0.708856507733608\\
3.44891049346576	0.708297518741606\\
3.45679938924632	0.707738970556296\\
3.46468828502688	0.707180862830069\\
3.47257718080744	0.706623195215588\\
3.48050381756295	0.70606330260902\\
3.48843045431847	0.705503853633135\\
3.49635709107398	0.704944847936423\\
3.50428372782949	0.704386285167652\\
3.51224850159594	0.703825480790743\\
3.52021327536239	0.703265122904\\
3.52817804912884	0.702705211151945\\
3.53614282289529	0.702145745179383\\
3.54414613607264	0.701584020762706\\
3.55214944924998	0.70102274573182\\
3.56015276242733	0.700461919727209\\
3.56815607560468	0.699901542389649\\
3.57619833889781	0.699338889421824\\
3.58424060219095	0.698776688772422\\
3.59228286548408	0.698214940077823\\
3.60032512877721	0.697653642974701\\
3.60840675922679	0.697090052849298\\
3.61648838967638	0.69652691801261\\
3.62457002012596	0.695964238096835\\
3.63265165057554	0.695402012734474\\
3.64077307248204	0.694837476693267\\
3.64889449438855	0.694273398949471\\
3.65701591629505	0.693709779131034\\
3.66513733820155	0.693146616866207\\
3.67329898353354	0.692581125979407\\
3.68146062886554	0.69201609643792\\
3.68962227419754	0.691451527865367\\
3.69778391952953	0.690887419885673\\
3.70598622860803	0.690320965013693\\
3.71418853768653	0.68975497457501\\
3.72239084676502	0.689189448188837\\
3.73059315584352	0.688624385474701\\
3.73883657626523	0.688056957352052\\
3.74707999668693	0.687489996791518\\
3.75532341710863	0.686923503407827\\
3.76356683753034	0.686357476816024\\
3.77185182516863	0.685789065991284\\
3.78013681280692	0.685221125899181\\
3.78842180044522	0.684653656149876\\
3.79670678808351	0.684086656353853\\
3.80503380745825	0.683517253174174\\
3.81336082683299	0.68294832394027\\
3.82168784620774	0.682379868257652\\
3.83001486558248	0.681811885732154\\
3.83838438958511	0.681241480372227\\
3.84675391358775	0.680671552214709\\
3.85512343759039	0.680102100860372\\
3.86349296159302	0.679533125910321\\
3.8719054722877	0.678961708327936\\
3.88031798298237	0.678390771249063\\
3.88873049367705	0.677820314269648\\
3.89714300437172	0.677250336985976\\
3.90559899284027	0.676677896943418\\
3.91405498130882	0.6761059407509\\
3.92251096977737	0.67553446799945\\
3.93096695824593	0.674963478280447\\
3.93946692479213	0.674390005341485\\
3.94796689133833	0.673817019645496\\
3.95646685788454	0.6732445207785\\
3.96496682443074	0.672672508326872\\
3.97351127881372	0.672097991852108\\
3.98205573319669	0.671523966060664\\
3.99060018757967	0.670950430533457\\
3.99914464196265	0.670377384851762\\
4.00773410425464	0.669801813952518\\
4.01632356654662	0.669226737225463\\
4.02491302883861	0.668652154246312\\
4.0335024911306	0.668078064591145\\
4.04213749081375	0.667501428202121\\
4.05077249049689	0.666925289523683\\
4.05940749018004	0.666349648126243\\
4.06804248986318	0.665774503580586\\
4.07672356764034	0.665196790350778\\
4.08540464541749	0.664619578420695\\
4.09408572319465	0.664042867355345\\
4.10276680097181	0.663466656720111\\
4.11149450824228	0.662887855061067\\
4.12022221551276	0.662309558342786\\
4.12894992278323	0.661731766124765\\
4.1376776300537	0.661154477966881\\
4.14645252883691	0.660574576070691\\
4.15522742762012	0.659995182809351\\
4.16400232640333	0.659416297736735\\
4.17277722518654	0.65883792040711\\
4.18159988926615	0.658256906183998\\
4.19042255334575	0.657676404344155\\
4.19924521742536	0.657096414435723\\
4.20806788150496	0.656516936007239\\
4.21693889629988	0.655934797108311\\
4.22580991109479	0.655353174396673\\
4.23468092588971	0.654772067414616\\
4.24355194068463	0.65419147570484\\
4.2524719033702	0.65360819952835\\
4.26139186605577	0.653025443400057\\
4.27031182874135	0.652443206856288\\
4.27923179142692	0.651861489433783\\
4.28820131218307	0.651277063058229\\
4.29717083293922	0.650693160650104\\
4.30614035369538	0.650109781739644\\
4.31510987445153	0.649526925857508\\
4.32412957614907	0.64894133607789\\
4.33314927784661	0.648356274244652\\
4.34216897954416	0.647771739881819\\
4.3511886812417	0.647187732513841\\
4.36025919971018	0.646600965839935\\
4.36932971817866	0.646014731152517\\
4.37840023664715	0.645429027969268\\
4.38747075511563	0.644843855808304\\
4.39659274002011	0.644255898424151\\
4.40571472492458	0.643668477129288\\
4.41483670982906	0.64308159143492\\
4.42395869473353	0.642495240852696\\
4.43313280988389	0.641906078613815\\
4.44230692503424	0.6413174566313\\
4.45148104018459	0.640729374409741\\
4.46065515533495	0.640141831454181\\
4.469882079136	0.639551449876289\\
4.47910900293706	0.638961612787743\\
4.48833592673811	0.638372319686379\\
4.49756285053916	0.637783570070497\\
4.50684327607396	0.637191954342308\\
4.51612370160877	0.636600887404\\
4.52540412714357	0.63601036874651\\
4.53468455267837	0.635420397861246\\
4.54401918924822	0.634827532764995\\
4.55335382581807	0.634235220828548\\
4.56268846238792	0.633643461535791\\
4.57202309895777	0.633052254371092\\
4.58141267108196	0.632458124365989\\
4.59080224320616	0.631864551961717\\
4.60019181533036	0.631271536634958\\
4.60958138745456	0.630679077862886\\
4.61902663678989	0.630083666983262\\
4.62847188612523	0.629488818218553\\
4.63791713546056	0.628894531038076\\
4.6473623847959	0.628300804911653\\
4.65686406984621	0.627704096806657\\
4.66636575489653	0.627107955405635\\
4.67586743994685	0.626512380170377\\
4.68536912499717	0.625917370563187\\
4.69492802222378	0.625319348448563\\
4.7044869194504	0.624721897704004\\
4.71404581667701	0.624125017783607\\
4.72360471390363	0.623528708141986\\
4.73320764525141	0.622930225209737\\
4.74281057659919	0.622332316720678\\
4.75241350794698	0.62173498212344\\
4.76201643929476	0.621138220867184\\
4.77165794452804	0.620539638734484\\
4.78129944976131	0.619941633446933\\
4.79094095499459	0.619344204448633\\
4.80058246022786	0.618747351184223\\
4.81026314694842	0.618148651095781\\
4.81994383366897	0.61755053030969\\
4.82962452038952	0.616952988265419\\
4.83930520711008	0.616356024402976\\
4.84902565638322	0.61575718975028\\
4.85874610565637	0.615158936908951\\
4.86846655492951	0.614561265313718\\
4.87818700420266	0.613964174399858\\
4.88794780523678	0.613365188561654\\
4.89770860627089	0.612766787096365\\
4.90746940730501	0.612168969433875\\
4.91723020833913	0.611571735004624\\
4.92703195872998	0.610972581341064\\
4.93683370912084	0.610374014665247\\
4.9466354595117	0.609776034402105\\
4.95643720990256	0.609178639977132\\
4.96628051524823	0.608579301862592\\
4.97612382059389	0.607980553404602\\
4.98596712593956	0.607382394023032\\
4.99581043128522	0.606784823138321\\
5.00569590732134	0.606185283840891\\
5.01558138335745	0.605586336923751\\
5.02546685939356	0.604987981801594\\
5.03535233542967	0.604390217889691\\
5.04528060635073	0.603790460684033\\
5.05520887727179	0.603191298638083\\
5.06513714819285	0.602592731161244\\
5.07506541911391	0.601994757663504\\
5.08503711889176	0.601394765761051\\
5.0950088186696	0.600795371854309\\
5.10498051844744	0.600196575347272\\
5.11495221822528	0.599598375644527\\
5.12496799094167	0.598998132184652\\
5.13498376365806	0.598398489614014\\
5.14499953637444	0.597799447331079\\
5.15501530909083	0.597201004734914\\
5.16507580953502	0.596600492760772\\
5.17513630997921	0.596000584627931\\
5.18519681042339	0.5954012797292\\
5.19525731086758	0.594802577458003\\
5.2053632039849	0.594201779961529\\
5.21546909710223	0.593601589317893\\
5.22557499021955	0.593002004914125\\
5.23568088333688	0.592403026137876\\
5.24583284576096	0.591801925980685\\
5.25598480818505	0.591201435748466\\
5.26613677060914	0.590601554822339\\
5.27628873303322	0.590002282584054\\
5.28648745229033	0.589400862557772\\
5.29668617154743	0.588800055590216\\
5.30688489080453	0.588199861056464\\
5.31708361006163	0.587600278332231\\
5.32732978639303	0.586998521064014\\
5.33757596272443	0.586397380051137\\
5.34782213905582	0.585796854662499\\
5.35806831538722	0.585196944267643\\
5.36836266133408	0.584594832259148\\
5.37865700728095	0.583993339766654\\
5.38895135322782	0.583392466152738\\
5.39924569917468	0.582792210780633\\
5.40958894010065	0.582189726391818\\
5.41993218102661	0.581587864844955\\
5.43027542195257	0.580986625496158\\
5.44061866287853	0.580386007702205\\
5.45101153706255	0.579783133160811\\
5.46140441124658	0.579180884853867\\
5.4717972854306	0.578579262130874\\
5.48219015961463	0.577978264342009\\
5.49263341945276	0.577374981689182\\
5.50307667929089	0.576772328731244\\
5.51351993912902	0.576170304810932\\
5.52396319896715	0.575568909271668\\
5.53445761187213	0.574965200326705\\
5.54495202477712	0.574362124606512\\
5.5554464376821	0.573759681446905\\
5.56594085058709	0.573157870184399\\
5.57648719831941	0.572553716600637\\
5.58703354605173	0.571950199842385\\
5.59757989378405	0.571347319238379\\
5.60812624151637	0.570745074118061\\
5.61872532103302	0.570140457351291\\
5.62932440054967	0.569536481083146\\
5.63992348006633	0.568933144635114\\
5.65052255958298	0.568330447329406\\
5.66117518401188	0.567725348600411\\
5.67182780844078	0.567120894117161\\
5.68248043286967	0.566517083193729\\
5.69313305729857	0.565913915144921\\
5.70384005671797	0.565308315414485\\
5.71454705613736	0.564703363752641\\
5.72525405555675	0.564099059465871\\
5.73596105497615	0.563495401861404\\
5.74672327665992	0.562889281836549\\
5.75748549834369	0.562283813780607\\
5.76824772002746	0.561678996992292\\
5.77900994171123	0.561074830771072\\
5.78982825023888	0.560468170919274\\
5.80064655876654	0.559862167015943\\
5.81146486729419	0.55925681835184\\
5.82228317582184	0.558652124218488\\
5.83315845473553	0.55804490469576\\
5.84403373364921	0.557438345182243\\
5.8549090125629	0.556832444960552\\
5.86578429147658	0.556227203314077\\
5.87671744355088	0.555619403972583\\
5.88765059562518	0.555012268784221\\
5.89858374769948	0.55440579702326\\
5.90951689977378	0.553799987964761\\
5.92050884813521	0.553191588313869\\
5.93150079649664	0.552583857045322\\
5.94249274485807	0.551976793424843\\
5.9534846932195	0.551370396718959\\
5.96453638108293	0.550761375964145\\
5.97558806894636	0.550153027908266\\
5.98663975680979	0.54954535180829\\
5.99769144467322	0.548938346922001\\
6.00880383710542	0.548328683892273\\
6.01991622953761	0.547719697967745\\
6.0310286219698	0.54711138839641\\
6.042141014402	0.546503754427096\\
6.05331509935885	0.545893427544702\\
6.0644891843157	0.545283782265865\\
6.07566326927256	0.544674817829382\\
6.08683735422941	0.544066533474899\\
6.09807414230094	0.543455520796842\\
6.10931093037248	0.542845194315177\\
6.12054771844401	0.542235553259274\\
6.13178450651555	0.541626596859368\\
6.14308503319571	0.541014875982021\\
6.15438555987588	0.540403845991041\\
6.16568608655605	0.539793506106132\\
6.17698661323621	0.539183855547877\\
6.18835193888863	0.538571403639089\\
6.19971726454106	0.537959647406435\\
6.21108259019348	0.537348586059707\\
6.2224479158459	0.536738218809594\\
6.23387912710053	0.53612501256411\\
6.24531033835517	0.535512506887146\\
6.2567415496098	0.534900700978325\\
6.26817276086443	0.534289594038184\\
6.2796709718506	0.533675609643938\\
6.29116918283678	0.533062330816187\\
6.30266739382295	0.532449756744122\\
6.31416560480912	0.531837886617867\\
6.32573195844141	0.531223099721371\\
6.33729831207371	0.530609023498066\\
6.34886466570601	0.529995657126438\\
6.36043101933831	0.529382999785922\\
6.37206668825559	0.528767385477996\\
6.38370235717287	0.528152487061923\\
6.39533802609016	0.527538303705197\\
6.40697369500744	0.526924834576286\\
6.41867988207414	0.526308367402323\\
6.43038606914084	0.525692621454143\\
6.44209225620754	0.52507759588796\\
6.45379844327423	0.524463289860975\\
6.46557638449348	0.523845943705713\\
6.47735432571273	0.523229324229101\\
6.48913226693197	0.522613430575764\\
6.50091020815122	0.521998261891338\\
6.51276117329228	0.521380009987248\\
6.52461213843333	0.520762490337353\\
6.53646310357439	0.520145702074375\\
6.54831406871544	0.519529644332063\\
6.56023936237899	0.518910459246085\\
6.57216465604253	0.5182920121164\\
6.58408994970607	0.517674302063498\\
6.59601524336962	0.517057328208918\\
6.608016207688	0.516437181746487\\
6.62001717200638	0.515817779073214\\
6.63201813632476	0.515199119297015\\
6.64401910064313	0.514581201526879\\
6.65609711612489	0.513960064735511\\
6.66817513160664	0.513339677701251\\
6.68025314708839	0.512720039519088\\
6.69233116257014	0.512101149285104\\
6.70448764976455	0.511478992417469\\
6.71664413695897	0.510857591414508\\
6.72880062415338	0.510236945357912\\
6.7409571113478	0.509617053330491\\
6.75319353351478	0.508993845758994\\
6.76542995568177	0.508371400304217\\
6.77766637784875	0.507749716034172\\
6.78990280001574	0.507128792018012\\
6.80222066442117	0.506504502223184\\
6.8145385288266	0.505880980946637\\
6.82685639323203	0.505258227242304\\
6.83917425763746	0.504636240165283\\
6.85157511856051	0.504010835640055\\
6.86397597948357	0.5033862061896\\
6.87637684040662	0.502762350853356\\
6.88877770132968	0.502139268671951\\
6.90126316148383	0.501512715910237\\
6.91374862163799	0.500886944940323\\
6.92623408179214	0.500261954786717\\
6.93871954194629	0.499637744475142\\
6.95129125486549	0.499010008914668\\
6.96386296778468	0.498383062029465\\
6.97643468070388	0.497756902828658\\
6.98900639362307	0.497131530322614\\
7.00166606764739	0.496502576212206\\
7.0143257416717	0.495874417833408\\
7.02698541569601	0.495247054179487\\
7.03964508972033	0.494620484244984\\
7.05239448925109	0.49399027465307\\
7.06514388878186	0.493360868028566\\
7.07789328831262	0.492732263348391\\
7.09064268784339	0.492104459590765\\
7.10348363709426	0.491472956293455\\
7.11632458634514	0.490842263385904\\
7.12916553559601	0.490212379828162\\
7.14200648484689	0.489583304581616\\
7.15494087138423	0.488950467967145\\
7.16787525792157	0.488318449358876\\
7.18080964445891	0.487687247699455\\
7.19374403099625	0.487056861932892\\
7.20677380774895	0.486422650986287\\
7.21980358450165	0.485789265864253\\
7.23283336125435	0.485156705491462\\
7.24586313800706	0.484524968793984\\
7.25899032889946	0.483889340909941\\
7.27211751979186	0.483254546879327\\
7.28524471068426	0.482620585608245\\
7.29837190157667	0.481987456004235\\
7.31159860422458	0.48135036695043\\
7.32482530687249	0.480714119998338\\
7.3380520095204	0.480078714034872\\
7.35127871216831	0.479444147948419\\
7.36460710304072	0.478805551715431\\
7.37793549391314	0.478167806061494\\
7.39126388478556	0.47753090985368\\
7.40459227565797	0.476894861960567\\
7.41802461355885	0.476254710700316\\
7.43145695145972	0.47561541873562\\
7.4448892893606	0.474976984913021\\
7.45832162726148	0.474339408080609\\
7.47186025959003	0.473697651919176\\
7.48539889191859	0.473056764019926\\
7.49893752424715	0.472416743208148\\
7.51247615657571	0.47177758831072\\
7.52612352432791	0.471134175230837\\
7.53977089208011	0.470491639641537\\
7.55341825983232	0.469849980346094\\
7.56706562758452	0.469209196149415\\
7.58082427117841	0.468564071842874\\
7.5945829147723	0.467919834529542\\
7.60834155836619	0.467276482989877\\
7.62210020196009	0.466634016006013\\
7.63597276754062	0.46598712371363\\
7.64984533312116	0.465341128204643\\
7.66371789870169	0.464696028235847\\
7.67759046428223	0.46405182256576\\
7.69157971106386	0.46340310288001\\
7.70556895784549	0.462755290069763\\
7.71955820462712	0.462108382867258\\
7.73354745140875	0.461462380006504\\
7.7476562591015	0.460811770682825\\
7.76176506679424	0.460162078644086\\
7.77587387448699	0.459513302597022\\
7.78998268217974	0.458865441250189\\
7.80421406003167	0.458212876957016\\
7.81844543788359	0.457561240692232\\
7.83267681573552	0.456910531136066\\
7.84690819358745	0.456260746970623\\
7.86126528820976	0.455606159110384\\
7.87562238283208	0.454952510373814\\
7.8899794774544	0.454299799413572\\
7.90433657207672	0.453648024884251\\
7.91882267896367	0.452991341260793\\
7.93330878585062	0.452335608227565\\
7.94779489273757	0.451680824408532\\
7.96228099962452	0.451026988429649\\
7.97689957365268	0.450368132981433\\
7.99151814768085	0.449710239982277\\
8.00613672170902	0.449053308026248\\
8.02075529573718	0.448397335709464\\
8.03550996276884	0.447736228211823\\
8.0502646298005	0.447076095437018\\
8.06501929683216	0.44641693594794\\
8.07977396386382	0.4457587483096\\
8.09466853533245	0.445095303964768\\
8.10956310680107	0.4444328470563\\
8.12445767826969	0.443771376114551\\
8.13935224973832	0.443110889672058\\
8.15439073744049	0.442445018715657\\
8.16942922514267	0.441780148375896\\
8.18446771284484	0.441116277149134\\
8.19950620054702	0.440453403533984\\
8.2146928332504	0.43978501078796\\
8.22987946595378	0.439117632335069\\
8.24506609865717	0.438451266636111\\
8.26025273136055	0.437785912154223\\
8.27559197362809	0.437114896512584\\
8.29093121589562	0.436444909369072\\
8.30627045816316	0.43577594914726\\
8.32160970043069	0.435108014273137\\
8.33710627308224	0.434434268150309\\
8.35260284573378	0.433761565294492\\
8.36809941838533	0.433089904090233\\
8.38359599103687	0.432419282924581\\
8.39925489573158	0.431742691562868\\
8.41491380042629	0.431067158840045\\
8.430572705121	0.430392683099701\\
8.44623160981571	0.429719262688011\\
8.46205815432056	0.429039703480009\\
8.4778846988254	0.428361218928779\\
8.49371124333024	0.427683807334861\\
8.50953778783509	0.427007467001478\\
8.52553761739547	0.426324808598711\\
8.54153744695585	0.425643241564432\\
8.55753727651623	0.42496276415387\\
8.57353710607661	0.424283374625041\\
8.58971623619016	0.423597476044473\\
8.60589536630372	0.422912686290906\\
8.62207449641727	0.422229003571808\\
8.63825362653082	0.421546426097543\\
8.65461848330403	0.420857135569779\\
8.67098334007723	0.420168972133602\\
8.68734819685043	0.419481933946057\\
8.70371305362364	0.418796019167195\\
8.72018188980809	0.418106878486767\\
8.73665072599255	0.417418871806793\\
8.75311956217701	0.416731997261249\\
8.76958839836146	0.416046252987176\\
8.78614348346302	0.415358054690985\\
8.80269856856458	0.414670994770372\\
8.81925365366614	0.41398507134231\\
8.83580873876771	0.413300282526883\\
8.85245111606455	0.412613024641889\\
8.86909349336139	0.411926909566177\\
8.88573587065824	0.411241935399427\\
8.90237824795508	0.410558100244476\\
8.91910883699713	0.409871786640802\\
8.93583942603918	0.409186620320207\\
8.95257001508123	0.408502599364828\\
8.96930060412328	0.407819721860005\\
8.98612034063892	0.407134356378168\\
9.00294007715455	0.406450142694091\\
9.01975981367019	0.405767078872114\\
9.03657955018582	0.405085162979827\\
9.05348938408815	0.404400749526576\\
9.07039921799048	0.40371749242714\\
9.08730905189281	0.403035389727801\\
9.10421888579513	0.402354439478138\\
9.12121978296026	0.401670981967349\\
9.13822068012539	0.400988685408501\\
9.15522157729052	0.400307547829554\\
9.17222247445565	0.399627567261813\\
9.18931541587169	0.398945069663014\\
9.20640835728774	0.398263737656904\\
9.22350129870378	0.397583569252851\\
9.24059424011982	0.396904562463615\\
9.25778022278737	0.396223028779549\\
9.27496620545493	0.395542665372183\\
9.29215218812249	0.394863470232013\\
9.30933817079004	0.394185441352981\\
9.32661820779546	0.393504875631015\\
9.34389824480088	0.392825484913633\\
9.3611782818063	0.39214726717218\\
9.37845831881172	0.391470220381503\\
9.39583344002363	0.390790626700606\\
9.41320856123555	0.390112212796729\\
9.43058368244747	0.389434976621779\\
9.44795880365938	0.388758916131216\\
9.46543005570071	0.388080298618013\\
9.48290130774205	0.38740286569945\\
9.50037255978338	0.386726615307702\\
9.51784381182471	0.386051545378553\\
9.53541225878365	0.385373908194654\\
9.55298070574258	0.384697460468892\\
9.57054915270152	0.384022200113413\\
9.58811759966045	0.383348125044023\\
9.60578432346384	0.382671472387228\\
9.62345104726722	0.381996014098649\\
9.64111777107061	0.381321748070098\\
9.658784494874	0.380648672197103\\
9.67655059520445	0.379973008320344\\
9.6943166955349	0.379298543769121\\
9.71208279586536	0.378625276414597\\
9.72984889619581	0.377953204131711\\
9.74771549159443	0.377278533316924\\
9.76558208699305	0.376605066832995\\
9.78344868239167	0.375932802530123\\
9.80131527779029	0.375261738262337\\
9.81928350543996	0.374588064845323\\
9.83725173308963	0.373915600813187\\
9.8552199607393	0.373244343994835\\
9.87318818838897	0.372574292223067\\
9.89125920482513	0.371901620583293\\
9.90933022126129	0.371230163431858\\
9.92740123769745	0.370559918576047\\
9.94547225413361	0.369890883827097\\
9.96364723590061	0.36921921838042\\
9.9818222176676	0.368548772575779\\
9.99999719943459	0.367879544198497\\
10.0181721812016	0.367211531037911\\
10.0364523247657	0.366540876257377\\
10.0547324683298	0.365871446323546\\
10.0730126118939	0.365203238999431\\
10.0912927554579	0.364536252052129\\
10.1096792783833	0.363866612442436\\
10.1280658013086	0.363198202935929\\
10.146452324234	0.362531021272962\\
10.1648388471593	0.361865065198031\\
10.1833329877765	0.361196445326081\\
10.2018271283938	0.360529060866373\\
10.220321269011	0.359862909536236\\
10.2388154096282	0.359197989057212\\
10.257418428294	0.358530393524836\\
10.2760214469598	0.357864038767204\\
10.2946244656257	0.357198922478255\\
10.3132274842915	0.356535042356204\\
10.3319406634311	0.355868475817958\\
10.3506538425707	0.355203155471236\\
10.3693670217104	0.354539078986207\\
10.38808020085	0.353876244037388\\
10.4069048454226	0.353210711203208\\
10.4257294899953	0.352546430032452\\
10.4445541345679	0.351883398171133\\
10.4633787791406	0.351221613269684\\
10.4823162175897	0.350557118891014\\
10.5012536560388	0.349893881703704\\
10.5201910944879	0.349231899329216\\
10.539128532937	0.348571169393504\\
10.5581801172831	0.347907718280489\\
10.5772317016292	0.347245529943677\\
10.5962832859753	0.346584601979576\\
10.6153348703214	0.345924931989264\\
10.6345019773376	0.345262528990417\\
10.6536690843538	0.344601394410498\\
10.67283619137	0.343941525820649\\
10.6920032983861	0.34328292079666\\
10.7112873293394	0.342621570829238\\
10.7305713602926	0.34196149498223\\
10.7498553912458	0.341302690800996\\
10.769139422199	0.340645155835615\\
10.7885418041401	0.33998486386262\\
10.8079441860811	0.339325851771286\\
10.8273465680222	0.338668117080752\\
10.8467489499633	0.338011657314962\\
10.8662711360792	0.337352428355508\\
10.8857933221952	0.336694485099696\\
10.9053155083112	0.336037825040009\\
10.9248376944271	0.335382445673813\\
10.9444811647161	0.334724284802921\\
10.9641246350051	0.334067415519396\\
10.9837681052941	0.333411835288612\\
11.0034115755831	0.332757541580908\\
11.02317783773	0.33210045392337\\
11.0429440998769	0.331444663799722\\
11.0627103620238	0.330790168647769\\
11.0824766241707	0.330136965910363\\
11.102367213876	0.32948095665365\\
11.1222578035813	0.328826250941184\\
11.1421483932866	0.328172846182725\\
11.1620389829919	0.327520739793173\\
11.182055464688	0.326865814187848\\
11.202071946384	0.326212198202021\\
11.22208842808	0.325559889216924\\
11.2421049097761	0.32490888461902\\
11.2622488777205	0.324255047967681\\
11.282392845665	0.323602527077113\\
11.3025368136094	0.322951319299529\\
11.3226807815539	0.322301421992462\\
11.3429428732039	0.321649033058746\\
11.363204964854	0.320997964663165\\
11.3834670565041	0.320348214132751\\
11.4037291481542	0.319699778799939\\
11.4240073959591	0.319052140536561\\
11.444285643764	0.318405814239426\\
11.464563891569	0.317760797250804\\
11.4848421393739	0.317117086918338\\
11.5051410232308	0.316474027516911\\
11.5254399070878	0.315832272130339\\
11.5457387909448	0.315191818114309\\
11.5660376748017	0.314552662829868\\
11.5863613245615	0.313914026206012\\
11.6066849743212	0.313276686206804\\
11.627008624081	0.312640640199713\\
11.6473322738407	0.312005885557548\\
11.6676846007299	0.311371526733005\\
11.6880369276191	0.310738457663351\\
11.7083892545083	0.310106675726313\\
11.7287415813975	0.309476178304943\\
11.7491263024508	0.308845962305306\\
11.7695110235041	0.308217029674897\\
11.7898957445574	0.307589377800278\\
11.8102804656107	0.306963004073325\\
11.8307011258767	0.306336804938883\\
11.8511217861426	0.305711883239665\\
11.8715424464086	0.305088236369736\\
11.8919631066745	0.304465861728468\\
11.9124230970895	0.303843561696921\\
11.9328830875046	0.303222533589016\\
11.9533430779196	0.302602774805066\\
11.9738030683347	0.301984282750694\\
11.9943056417768	0.301365771527399\\
12.014808215219	0.300748527112192\\
12.0353107886612	0.300132546910458\\
12.0558133621033	0.299517828332893\\
12.0763616495731	0.298903002386353\\
12.0969099370428	0.298289438504728\\
12.1174582245125	0.297677134097363\\
12.1380065119823	0.297066086578913\\
12.1586035338403	0.296454848611242\\
12.1792005556983	0.295844868316083\\
12.1997975775563	0.295236143105679\\
12.2203945994143	0.294628670397594\\
12.2410432774846	0.294020928810018\\
12.2616919555548	0.29341444083376\\
12.2823406336251	0.292809203882956\\
12.3029893116954	0.292205215377072\\
12.3236924799435	0.291600883798156\\
12.3443956481917	0.290997802082816\\
12.3650988164398	0.290395967646125\\
12.385801984688	0.289795377908493\\
12.406562398648	0.289194374777133\\
12.4273228126081	0.288594618058896\\
12.4480832265682	0.287996105168872\\
12.4688436405282	0.2873988335275\\
12.4896639845707	0.286801081754166\\
12.5104843286133	0.286204573225893\\
12.5313046726558	0.285609305356907\\
12.5521250166983	0.285015275566799\\
12.5730079121072	0.284420702184605\\
12.5938908075161	0.283827369148228\\
12.614773702925	0.28323527387018\\
12.6356565983338	0.282644413768361\\
12.6566046110283	0.282052949607826\\
12.6775526237227	0.281462723150334\\
12.6985006364172	0.280873731805868\\
12.7194486491116	0.280285972989822\\
12.7404642942679	0.279697552453187\\
12.7614799394242	0.279110367221773\\
12.7824955845805	0.278524414702244\\
12.8035112297368	0.277939692306698\\
12.8245969793381	0.277354253069243\\
12.8456827289394	0.276770046973756\\
12.8667684785407	0.276187071422813\\
12.887854228142	0.275605323824454\\
12.9090125143938	0.275022806661617\\
12.9301708006455	0.274441520702292\\
12.9513290868973	0.273861463344231\\
12.972487373149	0.273282631990672\\
12.9937205946389	0.2727029805364\\
13.0149538161287	0.272124558563139\\
13.0361870376186	0.271547363463084\\
13.0574202591085	0.270971392633954\\
13.0787307850171	0.270394553201521\\
13.1000413109258	0.269818941735352\\
13.1213518368345	0.269244555621385\\
13.1426623627431	0.268671392251111\\
13.1640525369264	0.268097313664216\\
13.1854427111097	0.267524461729028\\
13.206832885293	0.266952833824533\\
13.2282230594763	0.266382427335308\\
13.2496952043377	0.265811060771291\\
13.271167349199	0.265240919737622\\
13.2926394940603	0.264672001605656\\
13.3141116389217	0.264104303752381\\
13.3356680588802	0.263535602602189\\
13.3572244788388	0.262968126047702\\
13.3787808987974	0.262401871451986\\
13.400337318756	0.261836836183773\\
13.4219803036483	0.261270755919098\\
13.4436232885406	0.260705899496219\\
13.4652662734329	0.260142264269254\\
13.4869092583252	0.25957984759803\\
13.5086410871513	0.259016345634744\\
13.5303729159775	0.258454066934602\\
13.5521047448036	0.257893008842129\\
13.5738365736297	0.257333168707602\\
13.5956595174591	0.256772204296359\\
13.6174824612885	0.256212462739792\\
13.639305405118	0.255653941372194\\
13.6611283489474	0.255096637533659\\
13.6830446722324	0.254538171695381\\
13.7049609955174	0.253980928468634\\
13.7268773188025	0.253424905176848\\
13.7487936420875	0.252870099149305\\
13.770805608124	0.252314094508157\\
13.7928175741606	0.251759312396521\\
13.8148295401972	0.251205750126341\\
13.8368415062338	0.250653405015461\\
13.858951377984	0.250099825755433\\
13.8810612497342	0.249547469099961\\
13.9031711214845	0.248996332348882\\
13.9252809932347	0.248446412807983\\
13.9474910360997	0.247895224582458\\
13.9697010789646	0.247345259189875\\
13.9919111218296	0.246796513917344\\
14.0141211646945	0.24624898605798\\
14.0364336494706	0.245700155898653\\
14.0587461342466	0.245152548950624\\
14.0810586190226	0.244606162487659\\
14.1033711037986	0.244060993789591\\
14.1257883079129	0.243514490061325\\
14.1482055120272	0.242969210069464\\
14.1706227161414	0.242425151073818\\
14.1930399202557	0.241882310340321\\
14.2155641320733	0.241338102623597\\
14.2380883438909	0.240795119312397\\
14.2606125557085	0.24025335765196\\
14.283136767526	0.239712814893712\\
14.3057702854045	0.239170873998356\\
14.3284038032829	0.238630158318815\\
14.3510373211614	0.238090665085144\\
14.3736708390398	0.237552391533649\\
14.3964159769655	0.237012689355351\\
14.4191611148912	0.236474213342099\\
14.4419062528169	0.235936960708146\\
14.4646513907425	0.235400928674064\\
14.4875104777278	0.234863438204689\\
14.510369564713	0.234327174986218\\
14.5332286516983	0.233792136216485\\
14.5560877386835	0.233258319099711\\
14.5790631229979	0.232723014324253\\
14.6020385073122	0.23218893802031\\
14.6250138916265	0.231656087368674\\
14.6479892759408	0.231124459556597\\
14.6710833264109	0.230591315423044\\
14.6941773768809	0.230059401114606\\
14.7172714273509	0.229528713794409\\
14.740365477821	0.228999250632112\\
14.7635805844029	0.228468243039432\\
14.7867956909849	0.227938466756725\\
14.8100107975669	0.227409918928823\\
14.8332259041488	0.226882596707164\\
14.8565644820063	0.226353702414492\\
14.8799030598637	0.225826041046579\\
14.9032416377212	0.225299609729322\\
14.9265802155787	0.224774405595307\\
14.9500447068131	0.224247602188731\\
14.9735091980476	0.223722033450428\\
14.996973689282	0.223197696486721\\
15.0204381805165	0.222674588410705\\
15.0440310551214	0.222149854289215\\
15.0676239297264	0.221626356707099\\
15.0912168043314	0.221104092750459\\
15.1148096789363	0.220583059512249\\
15.1385334385982	0.220060373811961\\
15.16225719826	0.219538926648938\\
15.1859809579218	0.219018715088401\\
15.2097047175837	0.218499736202513\\
15.2335618956234	0.21797907880945\\
15.2574190736632	0.217459662077467\\
15.281276251703	0.21694148305024\\
15.3051334297427	0.216424538778477\\
15.3291265950906	0.215905890253855\\
15.3531197604384	0.215388484639541\\
15.3771129257863	0.214872318956988\\
15.4011060911341	0.214357390234776\\
15.4252378496272	0.213840731802096\\
15.4493696081203	0.213325318653909\\
15.4735013666134	0.212811147788762\\
15.4976331251065	0.212298216212424\\
15.5219061221034	0.211783529717858\\
15.5461791191004	0.21127009100667\\
15.5704521160973	0.210757897053805\\
15.5947251130942	0.210246944841529\\
15.619142035055	0.209734212738787\\
15.6435589570157	0.209222731042833\\
15.6679758789765	0.208712496704299\\
15.6923928009372	0.20820350668124\\
15.7169563780926	0.20769271199527\\
15.741519955248	0.20718317046403\\
15.7660835324033	0.206674879013116\\
15.7906471095587	0.206167834575654\\
15.8153601179615	0.205658960881813\\
15.8400731263644	0.205151343215298\\
15.8647861347672	0.204644978475937\\
15.88949914317	0.204139863571196\\
15.9143644081624	0.203632894947545\\
15.9392296731548	0.203127185348852\\
15.9640949381471	0.202622731648423\\
15.9889602031395	0.202119530727316\\
16.0139806003288	0.201614451760779\\
16.039000997518	0.201110634942235\\
16.0640213947072	0.200608077117704\\
16.0890417918965	0.200106775141075\\
16.1142202516767	0.199603570863388\\
16.1393987114569	0.199101631982851\\
16.1645771712372	0.198600955317412\\
16.1897556310174	0.19810153769301\\
16.2150951374624	0.197600193632709\\
16.2404346439074	0.197100118345318\\
16.2657741503524	0.196601308619917\\
16.2911136567974	0.196103761253696\\
16.3166172552097	0.195604263316767\\
16.342120853622	0.195106037656232\\
16.3676244520343	0.194609081031484\\
16.3931280504465	0.194113390210151\\
16.4187988465862	0.193615724730099\\
16.4444696427259	0.193119335158563\\
16.4701404388656	0.192624218224403\\
16.4958112350053	0.192130370664852\\
16.5216524004798	0.191634524333033\\
16.5474935659542	0.191139957671792\\
16.5733347314287	0.190646667378599\\
16.5991758969032	0.19015465015943\\
16.6251906694304	0.189660610058212\\
16.6512054419577	0.189167853520761\\
16.677220214485	0.188676377212275\\
16.7032349870123	0.188186177806599\\
16.7294266758655	0.187693931344755\\
16.7556183647187	0.187202972472571\\
16.781810053572	0.186713297822065\\
16.8080017424252	0.186224904034048\\
16.8343737319713	0.185734438923108\\
16.8607457215174	0.185245265562182\\
16.8871177110636	0.184757380549178\\
16.9134897006097	0.184270780490945\\
16.9400454511807	0.183782084773798\\
16.9666012017517	0.183294685103156\\
16.9931569523226	0.182808578041861\\
17.0197127028936	0.182323760161846\\
17.0464557582069	0.181836822122315\\
17.0731988135202	0.181351184564143\\
17.0999418688336	0.180866844014117\\
17.1266849241469	0.180383797008276\\
17.1536189111414	0.17989860522479\\
17.1805528981359	0.1794147184979\\
17.2074868851304	0.17893213331732\\
17.2344208721249	0.178450846182187\\
17.2615495082445	0.177967389447658\\
17.2886781443641	0.177485242487873\\
17.3158067804836	0.177004401754431\\
17.3429354166032	0.176524863708523\\
17.3702625121198	0.176043131044878\\
17.3975896076364	0.175562713019588\\
17.424916703153	0.175083606045056\\
17.4522437986696	0.174605806543456\\
17.4797732620512	0.174125787164672\\
17.5073027254328	0.173647087436167\\
17.5348321888143	0.173169703730034\\
17.5623616521959	0.17269363242832\\
17.5900974943972	0.172215315726965\\
17.6178333365984	0.171738323839158\\
17.6455691787997	0.171262653095534\\
17.6733050210009	0.170788299836868\\
17.7012513604556	0.17031167536233\\
17.7291976999103	0.16983638101923\\
17.757144039365	0.169362413095552\\
17.7850903788197	0.168889767889619\\
17.8132514459017	0.168414825339651\\
17.8414125129837	0.167941218397077\\
17.8695735800657	0.167468943306002\\
17.8977346471477	0.16699799632107\\
17.9261147917276	0.166524725485453\\
17.9544949363076	0.166052795895155\\
17.9828750808876	0.165582203749126\\
18.0112552254676	0.165112945257065\\
18.0398589202161	0.164641336038409\\
18.0684626149646	0.1641710738689\\
18.097066309713	0.163702154901013\\
18.1256700044615	0.163234575298189\\
18.1545018545664	0.162764617630508\\
18.1833337046712	0.162296012986264\\
18.212165554776	0.161828757470089\\
18.2409974048809	0.161362847197804\\
18.2700621491713	0.160894531112675\\
18.2991268934616	0.160427574200095\\
18.328191637752	0.159961972515429\\
18.3572563820424	0.159497722125466\\
18.3865589068877	0.159031037614587\\
18.4158614317331	0.158565718605526\\
18.4451639565785	0.158101761102905\\
18.4744664814239	0.15763916112301\\
18.5040118216459	0.157174098218638\\
18.5335571618679	0.156710407330586\\
18.56310250209	0.1562480844112\\
18.592647842312	0.155787125424738\\
18.622441195706	0.155323674067863\\
18.6522345491	0.154861601433114\\
18.682027902494	0.154400903418961\\
18.711821255888	0.153941575936049\\
18.74186798467	0.15347972605938\\
18.771914713452	0.153019261807769\\
18.801961442234	0.152560179024148\\
18.8320081710161	0.152102473563894\\
18.8623138179204	0.151642214952203\\
18.8926194648247	0.151183349072538\\
18.922925111729	0.150725871710558\\
18.9532307586333	0.150269778664647\\
18.9838010512769	0.149811101006451\\
19.0143713439205	0.149353823398168\\
19.0449416365641	0.148897941566376\\
19.0755119292076	0.148443451250667\\
19.1063527942164	0.147986344047923\\
19.1371936592252	0.147530644431634\\
19.1680345242339	0.147076348067404\\
19.1988753892427	0.146623450634156\\
19.2299929618915	0.146167903190575\\
19.2611105345403	0.145713771096781\\
19.2922281071891	0.145261049955432\\
19.3233456798379	0.144809735382818\\
19.3547463174076	0.144355736746461\\
19.3861469549774	0.143903161458869\\
19.4175475925471	0.143452005057685\\
19.4489482301169	0.14300226309451\\
19.4806385226382	0.142549802049457\\
19.5123288151595	0.142098772597087\\
19.5440191076808	0.141649170207861\\
19.5757094002021	0.141200990366541\\
19.6076961882754	0.140750055333737\\
19.6396829763487	0.140300560392814\\
19.671669764422	0.139852500944782\\
19.7036565524953	0.139405872405304\\
19.7359469398787	0.138956451432329\\
19.768237327262	0.138508479316589\\
19.8005277146453	0.138061951387257\\
19.8328181020287	0.137616862988528\\
19.8654194755566	0.137168943649144\\
19.8980208490846	0.13672248221053\\
19.9306222226126	0.136277473927513\\
19.9632235961405	0.135833914070328\\
19.9961436388709	0.135387483473685\\
20.0290636816012	0.134942520112046\\
20.0619837243316	0.134499019163256\\
20.0949037670619	0.134056975820967\\
20.1281504830269	0.133612020476209\\
20.1613971989918	0.133168542005394\\
20.1946439149567	0.132726535506603\\
20.2278906309217	0.132285996094147\\
20.2614723645969	0.131842501867214\\
20.2950540982721	0.131400494472832\\
20.3286358319473	0.130959968926383\\
20.3622175656225	0.130520920259919\\
20.3961430262838	0.13007887228339\\
20.4300684869452	0.129638321433977\\
20.4639939476066	0.129199262641266\\
20.497919408268	0.128761690851972\\
20.5321976950164	0.128321073447784\\
20.5664759817649	0.127881963818857\\
20.6007542685134	0.127444356805701\\
20.6350325552618	0.127008247266432\\
20.6696731863085	0.126569043836491\\
20.7043138173553	0.126131359202787\\
20.738954448402	0.125695188113274\\
20.7735950794487	0.125260525334019\\
20.8086080231198	0.124822718255246\\
20.8436209667909	0.124386441387469\\
20.878633910462	0.123951689382397\\
20.9136468541332	0.123518456910383\\
20.949042565344	0.123082027389712\\
20.9844382765548	0.122647139911633\\
21.0198339877656	0.122213789027686\\
21.0552296989764	0.121781969308613\\
21.0910191546516	0.121346897283311\\
21.1268086103268	0.120913379573997\\
21.162598066002	0.120481410627866\\
21.1983875216773	0.120050984911897\\
21.234582265905	0.119617248864376\\
21.2707770101326	0.119185079875665\\
21.3069717543603	0.118754472284154\\
21.343166498588	0.118325420448632\\
21.3797786897193	0.117892997238826\\
21.4163908808506	0.117462154330415\\
21.453003071982	0.117032885948216\\
21.4896152631133	0.116605186338096\\
21.5266577262528	0.116174051015129\\
21.5637001893923	0.115744509769294\\
21.6007426525319	0.115316556706733\\
21.6377851156714	0.114890185955319\\
21.6752714017348	0.114460311540828\\
21.7127576877983	0.114032045549241\\
21.7502439738617	0.113605381962539\\
21.7877302599251	0.11318031478516\\
21.8256747161126	0.11275167198081\\
21.8636191723001	0.112324652556356\\
21.9015636284876	0.111899250363725\\
21.9395080846751	0.111475459278057\\
21.9779259262067	0.111048016222861\\
22.0163437677384	0.110622212161219\\
22.0547616092701	0.110198040808635\\
22.0931794508017	0.109775495904642\\
22.1320868496398	0.109349217810814\\
22.1709942484778	0.108924595031822\\
22.2099016473158	0.108501621139865\\
22.2488090461539	0.108080289732033\\
22.2882232273236	0.107655138519745\\
22.3276374084934	0.107231659708105\\
22.3670515896632	0.10680984671855\\
22.4064657708329	0.106389692998315\\
22.4464051256478	0.105965626838754\\
22.4863444804626	0.105543250994257\\
22.5262838352775	0.105122558727376\\
22.5662231900924	0.104703543327437\\
22.6067074061995	0.104280516113964\\
22.6471916223066	0.103859198030961\\
22.6876758384138	0.103439582173237\\
22.7281600545209	0.103021661663413\\
22.7692102626757	0.102599622430435\\
22.8102604708306	0.102179312126148\\
22.8513106789854	0.101760723667922\\
22.8923608871403	0.101343850002046\\
22.9339998357608	0.100922742198486\\
22.9756387843813	0.100503384197943\\
23.0172777330018	0.100085768729685\\
23.0589166816223	0.0996698885530931\\
23.1011689388352	0.0992496492046123\\
23.143421196048	0.0988311817162646\\
23.1856734532609	0.0984144786174562\\
23.2279257104737	0.0979995324689895\\
23.2707449436263	0.097580803109335\\
23.3135641767789	0.0971638628834033\\
23.3563834099314	0.0967487041467667\\
23.399202643084	0.0963353192875533\\
23.4425351018712	0.0959187778024229\\
23.4858675606585	0.0955040373888167\\
23.5292000194457	0.0950910902592668\\
23.5725324782329	0.0946799286598654\\
23.6163923812599	0.0942655727556639\\
23.6602522842869	0.0938530302327043\\
23.704112187314	0.0934422931550695\\
23.747972090341	0.093033353621456\\
23.7923730104802	0.0926211926644557\\
23.8367739306194	0.0922108576831823\\
23.8811748507586	0.0918023405882533\\
23.9255757708978	0.0913956333260018\\
23.9705318120247	0.090985676932245\\
24.0154878531517	0.0905775594035353\\
24.0604438942787	0.0901712724917694\\
24.1053999354057	0.0897668079857136\\
24.1509257391711	0.0893590662200637\\
24.1964515429365	0.0889531765124487\\
24.2419773467019	0.088549130450548\\
24.2875031504673	0.0881469196601185\\
24.3336139249026	0.0877414030399698\\
24.3797246993379	0.0873377519838151\\
24.4258354737732	0.086935957909363\\
24.4719462482085	0.0865360122736645\\
24.518657797687	0.0861327317779306\\
24.5653693471654	0.0857313306748791\\
24.6120808966439	0.0853318002062218\\
24.6587924461223	0.0849341316543399\\
24.7061212033434	0.0845330987327796\\
24.7534499605645	0.0841339593655625\\
24.8007787177855	0.0837367046120884\\
24.8481074750066	0.0833413255738182\\
24.8960705351928	0.0829425521565451\\
24.944033595379	0.0825456867990088\\
24.9919966555653	0.0821507203716743\\
25.0399597157515	0.0817576437885288\\
25.0885748730688	0.0813611422971422\\
25.1371900303861	0.0809665637259192\\
25.1858051877034	0.0805738987494408\\
25.2344203450208	0.0801831380873448\\
25.2837061315614	0.0797889214464446\\
25.332991918102	0.0793966429530456\\
25.3822777046427	0.0790062930785477\\
25.4315634911833	0.0786178623410201\\
25.4815392186928	0.0782259439904308\\
25.5315149462023	0.0778359793940823\\
25.5814906737117	0.0774479588125251\\
25.6314664012212	0.0770618725546757\\
25.6821522057685	0.0766722664626975\\
25.7328380103157	0.0762846301242425\\
25.783523814863	0.0758989535809493\\
25.8342096194102	0.0755152269246073\\
25.8856265092156	0.0751279476026346\\
25.9370433990211	0.0747426544398382\\
25.9884602888265	0.0743593372504602\\
26.0398771786319	0.073977985900774\\
26.0920470854784	0.0735930484182132\\
26.144216992325	0.0732101139214184\\
26.1963868991715	0.0728291719883009\\
26.2485568060181	0.0724502122507849\\
26.3015026403881	0.07206763225164\\
26.3544484747581	0.0716870725011328\\
26.4073943091281	0.0713085223314283\\
26.4603401434981	0.0709319711307941\\
26.5140858539001	0.0705517648510394\\
26.5678315643021	0.0701735965352561\\
26.6215772747041	0.0697974552599322\\
26.6753229851061	0.0694233301598652\\
26.7298936218998	0.069045514446213\\
26.7844642586936	0.068669754881396\\
26.8390348954873	0.0682960402757438\\
26.893605532281	0.0679243595002267\\
26.9490273164448	0.0675489518301775\\
27.0044491006085	0.0671756189813743\\
27.0598708847722	0.0668043494869158\\
27.1152926689359	0.0664351319430061\\
27.1715930662283	0.0660621504463515\\
27.2278934635207	0.0656912629499304\\
27.2841938608131	0.0653224576979045\\
27.3404942581055	0.064955723000149\\
27.3972810276665	0.0645879057788727\\
27.4540677972275	0.0642221713531832\\
27.5108545667884	0.0638585079294228\\
27.5676413363494	0.0634969037804266\\
27.6246835621125	0.0631357343830617\\
27.6817257878755	0.0627766193113579\\
27.7387680136386	0.0624195468806901\\
27.7958102394016	0.0620645054726051\\
27.8531117250382	0.0617098836240515\\
27.9104132106748	0.0613572880003119\\
27.9677146963114	0.0610067070243659\\
28.025016181948	0.0606581291850512\\
28.0825792753208	0.0603099652607105\\
28.1401423686935	0.0599637997182149\\
28.1977054620663	0.0596196210876523\\
28.2552685554391	0.0592774179646552\\
28.3130956457549	0.0589356231070616\\
28.3709227360708	0.0585957990456821\\
28.4287498263866	0.0582579344172108\\
28.4865769167025	0.0579220179235726\\
28.5446704260258	0.0575865040971926\\
28.602763935349	0.0572529337375335\\
28.6608574446723	0.0569212955873932\\
28.7189509539956	0.0565915784544872\\
28.7773133374	0.0562622584444395\\
28.8356757208045	0.055934854826527\\
28.8940381042089	0.0556093564491499\\
28.9524004876134	0.0552857522253123\\
29.0110342352743	0.0549625396251478\\
29.0696679829352	0.0546412165963079\\
29.1283017305961	0.0543217720922923\\
29.186935478257	0.0540041951308911\\
29.2458431140456	0.0536870043503704\\
29.3047507498342	0.0533716765728713\\
29.3636583856228	0.053058200856491\\
29.4225660214114	0.0527465663233039\\
29.4817501049045	0.0524353125767768\\
29.5409341883977	0.0521258955162984\\
29.6001182718908	0.0518183043040613\\
29.659302355384	0.0515125281659219\\
29.7187654815208	0.0512071274738736\\
29.7782286076576	0.0509035374009904\\
29.8376917337944	0.0506017472130587\\
29.8971548599312	0.050301746239215\\
29.9568996607722	0.0500021154186347\\
30.0166444616132	0.0497042693992526\\
30.0763892624543	0.0494081975499468\\
30.1361340632953	0.0491138893026317\\
30.1961632086509	0.048819945963771\\
30.2561923540065	0.0485277618558791\\
30.316221499362	0.0482373264504232\\
30.3762506447176	0.0479486292815935\\
30.436566841334	0.0476602918316142\\
30.4968830379503	0.0473736882888429\\
30.5571992345667	0.0470888082268345\\
30.617515431183	0.0468056412815533\\
30.6781214246657	0.0465228289144672\\
30.7387274181483	0.0462417253761016\\
30.799333411631	0.0459623203415973\\
30.8599394051136	0.045684603548191\\
30.9208379817972	0.0454072362359605\\
30.9817365584807	0.0451315529181025\\
31.0426351351642	0.0448575433708411\\
31.1035337118477	0.044585197432183\\
31.1647276970968	0.0443131959324507\\
31.2259216823458	0.0440428538356091\\
31.2871156675948	0.0437741610184641\\
31.3483096528438	0.0435071074192905\\
31.4098019130839	0.0432403932654385\\
31.4712941733241	0.0429753141646614\\
31.5327864335642	0.0427118600938449\\
31.5942786938043	0.0424500210910303\\
31.656072138367	0.042188516583934\\
31.7178655829298	0.0419286230206164\\
31.7796590274925	0.0416703304775406\\
31.8414524720552	0.0414136290920125\\
31.903550051626	0.0411572573065\\
31.9656476311968	0.0409024725947175\\
32.0277452107676	0.0406492651322043\\
32.0898427903383	0.0403976251550288\\
32.1522474993513	0.0401463099291097\\
32.2146522083642	0.0398965581449315\\
32.2770569173772	0.0396483600766079\\
32.3394616263901	0.0394017060584682\\
32.4021765035977	0.0391553719907521\\
32.4648913808053	0.0389105779696441\\
32.527606258013	0.0386673143673306\\
32.5903211352206	0.0384255716159002\\
32.6533492654047	0.0381841440585643\\
32.7163773955889	0.0379442333883968\\
32.7794055257731	0.0377058300751544\\
32.8424336559572	0.037468924648183\\
32.9057781690946	0.0372323297099736\\
32.969122682232	0.0369972287339373\\
33.0324671953693	0.0367636122868995\\
33.0958117085067	0.0365314709949619\\
33.1594757818323	0.0362996355327643\\
33.2231398551579	0.0360692713409971\\
33.2868039284835	0.0358403690830529\\
33.3504680018091	0.0356129194812871\\
33.4144548595075	0.0353857711008235\\
33.478441717206	0.0351600715310509\\
33.5424285749044	0.0349358115314278\\
33.6064154326028	0.034712981920062\\
33.6707283499096	0.0344904489607821\\
33.7350412672163	0.0342693425833507\\
33.7993541845231	0.0340496536427897\\
33.8636671018298	0.0338313730524574\\
33.9283094023156	0.0336133845971003\\
33.9929517028013	0.0333968007243697\\
34.057594003287	0.0331816123843498\\
34.1222363037727	0.0329678105851478\\
34.1872113622324	0.0327542964490607\\
34.2521864206922	0.032542165124809\\
34.3171614791519	0.0323314076570373\\
34.3821365376116	0.0321220151481003\\
34.447447781095	0.0319129058755605\\
34.5127590245783	0.0317051578713038\\
34.5780702680616	0.0314987622740339\\
34.643381511545	0.0312937102798511\\
34.7090324201268	0.0310889371412626\\
34.7746833287086	0.030885503953439\\
34.8403342372905	0.0306834019486414\\
34.9059851458723	0.0304826224162141\\
34.9719792533529	0.0302821174054506\\
35.0379733608335	0.0300829312527472\\
35.1039674683141	0.0298850552834209\\
35.1699615757948	0.0296884808795587\\
35.2363024710846	0.0294921767091833\\
35.3026433663745	0.0292971705277679\\
35.3689842616644	0.0291034537531836\\
35.4353251569542	0.0289110178597592\\
35.5020164861975	0.0287188479543381\\
35.5687078154407	0.0285279553912099\\
35.6353991446839	0.0283383316802987\\
35.7020904739271	0.0281499683876728\\
35.76913594038	0.0279618668838935\\
35.8361814068328	0.02777502229695\\
35.9032268732857	0.0275894262283178\\
35.9702723397385	0.0274050703353036\\
36.0376757049608	0.0272209720780044\\
36.1050790701831	0.0270381105320717\\
36.1724824354054	0.0268564773900311\\
36.2398858006277	0.0266760643999265\\
36.3076508877251	0.0264959049327787\\
36.3754159748225	0.0263169621903669\\
36.4431810619199	0.0261392279557653\\
36.5109461490173	0.0259626940672529\\
36.5790768402747	0.025786409638851\\
36.6472075315322	0.0256113221661115\\
36.7153382227897	0.0254374235221559\\
36.7834689140472	0.0252647056349982\\
36.8519691566203	0.0250922331832659\\
36.9204693991935	0.024920938134555\\
36.9889696417667	0.0247508124515328\\
37.0574698843398	0.024581848151446\\
37.1263436888854	0.0244131253071965\\
37.1952174934311	0.0242455605285366\\
37.2640912979767	0.0240791458671779\\
37.3329651025223	0.0239138734290988\\
37.4022165438372	0.0237488385137789\\
37.4714679851522	0.0235849425405802\\
37.5407194264671	0.023422177649758\\
37.609970867782	0.0232605360355208\\
37.6796040887873	0.0230991280510355\\
37.7492373097926	0.0229388400980071\\
37.8188705307979	0.0227796644047328\\
37.8885037518033	0.0226215932531502\\
37.9585229632751	0.0224637518823313\\
38.0285421747469	0.0223070118438907\\
38.0985613862187	0.0221513654536663\\
38.1685805976905	0.021996805080824\\
38.2389900798134	0.0218424706833319\\
38.3093995619363	0.0216892191295102\\
38.3798090440592	0.0215370428222367\\
38.4502185261821	0.0213859342174039\\
38.5209622984945	0.0212351759405899\\
38.591706070807	0.02108548042086\\
38.6624498431194	0.0209368401667451\\
38.7331936154318	0.0207892477393\\
38.8042656863826	0.0206420180659731\\
38.8753377573333	0.0204958310745268\\
38.9464098282841	0.0203506793810197\\
39.0174818992348	0.0202065556535179\\
39.0888909542047	0.0200627765168359\\
39.1603000091745	0.0199200204360313\\
39.2317090641443	0.0197782801319219\\
39.3031181191142	0.0196375483768362\\
39.3748725127202	0.0194971446708927\\
39.4466269063262	0.0193577448171507\\
39.5183812999322	0.0192193416386501\\
39.5901356935382	0.0190819280094616\\
39.662243568804	0.0189448271775968\\
39.7343514440698	0.0188087113947145\\
39.8064593193355	0.0186735735837216\\
39.8785671946013	0.0185394067180912\\
39.95103651206	0.0184055385556614\\
40.0235058295186	0.0182726370196721\\
40.0959751469772	0.0181406951307057\\
40.1684444644358	0.0180097059594607\\
40.2412830193083	0.0178790024545455\\
40.3141215741807	0.0177492475160411\\
40.3869601290532	0.0176204342601645\\
40.4597986839256	0.0174925558528114\\
40.5330141309314	0.0173649510255249\\
40.6062295779373	0.0172382770506626\\
40.6794450249431	0.0171125271381699\\
40.752660471949	0.0169876945472457\\
40.8262603410054	0.0168631243178462\\
40.8998602100619	0.016739467557794\\
40.9734600791183	0.0166167175689809\\
41.0470599481748	0.0164948677021377\\
41.1210516636287	0.0163732697632354\\
41.1950433790826	0.0162525682277073\\
41.2690350945364	0.0161327564877213\\
41.3430268099903	0.0160138279838805\\
41.4174177047231	0.0158951416910883\\
41.4918085994559	0.0157773350400996\\
41.5661994941887	0.015660401511792\\
41.6405903889215	0.0155443346350828\\
41.7153877210576	0.0154285009018519\\
41.7901850531936	0.0153135303416431\\
41.8649823853296	0.015199416522569\\
41.9397797174656	0.0150861530603963\\
42.0149906816255	0.0149731142694313\\
42.0902016457854	0.0148609224647634\\
42.1654126099454	0.0147495713003533\\
42.2406235741053	0.0146390544774366\\
42.316255315397	0.0145287543945537\\
42.3918870566888	0.0144192853834496\\
42.4675187979805	0.0143106411826232\\
42.5431505392722	0.0142028155774767\\
42.6192101664345	0.0140951992734046\\
42.6952697935968	0.0139883983897471\\
42.771329420759	0.0138824067483026\\
42.8473890479213	0.0137772182174087\\
42.9238836429179	0.0136722320005469\\
43.0003782379145	0.0135680458075013\\
43.0768728329111	0.0134646535421975\\
43.1533674279077	0.0133620491547413\\
43.2303040564086	0.0132596405075829\\
43.3072406849096	0.0131580167346568\\
43.3841773134105	0.0130571718209032\\
43.4611139419115	0.0129570997970895\\
43.5384996649489	0.0128572173158944\\
43.6158853879863	0.0127581047992253\\
43.6932711110238	0.0126597563119789\\
43.7706568340612	0.0125621659645316\\
43.8484987166093	0.0124647593075753\\
43.9263405991575	0.0123681079386871\\
44.0041824817057	0.0122722060017131\\
44.0820243642538	0.0121770476856358\\
44.1603294838729	0.0120820675251092\\
44.2386346034919	0.0119878282032865\\
44.3169397231109	0.0118943239420019\\
44.39524484273	0.011801549007887\\
44.4740202988846	0.0117089469845201\\
44.5527957550393	0.0116170715718841\\
44.631571211194	0.0115259170688818\\
44.7103466673487	0.0114354778188795\\
44.7895995895968	0.0113452065012878\\
44.8688525118449	0.0112556477825406\\
44.948105434093	0.0111667960377299\\
45.0273583563411	0.0110786456860794\\
45.1070959123988	0.0109906585332237\\
45.1868334684566	0.0109033701784763\\
45.2665710245144	0.0108167750722735\\
45.3463085805721	0.0107308677088558\\
45.4265379845256	0.0106451190351097\\
45.5067673884792	0.0105600555652918\\
45.5869967924327	0.0104756718243707\\
45.6672261963862	0.0103919623807946\\
45.7479547148514	0.0103084073257406\\
45.8286832333165	0.0102255240826796\\
45.9094117517817	0.0101433072503305\\
45.9901402702469	0.0100617514705701\\
46.0713752310891	0.00998034596827876\\
46.1526101919314	0.00989959908428378\\
46.2338451527736	0.00981950549029933\\
46.3150801136159	0.00974005990087875\\
46.3968289125376	0.00966076065364945\\
46.4785777114593	0.00958210702549797\\
46.560326510381	0.00950409376040448\\
46.6420753093027	0.00942671564487172\\
46.7243454163926	0.00934948009884547\\
46.8066155234824	0.00927287736362463\\
46.8888856305723	0.0091969022547489\\
46.9711557376621	0.0091215496299662\\
47.0539547042701	0.00904633595215389\\
47.136753670878	0.00897174246450408\\
47.2195526374859	0.00889776405343073\\
47.3023516040939	0.00882439564724376\\
47.3856870701744	0.0087511627037969\\
47.4690225362549	0.00867853751438819\\
47.5523580023354	0.00860651503563984\\
47.6356934684159	0.00853509026575965\\
47.7195731680798	0.00846379760337656\\
47.8034528677438	0.00839310044047834\\
47.8873325674077	0.00832299380324728\\
47.9712122670716	0.00825347275914265\\
48.0556440359198	0.00818408058621986\\
48.1400758047679	0.0081152718370282\\
48.2245075736161	0.00804704160667811\\
48.3089393424642	0.00797938503124993\\
48.3939311234603	0.00791185420160151\\
48.4789229044563	0.00784489489609994\\
48.5639146854524	0.00777850227816674\\
48.6489064664485	0.00771267155188779\\
48.7344663172648	0.00764696354841467\\
48.820026168081	0.00758181534319296\\
48.9055860188973	0.00751722216735175\\
48.9911458697136	0.00745317929238023\\
49.0772819670679	0.00738925621418895\\
49.1634180644223	0.00732588137965989\\
49.2495541617767	0.00726305008703885\\
49.3356902591311	0.00720075767462872\\
49.4224109073406	0.00713858222182016\\
49.5091315555501	0.00707694362715165\\
49.5958522037596	0.00701583725540613\\
49.6825728519691	0.00695525851112169\\
49.769886487838	0.00689479397335669\\
49.857200123707	0.00683485507512259\\
49.944513759576	0.00677543724717035\\
50.0318273954449	0.00671653595970522\\
50.1197425956162	0.00665774620375686\\
50.2076577957875	0.00659947103359668\\
50.2955729959588	0.006541705945384\\
50.3834881961302	0.00648444647443254\\
50.4720136825001	0.00642729593390276\\
50.5605391688701	0.00637064908831327\\
50.64906465524	0.00631450149868091\\
50.73759014161	0.00625884876487789\\
50.8267347882915	0.00620330242953685\\
50.915879434973	0.0061482490592972\\
51.0050240816545	0.00609368427949039\\
51.094168728336	0.00603960375400498\\
51.1839415683769	0.00598562716001514\\
51.2737144084178	0.00593213296048548\\
51.3634872484587	0.00587911684452637\\
51.4532600884996	0.00582657453950779\\
51.5436703203087	0.00577413376045984\\
51.6340805521178	0.00572216496268131\\
51.7244907839268	0.00567066389853237\\
51.8149010157359	0.005619626358336\\
51.9059580087448	0.0055686879972606\\
51.9970150017537	0.00551821136001682\\
52.0880719947626	0.00546819226169173\\
52.1791289877716	0.00541862655503897\\
52.270842290771	0.00536915773539172\\
52.3625555937704	0.00532014053626519\\
52.4542688967698	0.00527157083495582\\
52.5459821997693	0.00522344454613094\\
52.6383615465513	0.00517541290485362\\
52.7307408933333	0.0051278229333241\\
52.8231202401153	0.00508067057053573\\
52.9154995868973	0.00503395179255758\\
53.0085549039533	0.00498732547242669\\
53.1016102210094	0.00494113102227217\\
53.1946655380654	0.00489536444227633\\
53.2877208551214	0.00485002176940259\\
53.3814622684928	0.00480476941229111\\
53.4752036818643	0.00475993927488794\\
53.5689450952358	0.00471552741806063\\
53.6626865086073	0.00467152993916358\\
53.7571243510415	0.00462762067952606\\
53.8515621934757	0.00458412413737049\\
53.9460000359099	0.00454103643374966\\
54.0404378783442	0.0044983537259094\\
54.1355826971555	0.00445575718430989\\
54.2307275159668	0.00441356400458359\\
54.325872334778	0.00437177036747199\\
54.4210171535893	0.00433037248961613\\
54.5168797176294	0.00428905876702625\\
54.6127422816695	0.00424813919591958\\
54.7086048457095	0.00420761001623291\\
54.8044674097496	0.00416746750350947\\
54.9010587178797	0.00412740717524474\\
54.9976500260098	0.00408773193192671\\
55.0942413341398	0.00404843807219686\\
55.1908326422699	0.00400952193001031\\
55.2881639308243	0.00397068604043484\\
55.3854952193786	0.0039322263117784\\
55.4828265079329	0.00389413910089923\\
55.5801577964873	0.00385642079967663\\
55.6782405480441	0.00381878085664034\\
55.776323299601	0.00378150829168688\\
55.8744060511579	0.00374459951940551\\
55.9724888027147	0.00370805098911433\\
56.0713347542828	0.00367157895892164\\
56.1701807058509	0.00363546566386786\\
56.2690266574189	0.00359970757578977\\
56.367872608987	0.00356430120096097\\
56.4674937609263	0.00352896950330633\\
56.5671149128656	0.00349398803645998\\
56.6667360648049	0.00345935332902458\\
56.7663572167441	0.00342506194374778\\
56.8667658411891	0.00339084344707514\\
56.9671744656341	0.00335696681422726\\
57.0675830900791	0.00332342863009284\\
57.1679917145241	0.00329022551341397\\
57.2692003658787	0.00325709352978088\\
57.3704090172333	0.00322429517909354\\
57.4716176685879	0.00319182710204835\\
57.5728263199425	0.00315968597290375\\
57.6748478434472	0.0031276142537602\\
57.7768693669519	0.00309586807155185\\
57.8788908904565	0.00306444412230596\\
57.9809124139612	0.00303333913532059\\
58.0837599560178	0.00300230186688364\\
58.1866074980744	0.00297158217301892\\
58.2894550401309	0.00294117680460899\\
58.3923025821875	0.00291108254551613\\
58.4959896007833	0.0028810543443434\\
58.5996766193791	0.0028513358878187\\
58.7033636379749	0.00282192398120545\\
58.8070506565707	0.00279281546245592\\
58.911590931702	0.0027637713712179\\
59.0161312068333	0.00273502932594229\\
59.1206714819646	0.00270658618580006\\
59.2252117570959	0.00267843884236029\\
59.3306194021433	0.0026503543255257\\
59.4360270471907	0.00262256428601738\\
59.541434692238	0.00259506563644149\\
59.6468423372854	0.00256785532151165\\
59.7351317529641	0.00254528366553337\\
59.8234211686427	0.00252291041615296\\
59.9117105843214	0.00250073382946013\\
60	0.00247875217676654\\
};
\addlegendentry{Infetti}

\addplot [color=mycolor3, line width=2.0pt]
  table[row sep=crcr]{%
0	0\\
0.00199053585276749	0.000199033775426294\\
0.00398107170553497	0.000398027936408751\\
0.00597160755830246	0.000596982490832176\\
0.00796214341106994	0.000795897446579374\\
0.0123567669950473	0.00123491356541328\\
0.0167513905790247	0.00167373679557492\\
0.0211460141630021	0.00211236722181307\\
0.0255406377469794	0.00255080492883919\\
0.0299388563931551	0.00298940843290074\\
0.0343370750393307	0.00342781907196765\\
0.0387352936855064	0.00386603693084731\\
0.043133512331682	0.00430406209431011\\
0.0475471725191068	0.0047434314778112\\
0.0519608327065316	0.00518260698138595\\
0.0563744928939564	0.00562158869058749\\
0.0607881530813811	0.00606037669093107\\
0.0652173199794049	0.00650051165999344\\
0.0696464868774286	0.00694045172909763\\
0.0740756537754524	0.00738019698454906\\
0.0785048206734761	0.00781974751261472\\
0.0829496187548012	0.00826065360668626\\
0.0873944168361263	0.0087013637704485\\
0.0918392149174515	0.00914187809096911\\
0.0962840129987766	0.00958219665527704\\
0.100744568101794	0.0100238794591123\\
0.105205123204811	0.0104653652918325\\
0.109665678307829	0.010906654241278\\
0.114126233410846	0.01134774639525\\
0.118602673075784	0.011790211570715\\
0.123079112740721	0.0122324787236389\\
0.127555552405659	0.0126745479426453\\
0.132031992070596	0.0131164193163186\\
0.136524445545197	0.0135596726012327\\
0.141016899019798	0.0140027268013931\\
0.1455093524944	0.0144455820062176\\
0.150001805969001	0.0148882383050839\\
0.154510404254387	0.0153322855160064\\
0.159019002539772	0.0157761325690045\\
0.163527600825158	0.0162197795543009\\
0.168036199110544	0.016663226562078\\
0.172561074936287	0.0171080735900792\\
0.17708595076203	0.0175527193758572\\
0.181610826587773	0.0179971640104511\\
0.186135702413515	0.0184414075848589\\
0.190676990405524	0.0188870604102275\\
0.195218278397533	0.0193325108977608\\
0.199759566389542	0.0197777591393254\\
0.204300854381551	0.0202228052267462\\
0.208858690959197	0.0206692699069928\\
0.213416527536843	0.0211155311423014\\
0.217974364114488	0.0215615890253775\\
0.222532200692134	0.0220074436488849\\
0.227106724202768	0.0224547263300531\\
0.231681247713401	0.0229018044475\\
0.236255771224035	0.0233486780947821\\
0.240830294734669	0.0237953473654136\\
0.245421645441331	0.0242434542789139\\
0.250012996147992	0.0246913554980392\\
0.254604346854654	0.0251390511172095\\
0.259195697561315	0.0255865412308008\\
0.263804017673384	0.0260354786944303\\
0.268412337785452	0.026484209320968\\
0.273020657897521	0.0269327332057087\\
0.27762897800959	0.0273810504439037\\
0.282254411746462	0.0278308248660353\\
0.286879845483334	0.0282803912960946\\
0.291505279220206	0.0287297498302648\\
0.296130712957079	0.0291789005646845\\
0.300773406613671	0.0296295184485543\\
0.305416100270264	0.030079927172902\\
0.310058793926856	0.0305301268348113\\
0.314701487583449	0.030980117531321\\
0.319361589521735	0.0314315854720725\\
0.324021691460022	0.0318828430731755\\
0.328681793398309	0.0323338904326275\\
0.333341895336596	0.0327847276483804\\
0.338019556071109	0.0332370523391906\\
0.342697216805623	0.0336891654973341\\
0.347374877540136	0.0341410672217355\\
0.35205253827465	0.034592757611273\\
0.356747910447269	0.0350459458386931\\
0.361443282619888	0.0354989213273224\\
0.366138654792507	0.035951684177026\\
0.370834026965126	0.0364042344876224\\
0.375547265516765	0.0368582931458991\\
0.380260504068404	0.0373121378459241\\
0.384973742620042	0.0377657686885177\\
0.389686981171681	0.0382191857744518\\
0.394418243246227	0.0386741218539115\\
0.399149505320772	0.0391288427420997\\
0.403880767395317	0.0395833485408048\\
0.408612029469863	0.0400376393517673\\
0.41336147459102	0.0404934599534799\\
0.418110919712177	0.0409490641171012\\
0.422860364833334	0.041404451945403\\
0.427609809954491	0.0418596235411082\\
0.432377599961091	0.0423163358682133\\
0.437145389967691	0.0427728304963726\\
0.44191317997429	0.0432291075293558\\
0.44668096998089	0.0436851670708831\\
0.451467269141767	0.0441427784371404\\
0.456253568302644	0.0446001708293159\\
0.461039867463521	0.0450573443521919\\
0.465826166624398	0.0455142991105008\\
0.47063114173014	0.0459728169464957\\
0.475436116835881	0.0464311145187341\\
0.480241091941623	0.0468891919330269\\
0.485046067047365	0.047347049295134\\
0.489869887370685	0.0478064811418594\\
0.494693707694006	0.0482656914203613\\
0.499517528017326	0.0487246802374945\\
0.504341348340647	0.0491834477000624\\
0.509184185682064	0.0496438012106382\\
0.51402702302348	0.0501039318334716\\
0.518869860364897	0.0505638396764775\\
0.523712697706314	0.0510235248475183\\
0.528574726619363	0.0514848078058767\\
0.533436755532412	0.0519458665416402\\
0.53829878444546	0.0524067011638003\\
0.543160813358509	0.0528673117812948\\
0.548042211039569	0.0533295320888524\\
0.55292360872063	0.0537915268233564\\
0.55780500640169	0.054253296094891\\
0.56268640408275	0.0547148400134868\\
0.56758735048985	0.0551780056975584\\
0.572488296896949	0.0556409444422259\\
0.577389243304049	0.0561036563586838\\
0.582290189711149	0.0565661415580722\\
0.587210867601438	0.0570302607724681\\
0.592131545491728	0.0574941516649279\\
0.597052223382018	0.0579578143477738\\
0.601972901272307	0.0584212489332728\\
0.606913496388328	0.0588863299728438\\
0.611854091504349	0.0593511812914561\\
0.61679468662037	0.0598158030025775\\
0.621735281736391	0.0602801952196198\\
0.626695982628195	0.0607462465003742\\
0.631656683519999	0.061212066644363\\
0.636617384411802	0.0616776557662176\\
0.641578085303606	0.062143013980513\\
0.646559083672327	0.0626100440687973\\
0.651540082041047	0.063076841587397\\
0.656521080409768	0.0635434066521263\\
0.661502078778488	0.0640097393787415\\
0.666503569383853	0.0644777569792249\\
0.671505059989219	0.0649455405596723\\
0.676506550594584	0.0654130902370995\\
0.68150804119995	0.0658804061284636\\
0.686530221876942	0.0663494200823989\\
0.691552402553934	0.0668181985481907\\
0.696574583230927	0.0672867416440757\\
0.701596763907919	0.0677550494882312\\
0.706639835866396	0.0682250687980454\\
0.711682907824873	0.0686948511334985\\
0.71672597978335	0.069164396614068\\
0.721769051741827	0.0696337053591717\\
0.72683321952555	0.0701047391810279\\
0.731897387309272	0.0705755345238435\\
0.736961555092995	0.0710460915083577\\
0.742025722876718	0.0715164102552487\\
0.747111194287107	0.0719884678883614\\
0.752196665697496	0.0724602855189459\\
0.757282137107885	0.0729318632690237\\
0.762367608518274	0.0734032012605543\\
0.767474594900923	0.0738762921701672\\
0.772581581283573	0.0743491415345804\\
0.777688567666222	0.0748217494771193\\
0.782795554048871	0.0752941161210461\\
0.78792427039055	0.0757682499435487\\
0.793052986732229	0.0762421406586098\\
0.798181703073909	0.0767157883908801\\
0.803310419415588	0.0771891932649469\\
0.808461084325784	0.0776643798022104\\
0.81361174923598	0.0781393216498327\\
0.818762414146177	0.0786140189338129\\
0.823913079056373	0.0790884717800848\\
0.82908591482452	0.0795647210007625\\
0.834258750592667	0.0800407239292469\\
0.839431586360813	0.0805164806929077\\
0.84460442212896	0.0809919914190492\\
0.849799655005291	0.081469313480329\\
0.854994887881621	0.081946387626087\\
0.860190120757951	0.0824232139850872\\
0.865385353634282	0.0828997926860274\\
0.870603213699137	0.083378197917317\\
0.875821073763992	0.0838563535885657\\
0.881038933828848	0.0843342598299563\\
0.886256793893703	0.0848119167716039\\
0.891497515445601	0.0852914157057909\\
0.896738236997499	0.0857706634137742\\
0.901978958549397	0.0862496600271798\\
0.907219680101295	0.086728405677565\\
0.912483501408722	0.0872090090237274\\
0.917747322716149	0.0876893594554472\\
0.923011144023576	0.0881694571058188\\
0.928274965331004	0.0886493021078669\\
0.933562128954513	0.0891310207761897\\
0.938849292578023	0.0896124848192889\\
0.944136456201532	0.090093694371753\\
0.949423619825042	0.0905746495681001\\
0.954734372782572	0.0910574946801422\\
0.960045125740102	0.0915400834331526\\
0.965355878697632	0.0920224159632411\\
0.970666631655162	0.0925044924064453\\
0.976001225418392	0.0929884752898608\\
0.981335819181621	0.0934722000569226\\
0.98667041294485	0.0939556668452886\\
0.99200500670808	0.0944388757925432\\
0.997363697280812	0.0949240079832017\\
1.00272238785355	0.0954088802761722\\
1.00808107842628	0.0958934928106889\\
1.01343976899901	0.0963778457259106\\
1.0188228172289	0.0968641389909677\\
1.02420586545878	0.0973501705524594\\
1.02958891368867	0.0978359405512239\\
1.03497196191855	0.098321449128024\\
1.04037963341625	0.0988089154532208\\
1.04578730491395	0.0992961182439042\\
1.05119497641164	0.0997830576425464\\
1.05660264790934	0.100269733791543\\
1.06203521338424	0.100758385406022\\
1.06746777885913	0.101246771629407\\
1.07290034433402	0.101734892605833\\
1.07833290980891	0.102222748479359\\
1.08379064506731	0.102712597849773\\
1.08924838032571	0.10320217994631\\
1.09470611558411	0.103691494914803\\
1.10016385084252	0.104180542901002\\
1.1056470369148	0.10467160273712\\
1.11113022298707	0.105162393389799\\
1.11661340905935	0.105652915006596\\
1.12209659513163	0.106143167734989\\
1.12760551858231	0.106635451011475\\
1.13311444203298	0.107127463167559\\
1.13862336548365	0.107619204352557\\
1.14413228893432	0.108110674715705\\
1.14966724184116	0.108604194663977\\
1.15520219474799	0.109097441526865\\
1.16073714765482	0.109590415455479\\
1.16627210056166	0.110083116600843\\
1.17183338076584	0.110577886725028\\
1.17739466097002	0.111072381770179\\
1.18295594117419	0.111566601889234\\
1.18851722137837	0.112060547235044\\
1.19410513262804	0.112556581318129\\
1.19969304387772	0.113052338299198\\
1.20528095512739	0.113547818333051\\
1.21086886637706	0.114043021574401\\
1.21648371859449	0.114540333695119\\
1.22209857081192	0.115037366660808\\
1.22771342302936	0.115534120628167\\
1.23332827524679	0.116030595753804\\
1.23897038457538	0.116529200283702\\
1.24461249390397	0.11702752357482\\
1.25025460323256	0.11752556578579\\
1.25589671256115	0.118023327075157\\
1.26156640151281	0.118523238684409\\
1.26723609046446	0.119022866939663\\
1.27290577941612	0.119522212001525\\
1.27857546836778	0.120021274030511\\
1.28427306633634	0.120522507725694\\
1.28997066430491	0.12102345591941\\
1.29566826227347	0.12152411877428\\
1.30136586024204	0.122024496452832\\
1.30709170327692	0.122527067549119\\
1.31281754631181	0.123029350963454\\
1.31854338934669	0.123531346860512\\
1.32426923238158	0.124033055404873\\
1.33002366371043	0.124536979564021\\
1.33577809503928	0.125040613826889\\
1.34153252636813	0.125543958360249\\
1.34728695769699	0.126047013330775\\
1.35307032782915	0.126552306561804\\
1.35885369796131	0.127057307647542\\
1.36463706809348	0.127562016756899\\
1.37042043822564	0.128066434058687\\
1.37622881553645	0.128572738918064\\
1.38203719284726	0.129078749781866\\
1.38784557015807	0.129584466820806\\
1.39365394746888	0.1300898902055\\
1.39948194569202	0.130596725956678\\
1.40530994391516	0.131103266410129\\
1.41113794213829	0.131609511737902\\
1.41696594036143	0.132115462111945\\
1.4228137248248	0.132622833919104\\
1.42866150928816	0.1331299091129\\
1.43450929375153	0.133636687866737\\
1.44035707821489	0.134143170353915\\
1.44622478964653	0.134651081127397\\
1.45209250107816	0.135158693960913\\
1.45796021250979	0.135666009029236\\
1.46382792394143	0.136173026507033\\
1.46971570462452	0.136681479186257\\
1.4756034853076	0.137189632587809\\
1.48149126599069	0.137697486887842\\
1.48737904667378	0.138205042262411\\
1.49328704060906	0.138714039828256\\
1.49919503454433	0.139222736767462\\
1.50510302847961	0.139731133257585\\
1.51101102241489	0.14023922947608\\
1.51693937480906	0.140748774905147\\
1.52286772720322	0.141258018347252\\
1.52879607959739	0.141766959981368\\
1.53472443199155	0.142275599986366\\
1.540673290528	0.14278569635866\\
1.54662214906446	0.143295489372078\\
1.55257100760091	0.143804979207032\\
1.55851986613736	0.144314166043824\\
1.56448937947106	0.144824816411477\\
1.57045889280476	0.145335162036679\\
1.57642840613847	0.145845203101293\\
1.58239791947216	0.146354939787072\\
1.58838823791446	0.146866147231752\\
1.59437855635675	0.147377048538597\\
1.60036887479904	0.147887643890936\\
1.60635919324133	0.148397933471993\\
1.61237046959671	0.148909701174399\\
1.61838174595208	0.149421161331542\\
1.62439302230746	0.149932314128241\\
1.63040429866284	0.150443159749203\\
1.63643668672632	0.150955490858901\\
1.6424690747898	0.151467513003791\\
1.64850146285328	0.151979226370196\\
1.65453385091676	0.152490631144328\\
1.66058750665404	0.153003528878666\\
1.66664116239132	0.153516116216334\\
1.67269481812859	0.154028393345179\\
1.67874847386587	0.154540360452933\\
1.68482355500326	0.155053828060135\\
1.69089863614064	0.155566983826333\\
1.69697371727803	0.156079827940914\\
1.70304879841541	0.156592360593152\\
1.70914546464501	0.157106401367774\\
1.71524213087462	0.157620128844407\\
1.72133879710422	0.158133543213999\\
1.72743546333383	0.158646644667383\\
1.73355387572031	0.159161261898375\\
1.73967228810678	0.159675564361626\\
1.74579070049326	0.160189552249665\\
1.75190911287974	0.160703225754905\\
1.75804943463327	0.161218422788997\\
1.76418975638681	0.161733303572637\\
1.77033007814035	0.162247868299954\\
1.77647039989388	0.162762117164956\\
1.78263279664118	0.163277897427093\\
1.78879519338848	0.163793359942884\\
1.79495759013578	0.164308504908074\\
1.80111998688309	0.164823332518293\\
1.80730462566597	0.16533969942552\\
1.81348926444885	0.165855747077202\\
1.81967390323173	0.166371475670725\\
1.82585854201462	0.166886885403353\\
1.83206559217232	0.167403842436778\\
1.83827264233002	0.167920478691944\\
1.84447969248772	0.168436794367898\\
1.85068674264543	0.168952789663563\\
1.85691637550678	0.169470340340375\\
1.86314600836813	0.169987568702522\\
1.86937564122948	0.170504474950731\\
1.87560527409083	0.171021059285606\\
1.88185766319929	0.171539207175622\\
1.88811005230775	0.172057031200673\\
1.89436244141622	0.172574531563187\\
1.90061483052468	0.173091708465469\\
1.90689015093446	0.17361045713063\\
1.91316547134424	0.174128880366502\\
1.91944079175402	0.174646978377237\\
1.92571611216381	0.175164751366862\\
1.93201454150218	0.17568410444719\\
1.93831297084055	0.176203130519644\\
1.94461140017892	0.17672182979012\\
1.9509098295173	0.17724020246439\\
1.95723154808653	0.177760163683658\\
1.96355326665576	0.178279796301955\\
1.96987498522499	0.178799100526945\\
1.97619670379422	0.179318076566165\\
1.98254189319299	0.179838649616003\\
1.98888708259176	0.180358892457154\\
1.99523227199053	0.180878805299075\\
2.0015774613893	0.181398388351091\\
2.00794630624256	0.181919577031104\\
2.01431515109582	0.182440433879813\\
2.02068399594908	0.182960959108489\\
2.02705284080234	0.183481152928268\\
2.03344552746868	0.18400296103736\\
2.03983821413503	0.184524435677477\\
2.04623090080137	0.185045577061728\\
2.05262358746771	0.185566385403084\\
2.05904030495676	0.186088816812266\\
2.06545702244581	0.186610913099502\\
2.07187373993486	0.187132674479762\\
2.07829045742391	0.187654101167878\\
2.084731396657	0.188177159757429\\
2.0911723358901	0.188699881556595\\
2.0976132751232	0.18922226678223\\
2.1040542143563	0.189744315651051\\
2.11051956957009	0.19026800542212\\
2.11698492478388	0.190791356718579\\
2.12345027999767	0.191314369759195\\
2.12991563521146	0.191837044762589\\
2.13640560238783	0.192361369706867\\
2.1428955695642	0.192885354476374\\
2.14938553674057	0.193408999291813\\
2.15587550391694	0.19393230437374\\
2.16239028199053	0.194457268568841\\
2.16890506006412	0.194981890872798\\
2.17541983813771	0.195506171508274\\
2.1819346162113	0.196030110697785\\
2.18847440630977	0.196555718243055\\
2.19501419640825	0.197080982164396\\
2.20155398650673	0.197605902686458\\
2.20809377660521	0.198130480033743\\
2.21465878275268	0.198656735104258\\
2.22122378890015	0.199182644801377\\
2.22778879504762	0.199708209351764\\
2.23435380119508	0.200233428981932\\
2.24094422983074	0.200760335786924\\
2.24753465846639	0.201286895452148\\
2.25412508710205	0.20181310820631\\
2.2607155157377	0.202338974277964\\
2.2673315759421	0.202866537076246\\
2.27394763614649	0.203393750951242\\
2.28056369635089	0.203920616133726\\
2.28717975655528	0.204447132854318\\
2.29382166035897	0.204975355975974\\
2.30046356416266	0.205503228373401\\
2.30710546796636	0.206030750279469\\
2.31374737177005	0.206557921926894\\
2.32041533418655	0.207086809772711\\
2.32708329660306	0.20761534507565\\
2.33375125901957	0.208143528070707\\
2.34041922143607	0.208671358992722\\
2.34711345989709	0.209200915986299\\
2.3538076983581	0.209730118600426\\
2.36050193681912	0.210258967072253\\
2.36719617528013	0.21078746163877\\
2.37391691043736	0.211317692287134\\
2.3806376455946	0.211847566701242\\
2.38735838075183	0.212377085120432\\
2.39407911590907	0.212906247783875\\
2.40082657131283	0.213437156648704\\
2.40757402671659	0.213967707405973\\
2.41432148212035	0.214497900297233\\
2.42106893752412	0.21502773556387\\
2.42784333963791	0.215559327259491\\
2.4346177417517	0.216090558955474\\
2.4413921438655	0.216621430895615\\
2.44816654597929	0.217151943323545\\
2.4549681243632	0.217684222528106\\
2.4617697027471	0.218216139821886\\
2.46857128113101	0.218747695450957\\
2.47537285951492	0.219278889661226\\
2.48220184698832	0.219811861126013\\
2.48903083446173	0.220344468749502\\
2.49585982193513	0.220876712780074\\
2.50268880940853	0.221408593465943\\
2.50954544177924	0.221942261990751\\
2.51640207414995	0.22247556472409\\
2.52325870652066	0.223008501916683\\
2.53011533889137	0.223541073819084\\
2.53699985562036	0.224075444300518\\
2.54388437234935	0.224609447020308\\
2.55076888907833	0.225143082231552\\
2.55765340580732	0.225676350187176\\
2.5645660494942	0.226211427574851\\
2.57147869318108	0.226746135210406\\
2.57839133686796	0.22728047334935\\
2.58530398055484	0.227814442247013\\
2.59224499685821	0.228350231533953\\
2.59918601316159	0.228885649057711\\
2.60612702946496	0.229420695076238\\
2.61306804576833	0.229955369847309\\
2.62003768421203	0.230491876128618\\
2.62700732265572	0.231028008614723\\
2.63397696109942	0.231563767566055\\
2.64094659954311	0.232099153242863\\
2.64794511298781	0.232636381670939\\
2.65494362643251	0.233173234250512\\
2.6619421398772	0.233709711244527\\
2.6689406533219	0.234245812915747\\
2.67596829852112	0.234783768739138\\
2.68299594372033	0.235321346639075\\
2.69002358891955	0.235858546881054\\
2.69705123411877	0.236395369730388\\
2.70410827073433	0.236934058214376\\
2.71116530734989	0.237472366678035\\
2.71822234396545	0.238010295389453\\
2.72527938058101	0.238547844616526\\
2.73236607296869	0.23908727117576\\
2.73945276535637	0.239626315595406\\
2.74653945774405	0.240164978146179\\
2.75362615013173	0.240703259098601\\
2.76074276611797	0.241243429199104\\
2.76785938210422	0.241783215018046\\
2.77497599809046	0.242322616828808\\
2.78209261407671	0.242861634904575\\
2.78923942490227	0.24340255405562\\
2.79638623572783	0.24394308676009\\
2.80353304655339	0.244483233294073\\
2.81067985737896	0.245022993933458\\
2.81785713902212	0.245564667783582\\
2.82503442066529	0.246105952998611\\
2.83221170230846	0.246646849857381\\
2.83938898395163	0.247187358638524\\
2.84659701614953	0.247729792896651\\
2.85380504834743	0.248271836307297\\
2.86101308054533	0.248813489152084\\
2.86822111274323	0.249354751712433\\
2.87546017926332	0.249897952164036\\
2.88269924578341	0.250440759531514\\
2.88993831230351	0.25098317409932\\
2.8971773788236	0.251525196151702\\
2.90444776780364	0.25206916867972\\
2.91171815678368	0.252612745862285\\
2.91898854576373	0.253155927986723\\
2.92625893474377	0.253698715340152\\
2.93356093845992	0.25424346590281\\
2.94086294217606	0.254787818833599\\
2.94816494589221	0.255331774422762\\
2.95546694960836	0.255875332960334\\
2.96280086487659	0.256420867616342\\
2.97013478014483	0.256966002328532\\
2.97746869541306	0.25751073739011\\
2.9848026106813	0.258055073094071\\
2.99216873909201	0.258601398014918\\
2.99953486750271	0.259147320653994\\
3.00690099591342	0.259692841307517\\
3.01426712432413	0.260237960271483\\
3.02166577186631	0.260785081708295\\
3.02906441940849	0.26133179849895\\
3.03646306695067	0.261878110942721\\
3.04386171449286	0.262424019338658\\
3.05129319192691	0.262971943644704\\
3.05872466936097	0.263519460913302\\
3.06615614679503	0.264066571446828\\
3.07358762422909	0.264613275547435\\
3.08105224696524	0.265162009163805\\
3.0885168697014	0.265710333324073\\
3.09598149243755	0.266258248333768\\
3.1034461151737	0.266805754498193\\
3.11094420406679	0.26735530400676\\
3.11844229295988	0.267904441612663\\
3.12594038185297	0.268453167624636\\
3.13343847074605	0.26900148235118\\
3.14097035080817	0.269551854374409\\
3.14850223087028	0.270101812020104\\
3.1560341109324	0.27065135560025\\
3.16356599099451	0.271200485426599\\
3.17113199342146	0.271751686770372\\
3.17869799584842	0.2723024712328\\
3.18626399827537	0.272852839129178\\
3.19383000070233	0.273402790774559\\
3.20143046110215	0.273954828292554\\
3.20903092150198	0.274506446396027\\
3.21663138190181	0.275057645403631\\
3.22423184230164	0.275608425633777\\
3.2318671021242	0.276161306325887\\
3.23950236194676	0.27671376504034\\
3.24713762176933	0.277265802099202\\
3.25477288159189	0.277817417824298\\
3.26244328804483	0.278371148823909\\
3.27011369449776	0.278924455252189\\
3.2777841009507	0.279477337434678\\
3.28545450740363	0.280029795696663\\
3.29316041266953	0.280584384207756\\
3.30086631793543	0.281138545522814\\
3.30857222320133	0.281692279970904\\
3.31627812846722	0.282245587880838\\
3.32401989110489	0.282801041272756\\
3.33176165374256	0.283356064812255\\
3.33950341638023	0.283910658831987\\
3.3472451790179	0.284464823664348\\
3.35502316313618	0.285021149405423\\
3.36280114725446	0.285577042605458\\
3.37057913137275	0.28613250360075\\
3.37835711549103	0.286687532727337\\
3.38617169154667	0.287244738435355\\
3.3939862676023	0.287801508880827\\
3.40180084365794	0.288357844403758\\
3.40961541971357	0.28891374534389\\
3.41746696420648	0.289471838757129\\
3.42531850869938	0.29002949415282\\
3.43317005319229	0.290586711874738\\
3.4410215976852	0.291143492266392\\
3.44891049346576	0.291702481258394\\
3.45679938924632	0.292261029443704\\
3.46468828502688	0.292819137169931\\
3.47257718080744	0.293376804784412\\
3.48050381756295	0.29393669739098\\
3.48843045431847	0.294496146366865\\
3.49635709107398	0.295055152063577\\
3.50428372782949	0.295613714832348\\
3.51224850159594	0.296174519209257\\
3.52021327536239	0.296734877096\\
3.52817804912884	0.297294788848055\\
3.53614282289529	0.297854254820617\\
3.54414613607264	0.298415979237294\\
3.55214944924998	0.29897725426818\\
3.56015276242733	0.299538080272791\\
3.56815607560468	0.300098457610351\\
3.57619833889781	0.300661110578176\\
3.58424060219095	0.301223311227578\\
3.59228286548408	0.301785059922177\\
3.60032512877721	0.302346357025299\\
3.60840675922679	0.302909947150702\\
3.61648838967638	0.30347308198739\\
3.62457002012596	0.304035761903165\\
3.63265165057554	0.304597987265526\\
3.64077307248204	0.305162523306733\\
3.64889449438855	0.305726601050529\\
3.65701591629505	0.306290220868966\\
3.66513733820155	0.306853383133793\\
3.67329898353354	0.307418874020593\\
3.68146062886554	0.30798390356208\\
3.68962227419754	0.308548472134633\\
3.69778391952953	0.309112580114327\\
3.70598622860803	0.309679034986307\\
3.71418853768653	0.31024502542499\\
3.72239084676502	0.310810551811163\\
3.73059315584352	0.311375614525299\\
3.73883657626523	0.311943042647948\\
3.74707999668693	0.312510003208482\\
3.75532341710863	0.313076496592173\\
3.76356683753034	0.313642523183976\\
3.77185182516863	0.314210934008716\\
3.78013681280692	0.314778874100819\\
3.78842180044522	0.315346343850124\\
3.79670678808351	0.315913343646147\\
3.80503380745825	0.316482746825826\\
3.81336082683299	0.31705167605973\\
3.82168784620774	0.317620131742348\\
3.83001486558248	0.318188114267846\\
3.83838438958511	0.318758519627773\\
3.84675391358775	0.319328447785291\\
3.85512343759039	0.319897899139628\\
3.86349296159302	0.320466874089679\\
3.8719054722877	0.321038291672064\\
3.88031798298237	0.321609228750937\\
3.88873049367705	0.322179685730352\\
3.89714300437172	0.322749663014024\\
3.90559899284027	0.323322103056582\\
3.91405498130882	0.3238940592491\\
3.92251096977737	0.32446553200055\\
3.93096695824593	0.325036521719553\\
3.93946692479213	0.325609994658515\\
3.94796689133833	0.326182980354504\\
3.95646685788454	0.3267554792215\\
3.96496682443074	0.327327491673128\\
3.97351127881372	0.327902008147892\\
3.98205573319669	0.328476033939336\\
3.99060018757967	0.329049569466543\\
3.99914464196265	0.329622615148238\\
4.00773410425464	0.330198186047482\\
4.01632356654662	0.330773262774537\\
4.02491302883861	0.331347845753688\\
4.0335024911306	0.331921935408855\\
4.04213749081375	0.332498571797879\\
4.05077249049689	0.333074710476317\\
4.05940749018004	0.333650351873757\\
4.06804248986318	0.334225496419414\\
4.07672356764034	0.334803209649222\\
4.08540464541749	0.335380421579305\\
4.09408572319465	0.335957132644655\\
4.10276680097181	0.336533343279889\\
4.11149450824228	0.337112144938933\\
4.12022221551276	0.337690441657214\\
4.12894992278323	0.338268233875235\\
4.1376776300537	0.338845522033119\\
4.14645252883691	0.339425423929309\\
4.15522742762012	0.340004817190649\\
4.16400232640333	0.340583702263265\\
4.17277722518654	0.34116207959289\\
4.18159988926615	0.341743093816002\\
4.19042255334575	0.342323595655845\\
4.19924521742536	0.342903585564277\\
4.20806788150496	0.343483063992761\\
4.21693889629988	0.344065202891689\\
4.22580991109479	0.344646825603327\\
4.23468092588971	0.345227932585384\\
4.24355194068463	0.34580852429516\\
4.2524719033702	0.34639180047165\\
4.26139186605577	0.346974556599943\\
4.27031182874135	0.347556793143712\\
4.27923179142692	0.348138510566217\\
4.28820131218307	0.348722936941771\\
4.29717083293922	0.349306839349896\\
4.30614035369538	0.349890218260356\\
4.31510987445153	0.350473074142492\\
4.32412957614907	0.35105866392211\\
4.33314927784661	0.351643725755348\\
4.34216897954416	0.352228260118181\\
4.3511886812417	0.352812267486159\\
4.36025919971018	0.353399034160065\\
4.36932971817866	0.353985268847483\\
4.37840023664715	0.354570972030732\\
4.38747075511563	0.355156144191696\\
4.39659274002011	0.355744101575849\\
4.40571472492458	0.356331522870712\\
4.41483670982906	0.35691840856508\\
4.42395869473353	0.357504759147304\\
4.43313280988389	0.358093921386185\\
4.44230692503424	0.3586825433687\\
4.45148104018459	0.359270625590259\\
4.46065515533495	0.359858168545819\\
4.469882079136	0.360448550123711\\
4.47910900293706	0.361038387212257\\
4.48833592673811	0.361627680313621\\
4.49756285053916	0.362216429929503\\
4.50684327607396	0.362808045657692\\
4.51612370160877	0.363399112596\\
4.52540412714357	0.36398963125349\\
4.53468455267837	0.364579602138754\\
4.54401918924822	0.365172467235005\\
4.55335382581807	0.365764779171452\\
4.56268846238792	0.366356538464209\\
4.57202309895777	0.366947745628908\\
4.58141267108196	0.367541875634011\\
4.59080224320616	0.368135448038283\\
4.60019181533036	0.368728463365042\\
4.60958138745456	0.369320922137114\\
4.61902663678989	0.369916333016738\\
4.62847188612523	0.370511181781447\\
4.63791713546056	0.371105468961924\\
4.6473623847959	0.371699195088347\\
4.65686406984621	0.372295903193343\\
4.66636575489653	0.372892044594365\\
4.67586743994685	0.373487619829623\\
4.68536912499717	0.374082629436813\\
4.69492802222378	0.374680651551437\\
4.7044869194504	0.375278102295996\\
4.71404581667701	0.375874982216393\\
4.72360471390363	0.376471291858014\\
4.73320764525141	0.377069774790263\\
4.74281057659919	0.377667683279322\\
4.75241350794698	0.37826501787656\\
4.76201643929476	0.378861779132816\\
4.77165794452804	0.379460361265516\\
4.78129944976131	0.380058366553067\\
4.79094095499459	0.380655795551367\\
4.80058246022786	0.381252648815777\\
4.81026314694842	0.381851348904219\\
4.81994383366897	0.38244946969031\\
4.82962452038952	0.383047011734581\\
4.83930520711008	0.383643975597024\\
4.84902565638322	0.38424281024972\\
4.85874610565637	0.384841063091049\\
4.86846655492951	0.385438734686282\\
4.87818700420266	0.386035825600142\\
4.88794780523678	0.386634811438346\\
4.89770860627089	0.387233212903635\\
4.90746940730501	0.387831030566125\\
4.91723020833913	0.388428264995376\\
4.92703195872998	0.389027418658936\\
4.93683370912084	0.389625985334753\\
4.9466354595117	0.390223965597895\\
4.95643720990256	0.390821360022868\\
4.96628051524823	0.391420698137408\\
4.97612382059389	0.392019446595398\\
4.98596712593956	0.392617605976968\\
4.99581043128522	0.393215176861679\\
5.00569590732134	0.393814716159109\\
5.01558138335745	0.394413663076249\\
5.02546685939356	0.395012018198406\\
5.03535233542967	0.395609782110309\\
5.04528060635073	0.396209539315967\\
5.05520887727179	0.396808701361917\\
5.06513714819285	0.397407268838756\\
5.07506541911391	0.398005242336496\\
5.08503711889176	0.398605234238949\\
5.0950088186696	0.399204628145691\\
5.10498051844744	0.399803424652728\\
5.11495221822528	0.400401624355473\\
5.12496799094167	0.401001867815348\\
5.13498376365806	0.401601510385986\\
5.14499953637444	0.402200552668921\\
5.15501530909083	0.402798995265086\\
5.16507580953502	0.403399507239228\\
5.17513630997921	0.403999415372069\\
5.18519681042339	0.4045987202708\\
5.19525731086758	0.405197422541997\\
5.2053632039849	0.405798220038471\\
5.21546909710223	0.406398410682107\\
5.22557499021955	0.406997995085875\\
5.23568088333688	0.407596973862124\\
5.24583284576096	0.408198074019315\\
5.25598480818505	0.408798564251534\\
5.26613677060914	0.409398445177661\\
5.27628873303322	0.409997717415946\\
5.28648745229033	0.410599137442228\\
5.29668617154743	0.411199944409784\\
5.30688489080453	0.411800138943536\\
5.31708361006163	0.412399721667769\\
5.32732978639303	0.413001478935986\\
5.33757596272443	0.413602619948863\\
5.34782213905582	0.414203145337501\\
5.35806831538722	0.414803055732357\\
5.36836266133408	0.415405167740852\\
5.37865700728095	0.416006660233346\\
5.38895135322782	0.416607533847262\\
5.39924569917468	0.417207789219367\\
5.40958894010065	0.417810273608182\\
5.41993218102661	0.418412135155045\\
5.43027542195257	0.419013374503842\\
5.44061866287853	0.419613992297795\\
5.45101153706255	0.420216866839189\\
5.46140441124658	0.420819115146133\\
5.4717972854306	0.421420737869126\\
5.48219015961463	0.422021735657991\\
5.49263341945276	0.422625018310818\\
5.50307667929089	0.423227671268756\\
5.51351993912902	0.423829695189068\\
5.52396319896715	0.424431090728332\\
5.53445761187213	0.425034799673295\\
5.54495202477712	0.425637875393488\\
5.5554464376821	0.426240318553095\\
5.56594085058709	0.426842129815601\\
5.57648719831941	0.427446283399363\\
5.58703354605173	0.428049800157615\\
5.59757989378405	0.428652680761621\\
5.60812624151637	0.429254925881939\\
5.61872532103302	0.429859542648709\\
5.62932440054967	0.430463518916854\\
5.63992348006633	0.431066855364886\\
5.65052255958298	0.431669552670594\\
5.66117518401188	0.432274651399589\\
5.67182780844078	0.432879105882839\\
5.68248043286967	0.433482916806271\\
5.69313305729857	0.434086084855079\\
5.70384005671797	0.434691684585515\\
5.71454705613736	0.435296636247359\\
5.72525405555675	0.435900940534129\\
5.73596105497615	0.436504598138596\\
5.74672327665992	0.437110718163451\\
5.75748549834369	0.437716186219393\\
5.76824772002746	0.438321003007708\\
5.77900994171123	0.438925169228928\\
5.78982825023888	0.439531829080726\\
5.80064655876654	0.440137832984057\\
5.81146486729419	0.44074318164816\\
5.82228317582184	0.441347875781512\\
5.83315845473553	0.44195509530424\\
5.84403373364921	0.442561654817757\\
5.8549090125629	0.443167555039448\\
5.86578429147658	0.443772796685923\\
5.87671744355088	0.444380596027417\\
5.88765059562518	0.444987731215779\\
5.89858374769948	0.44559420297674\\
5.90951689977378	0.446200012035239\\
5.92050884813521	0.446808411686131\\
5.93150079649664	0.447416142954678\\
5.94249274485807	0.448023206575157\\
5.9534846932195	0.448629603281041\\
5.96453638108293	0.449238624035855\\
5.97558806894636	0.449846972091734\\
5.98663975680979	0.45045464819171\\
5.99769144467322	0.451061653077999\\
6.00880383710542	0.451671316107727\\
6.01991622953761	0.452280302032255\\
6.0310286219698	0.45288861160359\\
6.042141014402	0.453496245572904\\
6.05331509935885	0.454106572455298\\
6.0644891843157	0.454716217734135\\
6.07566326927256	0.455325182170618\\
6.08683735422941	0.455933466525101\\
6.09807414230094	0.456544479203158\\
6.10931093037248	0.457154805684823\\
6.12054771844401	0.457764446740726\\
6.13178450651555	0.458373403140632\\
6.14308503319571	0.458985124017979\\
6.15438555987588	0.459596154008959\\
6.16568608655605	0.460206493893868\\
6.17698661323621	0.460816144452123\\
6.18835193888863	0.461428596360911\\
6.19971726454106	0.462040352593565\\
6.21108259019348	0.462651413940293\\
6.2224479158459	0.463261781190406\\
6.23387912710053	0.46387498743589\\
6.24531033835517	0.464487493112854\\
6.2567415496098	0.465099299021675\\
6.26817276086443	0.465710405961816\\
6.2796709718506	0.466324390356062\\
6.29116918283678	0.466937669183813\\
6.30266739382295	0.467550243255878\\
6.31416560480912	0.468162113382133\\
6.32573195844141	0.468776900278629\\
6.33729831207371	0.469390976501934\\
6.34886466570601	0.470004342873562\\
6.36043101933831	0.470617000214078\\
6.37206668825559	0.471232614522004\\
6.38370235717287	0.471847512938077\\
6.39533802609016	0.472461696294803\\
6.40697369500744	0.473075165423714\\
6.41867988207414	0.473691632597677\\
6.43038606914084	0.474307378545857\\
6.44209225620754	0.47492240411204\\
6.45379844327423	0.475536710139025\\
6.46557638449348	0.476154056294287\\
6.47735432571273	0.476770675770899\\
6.48913226693197	0.477386569424236\\
6.50091020815122	0.478001738108662\\
6.51276117329228	0.478619990012752\\
6.52461213843333	0.479237509662647\\
6.53646310357439	0.479854297925625\\
6.54831406871544	0.480470355667937\\
6.56023936237899	0.481089540753915\\
6.57216465604253	0.4817079878836\\
6.58408994970607	0.482325697936502\\
6.59601524336962	0.482942671791082\\
6.608016207688	0.483562818253513\\
6.62001717200638	0.484182220926786\\
6.63201813632476	0.484800880702985\\
6.64401910064313	0.485418798473121\\
6.65609711612489	0.486039935264489\\
6.66817513160664	0.486660322298749\\
6.68025314708839	0.487279960480912\\
6.69233116257014	0.487898850714896\\
6.70448764976455	0.488521007582531\\
6.71664413695897	0.489142408585492\\
6.72880062415338	0.489763054642088\\
6.7409571113478	0.490382946669509\\
6.75319353351478	0.491006154241006\\
6.76542995568177	0.491628599695783\\
6.77766637784875	0.492250283965828\\
6.78990280001574	0.492871207981988\\
6.80222066442117	0.493495497776816\\
6.8145385288266	0.494119019053363\\
6.82685639323203	0.494741772757696\\
6.83917425763746	0.495363759834717\\
6.85157511856051	0.495989164359945\\
6.86397597948357	0.4966137938104\\
6.87637684040662	0.497237649146644\\
6.88877770132968	0.497860731328049\\
6.90126316148383	0.498487284089763\\
6.91374862163799	0.499113055059677\\
6.92623408179214	0.499738045213283\\
6.93871954194629	0.500362255524858\\
6.95129125486549	0.500989991085332\\
6.96386296778468	0.501616937970535\\
6.97643468070388	0.502243097171342\\
6.98900639362307	0.502868469677386\\
7.00166606764739	0.503497423787794\\
7.0143257416717	0.504125582166592\\
7.02698541569601	0.504752945820513\\
7.03964508972033	0.505379515755016\\
7.05239448925109	0.50600972534693\\
7.06514388878186	0.506639131971434\\
7.07789328831262	0.507267736651609\\
7.09064268784339	0.507895540409235\\
7.10348363709426	0.508527043706545\\
7.11632458634514	0.509157736614096\\
7.12916553559601	0.509787620171838\\
7.14200648484689	0.510416695418384\\
7.15494087138423	0.511049532032855\\
7.16787525792157	0.511681550641124\\
7.18080964445891	0.512312752300545\\
7.19374403099625	0.512943138067108\\
7.20677380774895	0.513577349013713\\
7.21980358450165	0.514210734135747\\
7.23283336125435	0.514843294508538\\
7.24586313800706	0.515475031206016\\
7.25899032889946	0.516110659090059\\
7.27211751979186	0.516745453120673\\
7.28524471068426	0.517379414391754\\
7.29837190157667	0.518012543995765\\
7.31159860422458	0.51864963304957\\
7.32482530687249	0.519285880001662\\
7.3380520095204	0.519921285965128\\
7.35127871216831	0.520555852051581\\
7.36460710304072	0.521194448284569\\
7.37793549391314	0.521832193938506\\
7.39126388478556	0.52246909014632\\
7.40459227565797	0.523105138039433\\
7.41802461355885	0.523745289299684\\
7.43145695145972	0.52438458126438\\
7.4448892893606	0.525023015086979\\
7.45832162726148	0.525660591919391\\
7.47186025959003	0.526302348080824\\
7.48539889191859	0.526943235980074\\
7.49893752424715	0.527583256791852\\
7.51247615657571	0.52822241168928\\
7.52612352432791	0.528865824769163\\
7.53977089208011	0.529508360358463\\
7.55341825983232	0.530150019653906\\
7.56706562758452	0.530790803850586\\
7.58082427117841	0.531435928157126\\
7.5945829147723	0.532080165470458\\
7.60834155836619	0.532723517010123\\
7.62210020196009	0.533365983993987\\
7.63597276754062	0.53401287628637\\
7.64984533312116	0.534658871795357\\
7.66371789870169	0.535303971764153\\
7.67759046428223	0.53594817743424\\
7.69157971106386	0.53659689711999\\
7.70556895784549	0.537244709930237\\
7.71955820462712	0.537891617132742\\
7.73354745140875	0.538537619993496\\
7.7476562591015	0.539188229317175\\
7.76176506679424	0.539837921355914\\
7.77587387448699	0.540486697402978\\
7.78998268217974	0.541134558749811\\
7.80421406003167	0.541787123042984\\
7.81844543788359	0.542438759307768\\
7.83267681573552	0.543089468863934\\
7.84690819358745	0.543739253029377\\
7.86126528820976	0.544393840889616\\
7.87562238283208	0.545047489626186\\
7.8899794774544	0.545700200586428\\
7.90433657207672	0.546351975115749\\
7.91882267896367	0.547008658739207\\
7.93330878585062	0.547664391772435\\
7.94779489273757	0.548319175591468\\
7.96228099962452	0.548973011570351\\
7.97689957365268	0.549631867018567\\
7.99151814768085	0.550289760017723\\
8.00613672170902	0.550946691973752\\
8.02075529573718	0.551602664290536\\
8.03550996276884	0.552263771788177\\
8.0502646298005	0.552923904562982\\
8.06501929683216	0.55358306405206\\
8.07977396386382	0.554241251690401\\
8.09466853533245	0.554904696035232\\
8.10956310680107	0.5555671529437\\
8.12445767826969	0.556228623885449\\
8.13935224973832	0.556889110327942\\
8.15439073744049	0.557554981284343\\
8.16942922514267	0.558219851624104\\
8.18446771284484	0.558883722850867\\
8.19950620054702	0.559546596466016\\
8.2146928332504	0.56021498921204\\
8.22987946595378	0.560882367664931\\
8.24506609865717	0.561548733363889\\
8.26025273136055	0.562214087845777\\
8.27559197362809	0.562885103487416\\
8.29093121589562	0.563555090630928\\
8.30627045816316	0.56422405085274\\
8.32160970043069	0.564891985726863\\
8.33710627308224	0.565565731849691\\
8.35260284573378	0.566238434705508\\
8.36809941838533	0.566910095909767\\
8.38359599103687	0.567580717075419\\
8.39925489573158	0.568257308437132\\
8.41491380042629	0.568932841159955\\
8.430572705121	0.569607316900299\\
8.44623160981571	0.570280737311989\\
8.46205815432056	0.570960296519991\\
8.4778846988254	0.571638781071221\\
8.49371124333024	0.572316192665139\\
8.50953778783509	0.572992532998522\\
8.52553761739547	0.573675191401289\\
8.54153744695585	0.574356758435568\\
8.55753727651623	0.57503723584613\\
8.57353710607661	0.575716625374959\\
8.58971623619016	0.576402523955527\\
8.60589536630372	0.577087313709094\\
8.62207449641727	0.577770996428192\\
8.63825362653082	0.578453573902457\\
8.65461848330403	0.579142864430221\\
8.67098334007723	0.579831027866398\\
8.68734819685043	0.580518066053943\\
8.70371305362364	0.581203980832805\\
8.72018188980809	0.581893121513233\\
8.73665072599255	0.582581128193207\\
8.75311956217701	0.583268002738751\\
8.76958839836146	0.583953747012824\\
8.78614348346302	0.584641945309015\\
8.80269856856458	0.585329005229628\\
8.81925365366614	0.58601492865769\\
8.83580873876771	0.586699717473117\\
8.85245111606455	0.587386975358111\\
8.86909349336139	0.588073090433823\\
8.88573587065824	0.588758064600573\\
8.90237824795508	0.589441899755524\\
8.91910883699713	0.590128213359198\\
8.93583942603918	0.590813379679793\\
8.95257001508123	0.591497400635172\\
8.96930060412328	0.592180278139995\\
8.98612034063892	0.592865643621832\\
9.00294007715455	0.593549857305909\\
9.01975981367019	0.594232921127886\\
9.03657955018582	0.594914837020173\\
9.05348938408815	0.595599250473424\\
9.07039921799048	0.59628250757286\\
9.08730905189281	0.596964610272199\\
9.10421888579513	0.597645560521862\\
9.12121978296026	0.598329018032651\\
9.13822068012539	0.599011314591499\\
9.15522157729052	0.599692452170446\\
9.17222247445565	0.600372432738187\\
9.18931541587169	0.601054930336986\\
9.20640835728774	0.601736262343096\\
9.22350129870378	0.602416430747149\\
9.24059424011982	0.603095437536385\\
9.25778022278737	0.603776971220451\\
9.27496620545493	0.604457334627817\\
9.29215218812249	0.605136529767987\\
9.30933817079004	0.605814558647019\\
9.32661820779546	0.606495124368985\\
9.34389824480088	0.607174515086367\\
9.3611782818063	0.60785273282782\\
9.37845831881172	0.608529779618497\\
9.39583344002363	0.609209373299394\\
9.41320856123555	0.609887787203271\\
9.43058368244747	0.610565023378221\\
9.44795880365938	0.611241083868784\\
9.46543005570071	0.611919701381987\\
9.48290130774205	0.61259713430055\\
9.50037255978338	0.613273384692298\\
9.51784381182471	0.613948454621447\\
9.53541225878365	0.614626091805346\\
9.55298070574258	0.615302539531108\\
9.57054915270152	0.615977799886587\\
9.58811759966045	0.616651874955977\\
9.60578432346384	0.617328527612772\\
9.62345104726722	0.618003985901351\\
9.64111777107061	0.618678251929902\\
9.658784494874	0.619351327802897\\
9.67655059520445	0.620026991679656\\
9.6943166955349	0.620701456230879\\
9.71208279586536	0.621374723585403\\
9.72984889619581	0.622046795868289\\
9.74771549159443	0.622721466683076\\
9.76558208699305	0.623394933167005\\
9.78344868239167	0.624067197469877\\
9.80131527779029	0.624738261737663\\
9.81928350543996	0.625411935154677\\
9.83725173308963	0.626084399186813\\
9.8552199607393	0.626755656005165\\
9.87318818838897	0.627425707776933\\
9.89125920482513	0.628098379416707\\
9.90933022126129	0.628769836568142\\
9.92740123769745	0.629440081423953\\
9.94547225413361	0.630109116172903\\
9.96364723590061	0.63078078161958\\
9.9818222176676	0.631451227424221\\
9.99999719943459	0.632120455801503\\
10.0181721812016	0.632788468962089\\
10.0364523247657	0.633459123742623\\
10.0547324683298	0.634128553676454\\
10.0730126118939	0.634796761000568\\
10.0912927554579	0.635463747947871\\
10.1096792783833	0.636133387557564\\
10.1280658013086	0.636801797064071\\
10.146452324234	0.637468978727038\\
10.1648388471593	0.638134934801969\\
10.1833329877765	0.638803554673919\\
10.2018271283938	0.639470939133627\\
10.220321269011	0.640137090463764\\
10.2388154096282	0.640802010942788\\
10.257418428294	0.641469606475164\\
10.2760214469598	0.642135961232796\\
10.2946244656257	0.642801077521745\\
10.3132274842915	0.643464957643796\\
10.3319406634311	0.644131524182042\\
10.3506538425707	0.644796844528764\\
10.3693670217104	0.645460921013793\\
10.38808020085	0.646123755962612\\
10.4069048454226	0.646789288796792\\
10.4257294899953	0.647453569967548\\
10.4445541345679	0.648116601828867\\
10.4633787791406	0.648778386730316\\
10.4823162175897	0.649442881108986\\
10.5012536560388	0.650106118296296\\
10.5201910944879	0.650768100670784\\
10.539128532937	0.651428830606496\\
10.5581801172831	0.652092281719511\\
10.5772317016292	0.652754470056323\\
10.5962832859753	0.653415398020424\\
10.6153348703214	0.654075068010736\\
10.6345019773376	0.654737471009583\\
10.6536690843538	0.655398605589502\\
10.67283619137	0.656058474179351\\
10.6920032983861	0.65671707920334\\
10.7112873293394	0.657378429170762\\
10.7305713602926	0.65803850501777\\
10.7498553912458	0.658697309199004\\
10.769139422199	0.659354844164385\\
10.7885418041401	0.66001513613738\\
10.8079441860811	0.660674148228714\\
10.8273465680222	0.661331882919248\\
10.8467489499633	0.661988342685038\\
10.8662711360792	0.662647571644492\\
10.8857933221952	0.663305514900304\\
10.9053155083112	0.663962174959991\\
10.9248376944271	0.664617554326187\\
10.9444811647161	0.665275715197079\\
10.9641246350051	0.665932584480604\\
10.9837681052941	0.666588164711388\\
11.0034115755831	0.667242458419092\\
11.02317783773	0.66789954607663\\
11.0429440998769	0.668555336200278\\
11.0627103620238	0.669209831352231\\
11.0824766241707	0.669863034089637\\
11.102367213876	0.67051904334635\\
11.1222578035813	0.671173749058817\\
11.1421483932866	0.671827153817275\\
11.1620389829919	0.672479260206827\\
11.182055464688	0.673134185812152\\
11.202071946384	0.673787801797979\\
11.22208842808	0.674440110783076\\
11.2421049097761	0.67509111538098\\
11.2622488777205	0.675744952032319\\
11.282392845665	0.676397472922887\\
11.3025368136094	0.677048680700471\\
11.3226807815539	0.677698578007538\\
11.3429428732039	0.678350966941254\\
11.363204964854	0.679002035336835\\
11.3834670565041	0.679651785867249\\
11.4037291481542	0.680300221200061\\
11.4240073959591	0.680947859463439\\
11.444285643764	0.681594185760574\\
11.464563891569	0.682239202749196\\
11.4848421393739	0.682882913081662\\
11.5051410232308	0.683525972483089\\
11.5254399070878	0.684167727869661\\
11.5457387909448	0.684808181885691\\
11.5660376748017	0.685447337170132\\
11.5863613245615	0.686085973793988\\
11.6066849743212	0.686723313793196\\
11.627008624081	0.687359359800287\\
11.6473322738407	0.687994114442452\\
11.6676846007299	0.688628473266995\\
11.6880369276191	0.689261542336649\\
11.7083892545083	0.689893324273687\\
11.7287415813975	0.690523821695057\\
11.7491263024508	0.691154037694694\\
11.7695110235041	0.691782970325103\\
11.7898957445574	0.692410622199722\\
11.8102804656107	0.693036995926675\\
11.8307011258767	0.693663195061117\\
11.8511217861426	0.694288116760335\\
11.8715424464086	0.694911763630264\\
11.8919631066745	0.695534138271532\\
11.9124230970895	0.696156438303079\\
11.9328830875046	0.696777466410984\\
11.9533430779196	0.697397225194934\\
11.9738030683347	0.698015717249306\\
11.9943056417768	0.698634228472601\\
12.014808215219	0.699251472887808\\
12.0353107886612	0.699867453089542\\
12.0558133621033	0.700482171667107\\
12.0763616495731	0.701096997613647\\
12.0969099370428	0.701710561495272\\
12.1174582245125	0.702322865902637\\
12.1380065119823	0.702933913421087\\
12.1586035338403	0.703545151388758\\
12.1792005556983	0.704155131683917\\
12.1997975775563	0.704763856894321\\
12.2203945994143	0.705371329602406\\
12.2410432774846	0.705979071189982\\
12.2616919555548	0.70658555916624\\
12.2823406336251	0.707190796117044\\
12.3029893116954	0.707794784622928\\
12.3236924799435	0.708399116201844\\
12.3443956481917	0.709002197917184\\
12.3650988164398	0.709604032353875\\
12.385801984688	0.710204622091507\\
12.406562398648	0.710805625222867\\
12.4273228126081	0.711405381941103\\
12.4480832265682	0.712003894831128\\
12.4688436405282	0.7126011664725\\
12.4896639845707	0.713198918245834\\
12.5104843286133	0.713795426774107\\
12.5313046726558	0.714390694643093\\
12.5521250166983	0.714984724433201\\
12.5730079121072	0.715579297815395\\
12.5938908075161	0.716172630851772\\
12.614773702925	0.71676472612982\\
12.6356565983338	0.717355586231639\\
12.6566046110283	0.717947050392174\\
12.6775526237227	0.718537276849666\\
12.6985006364172	0.719126268194132\\
12.7194486491116	0.719714027010178\\
12.7404642942679	0.720302447546813\\
12.7614799394242	0.720889632778227\\
12.7824955845805	0.721475585297756\\
12.8035112297368	0.722060307693302\\
12.8245969793381	0.722645746930757\\
12.8456827289394	0.723229953026244\\
12.8667684785407	0.723812928577187\\
12.887854228142	0.724394676175546\\
12.9090125143938	0.724977193338384\\
12.9301708006455	0.725558479297708\\
12.9513290868973	0.726138536655769\\
12.972487373149	0.726717368009328\\
12.9937205946389	0.7272970194636\\
13.0149538161287	0.727875441436861\\
13.0361870376186	0.728452636536916\\
13.0574202591085	0.729028607366046\\
13.0787307850171	0.729605446798479\\
13.1000413109258	0.730181058264648\\
13.1213518368345	0.730755444378615\\
13.1426623627431	0.731328607748889\\
13.1640525369264	0.731902686335784\\
13.1854427111097	0.732475538270972\\
13.206832885293	0.733047166175467\\
13.2282230594763	0.733617572664692\\
13.2496952043377	0.734188939228709\\
13.271167349199	0.734759080262378\\
13.2926394940603	0.735327998394344\\
13.3141116389217	0.735895696247619\\
13.3356680588802	0.736464397397811\\
13.3572244788388	0.737031873952298\\
13.3787808987974	0.737598128548014\\
13.400337318756	0.738163163816227\\
13.4219803036483	0.738729244080902\\
13.4436232885406	0.739294100503781\\
13.4652662734329	0.739857735730746\\
13.4869092583252	0.74042015240197\\
13.5086410871513	0.740983654365256\\
13.5303729159775	0.741545933065398\\
13.5521047448036	0.742106991157871\\
13.5738365736297	0.742666831292398\\
13.5956595174591	0.743227795703641\\
13.6174824612885	0.743787537260208\\
13.639305405118	0.744346058627806\\
13.6611283489474	0.744903362466341\\
13.6830446722324	0.745461828304619\\
13.7049609955174	0.746019071531366\\
13.7268773188025	0.746575094823152\\
13.7487936420875	0.747129900850695\\
13.770805608124	0.747685905491843\\
13.7928175741606	0.748240687603479\\
13.8148295401972	0.748794249873659\\
13.8368415062338	0.749346594984539\\
13.858951377984	0.749900174244567\\
13.8810612497342	0.750452530900039\\
13.9031711214845	0.751003667651118\\
13.9252809932347	0.751553587192017\\
13.9474910360997	0.752104775417542\\
13.9697010789646	0.752654740810125\\
13.9919111218296	0.753203486082656\\
14.0141211646945	0.75375101394202\\
14.0364336494706	0.754299844101347\\
14.0587461342466	0.754847451049376\\
14.0810586190226	0.755393837512341\\
14.1033711037986	0.755939006210409\\
14.1257883079129	0.756485509938675\\
14.1482055120272	0.757030789930536\\
14.1706227161414	0.757574848926182\\
14.1930399202557	0.758117689659679\\
14.2155641320733	0.758661897376403\\
14.2380883438909	0.759204880687603\\
14.2606125557085	0.75974664234804\\
14.283136767526	0.760287185106288\\
14.3057702854045	0.760829126001644\\
14.3284038032829	0.761369841681185\\
14.3510373211614	0.761909334914856\\
14.3736708390398	0.762447608466351\\
14.3964159769655	0.762987310644649\\
14.4191611148912	0.763525786657901\\
14.4419062528169	0.764063039291854\\
14.4646513907425	0.764599071325936\\
14.4875104777278	0.765136561795311\\
14.510369564713	0.765672825013782\\
14.5332286516983	0.766207863783515\\
14.5560877386835	0.766741680900289\\
14.5790631229979	0.767276985675747\\
14.6020385073122	0.76781106197969\\
14.6250138916265	0.768343912631326\\
14.6479892759408	0.768875540443403\\
14.6710833264109	0.769408684576956\\
14.6941773768809	0.769940598885394\\
14.7172714273509	0.770471286205591\\
14.740365477821	0.771000749367888\\
14.7635805844029	0.771531756960568\\
14.7867956909849	0.772061533243275\\
14.8100107975669	0.772590081071177\\
14.8332259041488	0.773117403292836\\
14.8565644820063	0.773646297585508\\
14.8799030598637	0.774173958953421\\
14.9032416377212	0.774700390270678\\
14.9265802155787	0.775225594404693\\
14.9500447068131	0.775752397811269\\
14.9735091980476	0.776277966549572\\
14.996973689282	0.776802303513279\\
15.0204381805165	0.777325411589295\\
15.0440310551214	0.777850145710785\\
15.0676239297264	0.778373643292901\\
15.0912168043314	0.778895907249541\\
15.1148096789363	0.779416940487751\\
15.1385334385982	0.779939626188039\\
15.16225719826	0.780461073351062\\
15.1859809579218	0.780981284911599\\
15.2097047175837	0.781500263797487\\
15.2335618956234	0.78202092119055\\
15.2574190736632	0.782540337922533\\
15.281276251703	0.78305851694976\\
15.3051334297427	0.783575461221523\\
15.3291265950906	0.784094109746144\\
15.3531197604384	0.784611515360459\\
15.3771129257863	0.785127681043012\\
15.4011060911341	0.785642609765224\\
15.4252378496272	0.786159268197903\\
15.4493696081203	0.786674681346091\\
15.4735013666134	0.787188852211238\\
15.4976331251065	0.787701783787576\\
15.5219061221034	0.788216470282142\\
15.5461791191004	0.78872990899333\\
15.5704521160973	0.789242102946195\\
15.5947251130942	0.789753055158471\\
15.619142035055	0.790265787261213\\
15.6435589570157	0.790777268957167\\
15.6679758789765	0.791287503295701\\
15.6923928009372	0.79179649331876\\
15.7169563780926	0.79230728800473\\
15.741519955248	0.79281682953597\\
15.7660835324033	0.793325120986884\\
15.7906471095587	0.793832165424346\\
15.8153601179615	0.794341039118187\\
15.8400731263644	0.794848656784702\\
15.8647861347672	0.795355021524063\\
15.88949914317	0.795860136428804\\
15.9143644081624	0.796367105052455\\
15.9392296731548	0.796872814651148\\
15.9640949381471	0.797377268351577\\
15.9889602031395	0.797880469272684\\
16.0139806003288	0.798385548239221\\
16.039000997518	0.798889365057765\\
16.0640213947072	0.799391922882296\\
16.0890417918965	0.799893224858925\\
16.1142202516767	0.800396429136611\\
16.1393987114569	0.800898368017149\\
16.1645771712372	0.801399044682588\\
16.1897556310174	0.80189846230699\\
16.2150951374624	0.802399806367291\\
16.2404346439074	0.802899881654682\\
16.2657741503524	0.803398691380083\\
16.2911136567974	0.803896238746304\\
16.3166172552097	0.804395736683233\\
16.342120853622	0.804893962343768\\
16.3676244520343	0.805390918968516\\
16.3931280504465	0.805886609789849\\
16.4187988465862	0.806384275269901\\
16.4444696427259	0.806880664841437\\
16.4701404388656	0.807375781775597\\
16.4958112350053	0.807869629335148\\
16.5216524004798	0.808365475666966\\
16.5474935659542	0.808860042328208\\
16.5733347314287	0.809353332621401\\
16.5991758969032	0.80984534984057\\
16.6251906694304	0.810339389941788\\
16.6512054419577	0.810832146479239\\
16.677220214485	0.811323622787725\\
16.7032349870123	0.811813822193401\\
16.7294266758655	0.812306068655245\\
16.7556183647187	0.812797027527429\\
16.781810053572	0.813286702177935\\
16.8080017424252	0.813775095965952\\
16.8343737319713	0.814265561076892\\
16.8607457215174	0.814754734437818\\
16.8871177110636	0.815242619450822\\
16.9134897006097	0.815729219509055\\
16.9400454511807	0.816217915226202\\
16.9666012017517	0.816705314896844\\
16.9931569523226	0.817191421958139\\
17.0197127028936	0.817676239838154\\
17.0464557582069	0.818163177877685\\
17.0731988135202	0.818648815435857\\
17.0999418688336	0.819133155985883\\
17.1266849241469	0.819616202991724\\
17.1536189111414	0.82010139477521\\
17.1805528981359	0.8205852815021\\
17.2074868851304	0.82106786668268\\
17.2344208721249	0.821549153817813\\
17.2615495082445	0.822032610552342\\
17.2886781443641	0.822514757512127\\
17.3158067804836	0.822995598245569\\
17.3429354166032	0.823475136291477\\
17.3702625121198	0.823956868955122\\
17.3975896076364	0.824437286980412\\
17.424916703153	0.824916393954944\\
17.4522437986696	0.825394193456544\\
17.4797732620512	0.825874212835328\\
17.5073027254328	0.826352912563833\\
17.5348321888143	0.826830296269966\\
17.5623616521959	0.82730636757168\\
17.5900974943972	0.827784684273035\\
17.6178333365984	0.828261676160842\\
17.6455691787997	0.828737346904466\\
17.6733050210009	0.829211700163132\\
17.7012513604556	0.82968832463767\\
17.7291976999103	0.83016361898077\\
17.757144039365	0.830637586904448\\
17.7850903788197	0.831110232110381\\
17.8132514459017	0.831585174660349\\
17.8414125129837	0.832058781602923\\
17.8695735800657	0.832531056693998\\
17.8977346471477	0.83300200367893\\
17.9261147917276	0.833475274514547\\
17.9544949363076	0.833947204104845\\
17.9828750808876	0.834417796250874\\
18.0112552254676	0.834887054742935\\
18.0398589202161	0.835358663961591\\
18.0684626149646	0.8358289261311\\
18.097066309713	0.836297845098987\\
18.1256700044615	0.836765424701811\\
18.1545018545664	0.837235382369492\\
18.1833337046712	0.837703987013735\\
18.212165554776	0.838171242529911\\
18.2409974048809	0.838637152802196\\
18.2700621491713	0.839105468887325\\
18.2991268934616	0.839572425799905\\
18.328191637752	0.840038027484571\\
18.3572563820424	0.840502277874534\\
18.3865589068877	0.840968962385413\\
18.4158614317331	0.841434281394474\\
18.4451639565785	0.841898238897095\\
18.4744664814239	0.84236083887699\\
18.5040118216459	0.842825901781362\\
18.5335571618679	0.843289592669414\\
18.56310250209	0.8437519155888\\
18.592647842312	0.844212874575262\\
18.622441195706	0.844676325932137\\
18.6522345491	0.845138398566886\\
18.682027902494	0.845599096581039\\
18.711821255888	0.846058424063951\\
18.74186798467	0.84652027394062\\
18.771914713452	0.846980738192231\\
18.801961442234	0.847439820975852\\
18.8320081710161	0.847897526436106\\
18.8623138179204	0.848357785047797\\
18.8926194648247	0.848816650927462\\
18.922925111729	0.849274128289442\\
18.9532307586333	0.849730221335353\\
18.9838010512769	0.850188898993549\\
19.0143713439205	0.850646176601832\\
19.0449416365641	0.851102058433624\\
19.0755119292076	0.851556548749333\\
19.1063527942164	0.852013655952077\\
19.1371936592252	0.852469355568366\\
19.1680345242339	0.852923651932596\\
19.1988753892427	0.853376549365844\\
19.2299929618915	0.853832096809425\\
19.2611105345403	0.854286228903219\\
19.2922281071891	0.854738950044568\\
19.3233456798379	0.855190264617182\\
19.3547463174076	0.855644263253539\\
19.3861469549774	0.856096838541131\\
19.4175475925471	0.856547994942315\\
19.4489482301169	0.85699773690549\\
19.4806385226382	0.857450197950543\\
19.5123288151595	0.857901227402913\\
19.5440191076808	0.858350829792139\\
19.5757094002021	0.858799009633459\\
19.6076961882754	0.859249944666263\\
19.6396829763487	0.859699439607186\\
19.671669764422	0.860147499055218\\
19.7036565524953	0.860594127594696\\
19.7359469398787	0.861043548567671\\
19.768237327262	0.861491520683411\\
19.8005277146453	0.861938048612743\\
19.8328181020287	0.862383137011472\\
19.8654194755566	0.862831056350856\\
19.8980208490846	0.86327751778947\\
19.9306222226126	0.863722526072487\\
19.9632235961405	0.864166085929672\\
19.9961436388709	0.864612516526315\\
20.0290636816012	0.865057479887954\\
20.0619837243316	0.865500980836744\\
20.0949037670619	0.865943024179033\\
20.1281504830269	0.866387979523791\\
20.1613971989918	0.866831457994606\\
20.1946439149567	0.867273464493397\\
20.2278906309217	0.867714003905853\\
20.2614723645969	0.868157498132786\\
20.2950540982721	0.868599505527168\\
20.3286358319473	0.869040031073617\\
20.3622175656225	0.869479079740081\\
20.3961430262838	0.86992112771661\\
20.4300684869452	0.870361678566023\\
20.4639939476066	0.870800737358734\\
20.497919408268	0.871238309148028\\
20.5321976950164	0.871678926552216\\
20.5664759817649	0.872118036181143\\
20.6007542685134	0.872555643194299\\
20.6350325552618	0.872991752733568\\
20.6696731863085	0.873430956163509\\
20.7043138173553	0.873868640797213\\
20.738954448402	0.874304811886726\\
20.7735950794487	0.874739474665981\\
20.8086080231198	0.875177281744755\\
20.8436209667909	0.875613558612531\\
20.878633910462	0.876048310617603\\
20.9136468541332	0.876481543089617\\
20.949042565344	0.876917972610288\\
20.9844382765548	0.877352860088367\\
21.0198339877656	0.877786210972314\\
21.0552296989764	0.878218030691387\\
21.0910191546516	0.878653102716689\\
21.1268086103268	0.879086620426003\\
21.162598066002	0.879518589372134\\
21.1983875216773	0.879949015088103\\
21.234582265905	0.880382751135624\\
21.2707770101326	0.880814920124335\\
21.3069717543603	0.881245527715846\\
21.343166498588	0.881674579551368\\
21.3797786897193	0.882107002761174\\
21.4163908808506	0.882537845669585\\
21.453003071982	0.882967114051784\\
21.4896152631133	0.883394813661904\\
21.5266577262528	0.883825948984871\\
21.5637001893923	0.884255490230706\\
21.6007426525319	0.884683443293267\\
21.6377851156714	0.885109814044681\\
21.6752714017348	0.885539688459172\\
21.7127576877983	0.885967954450759\\
21.7502439738617	0.886394618037461\\
21.7877302599251	0.88681968521484\\
21.8256747161126	0.88724832801919\\
21.8636191723001	0.887675347443644\\
21.9015636284876	0.888100749636275\\
21.9395080846751	0.888524540721943\\
21.9779259262067	0.888951983777139\\
22.0163437677384	0.889377787838781\\
22.0547616092701	0.889801959191365\\
22.0931794508017	0.890224504095358\\
22.1320868496398	0.890650782189185\\
22.1709942484778	0.891075404968178\\
22.2099016473158	0.891498378860135\\
22.2488090461539	0.891919710267967\\
22.2882232273236	0.892344861480255\\
22.3276374084934	0.892768340291895\\
22.3670515896632	0.893190153281449\\
22.4064657708329	0.893610307001685\\
22.4464051256478	0.894034373161246\\
22.4863444804626	0.894456749005743\\
22.5262838352775	0.894877441272624\\
22.5662231900924	0.895296456672563\\
22.6067074061995	0.895719483886036\\
22.6471916223066	0.896140801969039\\
22.6876758384138	0.896560417826763\\
22.7281600545209	0.896978338336587\\
22.7692102626757	0.897400377569565\\
22.8102604708306	0.897820687873852\\
22.8513106789854	0.898239276332078\\
22.8923608871403	0.898656149997954\\
22.9339998357608	0.899077257801514\\
22.9756387843813	0.899496615802057\\
23.0172777330018	0.899914231270315\\
23.0589166816223	0.900330111446907\\
23.1011689388352	0.900750350795388\\
23.143421196048	0.901168818283735\\
23.1856734532609	0.901585521382544\\
23.2279257104737	0.90200046753101\\
23.2707449436263	0.902419196890665\\
23.3135641767789	0.902836137116597\\
23.3563834099314	0.903251295853233\\
23.399202643084	0.903664680712447\\
23.4425351018712	0.904081222197577\\
23.4858675606585	0.904495962611183\\
23.5292000194457	0.904908909740733\\
23.5725324782329	0.905320071340135\\
23.6163923812599	0.905734427244336\\
23.6602522842869	0.906146969767296\\
23.704112187314	0.906557706844931\\
23.747972090341	0.906966646378544\\
23.7923730104802	0.907378807335544\\
23.8367739306194	0.907789142316818\\
23.8811748507586	0.908197659411747\\
23.9255757708978	0.908604366673998\\
23.9705318120247	0.909014323067755\\
24.0154878531517	0.909422440596465\\
24.0604438942787	0.909828727508231\\
24.1053999354057	0.910233192014286\\
24.1509257391711	0.910640933779936\\
24.1964515429365	0.911046823487551\\
24.2419773467019	0.911450869549452\\
24.2875031504673	0.911853080339882\\
24.3336139249026	0.91225859696003\\
24.3797246993379	0.912662248016185\\
24.4258354737732	0.913064042090637\\
24.4719462482085	0.913463987726335\\
24.518657797687	0.913867268222069\\
24.5653693471654	0.914268669325121\\
24.6120808966439	0.914668199793778\\
24.6587924461223	0.91506586834566\\
24.7061212033434	0.91546690126722\\
24.7534499605645	0.915866040634437\\
24.8007787177855	0.916263295387912\\
24.8481074750066	0.916658674426182\\
24.8960705351928	0.917057447843455\\
24.944033595379	0.917454313200991\\
24.9919966555653	0.917849279628326\\
25.0399597157515	0.918242356211471\\
25.0885748730688	0.918638857702858\\
25.1371900303861	0.919033436274081\\
25.1858051877034	0.919426101250559\\
25.2344203450208	0.919816861912655\\
25.2837061315614	0.920211078553555\\
25.332991918102	0.920603357046954\\
25.3822777046427	0.920993706921452\\
25.4315634911833	0.92138213765898\\
25.4815392186928	0.921774056009569\\
25.5315149462023	0.922164020605918\\
25.5814906737117	0.922552041187475\\
25.6314664012212	0.922938127445324\\
25.6821522057685	0.923327733537302\\
25.7328380103157	0.923715369875757\\
25.783523814863	0.924101046419051\\
25.8342096194102	0.924484773075393\\
25.8856265092156	0.924872052397365\\
25.9370433990211	0.925257345560162\\
25.9884602888265	0.92564066274954\\
26.0398771786319	0.926022014099226\\
26.0920470854784	0.926406951581787\\
26.144216992325	0.926789886078582\\
26.1963868991715	0.927170828011699\\
26.2485568060181	0.927549787749215\\
26.3015026403881	0.92793236774836\\
26.3544484747581	0.928312927498867\\
26.4073943091281	0.928691477668572\\
26.4603401434981	0.929068028869206\\
26.5140858539001	0.929448235148961\\
26.5678315643021	0.929826403464744\\
26.6215772747041	0.930202544740068\\
26.6753229851061	0.930576669840135\\
26.7298936218998	0.930954485553787\\
26.7844642586936	0.931330245118604\\
26.8390348954873	0.931703959724256\\
26.893605532281	0.932075640499773\\
26.9490273164448	0.932451048169822\\
27.0044491006085	0.932824381018626\\
27.0598708847722	0.933195650513084\\
27.1152926689359	0.933564868056994\\
27.1715930662283	0.933937849553648\\
27.2278934635207	0.93430873705007\\
27.2841938608131	0.934677542302095\\
27.3404942581055	0.935044276999851\\
27.3972810276665	0.935412094221127\\
27.4540677972275	0.935777828646817\\
27.5108545667884	0.936141492070577\\
27.5676413363494	0.936503096219573\\
27.6246835621125	0.936864265616938\\
27.6817257878755	0.937223380688642\\
27.7387680136386	0.93758045311931\\
27.7958102394016	0.937935494527395\\
27.8531117250382	0.938290116375949\\
27.9104132106748	0.938642711999688\\
27.9677146963114	0.938993292975634\\
28.025016181948	0.939341870814949\\
28.0825792753208	0.939690034739289\\
28.1401423686935	0.940036200281785\\
28.1977054620663	0.940380378912348\\
28.2552685554391	0.940722582035345\\
28.3130956457549	0.941064376892938\\
28.3709227360708	0.941404200954318\\
28.4287498263866	0.941742065582789\\
28.4865769167025	0.942077982076427\\
28.5446704260258	0.942413495902807\\
28.602763935349	0.942747066262467\\
28.6608574446723	0.943078704412607\\
28.7189509539956	0.943408421545513\\
28.7773133374	0.943737741555561\\
28.8356757208045	0.944065145173473\\
28.8940381042089	0.94439064355085\\
28.9524004876134	0.944714247774688\\
29.0110342352743	0.945037460374852\\
29.0696679829352	0.945358783403692\\
29.1283017305961	0.945678227907708\\
29.186935478257	0.945995804869109\\
29.2458431140456	0.94631299564963\\
29.3047507498342	0.946628323427129\\
29.3636583856228	0.946941799143509\\
29.4225660214114	0.947253433676696\\
29.4817501049045	0.947564687423223\\
29.5409341883977	0.947874104483702\\
29.6001182718908	0.948181695695939\\
29.659302355384	0.948487471834078\\
29.7187654815208	0.948792872526126\\
29.7782286076576	0.94909646259901\\
29.8376917337944	0.949398252786941\\
29.8971548599312	0.949698253760785\\
29.9568996607722	0.949997884581365\\
30.0166444616132	0.950295730600747\\
30.0763892624543	0.950591802450053\\
30.1361340632953	0.950886110697368\\
30.1961632086509	0.951180054036229\\
30.2561923540065	0.951472238144121\\
30.316221499362	0.951762673549577\\
30.3762506447176	0.952051370718407\\
30.436566841334	0.952339708168386\\
30.4968830379503	0.952626311711157\\
30.5571992345667	0.952911191773166\\
30.617515431183	0.953194358718447\\
30.6781214246657	0.953477171085533\\
30.7387274181483	0.953758274623898\\
30.799333411631	0.954037679658403\\
30.8599394051136	0.954315396451809\\
30.9208379817972	0.95459276376404\\
30.9817365584807	0.954868447081897\\
31.0426351351642	0.955142456629159\\
31.1035337118477	0.955414802567817\\
31.1647276970968	0.955686804067549\\
31.2259216823458	0.955957146164391\\
31.2871156675948	0.956225838981536\\
31.3483096528438	0.956492892580709\\
31.4098019130839	0.956759606734562\\
31.4712941733241	0.957024685835339\\
31.5327864335642	0.957288139906155\\
31.5942786938043	0.95754997890897\\
31.656072138367	0.957811483416066\\
31.7178655829298	0.958071376979384\\
31.7796590274925	0.958329669522459\\
31.8414524720552	0.958586370907987\\
31.903550051626	0.9588427426935\\
31.9656476311968	0.959097527405283\\
32.0277452107676	0.959350734867796\\
32.0898427903383	0.959602374844971\\
32.1522474993513	0.95985369007089\\
32.2146522083642	0.960103441855069\\
32.2770569173772	0.960351639923392\\
32.3394616263901	0.960598293941532\\
32.4021765035977	0.960844628009248\\
32.4648913808053	0.961089422030356\\
32.527606258013	0.961332685632669\\
32.5903211352206	0.9615744283841\\
32.6533492654047	0.961815855941436\\
32.7163773955889	0.962055766611603\\
32.7794055257731	0.962294169924846\\
32.8424336559572	0.962531075351817\\
32.9057781690946	0.962767670290026\\
32.969122682232	0.963002771266063\\
33.0324671953693	0.9632363877131\\
33.0958117085067	0.963468529005038\\
33.1594757818323	0.963700364467236\\
33.2231398551579	0.963930728659003\\
33.2868039284835	0.964159630916947\\
33.3504680018091	0.964387080518713\\
33.4144548595075	0.964614228899176\\
33.478441717206	0.964839928468949\\
33.5424285749044	0.965064188468572\\
33.6064154326028	0.965287018079938\\
33.6707283499096	0.965509551039218\\
33.7350412672163	0.965730657416649\\
33.7993541845231	0.96595034635721\\
33.8636671018298	0.966168626947543\\
33.9283094023156	0.9663866154029\\
33.9929517028013	0.96660319927563\\
34.057594003287	0.96681838761565\\
34.1222363037727	0.967032189414852\\
34.1872113622324	0.967245703550939\\
34.2521864206922	0.967457834875191\\
34.3171614791519	0.967668592342963\\
34.3821365376116	0.9678779848519\\
34.447447781095	0.968087094124439\\
34.5127590245783	0.968294842128696\\
34.5780702680616	0.968501237725966\\
34.643381511545	0.968706289720149\\
34.7090324201268	0.968911062858737\\
34.7746833287086	0.969114496046561\\
34.8403342372905	0.969316598051359\\
34.9059851458723	0.969517377583786\\
34.9719792533529	0.969717882594549\\
35.0379733608335	0.969917068747253\\
35.1039674683141	0.970114944716579\\
35.1699615757948	0.970311519120441\\
35.2363024710846	0.970507823290817\\
35.3026433663745	0.970702829472232\\
35.3689842616644	0.970896546246816\\
35.4353251569542	0.971088982140241\\
35.5020164861975	0.971281152045662\\
35.5687078154407	0.97147204460879\\
35.6353991446839	0.971661668319701\\
35.7020904739271	0.971850031612327\\
35.76913594038	0.972038133116107\\
35.8361814068328	0.97222497770305\\
35.9032268732857	0.972410573771682\\
35.9702723397385	0.972594929664696\\
36.0376757049608	0.972779027921996\\
36.1050790701831	0.972961889467928\\
36.1724824354054	0.973143522609969\\
36.2398858006277	0.973323935600073\\
36.3076508877251	0.973504095067221\\
36.3754159748225	0.973683037809633\\
36.4431810619199	0.973860772044235\\
36.5109461490173	0.974037305932747\\
36.5790768402747	0.974213590361149\\
36.6472075315322	0.974388677833888\\
36.7153382227897	0.974562576477844\\
36.7834689140472	0.974735294365002\\
36.8519691566203	0.974907766816734\\
36.9204693991935	0.975079061865445\\
36.9889696417667	0.975249187548467\\
37.0574698843398	0.975418151848554\\
37.1263436888854	0.975586874692803\\
37.1952174934311	0.975754439471463\\
37.2640912979767	0.975920854132822\\
37.3329651025223	0.976086126570901\\
37.4022165438372	0.976251161486221\\
37.4714679851522	0.97641505745942\\
37.5407194264671	0.976577822350242\\
37.609970867782	0.976739463964479\\
37.6796040887873	0.976900871948964\\
37.7492373097926	0.977061159901993\\
37.8188705307979	0.977220335595267\\
37.8885037518033	0.97737840674685\\
37.9585229632751	0.977536248117669\\
38.0285421747469	0.977692988156109\\
38.0985613862187	0.977848634546334\\
38.1685805976905	0.978003194919176\\
38.2389900798134	0.978157529316668\\
38.3093995619363	0.97831078087049\\
38.3798090440592	0.978462957177763\\
38.4502185261821	0.978614065782596\\
38.5209622984945	0.97876482405941\\
38.591706070807	0.97891451957914\\
38.6624498431194	0.979063159833255\\
38.7331936154318	0.9792107522607\\
38.8042656863826	0.979357981934027\\
38.8753377573333	0.979504168925473\\
38.9464098282841	0.97964932061898\\
39.0174818992348	0.979793444346482\\
39.0888909542047	0.979937223483164\\
39.1603000091745	0.980079979563969\\
39.2317090641443	0.980221719868078\\
39.3031181191142	0.980362451623164\\
39.3748725127202	0.980502855329107\\
39.4466269063262	0.980642255182849\\
39.5183812999322	0.98078065836135\\
39.5901356935382	0.980918071990538\\
39.662243568804	0.981055172822403\\
39.7343514440698	0.981191288605285\\
39.8064593193355	0.981326426416278\\
39.8785671946013	0.981460593281909\\
39.95103651206	0.981594461444339\\
40.0235058295186	0.981727362980328\\
40.0959751469772	0.981859304869294\\
40.1684444644358	0.981990294040539\\
40.2412830193083	0.982120997545454\\
40.3141215741807	0.982250752483959\\
40.3869601290532	0.982379565739836\\
40.4597986839256	0.982507444147189\\
40.5330141309314	0.982635048974475\\
40.6062295779373	0.982761722949337\\
40.6794450249431	0.98288747286183\\
40.752660471949	0.983012305452754\\
40.8262603410054	0.983136875682154\\
40.8998602100619	0.983260532442206\\
40.9734600791183	0.983383282431019\\
41.0470599481748	0.983505132297862\\
41.1210516636287	0.983626730236765\\
41.1950433790826	0.983747431772293\\
41.2690350945364	0.983867243512279\\
41.3430268099903	0.983986172016119\\
41.4174177047231	0.984104858308912\\
41.4918085994559	0.9842226649599\\
41.5661994941887	0.984339598488208\\
41.6405903889215	0.984455665364917\\
41.7153877210576	0.984571499098148\\
41.7901850531936	0.984686469658357\\
41.8649823853296	0.984800583477431\\
41.9397797174656	0.984913846939604\\
42.0149906816255	0.985026885730569\\
42.0902016457854	0.985139077535237\\
42.1654126099454	0.985250428699647\\
42.2406235741053	0.985360945522563\\
42.316255315397	0.985471245605446\\
42.3918870566888	0.98558071461655\\
42.4675187979805	0.985689358817377\\
42.5431505392722	0.985797184422523\\
42.6192101664345	0.985904800726595\\
42.6952697935968	0.986011601610253\\
42.771329420759	0.986117593251697\\
42.8473890479213	0.986222781782591\\
42.9238836429179	0.986327767999453\\
43.0003782379145	0.986431954192499\\
43.0768728329111	0.986535346457802\\
43.1533674279077	0.986637950845259\\
43.2303040564086	0.986740359492417\\
43.3072406849096	0.986841983265343\\
43.3841773134105	0.986942828179097\\
43.4611139419115	0.987042900202911\\
43.5384996649489	0.987142782684106\\
43.6158853879863	0.987241895200775\\
43.6932711110238	0.987340243688021\\
43.7706568340612	0.987437834035468\\
43.8484987166093	0.987535240692425\\
43.9263405991575	0.987631892061313\\
44.0041824817057	0.987727793998287\\
44.0820243642538	0.987822952314364\\
44.1603294838729	0.987917932474891\\
44.2386346034919	0.988012171796713\\
44.3169397231109	0.988105676057998\\
44.39524484273	0.988198450992113\\
44.4740202988846	0.98829105301548\\
44.5527957550393	0.988382928428116\\
44.631571211194	0.988474082931118\\
44.7103466673487	0.98856452218112\\
44.7895995895968	0.988654793498712\\
44.8688525118449	0.988744352217459\\
44.948105434093	0.98883320396227\\
45.0273583563411	0.988921354313921\\
45.1070959123988	0.989009341466776\\
45.1868334684566	0.989096629821524\\
45.2665710245144	0.989183224927726\\
45.3463085805721	0.989269132291144\\
45.4265379845256	0.98935488096489\\
45.5067673884792	0.989439944434708\\
45.5869967924327	0.989524328175629\\
45.6672261963862	0.989608037619205\\
45.7479547148514	0.989691592674259\\
45.8286832333165	0.98977447591732\\
45.9094117517817	0.989856692749669\\
45.9901402702469	0.98993824852943\\
46.0713752310891	0.990019654031721\\
46.1526101919314	0.990100400915716\\
46.2338451527736	0.990180494509701\\
46.3150801136159	0.990259940099121\\
46.3968289125376	0.990339239346351\\
46.4785777114593	0.990417892974502\\
46.560326510381	0.990495906239595\\
46.6420753093027	0.990573284355128\\
46.7243454163926	0.990650519901154\\
46.8066155234824	0.990727122636375\\
46.8888856305723	0.990803097745251\\
46.9711557376621	0.990878450370034\\
47.0539547042701	0.990953664047846\\
47.136753670878	0.991028257535496\\
47.2195526374859	0.991102235946569\\
47.3023516040939	0.991175604352756\\
47.3856870701744	0.991248837296203\\
47.4690225362549	0.991321462485612\\
47.5523580023354	0.99139348496436\\
47.6356934684159	0.99146490973424\\
47.7195731680798	0.991536202396623\\
47.8034528677438	0.991606899559522\\
47.8873325674077	0.991677006196753\\
47.9712122670716	0.991746527240857\\
48.0556440359198	0.99181591941378\\
48.1400758047679	0.991884728162972\\
48.2245075736161	0.991952958393322\\
48.3089393424642	0.99202061496875\\
48.3939311234603	0.992088145798398\\
48.4789229044563	0.9921551051039\\
48.5639146854524	0.992221497721833\\
48.6489064664485	0.992287328448112\\
48.7344663172648	0.992353036451585\\
48.820026168081	0.992418184656807\\
48.9055860188973	0.992482777832648\\
48.9911458697136	0.99254682070762\\
49.0772819670679	0.992610743785811\\
49.1634180644223	0.99267411862034\\
49.2495541617767	0.992736949912961\\
49.3356902591311	0.992799242325371\\
49.4224109073406	0.99286141777818\\
49.5091315555501	0.992923056372848\\
49.5958522037596	0.992984162744594\\
49.6825728519691	0.993044741488878\\
49.769886487838	0.993105206026643\\
49.857200123707	0.993165144924877\\
49.944513759576	0.99322456275283\\
50.0318273954449	0.993283464040295\\
50.1197425956162	0.993342253796243\\
50.2076577957875	0.993400528966403\\
50.2955729959588	0.993458294054616\\
50.3834881961302	0.993515553525567\\
50.4720136825001	0.993572704066097\\
50.5605391688701	0.993629350911687\\
50.64906465524	0.993685498501319\\
50.73759014161	0.993741151235122\\
50.8267347882915	0.993796697570463\\
50.915879434973	0.993851750940703\\
51.0050240816545	0.99390631572051\\
51.094168728336	0.993960396245995\\
51.1839415683769	0.994014372839985\\
51.2737144084178	0.994067867039515\\
51.3634872484587	0.994120883155474\\
51.4532600884996	0.994173425460492\\
51.5436703203087	0.99422586623954\\
51.6340805521178	0.994277835037319\\
51.7244907839268	0.994329336101468\\
51.8149010157359	0.994380373641664\\
51.9059580087448	0.994431312002739\\
51.9970150017537	0.994481788639983\\
52.0880719947626	0.994531807738308\\
52.1791289877716	0.994581373444961\\
52.270842290771	0.994630842264608\\
52.3625555937704	0.994679859463735\\
52.4542688967698	0.994728429165044\\
52.5459821997693	0.994776555453869\\
52.6383615465513	0.994824587095146\\
52.7307408933333	0.994872177066676\\
52.8231202401153	0.994919329429464\\
52.9154995868973	0.994966048207442\\
53.0085549039533	0.995012674527573\\
53.1016102210094	0.995058868977728\\
53.1946655380654	0.995104635557724\\
53.2877208551214	0.995149978230597\\
53.3814622684928	0.995195230587709\\
53.4752036818643	0.995240060725112\\
53.5689450952358	0.995284472581939\\
53.6626865086073	0.995328470060836\\
53.7571243510415	0.995372379320474\\
53.8515621934757	0.995415875862629\\
53.9460000359099	0.99545896356625\\
54.0404378783442	0.995501646274091\\
54.1355826971555	0.99554424281569\\
54.2307275159668	0.995586435995416\\
54.325872334778	0.995628229632528\\
54.4210171535893	0.995669627510384\\
54.5168797176294	0.995710941232974\\
54.6127422816695	0.99575186080408\\
54.7086048457095	0.995792389983767\\
54.8044674097496	0.995832532496491\\
54.9010587178797	0.995872592824755\\
54.9976500260098	0.995912268068073\\
55.0942413341398	0.995951561927803\\
55.1908326422699	0.99599047806999\\
55.2881639308243	0.996029313959565\\
55.3854952193786	0.996067773688222\\
55.4828265079329	0.996105860899101\\
55.5801577964873	0.996143579200323\\
55.6782405480441	0.99618121914336\\
55.776323299601	0.996218491708313\\
55.8744060511579	0.996255400480594\\
55.9724888027147	0.996291949010886\\
56.0713347542828	0.996328421041078\\
56.1701807058509	0.996364534336132\\
56.2690266574189	0.99640029242421\\
56.367872608987	0.996435698799039\\
56.4674937609263	0.996471030496694\\
56.5671149128656	0.99650601196354\\
56.6667360648049	0.996540646670975\\
56.7663572167441	0.996574938056252\\
56.8667658411891	0.996609156552925\\
56.9671744656341	0.996643033185773\\
57.0675830900791	0.996676571369907\\
57.1679917145241	0.996709774486586\\
57.2692003658787	0.996742906470219\\
57.3704090172333	0.996775704820906\\
57.4716176685879	0.996808172897952\\
57.5728263199425	0.996840314027096\\
57.6748478434472	0.99687238574624\\
57.7768693669519	0.996904131928448\\
57.8788908904565	0.996935555877694\\
57.9809124139612	0.996966660864679\\
58.0837599560178	0.996997698133116\\
58.1866074980744	0.997028417826981\\
58.2894550401309	0.997058823195391\\
58.3923025821875	0.997088917454484\\
58.4959896007833	0.997118945655657\\
58.5996766193791	0.997148664112181\\
58.7033636379749	0.997178076018795\\
58.8070506565707	0.997207184537544\\
58.911590931702	0.997236228628782\\
59.0161312068333	0.997264970674058\\
59.1206714819646	0.9972934138142\\
59.2252117570959	0.99732156115764\\
59.3306194021433	0.997349645674474\\
59.4360270471907	0.997377435713983\\
59.541434692238	0.997404934363558\\
59.6468423372854	0.997432144678488\\
59.7351317529641	0.997454716334467\\
59.8234211686427	0.997477089583847\\
59.9117105843214	0.99749926617054\\
60	0.997521247823233\\
};
\addlegendentry{Rimossi}

\end{axis}
\end{tikzpicture}%}
}
\caption[Sperimentazione in MATLAB relativo al grafo~\ref{fig::lollipop}]{Divisione in classe dei singoli nodi per il grafo~\ref{fig::lollipop}.\\
	Per ottenere i grafici abbiamo risolto numericamente, usando MATLAB, i due problemi di Cauchy con condizioni iniziali di stati puri~\eqref{statipuri_lollipop}. Abbiamo inoltre supposto l'indipendenza statica delle condizioni iniziali.\\
	Per la sperimentazione abbiamo usato come parametri $\tau = 0.3$ e $\gamma = 0.1$.}	\label{fig::plot_lollipop}
\end{figure}

\begin{figure}[!htb]
	\subfloat[][Nodo 1]
{\resizebox{0.45\textwidth}{!}{% This file was created by matlab2tikz.
%
%The latest updates can be retrieved from
%  http://www.mathworks.com/matlabcentral/fileexchange/22022-matlab2tikz-matlab2tikz
%where you can also make suggestions and rate matlab2tikz.
%
\definecolor{mycolor1}{rgb}{0.00000,0.44700,0.74100}%
\definecolor{mycolor2}{rgb}{0.85000,0.32500,0.09800}%
%
\begin{tikzpicture}

\begin{axis}[%
width=6.028in,
height=4.754in,
at={(1.011in,0.642in)},
scale only axis,
xmin=0,
xmax=60,
xlabel style={font=\color{white!15!black}},
xlabel={T},
ymode=log,
ymin=1e-16,
ymax=1e-09,
yminorticks=true,
ylabel style={font=\color{white!15!black}},
ylabel={Errore assoluto},
axis background/.style={fill=white},
legend style={legend cell align=left, align=left, draw=white!15!black}
]
\addplot [color=mycolor1, line width=2.0pt]
  table[row sep=crcr]{%
0	0\\
0.000265404780368998	0\\
0.000530809560737996	0\\
0.000796214341106994	0\\
0.00106161912147599	0\\
0.00238864302332098	0\\
0.00371566692516597	0\\
0.00504269082701096	0\\
0.00636971472885596	0\\
0.0130048342380809	0\\
0.0196399537473059	0\\
0.0262750732565308	0\\
0.0329101927657558	0\\
0.0481555938880245	0\\
0.0634009950102933	0\\
0.0786463961325621	0\\
0.0938917972548308	0\\
0.112426119267152	0\\
0.130960441279472	0\\
0.149494763291793	0\\
0.168029085304114	0\\
0.18883016427751	0\\
0.209631243250907	0\\
0.230432322224303	0\\
0.2512334011977	0\\
0.273784345742275	0\\
0.29633529028685	0\\
0.318886234831425	0\\
0.341437179376	0\\
0.365441427620669	0\\
0.389445675865339	0\\
0.413449924110009	0\\
0.437454172354678	0\\
0.462718995465789	0\\
0.487983818576899	0\\
0.513248641688009	0\\
0.538513464799119	0\\
0.564903949168048	0\\
0.591294433536978	0\\
0.617684917905907	0\\
0.644075402274836	0\\
0.671492335525317	0\\
0.698909268775797	0\\
0.726326202026278	0\\
0.753743135276759	0\\
0.782111039573039	0\\
0.810478943869319	0\\
0.838846848165599	0\\
0.867214752461879	0\\
0.896474834868797	0\\
0.925734917275715	0\\
0.954994999682632	0\\
0.98425508208955	0\\
1.01436081133407	0\\
1.04446654057859	0\\
1.0745722698231	0\\
1.10467799906762	0\\
1.13559219349808	0\\
1.16650638792855	0\\
1.19742058235901	0\\
1.22833477678948	0\\
1.26002761487183	0\\
1.29172045295419	0\\
1.32341329103654	0\\
1.35510612911889	0\\
1.38755375131887	0\\
1.42000137351885	0\\
1.45244899571883	0\\
1.48489661791881	0\\
1.51811813384306	0\\
1.5513396497673	0\\
1.58456116569155	0\\
1.61778268161579	0\\
1.65178046131382	0\\
1.68577824101184	0\\
1.71977602070987	0\\
1.7537738004079	0\\
1.78853445564131	0\\
1.82329511087471	0\\
1.85805576610812	0\\
1.89281642134153	0\\
1.92833097625931	0\\
1.96384553117709	0\\
1.99936008609487	0\\
2.03487464101265	0\\
2.07113752249011	0\\
2.10740040396758	0\\
2.14366328544504	0\\
2.1799261669225	0\\
2.21693501810426	0\\
2.25394386928602	0\\
2.29095272046778	0\\
2.32796157164954	0\\
2.3657171313366	0\\
2.40347269102366	0\\
2.44122825071071	0\\
2.47898381039777	0\\
2.51748987308115	0\\
2.55599593576452	0\\
2.5945019984479	0\\
2.63300806113127	0\\
2.67227151262563	0\\
2.71153496411999	0\\
2.75079841561434	0\\
2.7900618671087	0\\
2.83009279630781	0\\
2.87012372550692	0\\
2.91015465470604	0\\
2.95018558390515	0\\
2.99099747582647	0\\
3.03180936774779	0\\
3.07262125966911	0\\
3.11343315159043	0\\
3.15504317204847	0\\
3.1966531925065	0\\
3.23826321296453	0\\
3.27987323342256	0\\
3.32230262192904	0\\
3.36473201043552	0\\
3.407161398942	0\\
3.44959078744848	0\\
3.49286538309579	0\\
3.5361399787431	0\\
3.57941457439041	0\\
3.62268917003772	0\\
3.66684010640848	0\\
3.71099104277924	0\\
3.75514197915	0\\
3.79929291552076	0\\
3.84435753680729	0\\
3.88942215809383	0\\
3.93448677938037	0\\
3.9795514006669	0\\
4.02557447530949	0\\
4.07159754995208	0\\
4.11762062459467	0\\
4.16364369923726	0\\
4.21041243310513	0\\
4.257181166973	0\\
4.30394990084087	0\\
4.35071863470874	0\\
4.39747966483664	0\\
4.44424069496454	0\\
4.49100172509244	0\\
4.53776275522034	0\\
4.58452386492069	0\\
4.63128497462103	0\\
4.67804608432138	0\\
4.72480719402173	0\\
4.77156830290013	0\\
4.81832941177852	0\\
4.86509052065691	0\\
4.91185162953531	0\\
4.95861273842228	0\\
5.00537384730925	0\\
5.05213495619622	0\\
5.09889606508318	0\\
5.14565717397096	0\\
5.19241828285873	0\\
5.2391793917465	0\\
5.28594050063428	0\\
5.33270160952073	0\\
5.37946271840718	0\\
5.42622382729363	0\\
5.47298493618008	0\\
5.51974604506734	0\\
5.56650715395461	0\\
5.61326826284187	0\\
5.66002937172913	0\\
5.70679048061676	0\\
5.7535515895044	0\\
5.80031269839203	0\\
5.84707380727967	0\\
5.89383491616663	0\\
5.9405960250536	0\\
5.98735713394057	0\\
6.03411824282754	0\\
6.08087935171415	0\\
6.12764046060076	0\\
6.17440156948738	0\\
6.22116267837399	0\\
6.26792378726144	0\\
6.3146848961489	0\\
6.36144600503635	0\\
6.4082071139238	0\\
6.45496822281113	0\\
6.50172933169847	0\\
6.5484904405858	0\\
6.59525154947314	0\\
6.64201265836067	0\\
6.68877376724821	0\\
6.73553487613575	0\\
6.78229598502329	0\\
6.82916482319815	0\\
6.87603366137301	0\\
6.92290249954786	0\\
6.96977133772272	0\\
7.01788255937152	0\\
7.06599378102033	0\\
7.11410500266913	0\\
7.16221622431793	0\\
7.21159336850751	0\\
7.26097051269708	0\\
7.31034765688666	0\\
7.35972480107623	0\\
7.41043619055615	0\\
7.46114758003607	0\\
7.511858969516	0\\
7.56257035899592	0\\
7.61468820401402	0\\
7.66680604903213	0\\
7.71892389405023	0\\
7.77104173906833	0\\
7.8246442387161	0\\
7.87824673836387	0\\
7.93184923801164	0\\
7.98545173765941	0\\
8.04062368192306	0\\
8.09579562618671	0\\
8.15096757045037	0\\
8.20613951471402	0\\
8.26297303935407	0\\
8.31980656399412	0\\
8.37664008863416	0\\
8.43347361327421	0\\
8.49206906178453	0\\
8.55066451029485	0\\
8.60925995880517	0\\
8.66785540731549	0\\
8.7283223218246	0\\
8.7887892363337	0\\
8.84925615084281	0\\
8.90972306535191	0\\
8.97139693657501	0\\
9.03307080779811	0\\
9.09474467902121	0\\
9.15641855024431	0\\
9.21815611731334	0\\
9.27989368438236	0\\
9.34163125145139	0\\
9.40336881852041	0\\
9.46537482585172	0\\
9.52738083318303	0\\
9.58938684051434	0\\
9.65139284784564	0\\
9.71505322212012	0\\
9.77871359639459	0\\
9.84237397066907	0\\
9.90603434494354	0\\
9.97140380506578	0\\
10.036773265188	0\\
10.1021427253103	0\\
10.1675121854325	0\\
10.2347010682591	0\\
10.3018899510858	0\\
10.3690788339124	0\\
10.436267716739	0\\
10.505392885882	0\\
10.574518055025	0\\
10.643643224168	0\\
10.712768393311	0\\
10.7839560546811	0\\
10.8551437160513	0\\
10.9263313774214	0\\
10.9975190387916	0\\
11.07090572606	0\\
11.1442924133284	0\\
11.2176791005968	0\\
11.2910657878652	0\\
11.3667995476522	0\\
11.4425333074392	0\\
11.5182670672261	0\\
11.5940008270131	0\\
11.6722425439244	0\\
11.7504842608358	0\\
11.8287259777471	0\\
11.9069676946584	0\\
11.9878925800102	0\\
12.068817465362	0\\
12.1497423507139	0\\
12.2306672360657	0\\
12.3144665036616	0\\
12.3982657712575	0\\
12.4820650388535	0\\
12.5658643064494	0\\
12.652747062149	0\\
12.7396298178485	0\\
12.826512573548	0\\
12.9133953292476	0\\
13.0035907048582	0\\
13.0937860804688	0\\
13.1839814560794	0\\
13.27417683169	0\\
13.3679363981715	0\\
13.461695964653	0\\
13.5554555311345	0\\
13.649215097616	0\\
13.7468155979555	0\\
13.8444160982949	0\\
13.9420165986344	0\\
14.0396170989738	0\\
14.1413635477079	0\\
14.2431099964419	0\\
14.344856445176	0\\
14.44660289391	0\\
14.5528320987134	0\\
14.6590613035167	0\\
14.7652905083201	0\\
14.8715197131234	0\\
14.9753724014636	0\\
15.0792250898038	0\\
15.183077778144	0\\
15.2869304664842	0\\
15.3888306269013	0\\
15.4907307873185	0\\
15.5926309477356	0\\
15.6945311081528	0\\
15.7977993001492	0\\
15.9010674921457	0\\
16.0043356841422	0\\
16.1076038761386	0\\
16.2124869885142	0\\
16.3173701008898	0\\
16.4222532132654	0\\
16.527136325641	0\\
16.6339006004935	0\\
16.7406648753461	0\\
16.8474291501986	0\\
16.9541934250511	0\\
17.0630876639797	0\\
17.1719819029082	0\\
17.2808761418368	0\\
17.3897703807653	0\\
17.5010359523856	0\\
17.6123015240059	0\\
17.7235670956262	0\\
17.8348326672466	0\\
17.9487100003084	0\\
18.0625873333702	0\\
18.1764646664321	0\\
18.2903419994939	0\\
18.4070757021241	0\\
18.5238094047543	0\\
18.6405431073846	0\\
18.7572768100148	0\\
18.8771200075905	0\\
18.9969632051661	0\\
19.1168064027418	0\\
19.2366496003175	0\\
19.3598678839536	0\\
19.4830861675897	0\\
19.6063044512258	0\\
19.7295227348619	0\\
19.8563979903555	0\\
19.9832732458491	0\\
20.1101485013428	0\\
20.2370237568364	0\\
20.3678580796439	0\\
20.4986924024514	0\\
20.6295267252589	0\\
20.7603610480664	0\\
20.8954809190363	0\\
21.0306007900061	0\\
21.1657206609759	0\\
21.3008405319458	0\\
21.4406014192669	0\\
21.5803623065881	0\\
21.7201231939092	0\\
21.8598840812304	0\\
22.0046756408933	0\\
22.1494672005562	0\\
22.2942587602192	0\\
22.4390503198821	0\\
22.5893023876591	0\\
22.7395544554361	0\\
22.8898065232131	0\\
23.0400585909901	0\\
23.1962481938828	0\\
23.3524377967755	0\\
23.5086273996682	0\\
23.6648170025609	0\\
23.8274766689118	0\\
23.9901363352628	0\\
24.1527960016138	0\\
24.3154556679648	0\\
24.4851833872328	0\\
24.6549111065008	0\\
24.8246388257688	0\\
24.9943665450368	0\\
25.1718378270833	0\\
25.3493091091298	0\\
25.5267803911763	0\\
25.7042516732229	0\\
25.8902342929274	0\\
26.0762169126319	0\\
26.2621995323364	0\\
26.4481821520409	0\\
26.6435543528394	0\\
26.838926553638	0\\
27.0342987544366	0\\
27.2296709552352	0\\
27.4354441469161	0\\
27.641217338597	0\\
27.8469905302779	0\\
28.0527637219589	0\\
28.2701110770545	0\\
28.4874584321501	0\\
28.7048057872458	0\\
28.9221531423414	0\\
29.1524460226538	0\\
29.3827389029661	0\\
29.6130317832784	0\\
29.8433246635908	0\\
30.088179573096	0\\
30.3330344826013	0\\
30.5778893921066	0\\
30.8227443016119	0\\
31.0840841898221	0\\
31.3454240780323	0\\
31.6067639662425	0\\
31.8681038544528	0\\
32.1482392887021	0\\
32.4283747229513	0\\
32.7085101572006	0\\
32.9886455914499	0\\
33.2903839143714	0\\
33.5921222372929	0\\
33.8938605602144	0\\
34.195598883136	0\\
34.5223935450538	0\\
34.8491882069717	0\\
35.1759828688896	0\\
35.5027775308075	0\\
35.8301107995782	0\\
36.1574440683489	0\\
36.4847773371196	0\\
36.8121106058903	0\\
37.1394383112246	0\\
37.4667660165588	0\\
37.7940937218931	0\\
38.1214214272273	0\\
38.448749190017	0\\
38.7760769528068	0\\
39.1034047155965	0\\
39.4307324783862	0\\
39.7580602405942	0\\
40.0853880028023	0\\
40.4127157650103	0\\
40.7400435272184	0\\
41.0673712894278	0\\
41.3946990516372	0\\
41.7220268138466	0\\
42.049354576056	0\\
42.3766823382676	0\\
42.7040101004792	0\\
43.0313378626908	0\\
43.3586656249024	0\\
43.6859933871135	0\\
44.0133211493247	0\\
44.3406489115359	0\\
44.667976673747	0\\
44.9953044359533	0\\
45.3226321981596	0\\
45.649959960366	0\\
45.9772877225723	0\\
46.3046154847852	0\\
46.6319432469981	0\\
46.9592710092111	0\\
47.286598771424	0\\
47.6139265336322	0\\
47.9412542958405	0\\
48.2685820580487	0\\
48.595909820257	0\\
48.926850153611	0\\
49.257790486965	0\\
49.5887308203189	0\\
49.9196711536729	0\\
50.2593448649822	0\\
50.5990185762914	0\\
50.9386922876007	0\\
51.2783659989099	0\\
51.6270508037435	0\\
51.9757356085771	0\\
52.3244204134106	0\\
52.6731052182442	0\\
53.0312891287569	0\\
53.3894730392697	0\\
53.7476569497824	0\\
54.1058408602952	0\\
54.4740437546715	0\\
54.8422466490478	0\\
55.2104495434241	0\\
55.5786524378004	0\\
55.95743714694	0\\
56.3362218560795	0\\
56.7150065652191	0\\
57.0937912743587	0\\
57.4837682235858	0\\
57.8737451728129	0\\
58.26372212204	0\\
58.653699071267	0\\
58.9902743034503	0\\
59.3268495356335	0\\
59.6634247678168	0\\
60	0\\
};
\addlegendentry{Sani}

\addplot [color=mycolor2, line width=2.0pt]
  table[row sep=crcr]{%
0	0\\
0.000265404780368998	0\\
0.000530809560737996	0\\
0.000796214341106994	1.11022302462516e-16\\
0.00106161912147599	0\\
0.00238864302332098	1.11022302462516e-16\\
0.00371566692516597	1.11022302462516e-16\\
0.00504269082701096	0\\
0.00636971472885596	1.11022302462516e-16\\
0.0130048342380809	1.11022302462516e-16\\
0.0196399537473059	2.22044604925031e-16\\
0.0262750732565308	3.33066907387547e-16\\
0.0329101927657558	2.22044604925031e-16\\
0.0481555938880245	1.88737914186277e-15\\
0.0634009950102933	7.7715611723761e-16\\
0.0786463961325621	2.77555756156289e-15\\
0.0938917972548308	1.22124532708767e-15\\
0.112426119267152	5.6621374255883e-15\\
0.130960441279472	6.66133814775094e-16\\
0.149494763291793	6.21724893790088e-15\\
0.168029085304114	2.66453525910038e-15\\
0.18883016427751	1.0325074129014e-14\\
0.209631243250907	3.33066907387547e-16\\
0.230432322224303	1.01030295240889e-14\\
0.2512334011977	4.44089209850063e-15\\
0.273784345742275	1.57651669496772e-14\\
0.29633529028685	1.11022302462516e-16\\
0.318886234831425	1.4432899320127e-14\\
0.341437179376	6.32827124036339e-15\\
0.365441427620669	2.14273043752655e-14\\
0.389445675865339	5.55111512312578e-16\\
0.413449924110009	1.92068583260152e-14\\
0.437454172354678	8.43769498715119e-15\\
0.462718995465789	2.76445533131664e-14\\
0.487983818576899	1.11022302462516e-15\\
0.513248641688009	2.42028619368284e-14\\
0.538513464799119	1.0547118733939e-14\\
0.564903949168048	3.40838468559923e-14\\
0.591294433536978	1.66533453693773e-15\\
0.617684917905907	2.95319324550292e-14\\
0.644075402274836	1.28785870856518e-14\\
0.671492335525317	4.09672296086683e-14\\
0.698909268775797	2.33146835171283e-15\\
0.726326202026278	3.51940698806175e-14\\
0.753743135276759	1.53210777398272e-14\\
0.782111039573039	4.82947015711943e-14\\
0.810478943869319	3.10862446895044e-15\\
0.838846848165599	4.08562073062058e-14\\
0.867214752461879	1.77635683940025e-14\\
0.896474834868797	5.60662627435704e-14\\
0.925734917275715	3.88578058618805e-15\\
0.954994999682632	4.67403893367191e-14\\
0.98425508208955	2.03170813506404e-14\\
1.01436081133407	6.40598685208715e-14\\
1.04446654057859	4.77395900588817e-15\\
1.0745722698231	5.3179682879545e-14\\
1.10467799906762	2.32036612146658e-14\\
1.13559219349808	7.22755189030977e-14\\
1.16650638792855	5.55111512312578e-15\\
1.19742058235901	5.98410210272959e-14\\
1.22833477678948	2.60902410786912e-14\\
1.26002761487183	8.08242361927114e-14\\
1.29172045295419	6.32827124036339e-15\\
1.32341329103654	6.68354260824344e-14\\
1.35510612911889	2.90878432451791e-14\\
1.38755375131887	9.00390872971002e-14\\
1.42000137351885	7.32747196252603e-15\\
1.45244899571883	7.40518757424979e-14\\
1.48489661791881	3.21964677141295e-14\\
1.51811813384306	9.99200722162641e-14\\
1.5513396497673	7.99360577730113e-15\\
1.58456116569155	8.21565038222616e-14\\
1.61778268161579	3.574918139293e-14\\
1.65178046131382	1.10911280160053e-13\\
1.68577824101184	9.10382880192628e-15\\
1.71977602070987	9.09272657168003e-14\\
1.7537738004079	3.96349619791181e-14\\
1.78853445564131	1.2223555501123e-13\\
1.82329511087471	1.02140518265514e-14\\
1.85805576610812	1.00142116821189e-13\\
1.89281642134153	4.36317648677687e-14\\
1.92833097625931	1.34336985979644e-13\\
1.96384553117709	1.13242748511766e-14\\
1.99936008609487	1.09801057135428e-13\\
2.03487464101265	4.78506123613442e-14\\
2.07113752249011	1.47104550762833e-13\\
2.10740040396758	1.25455201782643e-14\\
2.14366328544504	1.20126131264442e-13\\
2.1799261669225	5.22915044598449e-14\\
2.21693501810426	1.6064927166326e-13\\
2.25394386928602	1.38777878078145e-14\\
2.29095272046778	1.30895294603306e-13\\
2.32796157164954	5.69544411632705e-14\\
2.3657171313366	1.75082170983387e-13\\
2.40347269102366	1.53210777398272e-14\\
2.44122825071071	1.42330591756945e-13\\
2.47898381039777	6.19504447740837e-14\\
2.51748987308115	1.90403248723214e-13\\
2.55599593576452	1.67643676718399e-14\\
2.5945019984479	1.54654067330284e-13\\
2.63300806113127	6.7390537594747e-14\\
2.67227151262563	2.06834549487667e-13\\
2.71153496411999	1.83186799063151e-14\\
2.75079841561434	1.67865721323324e-13\\
2.7900618671087	7.31636973227978e-14\\
2.83009279630781	2.24265050974282e-13\\
2.87012372550692	1.99840144432528e-14\\
2.91015465470604	1.82187598340988e-13\\
2.95018558390515	7.93809462606987e-14\\
2.99099747582647	2.43249864695372e-13\\
3.03180936774779	2.16493489801906e-14\\
3.07262125966911	1.97397653778353e-13\\
3.11343315159043	8.60422844084496e-14\\
3.15504317204847	2.63677968348475e-13\\
3.1966531925065	2.35367281220533e-14\\
3.23826321296453	2.1405099914773e-13\\
3.27987323342256	9.32587340685131e-14\\
3.32230262192904	2.8599345114344e-13\\
3.36473201043552	2.55351295663786e-14\\
3.407161398942	2.32258656751583e-13\\
3.44959078744848	1.01252339845814e-13\\
3.49286538309579	3.10307335382731e-13\\
3.5361399787431	2.75335310107039e-14\\
3.57941457439041	2.52242671194836e-13\\
3.62268917003772	1.10023101740353e-13\\
3.66684010640848	3.3717473257866e-13\\
3.71099104277924	2.98649993624167e-14\\
3.75514197915	2.74225087082414e-13\\
3.79929291552076	1.19682042054592e-13\\
3.84435753680729	3.67039731941077e-13\\
3.88942215809383	3.23074900165921e-14\\
3.93448677938037	2.98872038229092e-13\\
3.9795514006669	1.30451205393456e-13\\
4.02557447530949	4.00568467284756e-13\\
4.07159754995208	3.49720252756924e-14\\
4.11762062459467	3.26516591542259e-13\\
4.16364369923726	1.4255263636187e-13\\
4.21041243310513	4.29767332832398e-13\\
4.257181166973	4.9737991503207e-14\\
4.30394990084087	3.33955085807247e-13\\
4.35071863470874	1.45772283133283e-13\\
4.39747966483664	4.21440660147709e-13\\
4.44424069496454	4.87387907810444e-14\\
4.49100172509244	3.27515792264421e-13\\
4.53776275522034	1.42885703269258e-13\\
4.58452386492069	4.13669098975333e-13\\
4.63128497462103	4.78506123613442e-14\\
4.67804608432138	3.21409565628983e-13\\
4.72480719402173	1.40221168010157e-13\\
4.77156830290013	4.0600856010542e-13\\
4.81832941177852	4.68514116391816e-14\\
4.86509052065691	3.15414361296007e-13\\
4.91185162953531	1.37667655053519e-13\\
4.95861273842228	3.98459043537969e-13\\
5.00537384730925	4.59632332194815e-14\\
5.05213495619622	3.09641201567956e-13\\
5.09889606508318	1.35114142096882e-13\\
5.14565717397096	3.9102054927298e-13\\
5.19241828285873	4.51860771022439e-14\\
5.2391793917465	3.03868041839905e-13\\
5.28594050063428	1.32671651442706e-13\\
5.33270160952073	3.83804099612917e-13\\
5.37946271840718	4.42978986825437e-14\\
5.42622382729363	2.9831692671678e-13\\
5.47298493618008	1.30229160788531e-13\\
5.51974604506734	3.76698672255316e-13\\
5.56650715395461	4.35207425653061e-14\\
5.61326826284187	2.92876833896116e-13\\
5.66002937172913	1.27897692436818e-13\\
5.70679048061676	3.69593244897715e-13\\
5.7535515895044	4.25215418431435e-14\\
5.80031269839203	2.87547763377916e-13\\
5.84707380727967	1.25455201782643e-13\\
5.89383491616663	3.62709862145039e-13\\
5.9405960250536	4.18554080283684e-14\\
5.98735713394057	2.82107670557252e-13\\
6.03411824282754	1.23234755733392e-13\\
6.08087935171415	3.55937501694825e-13\\
6.12764046060076	4.09672296086683e-14\\
6.17440156948738	2.77000644643977e-13\\
6.22116267837399	1.2090328738168e-13\\
6.26792378726144	3.49498208151999e-13\\
6.3146848961489	4.01900734914307e-14\\
6.36144600503635	2.71782596428238e-13\\
6.4082071139238	1.18682841332429e-13\\
6.45496822281113	3.42947892306711e-13\\
6.50172933169847	3.94129173741931e-14\\
6.5484904405858	2.66897615119888e-13\\
6.59525154947314	1.16573417585641e-13\\
6.64201265836067	3.36619621066347e-13\\
6.68877376724821	3.86357612569554e-14\\
6.73553487613575	2.61901611509074e-13\\
6.78229598502329	1.14352971536391e-13\\
6.82916482319815	3.32178728967847e-13\\
6.87603366137301	2.76445533131664e-14\\
6.92290249954786	2.73558953267639e-13\\
6.96977133772272	1.20070620113211e-13\\
7.01788255937152	3.71647157493271e-13\\
7.06599378102033	3.11972669919669e-14\\
7.11410500266913	3.05921954435462e-13\\
7.16221622431793	1.34281474828413e-13\\
7.21159336850751	4.15167900058577e-13\\
7.26097051269708	3.45279360658424e-14\\
7.31034765688666	3.42226247340705e-13\\
7.35972480107623	1.50268686383015e-13\\
7.41043619055615	4.65183447317941e-13\\
7.46114758003607	3.84137166520304e-14\\
7.511858969516	3.8397063306661e-13\\
7.56257035899592	1.6858736628933e-13\\
7.61468820401402	5.22804022295986e-13\\
7.66680604903213	4.27990975992998e-14\\
7.71892389405023	4.32154312335342e-13\\
7.77104173906833	1.89792626059671e-13\\
7.8246442387161	5.89361892622264e-13\\
7.87824673836387	4.79061235125755e-14\\
7.93184923801164	4.87776485869063e-13\\
7.98545173765941	2.14328554903886e-13\\
8.04062368192306	6.66577903984944e-13\\
8.09579562618671	5.37347943918576e-14\\
8.15096757045037	5.52502488204709e-13\\
8.20613951471402	2.42916797787984e-13\\
8.26297303935407	7.5645045782835e-13\\
8.31980656399412	6.03961325396085e-14\\
8.37664008863416	6.28108676181682e-13\\
8.43347361327421	2.76223488526739e-13\\
8.49206906178453	8.61588578260353e-13\\
8.55066451029485	6.81121825607534e-14\\
8.60925995880517	7.16648962395539e-13\\
8.66785540731549	3.15303338993544e-13\\
8.7283223218246	9.85267423203595e-13\\
8.7887892363337	7.71049890602171e-14\\
8.84925615084281	8.20898904407841e-13\\
8.90972306535191	3.61322083364257e-13\\
8.97139693657501	1.07908126878442e-12\\
9.03307080779811	1.31006316905768e-13\\
9.09474467902121	8.38107361289531e-13\\
9.15641855024431	3.70536934468646e-13\\
9.21815611731334	1.05782049786285e-12\\
9.27989368438236	1.27231558622043e-13\\
9.34163125145139	8.23119350457091e-13\\
9.40336881852041	3.63820085169664e-13\\
9.46537482585172	1.04816155754861e-12\\
9.52738083318303	9.43689570931383e-14\\
9.58938684051434	8.6103346674804e-13\\
9.65139284784564	3.82027742773516e-13\\
9.71505322212012	1.16673337657858e-12\\
9.77871359639459	1.05193631583234e-13\\
9.84237397066907	9.58622070612591e-13\\
9.90603434494354	4.2543746303636e-13\\
9.97140380506578	1.29873889420651e-12\\
10.036773265188	1.15907283770866e-13\\
10.1021427253103	1.06914477271403e-12\\
10.1675121854325	4.74731365329717e-13\\
10.2347010682591	1.45161660469739e-12\\
10.3018899510858	1.28286270495437e-13\\
10.3690788339124	1.19737553205823e-12\\
10.436267716739	5.31907851097912e-13\\
10.505392885882	1.62919677748619e-12\\
10.574518055025	1.42441614059408e-13\\
10.643643224168	1.34653399541662e-12\\
10.712768393311	5.98521232575422e-13\\
10.7839560546811	1.83619786042755e-12\\
10.8551437160513	1.5887291482386e-13\\
10.9263313774214	1.52078349913154e-12\\
10.9975190387916	6.76292355450414e-13\\
11.07090572606	2.07855954670322e-12\\
11.1442924133284	1.77913239696181e-13\\
11.2176791005968	1.72506453566257e-12\\
11.2910657878652	7.67608199225833e-13\\
11.3667995476522	2.36327624136834e-12\\
11.4425333074392	1.99951166734991e-13\\
11.5182670672261	1.96570537625007e-12\\
11.5940008270131	8.75244321463242e-13\\
11.6722425439244	2.69939626207361e-12\\
11.7504842608358	2.25819363208757e-13\\
11.8287259777471	2.25036655976396e-12\\
11.9069676946584	1.00269792469021e-12\\
11.9878925800102	3.09796632791404e-12\\
12.068817465362	2.56017429478561e-13\\
12.1497423507139	2.58870702651848e-12\\
12.2306672360657	1.15424336755154e-12\\
12.3144665036616	3.57280871554622e-12\\
12.3982657712575	2.91544566266566e-13\\
12.4820650388535	2.99282820748203e-12\\
12.5658643064494	1.3354872763216e-12\\
12.652747062149	4.14140943760799e-12\\
12.7396298178485	3.33510996597397e-13\\
12.826512573548	3.477940158092e-12\\
12.9133953292476	1.55325752260183e-12\\
13.0035907048582	4.82569539883571e-12\\
13.0937860804688	3.83415521554298e-13\\
13.1839814560794	4.06330524782561e-12\\
13.27417683169	1.81638037943799e-12\\
13.3679363981715	5.65364421944992e-12\\
13.461695964653	4.42812453371744e-13\\
13.5554555311345	4.77357042782955e-12\\
13.649215097616	2.13590256592511e-12\\
13.7468155979555	6.66083854738986e-12\\
13.8444160982949	5.14199793855141e-13\\
13.9420165986344	5.64004398739826e-12\\
14.0396170989738	2.52628473695893e-12\\
14.1413635477079	7.89326937145063e-12\\
14.2431099964419	6.00158811536744e-13\\
14.344856445176	6.70349886711108e-12\\
14.44660289391	3.00603986147507e-12\\
14.5528320987134	9.50228784546425e-12\\
14.6590613035167	1.39546707522697e-12\\
14.7652905083201	7.0770056481706e-12\\
14.8715197131234	3.13485348790721e-12\\
14.9753724014636	8.17504397510049e-12\\
15.0792250898038	1.26842980563424e-12\\
15.183077778144	6.0105809218669e-12\\
15.2869304664842	2.67366684347792e-12\\
15.3888306269013	7.06804059724675e-12\\
15.4907307873185	8.05133737458164e-13\\
15.5926309477356	5.61298230117302e-12\\
15.6945311081528	2.50460763240312e-12\\
15.7977993001492	7.24609261482101e-12\\
15.9010674921457	8.02247157594138e-13\\
16.0043356841422	5.78581627053154e-12\\
16.1076038761386	2.58082444304364e-12\\
16.2124869885142	7.50885464917417e-12\\
16.3173701008898	8.09435851678586e-13\\
16.4222532132654	6.02684568917766e-12\\
16.527136325641	2.68840505412982e-12\\
16.6339006004935	7.86670728558647e-12\\
16.7406648753461	8.2903128806322e-13\\
16.8474291501986	6.34212127259559e-12\\
16.9541934250511	2.82951440055967e-12\\
17.0630876639797	8.31903990139438e-12\\
17.1719819029082	8.59645687967259e-13\\
17.2808761418368	6.73305855514172e-12\\
17.3897703807653	3.0048463717236e-12\\
17.5010359523856	8.87045992215008e-12\\
17.6123015240059	9.00696184302774e-13\\
17.7235670956262	7.20482007388057e-12\\
17.8348326672466	3.21673243597331e-12\\
17.9487100003084	9.52965484302126e-12\\
18.0625873333702	9.52432577250306e-13\\
18.1764646664321	7.7657047459212e-12\\
18.2903419994939	3.4688918404413e-12\\
18.4070757021241	1.0309114673035e-11\\
18.5238094047543	1.01538222274655e-12\\
18.6405431073846	8.42712011284164e-12\\
18.7572768100148	3.76668141122138e-12\\
18.8771200075905	1.12257148021655e-11\\
18.9969632051661	1.09059983266491e-12\\
19.1168064027418	9.20399867432309e-12\\
19.2366496003175	4.11681799761254e-12\\
19.3598678839536	1.23008270236369e-11\\
19.4830861675897	1.17952869693738e-12\\
19.6063044512258	1.01153252440866e-11\\
19.7295227348619	4.52801685035809e-12\\
19.8563979903555	1.35617905794305e-11\\
19.9832732458491	1.28413946143269e-12\\
20.1101485013428	1.11849141060105e-11\\
20.2370237568364	5.01124142182618e-12\\
20.3678580796439	1.50426060496756e-11\\
20.4986924024514	1.406874616805e-12\\
20.6295267252589	1.24428523040621e-11\\
20.7603610480664	5.58028623309781e-12\\
20.8954809190363	1.67861974320616e-11\\
21.0306007900061	1.55077339858423e-12\\
21.1657206609759	1.39267625209882e-11\\
21.3008405319458	6.2524568855693e-12\\
21.4406014192669	1.88463827877072e-11\\
21.5803623065881	1.71977709850779e-12\\
21.7201231939092	1.5684162302243e-11\\
21.8598840812304	7.04972191734043e-12\\
22.0046756408933	2.12913020547489e-11\\
22.1494672005562	1.91867355336939e-12\\
22.2942587602192	1.7775378391427e-11\\
22.4390503198821	7.99987853739026e-12\\
22.5893023876591	2.42076775291977e-11\\
22.7395544554361	2.15352735644103e-12\\
22.8898065232131	2.02774325108734e-11\\
23.0400585909901	9.13864817153609e-12\\
23.1962481938828	2.77066425358186e-11\\
23.3524377967755	2.43198516880483e-12\\
23.5086273996682	2.32895092100449e-11\\
23.6648170025609	1.05119662974218e-11\\
23.8274766689118	3.19319154451492e-11\\
23.9901363352628	2.7636781751994e-12\\
24.1527960016138	2.69405053821004e-11\\
24.3154556679648	1.21797988361649e-11\\
24.4851833872328	3.70710961927756e-11\\
24.6549111065008	3.16091597341028e-12\\
24.8246388257688	3.13995357492658e-11\\
24.9943665450368	1.42209855003017e-11\\
25.1718378270833	4.33715979797356e-11\\
25.3493091091298	3.63942209702373e-12\\
25.5267803911763	3.6891045773757e-11\\
25.7042516732229	1.6740386854508e-11\\
25.8902342929274	5.11634207223111e-11\\
26.0762169126319	4.21954138296599e-12\\
26.2621995323364	4.3716294473306e-11\\
26.4481821520409	1.98791955119404e-11\\
26.6435543528394	6.08921246758598e-11\\
26.838926553638	4.92769713922314e-12\\
27.0342987544366	5.22849541439996e-11\\
27.2296709552352	2.38299935340081e-11\\
27.4354441469161	7.31676108589596e-11\\
27.641217338597	5.79861159089035e-12\\
27.8469905302779	6.31620866720084e-11\\
28.0527637219589	2.88592205688332e-11\\
28.2701110770545	8.8836868417097e-11\\
28.4874584321501	6.878109193309e-12\\
28.7048057872458	7.71391006626487e-11\\
28.9221531423414	3.53413479037279e-11\\
29.1524460226538	1.09095302236462e-10\\
29.3827389029661	8.22728690730301e-12\\
29.6130317832784	9.53433929029579e-11\\
29.8433246635908	4.3811829164575e-11\\
30.088179573096	1.3566012896371e-10\\
30.3330344826013	9.92796816978725e-12\\
30.5778893921066	1.19411244781098e-10\\
30.8227443016119	5.5050797254097e-11\\
31.0840841898221	1.7104669763901e-10\\
31.3454240780323	1.20904050660009e-11\\
31.6067639662425	1.51766862965808e-10\\
31.8681038544528	7.02190111612211e-11\\
32.1482392887021	2.19020003522363e-10\\
32.4283747229513	1.48631662533205e-11\\
32.7085101572006	1.96085328529882e-10\\
32.9886455914499	9.10841194134981e-11\\
33.2903839143714	2.8535295654164e-10\\
33.5921222372929	1.8445342475637e-11\\
33.8938605602144	2.5808256226556e-10\\
34.195598883136	1.2040873853536e-10\\
34.5223935450538	3.83764887923466e-10\\
34.8491882069717	5.59923947485164e-11\\
35.1759828688896	3.0208310852764e-10\\
35.5027775308075	1.40671928966496e-10\\
35.8301107995782	3.45089991304892e-10\\
36.1574440683489	5.8957883714017e-11\\
36.4847773371196	2.61138544815909e-10\\
36.8121106058903	1.23054916950638e-10\\
37.1394383112246	3.02663737822284e-10\\
37.4667660165588	5.16331317979013e-11\\
37.7940937218931	2.29126346368291e-10\\
38.1214214272273	1.07956598521808e-10\\
38.448749190017	2.65520994080504e-10\\
38.7760769528068	4.52974290021668e-11\\
39.1034047155965	2.01007231692696e-10\\
39.4307324783862	9.47079428770259e-11\\
39.7580602405942	2.32935778304855e-10\\
40.0853880028023	3.97384417039071e-11\\
40.4127157650103	1.76339269575276e-10\\
40.7400435272184	8.30852123212544e-11\\
41.0673712894278	2.04349481780497e-10\\
41.3946990516372	3.48616656375977e-11\\
41.7220268138466	1.54698599416636e-10\\
42.049354576056	7.28888494699875e-11\\
42.3766823382676	1.79271349501353e-10\\
42.7040101004792	3.05833813135648e-11\\
43.0313378626908	1.35713709367713e-10\\
43.3586656249024	6.39437981730939e-11\\
43.6859933871135	1.57270849121538e-10\\
44.0133211493247	2.68301353356648e-11\\
44.3406489115359	1.19058681316564e-10\\
44.667976673747	5.60965024432036e-11\\
44.9953044359533	1.37970289706968e-10\\
45.3226321981596	2.35374931351062e-11\\
45.649959960366	1.04447589466261e-10\\
45.9772877225723	4.92122384510019e-11\\
46.3046154847852	1.2103832866539e-10\\
46.6319432469981	2.06489304821478e-11\\
46.9592710092111	9.16295928021782e-11\\
47.286598771424	4.31728246902408e-11\\
47.6139265336322	1.06172357364209e-10\\
47.9412542958405	1.80278049433102e-11\\
48.2685820580487	8.05028856076806e-11\\
48.595909820257	3.79308650461097e-11\\
48.926850153611	9.71533478666409e-11\\
49.257790486965	1.31989757665152e-11\\
49.5887308203189	7.81821551942841e-11\\
49.9196711536729	3.65696804435767e-11\\
50.2593448649822	9.73106768775711e-11\\
50.5990185762914	1.34524613670806e-11\\
50.9386922876007	7.81527516313663e-11\\
51.2783659989099	3.66342503207995e-11\\
51.6270508037435	9.71504977159698e-11\\
51.9757356085771	1.3538086450493e-11\\
52.3244204134106	7.80424188814854e-11\\
52.6731052182442	3.66539637183805e-11\\
53.0312891287569	9.69940282605219e-11\\
53.3894730392697	1.36311283441271e-11\\
53.7476569497824	7.79349041898914e-11\\
54.1058408602952	3.66799247225602e-11\\
54.4740437546715	9.68284905381034e-11\\
54.8422466490478	1.37254886276028e-11\\
55.2104495434241	7.78260823841581e-11\\
55.5786524378004	3.67098838306712e-11\\
55.95743714694	9.66532873701575e-11\\
56.3362218560795	1.38216183290241e-11\\
56.7150065652191	7.77151732724024e-11\\
57.0937912743587	3.6743617263385e-11\\
57.4837682235858	9.64674620566874e-11\\
57.8737451728129	1.39197672485714e-11\\
58.26372212204	7.76013637029949e-11\\
58.653699071267	3.6780892567756e-11\\
58.9902743034503	4.95332737500409e-11\\
59.3268495356335	4.97881133022793e-11\\
59.6634247678168	1.71213421044925e-11\\
60	1.51363817313399e-12\\
};
\addlegendentry{Infetti}

\end{axis}
\end{tikzpicture}%}}
 \quad 
\subfloat[][Nodo 2]
{\resizebox{0.45\textwidth}{!}{ % This file was created by matlab2tikz.
%
%The latest updates can be retrieved from
%  http://www.mathworks.com/matlabcentral/fileexchange/22022-matlab2tikz-matlab2tikz
%where you can also make suggestions and rate matlab2tikz.
%
\definecolor{mycolor1}{rgb}{0.00000,0.44700,0.74100}%
\definecolor{mycolor2}{rgb}{0.85000,0.32500,0.09800}%
%
\begin{tikzpicture}

\begin{axis}[%
width=6.028in,
height=4.754in,
at={(1.011in,0.642in)},
scale only axis,
xmin=0,
xmax=60,
xlabel style={font=\color{white!15!black}},
xlabel={T},
ymode=log,
ymin=1e-20,
ymax=1e-08,
yminorticks=true,
ylabel style={font=\color{white!15!black}},
ylabel={Errore assoluto},
axis background/.style={fill=white},
legend style={legend cell align=left, align=left, draw=white!15!black}
]
\addplot [color=mycolor1, line width=2.0pt]
  table[row sep=crcr]{%
0	0\\
0.000265404780368998	0\\
0.000530809560737996	0\\
0.000796214341106994	0\\
0.00106161912147599	0\\
0.00238864302332098	1.11022302462516e-16\\
0.00371566692516597	7.7715611723761e-16\\
0.00504269082701096	1.22124532708767e-15\\
0.00636971472885596	6.66133814775094e-16\\
0.0130048342380809	3.99014155050281e-13\\
0.0196399537473059	2.47679654563626e-12\\
0.0262750732565308	3.25750537655267e-12\\
0.0329101927657558	1.78224102143076e-12\\
0.0481555938880245	3.89315246707156e-11\\
0.0634009950102933	1.06853415005048e-11\\
0.0786463961325621	5.16733322797336e-11\\
0.0938917972548308	2.28428387316626e-11\\
0.112426119267152	1.06134878663511e-10\\
0.130960441279472	7.17204073907851e-12\\
0.149494763291793	1.11880948949761e-10\\
0.168029085304114	5.00134378356165e-11\\
0.18883016427751	1.83187021107756e-10\\
0.209631243250907	3.58379992349001e-12\\
0.230432322224303	1.72891923000407e-10\\
0.2512334011977	7.84514675444825e-11\\
0.273784345742275	2.57950327764433e-10\\
0.29633529028685	1.5595635893817e-11\\
0.318886234831425	2.30505392551095e-10\\
0.341437179376	1.05659703208971e-10\\
0.365441427620669	3.26388582827519e-10\\
0.389445675865339	2.76717537772697e-11\\
0.413449924110009	2.82246115368423e-10\\
0.437454172354678	1.30386368368818e-10\\
0.462718995465789	3.86338183666624e-10\\
0.487983818576899	3.91074950201187e-11\\
0.513248641688009	3.26830673635925e-10\\
0.538513464799119	1.51952561644464e-10\\
0.564903949168048	4.36734315378828e-10\\
0.591294433536978	4.94934093708821e-11\\
0.617684917905907	3.63671426306666e-10\\
0.644075402274836	1.70021663414843e-10\\
0.671492335525317	4.77198947024249e-10\\
0.698909268775797	5.85946846598517e-11\\
0.726326202026278	3.92633925372365e-10\\
0.753743135276759	1.84480541953747e-10\\
0.782111039573039	5.07810127281516e-10\\
0.810478943869319	6.62910837334607e-11\\
0.838846848165599	4.13888368022697e-10\\
0.867214752461879	1.95368166089338e-10\\
0.896474834868797	5.28953991718595e-10\\
0.925734917275715	7.25416393621003e-11\\
0.954994999682632	4.27813118264453e-10\\
0.98425508208955	2.02828531747912e-10\\
1.01436081133407	5.41223732497542e-10\\
1.04446654057859	7.73612285343006e-11\\
1.0745722698231	4.34923985714875e-10\\
1.10467799906762	2.07076578107035e-10\\
1.13559219349808	5.45345879565673e-10\\
1.16650638792855	8.08044742228731e-11\\
1.19742058235901	4.35823932498636e-10\\
1.22833477678948	2.08373984733612e-10\\
1.26002761487183	5.42126565861167e-10\\
1.29172045295419	8.29537549762449e-11\\
1.32341329103654	4.31164548508889e-10\\
1.35510612911889	2.07009853703255e-10\\
1.38755375131887	5.32409449860438e-10\\
1.42000137351885	8.39094349558422e-11\\
1.45244899571883	4.2161740765323e-10\\
1.48489661791881	2.03287275901687e-10\\
1.51811813384306	5.193782071089e-10\\
1.5513396497673	8.28700441601882e-11\\
1.58456116569155	4.11392031551827e-10\\
1.61778268161579	1.99051664040439e-10\\
1.65178046131382	5.03212693736543e-10\\
1.68577824101184	8.27061752417535e-11\\
1.71977602070987	3.96937815949627e-10\\
1.7537738004079	1.92981075564091e-10\\
1.78853445564131	4.82161199855113e-10\\
1.82329511087471	8.15173484269849e-11\\
1.85805576610812	3.78912456966418e-10\\
1.89281642134153	1.85144899411682e-10\\
1.92833097625931	4.57089810446121e-10\\
1.96384553117709	7.94829757566617e-11\\
1.99936008609487	3.57924134775089e-10\\
2.03487464101265	1.75832792770336e-10\\
2.07113752249011	4.28740487556922e-10\\
2.10740040396758	7.67157448677835e-11\\
2.14366328544504	3.34547833880094e-10\\
2.1799261669225	1.65313152056257e-10\\
2.21693501810426	3.97809285512807e-10\\
2.25394386928602	7.33275107300813e-11\\
2.29095272046778	3.09317904623185e-10\\
2.32796157164954	1.53837664829126e-10\\
2.3657171313366	3.64935082153295e-10\\
2.40347269102366	6.94256319100361e-11\\
2.44122825071071	2.82721512867568e-10\\
2.47898381039777	1.4163742401152e-10\\
2.51748987308115	3.30695526518099e-10\\
2.55599593576452	6.51107501248305e-11\\
2.5945019984479	2.55196697107607e-10\\
2.63300806113127	1.28921706643581e-10\\
2.67227151262563	2.95603652666898e-10\\
2.71153496411999	6.04761241085328e-11\\
2.75079841561434	2.27130148022781e-10\\
2.7900618671087	1.15876475081933e-10\\
2.83009279630781	2.60107380078978e-10\\
2.87012372550692	5.56066304113756e-11\\
2.91015465470604	1.9885648683271e-10\\
2.95018558390515	1.02664432510835e-10\\
2.99099747582647	2.24588958541716e-10\\
3.03180936774779	5.0578596866302e-11\\
3.07262125966911	1.70659042453281e-10\\
3.11343315159043	8.94241347637603e-11\\
3.15504317204847	1.89365967351307e-10\\
3.1966531925065	4.54591364551504e-11\\
3.23826321296453	1.42769907007789e-10\\
3.27987323342256	7.6270822990665e-11\\
3.32230262192904	1.54692370024634e-10\\
3.36473201043552	4.03065358867138e-11\\
3.407161398942	1.15371212583426e-10\\
3.44959078744848	6.32963681468368e-11\\
3.49286538309579	1.20759791055747e-10\\
3.5361399787431	3.5170200085588e-11\\
3.57941457439041	8.85955198093313e-11\\
3.62268917003772	5.05692154817439e-11\\
3.66684010640848	8.7697460404712e-11\\
3.71099104277924	3.00905411698693e-11\\
3.75514197915	6.25260954123519e-11\\
3.79929291552076	3.81348841393958e-11\\
3.84435753680729	5.55722690087634e-11\\
3.88942215809383	2.50997000961206e-11\\
3.93448677938037	3.71956909717142e-11\\
3.9795514006669	2.60148569353191e-11\\
4.02557447530949	2.43860487358916e-11\\
4.07159754995208	2.02216021705226e-11\\
4.11762062459467	1.25835453168577e-11\\
4.16364369923726	1.42054146223813e-11\\
4.21041243310513	5.8795190938099e-12\\
4.257181166973	1.47871159761337e-11\\
4.30394990084087	1.10385589557893e-11\\
4.35071863470874	2.6771918015811e-12\\
4.39747966483664	3.14919201827024e-11\\
4.44424069496454	8.08497713222778e-12\\
4.49100172509244	2.94467783490404e-11\\
4.53776275522034	7.20246084995324e-12\\
4.58452386492069	5.04670749634784e-11\\
4.63128497462103	2.87098123052942e-12\\
4.67804608432138	4.29807855972797e-11\\
4.72480719402173	1.45960465935957e-11\\
4.77156830290013	6.40961728137768e-11\\
4.81832941177852	1.13975495708019e-12\\
4.86509052065691	5.25909316095863e-11\\
4.91185162953531	1.99869565342681e-11\\
4.95861273842228	7.34435290361546e-11\\
5.00537384730925	4.17149648157533e-12\\
5.05213495619622	5.90645310438731e-11\\
5.09889606508318	2.37692088234098e-11\\
5.14565717397096	7.93868026871536e-11\\
5.19241828285873	6.4105110109125e-12\\
5.2391793917465	6.30504259913067e-11\\
5.28594050063428	2.62682653406898e-11\\
5.33270160952073	8.26475277104777e-11\\
5.37946271840718	8.01103627878774e-12\\
5.42622382729363	6.50819398373415e-11\\
5.47298493618008	2.77519118796477e-11\\
5.51974604506734	8.38173697115252e-11\\
5.56650715395461	9.1006646663061e-12\\
5.61326826284187	6.55955012529574e-11\\
5.66002937172913	2.84401391326128e-11\\
5.70679048061676	8.33795255061887e-11\\
5.7535515895044	9.78458980505081e-12\\
5.80031269839203	6.49473808067569e-11\\
5.84707380727967	2.85128309851501e-11\\
5.89383491616663	8.17269307784585e-11\\
5.9405960250536	1.01491592907621e-11\\
5.98735713394057	6.34267915966547e-11\\
6.03411824282754	2.81165923876614e-11\\
6.08087935171415	7.91778032027679e-11\\
6.12764046060076	1.02652886191379e-11\\
6.17440156948738	6.12676576139393e-11\\
6.22116267837399	2.73704114928108e-11\\
6.26792378726144	7.59881602085954e-11\\
6.3146848961489	1.01907648986099e-11\\
6.36144600503635	5.86578008388017e-11\\
6.4082071139238	2.63702393255016e-11\\
6.45496822281113	7.23629767218625e-11\\
6.50172933169847	9.97252280754424e-12\\
6.5484904405858	5.57472956685956e-11\\
6.59525154947314	2.51930976080672e-11\\
6.64201265836067	6.84647616377987e-11\\
6.68877376724821	9.64867075126108e-12\\
6.73553487613575	5.26549082113803e-11\\
6.78229598502329	2.39002984070424e-11\\
6.82916482319815	6.4772909258437e-11\\
6.87603366137301	7.25658422240372e-12\\
6.92290249954786	5.27233812164241e-11\\
6.96977133772272	2.40805431150903e-11\\
7.01788255937152	6.92392254642016e-11\\
7.06599378102033	8.00293165070798e-12\\
7.11410500266913	5.62420665595198e-11\\
7.16221622431793	2.57981691564879e-11\\
7.21159336850751	7.35034255683331e-11\\
7.26097051269708	8.66109961528139e-12\\
7.31034765688666	5.97107363553562e-11\\
7.35972480107623	2.75042211228538e-11\\
7.41043619055615	7.78713482585403e-11\\
7.46114758003607	9.34410882003078e-12\\
7.511858969516	6.32669749922599e-11\\
7.56257035899592	2.92604829255083e-11\\
7.61468820401402	8.23408008443494e-11\\
7.66680604903213	1.00495445298776e-11\\
7.71892389405023	6.69138910946998e-11\\
7.77104173906833	3.10692582772276e-11\\
7.8246442387161	8.69225802446749e-11\\
7.87824673836387	1.07797382131736e-11\\
7.93184923801164	7.06609770695366e-11\\
7.98545173765941	3.29359040041055e-11\\
8.04062368192306	9.16285103347292e-11\\
8.09579562618671	1.15369380715435e-11\\
8.15096757045037	7.45188355466553e-11\\
8.20613951471402	3.48663320437481e-11\\
8.26297303935407	9.6472246857715e-11\\
8.31980656399412	1.23238919069735e-11\\
8.37664008863416	7.84996512237512e-11\\
8.43347361327421	3.68673702677569e-11\\
8.49206906178453	1.01469388447128e-10\\
8.55066451029485	1.31435695660542e-11\\
8.60925995880517	8.26176349555396e-11\\
8.66785540731549	3.89468457484554e-11\\
8.7283223218246	1.06638170516149e-10\\
8.7887892363337	1.39993849845865e-11\\
8.84925615084281	8.68890792649069e-11\\
8.90972306535191	4.11139178257969e-11\\
8.97139693657501	1.06793573984021e-10\\
9.03307080779811	1.88154769542592e-11\\
9.09474467902121	8.07490463383687e-11\\
9.15641855024431	3.86704002153238e-11\\
9.21815611731334	9.51361767143055e-11\\
9.27989368438236	1.67453828581188e-11\\
9.34163125145139	7.20392634434575e-11\\
9.40336881852041	3.45352357822293e-11\\
9.46537482585172	8.55275850142334e-11\\
9.52738083318303	1.26068322447992e-11\\
9.58938684051434	6.84343137713483e-11\\
9.65139284784564	3.28187199638563e-11\\
9.71505322212012	8.6399915000257e-11\\
9.77871359639459	1.29192767595043e-11\\
9.84237397066907	6.90766888133965e-11\\
9.90603434494354	3.3226948970011e-11\\
9.97140380506578	8.69434246819623e-11\\
10.036773265188	1.30929433961313e-11\\
10.1021427253103	6.95897506286514e-11\\
10.1675121854325	3.3572838953333e-11\\
10.2347010682591	8.75267358591003e-11\\
10.3018899510858	1.32778787964583e-11\\
10.3690788339124	7.01327051988443e-11\\
10.436267716739	3.39354655487512e-11\\
10.505392885882	8.81408279695961e-11\\
10.574518055025	1.34697530906891e-11\\
10.643643224168	7.07034153446529e-11\\
10.712768393311	3.43140793557239e-11\\
10.7839560546811	8.87907247726361e-11\\
10.8551437160513	1.36697597685753e-11\\
10.9263313774214	7.13062386914487e-11\\
10.9975190387916	3.47111783760568e-11\\
11.07090572606	8.94812557383773e-11\\
11.1442924133284	1.38789535419903e-11\\
11.2176791005968	7.19454495978766e-11\\
11.2910657878652	3.51291773448281e-11\\
11.3667995476522	9.02177776929136e-11\\
11.4425333074392	1.40984446339587e-11\\
11.5182670672261	7.26257110006401e-11\\
11.5940008270131	3.55707685528728e-11\\
11.6722425439244	9.10058417513682e-11\\
11.7504842608358	1.43295098009588e-11\\
11.8287259777471	7.33521021700767e-11\\
11.9069676946584	3.60388663356304e-11\\
11.9878925800102	9.1851581895952e-11\\
12.068817465362	1.45733980438933e-11\\
12.1497423507139	7.41300354434316e-11\\
12.2306672360657	3.65365515619942e-11\\
12.3144665036616	9.27614651757835e-11\\
12.3982657712575	1.48316081638455e-11\\
12.4820650388535	7.49653672471595e-11\\
12.5658643064494	3.70671826566138e-11\\
12.652747062149	9.37425137514936e-11\\
12.7396298178485	1.5105611206323e-11\\
12.826512573548	7.58643425857741e-11\\
12.9133953292476	3.76343400887436e-11\\
13.0035907048582	9.48021383617714e-11\\
13.0937860804688	1.53972390393164e-11\\
13.1839814560794	7.68335950418475e-11\\
13.27417683169	3.82418818833941e-11\\
13.3679363981715	9.59479995454871e-11\\
13.461695964653	1.57082680196652e-11\\
13.5554555311345	7.78800635092836e-11\\
13.649215097616	3.88939158657564e-11\\
13.7468155979555	9.7188146419569e-11\\
13.8444160982949	1.60408353266916e-11\\
13.9420165986344	7.90108534154399e-11\\
14.0396170989738	3.9594744150051e-11\\
14.1413635477079	9.85304893230676e-11\\
14.2431099964419	1.63971336508695e-11\\
14.344856445176	8.02332089655522e-11\\
14.44660289391	4.03487798728008e-11\\
14.5528320987134	1.00914637757299e-10\\
14.6590613035167	2.38614406011806e-11\\
14.7652905083201	7.17640669112285e-11\\
14.8715197131234	3.64679120234968e-11\\
14.9753724014636	7.27198301575527e-11\\
15.0792250898038	1.76065828583205e-11\\
15.183077778144	5.10618769489213e-11\\
15.2869304664842	2.60055310619123e-11\\
15.3888306269013	5.29564170292929e-11\\
15.4907307873185	1.05691011498266e-11\\
15.5926309477356	4.03836686313497e-11\\
15.6945311081528	2.0349888441018e-11\\
15.7977993001492	4.62287708113962e-11\\
15.9010674921457	9.08637054486405e-12\\
16.0043356841422	3.54471452190808e-11\\
16.1076038761386	1.78404790940334e-11\\
16.2124869885142	4.08002798213403e-11\\
16.3173701008898	7.91344767492319e-12\\
16.4222532132654	3.14417936131406e-11\\
16.527136325641	1.58127400062824e-11\\
16.6339006004935	3.64034913324929e-11\\
16.7406648753461	6.98913149577152e-12\\
16.8474291501986	2.81682177583065e-11\\
16.9541934250511	1.41611444792744e-11\\
17.0630876639797	3.2773395108876e-11\\
17.1719819029082	6.24306162322341e-12\\
17.2808761418368	2.54470333693746e-11\\
17.3897703807653	1.27924615345165e-11\\
17.5010359523856	2.97307733987395e-11\\
17.6123015240059	5.62958013539117e-12\\
17.7235670956262	2.31541730233431e-11\\
17.8348326672466	1.16425202811854e-11\\
17.9487100003084	2.71487277103688e-11\\
18.0625873333702	5.11762854316089e-12\\
18.1764646664321	2.12003470334565e-11\\
18.2903419994939	1.06653852416372e-11\\
18.4070757021241	2.49344433989052e-11\\
18.5238094047543	4.68519667506939e-12\\
18.6405431073846	1.9519469374174e-11\\
18.7572768100148	9.82708359131834e-12\\
18.8771200075905	2.30185315253095e-11\\
18.9969632051661	4.31624180841084e-12\\
19.1168064027418	1.80614134759338e-11\\
19.2366496003175	9.10210795623811e-12\\
19.3598678839536	2.13476458732487e-11\\
19.4830861675897	3.9988290456705e-12\\
19.6063044512258	1.67874325551765e-11\\
19.7295227348619	8.47061309983133e-12\\
19.8563979903555	1.98802918571772e-11\\
19.9832732458491	3.72382680247085e-12\\
20.1101485013428	1.56669122119979e-11\\
20.2370237568364	7.91708365532884e-12\\
20.3678580796439	1.85833570753857e-11\\
20.4986924024514	3.4841574070299e-12\\
20.6295267252589	1.46754275398564e-11\\
20.7603610480664	7.42911288043047e-12\\
20.8954809190363	1.7430140664132e-11\\
21.0306007900061	3.27413096634643e-12\\
21.1657206609759	1.37930777910356e-11\\
21.3008405319458	6.99668101233897e-12\\
21.4406014192669	1.63988544965576e-11\\
21.5803623065881	3.08933434389758e-12\\
21.7201231939092	1.30036537093758e-11\\
21.8598840812304	6.61157240067212e-12\\
22.0046756408933	1.54714574485126e-11\\
22.1494672005562	2.9259927813996e-12\\
22.2942587602192	1.22934440405231e-11\\
22.4390503198821	6.26687590710162e-12\\
22.5893023876591	1.46327949757108e-11\\
22.7395544554361	2.7810809211104e-12\\
22.8898065232131	1.16511522652019e-11\\
23.0400585909901	5.95690163862628e-12\\
23.1962481938828	1.38701272689445e-11\\
23.3524377967755	2.65201749449773e-12\\
23.5086273996682	1.10670639319466e-11\\
23.6648170025609	5.67676461393773e-12\\
23.8274766689118	1.31725463869969e-11\\
23.9901363352628	2.53669307781479e-12\\
24.1527960016138	1.05328801236482e-11\\
24.3154556679648	5.42224598554242e-12\\
24.4851833872328	1.25306709453099e-11\\
24.6549111065008	2.43319253634411e-12\\
24.8246388257688	1.00415231685247e-11\\
24.9943665450368	5.18979303976153e-12\\
25.1718378270833	1.19363408046524e-11\\
25.3493091091298	2.33996155785121e-12\\
25.5267803911763	9.58677581763823e-12\\
25.7042516732229	4.97629715212611e-12\\
25.8902342929274	1.13825060488182e-11\\
26.0762169126319	2.25550134125285e-12\\
26.2621995323364	9.16322573374373e-12\\
26.4481821520409	4.77895500949899e-12\\
26.6435543528394	1.0862893917718e-11\\
26.838926553638	2.17859064122194e-12\\
27.0342987544366	8.76609895783531e-12\\
27.2296709552352	4.59529636565037e-12\\
27.4354441469161	1.03719810518044e-11\\
27.641217338597	2.10811923473386e-12\\
27.8469905302779	8.3911766424194e-12\\
28.0527637219589	4.42318404125785e-12\\
28.2701110770545	9.90488246976895e-12\\
28.4874584321501	2.04303240991521e-12\\
28.7048057872458	8.03471178478787e-12\\
28.9221531423414	4.26061963487712e-12\\
29.1524460226538	9.45715727951324e-12\\
29.3827389029661	1.98233096604383e-12\\
29.6130317832784	7.69331820471564e-12\\
29.8433246635908	4.10574352294191e-12\\
30.088179573096	9.0248364337242e-12\\
30.3330344826013	1.92512672470002e-12\\
30.5778893921066	7.36388727773374e-12\\
30.8227443016119	3.95689037091529e-12\\
31.0840841898221	8.60420068526935e-12\\
31.3454240780323	1.870392729586e-12\\
31.6067639662425	7.04367120185623e-12\\
31.8681038544528	3.81236708868471e-12\\
32.1482392887021	8.19189160949918e-12\\
32.4283747229513	1.81735182458453e-12\\
32.7085101572006	6.72997768624839e-12\\
32.9886455914499	3.67059160844008e-12\\
33.2903839143714	7.7847173152179e-12\\
33.5921222372929	1.76497705339784e-12\\
33.8938605602144	6.4202809735292e-12\\
34.195598883136	3.52989859564445e-12\\
34.5223935450538	7.46036565857366e-12\\
34.8491882069717	2.29719021582753e-12\\
35.1759828688896	5.30417376687353e-12\\
35.5027775308075	3.03815306246236e-12\\
35.8301107995782	4.55419035816362e-12\\
36.1574440683489	1.52086676585839e-12\\
36.4847773371196	3.06446534814597e-12\\
36.8121106058903	1.79040116066176e-12\\
37.1394383112246	2.69401168040417e-12\\
37.4667660165588	8.98642271707217e-13\\
37.7940937218931	1.81415993338874e-12\\
38.1214214272273	1.05959685470225e-12\\
38.448749190017	1.5945855746935e-12\\
38.7760769528068	5.31741317644219e-13\\
39.1034047155965	1.07402975402238e-12\\
39.4307324783862	6.27276008913213e-13\\
39.7580602405942	9.43994882263155e-13\\
40.0853880028023	3.14720471905616e-13\\
40.4127157650103	6.35963504080905e-13\\
40.7400435272184	3.71480624039577e-13\\
41.0673712894278	5.58886270596304e-13\\
41.3946990516372	1.8623991238087e-13\\
41.7220268138466	3.76698672255316e-13\\
42.049354576056	2.20046203480706e-13\\
42.3766823382676	3.30901972489528e-13\\
42.7040101004792	1.10217390769662e-13\\
43.0313378626908	2.23154827949656e-13\\
43.3586656249024	1.30395694242225e-13\\
43.6859933871135	1.95898852695109e-13\\
44.0133211493247	6.51978471211123e-14\\
44.3406489115359	1.32227562232856e-13\\
44.667976673747	7.72715225139109e-14\\
44.9953044359533	1.15962794922098e-13\\
45.3226321981596	3.85247389544929e-14\\
45.649959960366	7.84095011141517e-14\\
45.9772877225723	4.5852210917019e-14\\
46.3046154847852	6.8584027346219e-14\\
46.6319432469981	2.27595720048157e-14\\
46.9592710092111	4.64905891561784e-14\\
47.286598771424	2.7228219678932e-14\\
47.6139265336322	4.05786515500495e-14\\
47.9412542958405	1.34059430223488e-14\\
48.2685820580487	2.76445533131664e-14\\
48.595909820257	1.62092561595273e-14\\
48.926850153611	2.50355292052973e-14\\
49.257790486965	7.43849426498855e-15\\
49.5887308203189	1.83464354819307e-14\\
49.9196711536729	1.0547118733939e-14\\
50.2593448649822	1.69864122767649e-14\\
50.5990185762914	5.10702591327572e-15\\
50.9386922876007	1.24344978758018e-14\\
51.2783659989099	7.18869408444789e-15\\
51.6270508037435	1.13797860024079e-14\\
51.9757356085771	3.44169137633799e-15\\
52.3244204134106	8.32667268468867e-15\\
52.6731052182442	4.85722573273506e-15\\
53.0312891287569	7.52176099183544e-15\\
53.3894730392697	2.3037127760972e-15\\
53.7476569497824	5.55111512312578e-15\\
54.1058408602952	3.27515792264421e-15\\
54.4740437546715	4.88498130835069e-15\\
54.8422466490478	1.47104550762833e-15\\
55.2104495434241	3.69149155687865e-15\\
55.5786524378004	2.19269047363468e-15\\
55.95743714694	3.13638004456607e-15\\
56.3362218560795	9.43689570931383e-16\\
56.7150065652191	2.41473507855972e-15\\
57.0937912743587	1.47104550762833e-15\\
57.4837682235858	1.97064586870965e-15\\
57.8737451728129	5.55111512312578e-16\\
58.26372212204	1.60982338570648e-15\\
58.653699071267	9.99200722162641e-16\\
58.9902743034503	5.55111512312578e-16\\
59.3268495356335	5.82867087928207e-16\\
59.6634247678168	1.11022302462516e-16\\
60	2.22044604925031e-16\\
};
\addlegendentry{Sani}

\addplot [color=mycolor2, line width=2.0pt]
  table[row sep=crcr]{%
0	0\\
0.000265404780368998	1.0842021724855e-19\\
0.000530809560737996	5.42101086242752e-20\\
0.000796214341106994	2.43945488809238e-19\\
0.00106161912147599	4.87890977618477e-19\\
0.00238864302332098	2.16840434497101e-17\\
0.00371566692516597	7.86697096355482e-16\\
0.00504269082701096	1.1668183780289e-15\\
0.00636971472885596	7.06682976026052e-16\\
0.0130048342380809	4.3647117170531e-13\\
0.0196399537473059	2.70952271036462e-12\\
0.0262750732565308	3.56387315492146e-12\\
0.0329101927657558	1.9498673509144e-12\\
0.0481555938880245	4.25907156764715e-11\\
0.0634009950102933	1.16701821817333e-11\\
0.0786463961325621	5.6533274589432e-11\\
0.0938917972548308	2.50039641769284e-11\\
0.112426119267152	1.16034216957051e-10\\
0.130960441279472	7.76954889314396e-12\\
0.149494763291793	1.22305221506025e-10\\
0.168029085304114	5.47130396544304e-11\\
0.18883016427751	2.00081902435034e-10\\
0.209631243250907	4.05935701719429e-12\\
0.230432322224303	1.88800988731863e-10\\
0.2512334011977	8.57472842730189e-11\\
0.273784345742275	2.81409118230158e-10\\
0.29633529028685	1.72443725965366e-11\\
0.318886234831425	2.5140124704226e-10\\
0.341437179376	1.15358680941036e-10\\
0.365441427620669	3.55584575673085e-10\\
0.389445675865339	3.04713060339523e-11\\
0.413449924110009	3.07391959109005e-10\\
0.437454172354678	1.42171399653002e-10\\
0.462718995465789	4.20242951548744e-10\\
0.487983818576899	4.29635493848224e-11\\
0.513248641688009	3.55374257798857e-10\\
0.538513464799119	1.65442964883411e-10\\
0.564903949168048	4.74233097236265e-10\\
0.591294433536978	5.42711708906296e-11\\
0.617684917905907	3.94718535634553e-10\\
0.644075402274836	1.84809501035943e-10\\
0.671492335525317	5.17162618285383e-10\\
0.698909268775797	6.41371955545367e-11\\
0.726326202026278	4.25295715311691e-10\\
0.753743135276759	2.001544485708e-10\\
0.782111039573039	5.49142536998559e-10\\
0.810478943869319	7.24316440159356e-11\\
0.838846848165599	4.47313852625086e-10\\
0.867214752461879	2.11529543880928e-10\\
0.896474834868797	5.70623992501851e-10\\
0.925734917275715	7.91123833110419e-11\\
0.954994999682632	4.6120943175687e-10\\
0.98425508208955	2.1910229186517e-10\\
1.01436081133407	5.8228560861906e-10\\
1.04446654057859	8.41994807210256e-11\\
1.0745722698231	4.67570482332036e-10\\
1.10467799906762	2.23118357123298e-10\\
1.13559219349808	5.84951864723848e-10\\
1.16650638792855	8.77565797807733e-11\\
1.19742058235901	4.67079486199395e-10\\
1.22833477678948	2.23874585536521e-10\\
1.26002761487183	5.79532810629502e-10\\
1.29172045295419	8.98778274027734e-11\\
1.32341329103654	4.6047160529028e-10\\
1.35510612911889	2.2169738267408e-10\\
1.38755375131887	5.6697840866704e-10\\
1.42000137351885	9.06775210474109e-11\\
1.45244899571883	4.48501735750284e-10\\
1.48489661791881	2.16927753537988e-10\\
1.51811813384306	5.50718637359893e-10\\
1.5513396497673	8.93214391339825e-11\\
1.58456116569155	4.35676161814058e-10\\
1.61778268161579	2.11546558048781e-10\\
1.65178046131382	5.30960497791e-10\\
1.68577824101184	8.88615847571828e-11\\
1.71977602070987	4.18226731024873e-10\\
1.7537738004079	2.04150307769879e-10\\
1.78853445564131	5.0588067068702e-10\\
1.82329511087471	8.72744099211786e-11\\
1.85805576610812	3.96892074761013e-10\\
1.89281642134153	1.94827431965194e-10\\
1.92833097625931	4.76443051677933e-10\\
1.96384553117709	8.4757922902412e-11\\
1.99936008609487	3.72351705024698e-10\\
2.03487464101265	1.83901505135253e-10\\
2.07113752249011	4.43476033673562e-10\\
2.10740040396758	8.14386891256902e-11\\
2.14366328544504	3.45246775612651e-10\\
2.1799261669225	1.71672398519007e-10\\
2.21693501810426	4.07754052744735e-10\\
2.25394386928602	7.74422748150982e-11\\
2.29095272046778	3.1617103379844e-10\\
2.32796157164954	1.58419999340964e-10\\
2.3657171313366	3.69986208337281e-10\\
2.40347269102366	7.2888417523842e-11\\
2.44122825071071	2.85664991661605e-10\\
2.47898381039777	1.44400880142115e-10\\
2.51748987308115	3.30811378290718e-10\\
2.55599593576452	6.7889582844316e-11\\
2.5945019984479	2.54212095818218e-10\\
2.63300806113127	1.29846411400791e-10\\
2.67227151262563	2.90794721635734e-10\\
2.71153496411999	6.2549410095869e-11\\
2.75079841561434	2.22237561686711e-10\\
2.7900618671087	1.14961540287339e-10\\
2.83009279630781	2.50427123482666e-10\\
2.87012372550692	5.69621017021404e-11\\
2.91015465470604	1.90107374287152e-10\\
2.95018558390515	9.99244020860601e-11\\
2.99099747582647	2.10124473376538e-10\\
3.03180936774779	5.12119235906994e-11\\
3.07262125966911	1.58128510285849e-10\\
3.11343315159043	8.48856540613951e-11\\
3.15504317204847	1.70229053075843e-10\\
3.1966531925065	4.53727055926834e-11\\
3.23826321296453	1.26549992707226e-10\\
3.27987323342256	6.99693636363463e-11\\
3.32230262192904	1.3101042473096e-10\\
3.36473201043552	3.95082855320084e-11\\
3.407161398942	9.55632240007276e-11\\
3.44959078744848	5.52726753255683e-11\\
3.49286538309579	9.26662080402707e-11\\
3.5361399787431	3.36720651361588e-11\\
3.57941457439041	6.53028742192419e-11\\
3.62268917003772	4.08659772688225e-11\\
3.66684010640848	5.53228574062814e-11\\
3.71099104277924	2.79074541253976e-11\\
3.75514197915	3.58462148852823e-11\\
3.79929291552076	2.67924571417666e-11\\
3.84435753680729	1.90336635341737e-11\\
3.88942215809383	2.22483143019758e-11\\
3.93448677938037	7.21167570105763e-12\\
3.9795514006669	1.30678801113504e-11\\
4.02557447530949	1.62244662149647e-11\\
4.07159754995208	1.67189595501327e-11\\
4.11762062459467	2.0645374299022e-11\\
4.16364369923726	3.22075699443758e-13\\
4.21041243310513	4.96264140892322e-11\\
4.257181166973	9.4592111921088e-12\\
4.30394990084087	4.50202097823649e-11\\
4.35071863470874	1.23285825992525e-11\\
4.39747966483664	7.40306704827276e-11\\
4.44424069496454	2.66353605837821e-12\\
4.49100172509244	6.24006402105692e-11\\
4.53776275522034	2.1877499811751e-11\\
4.58452386492069	9.14487374714668e-11\\
4.63128497462103	2.54951615374921e-12\\
4.67804608432138	7.46460671052773e-11\\
4.72480719402173	2.87936341436534e-11\\
4.77156830290013	1.03249186977905e-10\\
4.81832941177852	6.47393250119421e-12\\
4.86509052065691	8.27782287160517e-11\\
4.91185162953531	3.35981242827188e-11\\
4.95861273842228	1.10590536728239e-10\\
5.00537384730925	9.35618249542358e-12\\
5.05213495619622	8.76534400617857e-11\\
5.09889606508318	3.67204044948721e-11\\
5.14565717397096	1.14424580921479e-10\\
5.19241828285873	1.13998810391536e-11\\
5.2391793917465	8.99745833393695e-11\\
5.28594050063428	3.85139697911541e-11\\
5.33270160952073	1.15530918165518e-10\\
5.37946271840718	1.27733379429174e-11\\
5.42622382729363	9.03174202093737e-11\\
5.47298493618008	3.92689214478992e-11\\
5.51974604506734	1.14546150342676e-10\\
5.56650715395461	1.36153310847931e-11\\
5.61326826284187	8.91505758104927e-11\\
5.66002937172913	3.922207003626e-11\\
5.70679048061676	1.11987530360125e-10\\
5.7535515895044	1.40399913917122e-11\\
5.80031269839203	8.68549676624752e-11\\
5.84707380727967	3.85664833402188e-11\\
5.89383491616663	1.08273501275846e-10\\
5.9405960250536	1.41406886200457e-11\\
5.98735713394057	8.37379054985377e-11\\
6.03411824282754	3.74585917839454e-11\\
6.08087935171415	1.03740682710907e-10\\
6.12764046060076	1.39933620246779e-11\\
6.17440156948738	8.00461918970541e-11\\
6.22116267837399	3.60246277253395e-11\\
6.26792378726144	9.86574155703579e-11\\
6.3146848961489	1.3659962050383e-11\\
6.36144600503635	7.59766694002906e-11\\
6.4082071139238	3.43658435042471e-11\\
6.45496822281113	9.32371957418354e-11\\
6.50172933169847	1.31905597555715e-11\\
6.5484904405858	7.1685102298602e-11\\
6.59525154947314	3.25627302899534e-11\\
6.64201265836067	8.76471117905453e-11\\
6.68877376724821	1.26250121468274e-11\\
6.73553487613575	6.72927824574288e-11\\
6.78229598502329	3.06784042614083e-11\\
6.82916482319815	8.24652013342586e-11\\
6.87603366137301	9.46298595039252e-12\\
6.92290249954786	6.70298261340463e-11\\
6.96977133772272	3.0722480115486e-11\\
7.01788255937152	8.77410921695798e-11\\
7.06599378102033	1.03533293049907e-11\\
7.11410500266913	7.11812830900271e-11\\
7.16221622431793	3.27526894494667e-11\\
7.21159336850751	9.27568022390801e-11\\
7.26097051269708	1.11294307103549e-11\\
7.31034765688666	7.52666817760428e-11\\
7.35972480107623	3.47656903265658e-11\\
7.41043619055615	9.7901797779798e-11\\
7.46114758003607	1.19361742711988e-11\\
7.511858969516	7.9460715785018e-11\\
7.56257035899592	3.68406971595903e-11\\
7.61468820401402	1.0317274812266e-10\\
7.66680604903213	1.2770617896507e-11\\
7.71892389405023	8.37667712971779e-11\\
7.77104173906833	3.89802634614966e-11\\
7.8246442387161	1.08582587365902e-10\\
7.87824673836387	1.363548163269e-11\\
7.93184923801164	8.81962836096761e-11\\
7.98545173765941	4.1190884036979e-11\\
8.04062368192306	1.14145470853089e-10\\
8.09579562618671	1.45335965484605e-11\\
8.15096757045037	9.27618537538422e-11\\
8.20613951471402	4.3479664313395e-11\\
8.26297303935407	1.19877441306926e-10\\
8.31980656399412	1.54682933128925e-11\\
8.37664008863416	9.74780256512986e-11\\
8.43347361327421	4.58548754522781e-11\\
8.49206906178453	1.25796761896169e-10\\
8.55066451029485	1.6443235661967e-11\\
8.60925995880517	1.02361286113961e-10\\
8.66785540731549	4.83257323047326e-11\\
8.7283223218246	1.31924471347133e-10\\
8.7887892363337	1.74626979543291e-11\\
8.84925615084281	1.074305644444e-10\\
8.90972306535191	5.09029485229462e-11\\
8.97139693657501	1.31856192631119e-10\\
9.03307080779811	2.33627006629433e-11\\
9.09474467902121	9.96397409025462e-11\\
9.15641855024431	4.77817230226663e-11\\
9.21815611731334	1.1724288206949e-10\\
9.27989368438236	2.07439621036087e-11\\
9.34163125145139	8.87308004848819e-11\\
9.40336881852041	4.25899870926116e-11\\
9.46537482585172	1.05220610002732e-10\\
9.52738083318303	1.56011314977889e-11\\
9.58938684051434	8.41509639748494e-11\\
9.65139284784564	4.04002387099922e-11\\
9.71505322212012	1.06126218923919e-10\\
9.77871359639459	1.59586788228694e-11\\
9.84237397066907	8.48087711169399e-11\\
9.90603434494354	4.08381661820556e-11\\
9.97140380506578	1.06629538532133e-10\\
10.036773265188	1.6147028158997e-11\\
10.1021427253103	8.53082604557187e-11\\
10.1675121854325	4.12000988880834e-11\\
10.2347010682591	1.07180486708103e-10\\
10.3018899510858	1.6350809595167e-11\\
10.3690788339124	8.58424442640171e-11\\
10.436267716739	4.15824041866131e-11\\
10.505392885882	1.07764019929846e-10\\
10.574518055025	1.65644165051049e-11\\
10.643643224168	8.64053273375021e-11\\
10.712768393311	4.19826950981417e-11\\
10.7839560546811	1.08381470464991e-10\\
10.8551437160513	1.67890146229865e-11\\
10.9263313774214	8.69980754103494e-11\\
10.9975190387916	4.24020263345426e-11\\
11.07090572606	1.09033226891597e-10\\
11.1442924133284	1.70255476383829e-11\\
11.2176791005968	8.76211880829203e-11\\
11.2910657878652	4.28414526076892e-11\\
11.3667995476522	1.09718512053547e-10\\
11.4425333074392	1.72750147520162e-11\\
11.5182670672261	8.82744433106097e-11\\
11.5940008270131	4.33016400513964e-11\\
11.6722425439244	1.10435771638606e-10\\
11.7504842608358	1.75383596534573e-11\\
11.8287259777471	8.89565088257882e-11\\
11.9069676946584	4.37828107102689e-11\\
11.9878925800102	1.11181674977701e-10\\
12.068817465362	1.78168035880333e-11\\
12.1497423507139	8.96648866266503e-11\\
12.2306672360657	4.42847425397019e-11\\
12.3144665036616	1.11950670955707e-10\\
12.3982657712575	1.81112347341639e-11\\
12.4820650388535	9.03949137764926e-11\\
12.5658643064494	4.48063253166708e-11\\
12.652747062149	1.12734210855336e-10\\
12.7396298178485	1.84228188260249e-11\\
12.826512573548	9.11392072922013e-11\\
12.9133953292476	4.53453941062776e-11\\
13.0035907048582	1.135193050672e-10\\
13.0937860804688	1.87524440420361e-11\\
13.1839814560794	9.18868314769838e-11\\
13.27417683169	4.58982296613897e-11\\
13.3679363981715	1.14287357355636e-10\\
13.461695964653	1.91011095829197e-11\\
13.5554555311345	9.26215215635295e-11\\
13.649215097616	4.64588367776742e-11\\
13.7468155979555	1.15011722368052e-10\\
13.8444160982949	1.94692595378854e-11\\
13.9420165986344	9.33201294017749e-11\\
14.0396170989738	4.70180561151778e-11\\
14.1413635477079	1.15655263144276e-10\\
14.2431099964419	1.98575045295968e-11\\
14.344856445176	9.39495703455862e-11\\
14.44660289391	4.75626205087565e-11\\
14.5528320987134	1.17244519648452e-10\\
14.6590613035167	2.84726409116587e-11\\
14.7652905083201	8.310344079554e-11\\
14.8715197131234	4.26394475283587e-11\\
14.9753724014636	8.3321155530669e-11\\
15.0792250898038	2.08169592674778e-11\\
15.183077778144	5.82672243787385e-11\\
15.2869304664842	3.00280911247341e-11\\
15.3888306269013	5.96701021926549e-11\\
15.4907307873185	1.24704135906484e-11\\
15.5926309477356	4.53015125412293e-11\\
15.6945311081528	2.31303032283137e-11\\
15.7977993001492	5.10944064835428e-11\\
15.9010674921457	1.06314401726593e-11\\
16.0043356841422	3.89652476950886e-11\\
16.1076038761386	1.99265048905772e-11\\
16.2124869885142	4.40428249426361e-11\\
16.3173701008898	9.16602904688091e-12\\
16.4222532132654	3.37133654326749e-11\\
16.527136325641	1.72872827164383e-11\\
16.6339006004935	3.81736309229552e-11\\
16.7406648753461	7.996575623892e-12\\
16.8474291501986	2.92923740818907e-11\\
16.9541934250511	1.5079687498698e-11\\
17.0630876639797	3.31571159417621e-11\\
17.1719819029082	7.0368710858304e-12\\
17.2808761418368	2.54770649021907e-11\\
17.3897703807653	1.31868960195902e-11\\
17.5010359523856	2.87649071228913e-11\\
17.6123015240059	6.23071039207446e-12\\
17.7235670956262	2.21074270001509e-11\\
17.8348326672466	1.15265019751121e-11\\
17.9487100003084	2.48295828342293e-11\\
18.0625873333702	5.53920798118668e-12\\
18.1764646664321	1.90633620000824e-11\\
18.2903419994939	1.00373320766067e-11\\
18.4070757021241	2.12220796491636e-11\\
18.5238094047543	4.93433072179528e-12\\
18.6405431073846	1.62503899225896e-11\\
18.7572768100148	8.67161897843971e-12\\
18.8771200075905	1.78379255810768e-11\\
18.9969632051661	4.39501213200799e-12\\
19.1168064027418	1.35901290221341e-11\\
19.2366496003175	7.39014405226612e-12\\
19.3598678839536	1.45878031876379e-11\\
19.4830861675897	3.90484866663598e-12\\
19.6063044512258	1.10139120046426e-11\\
19.7295227348619	6.15907325141052e-12\\
19.8563979903555	1.1390749454776e-11\\
19.9832732458491	3.45040662708129e-12\\
20.1101485013428	8.45820635753114e-12\\
20.2370237568364	4.94734808675901e-12\\
20.3678580796439	8.16915979306998e-12\\
20.4986924024514	3.02025071619028e-12\\
20.6295267252589	5.8605065245132e-12\\
20.7603610480664	3.72477049204178e-12\\
20.8954809190363	4.8439863231664e-12\\
21.0306007900061	2.60411137098515e-12\\
21.1657206609759	3.15572568077016e-12\\
21.3008405319458	2.46033748929619e-12\\
21.4406014192669	1.33026922810586e-12\\
21.5803623065881	2.19213536212237e-12\\
21.7201231939092	2.72712408211362e-13\\
21.8598840812304	1.12035380972486e-12\\
22.0046756408933	2.46730413877572e-12\\
22.1494672005562	1.77462211592427e-12\\
22.2942587602192	2.87000978538288e-12\\
22.4390503198821	3.3355262996082e-13\\
22.5893023876591	6.66040833596782e-12\\
22.7395544554361	1.34131594720088e-12\\
22.8898065232131	6.3691274476696e-12\\
23.0400585909901	1.94670668474117e-12\\
23.1962481938828	1.13839909721136e-11\\
23.3524377967755	8.80823192161984e-13\\
23.5086273996682	1.03428932085592e-11\\
23.6648170025609	3.77442521681814e-12\\
23.8274766689118	1.68055708238413e-11\\
23.9901363352628	3.80168119207269e-13\\
24.1527960016138	1.49396467419294e-11\\
24.3154556679648	5.886083287443e-12\\
24.4851833872328	2.3137713967003e-11\\
24.6549111065008	1.76136882856781e-13\\
24.8246388257688	2.03494859851716e-11\\
24.9943665450368	8.37062363867602e-12\\
25.1718378270833	3.06556308116157e-11\\
25.3493091091298	8.06979483236603e-13\\
25.5267803911763	2.68204625175628e-11\\
25.7042516732229	1.13442588656198e-11\\
25.8902342929274	3.97216148861901e-11\\
26.0762169126319	1.53586865447863e-12\\
26.2621995323364	3.46821044105994e-11\\
26.4481821520409	1.49617263023316e-11\\
26.6435543528394	5.0820861408063e-11\\
26.838926553638	2.39271102930871e-12\\
27.0342987544366	4.43792363968853e-11\\
27.2296709552352	1.94323446223166e-11\\
27.4354441469161	6.46136200099789e-11\\
27.641217338597	3.41610073562038e-12\\
27.8469905302779	5.65224811399645e-11\\
28.0527637219589	2.5044355478343e-11\\
28.2701110770545	8.20134654633264e-11\\
28.4874584321501	4.6564418987316e-12\\
28.7048057872458	7.19644760449611e-11\\
28.9221531423414	3.22014012676952e-11\\
29.1524460226538	1.04306716042224e-10\\
29.3827389029661	6.18038259458942e-12\\
29.6130317832784	9.19173487323732e-11\\
29.8433246635908	4.14794934511242e-11\\
30.088179573096	1.3333817383554e-10\\
30.3330344826013	8.07678923742117e-12\\
30.5778893921066	1.18138047955352e-10\\
30.8227443016119	5.37166144898293e-11\\
31.0840841898221	1.71807464088847e-10\\
31.3454240780323	1.04649205967533e-11\\
31.6067639662425	1.53227493193686e-10\\
31.8681038544528	7.01585109452729e-11\\
32.1482392887021	2.2375275937625e-10\\
32.4283747229513	1.3505141449599e-11\\
32.7085101572006	2.01126559851161e-10\\
32.9886455914499	9.27011106766074e-11\\
33.2903839143714	2.95363344893218e-10\\
33.5921222372929	1.74120023954671e-11\\
33.8938605602144	2.67962603495153e-10\\
34.195598883136	1.24306481774639e-10\\
34.5223935450538	4.01234732938516e-10\\
34.8491882069717	5.69309530074058e-11\\
35.1759828688896	3.16591815263667e-10\\
35.5027775308075	1.46671435413692e-10\\
35.8301107995782	3.6366907749108e-10\\
36.1574440683489	6.11421364926645e-11\\
36.4847773371196	2.75706863589464e-10\\
36.8121106058903	1.29458437025942e-10\\
37.1394383112246	3.20691292965414e-10\\
37.4667660165588	5.41232267337044e-11\\
37.7940937218931	2.43074387007436e-10\\
38.1214214272273	1.14255563343768e-10\\
38.448749190017	2.82360822223549e-10\\
38.7760769528068	4.78239045587081e-11\\
39.1034047155965	2.13933305098779e-10\\
39.4307324783862	1.00636783628705e-10\\
39.7580602405942	2.48315305817481e-10\\
40.0853880028023	4.21571388908859e-11\\
40.4127157650103	1.88087257574354e-10\\
40.7400435272184	8.85248714610487e-11\\
41.0673712894278	2.18200506274524e-10\\
41.3946990516372	3.71032475576794e-11\\
41.7220268138466	1.65246514388651e-10\\
42.049354576056	7.78020668723567e-11\\
42.3766823382676	1.91635034230142e-10\\
42.7040101004792	3.26207325040961e-11\\
43.0313378626908	1.45110257143699e-10\\
43.3586656249024	6.8337615080627e-11\\
43.6859933871135	1.68243063577989e-10\\
44.0133211493247	2.86594185883171e-11\\
44.3406489115359	1.27386793821738e-10\\
44.667976673747	6.00005756884459e-11\\
44.9953044359533	1.47670483480189e-10\\
45.3226321981596	2.51671287976318e-11\\
45.649959960366	1.1180383875653e-10\\
45.9772877225723	5.26664996336468e-11\\
46.3046154847852	1.29592225839303e-10\\
46.6319432469981	2.20932872690982e-11\\
46.9592710092111	9.81127904614576e-11\\
47.286598771424	4.62205239359914e-11\\
47.6139265336322	1.13701796242149e-10\\
47.9412542958405	1.92974455859307e-11\\
48.2685820580487	8.62164437065438e-11\\
48.595909820257	4.06188190721357e-11\\
48.926850153611	1.04059299371739e-10\\
49.257790486965	1.41316481086506e-11\\
49.5887308203189	8.37423273086779e-11\\
49.9196711536729	3.91679275818513e-11\\
50.2593448649822	1.04238591716599e-10\\
50.5990185762914	1.44063372342629e-11\\
50.9386922876007	8.37187099159142e-11\\
51.2783659989099	3.92415752670239e-11\\
51.6270508037435	1.04074482538963e-10\\
51.9757356085771	1.45003184806591e-11\\
52.3244204134106	8.36059390121879e-11\\
52.6731052182442	3.92657972109189e-11\\
53.0312891287569	1.03911984919436e-10\\
53.3894730392697	1.46015621468876e-11\\
53.7476569497824	8.34944552732808e-11\\
54.1058408602952	3.92957342013056e-11\\
54.4740437546715	1.03738118789609e-10\\
54.8422466490478	1.47037364922609e-11\\
55.2104495434241	8.33803625102658e-11\\
55.5786524378004	3.93292698755432e-11\\
55.95743714694	1.0355274797369e-10\\
56.3362218560795	1.48074651498686e-11\\
56.7150065652191	8.3263203623507e-11\\
57.0937912743587	3.93663769117369e-11\\
57.4837682235858	1.03355208506267e-10\\
57.8737451728129	1.4913118483173e-11\\
58.26372212204	8.31423692913835e-11\\
58.653699071267	3.94069542622444e-11\\
58.9902743034503	5.30704295613471e-11\\
59.3268495356335	5.3343428198932e-11\\
59.6634247678168	1.83439461537427e-11\\
60	1.62166764391958e-12\\
};
\addlegendentry{Infetti}

\end{axis}
\end{tikzpicture}%}}
\\
\subfloat[][Nodo 3]
{\resizebox{0.45\textwidth}{!}
{% This file was created by matlab2tikz.
%
%The latest updates can be retrieved from
%  http://www.mathworks.com/matlabcentral/fileexchange/22022-matlab2tikz-matlab2tikz
%where you can also make suggestions and rate matlab2tikz.
%
\definecolor{mycolor1}{rgb}{0.00000,0.44700,0.74100}%
\definecolor{mycolor2}{rgb}{0.85000,0.32500,0.09800}%
%
\begin{tikzpicture}

\begin{axis}[%
width=6.028in,
height=4.754in,
at={(1.011in,0.642in)},
scale only axis,
xmin=0,
xmax=60,
xlabel style={font=\color{white!15!black}},
xlabel={T},
ymin=0,
ymax=6e-10,
ylabel style={font=\color{white!15!black}},
ylabel={Errore assoluto},
axis background/.style={fill=white},
legend style={legend cell align=left, align=left, draw=white!15!black}
]
\addplot [color=mycolor1, line width=2.0pt]
  table[row sep=crcr]{%
0	0\\
0.000265404780368998	0\\
0.000530809560737996	0\\
0.000796214341106994	0\\
0.00106161912147599	0\\
0.00238864302332098	1.11022302462516e-16\\
0.00371566692516597	7.7715611723761e-16\\
0.00504269082701096	1.22124532708767e-15\\
0.00636971472885596	6.66133814775094e-16\\
0.0130048342380809	3.99014155050281e-13\\
0.0196399537473059	2.47679654563626e-12\\
0.0262750732565308	3.25750537655267e-12\\
0.0329101927657558	1.78224102143076e-12\\
0.0481555938880245	3.89315246707156e-11\\
0.0634009950102933	1.06853415005048e-11\\
0.0786463961325621	5.16733322797336e-11\\
0.0938917972548308	2.28428387316626e-11\\
0.112426119267152	1.06134878663511e-10\\
0.130960441279472	7.17204073907851e-12\\
0.149494763291793	1.11880948949761e-10\\
0.168029085304114	5.00134378356165e-11\\
0.18883016427751	1.83187021107756e-10\\
0.209631243250907	3.58379992349001e-12\\
0.230432322224303	1.72891923000407e-10\\
0.2512334011977	7.84514675444825e-11\\
0.273784345742275	2.57950327764433e-10\\
0.29633529028685	1.5595635893817e-11\\
0.318886234831425	2.30505392551095e-10\\
0.341437179376	1.05659703208971e-10\\
0.365441427620669	3.26388582827519e-10\\
0.389445675865339	2.76717537772697e-11\\
0.413449924110009	2.82246115368423e-10\\
0.437454172354678	1.30386368368818e-10\\
0.462718995465789	3.86338183666624e-10\\
0.487983818576899	3.91074950201187e-11\\
0.513248641688009	3.26830673635925e-10\\
0.538513464799119	1.51952561644464e-10\\
0.564903949168048	4.36734315378828e-10\\
0.591294433536978	4.94934093708821e-11\\
0.617684917905907	3.63671426306666e-10\\
0.644075402274836	1.70021663414843e-10\\
0.671492335525317	4.77198947024249e-10\\
0.698909268775797	5.85946846598517e-11\\
0.726326202026278	3.92633925372365e-10\\
0.753743135276759	1.84480541953747e-10\\
0.782111039573039	5.07810127281516e-10\\
0.810478943869319	6.62910837334607e-11\\
0.838846848165599	4.13888368022697e-10\\
0.867214752461879	1.95368166089338e-10\\
0.896474834868797	5.28953991718595e-10\\
0.925734917275715	7.25416393621003e-11\\
0.954994999682632	4.27813118264453e-10\\
0.98425508208955	2.02828531747912e-10\\
1.01436081133407	5.41223732497542e-10\\
1.04446654057859	7.73612285343006e-11\\
1.0745722698231	4.34923985714875e-10\\
1.10467799906762	2.07076578107035e-10\\
1.13559219349808	5.45345879565673e-10\\
1.16650638792855	8.08044742228731e-11\\
1.19742058235901	4.35823932498636e-10\\
1.22833477678948	2.08373984733612e-10\\
1.26002761487183	5.42126565861167e-10\\
1.29172045295419	8.29537549762449e-11\\
1.32341329103654	4.31164548508889e-10\\
1.35510612911889	2.07009853703255e-10\\
1.38755375131887	5.32409449860438e-10\\
1.42000137351885	8.39094349558422e-11\\
1.45244899571883	4.2161740765323e-10\\
1.48489661791881	2.03287275901687e-10\\
1.51811813384306	5.193782071089e-10\\
1.5513396497673	8.28700441601882e-11\\
1.58456116569155	4.11392031551827e-10\\
1.61778268161579	1.99051664040439e-10\\
1.65178046131382	5.03212693736543e-10\\
1.68577824101184	8.27061752417535e-11\\
1.71977602070987	3.96937815949627e-10\\
1.7537738004079	1.92981075564091e-10\\
1.78853445564131	4.82161199855113e-10\\
1.82329511087471	8.15173484269849e-11\\
1.85805576610812	3.78912456966418e-10\\
1.89281642134153	1.85144899411682e-10\\
1.92833097625931	4.57089810446121e-10\\
1.96384553117709	7.94829757566617e-11\\
1.99936008609487	3.57924134775089e-10\\
2.03487464101265	1.75832792770336e-10\\
2.07113752249011	4.28740487556922e-10\\
2.10740040396758	7.67157448677835e-11\\
2.14366328544504	3.34547833880094e-10\\
2.1799261669225	1.65313152056257e-10\\
2.21693501810426	3.97809285512807e-10\\
2.25394386928602	7.33275107300813e-11\\
2.29095272046778	3.09317904623185e-10\\
2.32796157164954	1.53837664829126e-10\\
2.3657171313366	3.64935082153295e-10\\
2.40347269102366	6.94256319100361e-11\\
2.44122825071071	2.82721512867568e-10\\
2.47898381039777	1.4163742401152e-10\\
2.51748987308115	3.30695526518099e-10\\
2.55599593576452	6.51107501248305e-11\\
2.5945019984479	2.55196697107607e-10\\
2.63300806113127	1.28921706643581e-10\\
2.67227151262563	2.95603652666898e-10\\
2.71153496411999	6.04761241085328e-11\\
2.75079841561434	2.27130148022781e-10\\
2.7900618671087	1.15876475081933e-10\\
2.83009279630781	2.60107380078978e-10\\
2.87012372550692	5.56066304113756e-11\\
2.91015465470604	1.9885648683271e-10\\
2.95018558390515	1.02664432510835e-10\\
2.99099747582647	2.24588958541716e-10\\
3.03180936774779	5.0578596866302e-11\\
3.07262125966911	1.70659042453281e-10\\
3.11343315159043	8.94241347637603e-11\\
3.15504317204847	1.89365967351307e-10\\
3.1966531925065	4.54591364551504e-11\\
3.23826321296453	1.42769907007789e-10\\
3.27987323342256	7.6270822990665e-11\\
3.32230262192904	1.54692370024634e-10\\
3.36473201043552	4.03065358867138e-11\\
3.407161398942	1.15371212583426e-10\\
3.44959078744848	6.32963681468368e-11\\
3.49286538309579	1.20759791055747e-10\\
3.5361399787431	3.5170200085588e-11\\
3.57941457439041	8.85955198093313e-11\\
3.62268917003772	5.05692154817439e-11\\
3.66684010640848	8.7697460404712e-11\\
3.71099104277924	3.00905411698693e-11\\
3.75514197915	6.25260954123519e-11\\
3.79929291552076	3.81348841393958e-11\\
3.84435753680729	5.55722690087634e-11\\
3.88942215809383	2.50997000961206e-11\\
3.93448677938037	3.71956909717142e-11\\
3.9795514006669	2.60148569353191e-11\\
4.02557447530949	2.43860487358916e-11\\
4.07159754995208	2.02216021705226e-11\\
4.11762062459467	1.25835453168577e-11\\
4.16364369923726	1.42054146223813e-11\\
4.21041243310513	5.8795190938099e-12\\
4.257181166973	1.47871159761337e-11\\
4.30394990084087	1.10385589557893e-11\\
4.35071863470874	2.6771918015811e-12\\
4.39747966483664	3.14919201827024e-11\\
4.44424069496454	8.08497713222778e-12\\
4.49100172509244	2.94467783490404e-11\\
4.53776275522034	7.20246084995324e-12\\
4.58452386492069	5.04670749634784e-11\\
4.63128497462103	2.87098123052942e-12\\
4.67804608432138	4.29807855972797e-11\\
4.72480719402173	1.45960465935957e-11\\
4.77156830290013	6.40961728137768e-11\\
4.81832941177852	1.13975495708019e-12\\
4.86509052065691	5.25909316095863e-11\\
4.91185162953531	1.99869565342681e-11\\
4.95861273842228	7.34435290361546e-11\\
5.00537384730925	4.17149648157533e-12\\
5.05213495619622	5.90645310438731e-11\\
5.09889606508318	2.37692088234098e-11\\
5.14565717397096	7.93868026871536e-11\\
5.19241828285873	6.4105110109125e-12\\
5.2391793917465	6.30504259913067e-11\\
5.28594050063428	2.62682653406898e-11\\
5.33270160952073	8.26475277104777e-11\\
5.37946271840718	8.01103627878774e-12\\
5.42622382729363	6.50819398373415e-11\\
5.47298493618008	2.77519118796477e-11\\
5.51974604506734	8.38173697115252e-11\\
5.56650715395461	9.1006646663061e-12\\
5.61326826284187	6.55955012529574e-11\\
5.66002937172913	2.84401391326128e-11\\
5.70679048061676	8.33795255061887e-11\\
5.7535515895044	9.78458980505081e-12\\
5.80031269839203	6.49473808067569e-11\\
5.84707380727967	2.85128309851501e-11\\
5.89383491616663	8.17269307784585e-11\\
5.9405960250536	1.01491592907621e-11\\
5.98735713394057	6.34267915966547e-11\\
6.03411824282754	2.81165923876614e-11\\
6.08087935171415	7.91778032027679e-11\\
6.12764046060076	1.02652886191379e-11\\
6.17440156948738	6.12676576139393e-11\\
6.22116267837399	2.73704114928108e-11\\
6.26792378726144	7.59881602085954e-11\\
6.3146848961489	1.01907648986099e-11\\
6.36144600503635	5.86578008388017e-11\\
6.4082071139238	2.63702393255016e-11\\
6.45496822281113	7.23629767218625e-11\\
6.50172933169847	9.97252280754424e-12\\
6.5484904405858	5.57472956685956e-11\\
6.59525154947314	2.51930976080672e-11\\
6.64201265836067	6.84647616377987e-11\\
6.68877376724821	9.64867075126108e-12\\
6.73553487613575	5.26549082113803e-11\\
6.78229598502329	2.39002984070424e-11\\
6.82916482319815	6.4772909258437e-11\\
6.87603366137301	7.25658422240372e-12\\
6.92290249954786	5.27233812164241e-11\\
6.96977133772272	2.40805431150903e-11\\
7.01788255937152	6.92392254642016e-11\\
7.06599378102033	8.00293165070798e-12\\
7.11410500266913	5.62420665595198e-11\\
7.16221622431793	2.57981691564879e-11\\
7.21159336850751	7.35034255683331e-11\\
7.26097051269708	8.66109961528139e-12\\
7.31034765688666	5.97107363553562e-11\\
7.35972480107623	2.75042211228538e-11\\
7.41043619055615	7.78713482585403e-11\\
7.46114758003607	9.34410882003078e-12\\
7.511858969516	6.32669749922599e-11\\
7.56257035899592	2.92604829255083e-11\\
7.61468820401402	8.23408008443494e-11\\
7.66680604903213	1.00495445298776e-11\\
7.71892389405023	6.69138910946998e-11\\
7.77104173906833	3.10692582772276e-11\\
7.8246442387161	8.69225802446749e-11\\
7.87824673836387	1.07797382131736e-11\\
7.93184923801164	7.06609770695366e-11\\
7.98545173765941	3.29359040041055e-11\\
8.04062368192306	9.16285103347292e-11\\
8.09579562618671	1.15369380715435e-11\\
8.15096757045037	7.45188355466553e-11\\
8.20613951471402	3.48663320437481e-11\\
8.26297303935407	9.6472246857715e-11\\
8.31980656399412	1.23238919069735e-11\\
8.37664008863416	7.84996512237512e-11\\
8.43347361327421	3.68673702677569e-11\\
8.49206906178453	1.01469388447128e-10\\
8.55066451029485	1.31435695660542e-11\\
8.60925995880517	8.26176349555396e-11\\
8.66785540731549	3.89468457484554e-11\\
8.7283223218246	1.06638170516149e-10\\
8.7887892363337	1.39993849845865e-11\\
8.84925615084281	8.68890792649069e-11\\
8.90972306535191	4.11139178257969e-11\\
8.97139693657501	1.06793573984021e-10\\
9.03307080779811	1.88154769542592e-11\\
9.09474467902121	8.07490463383687e-11\\
9.15641855024431	3.86704002153238e-11\\
9.21815611731334	9.51361767143055e-11\\
9.27989368438236	1.67453828581188e-11\\
9.34163125145139	7.20392634434575e-11\\
9.40336881852041	3.45352357822293e-11\\
9.46537482585172	8.55275850142334e-11\\
9.52738083318303	1.26068322447992e-11\\
9.58938684051434	6.84343137713483e-11\\
9.65139284784564	3.28187199638563e-11\\
9.71505322212012	8.6399915000257e-11\\
9.77871359639459	1.29192767595043e-11\\
9.84237397066907	6.90766888133965e-11\\
9.90603434494354	3.3226948970011e-11\\
9.97140380506578	8.69434246819623e-11\\
10.036773265188	1.30929433961313e-11\\
10.1021427253103	6.95897506286514e-11\\
10.1675121854325	3.3572838953333e-11\\
10.2347010682591	8.75267358591003e-11\\
10.3018899510858	1.32778787964583e-11\\
10.3690788339124	7.01327051988443e-11\\
10.436267716739	3.39354655487512e-11\\
10.505392885882	8.81408279695961e-11\\
10.574518055025	1.34697530906891e-11\\
10.643643224168	7.07034153446529e-11\\
10.712768393311	3.43140793557239e-11\\
10.7839560546811	8.87907247726361e-11\\
10.8551437160513	1.36697597685753e-11\\
10.9263313774214	7.13062386914487e-11\\
10.9975190387916	3.47111783760568e-11\\
11.07090572606	8.94812557383773e-11\\
11.1442924133284	1.38789535419903e-11\\
11.2176791005968	7.19454495978766e-11\\
11.2910657878652	3.51291773448281e-11\\
11.3667995476522	9.02177776929136e-11\\
11.4425333074392	1.40984446339587e-11\\
11.5182670672261	7.26257110006401e-11\\
11.5940008270131	3.55707685528728e-11\\
11.6722425439244	9.10058417513682e-11\\
11.7504842608358	1.43295098009588e-11\\
11.8287259777471	7.33521021700767e-11\\
11.9069676946584	3.60388663356304e-11\\
11.9878925800102	9.1851581895952e-11\\
12.068817465362	1.45733980438933e-11\\
12.1497423507139	7.41300354434316e-11\\
12.2306672360657	3.65365515619942e-11\\
12.3144665036616	9.27614651757835e-11\\
12.3982657712575	1.48316081638455e-11\\
12.4820650388535	7.49653672471595e-11\\
12.5658643064494	3.70671826566138e-11\\
12.652747062149	9.37425137514936e-11\\
12.7396298178485	1.5105611206323e-11\\
12.826512573548	7.58643425857741e-11\\
12.9133953292476	3.76343400887436e-11\\
13.0035907048582	9.48021383617714e-11\\
13.0937860804688	1.53972390393164e-11\\
13.1839814560794	7.68335950418475e-11\\
13.27417683169	3.82418818833941e-11\\
13.3679363981715	9.59479995454871e-11\\
13.461695964653	1.57082680196652e-11\\
13.5554555311345	7.78800635092836e-11\\
13.649215097616	3.88939158657564e-11\\
13.7468155979555	9.7188146419569e-11\\
13.8444160982949	1.60408353266916e-11\\
13.9420165986344	7.90108534154399e-11\\
14.0396170989738	3.9594744150051e-11\\
14.1413635477079	9.85304893230676e-11\\
14.2431099964419	1.63971336508695e-11\\
14.344856445176	8.02332089655522e-11\\
14.44660289391	4.03487798728008e-11\\
14.5528320987134	1.00914637757299e-10\\
14.6590613035167	2.38614406011806e-11\\
14.7652905083201	7.17640669112285e-11\\
14.8715197131234	3.64679120234968e-11\\
14.9753724014636	7.27198301575527e-11\\
15.0792250898038	1.76065828583205e-11\\
15.183077778144	5.10618769489213e-11\\
15.2869304664842	2.60055310619123e-11\\
15.3888306269013	5.29564170292929e-11\\
15.4907307873185	1.05691011498266e-11\\
15.5926309477356	4.03836686313497e-11\\
15.6945311081528	2.0349888441018e-11\\
15.7977993001492	4.62287708113962e-11\\
15.9010674921457	9.08639830043967e-12\\
16.0043356841422	3.54471452190808e-11\\
16.1076038761386	1.78404790940334e-11\\
16.2124869885142	4.08002798213403e-11\\
16.3173701008898	7.91344767492319e-12\\
16.4222532132654	3.14417936131406e-11\\
16.527136325641	1.58127400062824e-11\\
16.6339006004935	3.64034913324929e-11\\
16.7406648753461	6.98913149577152e-12\\
16.8474291501986	2.81682177583065e-11\\
16.9541934250511	1.41611444792744e-11\\
17.0630876639797	3.2773395108876e-11\\
17.1719819029082	6.24306162322341e-12\\
17.2808761418368	2.54470333693746e-11\\
17.3897703807653	1.27924615345165e-11\\
17.5010359523856	2.97307733987395e-11\\
17.6123015240059	5.62958013539117e-12\\
17.7235670956262	2.31541730233431e-11\\
17.8348326672466	1.16425202811854e-11\\
17.9487100003084	2.71487277103688e-11\\
18.0625873333702	5.11762854316089e-12\\
18.1764646664321	2.12003470334565e-11\\
18.2903419994939	1.06653852416372e-11\\
18.4070757021241	2.49344433989052e-11\\
18.5238094047543	4.68519667506939e-12\\
18.6405431073846	1.9519469374174e-11\\
18.7572768100148	9.82708359131834e-12\\
18.8771200075905	2.30185315253095e-11\\
18.9969632051661	4.31624180841084e-12\\
19.1168064027418	1.80614134759338e-11\\
19.2366496003175	9.10210795623811e-12\\
19.3598678839536	2.13476458732487e-11\\
19.4830861675897	3.9988290456705e-12\\
19.6063044512258	1.67874325551765e-11\\
19.7295227348619	8.47061309983133e-12\\
19.8563979903555	1.98802918571772e-11\\
19.9832732458491	3.72382680247085e-12\\
20.1101485013428	1.56669122119979e-11\\
20.2370237568364	7.91708365532884e-12\\
20.3678580796439	1.85833570753857e-11\\
20.4986924024514	3.4841574070299e-12\\
20.6295267252589	1.46754275398564e-11\\
20.7603610480664	7.42911288043047e-12\\
20.8954809190363	1.7430140664132e-11\\
21.0306007900061	3.27413096634643e-12\\
21.1657206609759	1.37930777910356e-11\\
21.3008405319458	6.99668101233897e-12\\
21.4406014192669	1.63988544965576e-11\\
21.5803623065881	3.08933434389758e-12\\
21.7201231939092	1.30036537093758e-11\\
21.8598840812304	6.61157240067212e-12\\
22.0046756408933	1.54714574485126e-11\\
22.1494672005562	2.9259927813996e-12\\
22.2942587602192	1.22934440405231e-11\\
22.4390503198821	6.26687590710162e-12\\
22.5893023876591	1.46327949757108e-11\\
22.7395544554361	2.7810809211104e-12\\
22.8898065232131	1.16511522652019e-11\\
23.0400585909901	5.95690163862628e-12\\
23.1962481938828	1.38701272689445e-11\\
23.3524377967755	2.65201749449773e-12\\
23.5086273996682	1.10670639319466e-11\\
23.6648170025609	5.67676461393773e-12\\
23.8274766689118	1.31725463869969e-11\\
23.9901363352628	2.53669307781479e-12\\
24.1527960016138	1.05328801236482e-11\\
24.3154556679648	5.42224598554242e-12\\
24.4851833872328	1.25306709453099e-11\\
24.6549111065008	2.43319253634411e-12\\
24.8246388257688	1.00415231685247e-11\\
24.9943665450368	5.18979303976153e-12\\
25.1718378270833	1.19363408046524e-11\\
25.3493091091298	2.33996155785121e-12\\
25.5267803911763	9.58677581763823e-12\\
25.7042516732229	4.97629715212611e-12\\
25.8902342929274	1.13825060488182e-11\\
26.0762169126319	2.25550134125285e-12\\
26.2621995323364	9.16322573374373e-12\\
26.4481821520409	4.77895500949899e-12\\
26.6435543528394	1.0862893917718e-11\\
26.838926553638	2.17859064122194e-12\\
27.0342987544366	8.76609895783531e-12\\
27.2296709552352	4.59529636565037e-12\\
27.4354441469161	1.03719810518044e-11\\
27.641217338597	2.10811923473386e-12\\
27.8469905302779	8.3911766424194e-12\\
28.0527637219589	4.42318404125785e-12\\
28.2701110770545	9.90488246976895e-12\\
28.4874584321501	2.04303240991521e-12\\
28.7048057872458	8.03471178478787e-12\\
28.9221531423414	4.26061963487712e-12\\
29.1524460226538	9.45715727951324e-12\\
29.3827389029661	1.98233096604383e-12\\
29.6130317832784	7.69331820471564e-12\\
29.8433246635908	4.10574352294191e-12\\
30.088179573096	9.0248364337242e-12\\
30.3330344826013	1.92512672470002e-12\\
30.5778893921066	7.36388727773374e-12\\
30.8227443016119	3.95689037091529e-12\\
31.0840841898221	8.60420068526935e-12\\
31.3454240780323	1.870392729586e-12\\
31.6067639662425	7.04367120185623e-12\\
31.8681038544528	3.81236708868471e-12\\
32.1482392887021	8.19189160949918e-12\\
32.4283747229513	1.81735182458453e-12\\
32.7085101572006	6.72997768624839e-12\\
32.9886455914499	3.67059160844008e-12\\
33.2903839143714	7.7847173152179e-12\\
33.5921222372929	1.76497705339784e-12\\
33.8938605602144	6.4202809735292e-12\\
34.195598883136	3.52989859564445e-12\\
34.5223935450538	7.46036565857366e-12\\
34.8491882069717	2.29719021582753e-12\\
35.1759828688896	5.30417376687353e-12\\
35.5027775308075	3.03815306246236e-12\\
35.8301107995782	4.55419035816362e-12\\
36.1574440683489	1.52086676585839e-12\\
36.4847773371196	3.06446534814597e-12\\
36.8121106058903	1.79040116066176e-12\\
37.1394383112246	2.69401168040417e-12\\
37.4667660165588	8.98642271707217e-13\\
37.7940937218931	1.81415993338874e-12\\
38.1214214272273	1.05959685470225e-12\\
38.448749190017	1.5945855746935e-12\\
38.7760769528068	5.31741317644219e-13\\
39.1034047155965	1.07402975402238e-12\\
39.4307324783862	6.27276008913213e-13\\
39.7580602405942	9.43994882263155e-13\\
40.0853880028023	3.14720471905616e-13\\
40.4127157650103	6.35963504080905e-13\\
40.7400435272184	3.71480624039577e-13\\
41.0673712894278	5.58886270596304e-13\\
41.3946990516372	1.8623991238087e-13\\
41.7220268138466	3.76698672255316e-13\\
42.049354576056	2.20046203480706e-13\\
42.3766823382676	3.30901972489528e-13\\
42.7040101004792	1.10217390769662e-13\\
43.0313378626908	2.23154827949656e-13\\
43.3586656249024	1.30395694242225e-13\\
43.6859933871135	1.95898852695109e-13\\
44.0133211493247	6.51978471211123e-14\\
44.3406489115359	1.32227562232856e-13\\
44.667976673747	7.72715225139109e-14\\
44.9953044359533	1.15962794922098e-13\\
45.3226321981596	3.85247389544929e-14\\
45.649959960366	7.84095011141517e-14\\
45.9772877225723	4.5852210917019e-14\\
46.3046154847852	6.8584027346219e-14\\
46.6319432469981	2.27595720048157e-14\\
46.9592710092111	4.64905891561784e-14\\
47.286598771424	2.7228219678932e-14\\
47.6139265336322	4.05786515500495e-14\\
47.9412542958405	1.34059430223488e-14\\
48.2685820580487	2.76445533131664e-14\\
48.595909820257	1.62092561595273e-14\\
48.926850153611	2.50355292052973e-14\\
49.257790486965	7.43849426498855e-15\\
49.5887308203189	1.83464354819307e-14\\
49.9196711536729	1.0547118733939e-14\\
50.2593448649822	1.69864122767649e-14\\
50.5990185762914	5.10702591327572e-15\\
50.9386922876007	1.24344978758018e-14\\
51.2783659989099	7.18869408444789e-15\\
51.6270508037435	1.13797860024079e-14\\
51.9757356085771	3.44169137633799e-15\\
52.3244204134106	8.32667268468867e-15\\
52.6731052182442	4.85722573273506e-15\\
53.0312891287569	7.52176099183544e-15\\
53.3894730392697	2.3037127760972e-15\\
53.7476569497824	5.55111512312578e-15\\
54.1058408602952	3.27515792264421e-15\\
54.4740437546715	4.88498130835069e-15\\
54.8422466490478	1.47104550762833e-15\\
55.2104495434241	3.69149155687865e-15\\
55.5786524378004	2.19269047363468e-15\\
55.95743714694	3.13638004456607e-15\\
56.3362218560795	9.43689570931383e-16\\
56.7150065652191	2.41473507855972e-15\\
57.0937912743587	1.47104550762833e-15\\
57.4837682235858	1.97064586870965e-15\\
57.8737451728129	5.55111512312578e-16\\
58.26372212204	1.60982338570648e-15\\
58.653699071267	9.99200722162641e-16\\
58.9902743034503	5.55111512312578e-16\\
59.3268495356335	5.82867087928207e-16\\
59.6634247678168	1.11022302462516e-16\\
60	2.22044604925031e-16\\
};
\addlegendentry{Sani}

\addplot [color=mycolor2, line width=2.0pt]
  table[row sep=crcr]{%
0	0\\
0.000265404780368998	1.0842021724855e-19\\
0.000530809560737996	5.42101086242752e-20\\
0.000796214341106994	2.43945488809238e-19\\
0.00106161912147599	4.87890977618477e-19\\
0.00238864302332098	2.16840434497101e-17\\
0.00371566692516597	7.86697096355482e-16\\
0.00504269082701096	1.1668183780289e-15\\
0.00636971472885596	7.06682976026052e-16\\
0.0130048342380809	4.3647117170531e-13\\
0.0196399537473059	2.70952271036462e-12\\
0.0262750732565308	3.56387315492146e-12\\
0.0329101927657558	1.9498673509144e-12\\
0.0481555938880245	4.25907156764715e-11\\
0.0634009950102933	1.16701821817333e-11\\
0.0786463961325621	5.6533274589432e-11\\
0.0938917972548308	2.50039641769284e-11\\
0.112426119267152	1.16034216957051e-10\\
0.130960441279472	7.76954889314396e-12\\
0.149494763291793	1.22305221506025e-10\\
0.168029085304114	5.47130396544304e-11\\
0.18883016427751	2.00081902435034e-10\\
0.209631243250907	4.05935701719429e-12\\
0.230432322224303	1.88800988731863e-10\\
0.2512334011977	8.57472842730189e-11\\
0.273784345742275	2.81409118230158e-10\\
0.29633529028685	1.72443725965366e-11\\
0.318886234831425	2.5140124704226e-10\\
0.341437179376	1.15358694818823e-10\\
0.365441427620669	3.55584575673085e-10\\
0.389445675865339	3.04713060339523e-11\\
0.413449924110009	3.07391959109005e-10\\
0.437454172354678	1.42171399653002e-10\\
0.462718995465789	4.20242951548744e-10\\
0.487983818576899	4.29635493848224e-11\\
0.513248641688009	3.55374257798857e-10\\
0.538513464799119	1.65442964883411e-10\\
0.564903949168048	4.74233097236265e-10\\
0.591294433536978	5.42711986462052e-11\\
0.617684917905907	3.94718535634553e-10\\
0.644075402274836	1.84809501035943e-10\\
0.671492335525317	5.17162618285383e-10\\
0.698909268775797	6.41371955545367e-11\\
0.726326202026278	4.25295715311691e-10\\
0.753743135276759	2.001544485708e-10\\
0.782111039573039	5.49142536998559e-10\\
0.810478943869319	7.24316440159356e-11\\
0.838846848165599	4.47313852625086e-10\\
0.867214752461879	2.11529543880928e-10\\
0.896474834868797	5.70623992501851e-10\\
0.925734917275715	7.91123833110419e-11\\
0.954994999682632	4.6120943175687e-10\\
0.98425508208955	2.1910229186517e-10\\
1.01436081133407	5.8228560861906e-10\\
1.04446654057859	8.41994807210256e-11\\
1.0745722698231	4.67570482332036e-10\\
1.10467799906762	2.23118357123298e-10\\
1.13559219349808	5.84951864723848e-10\\
1.16650638792855	8.77565797807733e-11\\
1.19742058235901	4.67079486199395e-10\\
1.22833477678948	2.23874585536521e-10\\
1.26002761487183	5.79532810629502e-10\\
1.29172045295419	8.98778274027734e-11\\
1.32341329103654	4.6047160529028e-10\\
1.35510612911889	2.2169738267408e-10\\
1.38755375131887	5.6697840866704e-10\\
1.42000137351885	9.06775210474109e-11\\
1.45244899571883	4.48501735750284e-10\\
1.48489661791881	2.16927753537988e-10\\
1.51811813384306	5.50718637359893e-10\\
1.5513396497673	8.93214391339825e-11\\
1.58456116569155	4.35676161814058e-10\\
1.61778268161579	2.11546558048781e-10\\
1.65178046131382	5.30960497791e-10\\
1.68577824101184	8.88615847571828e-11\\
1.71977602070987	4.18226731024873e-10\\
1.7537738004079	2.04150307769879e-10\\
1.78853445564131	5.0588067068702e-10\\
1.82329511087471	8.72744099211786e-11\\
1.85805576610812	3.96892074761013e-10\\
1.89281642134153	1.94827431965194e-10\\
1.92833097625931	4.76443051677933e-10\\
1.96384553117709	8.4757922902412e-11\\
1.99936008609487	3.72351705024698e-10\\
2.03487464101265	1.83901505135253e-10\\
2.07113752249011	4.43476033673562e-10\\
2.10740040396758	8.14386891256902e-11\\
2.14366328544504	3.45246775612651e-10\\
2.1799261669225	1.71672398519007e-10\\
2.21693501810426	4.07754052744735e-10\\
2.25394386928602	7.74422748150982e-11\\
2.29095272046778	3.1617103379844e-10\\
2.32796157164954	1.58419999340964e-10\\
2.3657171313366	3.69986208337281e-10\\
2.40347269102366	7.2888417523842e-11\\
2.44122825071071	2.85664991661605e-10\\
2.47898381039777	1.44400880142115e-10\\
2.51748987308115	3.30811378290718e-10\\
2.55599593576452	6.7889582844316e-11\\
2.5945019984479	2.54212095818218e-10\\
2.63300806113127	1.29846411400791e-10\\
2.67227151262563	2.90794721635734e-10\\
2.71153496411999	6.2549410095869e-11\\
2.75079841561434	2.22237561686711e-10\\
2.7900618671087	1.14961540287339e-10\\
2.83009279630781	2.50427123482666e-10\\
2.87012372550692	5.69621017021404e-11\\
2.91015465470604	1.90107374287152e-10\\
2.95018558390515	9.99244020860601e-11\\
2.99099747582647	2.10124473376538e-10\\
3.03180936774779	5.12119235906994e-11\\
3.07262125966911	1.58128510285849e-10\\
3.11343315159043	8.48856540613951e-11\\
3.15504317204847	1.70229053075843e-10\\
3.1966531925065	4.53727055926834e-11\\
3.23826321296453	1.26549992707226e-10\\
3.27987323342256	6.99693636363463e-11\\
3.32230262192904	1.3101042473096e-10\\
3.36473201043552	3.95082855320084e-11\\
3.407161398942	9.55632240007276e-11\\
3.44959078744848	5.52726753255683e-11\\
3.49286538309579	9.26662080402707e-11\\
3.5361399787431	3.36720651361588e-11\\
3.57941457439041	6.53028742192419e-11\\
3.62268917003772	4.08659772688225e-11\\
3.66684010640848	5.53228574062814e-11\\
3.71099104277924	2.79074541253976e-11\\
3.75514197915	3.58462148852823e-11\\
3.79929291552076	2.67924571417666e-11\\
3.84435753680729	1.90336635341737e-11\\
3.88942215809383	2.22483143019758e-11\\
3.93448677938037	7.21167570105763e-12\\
3.9795514006669	1.30678801113504e-11\\
4.02557447530949	1.62244662149647e-11\\
4.07159754995208	1.67189595501327e-11\\
4.11762062459467	2.0645374299022e-11\\
4.16364369923726	3.22075699443758e-13\\
4.21041243310513	4.96264140892322e-11\\
4.257181166973	9.4592111921088e-12\\
4.30394990084087	4.50202097823649e-11\\
4.35071863470874	1.23285825992525e-11\\
4.39747966483664	7.40306704827276e-11\\
4.44424069496454	2.66353605837821e-12\\
4.49100172509244	6.24006402105692e-11\\
4.53776275522034	2.1877499811751e-11\\
4.58452386492069	9.14487374714668e-11\\
4.63128497462103	2.54951615374921e-12\\
4.67804608432138	7.46460671052773e-11\\
4.72480719402173	2.87936341436534e-11\\
4.77156830290013	1.03249186977905e-10\\
4.81832941177852	6.47393250119421e-12\\
4.86509052065691	8.27782287160517e-11\\
4.91185162953531	3.35981242827188e-11\\
4.95861273842228	1.10590536728239e-10\\
5.00537384730925	9.35618249542358e-12\\
5.05213495619622	8.76534400617857e-11\\
5.09889606508318	3.67204044948721e-11\\
5.14565717397096	1.14424580921479e-10\\
5.19241828285873	1.13998810391536e-11\\
5.2391793917465	8.99745833393695e-11\\
5.28594050063428	3.85139697911541e-11\\
5.33270160952073	1.15530918165518e-10\\
5.37946271840718	1.27733379429174e-11\\
5.42622382729363	9.03174202093737e-11\\
5.47298493618008	3.92689214478992e-11\\
5.51974604506734	1.14546150342676e-10\\
5.56650715395461	1.36153310847931e-11\\
5.61326826284187	8.91505758104927e-11\\
5.66002937172913	3.922207003626e-11\\
5.70679048061676	1.11987530360125e-10\\
5.7535515895044	1.40399913917122e-11\\
5.80031269839203	8.68549676624752e-11\\
5.84707380727967	3.85664833402188e-11\\
5.89383491616663	1.08273501275846e-10\\
5.9405960250536	1.41406886200457e-11\\
5.98735713394057	8.37379054985377e-11\\
6.03411824282754	3.74585917839454e-11\\
6.08087935171415	1.03740682710907e-10\\
6.12764046060076	1.39933620246779e-11\\
6.17440156948738	8.00461918970541e-11\\
6.22116267837399	3.60246277253395e-11\\
6.26792378726144	9.86574155703579e-11\\
6.3146848961489	1.3659962050383e-11\\
6.36144600503635	7.59766694002906e-11\\
6.4082071139238	3.43658435042471e-11\\
6.45496822281113	9.32371957418354e-11\\
6.50172933169847	1.31905597555715e-11\\
6.5484904405858	7.1685102298602e-11\\
6.59525154947314	3.25627302899534e-11\\
6.64201265836067	8.76471117905453e-11\\
6.68877376724821	1.26250121468274e-11\\
6.73553487613575	6.72927824574288e-11\\
6.78229598502329	3.06784042614083e-11\\
6.82916482319815	8.24652013342586e-11\\
6.87603366137301	9.46298595039252e-12\\
6.92290249954786	6.70298261340463e-11\\
6.96977133772272	3.0722480115486e-11\\
7.01788255937152	8.77410921695798e-11\\
7.06599378102033	1.03533293049907e-11\\
7.11410500266913	7.11812830900271e-11\\
7.16221622431793	3.27526894494667e-11\\
7.21159336850751	9.27568022390801e-11\\
7.26097051269708	1.11294307103549e-11\\
7.31034765688666	7.52666817760428e-11\\
7.35972480107623	3.47656903265658e-11\\
7.41043619055615	9.7901797779798e-11\\
7.46114758003607	1.19361742711988e-11\\
7.511858969516	7.9460715785018e-11\\
7.56257035899592	3.68406971595903e-11\\
7.61468820401402	1.0317274812266e-10\\
7.66680604903213	1.2770617896507e-11\\
7.71892389405023	8.37667712971779e-11\\
7.77104173906833	3.89802634614966e-11\\
7.8246442387161	1.08582587365902e-10\\
7.87824673836387	1.363548163269e-11\\
7.93184923801164	8.81962836096761e-11\\
7.98545173765941	4.1190884036979e-11\\
8.04062368192306	1.14145470853089e-10\\
8.09579562618671	1.45335965484605e-11\\
8.15096757045037	9.27618537538422e-11\\
8.20613951471402	4.3479664313395e-11\\
8.26297303935407	1.19877441306926e-10\\
8.31980656399412	1.54682933128925e-11\\
8.37664008863416	9.74780256512986e-11\\
8.43347361327421	4.58548754522781e-11\\
8.49206906178453	1.25796761896169e-10\\
8.55066451029485	1.6443235661967e-11\\
8.60925995880517	1.02361286113961e-10\\
8.66785540731549	4.83257323047326e-11\\
8.7283223218246	1.31924471347133e-10\\
8.7887892363337	1.74626979543291e-11\\
8.84925615084281	1.074305644444e-10\\
8.90972306535191	5.09029485229462e-11\\
8.97139693657501	1.31856192631119e-10\\
9.03307080779811	2.33627006629433e-11\\
9.09474467902121	9.96397409025462e-11\\
9.15641855024431	4.77817230226663e-11\\
9.21815611731334	1.1724288206949e-10\\
9.27989368438236	2.07439621036087e-11\\
9.34163125145139	8.87308004848819e-11\\
9.40336881852041	4.25899870926116e-11\\
9.46537482585172	1.05220610002732e-10\\
9.52738083318303	1.56011314977889e-11\\
9.58938684051434	8.41509639748494e-11\\
9.65139284784564	4.04002387099922e-11\\
9.71505322212012	1.06126218923919e-10\\
9.77871359639459	1.59586788228694e-11\\
9.84237397066907	8.48087711169399e-11\\
9.90603434494354	4.08381661820556e-11\\
9.97140380506578	1.06629538532133e-10\\
10.036773265188	1.6147028158997e-11\\
10.1021427253103	8.53082604557187e-11\\
10.1675121854325	4.12000988880834e-11\\
10.2347010682591	1.07180486708103e-10\\
10.3018899510858	1.6350809595167e-11\\
10.3690788339124	8.58424442640171e-11\\
10.436267716739	4.15824041866131e-11\\
10.505392885882	1.07764019929846e-10\\
10.574518055025	1.65644165051049e-11\\
10.643643224168	8.64053273375021e-11\\
10.712768393311	4.19826950981417e-11\\
10.7839560546811	1.08381470464991e-10\\
10.8551437160513	1.67890146229865e-11\\
10.9263313774214	8.69980754103494e-11\\
10.9975190387916	4.24020263345426e-11\\
11.07090572606	1.09033226891597e-10\\
11.1442924133284	1.70255476383829e-11\\
11.2176791005968	8.76211880829203e-11\\
11.2910657878652	4.28414526076892e-11\\
11.3667995476522	1.09718512053547e-10\\
11.4425333074392	1.72750147520162e-11\\
11.5182670672261	8.82744433106097e-11\\
11.5940008270131	4.33016400513964e-11\\
11.6722425439244	1.10435771638606e-10\\
11.7504842608358	1.75383596534573e-11\\
11.8287259777471	8.89565088257882e-11\\
11.9069676946584	4.37828107102689e-11\\
11.9878925800102	1.11181674977701e-10\\
12.068817465362	1.78168035880333e-11\\
12.1497423507139	8.96648866266503e-11\\
12.2306672360657	4.42847425397019e-11\\
12.3144665036616	1.11950670955707e-10\\
12.3982657712575	1.81112347341639e-11\\
12.4820650388535	9.03949137764926e-11\\
12.5658643064494	4.48063253166708e-11\\
12.652747062149	1.12734210855336e-10\\
12.7396298178485	1.84228188260249e-11\\
12.826512573548	9.11392072922013e-11\\
12.9133953292476	4.53453941062776e-11\\
13.0035907048582	1.135193050672e-10\\
13.0937860804688	1.87524440420361e-11\\
13.1839814560794	9.18868314769838e-11\\
13.27417683169	4.58982296613897e-11\\
13.3679363981715	1.14287357355636e-10\\
13.461695964653	1.91011095829197e-11\\
13.5554555311345	9.26215215635295e-11\\
13.649215097616	4.64588367776742e-11\\
13.7468155979555	1.15011722368052e-10\\
13.8444160982949	1.94692595378854e-11\\
13.9420165986344	9.33201294017749e-11\\
14.0396170989738	4.70180561151778e-11\\
14.1413635477079	1.15655263144276e-10\\
14.2431099964419	1.98575045295968e-11\\
14.344856445176	9.39495703455862e-11\\
14.44660289391	4.75626205087565e-11\\
14.5528320987134	1.17244519648452e-10\\
14.6590613035167	2.84726409116587e-11\\
14.7652905083201	8.310344079554e-11\\
14.8715197131234	4.26394475283587e-11\\
14.9753724014636	8.3321155530669e-11\\
15.0792250898038	2.08169592674778e-11\\
15.183077778144	5.82672243787385e-11\\
15.2869304664842	3.00280911247341e-11\\
15.3888306269013	5.96701021926549e-11\\
15.4907307873185	1.24704135906484e-11\\
15.5926309477356	4.53015125412293e-11\\
15.6945311081528	2.31303032283137e-11\\
15.7977993001492	5.10944064835428e-11\\
15.9010674921457	1.06314401726593e-11\\
16.0043356841422	3.89652476950886e-11\\
16.1076038761386	1.99265048905772e-11\\
16.2124869885142	4.40428249426361e-11\\
16.3173701008898	9.16602904688091e-12\\
16.4222532132654	3.37133654326749e-11\\
16.527136325641	1.72872827164383e-11\\
16.6339006004935	3.81736309229552e-11\\
16.7406648753461	7.996575623892e-12\\
16.8474291501986	2.92923740818907e-11\\
16.9541934250511	1.5079687498698e-11\\
17.0630876639797	3.31571159417621e-11\\
17.1719819029082	7.0368710858304e-12\\
17.2808761418368	2.54770649021907e-11\\
17.3897703807653	1.31868960195902e-11\\
17.5010359523856	2.87649071228913e-11\\
17.6123015240059	6.23071039207446e-12\\
17.7235670956262	2.21074270001509e-11\\
17.8348326672466	1.15265019751121e-11\\
17.9487100003084	2.48295828342293e-11\\
18.0625873333702	5.53920798118668e-12\\
18.1764646664321	1.90633620000824e-11\\
18.2903419994939	1.00373320766067e-11\\
18.4070757021241	2.12220796491636e-11\\
18.5238094047543	4.93433072179528e-12\\
18.6405431073846	1.62503899225896e-11\\
18.7572768100148	8.67161897843971e-12\\
18.8771200075905	1.78379255810768e-11\\
18.9969632051661	4.39501213200799e-12\\
19.1168064027418	1.35901290221341e-11\\
19.2366496003175	7.39014405226612e-12\\
19.3598678839536	1.45878031876379e-11\\
19.4830861675897	3.90484866663598e-12\\
19.6063044512258	1.10139120046426e-11\\
19.7295227348619	6.15907325141052e-12\\
19.8563979903555	1.1390749454776e-11\\
19.9832732458491	3.45040662708129e-12\\
20.1101485013428	8.45820635753114e-12\\
20.2370237568364	4.94734808675901e-12\\
20.3678580796439	8.16915979306998e-12\\
20.4986924024514	3.02025071619028e-12\\
20.6295267252589	5.8605065245132e-12\\
20.7603610480664	3.72477049204178e-12\\
20.8954809190363	4.8439863231664e-12\\
21.0306007900061	2.60411137098515e-12\\
21.1657206609759	3.15572568077016e-12\\
21.3008405319458	2.46033748929619e-12\\
21.4406014192669	1.33026922810586e-12\\
21.5803623065881	2.19213536212237e-12\\
21.7201231939092	2.72712408211362e-13\\
21.8598840812304	1.12035380972486e-12\\
22.0046756408933	2.46730413877572e-12\\
22.1494672005562	1.77462211592427e-12\\
22.2942587602192	2.87000978538288e-12\\
22.4390503198821	3.3355262996082e-13\\
22.5893023876591	6.66040833596782e-12\\
22.7395544554361	1.34131594720088e-12\\
22.8898065232131	6.3691274476696e-12\\
23.0400585909901	1.94670668474117e-12\\
23.1962481938828	1.13839909721136e-11\\
23.3524377967755	8.80823192161984e-13\\
23.5086273996682	1.03428932085592e-11\\
23.6648170025609	3.77442521681814e-12\\
23.8274766689118	1.68055708238413e-11\\
23.9901363352628	3.80168119207269e-13\\
24.1527960016138	1.49396467419294e-11\\
24.3154556679648	5.886083287443e-12\\
24.4851833872328	2.3137713967003e-11\\
24.6549111065008	1.76136882856781e-13\\
24.8246388257688	2.03494859851716e-11\\
24.9943665450368	8.37062363867602e-12\\
25.1718378270833	3.06556308116157e-11\\
25.3493091091298	8.06979483236603e-13\\
25.5267803911763	2.68204625175628e-11\\
25.7042516732229	1.13442588656198e-11\\
25.8902342929274	3.97216148861901e-11\\
26.0762169126319	1.53586865447863e-12\\
26.2621995323364	3.46821044105994e-11\\
26.4481821520409	1.49617263023316e-11\\
26.6435543528394	5.0820861408063e-11\\
26.838926553638	2.39271102930871e-12\\
27.0342987544366	4.43792363968853e-11\\
27.2296709552352	1.94323446223166e-11\\
27.4354441469161	6.46136200099789e-11\\
27.641217338597	3.41610073562038e-12\\
27.8469905302779	5.65224811399645e-11\\
28.0527637219589	2.5044355478343e-11\\
28.2701110770545	8.20134654633264e-11\\
28.4874584321501	4.6564418987316e-12\\
28.7048057872458	7.19644760449611e-11\\
28.9221531423414	3.22014012676952e-11\\
29.1524460226538	1.04306716042224e-10\\
29.3827389029661	6.18038259458942e-12\\
29.6130317832784	9.19173487323732e-11\\
29.8433246635908	4.14794934511242e-11\\
30.088179573096	1.3333817383554e-10\\
30.3330344826013	8.07678923742117e-12\\
30.5778893921066	1.18138047955352e-10\\
30.8227443016119	5.37166144898293e-11\\
31.0840841898221	1.71807464088847e-10\\
31.3454240780323	1.04649205967533e-11\\
31.6067639662425	1.53227493193686e-10\\
31.8681038544528	7.01585109452729e-11\\
32.1482392887021	2.2375275937625e-10\\
32.4283747229513	1.3505141449599e-11\\
32.7085101572006	2.01126559851161e-10\\
32.9886455914499	9.27011106766074e-11\\
33.2903839143714	2.95363344893218e-10\\
33.5921222372929	1.74120023954671e-11\\
33.8938605602144	2.67962603495153e-10\\
34.195598883136	1.24306481774639e-10\\
34.5223935450538	4.01234732938516e-10\\
34.8491882069717	5.69309530074058e-11\\
35.1759828688896	3.16591815263667e-10\\
35.5027775308075	1.46671435413692e-10\\
35.8301107995782	3.6366907749108e-10\\
36.1574440683489	6.11421364926645e-11\\
36.4847773371196	2.75706863589464e-10\\
36.8121106058903	1.29458437025942e-10\\
37.1394383112246	3.20691292965414e-10\\
37.4667660165588	5.41232267337044e-11\\
37.7940937218931	2.43074387007436e-10\\
38.1214214272273	1.14255563343768e-10\\
38.448749190017	2.82360822223549e-10\\
38.7760769528068	4.78239045587081e-11\\
39.1034047155965	2.13933305098779e-10\\
39.4307324783862	1.00636783628705e-10\\
39.7580602405942	2.48315305817481e-10\\
40.0853880028023	4.21571388908859e-11\\
40.4127157650103	1.88087257574354e-10\\
40.7400435272184	8.85248714610487e-11\\
41.0673712894278	2.18200506274524e-10\\
41.3946990516372	3.71032475576794e-11\\
41.7220268138466	1.65246514388651e-10\\
42.049354576056	7.78020668723567e-11\\
42.3766823382676	1.91635034230142e-10\\
42.7040101004792	3.26207325040961e-11\\
43.0313378626908	1.45110257143699e-10\\
43.3586656249024	6.8337615080627e-11\\
43.6859933871135	1.68243063577989e-10\\
44.0133211493247	2.86594185883171e-11\\
44.3406489115359	1.27386793821738e-10\\
44.667976673747	6.00005756884459e-11\\
44.9953044359533	1.47670483480189e-10\\
45.3226321981596	2.51671287976318e-11\\
45.649959960366	1.1180383875653e-10\\
45.9772877225723	5.26664996336468e-11\\
46.3046154847852	1.29592225839303e-10\\
46.6319432469981	2.20932872690982e-11\\
46.9592710092111	9.81127904614576e-11\\
47.286598771424	4.62205239359914e-11\\
47.6139265336322	1.13701796242149e-10\\
47.9412542958405	1.92974455859307e-11\\
48.2685820580487	8.62164437065438e-11\\
48.595909820257	4.06188190721357e-11\\
48.926850153611	1.04059299371739e-10\\
49.257790486965	1.41316481086506e-11\\
49.5887308203189	8.37423273086779e-11\\
49.9196711536729	3.91679275818513e-11\\
50.2593448649822	1.04238591716599e-10\\
50.5990185762914	1.44063372342629e-11\\
50.9386922876007	8.37187099159142e-11\\
51.2783659989099	3.92415752670239e-11\\
51.6270508037435	1.04074482538963e-10\\
51.9757356085771	1.45003184806591e-11\\
52.3244204134106	8.36059390121879e-11\\
52.6731052182442	3.92657972109189e-11\\
53.0312891287569	1.03911984919436e-10\\
53.3894730392697	1.46015621468876e-11\\
53.7476569497824	8.34944552732808e-11\\
54.1058408602952	3.92957342013056e-11\\
54.4740437546715	1.03738118789609e-10\\
54.8422466490478	1.47037364922609e-11\\
55.2104495434241	8.33803625102658e-11\\
55.5786524378004	3.93292698755432e-11\\
55.95743714694	1.0355274797369e-10\\
56.3362218560795	1.48074651498686e-11\\
56.7150065652191	8.3263203623507e-11\\
57.0937912743587	3.93663769117369e-11\\
57.4837682235858	1.03355208506267e-10\\
57.8737451728129	1.4913118483173e-11\\
58.26372212204	8.31423692913835e-11\\
58.653699071267	3.94069542622444e-11\\
58.9902743034503	5.30704295613471e-11\\
59.3268495356335	5.3343428198932e-11\\
59.6634247678168	1.83439461537427e-11\\
60	1.62166764391958e-12\\
};
\addlegendentry{Infetti}

\end{axis}
\end{tikzpicture}%}
}
\quad
\subfloat[][Nodo 4]
{\resizebox{0.45\textwidth}{!}
{% This file was created by matlab2tikz.
%
%The latest updates can be retrieved from
%  http://www.mathworks.com/matlabcentral/fileexchange/22022-matlab2tikz-matlab2tikz
%where you can also make suggestions and rate matlab2tikz.
%
\definecolor{mycolor1}{rgb}{0.00000,0.44700,0.74100}%
\definecolor{mycolor2}{rgb}{0.85000,0.32500,0.09800}%
%
\begin{tikzpicture}

\begin{axis}[%
width=0.39\columnwidth,
height=1.7in,
at={(1.011in,0.642in)},
scale only axis,
xmin=0,
xmax=60,
xlabel style={font=\color{white!15!black}},
xlabel={T},
ymode=log,
ymin=1e-17,
ymax=1e-12,
yminorticks=true,
axis background/.style={fill=white},
legend columns=2,
legend style={legend cell align=left, align=left, draw=none,fill=none}
]
\addplot [color=mycolor1, line width=2.0pt]
  table[row sep=crcr]{%
0	0\\
0.00199053585276749	0\\
0.00398107170553497	0\\
0.00597160755830246	0\\
0.00796214341106994	0\\
0.0123567669950473	0\\
0.0167513905790247	0\\
0.0211460141630021	0\\
0.0255406377469794	0\\
0.0299388563931551	0\\
0.0343370750393307	0\\
0.0387352936855064	0\\
0.043133512331682	0\\
0.0475471725191068	0\\
0.0519608327065316	0\\
0.0563744928939564	0\\
0.0607881530813811	0\\
0.0652173199794049	0\\
0.0696464868774286	0\\
0.0740756537754524	0\\
0.0785048206734761	0\\
0.0829496187548012	0\\
0.0873944168361263	0\\
0.0918392149174515	0\\
0.0962840129987766	0\\
0.100744568101794	0\\
0.105205123204811	0\\
0.109665678307829	0\\
0.114126233410846	0\\
0.118602673075784	0\\
0.123079112740721	0\\
0.127555552405659	0\\
0.132031992070596	0\\
0.136524445545197	0\\
0.141016899019798	0\\
0.1455093524944	0\\
0.150001805969001	0\\
0.154510404254387	0\\
0.159019002539772	0\\
0.163527600825158	0\\
0.168036199110544	0\\
0.172561074936287	0\\
0.17708595076203	0\\
0.181610826587773	0\\
0.186135702413515	0\\
0.190676990405524	0\\
0.195218278397533	0\\
0.199759566389542	0\\
0.204300854381551	0\\
0.208858690959197	0\\
0.213416527536843	0\\
0.217974364114488	0\\
0.222532200692134	0\\
0.227106724202768	0\\
0.231681247713401	0\\
0.236255771224035	0\\
0.240830294734669	0\\
0.245421645441331	0\\
0.250012996147992	0\\
0.254604346854654	0\\
0.259195697561315	0\\
0.263804017673384	0\\
0.268412337785452	0\\
0.273020657897521	0\\
0.27762897800959	0\\
0.282254411746462	0\\
0.286879845483334	0\\
0.291505279220206	0\\
0.296130712957079	0\\
0.300773406613671	0\\
0.305416100270264	0\\
0.310058793926856	0\\
0.314701487583449	0\\
0.319361589521735	0\\
0.324021691460022	0\\
0.328681793398309	0\\
0.333341895336596	0\\
0.338019556071109	0\\
0.342697216805623	0\\
0.347374877540136	0\\
0.35205253827465	0\\
0.356747910447269	0\\
0.361443282619888	0\\
0.366138654792507	0\\
0.370834026965126	0\\
0.375547265516765	0\\
0.380260504068404	0\\
0.384973742620042	0\\
0.389686981171681	0\\
0.394418243246227	0\\
0.399149505320772	0\\
0.403880767395317	0\\
0.408612029469863	0\\
0.41336147459102	0\\
0.418110919712177	0\\
0.422860364833334	0\\
0.427609809954491	0\\
0.432377599961091	0\\
0.437145389967691	0\\
0.44191317997429	0\\
0.44668096998089	0\\
0.451467269141767	0\\
0.456253568302644	0\\
0.461039867463521	0\\
0.465826166624398	0\\
0.47063114173014	0\\
0.475436116835881	0\\
0.480241091941623	0\\
0.485046067047365	0\\
0.489869887370685	0\\
0.494693707694006	0\\
0.499517528017326	0\\
0.504341348340647	0\\
0.509184185682064	0\\
0.51402702302348	0\\
0.518869860364897	0\\
0.523712697706314	0\\
0.528574726619363	0\\
0.533436755532412	0\\
0.53829878444546	0\\
0.543160813358509	0\\
0.548042211039569	0\\
0.55292360872063	0\\
0.55780500640169	0\\
0.56268640408275	0\\
0.56758735048985	0\\
0.572488296896949	0\\
0.577389243304049	0\\
0.582290189711149	0\\
0.587210867601438	0\\
0.592131545491728	0\\
0.597052223382018	0\\
0.601972901272307	0\\
0.606913496388328	0\\
0.611854091504349	0\\
0.61679468662037	0\\
0.621735281736391	0\\
0.626695982628195	0\\
0.631656683519999	0\\
0.636617384411802	0\\
0.641578085303606	0\\
0.646559083672327	0\\
0.651540082041047	0\\
0.656521080409768	0\\
0.661502078778488	0\\
0.666503569383853	0\\
0.671505059989219	0\\
0.676506550594584	0\\
0.68150804119995	0\\
0.686530221876942	0\\
0.691552402553934	0\\
0.696574583230927	0\\
0.701596763907919	0\\
0.706639835866396	0\\
0.711682907824873	0\\
0.71672597978335	0\\
0.721769051741827	0\\
0.72683321952555	0\\
0.731897387309272	0\\
0.736961555092995	0\\
0.742025722876718	0\\
0.747111194287107	0\\
0.752196665697496	0\\
0.757282137107885	0\\
0.762367608518274	0\\
0.767474594900923	0\\
0.772581581283573	0\\
0.777688567666222	0\\
0.782795554048871	0\\
0.78792427039055	0\\
0.793052986732229	0\\
0.798181703073909	0\\
0.803310419415588	0\\
0.808461084325784	0\\
0.81361174923598	0\\
0.818762414146177	0\\
0.823913079056373	0\\
0.82908591482452	0\\
0.834258750592667	0\\
0.839431586360813	0\\
0.84460442212896	0\\
0.849799655005291	0\\
0.854994887881621	0\\
0.860190120757951	0\\
0.865385353634282	0\\
0.870603213699137	0\\
0.875821073763992	0\\
0.881038933828848	0\\
0.886256793893703	0\\
0.891497515445601	0\\
0.896738236997499	0\\
0.901978958549397	0\\
0.907219680101295	0\\
0.912483501408722	0\\
0.917747322716149	0\\
0.923011144023576	0\\
0.928274965331004	0\\
0.933562128954513	0\\
0.938849292578023	0\\
0.944136456201532	0\\
0.949423619825042	0\\
0.954734372782572	0\\
0.960045125740102	0\\
0.965355878697632	0\\
0.970666631655162	0\\
0.976001225418392	0\\
0.981335819181621	0\\
0.98667041294485	0\\
0.99200500670808	0\\
0.997363697280812	0\\
1.00272238785355	0\\
1.00808107842628	0\\
1.01343976899901	0\\
1.0188228172289	0\\
1.02420586545878	0\\
1.02958891368867	0\\
1.03497196191855	0\\
1.04037963341625	0\\
1.04578730491395	0\\
1.05119497641164	0\\
1.05660264790934	0\\
1.06203521338424	0\\
1.06746777885913	0\\
1.07290034433402	0\\
1.07833290980891	0\\
1.08379064506731	0\\
1.08924838032571	0\\
1.09470611558411	0\\
1.10016385084252	0\\
1.1056470369148	0\\
1.11113022298707	0\\
1.11661340905935	0\\
1.12209659513163	0\\
1.12760551858231	0\\
1.13311444203298	0\\
1.13862336548365	0\\
1.14413228893432	0\\
1.14966724184116	0\\
1.15520219474799	0\\
1.16073714765482	0\\
1.16627210056166	0\\
1.17183338076584	0\\
1.17739466097002	0\\
1.18295594117419	0\\
1.18851722137837	0\\
1.19410513262804	0\\
1.19969304387772	0\\
1.20528095512739	0\\
1.21086886637706	0\\
1.21648371859449	0\\
1.22209857081192	0\\
1.22771342302936	0\\
1.23332827524679	0\\
1.23897038457538	0\\
1.24461249390397	0\\
1.25025460323256	0\\
1.25589671256115	0\\
1.26156640151281	0\\
1.26723609046446	0\\
1.27290577941612	0\\
1.27857546836778	0\\
1.28427306633634	0\\
1.28997066430491	0\\
1.29566826227347	0\\
1.30136586024204	0\\
1.30709170327692	0\\
1.31281754631181	0\\
1.31854338934669	0\\
1.32426923238158	0\\
1.33002366371043	0\\
1.33577809503928	0\\
1.34153252636813	0\\
1.34728695769699	0\\
1.35307032782915	0\\
1.35885369796131	0\\
1.36463706809348	0\\
1.37042043822564	0\\
1.37622881553645	0\\
1.38203719284726	0\\
1.38784557015807	0\\
1.39365394746888	0\\
1.39948194569202	0\\
1.40530994391516	0\\
1.41113794213829	0\\
1.41696594036143	0\\
1.4228137248248	0\\
1.42866150928816	0\\
1.43450929375153	0\\
1.44035707821489	0\\
1.44622478964653	0\\
1.45209250107816	0\\
1.45796021250979	0\\
1.46382792394143	0\\
1.46971570462452	0\\
1.4756034853076	0\\
1.48149126599069	0\\
1.48737904667378	0\\
1.49328704060906	0\\
1.49919503454433	0\\
1.50510302847961	0\\
1.51101102241489	0\\
1.51693937480906	0\\
1.52286772720322	0\\
1.52879607959739	0\\
1.53472443199155	0\\
1.540673290528	0\\
1.54662214906446	0\\
1.55257100760091	0\\
1.55851986613736	0\\
1.56448937947106	0\\
1.57045889280476	0\\
1.57642840613847	0\\
1.58239791947216	0\\
1.58838823791446	0\\
1.59437855635675	0\\
1.60036887479904	0\\
1.60635919324133	0\\
1.61237046959671	0\\
1.61838174595208	0\\
1.62439302230746	0\\
1.63040429866284	0\\
1.63643668672632	0\\
1.6424690747898	0\\
1.64850146285328	0\\
1.65453385091676	0\\
1.66058750665404	0\\
1.66664116239132	0\\
1.67269481812859	0\\
1.67874847386587	0\\
1.68482355500326	0\\
1.69089863614064	0\\
1.69697371727803	0\\
1.70304879841541	0\\
1.70914546464501	0\\
1.71524213087462	0\\
1.72133879710422	0\\
1.72743546333383	0\\
1.73355387572031	0\\
1.73967228810678	0\\
1.74579070049326	0\\
1.75190911287974	0\\
1.75804943463327	0\\
1.76418975638681	0\\
1.77033007814035	0\\
1.77647039989388	0\\
1.78263279664118	0\\
1.78879519338848	0\\
1.79495759013578	0\\
1.80111998688309	0\\
1.80730462566597	0\\
1.81348926444885	0\\
1.81967390323173	0\\
1.82585854201462	0\\
1.83206559217232	0\\
1.83827264233002	0\\
1.84447969248772	0\\
1.85068674264543	0\\
1.85691637550678	0\\
1.86314600836813	0\\
1.86937564122948	0\\
1.87560527409083	0\\
1.88185766319929	0\\
1.88811005230775	0\\
1.89436244141622	0\\
1.90061483052468	0\\
1.90689015093446	0\\
1.91316547134424	0\\
1.91944079175402	0\\
1.92571611216381	0\\
1.93201454150218	0\\
1.93831297084055	0\\
1.94461140017892	0\\
1.9509098295173	0\\
1.95723154808653	0\\
1.96355326665576	0\\
1.96987498522499	0\\
1.97619670379422	0\\
1.98254189319299	0\\
1.98888708259176	0\\
1.99523227199053	0\\
2.0015774613893	0\\
2.00794630624256	0\\
2.01431515109582	0\\
2.02068399594908	0\\
2.02705284080234	0\\
2.03344552746868	0\\
2.03983821413503	0\\
2.04623090080137	0\\
2.05262358746771	0\\
2.05904030495676	0\\
2.06545702244581	0\\
2.07187373993486	0\\
2.07829045742391	0\\
2.084731396657	0\\
2.0911723358901	0\\
2.0976132751232	0\\
2.1040542143563	0\\
2.11051956957009	0\\
2.11698492478388	0\\
2.12345027999767	0\\
2.12991563521146	0\\
2.13640560238783	0\\
2.1428955695642	0\\
2.14938553674057	0\\
2.15587550391694	0\\
2.16239028199053	0\\
2.16890506006412	0\\
2.17541983813771	0\\
2.1819346162113	0\\
2.18847440630977	0\\
2.19501419640825	0\\
2.20155398650673	0\\
2.20809377660521	0\\
2.21465878275268	0\\
2.22122378890015	0\\
2.22778879504762	0\\
2.23435380119508	0\\
2.24094422983074	0\\
2.24753465846639	0\\
2.25412508710205	0\\
2.2607155157377	0\\
2.2673315759421	0\\
2.27394763614649	0\\
2.28056369635089	0\\
2.28717975655528	0\\
2.29382166035897	0\\
2.30046356416266	0\\
2.30710546796636	0\\
2.31374737177005	0\\
2.32041533418655	0\\
2.32708329660306	0\\
2.33375125901957	0\\
2.34041922143607	0\\
2.34711345989709	0\\
2.3538076983581	0\\
2.36050193681912	0\\
2.36719617528013	0\\
2.37391691043736	0\\
2.3806376455946	0\\
2.38735838075183	0\\
2.39407911590907	0\\
2.40082657131283	0\\
2.40757402671659	0\\
2.41432148212035	0\\
2.42106893752412	0\\
2.42784333963791	0\\
2.4346177417517	0\\
2.4413921438655	0\\
2.44816654597929	0\\
2.4549681243632	0\\
2.4617697027471	0\\
2.46857128113101	0\\
2.47537285951492	0\\
2.48220184698832	0\\
2.48903083446173	0\\
2.49585982193513	0\\
2.50268880940853	0\\
2.50954544177924	0\\
2.51640207414995	0\\
2.52325870652066	0\\
2.53011533889137	0\\
2.53699985562036	0\\
2.54388437234935	0\\
2.55076888907833	0\\
2.55765340580732	0\\
2.5645660494942	0\\
2.57147869318108	0\\
2.57839133686796	0\\
2.58530398055484	0\\
2.59224499685821	0\\
2.59918601316159	0\\
2.60612702946496	0\\
2.61306804576833	0\\
2.62003768421203	0\\
2.62700732265572	0\\
2.63397696109942	0\\
2.64094659954311	0\\
2.64794511298781	0\\
2.65494362643251	0\\
2.6619421398772	0\\
2.6689406533219	0\\
2.67596829852112	0\\
2.68299594372033	0\\
2.69002358891955	0\\
2.69705123411877	0\\
2.70410827073433	0\\
2.71116530734989	0\\
2.71822234396545	0\\
2.72527938058101	0\\
2.73236607296869	0\\
2.73945276535637	0\\
2.74653945774405	0\\
2.75362615013173	0\\
2.76074276611797	0\\
2.76785938210422	0\\
2.77497599809046	0\\
2.78209261407671	0\\
2.78923942490227	0\\
2.79638623572783	0\\
2.80353304655339	0\\
2.81067985737896	0\\
2.81785713902212	0\\
2.82503442066529	0\\
2.83221170230846	0\\
2.83938898395163	0\\
2.84659701614953	0\\
2.85380504834743	0\\
2.86101308054533	0\\
2.86822111274323	0\\
2.87546017926332	0\\
2.88269924578341	0\\
2.88993831230351	0\\
2.8971773788236	0\\
2.90444776780364	0\\
2.91171815678368	0\\
2.91898854576373	0\\
2.92625893474377	0\\
2.93356093845992	0\\
2.94086294217606	0\\
2.94816494589221	0\\
2.95546694960836	0\\
2.96280086487659	0\\
2.97013478014483	0\\
2.97746869541306	0\\
2.9848026106813	0\\
2.99216873909201	0\\
2.99953486750271	0\\
3.00690099591342	0\\
3.01426712432413	0\\
3.02166577186631	0\\
3.02906441940849	0\\
3.03646306695067	0\\
3.04386171449286	0\\
3.05129319192691	0\\
3.05872466936097	0\\
3.06615614679503	0\\
3.07358762422909	0\\
3.08105224696524	0\\
3.0885168697014	0\\
3.09598149243755	0\\
3.1034461151737	0\\
3.11094420406679	0\\
3.11844229295988	0\\
3.12594038185297	0\\
3.13343847074605	0\\
3.14097035080817	0\\
3.14850223087028	0\\
3.1560341109324	0\\
3.16356599099451	0\\
3.17113199342146	0\\
3.17869799584842	0\\
3.18626399827537	0\\
3.19383000070233	0\\
3.20143046110215	0\\
3.20903092150198	0\\
3.21663138190181	0\\
3.22423184230164	0\\
3.2318671021242	0\\
3.23950236194676	0\\
3.24713762176933	0\\
3.25477288159189	0\\
3.26244328804483	0\\
3.27011369449776	0\\
3.2777841009507	0\\
3.28545450740363	0\\
3.29316041266953	0\\
3.30086631793543	0\\
3.30857222320133	0\\
3.31627812846722	0\\
3.32401989110489	0\\
3.33176165374256	0\\
3.33950341638023	0\\
3.3472451790179	0\\
3.35502316313618	0\\
3.36280114725446	0\\
3.37057913137275	0\\
3.37835711549103	0\\
3.38617169154667	0\\
3.3939862676023	0\\
3.40180084365794	0\\
3.40961541971357	0\\
3.41746696420648	0\\
3.42531850869938	0\\
3.43317005319229	0\\
3.4410215976852	0\\
3.44891049346576	0\\
3.45679938924632	0\\
3.46468828502688	0\\
3.47257718080744	0\\
3.48050381756295	0\\
3.48843045431847	0\\
3.49635709107398	0\\
3.50428372782949	0\\
3.51224850159594	0\\
3.52021327536239	0\\
3.52817804912884	0\\
3.53614282289529	0\\
3.54414613607264	0\\
3.55214944924998	0\\
3.56015276242733	0\\
3.56815607560468	0\\
3.57619833889781	0\\
3.58424060219095	0\\
3.59228286548408	0\\
3.60032512877721	0\\
3.60840675922679	0\\
3.61648838967638	0\\
3.62457002012596	0\\
3.63265165057554	0\\
3.64077307248204	0\\
3.64889449438855	0\\
3.65701591629505	0\\
3.66513733820155	0\\
3.67329898353354	0\\
3.68146062886554	0\\
3.68962227419754	0\\
3.69778391952953	0\\
3.70598622860803	0\\
3.71418853768653	0\\
3.72239084676502	0\\
3.73059315584352	0\\
3.73883657626523	0\\
3.74707999668693	0\\
3.75532341710863	0\\
3.76356683753034	0\\
3.77185182516863	0\\
3.78013681280692	0\\
3.78842180044522	0\\
3.79670678808351	0\\
3.80503380745825	0\\
3.81336082683299	0\\
3.82168784620774	0\\
3.83001486558248	0\\
3.83838438958511	0\\
3.84675391358775	0\\
3.85512343759039	0\\
3.86349296159302	0\\
3.8719054722877	0\\
3.88031798298237	0\\
3.88873049367705	0\\
3.89714300437172	0\\
3.90559899284027	0\\
3.91405498130882	0\\
3.92251096977737	0\\
3.93096695824593	0\\
3.93946692479213	0\\
3.94796689133833	0\\
3.95646685788454	0\\
3.96496682443074	0\\
3.97351127881372	0\\
3.98205573319669	0\\
3.99060018757967	0\\
3.99914464196265	0\\
4.00773410425464	0\\
4.01632356654662	0\\
4.02491302883861	0\\
4.0335024911306	0\\
4.04213749081375	0\\
4.05077249049689	0\\
4.05940749018004	0\\
4.06804248986318	0\\
4.07672356764034	0\\
4.08540464541749	0\\
4.09408572319465	0\\
4.10276680097181	0\\
4.11149450824228	0\\
4.12022221551276	0\\
4.12894992278323	0\\
4.1376776300537	0\\
4.14645252883691	0\\
4.15522742762012	0\\
4.16400232640333	0\\
4.17277722518654	0\\
4.18159988926615	0\\
4.19042255334575	0\\
4.19924521742536	0\\
4.20806788150496	0\\
4.21693889629988	0\\
4.22580991109479	0\\
4.23468092588971	0\\
4.24355194068463	0\\
4.2524719033702	0\\
4.26139186605577	0\\
4.27031182874135	0\\
4.27923179142692	0\\
4.28820131218307	0\\
4.29717083293922	0\\
4.30614035369538	0\\
4.31510987445153	0\\
4.32412957614907	0\\
4.33314927784661	0\\
4.34216897954416	0\\
4.3511886812417	0\\
4.36025919971018	0\\
4.36932971817866	0\\
4.37840023664715	0\\
4.38747075511563	0\\
4.39659274002011	0\\
4.40571472492458	0\\
4.41483670982906	0\\
4.42395869473353	0\\
4.43313280988389	0\\
4.44230692503424	0\\
4.45148104018459	0\\
4.46065515533495	0\\
4.469882079136	0\\
4.47910900293706	0\\
4.48833592673811	0\\
4.49756285053916	0\\
4.50684327607396	0\\
4.51612370160877	0\\
4.52540412714357	0\\
4.53468455267837	0\\
4.54401918924822	0\\
4.55335382581807	0\\
4.56268846238792	0\\
4.57202309895777	0\\
4.58141267108196	0\\
4.59080224320616	0\\
4.60019181533036	0\\
4.60958138745456	0\\
4.61902663678989	0\\
4.62847188612523	0\\
4.63791713546056	0\\
4.6473623847959	0\\
4.65686406984621	0\\
4.66636575489653	0\\
4.67586743994685	0\\
4.68536912499717	0\\
4.69492802222378	0\\
4.7044869194504	0\\
4.71404581667701	0\\
4.72360471390363	0\\
4.73320764525141	0\\
4.74281057659919	0\\
4.75241350794698	0\\
4.76201643929476	0\\
4.77165794452804	0\\
4.78129944976131	0\\
4.79094095499459	0\\
4.80058246022786	0\\
4.81026314694842	0\\
4.81994383366897	0\\
4.82962452038952	0\\
4.83930520711008	0\\
4.84902565638322	0\\
4.85874610565637	0\\
4.86846655492951	0\\
4.87818700420266	0\\
4.88794780523678	0\\
4.89770860627089	0\\
4.90746940730501	0\\
4.91723020833913	0\\
4.92703195872998	0\\
4.93683370912084	0\\
4.9466354595117	0\\
4.95643720990256	0\\
4.96628051524823	0\\
4.97612382059389	0\\
4.98596712593956	0\\
4.99581043128522	0\\
5.00569590732134	0\\
5.01558138335745	0\\
5.02546685939356	0\\
5.03535233542967	0\\
5.04528060635073	0\\
5.05520887727179	0\\
5.06513714819285	0\\
5.07506541911391	0\\
5.08503711889176	0\\
5.0950088186696	0\\
5.10498051844744	0\\
5.11495221822528	0\\
5.12496799094167	0\\
5.13498376365806	0\\
5.14499953637444	0\\
5.15501530909083	0\\
5.16507580953502	0\\
5.17513630997921	0\\
5.18519681042339	0\\
5.19525731086758	0\\
5.2053632039849	0\\
5.21546909710223	0\\
5.22557499021955	0\\
5.23568088333688	0\\
5.24583284576096	0\\
5.25598480818505	0\\
5.26613677060914	0\\
5.27628873303322	0\\
5.28648745229033	0\\
5.29668617154743	0\\
5.30688489080453	0\\
5.31708361006163	0\\
5.32732978639303	0\\
5.33757596272443	0\\
5.34782213905582	0\\
5.35806831538722	0\\
5.36836266133408	0\\
5.37865700728095	0\\
5.38895135322782	0\\
5.39924569917468	0\\
5.40958894010065	0\\
5.41993218102661	0\\
5.43027542195257	0\\
5.44061866287853	0\\
5.45101153706255	0\\
5.46140441124658	0\\
5.4717972854306	0\\
5.48219015961463	0\\
5.49263341945276	0\\
5.50307667929089	0\\
5.51351993912902	0\\
5.52396319896715	0\\
5.53445761187213	0\\
5.54495202477712	0\\
5.5554464376821	0\\
5.56594085058709	0\\
5.57648719831941	0\\
5.58703354605173	0\\
5.59757989378405	0\\
5.60812624151637	0\\
5.61872532103302	0\\
5.62932440054967	0\\
5.63992348006633	0\\
5.65052255958298	0\\
5.66117518401188	0\\
5.67182780844078	0\\
5.68248043286967	0\\
5.69313305729857	0\\
5.70384005671797	0\\
5.71454705613736	0\\
5.72525405555675	0\\
5.73596105497615	0\\
5.74672327665992	0\\
5.75748549834369	0\\
5.76824772002746	0\\
5.77900994171123	0\\
5.78982825023888	0\\
5.80064655876654	0\\
5.81146486729419	0\\
5.82228317582184	0\\
5.83315845473553	0\\
5.84403373364921	0\\
5.8549090125629	0\\
5.86578429147658	0\\
5.87671744355088	0\\
5.88765059562518	0\\
5.89858374769948	0\\
5.90951689977378	0\\
5.92050884813521	0\\
5.93150079649664	0\\
5.94249274485807	0\\
5.9534846932195	0\\
5.96453638108293	0\\
5.97558806894636	0\\
5.98663975680979	0\\
5.99769144467322	0\\
6.00880383710542	0\\
6.01991622953761	0\\
6.0310286219698	0\\
6.042141014402	0\\
6.05331509935885	0\\
6.0644891843157	0\\
6.07566326927256	0\\
6.08683735422941	0\\
6.09807414230094	0\\
6.10931093037248	0\\
6.12054771844401	0\\
6.13178450651555	0\\
6.14308503319571	0\\
6.15438555987588	0\\
6.16568608655605	0\\
6.17698661323621	0\\
6.18835193888863	0\\
6.19971726454106	0\\
6.21108259019348	0\\
6.2224479158459	0\\
6.23387912710053	0\\
6.24531033835517	0\\
6.2567415496098	0\\
6.26817276086443	0\\
6.2796709718506	0\\
6.29116918283678	0\\
6.30266739382295	0\\
6.31416560480912	0\\
6.32573195844141	0\\
6.33729831207371	0\\
6.34886466570601	0\\
6.36043101933831	0\\
6.37206668825559	0\\
6.38370235717287	0\\
6.39533802609016	0\\
6.40697369500744	0\\
6.41867988207414	0\\
6.43038606914084	0\\
6.44209225620754	0\\
6.45379844327423	0\\
6.46557638449348	0\\
6.47735432571273	0\\
6.48913226693197	0\\
6.50091020815122	0\\
6.51276117329228	0\\
6.52461213843333	0\\
6.53646310357439	0\\
6.54831406871544	0\\
6.56023936237899	0\\
6.57216465604253	0\\
6.58408994970607	0\\
6.59601524336962	0\\
6.608016207688	0\\
6.62001717200638	0\\
6.63201813632476	0\\
6.64401910064313	0\\
6.65609711612489	0\\
6.66817513160664	0\\
6.68025314708839	0\\
6.69233116257014	0\\
6.70448764976455	0\\
6.71664413695897	0\\
6.72880062415338	0\\
6.7409571113478	0\\
6.75319353351478	0\\
6.76542995568177	0\\
6.77766637784875	0\\
6.78990280001574	0\\
6.80222066442117	0\\
6.8145385288266	0\\
6.82685639323203	0\\
6.83917425763746	0\\
6.85157511856051	0\\
6.86397597948357	0\\
6.87637684040662	0\\
6.88877770132968	0\\
6.90126316148383	0\\
6.91374862163799	0\\
6.92623408179214	0\\
6.93871954194629	0\\
6.95129125486549	0\\
6.96386296778468	0\\
6.97643468070388	0\\
6.98900639362307	0\\
7.00166606764739	0\\
7.0143257416717	0\\
7.02698541569601	0\\
7.03964508972033	0\\
7.05239448925109	0\\
7.06514388878186	0\\
7.07789328831262	0\\
7.09064268784339	0\\
7.10348363709426	0\\
7.11632458634514	0\\
7.12916553559601	0\\
7.14200648484689	0\\
7.15494087138423	0\\
7.16787525792157	0\\
7.18080964445891	0\\
7.19374403099625	0\\
7.20677380774895	0\\
7.21980358450165	0\\
7.23283336125435	0\\
7.24586313800706	0\\
7.25899032889946	0\\
7.27211751979186	0\\
7.28524471068426	0\\
7.29837190157667	0\\
7.31159860422458	0\\
7.32482530687249	0\\
7.3380520095204	0\\
7.35127871216831	0\\
7.36460710304072	0\\
7.37793549391314	0\\
7.39126388478556	0\\
7.40459227565797	0\\
7.41802461355885	0\\
7.43145695145972	0\\
7.4448892893606	0\\
7.45832162726148	0\\
7.47186025959003	0\\
7.48539889191859	0\\
7.49893752424715	0\\
7.51247615657571	0\\
7.52612352432791	0\\
7.53977089208011	0\\
7.55341825983232	0\\
7.56706562758452	0\\
7.58082427117841	0\\
7.5945829147723	0\\
7.60834155836619	0\\
7.62210020196009	0\\
7.63597276754062	0\\
7.64984533312116	0\\
7.66371789870169	0\\
7.67759046428223	0\\
7.69157971106386	0\\
7.70556895784549	0\\
7.71955820462712	0\\
7.73354745140875	0\\
7.7476562591015	0\\
7.76176506679424	0\\
7.77587387448699	0\\
7.78998268217974	0\\
7.80421406003167	0\\
7.81844543788359	0\\
7.83267681573552	0\\
7.84690819358745	0\\
7.86126528820976	0\\
7.87562238283208	0\\
7.8899794774544	0\\
7.90433657207672	0\\
7.91882267896367	0\\
7.93330878585062	0\\
7.94779489273757	0\\
7.96228099962452	0\\
7.97689957365268	0\\
7.99151814768085	0\\
8.00613672170902	0\\
8.02075529573718	0\\
8.03550996276884	0\\
8.0502646298005	0\\
8.06501929683216	0\\
8.07977396386382	0\\
8.09466853533245	0\\
8.10956310680107	0\\
8.12445767826969	0\\
8.13935224973832	0\\
8.15439073744049	0\\
8.16942922514267	0\\
8.18446771284484	0\\
8.19950620054702	0\\
8.2146928332504	0\\
8.22987946595378	0\\
8.24506609865717	0\\
8.26025273136055	0\\
8.27559197362809	0\\
8.29093121589562	0\\
8.30627045816316	0\\
8.32160970043069	0\\
8.33710627308224	0\\
8.35260284573378	0\\
8.36809941838533	0\\
8.38359599103687	0\\
8.39925489573158	0\\
8.41491380042629	0\\
8.430572705121	0\\
8.44623160981571	0\\
8.46205815432056	0\\
8.4778846988254	0\\
8.49371124333024	0\\
8.50953778783509	0\\
8.52553761739547	0\\
8.54153744695585	0\\
8.55753727651623	0\\
8.57353710607661	0\\
8.58971623619016	0\\
8.60589536630372	0\\
8.62207449641727	0\\
8.63825362653082	0\\
8.65461848330403	0\\
8.67098334007723	0\\
8.68734819685043	0\\
8.70371305362364	0\\
8.72018188980809	0\\
8.73665072599255	0\\
8.75311956217701	0\\
8.76958839836146	0\\
8.78614348346302	0\\
8.80269856856458	0\\
8.81925365366614	0\\
8.83580873876771	0\\
8.85245111606455	0\\
8.86909349336139	0\\
8.88573587065824	0\\
8.90237824795508	0\\
8.91910883699713	0\\
8.93583942603918	0\\
8.95257001508123	0\\
8.96930060412328	0\\
8.98612034063892	0\\
9.00294007715455	0\\
9.01975981367019	0\\
9.03657955018582	0\\
9.05348938408815	0\\
9.07039921799048	0\\
9.08730905189281	0\\
9.10421888579513	0\\
9.12121978296026	0\\
9.13822068012539	0\\
9.15522157729052	0\\
9.17222247445565	0\\
9.18931541587169	0\\
9.20640835728774	0\\
9.22350129870378	0\\
9.24059424011982	0\\
9.25778022278737	0\\
9.27496620545493	0\\
9.29215218812249	0\\
9.30933817079004	0\\
9.32661820779546	0\\
9.34389824480088	0\\
9.3611782818063	0\\
9.37845831881172	0\\
9.39583344002363	0\\
9.41320856123555	0\\
9.43058368244747	0\\
9.44795880365938	0\\
9.46543005570071	0\\
9.48290130774205	0\\
9.50037255978338	0\\
9.51784381182471	0\\
9.53541225878365	0\\
9.55298070574258	0\\
9.57054915270152	0\\
9.58811759966045	0\\
9.60578432346384	0\\
9.62345104726722	0\\
9.64111777107061	0\\
9.658784494874	0\\
9.67655059520445	0\\
9.6943166955349	0\\
9.71208279586536	0\\
9.72984889619581	0\\
9.74771549159443	0\\
9.76558208699305	0\\
9.78344868239167	0\\
9.80131527779029	0\\
9.81928350543996	0\\
9.83725173308963	0\\
9.8552199607393	0\\
9.87318818838897	0\\
9.89125920482513	0\\
9.90933022126129	0\\
9.92740123769745	0\\
9.94547225413361	0\\
9.96364723590061	0\\
9.9818222176676	0\\
9.99999719943459	0\\
10.0181721812016	0\\
10.0364523247657	0\\
10.0547324683298	0\\
10.0730126118939	0\\
10.0912927554579	0\\
10.1096792783833	0\\
10.1280658013086	0\\
10.146452324234	0\\
10.1648388471593	0\\
10.1833329877765	0\\
10.2018271283938	0\\
10.220321269011	0\\
10.2388154096282	0\\
10.257418428294	0\\
10.2760214469598	0\\
10.2946244656257	0\\
10.3132274842915	0\\
10.3319406634311	0\\
10.3506538425707	0\\
10.3693670217104	0\\
10.38808020085	0\\
10.4069048454226	0\\
10.4257294899953	0\\
10.4445541345679	0\\
10.4633787791406	0\\
10.4823162175897	0\\
10.5012536560388	0\\
10.5201910944879	0\\
10.539128532937	0\\
10.5581801172831	0\\
10.5772317016292	0\\
10.5962832859753	0\\
10.6153348703214	0\\
10.6345019773376	0\\
10.6536690843538	0\\
10.67283619137	0\\
10.6920032983861	0\\
10.7112873293394	0\\
10.7305713602926	0\\
10.7498553912458	0\\
10.769139422199	0\\
10.7885418041401	0\\
10.8079441860811	0\\
10.8273465680222	0\\
10.8467489499633	0\\
10.8662711360792	0\\
10.8857933221952	0\\
10.9053155083112	0\\
10.9248376944271	0\\
10.9444811647161	0\\
10.9641246350051	0\\
10.9837681052941	0\\
11.0034115755831	0\\
11.02317783773	0\\
11.0429440998769	0\\
11.0627103620238	0\\
11.0824766241707	0\\
11.102367213876	0\\
11.1222578035813	0\\
11.1421483932866	0\\
11.1620389829919	0\\
11.182055464688	0\\
11.202071946384	0\\
11.22208842808	0\\
11.2421049097761	0\\
11.2622488777205	0\\
11.282392845665	0\\
11.3025368136094	0\\
11.3226807815539	0\\
11.3429428732039	0\\
11.363204964854	0\\
11.3834670565041	0\\
11.4037291481542	0\\
11.4240073959591	0\\
11.444285643764	0\\
11.464563891569	0\\
11.4848421393739	0\\
11.5051410232308	0\\
11.5254399070878	0\\
11.5457387909448	0\\
11.5660376748017	0\\
11.5863613245615	0\\
11.6066849743212	0\\
11.627008624081	0\\
11.6473322738407	0\\
11.6676846007299	0\\
11.6880369276191	0\\
11.7083892545083	0\\
11.7287415813975	0\\
11.7491263024508	0\\
11.7695110235041	0\\
11.7898957445574	0\\
11.8102804656107	0\\
11.8307011258767	0\\
11.8511217861426	0\\
11.8715424464086	0\\
11.8919631066745	0\\
11.9124230970895	0\\
11.9328830875046	0\\
11.9533430779196	0\\
11.9738030683347	0\\
11.9943056417768	0\\
12.014808215219	0\\
12.0353107886612	0\\
12.0558133621033	0\\
12.0763616495731	0\\
12.0969099370428	0\\
12.1174582245125	0\\
12.1380065119823	0\\
12.1586035338403	0\\
12.1792005556983	0\\
12.1997975775563	0\\
12.2203945994143	0\\
12.2410432774846	0\\
12.2616919555548	0\\
12.2823406336251	0\\
12.3029893116954	0\\
12.3236924799435	0\\
12.3443956481917	0\\
12.3650988164398	0\\
12.385801984688	0\\
12.406562398648	0\\
12.4273228126081	0\\
12.4480832265682	0\\
12.4688436405282	0\\
12.4896639845707	0\\
12.5104843286133	0\\
12.5313046726558	0\\
12.5521250166983	0\\
12.5730079121072	0\\
12.5938908075161	0\\
12.614773702925	0\\
12.6356565983338	0\\
12.6566046110283	0\\
12.6775526237227	0\\
12.6985006364172	0\\
12.7194486491116	0\\
12.7404642942679	0\\
12.7614799394242	0\\
12.7824955845805	0\\
12.8035112297368	0\\
12.8245969793381	0\\
12.8456827289394	0\\
12.8667684785407	0\\
12.887854228142	0\\
12.9090125143938	0\\
12.9301708006455	0\\
12.9513290868973	0\\
12.972487373149	0\\
12.9937205946389	0\\
13.0149538161287	0\\
13.0361870376186	0\\
13.0574202591085	0\\
13.0787307850171	0\\
13.1000413109258	0\\
13.1213518368345	0\\
13.1426623627431	0\\
13.1640525369264	0\\
13.1854427111097	0\\
13.206832885293	0\\
13.2282230594763	0\\
13.2496952043377	0\\
13.271167349199	0\\
13.2926394940603	0\\
13.3141116389217	0\\
13.3356680588802	0\\
13.3572244788388	0\\
13.3787808987974	0\\
13.400337318756	0\\
13.4219803036483	0\\
13.4436232885406	0\\
13.4652662734329	0\\
13.4869092583252	0\\
13.5086410871513	0\\
13.5303729159775	0\\
13.5521047448036	0\\
13.5738365736297	0\\
13.5956595174591	0\\
13.6174824612885	0\\
13.639305405118	0\\
13.6611283489474	0\\
13.6830446722324	0\\
13.7049609955174	0\\
13.7268773188025	0\\
13.7487936420875	0\\
13.770805608124	0\\
13.7928175741606	0\\
13.8148295401972	0\\
13.8368415062338	0\\
13.858951377984	0\\
13.8810612497342	0\\
13.9031711214845	0\\
13.9252809932347	0\\
13.9474910360997	0\\
13.9697010789646	0\\
13.9919111218296	0\\
14.0141211646945	0\\
14.0364336494706	0\\
14.0587461342466	0\\
14.0810586190226	0\\
14.1033711037986	0\\
14.1257883079129	0\\
14.1482055120272	0\\
14.1706227161414	0\\
14.1930399202557	0\\
14.2155641320733	0\\
14.2380883438909	0\\
14.2606125557085	0\\
14.283136767526	0\\
14.3057702854045	0\\
14.3284038032829	0\\
14.3510373211614	0\\
14.3736708390398	0\\
14.3964159769655	0\\
14.4191611148912	0\\
14.4419062528169	0\\
14.4646513907425	0\\
14.4875104777278	0\\
14.510369564713	0\\
14.5332286516983	0\\
14.5560877386835	0\\
14.5790631229979	0\\
14.6020385073122	0\\
14.6250138916265	0\\
14.6479892759408	0\\
14.6710833264109	0\\
14.6941773768809	0\\
14.7172714273509	0\\
14.740365477821	0\\
14.7635805844029	0\\
14.7867956909849	0\\
14.8100107975669	0\\
14.8332259041488	0\\
14.8565644820063	0\\
14.8799030598637	0\\
14.9032416377212	0\\
14.9265802155787	0\\
14.9500447068131	0\\
14.9735091980476	0\\
14.996973689282	0\\
15.0204381805165	0\\
15.0440310551214	0\\
15.0676239297264	0\\
15.0912168043314	0\\
15.1148096789363	0\\
15.1385334385982	0\\
15.16225719826	0\\
15.1859809579218	0\\
15.2097047175837	0\\
15.2335618956234	0\\
15.2574190736632	0\\
15.281276251703	0\\
15.3051334297427	0\\
15.3291265950906	0\\
15.3531197604384	0\\
15.3771129257863	0\\
15.4011060911341	0\\
15.4252378496272	0\\
15.4493696081203	0\\
15.4735013666134	0\\
15.4976331251065	0\\
15.5219061221034	0\\
15.5461791191004	0\\
15.5704521160973	0\\
15.5947251130942	0\\
15.619142035055	0\\
15.6435589570157	0\\
15.6679758789765	0\\
15.6923928009372	0\\
15.7169563780926	0\\
15.741519955248	0\\
15.7660835324033	0\\
15.7906471095587	0\\
15.8153601179615	0\\
15.8400731263644	0\\
15.8647861347672	0\\
15.88949914317	0\\
15.9143644081624	0\\
15.9392296731548	0\\
15.9640949381471	0\\
15.9889602031395	0\\
16.0139806003288	0\\
16.039000997518	0\\
16.0640213947072	0\\
16.0890417918965	0\\
16.1142202516767	0\\
16.1393987114569	0\\
16.1645771712372	0\\
16.1897556310174	0\\
16.2150951374624	0\\
16.2404346439074	0\\
16.2657741503524	0\\
16.2911136567974	0\\
16.3166172552097	0\\
16.342120853622	0\\
16.3676244520343	0\\
16.3931280504465	0\\
16.4187988465862	0\\
16.4444696427259	0\\
16.4701404388656	0\\
16.4958112350053	0\\
16.5216524004798	0\\
16.5474935659542	0\\
16.5733347314287	0\\
16.5991758969032	0\\
16.6251906694304	0\\
16.6512054419577	0\\
16.677220214485	0\\
16.7032349870123	0\\
16.7294266758655	0\\
16.7556183647187	0\\
16.781810053572	0\\
16.8080017424252	0\\
16.8343737319713	0\\
16.8607457215174	0\\
16.8871177110636	0\\
16.9134897006097	0\\
16.9400454511807	0\\
16.9666012017517	0\\
16.9931569523226	0\\
17.0197127028936	0\\
17.0464557582069	0\\
17.0731988135202	0\\
17.0999418688336	0\\
17.1266849241469	0\\
17.1536189111414	0\\
17.1805528981359	0\\
17.2074868851304	0\\
17.2344208721249	0\\
17.2615495082445	0\\
17.2886781443641	0\\
17.3158067804836	0\\
17.3429354166032	0\\
17.3702625121198	0\\
17.3975896076364	0\\
17.424916703153	0\\
17.4522437986696	0\\
17.4797732620512	0\\
17.5073027254328	0\\
17.5348321888143	0\\
17.5623616521959	0\\
17.5900974943972	0\\
17.6178333365984	0\\
17.6455691787997	0\\
17.6733050210009	0\\
17.7012513604556	0\\
17.7291976999103	0\\
17.757144039365	0\\
17.7850903788197	0\\
17.8132514459017	0\\
17.8414125129837	0\\
17.8695735800657	0\\
17.8977346471477	0\\
17.9261147917276	0\\
17.9544949363076	0\\
17.9828750808876	0\\
18.0112552254676	0\\
18.0398589202161	0\\
18.0684626149646	0\\
18.097066309713	0\\
18.1256700044615	0\\
18.1545018545664	0\\
18.1833337046712	0\\
18.212165554776	0\\
18.2409974048809	0\\
18.2700621491713	0\\
18.2991268934616	0\\
18.328191637752	0\\
18.3572563820424	0\\
18.3865589068877	0\\
18.4158614317331	0\\
18.4451639565785	0\\
18.4744664814239	0\\
18.5040118216459	0\\
18.5335571618679	0\\
18.56310250209	0\\
18.592647842312	0\\
18.622441195706	0\\
18.6522345491	0\\
18.682027902494	0\\
18.711821255888	0\\
18.74186798467	0\\
18.771914713452	0\\
18.801961442234	0\\
18.8320081710161	0\\
18.8623138179204	0\\
18.8926194648247	0\\
18.922925111729	0\\
18.9532307586333	0\\
18.9838010512769	0\\
19.0143713439205	0\\
19.0449416365641	0\\
19.0755119292076	0\\
19.1063527942164	0\\
19.1371936592252	0\\
19.1680345242339	0\\
19.1988753892427	0\\
19.2299929618915	0\\
19.2611105345403	0\\
19.2922281071891	0\\
19.3233456798379	0\\
19.3547463174076	0\\
19.3861469549774	0\\
19.4175475925471	0\\
19.4489482301169	0\\
19.4806385226382	0\\
19.5123288151595	0\\
19.5440191076808	0\\
19.5757094002021	0\\
19.6076961882754	0\\
19.6396829763487	0\\
19.671669764422	0\\
19.7036565524953	0\\
19.7359469398787	0\\
19.768237327262	0\\
19.8005277146453	0\\
19.8328181020287	0\\
19.8654194755566	0\\
19.8980208490846	0\\
19.9306222226126	0\\
19.9632235961405	0\\
19.9961436388709	0\\
20.0290636816012	0\\
20.0619837243316	0\\
20.0949037670619	0\\
20.1281504830269	0\\
20.1613971989918	0\\
20.1946439149567	0\\
20.2278906309217	0\\
20.2614723645969	0\\
20.2950540982721	0\\
20.3286358319473	0\\
20.3622175656225	0\\
20.3961430262838	0\\
20.4300684869452	0\\
20.4639939476066	0\\
20.497919408268	0\\
20.5321976950164	0\\
20.5664759817649	0\\
20.6007542685134	0\\
20.6350325552618	0\\
20.6696731863085	0\\
20.7043138173553	0\\
20.738954448402	0\\
20.7735950794487	0\\
20.8086080231198	0\\
20.8436209667909	0\\
20.878633910462	0\\
20.9136468541332	0\\
20.949042565344	0\\
20.9844382765548	0\\
21.0198339877656	0\\
21.0552296989764	0\\
21.0910191546516	0\\
21.1268086103268	0\\
21.162598066002	0\\
21.1983875216773	0\\
21.234582265905	0\\
21.2707770101326	0\\
21.3069717543603	0\\
21.343166498588	0\\
21.3797786897193	0\\
21.4163908808506	0\\
21.453003071982	0\\
21.4896152631133	0\\
21.5266577262528	0\\
21.5637001893923	0\\
21.6007426525319	0\\
21.6377851156714	0\\
21.6752714017348	0\\
21.7127576877983	0\\
21.7502439738617	0\\
21.7877302599251	0\\
21.8256747161126	0\\
21.8636191723001	0\\
21.9015636284876	0\\
21.9395080846751	0\\
21.9779259262067	0\\
22.0163437677384	0\\
22.0547616092701	0\\
22.0931794508017	0\\
22.1320868496398	0\\
22.1709942484778	0\\
22.2099016473158	0\\
22.2488090461539	0\\
22.2882232273236	0\\
22.3276374084934	0\\
22.3670515896632	0\\
22.4064657708329	0\\
22.4464051256478	0\\
22.4863444804626	0\\
22.5262838352775	0\\
22.5662231900924	0\\
22.6067074061995	0\\
22.6471916223066	0\\
22.6876758384138	0\\
22.7281600545209	0\\
22.7692102626757	0\\
22.8102604708306	0\\
22.8513106789854	0\\
22.8923608871403	0\\
22.9339998357608	0\\
22.9756387843813	0\\
23.0172777330018	0\\
23.0589166816223	0\\
23.1011689388352	0\\
23.143421196048	0\\
23.1856734532609	0\\
23.2279257104737	0\\
23.2707449436263	0\\
23.3135641767789	0\\
23.3563834099314	0\\
23.399202643084	0\\
23.4425351018712	0\\
23.4858675606585	0\\
23.5292000194457	0\\
23.5725324782329	0\\
23.6163923812599	0\\
23.6602522842869	0\\
23.704112187314	0\\
23.747972090341	0\\
23.7923730104802	0\\
23.8367739306194	0\\
23.8811748507586	0\\
23.9255757708978	0\\
23.9705318120247	0\\
24.0154878531517	0\\
24.0604438942787	0\\
24.1053999354057	0\\
24.1509257391711	0\\
24.1964515429365	0\\
24.2419773467019	0\\
24.2875031504673	0\\
24.3336139249026	0\\
24.3797246993379	0\\
24.4258354737732	0\\
24.4719462482085	0\\
24.518657797687	0\\
24.5653693471654	0\\
24.6120808966439	0\\
24.6587924461223	0\\
24.7061212033434	0\\
24.7534499605645	0\\
24.8007787177855	0\\
24.8481074750066	0\\
24.8960705351928	0\\
24.944033595379	0\\
24.9919966555653	0\\
25.0399597157515	0\\
25.0885748730688	0\\
25.1371900303861	0\\
25.1858051877034	0\\
25.2344203450208	0\\
25.2837061315614	0\\
25.332991918102	0\\
25.3822777046427	0\\
25.4315634911833	0\\
25.4815392186928	0\\
25.5315149462023	0\\
25.5814906737117	0\\
25.6314664012212	0\\
25.6821522057685	0\\
25.7328380103157	0\\
25.783523814863	0\\
25.8342096194102	0\\
25.8856265092156	0\\
25.9370433990211	0\\
25.9884602888265	0\\
26.0398771786319	0\\
26.0920470854784	0\\
26.144216992325	0\\
26.1963868991715	0\\
26.2485568060181	0\\
26.3015026403881	0\\
26.3544484747581	0\\
26.4073943091281	0\\
26.4603401434981	0\\
26.5140858539001	0\\
26.5678315643021	0\\
26.6215772747041	0\\
26.6753229851061	0\\
26.7298936218998	0\\
26.7844642586936	0\\
26.8390348954873	0\\
26.893605532281	0\\
26.9490273164448	0\\
27.0044491006085	0\\
27.0598708847722	0\\
27.1152926689359	0\\
27.1715930662283	0\\
27.2278934635207	0\\
27.2841938608131	0\\
27.3404942581055	0\\
27.3972810276665	0\\
27.4540677972275	0\\
27.5108545667884	0\\
27.5676413363494	0\\
27.6246835621125	0\\
27.6817257878755	0\\
27.7387680136386	0\\
27.7958102394016	0\\
27.8531117250382	0\\
27.9104132106748	0\\
27.9677146963114	0\\
28.025016181948	0\\
28.0825792753208	0\\
28.1401423686935	0\\
28.1977054620663	0\\
28.2552685554391	0\\
28.3130956457549	0\\
28.3709227360708	0\\
28.4287498263866	0\\
28.4865769167025	0\\
28.5446704260258	0\\
28.602763935349	0\\
28.6608574446723	0\\
28.7189509539956	0\\
28.7773133374	0\\
28.8356757208045	0\\
28.8940381042089	0\\
28.9524004876134	0\\
29.0110342352743	0\\
29.0696679829352	0\\
29.1283017305961	0\\
29.186935478257	0\\
29.2458431140456	0\\
29.3047507498342	0\\
29.3636583856228	0\\
29.4225660214114	0\\
29.4817501049045	0\\
29.5409341883977	0\\
29.6001182718908	0\\
29.659302355384	0\\
29.7187654815208	0\\
29.7782286076576	0\\
29.8376917337944	0\\
29.8971548599312	0\\
29.9568996607722	0\\
30.0166444616132	0\\
30.0763892624543	0\\
30.1361340632953	0\\
30.1961632086509	0\\
30.2561923540065	0\\
30.316221499362	0\\
30.3762506447176	0\\
30.436566841334	0\\
30.4968830379503	0\\
30.5571992345667	0\\
30.617515431183	0\\
30.6781214246657	0\\
30.7387274181483	0\\
30.799333411631	0\\
30.8599394051136	0\\
30.9208379817972	0\\
30.9817365584807	0\\
31.0426351351642	0\\
31.1035337118477	0\\
31.1647276970968	0\\
31.2259216823458	0\\
31.2871156675948	0\\
31.3483096528438	0\\
31.4098019130839	0\\
31.4712941733241	0\\
31.5327864335642	0\\
31.5942786938043	0\\
31.656072138367	0\\
31.7178655829298	0\\
31.7796590274925	0\\
31.8414524720552	0\\
31.903550051626	0\\
31.9656476311968	0\\
32.0277452107676	0\\
32.0898427903383	0\\
32.1522474993513	0\\
32.2146522083642	0\\
32.2770569173772	0\\
32.3394616263901	0\\
32.4021765035977	0\\
32.4648913808053	0\\
32.527606258013	0\\
32.5903211352206	0\\
32.6533492654047	0\\
32.7163773955889	0\\
32.7794055257731	0\\
32.8424336559572	0\\
32.9057781690946	0\\
32.969122682232	0\\
33.0324671953693	0\\
33.0958117085067	0\\
33.1594757818323	0\\
33.2231398551579	0\\
33.2868039284835	0\\
33.3504680018091	0\\
33.4144548595075	0\\
33.478441717206	0\\
33.5424285749044	0\\
33.6064154326028	0\\
33.6707283499096	0\\
33.7350412672163	0\\
33.7993541845231	0\\
33.8636671018298	0\\
33.9283094023156	0\\
33.9929517028013	0\\
34.057594003287	0\\
34.1222363037727	0\\
34.1872113622324	0\\
34.2521864206922	0\\
34.3171614791519	0\\
34.3821365376116	0\\
34.447447781095	0\\
34.5127590245783	0\\
34.5780702680616	0\\
34.643381511545	0\\
34.7090324201268	0\\
34.7746833287086	0\\
34.8403342372905	0\\
34.9059851458723	0\\
34.9719792533529	0\\
35.0379733608335	0\\
35.1039674683141	0\\
35.1699615757948	0\\
35.2363024710846	0\\
35.3026433663745	0\\
35.3689842616644	0\\
35.4353251569542	0\\
35.5020164861975	0\\
35.5687078154407	0\\
35.6353991446839	0\\
35.7020904739271	0\\
35.76913594038	0\\
35.8361814068328	0\\
35.9032268732857	0\\
35.9702723397385	0\\
36.0376757049608	0\\
36.1050790701831	0\\
36.1724824354054	0\\
36.2398858006277	0\\
36.3076508877251	0\\
36.3754159748225	0\\
36.4431810619199	0\\
36.5109461490173	0\\
36.5790768402747	0\\
36.6472075315322	0\\
36.7153382227897	0\\
36.7834689140472	0\\
36.8519691566203	0\\
36.9204693991935	0\\
36.9889696417667	0\\
37.0574698843398	0\\
37.1263436888854	0\\
37.1952174934311	0\\
37.2640912979767	0\\
37.3329651025223	0\\
37.4022165438372	0\\
37.4714679851522	0\\
37.5407194264671	0\\
37.609970867782	0\\
37.6796040887873	0\\
37.7492373097926	0\\
37.8188705307979	0\\
37.8885037518033	0\\
37.9585229632751	0\\
38.0285421747469	0\\
38.0985613862187	0\\
38.1685805976905	0\\
38.2389900798134	0\\
38.3093995619363	0\\
38.3798090440592	0\\
38.4502185261821	0\\
38.5209622984945	0\\
38.591706070807	0\\
38.6624498431194	0\\
38.7331936154318	0\\
38.8042656863826	0\\
38.8753377573333	0\\
38.9464098282841	0\\
39.0174818992348	0\\
39.0888909542047	0\\
39.1603000091745	0\\
39.2317090641443	0\\
39.3031181191142	0\\
39.3748725127202	0\\
39.4466269063262	0\\
39.5183812999322	0\\
39.5901356935382	0\\
39.662243568804	0\\
39.7343514440698	0\\
39.8064593193355	0\\
39.8785671946013	0\\
39.95103651206	0\\
40.0235058295186	0\\
40.0959751469772	0\\
40.1684444644358	0\\
40.2412830193083	0\\
40.3141215741807	0\\
40.3869601290532	0\\
40.4597986839256	0\\
40.5330141309314	0\\
40.6062295779373	0\\
40.6794450249431	0\\
40.752660471949	0\\
40.8262603410054	0\\
40.8998602100619	0\\
40.9734600791183	0\\
41.0470599481748	0\\
41.1210516636287	0\\
41.1950433790826	0\\
41.2690350945364	0\\
41.3430268099903	0\\
41.4174177047231	0\\
41.4918085994559	0\\
41.5661994941887	0\\
41.6405903889215	0\\
41.7153877210576	0\\
41.7901850531936	0\\
41.8649823853296	0\\
41.9397797174656	0\\
42.0149906816255	0\\
42.0902016457854	0\\
42.1654126099454	0\\
42.2406235741053	0\\
42.316255315397	0\\
42.3918870566888	0\\
42.4675187979805	0\\
42.5431505392722	0\\
42.6192101664345	0\\
42.6952697935968	0\\
42.771329420759	0\\
42.8473890479213	0\\
42.9238836429179	0\\
43.0003782379145	0\\
43.0768728329111	0\\
43.1533674279077	0\\
43.2303040564086	0\\
43.3072406849096	0\\
43.3841773134105	0\\
43.4611139419115	0\\
43.5384996649489	0\\
43.6158853879863	0\\
43.6932711110238	0\\
43.7706568340612	0\\
43.8484987166093	0\\
43.9263405991575	0\\
44.0041824817057	0\\
44.0820243642538	0\\
44.1603294838729	0\\
44.2386346034919	0\\
44.3169397231109	0\\
44.39524484273	0\\
44.4740202988846	0\\
44.5527957550393	0\\
44.631571211194	0\\
44.7103466673487	0\\
44.7895995895968	0\\
44.8688525118449	0\\
44.948105434093	0\\
45.0273583563411	0\\
45.1070959123988	0\\
45.1868334684566	0\\
45.2665710245144	0\\
45.3463085805721	0\\
45.4265379845256	0\\
45.5067673884792	0\\
45.5869967924327	0\\
45.6672261963862	0\\
45.7479547148514	0\\
45.8286832333165	0\\
45.9094117517817	0\\
45.9901402702469	0\\
46.0713752310891	0\\
46.1526101919314	0\\
46.2338451527736	0\\
46.3150801136159	0\\
46.3968289125376	0\\
46.4785777114593	0\\
46.560326510381	0\\
46.6420753093027	0\\
46.7243454163926	0\\
46.8066155234824	0\\
46.8888856305723	0\\
46.9711557376621	0\\
47.0539547042701	0\\
47.136753670878	0\\
47.2195526374859	0\\
47.3023516040939	0\\
47.3856870701744	0\\
47.4690225362549	0\\
47.5523580023354	0\\
47.6356934684159	0\\
47.7195731680798	0\\
47.8034528677438	0\\
47.8873325674077	0\\
47.9712122670716	0\\
48.0556440359198	0\\
48.1400758047679	0\\
48.2245075736161	0\\
48.3089393424642	0\\
48.3939311234603	0\\
48.4789229044563	0\\
48.5639146854524	0\\
48.6489064664485	0\\
48.7344663172648	0\\
48.820026168081	0\\
48.9055860188973	0\\
48.9911458697136	0\\
49.0772819670679	0\\
49.1634180644223	0\\
49.2495541617767	0\\
49.3356902591311	0\\
49.4224109073406	0\\
49.5091315555501	0\\
49.5958522037596	0\\
49.6825728519691	0\\
49.769886487838	0\\
49.857200123707	0\\
49.944513759576	0\\
50.0318273954449	0\\
50.1197425956162	0\\
50.2076577957875	0\\
50.2955729959588	0\\
50.3834881961302	0\\
50.4720136825001	0\\
50.5605391688701	0\\
50.64906465524	0\\
50.73759014161	0\\
50.8267347882915	0\\
50.915879434973	0\\
51.0050240816545	0\\
51.094168728336	0\\
51.1839415683769	0\\
51.2737144084178	0\\
51.3634872484587	0\\
51.4532600884996	0\\
51.5436703203087	0\\
51.6340805521178	0\\
51.7244907839268	0\\
51.8149010157359	0\\
51.9059580087448	0\\
51.9970150017537	0\\
52.0880719947626	0\\
52.1791289877716	0\\
52.270842290771	0\\
52.3625555937704	0\\
52.4542688967698	0\\
52.5459821997693	0\\
52.6383615465513	0\\
52.7307408933333	0\\
52.8231202401153	0\\
52.9154995868973	0\\
53.0085549039533	0\\
53.1016102210094	0\\
53.1946655380654	0\\
53.2877208551214	0\\
53.3814622684928	0\\
53.4752036818643	0\\
53.5689450952358	0\\
53.6626865086073	0\\
53.7571243510415	0\\
53.8515621934757	0\\
53.9460000359099	0\\
54.0404378783442	0\\
54.1355826971555	0\\
54.2307275159668	0\\
54.325872334778	0\\
54.4210171535893	0\\
54.5168797176294	0\\
54.6127422816695	0\\
54.7086048457095	0\\
54.8044674097496	0\\
54.9010587178797	0\\
54.9976500260098	0\\
55.0942413341398	0\\
55.1908326422699	0\\
55.2881639308243	0\\
55.3854952193786	0\\
55.4828265079329	0\\
55.5801577964873	0\\
55.6782405480441	0\\
55.776323299601	0\\
55.8744060511579	0\\
55.9724888027147	0\\
56.0713347542828	0\\
56.1701807058509	0\\
56.2690266574189	0\\
56.367872608987	0\\
56.4674937609263	0\\
56.5671149128656	0\\
56.6667360648049	0\\
56.7663572167441	0\\
56.8667658411891	0\\
56.9671744656341	0\\
57.0675830900791	0\\
57.1679917145241	0\\
57.2692003658787	0\\
57.3704090172333	0\\
57.4716176685879	0\\
57.5728263199425	0\\
57.6748478434472	0\\
57.7768693669519	0\\
57.8788908904565	0\\
57.9809124139612	0\\
58.0837599560178	0\\
58.1866074980744	0\\
58.2894550401309	0\\
58.3923025821875	0\\
58.4959896007833	0\\
58.5996766193791	0\\
58.7033636379749	0\\
58.8070506565707	0\\
58.911590931702	0\\
59.0161312068333	0\\
59.1206714819646	0\\
59.2252117570959	0\\
59.3306194021433	0\\
59.4360270471907	0\\
59.541434692238	0\\
59.6468423372854	0\\
59.7351317529641	0\\
59.8234211686427	0\\
59.9117105843214	0\\
60	0\\
};
\addlegendentry{Sani}

\addplot [color=mycolor2, line width=2.0pt]
  table[row sep=crcr]{%
0	0\\
0.00199053585276749	0\\
0.00398107170553497	0\\
0.00597160755830246	0\\
0.00796214341106994	0\\
0.0123567669950473	0\\
0.0167513905790247	0\\
0.0211460141630021	1.11022302462516e-16\\
0.0255406377469794	1.11022302462516e-16\\
0.0299388563931551	1.11022302462516e-16\\
0.0343370750393307	1.11022302462516e-16\\
0.0387352936855064	1.11022302462516e-16\\
0.043133512331682	1.11022302462516e-16\\
0.0475471725191068	0\\
0.0519608327065316	1.11022302462516e-16\\
0.0563744928939564	1.11022302462516e-16\\
0.0607881530813811	1.11022302462516e-16\\
0.0652173199794049	1.11022302462516e-16\\
0.0696464868774286	0\\
0.0740756537754524	0\\
0.0785048206734761	0\\
0.0829496187548012	0\\
0.0873944168361263	0\\
0.0918392149174515	1.11022302462516e-16\\
0.0962840129987766	0\\
0.100744568101794	1.11022302462516e-16\\
0.105205123204811	0\\
0.109665678307829	0\\
0.114126233410846	0\\
0.118602673075784	1.11022302462516e-16\\
0.123079112740721	1.11022302462516e-16\\
0.127555552405659	0\\
0.132031992070596	0\\
0.136524445545197	0\\
0.141016899019798	1.11022302462516e-16\\
0.1455093524944	0\\
0.150001805969001	0\\
0.154510404254387	0\\
0.159019002539772	0\\
0.163527600825158	0\\
0.168036199110544	0\\
0.172561074936287	0\\
0.17708595076203	0\\
0.181610826587773	1.11022302462516e-16\\
0.186135702413515	1.11022302462516e-16\\
0.190676990405524	1.11022302462516e-16\\
0.195218278397533	1.11022302462516e-16\\
0.199759566389542	1.11022302462516e-16\\
0.204300854381551	1.11022302462516e-16\\
0.208858690959197	1.11022302462516e-16\\
0.213416527536843	0\\
0.217974364114488	0\\
0.222532200692134	0\\
0.227106724202768	1.11022302462516e-16\\
0.231681247713401	0\\
0.236255771224035	0\\
0.240830294734669	0\\
0.245421645441331	0\\
0.250012996147992	0\\
0.254604346854654	0\\
0.259195697561315	0\\
0.263804017673384	1.11022302462516e-16\\
0.268412337785452	0\\
0.273020657897521	0\\
0.27762897800959	1.11022302462516e-16\\
0.282254411746462	1.11022302462516e-16\\
0.286879845483334	1.11022302462516e-16\\
0.291505279220206	1.11022302462516e-16\\
0.296130712957079	0\\
0.300773406613671	0\\
0.305416100270264	0\\
0.310058793926856	1.11022302462516e-16\\
0.314701487583449	1.11022302462516e-16\\
0.319361589521735	2.22044604925031e-16\\
0.324021691460022	1.11022302462516e-16\\
0.328681793398309	1.11022302462516e-16\\
0.333341895336596	1.11022302462516e-16\\
0.338019556071109	1.11022302462516e-16\\
0.342697216805623	0\\
0.347374877540136	0\\
0.35205253827465	1.11022302462516e-16\\
0.356747910447269	1.11022302462516e-16\\
0.361443282619888	0\\
0.366138654792507	0\\
0.370834026965126	0\\
0.375547265516765	0\\
0.380260504068404	0\\
0.384973742620042	0\\
0.389686981171681	0\\
0.394418243246227	0\\
0.399149505320772	0\\
0.403880767395317	0\\
0.408612029469863	0\\
0.41336147459102	0\\
0.418110919712177	1.11022302462516e-16\\
0.422860364833334	0\\
0.427609809954491	0\\
0.432377599961091	1.11022302462516e-16\\
0.437145389967691	1.11022302462516e-16\\
0.44191317997429	0\\
0.44668096998089	1.11022302462516e-16\\
0.451467269141767	1.11022302462516e-16\\
0.456253568302644	1.11022302462516e-16\\
0.461039867463521	1.11022302462516e-16\\
0.465826166624398	0\\
0.47063114173014	0\\
0.475436116835881	0\\
0.480241091941623	0\\
0.485046067047365	1.11022302462516e-16\\
0.489869887370685	1.11022302462516e-16\\
0.494693707694006	1.11022302462516e-16\\
0.499517528017326	1.11022302462516e-16\\
0.504341348340647	1.11022302462516e-16\\
0.509184185682064	1.11022302462516e-16\\
0.51402702302348	1.11022302462516e-16\\
0.518869860364897	2.22044604925031e-16\\
0.523712697706314	1.11022302462516e-16\\
0.528574726619363	1.11022302462516e-16\\
0.533436755532412	1.11022302462516e-16\\
0.53829878444546	2.22044604925031e-16\\
0.543160813358509	1.11022302462516e-16\\
0.548042211039569	0\\
0.55292360872063	1.11022302462516e-16\\
0.55780500640169	1.11022302462516e-16\\
0.56268640408275	0\\
0.56758735048985	0\\
0.572488296896949	1.11022302462516e-16\\
0.577389243304049	0\\
0.582290189711149	1.11022302462516e-16\\
0.587210867601438	1.11022302462516e-16\\
0.592131545491728	0\\
0.597052223382018	0\\
0.601972901272307	0\\
0.606913496388328	0\\
0.611854091504349	0\\
0.61679468662037	0\\
0.621735281736391	0\\
0.626695982628195	0\\
0.631656683519999	0\\
0.636617384411802	0\\
0.641578085303606	0\\
0.646559083672327	0\\
0.651540082041047	0\\
0.656521080409768	1.11022302462516e-16\\
0.661502078778488	1.11022302462516e-16\\
0.666503569383853	1.11022302462516e-16\\
0.671505059989219	1.11022302462516e-16\\
0.676506550594584	1.11022302462516e-16\\
0.68150804119995	1.11022302462516e-16\\
0.686530221876942	1.11022302462516e-16\\
0.691552402553934	1.11022302462516e-16\\
0.696574583230927	1.11022302462516e-16\\
0.701596763907919	1.11022302462516e-16\\
0.706639835866396	1.11022302462516e-16\\
0.711682907824873	2.22044604925031e-16\\
0.71672597978335	1.11022302462516e-16\\
0.721769051741827	1.11022302462516e-16\\
0.72683321952555	1.11022302462516e-16\\
0.731897387309272	1.11022302462516e-16\\
0.736961555092995	1.11022302462516e-16\\
0.742025722876718	1.11022302462516e-16\\
0.747111194287107	1.11022302462516e-16\\
0.752196665697496	1.11022302462516e-16\\
0.757282137107885	1.11022302462516e-16\\
0.762367608518274	1.11022302462516e-16\\
0.767474594900923	1.11022302462516e-16\\
0.772581581283573	1.11022302462516e-16\\
0.777688567666222	2.22044604925031e-16\\
0.782795554048871	1.11022302462516e-16\\
0.78792427039055	1.11022302462516e-16\\
0.793052986732229	2.22044604925031e-16\\
0.798181703073909	1.11022302462516e-16\\
0.803310419415588	1.11022302462516e-16\\
0.808461084325784	1.11022302462516e-16\\
0.81361174923598	1.11022302462516e-16\\
0.818762414146177	2.22044604925031e-16\\
0.823913079056373	1.11022302462516e-16\\
0.82908591482452	1.11022302462516e-16\\
0.834258750592667	2.22044604925031e-16\\
0.839431586360813	2.22044604925031e-16\\
0.84460442212896	2.22044604925031e-16\\
0.849799655005291	2.22044604925031e-16\\
0.854994887881621	2.22044604925031e-16\\
0.860190120757951	3.33066907387547e-16\\
0.865385353634282	2.22044604925031e-16\\
0.870603213699137	2.22044604925031e-16\\
0.875821073763992	2.22044604925031e-16\\
0.881038933828848	3.33066907387547e-16\\
0.886256793893703	2.22044604925031e-16\\
0.891497515445601	2.22044604925031e-16\\
0.896738236997499	2.22044604925031e-16\\
0.901978958549397	2.22044604925031e-16\\
0.907219680101295	2.22044604925031e-16\\
0.912483501408722	1.11022302462516e-16\\
0.917747322716149	2.22044604925031e-16\\
0.923011144023576	2.22044604925031e-16\\
0.928274965331004	2.22044604925031e-16\\
0.933562128954513	1.11022302462516e-16\\
0.938849292578023	2.22044604925031e-16\\
0.944136456201532	1.11022302462516e-16\\
0.949423619825042	2.22044604925031e-16\\
0.954734372782572	2.22044604925031e-16\\
0.960045125740102	2.22044604925031e-16\\
0.965355878697632	2.22044604925031e-16\\
0.970666631655162	3.33066907387547e-16\\
0.976001225418392	2.22044604925031e-16\\
0.981335819181621	2.22044604925031e-16\\
0.98667041294485	2.22044604925031e-16\\
0.99200500670808	3.33066907387547e-16\\
0.997363697280812	3.33066907387547e-16\\
1.00272238785355	2.22044604925031e-16\\
1.00808107842628	3.33066907387547e-16\\
1.01343976899901	2.22044604925031e-16\\
1.0188228172289	1.11022302462516e-16\\
1.02420586545878	2.22044604925031e-16\\
1.02958891368867	2.22044604925031e-16\\
1.03497196191855	2.22044604925031e-16\\
1.04037963341625	2.22044604925031e-16\\
1.04578730491395	1.11022302462516e-16\\
1.05119497641164	1.11022302462516e-16\\
1.05660264790934	2.22044604925031e-16\\
1.06203521338424	2.22044604925031e-16\\
1.06746777885913	1.11022302462516e-16\\
1.07290034433402	1.11022302462516e-16\\
1.07833290980891	2.22044604925031e-16\\
1.08379064506731	2.22044604925031e-16\\
1.08924838032571	2.22044604925031e-16\\
1.09470611558411	2.22044604925031e-16\\
1.10016385084252	2.22044604925031e-16\\
1.1056470369148	1.11022302462516e-16\\
1.11113022298707	1.11022302462516e-16\\
1.11661340905935	2.22044604925031e-16\\
1.12209659513163	2.22044604925031e-16\\
1.12760551858231	1.11022302462516e-16\\
1.13311444203298	2.22044604925031e-16\\
1.13862336548365	2.22044604925031e-16\\
1.14413228893432	2.22044604925031e-16\\
1.14966724184116	1.11022302462516e-16\\
1.15520219474799	2.22044604925031e-16\\
1.16073714765482	2.22044604925031e-16\\
1.16627210056166	2.22044604925031e-16\\
1.17183338076584	3.33066907387547e-16\\
1.17739466097002	3.33066907387547e-16\\
1.18295594117419	2.22044604925031e-16\\
1.18851722137837	3.33066907387547e-16\\
1.19410513262804	3.33066907387547e-16\\
1.19969304387772	3.33066907387547e-16\\
1.20528095512739	2.22044604925031e-16\\
1.21086886637706	2.22044604925031e-16\\
1.21648371859449	1.11022302462516e-16\\
1.22209857081192	1.11022302462516e-16\\
1.22771342302936	2.22044604925031e-16\\
1.23332827524679	2.22044604925031e-16\\
1.23897038457538	2.22044604925031e-16\\
1.24461249390397	2.22044604925031e-16\\
1.25025460323256	3.33066907387547e-16\\
1.25589671256115	2.22044604925031e-16\\
1.26156640151281	1.11022302462516e-16\\
1.26723609046446	2.22044604925031e-16\\
1.27290577941612	2.22044604925031e-16\\
1.27857546836778	2.22044604925031e-16\\
1.28427306633634	2.22044604925031e-16\\
1.28997066430491	2.22044604925031e-16\\
1.29566826227347	2.22044604925031e-16\\
1.30136586024204	2.22044604925031e-16\\
1.30709170327692	1.11022302462516e-16\\
1.31281754631181	1.11022302462516e-16\\
1.31854338934669	1.11022302462516e-16\\
1.32426923238158	2.22044604925031e-16\\
1.33002366371043	2.22044604925031e-16\\
1.33577809503928	2.22044604925031e-16\\
1.34153252636813	2.22044604925031e-16\\
1.34728695769699	2.22044604925031e-16\\
1.35307032782915	2.22044604925031e-16\\
1.35885369796131	2.22044604925031e-16\\
1.36463706809348	3.33066907387547e-16\\
1.37042043822564	2.22044604925031e-16\\
1.37622881553645	2.22044604925031e-16\\
1.38203719284726	3.33066907387547e-16\\
1.38784557015807	2.22044604925031e-16\\
1.39365394746888	2.22044604925031e-16\\
1.39948194569202	2.22044604925031e-16\\
1.40530994391516	2.22044604925031e-16\\
1.41113794213829	2.22044604925031e-16\\
1.41696594036143	2.22044604925031e-16\\
1.4228137248248	2.22044604925031e-16\\
1.42866150928816	1.11022302462516e-16\\
1.43450929375153	1.11022302462516e-16\\
1.44035707821489	2.22044604925031e-16\\
1.44622478964653	1.11022302462516e-16\\
1.45209250107816	1.11022302462516e-16\\
1.45796021250979	1.11022302462516e-16\\
1.46382792394143	2.22044604925031e-16\\
1.46971570462452	1.11022302462516e-16\\
1.4756034853076	2.22044604925031e-16\\
1.48149126599069	2.22044604925031e-16\\
1.48737904667378	2.22044604925031e-16\\
1.49328704060906	1.11022302462516e-16\\
1.49919503454433	2.22044604925031e-16\\
1.50510302847961	1.11022302462516e-16\\
1.51101102241489	2.22044604925031e-16\\
1.51693937480906	1.11022302462516e-16\\
1.52286772720322	1.11022302462516e-16\\
1.52879607959739	1.11022302462516e-16\\
1.53472443199155	1.11022302462516e-16\\
1.540673290528	1.11022302462516e-16\\
1.54662214906446	1.11022302462516e-16\\
1.55257100760091	1.11022302462516e-16\\
1.55851986613736	1.11022302462516e-16\\
1.56448937947106	1.11022302462516e-16\\
1.57045889280476	2.22044604925031e-16\\
1.57642840613847	2.22044604925031e-16\\
1.58239791947216	1.11022302462516e-16\\
1.58838823791446	0\\
1.59437855635675	0\\
1.60036887479904	1.11022302462516e-16\\
1.60635919324133	0\\
1.61237046959671	0\\
1.61838174595208	0\\
1.62439302230746	0\\
1.63040429866284	1.11022302462516e-16\\
1.63643668672632	1.11022302462516e-16\\
1.6424690747898	1.11022302462516e-16\\
1.64850146285328	2.22044604925031e-16\\
1.65453385091676	2.22044604925031e-16\\
1.66058750665404	1.11022302462516e-16\\
1.66664116239132	1.11022302462516e-16\\
1.67269481812859	2.22044604925031e-16\\
1.67874847386587	1.11022302462516e-16\\
1.68482355500326	1.11022302462516e-16\\
1.69089863614064	2.22044604925031e-16\\
1.69697371727803	2.22044604925031e-16\\
1.70304879841541	1.11022302462516e-16\\
1.70914546464501	1.11022302462516e-16\\
1.71524213087462	1.11022302462516e-16\\
1.72133879710422	1.11022302462516e-16\\
1.72743546333383	1.11022302462516e-16\\
1.73355387572031	1.11022302462516e-16\\
1.73967228810678	1.11022302462516e-16\\
1.74579070049326	2.22044604925031e-16\\
1.75190911287974	1.11022302462516e-16\\
1.75804943463327	1.11022302462516e-16\\
1.76418975638681	1.11022302462516e-16\\
1.77033007814035	1.11022302462516e-16\\
1.77647039989388	2.22044604925031e-16\\
1.78263279664118	1.11022302462516e-16\\
1.78879519338848	2.22044604925031e-16\\
1.79495759013578	1.11022302462516e-16\\
1.80111998688309	2.22044604925031e-16\\
1.80730462566597	2.22044604925031e-16\\
1.81348926444885	1.11022302462516e-16\\
1.81967390323173	2.22044604925031e-16\\
1.82585854201462	2.22044604925031e-16\\
1.83206559217232	2.22044604925031e-16\\
1.83827264233002	2.22044604925031e-16\\
1.84447969248772	2.22044604925031e-16\\
1.85068674264543	2.22044604925031e-16\\
1.85691637550678	1.11022302462516e-16\\
1.86314600836813	2.22044604925031e-16\\
1.86937564122948	2.22044604925031e-16\\
1.87560527409083	1.11022302462516e-16\\
1.88185766319929	1.11022302462516e-16\\
1.88811005230775	1.11022302462516e-16\\
1.89436244141622	2.22044604925031e-16\\
1.90061483052468	1.11022302462516e-16\\
1.90689015093446	1.11022302462516e-16\\
1.91316547134424	1.11022302462516e-16\\
1.91944079175402	1.11022302462516e-16\\
1.92571611216381	1.11022302462516e-16\\
1.93201454150218	1.11022302462516e-16\\
1.93831297084055	2.22044604925031e-16\\
1.94461140017892	2.22044604925031e-16\\
1.9509098295173	2.22044604925031e-16\\
1.95723154808653	2.22044604925031e-16\\
1.96355326665576	2.22044604925031e-16\\
1.96987498522499	2.22044604925031e-16\\
1.97619670379422	3.33066907387547e-16\\
1.98254189319299	2.22044604925031e-16\\
1.98888708259176	2.22044604925031e-16\\
1.99523227199053	3.33066907387547e-16\\
2.0015774613893	2.22044604925031e-16\\
2.00794630624256	2.22044604925031e-16\\
2.01431515109582	2.22044604925031e-16\\
2.02068399594908	3.33066907387547e-16\\
2.02705284080234	2.22044604925031e-16\\
2.03344552746868	2.22044604925031e-16\\
2.03983821413503	2.22044604925031e-16\\
2.04623090080137	2.22044604925031e-16\\
2.05262358746771	2.22044604925031e-16\\
2.05904030495676	1.11022302462516e-16\\
2.06545702244581	2.22044604925031e-16\\
2.07187373993486	1.11022302462516e-16\\
2.07829045742391	2.22044604925031e-16\\
2.084731396657	2.22044604925031e-16\\
2.0911723358901	2.22044604925031e-16\\
2.0976132751232	3.33066907387547e-16\\
2.1040542143563	3.33066907387547e-16\\
2.11051956957009	3.33066907387547e-16\\
2.11698492478388	2.22044604925031e-16\\
2.12345027999767	2.22044604925031e-16\\
2.12991563521146	3.33066907387547e-16\\
2.13640560238783	3.33066907387547e-16\\
2.1428955695642	2.22044604925031e-16\\
2.14938553674057	3.33066907387547e-16\\
2.15587550391694	3.33066907387547e-16\\
2.16239028199053	3.33066907387547e-16\\
2.16890506006412	3.33066907387547e-16\\
2.17541983813771	4.44089209850063e-16\\
2.1819346162113	4.44089209850063e-16\\
2.18847440630977	3.33066907387547e-16\\
2.19501419640825	4.44089209850063e-16\\
2.20155398650673	4.44089209850063e-16\\
2.20809377660521	4.44089209850063e-16\\
2.21465878275268	3.33066907387547e-16\\
2.22122378890015	3.33066907387547e-16\\
2.22778879504762	4.44089209850063e-16\\
2.23435380119508	4.44089209850063e-16\\
2.24094422983074	4.44089209850063e-16\\
2.24753465846639	4.44089209850063e-16\\
2.25412508710205	5.55111512312578e-16\\
2.2607155157377	4.44089209850063e-16\\
2.2673315759421	4.44089209850063e-16\\
2.27394763614649	4.44089209850063e-16\\
2.28056369635089	5.55111512312578e-16\\
2.28717975655528	4.44089209850063e-16\\
2.29382166035897	4.44089209850063e-16\\
2.30046356416266	4.44089209850063e-16\\
2.30710546796636	5.55111512312578e-16\\
2.31374737177005	5.55111512312578e-16\\
2.32041533418655	4.44089209850063e-16\\
2.32708329660306	5.55111512312578e-16\\
2.33375125901957	5.55111512312578e-16\\
2.34041922143607	5.55111512312578e-16\\
2.34711345989709	4.44089209850063e-16\\
2.3538076983581	5.55111512312578e-16\\
2.36050193681912	5.55111512312578e-16\\
2.36719617528013	5.55111512312578e-16\\
2.37391691043736	4.44089209850063e-16\\
2.3806376455946	5.55111512312578e-16\\
2.38735838075183	5.55111512312578e-16\\
2.39407911590907	5.55111512312578e-16\\
2.40082657131283	5.55111512312578e-16\\
2.40757402671659	5.55111512312578e-16\\
2.41432148212035	6.66133814775094e-16\\
2.42106893752412	6.66133814775094e-16\\
2.42784333963791	5.55111512312578e-16\\
2.4346177417517	6.66133814775094e-16\\
2.4413921438655	5.55111512312578e-16\\
2.44816654597929	6.66133814775094e-16\\
2.4549681243632	6.66133814775094e-16\\
2.4617697027471	5.55111512312578e-16\\
2.46857128113101	6.66133814775094e-16\\
2.47537285951492	6.66133814775094e-16\\
2.48220184698832	5.55111512312578e-16\\
2.48903083446173	5.55111512312578e-16\\
2.49585982193513	6.66133814775094e-16\\
2.50268880940853	5.55111512312578e-16\\
2.50954544177924	4.44089209850063e-16\\
2.51640207414995	5.55111512312578e-16\\
2.52325870652066	5.55111512312578e-16\\
2.53011533889137	5.55111512312578e-16\\
2.53699985562036	4.44089209850063e-16\\
2.54388437234935	4.44089209850063e-16\\
2.55076888907833	4.44089209850063e-16\\
2.55765340580732	4.44089209850063e-16\\
2.5645660494942	4.44089209850063e-16\\
2.57147869318108	4.44089209850063e-16\\
2.57839133686796	5.55111512312578e-16\\
2.58530398055484	5.55111512312578e-16\\
2.59224499685821	4.44089209850063e-16\\
2.59918601316159	4.44089209850063e-16\\
2.60612702946496	5.55111512312578e-16\\
2.61306804576833	4.44089209850063e-16\\
2.62003768421203	3.33066907387547e-16\\
2.62700732265572	3.33066907387547e-16\\
2.63397696109942	4.44089209850063e-16\\
2.64094659954311	4.44089209850063e-16\\
2.64794511298781	4.44089209850063e-16\\
2.65494362643251	4.44089209850063e-16\\
2.6619421398772	5.55111512312578e-16\\
2.6689406533219	4.44089209850063e-16\\
2.67596829852112	4.44089209850063e-16\\
2.68299594372033	4.44089209850063e-16\\
2.69002358891955	5.55111512312578e-16\\
2.69705123411877	5.55111512312578e-16\\
2.70410827073433	4.44089209850063e-16\\
2.71116530734989	4.44089209850063e-16\\
2.71822234396545	5.55111512312578e-16\\
2.72527938058101	4.44089209850063e-16\\
2.73236607296869	4.44089209850063e-16\\
2.73945276535637	4.44089209850063e-16\\
2.74653945774405	5.55111512312578e-16\\
2.75362615013173	5.55111512312578e-16\\
2.76074276611797	4.44089209850063e-16\\
2.76785938210422	5.55111512312578e-16\\
2.77497599809046	6.66133814775094e-16\\
2.78209261407671	5.55111512312578e-16\\
2.78923942490227	5.55111512312578e-16\\
2.79638623572783	5.55111512312578e-16\\
2.80353304655339	6.66133814775094e-16\\
2.81067985737896	6.66133814775094e-16\\
2.81785713902212	5.55111512312578e-16\\
2.82503442066529	5.55111512312578e-16\\
2.83221170230846	5.55111512312578e-16\\
2.83938898395163	6.66133814775094e-16\\
2.84659701614953	4.44089209850063e-16\\
2.85380504834743	5.55111512312578e-16\\
2.86101308054533	5.55111512312578e-16\\
2.86822111274323	6.66133814775094e-16\\
2.87546017926332	5.55111512312578e-16\\
2.88269924578341	5.55111512312578e-16\\
2.88993831230351	6.66133814775094e-16\\
2.8971773788236	5.55111512312578e-16\\
2.90444776780364	5.55111512312578e-16\\
2.91171815678368	6.66133814775094e-16\\
2.91898854576373	6.66133814775094e-16\\
2.92625893474377	6.66133814775094e-16\\
2.93356093845992	6.66133814775094e-16\\
2.94086294217606	5.55111512312578e-16\\
2.94816494589221	6.66133814775094e-16\\
2.95546694960836	6.66133814775094e-16\\
2.96280086487659	5.55111512312578e-16\\
2.97013478014483	5.55111512312578e-16\\
2.97746869541306	6.66133814775094e-16\\
2.9848026106813	6.66133814775094e-16\\
2.99216873909201	4.44089209850063e-16\\
2.99953486750271	5.55111512312578e-16\\
3.00690099591342	5.55111512312578e-16\\
3.01426712432413	6.66133814775094e-16\\
3.02166577186631	5.55111512312578e-16\\
3.02906441940849	5.55111512312578e-16\\
3.03646306695067	6.66133814775094e-16\\
3.04386171449286	5.55111512312578e-16\\
3.05129319192691	5.55111512312578e-16\\
3.05872466936097	4.44089209850063e-16\\
3.06615614679503	5.55111512312578e-16\\
3.07358762422909	5.55111512312578e-16\\
3.08105224696524	4.44089209850063e-16\\
3.0885168697014	5.55111512312578e-16\\
3.09598149243755	5.55111512312578e-16\\
3.1034461151737	5.55111512312578e-16\\
3.11094420406679	5.55111512312578e-16\\
3.11844229295988	5.55111512312578e-16\\
3.12594038185297	5.55111512312578e-16\\
3.13343847074605	5.55111512312578e-16\\
3.14097035080817	5.55111512312578e-16\\
3.14850223087028	5.55111512312578e-16\\
3.1560341109324	6.66133814775094e-16\\
3.16356599099451	5.55111512312578e-16\\
3.17113199342146	4.44089209850063e-16\\
3.17869799584842	5.55111512312578e-16\\
3.18626399827537	5.55111512312578e-16\\
3.19383000070233	6.66133814775094e-16\\
3.20143046110215	4.44089209850063e-16\\
3.20903092150198	4.44089209850063e-16\\
3.21663138190181	6.66133814775094e-16\\
3.22423184230164	5.55111512312578e-16\\
3.2318671021242	4.44089209850063e-16\\
3.23950236194676	5.55111512312578e-16\\
3.24713762176933	4.44089209850063e-16\\
3.25477288159189	5.55111512312578e-16\\
3.26244328804483	4.44089209850063e-16\\
3.27011369449776	5.55111512312578e-16\\
3.2777841009507	6.66133814775094e-16\\
3.28545450740363	5.55111512312578e-16\\
3.29316041266953	5.55111512312578e-16\\
3.30086631793543	5.55111512312578e-16\\
3.30857222320133	5.55111512312578e-16\\
3.31627812846722	5.55111512312578e-16\\
3.32401989110489	4.44089209850063e-16\\
3.33176165374256	4.44089209850063e-16\\
3.33950341638023	5.55111512312578e-16\\
3.3472451790179	5.55111512312578e-16\\
3.35502316313618	4.44089209850063e-16\\
3.36280114725446	5.55111512312578e-16\\
3.37057913137275	5.55111512312578e-16\\
3.37835711549103	5.55111512312578e-16\\
3.38617169154667	3.33066907387547e-16\\
3.3939862676023	4.44089209850063e-16\\
3.40180084365794	5.55111512312578e-16\\
3.40961541971357	5.55111512312578e-16\\
3.41746696420648	5.55111512312578e-16\\
3.42531850869938	5.55111512312578e-16\\
3.43317005319229	5.55111512312578e-16\\
3.4410215976852	6.66133814775094e-16\\
3.44891049346576	4.44089209850063e-16\\
3.45679938924632	6.66133814775094e-16\\
3.46468828502688	6.66133814775094e-16\\
3.47257718080744	6.66133814775094e-16\\
3.48050381756295	5.55111512312578e-16\\
3.48843045431847	5.55111512312578e-16\\
3.49635709107398	6.66133814775094e-16\\
3.50428372782949	6.66133814775094e-16\\
3.51224850159594	5.55111512312578e-16\\
3.52021327536239	6.66133814775094e-16\\
3.52817804912884	6.66133814775094e-16\\
3.53614282289529	6.66133814775094e-16\\
3.54414613607264	5.55111512312578e-16\\
3.55214944924998	5.55111512312578e-16\\
3.56015276242733	6.66133814775094e-16\\
3.56815607560468	7.7715611723761e-16\\
3.57619833889781	5.55111512312578e-16\\
3.58424060219095	6.66133814775094e-16\\
3.59228286548408	6.66133814775094e-16\\
3.60032512877721	6.66133814775094e-16\\
3.60840675922679	5.55111512312578e-16\\
3.61648838967638	5.55111512312578e-16\\
3.62457002012596	6.66133814775094e-16\\
3.63265165057554	6.66133814775094e-16\\
3.64077307248204	4.44089209850063e-16\\
3.64889449438855	5.55111512312578e-16\\
3.65701591629505	6.66133814775094e-16\\
3.66513733820155	5.55111512312578e-16\\
3.67329898353354	4.44089209850063e-16\\
3.68146062886554	5.55111512312578e-16\\
3.68962227419754	6.66133814775094e-16\\
3.69778391952953	5.55111512312578e-16\\
3.70598622860803	5.55111512312578e-16\\
3.71418853768653	5.55111512312578e-16\\
3.72239084676502	6.66133814775094e-16\\
3.73059315584352	6.66133814775094e-16\\
3.73883657626523	4.44089209850063e-16\\
3.74707999668693	5.55111512312578e-16\\
3.75532341710863	7.7715611723761e-16\\
3.76356683753034	6.66133814775094e-16\\
3.77185182516863	4.44089209850063e-16\\
3.78013681280692	5.55111512312578e-16\\
3.78842180044522	5.55111512312578e-16\\
3.79670678808351	5.55111512312578e-16\\
3.80503380745825	4.44089209850063e-16\\
3.81336082683299	5.55111512312578e-16\\
3.82168784620774	6.66133814775094e-16\\
3.83001486558248	5.55111512312578e-16\\
3.83838438958511	3.33066907387547e-16\\
3.84675391358775	4.44089209850063e-16\\
3.85512343759039	5.55111512312578e-16\\
3.86349296159302	5.55111512312578e-16\\
3.8719054722877	4.44089209850063e-16\\
3.88031798298237	5.55111512312578e-16\\
3.88873049367705	6.66133814775094e-16\\
3.89714300437172	5.55111512312578e-16\\
3.90559899284027	4.44089209850063e-16\\
3.91405498130882	4.44089209850063e-16\\
3.92251096977737	6.66133814775094e-16\\
3.93096695824593	5.55111512312578e-16\\
3.93946692479213	3.33066907387547e-16\\
3.94796689133833	4.44089209850063e-16\\
3.95646685788454	5.55111512312578e-16\\
3.96496682443074	5.55111512312578e-16\\
3.97351127881372	4.44089209850063e-16\\
3.98205573319669	4.44089209850063e-16\\
3.99060018757967	5.55111512312578e-16\\
3.99914464196265	5.55111512312578e-16\\
4.00773410425464	3.33066907387547e-16\\
4.01632356654662	4.44089209850063e-16\\
4.02491302883861	5.55111512312578e-16\\
4.0335024911306	4.44089209850063e-16\\
4.04213749081375	3.33066907387547e-16\\
4.05077249049689	4.44089209850063e-16\\
4.05940749018004	4.44089209850063e-16\\
4.06804248986318	5.55111512312578e-16\\
4.07672356764034	3.33066907387547e-16\\
4.08540464541749	4.44089209850063e-16\\
4.09408572319465	5.55111512312578e-16\\
4.10276680097181	5.55111512312578e-16\\
4.11149450824228	3.33066907387547e-16\\
4.12022221551276	4.44089209850063e-16\\
4.12894992278323	5.55111512312578e-16\\
4.1376776300537	5.55111512312578e-16\\
4.14645252883691	3.33066907387547e-16\\
4.15522742762012	3.33066907387547e-16\\
4.16400232640333	4.44089209850063e-16\\
4.17277722518654	5.55111512312578e-16\\
4.18159988926615	3.33066907387547e-16\\
4.19042255334575	4.44089209850063e-16\\
4.19924521742536	5.55111512312578e-16\\
4.20806788150496	5.55111512312578e-16\\
4.21693889629988	4.44089209850063e-16\\
4.22580991109479	4.44089209850063e-16\\
4.23468092588971	6.66133814775094e-16\\
4.24355194068463	6.66133814775094e-16\\
4.2524719033702	3.33066907387547e-16\\
4.26139186605577	4.44089209850063e-16\\
4.27031182874135	6.66133814775094e-16\\
4.27923179142692	5.55111512312578e-16\\
4.28820131218307	3.33066907387547e-16\\
4.29717083293922	5.55111512312578e-16\\
4.30614035369538	6.66133814775094e-16\\
4.31510987445153	6.66133814775094e-16\\
4.32412957614907	4.44089209850063e-16\\
4.33314927784661	4.44089209850063e-16\\
4.34216897954416	6.66133814775094e-16\\
4.3511886812417	6.66133814775094e-16\\
4.36025919971018	4.44089209850063e-16\\
4.36932971817866	5.55111512312578e-16\\
4.37840023664715	6.66133814775094e-16\\
4.38747075511563	6.66133814775094e-16\\
4.39659274002011	4.44089209850063e-16\\
4.40571472492458	5.55111512312578e-16\\
4.41483670982906	6.66133814775094e-16\\
4.42395869473353	6.66133814775094e-16\\
4.43313280988389	5.55111512312578e-16\\
4.44230692503424	5.55111512312578e-16\\
4.45148104018459	7.7715611723761e-16\\
4.46065515533495	7.7715611723761e-16\\
4.469882079136	5.55111512312578e-16\\
4.47910900293706	6.66133814775094e-16\\
4.48833592673811	8.88178419700125e-16\\
4.49756285053916	7.7715611723761e-16\\
4.50684327607396	4.44089209850063e-16\\
4.51612370160877	5.55111512312578e-16\\
4.52540412714357	7.7715611723761e-16\\
4.53468455267837	6.66133814775094e-16\\
4.54401918924822	4.44089209850063e-16\\
4.55335382581807	5.55111512312578e-16\\
4.56268846238792	6.66133814775094e-16\\
4.57202309895777	6.66133814775094e-16\\
4.58141267108196	4.44089209850063e-16\\
4.59080224320616	4.44089209850063e-16\\
4.60019181533036	7.7715611723761e-16\\
4.60958138745456	6.66133814775094e-16\\
4.61902663678989	3.33066907387547e-16\\
4.62847188612523	4.44089209850063e-16\\
4.63791713546056	6.66133814775094e-16\\
4.6473623847959	6.66133814775094e-16\\
4.65686406984621	3.33066907387547e-16\\
4.66636575489653	4.44089209850063e-16\\
4.67586743994685	5.55111512312578e-16\\
4.68536912499717	5.55111512312578e-16\\
4.69492802222378	2.22044604925031e-16\\
4.7044869194504	4.44089209850063e-16\\
4.71404581667701	6.66133814775094e-16\\
4.72360471390363	5.55111512312578e-16\\
4.73320764525141	2.22044604925031e-16\\
4.74281057659919	3.33066907387547e-16\\
4.75241350794698	5.55111512312578e-16\\
4.76201643929476	5.55111512312578e-16\\
4.77165794452804	2.22044604925031e-16\\
4.78129944976131	3.33066907387547e-16\\
4.79094095499459	5.55111512312578e-16\\
4.80058246022786	4.44089209850063e-16\\
4.81026314694842	2.22044604925031e-16\\
4.81994383366897	3.33066907387547e-16\\
4.82962452038952	5.55111512312578e-16\\
4.83930520711008	4.44089209850063e-16\\
4.84902565638322	1.11022302462516e-16\\
4.85874610565637	3.33066907387547e-16\\
4.86846655492951	5.55111512312578e-16\\
4.87818700420266	5.55111512312578e-16\\
4.88794780523678	2.22044604925031e-16\\
4.89770860627089	4.44089209850063e-16\\
4.90746940730501	6.66133814775094e-16\\
4.91723020833913	5.55111512312578e-16\\
4.92703195872998	3.33066907387547e-16\\
4.93683370912084	4.44089209850063e-16\\
4.9466354595117	6.66133814775094e-16\\
4.95643720990256	6.66133814775094e-16\\
4.96628051524823	3.33066907387547e-16\\
4.97612382059389	4.44089209850063e-16\\
4.98596712593956	6.66133814775094e-16\\
4.99581043128522	5.55111512312578e-16\\
5.00569590732134	2.22044604925031e-16\\
5.01558138335745	4.44089209850063e-16\\
5.02546685939356	6.66133814775094e-16\\
5.03535233542967	6.66133814775094e-16\\
5.04528060635073	3.33066907387547e-16\\
5.05520887727179	4.44089209850063e-16\\
5.06513714819285	6.66133814775094e-16\\
5.07506541911391	5.55111512312578e-16\\
5.08503711889176	2.22044604925031e-16\\
5.0950088186696	4.44089209850063e-16\\
5.10498051844744	5.55111512312578e-16\\
5.11495221822528	5.55111512312578e-16\\
5.12496799094167	2.22044604925031e-16\\
5.13498376365806	3.33066907387547e-16\\
5.14499953637444	6.66133814775094e-16\\
5.15501530909083	5.55111512312578e-16\\
5.16507580953502	2.22044604925031e-16\\
5.17513630997921	3.33066907387547e-16\\
5.18519681042339	7.7715611723761e-16\\
5.19525731086758	5.55111512312578e-16\\
5.2053632039849	2.22044604925031e-16\\
5.21546909710223	3.33066907387547e-16\\
5.22557499021955	6.66133814775094e-16\\
5.23568088333688	5.55111512312578e-16\\
5.24583284576096	2.22044604925031e-16\\
5.25598480818505	3.33066907387547e-16\\
5.26613677060914	6.66133814775094e-16\\
5.27628873303322	5.55111512312578e-16\\
5.28648745229033	2.22044604925031e-16\\
5.29668617154743	3.33066907387547e-16\\
5.30688489080453	6.66133814775094e-16\\
5.31708361006163	5.55111512312578e-16\\
5.32732978639303	1.11022302462516e-16\\
5.33757596272443	3.33066907387547e-16\\
5.34782213905582	6.66133814775094e-16\\
5.35806831538722	5.55111512312578e-16\\
5.36836266133408	2.22044604925031e-16\\
5.37865700728095	4.44089209850063e-16\\
5.38895135322782	6.66133814775094e-16\\
5.39924569917468	5.55111512312578e-16\\
5.40958894010065	2.22044604925031e-16\\
5.41993218102661	3.33066907387547e-16\\
5.43027542195257	6.66133814775094e-16\\
5.44061866287853	6.66133814775094e-16\\
5.45101153706255	3.33066907387547e-16\\
5.46140441124658	4.44089209850063e-16\\
5.4717972854306	6.66133814775094e-16\\
5.48219015961463	6.66133814775094e-16\\
5.49263341945276	2.22044604925031e-16\\
5.50307667929089	4.44089209850063e-16\\
5.51351993912902	7.7715611723761e-16\\
5.52396319896715	6.66133814775094e-16\\
5.53445761187213	2.22044604925031e-16\\
5.54495202477712	4.44089209850063e-16\\
5.5554464376821	7.7715611723761e-16\\
5.56594085058709	6.66133814775094e-16\\
5.57648719831941	2.22044604925031e-16\\
5.58703354605173	4.44089209850063e-16\\
5.59757989378405	7.7715611723761e-16\\
5.60812624151637	7.7715611723761e-16\\
5.61872532103302	2.22044604925031e-16\\
5.62932440054967	4.44089209850063e-16\\
5.63992348006633	7.7715611723761e-16\\
5.65052255958298	6.66133814775094e-16\\
5.66117518401188	2.22044604925031e-16\\
5.67182780844078	4.44089209850063e-16\\
5.68248043286967	8.88178419700125e-16\\
5.69313305729857	6.66133814775094e-16\\
5.70384005671797	3.33066907387547e-16\\
5.71454705613736	4.44089209850063e-16\\
5.72525405555675	7.7715611723761e-16\\
5.73596105497615	6.66133814775094e-16\\
5.74672327665992	2.22044604925031e-16\\
5.75748549834369	4.44089209850063e-16\\
5.76824772002746	7.7715611723761e-16\\
5.77900994171123	6.66133814775094e-16\\
5.78982825023888	2.22044604925031e-16\\
5.80064655876654	4.44089209850063e-16\\
5.81146486729419	7.7715611723761e-16\\
5.82228317582184	6.66133814775094e-16\\
5.83315845473553	2.22044604925031e-16\\
5.84403373364921	5.55111512312578e-16\\
5.8549090125629	7.7715611723761e-16\\
5.86578429147658	7.7715611723761e-16\\
5.87671744355088	2.22044604925031e-16\\
5.88765059562518	5.55111512312578e-16\\
5.89858374769948	8.88178419700125e-16\\
5.90951689977378	7.7715611723761e-16\\
5.92050884813521	2.22044604925031e-16\\
5.93150079649664	4.44089209850063e-16\\
5.94249274485807	8.88178419700125e-16\\
5.9534846932195	7.7715611723761e-16\\
5.96453638108293	3.33066907387547e-16\\
5.97558806894636	5.55111512312578e-16\\
5.98663975680979	8.88178419700125e-16\\
5.99769144467322	8.88178419700125e-16\\
6.00880383710542	3.33066907387547e-16\\
6.01991622953761	5.55111512312578e-16\\
6.0310286219698	8.88178419700125e-16\\
6.042141014402	8.88178419700125e-16\\
6.05331509935885	3.33066907387547e-16\\
6.0644891843157	5.55111512312578e-16\\
6.07566326927256	9.99200722162641e-16\\
6.08683735422941	8.88178419700125e-16\\
6.09807414230094	3.33066907387547e-16\\
6.10931093037248	5.55111512312578e-16\\
6.12054771844401	9.99200722162641e-16\\
6.13178450651555	8.88178419700125e-16\\
6.14308503319571	3.33066907387547e-16\\
6.15438555987588	5.55111512312578e-16\\
6.16568608655605	9.99200722162641e-16\\
6.17698661323621	8.88178419700125e-16\\
6.18835193888863	2.22044604925031e-16\\
6.19971726454106	5.55111512312578e-16\\
6.21108259019348	9.99200722162641e-16\\
6.2224479158459	8.88178419700125e-16\\
6.23387912710053	3.33066907387547e-16\\
6.24531033835517	5.55111512312578e-16\\
6.2567415496098	9.99200722162641e-16\\
6.26817276086443	8.88178419700125e-16\\
6.2796709718506	3.33066907387547e-16\\
6.29116918283678	5.55111512312578e-16\\
6.30266739382295	1.11022302462516e-15\\
6.31416560480912	9.99200722162641e-16\\
6.32573195844141	4.44089209850063e-16\\
6.33729831207371	6.66133814775094e-16\\
6.34886466570601	1.11022302462516e-15\\
6.36043101933831	9.99200722162641e-16\\
6.37206668825559	4.44089209850063e-16\\
6.38370235717287	6.66133814775094e-16\\
6.39533802609016	1.22124532708767e-15\\
6.40697369500744	1.11022302462516e-15\\
6.41867988207414	4.44089209850063e-16\\
6.43038606914084	6.66133814775094e-16\\
6.44209225620754	1.22124532708767e-15\\
6.45379844327423	1.11022302462516e-15\\
6.46557638449348	3.33066907387547e-16\\
6.47735432571273	6.66133814775094e-16\\
6.48913226693197	1.11022302462516e-15\\
6.50091020815122	9.99200722162641e-16\\
6.51276117329228	3.33066907387547e-16\\
6.52461213843333	6.66133814775094e-16\\
6.53646310357439	1.22124532708767e-15\\
6.54831406871544	1.11022302462516e-15\\
6.56023936237899	3.33066907387547e-16\\
6.57216465604253	5.55111512312578e-16\\
6.58408994970607	1.11022302462516e-15\\
6.59601524336962	1.11022302462516e-15\\
6.608016207688	3.33066907387547e-16\\
6.62001717200638	6.66133814775094e-16\\
6.63201813632476	1.11022302462516e-15\\
6.64401910064313	1.11022302462516e-15\\
6.65609711612489	3.33066907387547e-16\\
6.66817513160664	7.7715611723761e-16\\
6.68025314708839	1.22124532708767e-15\\
6.69233116257014	1.11022302462516e-15\\
6.70448764976455	3.33066907387547e-16\\
6.71664413695897	6.66133814775094e-16\\
6.72880062415338	1.33226762955019e-15\\
6.7409571113478	1.11022302462516e-15\\
6.75319353351478	3.33066907387547e-16\\
6.76542995568177	6.66133814775094e-16\\
6.77766637784875	1.22124532708767e-15\\
6.78990280001574	1.11022302462516e-15\\
6.80222066442117	3.33066907387547e-16\\
6.8145385288266	6.66133814775094e-16\\
6.82685639323203	1.33226762955019e-15\\
6.83917425763746	1.22124532708767e-15\\
6.85157511856051	3.33066907387547e-16\\
6.86397597948357	6.66133814775094e-16\\
6.87637684040662	1.33226762955019e-15\\
6.88877770132968	1.22124532708767e-15\\
6.90126316148383	4.44089209850063e-16\\
6.91374862163799	6.66133814775094e-16\\
6.92623408179214	1.4432899320127e-15\\
6.93871954194629	1.22124532708767e-15\\
6.95129125486549	3.33066907387547e-16\\
6.96386296778468	7.21644966006352e-16\\
6.97643468070388	1.38777878078145e-15\\
6.98900639362307	1.22124532708767e-15\\
7.00166606764739	3.88578058618805e-16\\
7.0143257416717	7.21644966006352e-16\\
7.02698541569601	1.4432899320127e-15\\
7.03964508972033	1.27675647831893e-15\\
7.05239448925109	3.88578058618805e-16\\
7.06514388878186	8.32667268468867e-16\\
7.07789328831262	1.49880108324396e-15\\
7.09064268784339	1.33226762955019e-15\\
7.10348363709426	3.88578058618805e-16\\
7.11632458634514	7.7715611723761e-16\\
7.12916553559601	1.49880108324396e-15\\
7.14200648484689	1.33226762955019e-15\\
7.15494087138423	3.88578058618805e-16\\
7.16787525792157	7.7715611723761e-16\\
7.18080964445891	1.55431223447522e-15\\
7.19374403099625	1.33226762955019e-15\\
7.20677380774895	3.88578058618805e-16\\
7.21980358450165	7.7715611723761e-16\\
7.23283336125435	1.55431223447522e-15\\
7.24586313800706	1.33226762955019e-15\\
7.25899032889946	2.77555756156289e-16\\
7.27211751979186	7.7715611723761e-16\\
7.28524471068426	1.55431223447522e-15\\
7.29837190157667	1.33226762955019e-15\\
7.31159860422458	2.22044604925031e-16\\
7.32482530687249	7.7715611723761e-16\\
7.3380520095204	1.55431223447522e-15\\
7.35127871216831	1.38777878078145e-15\\
7.36460710304072	1.66533453693773e-16\\
7.37793549391314	7.21644966006352e-16\\
7.39126388478556	1.55431223447522e-15\\
7.40459227565797	1.33226762955019e-15\\
7.41802461355885	1.66533453693773e-16\\
7.43145695145972	7.21644966006352e-16\\
7.4448892893606	1.55431223447522e-15\\
7.45832162726148	1.33226762955019e-15\\
7.47186025959003	1.11022302462516e-16\\
7.48539889191859	6.66133814775094e-16\\
7.49893752424715	1.60982338570648e-15\\
7.51247615657571	1.33226762955019e-15\\
7.52612352432791	1.11022302462516e-16\\
7.53977089208011	6.66133814775094e-16\\
7.55341825983232	1.60982338570648e-15\\
7.56706562758452	1.38777878078145e-15\\
7.58082427117841	1.11022302462516e-16\\
7.5945829147723	6.66133814775094e-16\\
7.60834155836619	1.60982338570648e-15\\
7.62210020196009	1.38777878078145e-15\\
7.63597276754062	5.55111512312578e-17\\
7.64984533312116	6.66133814775094e-16\\
7.66371789870169	1.66533453693773e-15\\
7.67759046428223	1.38777878078145e-15\\
7.69157971106386	5.55111512312578e-17\\
7.70556895784549	6.10622663543836e-16\\
7.71955820462712	1.60982338570648e-15\\
7.73354745140875	1.33226762955019e-15\\
7.7476562591015	1.11022302462516e-16\\
7.76176506679424	5.55111512312578e-16\\
7.77587387448699	1.60982338570648e-15\\
7.78998268217974	1.38777878078145e-15\\
7.80421406003167	1.11022302462516e-16\\
7.81844543788359	5.55111512312578e-16\\
7.83267681573552	1.66533453693773e-15\\
7.84690819358745	1.38777878078145e-15\\
7.86126528820976	1.66533453693773e-16\\
7.87562238283208	5.55111512312578e-16\\
7.8899794774544	1.66533453693773e-15\\
7.90433657207672	1.38777878078145e-15\\
7.91882267896367	2.22044604925031e-16\\
7.93330878585062	4.9960036108132e-16\\
7.94779489273757	1.72084568816899e-15\\
7.96228099962452	1.38777878078145e-15\\
7.97689957365268	2.77555756156289e-16\\
7.99151814768085	4.9960036108132e-16\\
8.00613672170902	1.72084568816899e-15\\
8.02075529573718	1.4432899320127e-15\\
8.03550996276884	2.77555756156289e-16\\
8.0502646298005	4.44089209850063e-16\\
8.06501929683216	1.77635683940025e-15\\
8.07977396386382	1.4432899320127e-15\\
8.09466853533245	3.88578058618805e-16\\
8.10956310680107	4.44089209850063e-16\\
8.12445767826969	1.83186799063151e-15\\
8.13935224973832	1.4432899320127e-15\\
8.15439073744049	4.44089209850063e-16\\
8.16942922514267	4.44089209850063e-16\\
8.18446771284484	1.88737914186277e-15\\
8.19950620054702	1.49880108324396e-15\\
8.2146928332504	4.44089209850063e-16\\
8.22987946595378	4.44089209850063e-16\\
8.24506609865717	1.94289029309402e-15\\
8.26025273136055	1.55431223447522e-15\\
8.27559197362809	4.9960036108132e-16\\
8.29093121589562	3.88578058618805e-16\\
8.30627045816316	1.99840144432528e-15\\
8.32160970043069	1.55431223447522e-15\\
8.33710627308224	5.55111512312578e-16\\
8.35260284573378	3.88578058618805e-16\\
8.36809941838533	1.99840144432528e-15\\
8.38359599103687	1.60982338570648e-15\\
8.39925489573158	6.66133814775094e-16\\
8.41491380042629	3.88578058618805e-16\\
8.430572705121	2.1094237467878e-15\\
8.44623160981571	1.66533453693773e-15\\
8.46205815432056	7.21644966006352e-16\\
8.4778846988254	3.33066907387547e-16\\
8.49371124333024	2.16493489801906e-15\\
8.50953778783509	1.66533453693773e-15\\
8.52553761739547	7.7715611723761e-16\\
8.54153744695585	3.33066907387547e-16\\
8.55753727651623	2.22044604925031e-15\\
8.57353710607661	1.72084568816899e-15\\
8.58971623619016	8.32667268468867e-16\\
8.60589536630372	3.33066907387547e-16\\
8.62207449641727	2.33146835171283e-15\\
8.63825362653082	1.77635683940025e-15\\
8.65461848330403	9.99200722162641e-16\\
8.67098334007723	2.77555756156289e-16\\
8.68734819685043	2.33146835171283e-15\\
8.70371305362364	1.83186799063151e-15\\
8.72018188980809	1.0547118733939e-15\\
8.73665072599255	2.22044604925031e-16\\
8.75311956217701	2.33146835171283e-15\\
8.76958839836146	1.83186799063151e-15\\
8.78614348346302	1.0547118733939e-15\\
8.80269856856458	1.66533453693773e-16\\
8.81925365366614	2.38697950294409e-15\\
8.83580873876771	1.83186799063151e-15\\
8.85245111606455	1.22124532708767e-15\\
8.86909349336139	1.11022302462516e-16\\
8.88573587065824	2.38697950294409e-15\\
8.90237824795508	1.83186799063151e-15\\
8.91910883699713	1.22124532708767e-15\\
8.93583942603918	1.11022302462516e-16\\
8.95257001508123	2.38697950294409e-15\\
8.96930060412328	1.77635683940025e-15\\
8.98612034063892	1.27675647831893e-15\\
9.00294007715455	1.11022302462516e-16\\
9.01975981367019	2.38697950294409e-15\\
9.03657955018582	1.83186799063151e-15\\
9.05348938408815	1.27675647831893e-15\\
9.07039921799048	5.55111512312578e-17\\
9.08730905189281	2.44249065417534e-15\\
9.10421888579513	1.88737914186277e-15\\
9.12121978296026	1.38777878078145e-15\\
9.13822068012539	1.11022302462516e-16\\
9.15522157729052	2.4980018054066e-15\\
9.17222247445565	1.88737914186277e-15\\
9.18931541587169	1.38777878078145e-15\\
9.20640835728774	5.55111512312578e-17\\
9.22350129870378	2.55351295663786e-15\\
9.24059424011982	1.94289029309402e-15\\
9.25778022278737	1.4432899320127e-15\\
9.27496620545493	0\\
9.29215218812249	2.55351295663786e-15\\
9.30933817079004	1.94289029309402e-15\\
9.32661820779546	1.55431223447522e-15\\
9.34389824480088	0\\
9.3611782818063	2.55351295663786e-15\\
9.37845831881172	1.94289029309402e-15\\
9.39583344002363	1.55431223447522e-15\\
9.41320856123555	0\\
9.43058368244747	2.55351295663786e-15\\
9.44795880365938	1.94289029309402e-15\\
9.46543005570071	1.60982338570648e-15\\
9.48290130774205	5.55111512312578e-17\\
9.50037255978338	2.60902410786912e-15\\
9.51784381182471	1.94289029309402e-15\\
9.53541225878365	1.66533453693773e-15\\
9.55298070574258	5.55111512312578e-17\\
9.57054915270152	2.60902410786912e-15\\
9.58811759966045	1.99840144432528e-15\\
9.60578432346384	1.72084568816899e-15\\
9.62345104726722	1.11022302462516e-16\\
9.64111777107061	2.66453525910038e-15\\
9.658784494874	1.99840144432528e-15\\
9.67655059520445	1.77635683940025e-15\\
9.6943166955349	1.11022302462516e-16\\
9.71208279586536	2.72004641033163e-15\\
9.72984889619581	2.05391259555654e-15\\
9.74771549159443	1.83186799063151e-15\\
9.76558208699305	1.11022302462516e-16\\
9.78344868239167	2.77555756156289e-15\\
9.80131527779029	2.05391259555654e-15\\
9.81928350543996	1.88737914186277e-15\\
9.83725173308963	1.11022302462516e-16\\
9.8552199607393	2.83106871279415e-15\\
9.87318818838897	2.1094237467878e-15\\
9.89125920482513	1.88737914186277e-15\\
9.90933022126129	1.11022302462516e-16\\
9.92740123769745	2.88657986402541e-15\\
9.94547225413361	2.1094237467878e-15\\
9.96364723590061	1.99840144432528e-15\\
9.9818222176676	1.66533453693773e-16\\
9.99999719943459	2.94209101525666e-15\\
10.0181721812016	2.16493489801906e-15\\
10.0364523247657	2.1094237467878e-15\\
10.0547324683298	2.22044604925031e-16\\
10.0730126118939	2.94209101525666e-15\\
10.0912927554579	2.16493489801906e-15\\
10.1096792783833	2.1094237467878e-15\\
10.1280658013086	1.66533453693773e-16\\
10.146452324234	2.99760216648792e-15\\
10.1648388471593	2.22044604925031e-15\\
10.1833329877765	2.16493489801906e-15\\
10.2018271283938	2.77555756156289e-16\\
10.220321269011	2.99760216648792e-15\\
10.2388154096282	2.22044604925031e-15\\
10.257418428294	2.33146835171283e-15\\
10.2760214469598	2.77555756156289e-16\\
10.2946244656257	3.05311331771918e-15\\
10.3132274842915	2.27595720048157e-15\\
10.3319406634311	2.33146835171283e-15\\
10.3506538425707	2.77555756156289e-16\\
10.3693670217104	3.10862446895044e-15\\
10.38808020085	2.33146835171283e-15\\
10.4069048454226	2.38697950294409e-15\\
10.4257294899953	2.77555756156289e-16\\
10.4445541345679	3.21964677141295e-15\\
10.4633787791406	2.33146835171283e-15\\
10.4823162175897	2.4980018054066e-15\\
10.5012536560388	3.33066907387547e-16\\
10.5201910944879	3.21964677141295e-15\\
10.539128532937	2.33146835171283e-15\\
10.5581801172831	2.55351295663786e-15\\
10.5772317016292	4.44089209850063e-16\\
10.5962832859753	3.27515792264421e-15\\
10.6153348703214	2.38697950294409e-15\\
10.6345019773376	2.66453525910038e-15\\
10.6536690843538	4.44089209850063e-16\\
10.67283619137	3.33066907387547e-15\\
10.6920032983861	2.38697950294409e-15\\
10.7112873293394	2.77555756156289e-15\\
10.7305713602926	4.44089209850063e-16\\
10.7498553912458	3.38618022510673e-15\\
10.769139422199	2.44249065417534e-15\\
10.7885418041401	2.88657986402541e-15\\
10.8079441860811	5.55111512312578e-16\\
10.8273465680222	3.38618022510673e-15\\
10.8467489499633	2.38697950294409e-15\\
10.8662711360792	2.99760216648792e-15\\
10.8857933221952	5.55111512312578e-16\\
10.9053155083112	3.44169137633799e-15\\
10.9248376944271	2.4980018054066e-15\\
10.9444811647161	2.99760216648792e-15\\
10.9641246350051	5.55111512312578e-16\\
10.9837681052941	3.5527136788005e-15\\
11.0034115755831	2.55351295663786e-15\\
11.02317783773	3.05311331771918e-15\\
11.0429440998769	5.55111512312578e-16\\
11.0627103620238	3.66373598126302e-15\\
11.0824766241707	2.66453525910038e-15\\
11.102367213876	3.1641356201817e-15\\
11.1222578035813	5.55111512312578e-16\\
11.1421483932866	3.77475828372553e-15\\
11.1620389829919	2.72004641033163e-15\\
11.182055464688	3.21964677141295e-15\\
11.202071946384	6.10622663543836e-16\\
11.22208842808	3.83026943495679e-15\\
11.2421049097761	2.72004641033163e-15\\
11.2622488777205	3.27515792264421e-15\\
11.282392845665	6.10622663543836e-16\\
11.3025368136094	3.88578058618805e-15\\
11.3226807815539	2.83106871279415e-15\\
11.3429428732039	3.38618022510673e-15\\
11.363204964854	6.10622663543836e-16\\
11.3834670565041	3.99680288865056e-15\\
11.4037291481542	2.83106871279415e-15\\
11.4240073959591	3.38618022510673e-15\\
11.444285643764	6.66133814775094e-16\\
11.464563891569	3.94129173741931e-15\\
11.4848421393739	2.83106871279415e-15\\
11.5051410232308	3.38618022510673e-15\\
11.5254399070878	6.66133814775094e-16\\
11.5457387909448	3.94129173741931e-15\\
11.5660376748017	2.77555756156289e-15\\
11.5863613245615	3.38618022510673e-15\\
11.6066849743212	6.66133814775094e-16\\
11.627008624081	3.88578058618805e-15\\
11.6473322738407	2.77555756156289e-15\\
11.6676846007299	3.38618022510673e-15\\
11.6880369276191	7.21644966006352e-16\\
11.7083892545083	3.88578058618805e-15\\
11.7287415813975	2.77555756156289e-15\\
11.7491263024508	3.38618022510673e-15\\
11.7695110235041	6.66133814775094e-16\\
11.7898957445574	3.88578058618805e-15\\
11.8102804656107	2.77555756156289e-15\\
11.8307011258767	3.38618022510673e-15\\
11.8511217861426	6.66133814775094e-16\\
11.8715424464086	3.88578058618805e-15\\
11.8919631066745	2.77555756156289e-15\\
11.9124230970895	3.33066907387547e-15\\
11.9328830875046	6.66133814775094e-16\\
11.9533430779196	3.88578058618805e-15\\
11.9738030683347	2.77555756156289e-15\\
11.9943056417768	3.38618022510673e-15\\
12.014808215219	6.66133814775094e-16\\
12.0353107886612	3.94129173741931e-15\\
12.0558133621033	2.83106871279415e-15\\
12.0763616495731	3.38618022510673e-15\\
12.0969099370428	6.66133814775094e-16\\
12.1174582245125	3.94129173741931e-15\\
12.1380065119823	2.83106871279415e-15\\
12.1586035338403	3.38618022510673e-15\\
12.1792005556983	6.66133814775094e-16\\
12.1997975775563	3.94129173741931e-15\\
12.2203945994143	2.77555756156289e-15\\
12.2410432774846	3.44169137633799e-15\\
12.2616919555548	7.21644966006352e-16\\
12.2823406336251	3.88578058618805e-15\\
12.3029893116954	2.77555756156289e-15\\
12.3236924799435	3.49720252756924e-15\\
12.3443956481917	7.21644966006352e-16\\
12.3650988164398	3.94129173741931e-15\\
12.385801984688	2.77555756156289e-15\\
12.406562398648	3.5527136788005e-15\\
12.4273228126081	7.21644966006352e-16\\
12.4480832265682	3.94129173741931e-15\\
12.4688436405282	2.77555756156289e-15\\
12.4896639845707	3.60822483003176e-15\\
12.5104843286133	7.7715611723761e-16\\
12.5313046726558	3.94129173741931e-15\\
12.5521250166983	2.77555756156289e-15\\
12.5730079121072	3.60822483003176e-15\\
12.5938908075161	8.32667268468867e-16\\
12.614773702925	3.94129173741931e-15\\
12.6356565983338	2.77555756156289e-15\\
12.6566046110283	3.66373598126302e-15\\
12.6775526237227	8.32667268468867e-16\\
12.6985006364172	3.94129173741931e-15\\
12.7194486491116	2.77555756156289e-15\\
12.7404642942679	3.71924713249427e-15\\
12.7614799394242	8.32667268468867e-16\\
12.7824955845805	3.99680288865056e-15\\
12.8035112297368	2.77555756156289e-15\\
12.8245969793381	3.77475828372553e-15\\
12.8456827289394	8.88178419700125e-16\\
12.8667684785407	3.94129173741931e-15\\
12.887854228142	2.77555756156289e-15\\
12.9090125143938	3.83026943495679e-15\\
12.9301708006455	9.43689570931383e-16\\
12.9513290868973	3.99680288865056e-15\\
12.972487373149	2.77555756156289e-15\\
12.9937205946389	3.88578058618805e-15\\
13.0149538161287	8.88178419700125e-16\\
13.0361870376186	3.99680288865056e-15\\
13.0574202591085	2.77555756156289e-15\\
13.0787307850171	3.94129173741931e-15\\
13.1000413109258	9.43689570931383e-16\\
13.1213518368345	4.05231403988182e-15\\
13.1426623627431	2.83106871279415e-15\\
13.1640525369264	3.94129173741931e-15\\
13.1854427111097	9.43689570931383e-16\\
13.206832885293	4.10782519111308e-15\\
13.2282230594763	2.83106871279415e-15\\
13.2496952043377	3.99680288865056e-15\\
13.271167349199	9.99200722162641e-16\\
13.2926394940603	4.10782519111308e-15\\
13.3141116389217	2.83106871279415e-15\\
13.3356680588802	4.05231403988182e-15\\
13.3572244788388	1.0547118733939e-15\\
13.3787808987974	4.10782519111308e-15\\
13.400337318756	2.83106871279415e-15\\
13.4219803036483	4.16333634234434e-15\\
13.4436232885406	1.0547118733939e-15\\
13.4652662734329	4.16333634234434e-15\\
13.4869092583252	2.83106871279415e-15\\
13.5086410871513	4.21884749357559e-15\\
13.5303729159775	1.0547118733939e-15\\
13.5521047448036	4.21884749357559e-15\\
13.5738365736297	2.94209101525666e-15\\
13.5956595174591	4.27435864480685e-15\\
13.6174824612885	1.0547118733939e-15\\
13.639305405118	4.27435864480685e-15\\
13.6611283489474	2.94209101525666e-15\\
13.6830446722324	4.32986979603811e-15\\
13.7049609955174	1.11022302462516e-15\\
13.7268773188025	4.32986979603811e-15\\
13.7487936420875	2.99760216648792e-15\\
13.770805608124	4.32986979603811e-15\\
13.7928175741606	1.11022302462516e-15\\
13.8148295401972	4.38538094726937e-15\\
13.8368415062338	3.05311331771918e-15\\
13.858951377984	4.44089209850063e-15\\
13.8810612497342	1.11022302462516e-15\\
13.9031711214845	4.44089209850063e-15\\
13.9252809932347	3.05311331771918e-15\\
13.9474910360997	4.52415882534751e-15\\
13.9697010789646	1.16573417585641e-15\\
13.9919111218296	4.46864767411626e-15\\
14.0141211646945	3.08086889333481e-15\\
14.0364336494706	4.57966997657877e-15\\
14.0587461342466	1.19348975147204e-15\\
14.0810586190226	4.52415882534751e-15\\
14.1033711037986	3.10862446895044e-15\\
14.1257883079129	4.66293670342566e-15\\
14.1482055120272	1.22124532708767e-15\\
14.1706227161414	4.57966997657877e-15\\
14.1930399202557	3.1641356201817e-15\\
14.2155641320733	4.74620343027254e-15\\
14.2380883438909	1.22124532708767e-15\\
14.2606125557085	4.63518112781003e-15\\
14.283136767526	3.19189119579733e-15\\
14.3057702854045	4.82947015711943e-15\\
14.3284038032829	1.27675647831893e-15\\
14.3510373211614	4.66293670342566e-15\\
14.3736708390398	3.21964677141295e-15\\
14.3964159769655	4.94049245958195e-15\\
14.4191611148912	1.33226762955019e-15\\
14.4419062528169	4.74620343027254e-15\\
14.4646513907425	3.24740234702858e-15\\
14.4875104777278	5.02375918642883e-15\\
14.510369564713	1.36002320516582e-15\\
14.5332286516983	4.82947015711943e-15\\
14.5560877386835	3.30291349825984e-15\\
14.5790631229979	5.13478148889135e-15\\
14.6020385073122	1.36002320516582e-15\\
14.6250138916265	4.88498130835069e-15\\
14.6479892759408	3.3584246494911e-15\\
14.6710833264109	5.21804821573824e-15\\
14.6941773768809	1.41553435639707e-15\\
14.7172714273509	4.94049245958195e-15\\
14.740365477821	3.38618022510673e-15\\
14.7635805844029	5.32907051820075e-15\\
14.7867956909849	1.47104550762833e-15\\
14.8100107975669	5.02375918642883e-15\\
14.8332259041488	3.44169137633799e-15\\
14.8565644820063	5.41233724504764e-15\\
14.8799030598637	1.47104550762833e-15\\
14.9032416377212	5.10702591327572e-15\\
14.9265802155787	3.49720252756924e-15\\
14.9500447068131	5.49560397189452e-15\\
14.9735091980476	1.49880108324396e-15\\
14.996973689282	5.19029264012261e-15\\
15.0204381805165	3.58046925441613e-15\\
15.0440310551214	5.63438184997267e-15\\
15.0676239297264	1.55431223447522e-15\\
15.0912168043314	5.30131494258512e-15\\
15.1148096789363	3.63598040564739e-15\\
15.1385334385982	5.71764857681956e-15\\
15.16225719826	1.55431223447522e-15\\
15.1859809579218	5.38458166943201e-15\\
15.2097047175837	3.69149155687865e-15\\
15.2335618956234	5.82867087928207e-15\\
15.2574190736632	1.58206781009085e-15\\
15.281276251703	5.49560397189452e-15\\
15.3051334297427	3.7470027081099e-15\\
15.3291265950906	5.96744875736022e-15\\
15.3531197604384	1.66533453693773e-15\\
15.3771129257863	5.57887069874141e-15\\
15.4011060911341	3.80251385934116e-15\\
15.4252378496272	6.07847105982273e-15\\
15.4493696081203	1.69309011255336e-15\\
15.4735013666134	5.68989300120393e-15\\
15.4976331251065	3.88578058618805e-15\\
15.5219061221034	6.21724893790088e-15\\
15.5461791191004	1.74860126378462e-15\\
15.5704521160973	5.77315972805081e-15\\
15.5947251130942	3.94129173741931e-15\\
15.619142035055	6.35602681597902e-15\\
15.6435589570157	1.77635683940025e-15\\
15.6679758789765	5.88418203051333e-15\\
15.6923928009372	3.99680288865056e-15\\
15.7169563780926	6.49480469405717e-15\\
15.741519955248	1.83186799063151e-15\\
15.7660835324033	6.02295990859147e-15\\
15.7906471095587	4.08006961549745e-15\\
15.8153601179615	6.63358257213531e-15\\
15.8400731263644	1.88737914186277e-15\\
15.8647861347672	6.07847105982273e-15\\
15.88949914317	4.13558076672871e-15\\
15.9143644081624	6.82787160144471e-15\\
15.9392296731548	1.94289029309402e-15\\
15.9640949381471	6.21724893790088e-15\\
15.9889602031395	4.19109191795997e-15\\
16.0139806003288	6.96664947952286e-15\\
16.039000997518	1.99840144432528e-15\\
16.0640213947072	6.35602681597902e-15\\
16.0890417918965	4.27435864480685e-15\\
16.1142202516767	7.13318293321663e-15\\
16.1393987114569	2.05391259555654e-15\\
16.1645771712372	6.43929354282591e-15\\
16.1897556310174	4.38538094726937e-15\\
16.2150951374624	7.2997163869104e-15\\
16.2404346439074	2.1094237467878e-15\\
16.2657741503524	6.57807142090405e-15\\
16.2911136567974	4.44089209850063e-15\\
16.3166172552097	7.49400541621981e-15\\
16.342120853622	2.19269047363468e-15\\
16.3676244520343	6.7168492989822e-15\\
16.3931280504465	4.52415882534751e-15\\
16.4187988465862	7.68829444552921e-15\\
16.4444696427259	2.24820162486594e-15\\
16.4701404388656	6.85562717706034e-15\\
16.4958112350053	4.6074255521944e-15\\
16.5216524004798	7.85482789922298e-15\\
16.5474935659542	2.3037127760972e-15\\
16.5733347314287	6.99440505513849e-15\\
16.5991758969032	4.69069227904129e-15\\
16.6251906694304	8.07687250414801e-15\\
16.6512054419577	2.38697950294409e-15\\
16.677220214485	7.16093850883226e-15\\
16.7032349870123	4.8017145815038e-15\\
16.7294266758655	8.27116153345742e-15\\
16.7556183647187	2.47024622979097e-15\\
16.781810053572	7.2997163869104e-15\\
16.8080017424252	4.91273688396632e-15\\
16.8343737319713	8.49320613838245e-15\\
16.8607457215174	2.4980018054066e-15\\
16.8871177110636	7.46624984060418e-15\\
16.9134897006097	5.02375918642883e-15\\
16.9400454511807	8.68749516769185e-15\\
16.9666012017517	2.58126853225349e-15\\
16.9931569523226	7.66053886991358e-15\\
17.0197127028936	5.13478148889135e-15\\
17.0464557582069	8.90953977261688e-15\\
17.0731988135202	2.63677968348475e-15\\
17.0999418688336	7.82707232360735e-15\\
17.1266849241469	5.24580379135386e-15\\
17.1536189111414	9.13158437754191e-15\\
17.1805528981359	2.74780198594726e-15\\
17.2074868851304	8.02136135291676e-15\\
17.2344208721249	5.38458166943201e-15\\
17.2615495082445	9.4091401336982e-15\\
17.2886781443641	2.83106871279415e-15\\
17.3158067804836	8.21565038222616e-15\\
17.3429354166032	5.49560397189452e-15\\
17.3702625121198	9.68669588985449e-15\\
17.3975896076364	2.91433543964104e-15\\
17.424916703153	8.40993941153556e-15\\
17.4522437986696	5.63438184997267e-15\\
17.4797732620512	9.93649607039515e-15\\
17.5073027254328	2.99760216648792e-15\\
17.5348321888143	8.63198401646059e-15\\
17.5623616521959	5.77315972805081e-15\\
17.5900974943972	1.01862962509358e-14\\
17.6178333365984	3.08086889333481e-15\\
17.6455691787997	8.82627304576999e-15\\
17.6733050210009	5.91193760612896e-15\\
17.7012513604556	1.04916075827077e-14\\
17.7291976999103	3.19189119579733e-15\\
17.757144039365	9.07607322631065e-15\\
17.7850903788197	6.07847105982273e-15\\
17.8132514459017	1.07969189144796e-14\\
17.8414125129837	3.27515792264421e-15\\
17.8695735800657	9.29811783123569e-15\\
17.8977346471477	6.21724893790088e-15\\
17.9261147917276	1.11299858218672e-14\\
17.9544949363076	3.41393580072236e-15\\
17.9828750808876	9.54791801177635e-15\\
18.0112552254676	6.35602681597902e-15\\
18.0398589202161	1.14630527292547e-14\\
18.0684626149646	3.52495810318487e-15\\
18.097066309713	9.79771819231701e-15\\
18.1256700044615	6.52256026967279e-15\\
18.1545018545664	1.17961196366423e-14\\
18.1833337046712	3.60822483003176e-15\\
18.212165554776	1.00752739484733e-14\\
18.2409974048809	6.7168492989822e-15\\
18.2700621491713	1.21569421196455e-14\\
18.2991268934616	3.77475828372553e-15\\
18.328191637752	1.03528297046296e-14\\
18.3572563820424	6.88338275267597e-15\\
18.3865589068877	1.25455201782643e-14\\
18.4158614317331	3.85802501057242e-15\\
18.4451639565785	1.06581410364015e-14\\
18.4744664814239	7.07767178198537e-15\\
18.5040118216459	1.29063426612674e-14\\
18.5335571618679	3.96904731303493e-15\\
18.56310250209	1.09634523681734e-14\\
18.592647842312	7.2997163869104e-15\\
18.622441195706	1.32949207198862e-14\\
18.6522345491	4.10782519111308e-15\\
18.682027902494	1.13242748511766e-14\\
18.711821255888	7.52176099183544e-15\\
18.74186798467	1.37390099297363e-14\\
18.771914713452	4.24660306919122e-15\\
18.801961442234	1.16295861829485e-14\\
18.8320081710161	7.71605002114484e-15\\
18.8623138179204	1.41830991395864e-14\\
18.8926194648247	4.38538094726937e-15\\
18.922925111729	1.20181642415673e-14\\
18.9532307586333	7.9658502016855e-15\\
18.9838010512769	1.46549439250521e-14\\
19.0143713439205	4.55191440096314e-15\\
19.0449416365641	1.23789867245705e-14\\
19.0755119292076	8.18789480661053e-15\\
19.1063527942164	1.51267887105178e-14\\
19.1371936592252	4.71844785465692e-15\\
19.1680345242339	1.27398092075737e-14\\
19.1988753892427	8.46545056276682e-15\\
19.2299929618915	1.56541446472147e-14\\
19.2611105345403	4.85722573273506e-15\\
19.2922281071891	1.31838984174237e-14\\
19.3233456798379	8.74300631892311e-15\\
19.3547463174076	1.62092561595273e-14\\
19.3861469549774	5.05151476204446e-15\\
19.4175475925471	1.36002320516582e-14\\
19.4489482301169	9.0205620750794e-15\\
19.4806385226382	1.67643676718399e-14\\
19.5123288151595	5.24580379135386e-15\\
19.5440191076808	1.40443212615082e-14\\
19.5757094002021	9.32587340685131e-15\\
19.6076961882754	1.73472347597681e-14\\
19.6396829763487	5.44009282066327e-15\\
19.671669764422	1.45161660469739e-14\\
19.7036565524953	9.6034291630076e-15\\
19.7359469398787	1.79578574233119e-14\\
19.768237327262	5.63438184997267e-15\\
19.8005277146453	1.50435219836709e-14\\
19.8328181020287	9.93649607039515e-15\\
19.8654194755566	1.8623991238087e-14\\
19.8980208490846	5.82867087928207e-15\\
19.9306222226126	1.55431223447522e-14\\
19.9632235961405	1.02973185533983e-14\\
19.9961436388709	1.93178806284777e-14\\
20.0290636816012	6.0507154842071e-15\\
20.0619837243316	1.61259894326804e-14\\
20.0949037670619	1.06858966120171e-14\\
20.1281504830269	2.00117700188684e-14\\
20.1613971989918	6.27276008913213e-15\\
20.1946439149567	1.67366120962242e-14\\
20.2278906309217	1.10744746706359e-14\\
20.2614723645969	2.08166817117217e-14\\
20.2950540982721	6.52256026967279e-15\\
20.3286358319473	1.73194791841524e-14\\
20.3622175656225	1.14630527292547e-14\\
20.3961430262838	2.16215934045749e-14\\
20.4300684869452	6.77236045021345e-15\\
20.4639939476066	1.79856129989275e-14\\
20.497919408268	1.18793863634892e-14\\
20.5321976950164	2.24542606730438e-14\\
20.5664759817649	7.07767178198537e-15\\
20.6007542685134	1.86795023893183e-14\\
20.6350325552618	1.23234755733392e-14\\
20.6696731863085	2.33701946683595e-14\\
20.7043138173553	7.38298311375729e-15\\
20.738954448402	1.94011473553246e-14\\
20.7735950794487	1.28230759344206e-14\\
20.8086080231198	2.43555176027144e-14\\
20.8436209667909	7.68829444552921e-15\\
20.878633910462	2.01644256847544e-14\\
20.9136468541332	1.33365540833097e-14\\
20.949042565344	2.53685961126848e-14\\
20.9844382765548	8.00748356510894e-15\\
21.0198339877656	2.10109707410311e-14\\
21.0552296989764	1.38777878078145e-14\\
21.0910191546516	2.64510635616944e-14\\
21.1268086103268	8.3544282603043e-15\\
21.162598066002	2.18852713729234e-14\\
21.1983875216773	1.44467771079348e-14\\
21.234582265905	2.7602919949743e-14\\
21.2707770101326	8.72912853111529e-15\\
21.3069717543603	2.28289609438548e-14\\
21.343166498588	1.50573997714787e-14\\
21.3797786897193	2.88380430646384e-14\\
21.4163908808506	9.1177065897341e-15\\
21.453003071982	2.38142838782096e-14\\
21.4896152631133	1.5709655798446e-14\\
21.5266577262528	3.01286773307652e-14\\
21.5637001893923	9.53404022396853e-15\\
21.6007426525319	2.48967513272191e-14\\
21.6377851156714	1.64174229766445e-14\\
21.6752714017348	3.15303338993544e-14\\
21.7127576877983	9.97812943381859e-15\\
21.7502439738617	2.60347299274599e-14\\
21.7877302599251	1.71529457304587e-14\\
21.8256747161126	3.30430127704062e-14\\
21.8636191723001	1.04638520070921e-14\\
21.9015636284876	2.72420974667398e-14\\
21.9395080846751	1.79578574233119e-14\\
21.9779259262067	3.46389583683049e-14\\
22.0163437677384	1.09773301559812e-14\\
22.0547616092701	2.85604873084822e-14\\
22.0931794508017	1.88182802673964e-14\\
22.1320868496398	3.63598040564739e-14\\
22.1709942484778	1.1518563880486e-14\\
22.2099016473158	2.99621438770714e-14\\
22.2488090461539	1.97619698383278e-14\\
22.2882232273236	3.82055498349132e-14\\
22.3276374084934	1.20875531806064e-14\\
22.3670515896632	3.1488700535931e-14\\
22.4064657708329	2.07472927726826e-14\\
22.4464051256478	4.02039512792385e-14\\
22.4863444804626	1.27398092075737e-14\\
22.5262838352775	3.31124017094453e-14\\
22.5662231900924	2.18297602216921e-14\\
22.6067074061995	4.23411306016419e-14\\
22.6471916223066	1.3392065234541e-14\\
22.6876758384138	3.48887585488455e-14\\
22.7281600545209	2.29954943975486e-14\\
22.7692102626757	4.47003545289704e-14\\
22.8102604708306	1.41414657761629e-14\\
22.8513106789854	3.68038932663239e-14\\
22.8923608871403	2.42583730880597e-14\\
22.9339998357608	4.72261119099926e-14\\
22.9756387843813	1.49463774690162e-14\\
23.0172777330018	3.88994392253039e-14\\
23.0589166816223	2.56322740810333e-14\\
23.1011689388352	5.00016694715555e-14\\
23.143421196048	1.57929225252929e-14\\
23.1856734532609	4.11892742135933e-14\\
23.2279257104737	2.71449529520851e-14\\
23.2707449436263	5.25968157916168e-14\\
23.3135641767789	1.68753899743024e-14\\
23.3563834099314	4.30211422042248e-14\\
23.399202643084	2.83939538547884e-14\\
23.4425351018712	5.48866507799062e-14\\
23.4858675606585	1.76109127281165e-14\\
23.5292000194457	4.49085213460876e-14\\
23.5725324782329	2.96429547574917e-14\\
23.6163923812599	5.73430192218893e-14\\
23.6602522842869	1.83880688453542e-14\\
23.704112187314	4.6893045002605e-14\\
23.747972090341	3.09474668114262e-14\\
23.7923730104802	5.99381655419506e-14\\
23.8367739306194	1.9220736113823e-14\\
23.8811748507586	4.90024687493928e-14\\
23.9255757708978	3.23352455922077e-14\\
23.9705318120247	6.26720897400901e-14\\
24.0154878531517	2.01089145335231e-14\\
24.0604438942787	5.12506703742588e-14\\
24.1053999354057	3.38201688876438e-14\\
24.1509257391711	6.55725473919233e-14\\
24.1964515429365	2.10387263166467e-14\\
24.2419773467019	5.36098943015872e-14\\
24.2875031504673	3.53883589099269e-14\\
24.3336139249026	6.86395384974503e-14\\
24.3797246993379	2.20101714631937e-14\\
24.4258354737732	5.61495294704173e-14\\
24.4719462482085	3.70675712346724e-14\\
24.518657797687	7.19146964200945e-14\\
24.5653693471654	2.30648833365876e-14\\
24.6120808966439	5.88279425173255e-14\\
24.6587924461223	3.88439280740727e-14\\
24.7061212033434	7.53980211598559e-14\\
24.7534499605645	2.41751063612128e-14\\
24.8007787177855	6.16867668057353e-14\\
24.8481074750066	4.07174294281276e-14\\
24.8960705351928	7.91033905045424e-14\\
24.944033595379	2.53685961126848e-14\\
24.9919966555653	6.47121245478388e-14\\
25.0399597157515	4.27297086602607e-14\\
25.0885748730688	8.30446822419617e-14\\
25.1371900303861	2.66314748031959e-14\\
25.1858051877034	6.79456491070596e-14\\
25.2344203450208	4.4880765770472e-14\\
25.2837061315614	8.72496519477295e-14\\
25.332991918102	2.79637424327461e-14\\
25.3822777046427	7.13873404833976e-14\\
25.4315634911833	4.71706007587613e-14\\
25.4815392186928	9.17321774096536e-14\\
25.5315149462023	2.94070323647588e-14\\
25.5814906737117	7.50788320402762e-14\\
25.6314664012212	4.95992136251289e-14\\
25.6821522057685	9.65200142033495e-14\\
25.7328380103157	3.09335890236184e-14\\
25.783523814863	7.90062459898877e-14\\
25.8342096194102	5.2208237732998e-14\\
25.8856265092156	1.01640917904433e-13\\
25.9370433990211	3.25572901971327e-14\\
25.9884602888265	8.31973379078477e-14\\
26.0398771786319	5.49837952945609e-14\\
26.0920470854784	1.07108766300712e-13\\
26.144216992325	3.43197692487252e-14\\
26.1963868991715	8.76937411575796e-14\\
26.2485568060181	5.79536418854332e-14\\
26.3015026403881	1.12992948331225e-13\\
26.3544484747581	3.61932706027801e-14\\
26.4073943091281	9.24954557390834e-14\\
26.4603401434981	6.11455330812305e-14\\
26.5140858539001	1.19279586208165e-13\\
26.5678315643021	3.82055498349132e-14\\
26.6215772747041	9.76579928035903e-14\\
26.6753229851061	6.45733466697607e-14\\
26.7298936218998	1.26010313294955e-13\\
26.7844642586936	4.03566069451244e-14\\
26.8390348954873	1.03236863502332e-13\\
26.893605532281	6.82648382266393e-14\\
26.9490273164448	1.33254518530634e-13\\
27.0044491006085	4.26464419334138e-14\\
27.0598708847722	1.09190434471884e-13\\
27.1152926689359	7.22200077518664e-14\\
27.1715930662283	1.41053835278626e-13\\
27.2278934635207	4.51305659510126e-14\\
27.2841938608131	1.15615850226902e-13\\
27.3404942581055	7.64804886088655e-14\\
27.3972810276665	1.44398382140309e-13\\
27.4540677972275	4.88220575078913e-14\\
27.5108545667884	1.15379927834169e-13\\
27.5676413363494	7.68551888796765e-14\\
27.6246835621125	1.44384504352502e-13\\
27.6817257878755	4.87943019322756e-14\\
27.7387680136386	1.15372988940265e-13\\
27.7958102394016	7.68482499857726e-14\\
27.8531117250382	1.44384504352502e-13\\
27.9104132106748	4.88151186139874e-14\\
27.9677146963114	1.15359111152458e-13\\
28.025016181948	7.68482499857726e-14\\
28.0825792753208	1.44377565458598e-13\\
28.1401423686935	4.88151186139874e-14\\
28.1977054620663	1.15352172258554e-13\\
28.2552685554391	7.68551888796765e-14\\
28.3130956457549	1.4436368767079e-13\\
28.3709227360708	4.88151186139874e-14\\
28.4287498263866	1.15352172258554e-13\\
28.4865769167025	7.68621277735804e-14\\
28.5446704260258	1.44356748776886e-13\\
28.602763935349	4.88220575078913e-14\\
28.6608574446723	1.1534523336465e-13\\
28.7189509539956	7.68621277735804e-14\\
28.7773133374	1.44356748776886e-13\\
28.8356757208045	4.88220575078913e-14\\
28.8940381042089	1.15338294470746e-13\\
28.9524004876134	7.68621277735804e-14\\
29.0110342352743	1.44342870989078e-13\\
29.0696679829352	4.88220575078913e-14\\
29.1283017305961	1.15338294470746e-13\\
29.186935478257	7.68690666674843e-14\\
29.2458431140456	1.44335932095174e-13\\
29.3047507498342	4.88359352956991e-14\\
29.3636583856228	1.15331355576842e-13\\
29.4225660214114	7.68690666674843e-14\\
29.4817501049045	1.4432899320127e-13\\
29.5409341883977	4.88289964017952e-14\\
29.6001182718908	1.15324416682938e-13\\
29.659302355384	7.68690666674843e-14\\
29.7187654815208	1.4432899320127e-13\\
29.7782286076576	4.8842874189603e-14\\
29.8376917337944	1.1531053889513e-13\\
29.8971548599312	7.68690666674843e-14\\
29.9568996607722	1.44322054307366e-13\\
30.0166444616132	4.8842874189603e-14\\
30.0763892624543	1.15303600001226e-13\\
30.1361340632953	7.68690666674843e-14\\
30.1961632086509	1.44322054307366e-13\\
30.2561923540065	4.88498130835069e-14\\
30.316221499362	1.15289722213419e-13\\
30.3762506447176	7.68690666674843e-14\\
30.436566841334	1.44315115413463e-13\\
30.4968830379503	4.88636908713147e-14\\
30.5571992345667	1.15282783319515e-13\\
30.617515431183	7.68690666674843e-14\\
30.6781214246657	1.44315115413463e-13\\
30.7387274181483	4.88636908713147e-14\\
30.799333411631	1.15268905531707e-13\\
30.8599394051136	7.68690666674843e-14\\
30.9208379817972	1.44301237625655e-13\\
30.9817365584807	4.88706297652186e-14\\
31.0426351351642	1.15268905531707e-13\\
31.1035337118477	7.68760055613882e-14\\
31.1647276970968	1.44294298731751e-13\\
31.2259216823458	4.88636908713147e-14\\
31.2871156675948	1.15261966637803e-13\\
31.3483096528438	7.68829444552921e-14\\
31.4098019130839	1.44280420943943e-13\\
31.4712941733241	4.88775686591225e-14\\
31.5327864335642	1.15255027743899e-13\\
31.5942786938043	7.68829444552921e-14\\
31.656072138367	1.44273482050039e-13\\
31.7178655829298	4.88775686591225e-14\\
31.7796590274925	1.15248088849995e-13\\
31.8414524720552	7.6889883349196e-14\\
31.903550051626	1.44273482050039e-13\\
31.9656476311968	4.88775686591225e-14\\
32.0277452107676	1.15234211062187e-13\\
32.0898427903383	7.6889883349196e-14\\
32.1522474993513	1.44259604262231e-13\\
32.2146522083642	4.88845075530264e-14\\
32.2770569173772	1.15234211062187e-13\\
32.3394616263901	7.6889883349196e-14\\
32.4021765035977	1.44252665368327e-13\\
32.4648913808053	4.88914464469303e-14\\
32.527606258013	1.15227272168283e-13\\
32.5903211352206	7.68968222430999e-14\\
32.6533492654047	1.44245726474423e-13\\
32.7163773955889	4.88914464469303e-14\\
32.7794055257731	1.1522033327438e-13\\
32.8424336559572	7.69037611370038e-14\\
32.9057781690946	1.4423878758052e-13\\
32.969122682232	4.88914464469303e-14\\
33.0324671953693	1.1522033327438e-13\\
33.0958117085067	7.69107000309077e-14\\
33.1594757818323	1.44224909792712e-13\\
33.2231398551579	4.88914464469303e-14\\
33.2868039284835	1.15206455486572e-13\\
33.3504680018091	7.69107000309077e-14\\
33.4144548595075	1.44217970898808e-13\\
33.478441717206	4.89053242347381e-14\\
33.5424285749044	1.15199516592668e-13\\
33.6064154326028	7.69107000309077e-14\\
33.6707283499096	1.44211032004904e-13\\
33.7350412672163	4.89053242347381e-14\\
33.7993541845231	1.15192577698764e-13\\
33.8636671018298	7.69176389248116e-14\\
33.9283094023156	1.44204093111e-13\\
33.9929517028013	4.89053242347381e-14\\
34.057594003287	1.1518563880486e-13\\
34.1222363037727	7.69176389248116e-14\\
34.1872113622324	1.44190215323192e-13\\
34.2521864206922	4.89122631286421e-14\\
34.3171614791519	1.15178699910956e-13\\
34.3821365376116	7.69245778187155e-14\\
34.447447781095	1.44183276429288e-13\\
34.5127590245783	4.8919202022546e-14\\
34.5780702680616	1.15164822123148e-13\\
34.643381511545	7.69315167126194e-14\\
34.7090324201268	1.44176337535384e-13\\
34.7746833287086	4.8919202022546e-14\\
34.8403342372905	1.15161352676196e-13\\
34.9059851458723	7.69315167126194e-14\\
34.9719792533529	1.44165929194529e-13\\
35.0379733608335	4.89261409164499e-14\\
35.1039674683141	1.1515094433534e-13\\
35.1699615757948	7.69349861595714e-14\\
35.2363024710846	1.44162459747577e-13\\
35.3026433663745	4.89330798103538e-14\\
35.3689842616644	1.15144005441437e-13\\
35.4353251569542	7.69349861595714e-14\\
35.5020164861975	1.44148581959769e-13\\
35.5687078154407	4.89330798103538e-14\\
35.6353991446839	1.15133597100581e-13\\
35.7020904739271	7.69419250534753e-14\\
35.76913594038	1.44138173618913e-13\\
35.8361814068328	4.89400187042577e-14\\
35.9032268732857	1.15126658206677e-13\\
35.9702723397385	7.69419250534753e-14\\
36.0376757049608	1.44134704171961e-13\\
36.1050790701831	4.89434881512096e-14\\
36.1724824354054	1.15119719312773e-13\\
36.2398858006277	7.69453945004273e-14\\
36.3076508877251	1.44124295831105e-13\\
36.3754159748225	4.89538964920655e-14\\
36.4431810619199	1.15109310971917e-13\\
36.5109461490173	7.69488639473792e-14\\
36.5790768402747	1.44117356937201e-13\\
36.6472075315322	4.89538964920655e-14\\
36.7153382227897	1.15098902631061e-13\\
36.7834689140472	7.69523333943312e-14\\
36.8519691566203	1.44106948596345e-13\\
36.9204693991935	4.89573659390174e-14\\
36.9889696417667	1.15088494290205e-13\\
37.0574698843398	7.69592722882351e-14\\
37.1263436888854	1.44096540255489e-13\\
37.1952174934311	4.89643048329214e-14\\
37.2640912979767	1.15085024843253e-13\\
37.3329651025223	7.6962741735187e-14\\
37.4022165438372	1.44086131914634e-13\\
37.4714679851522	4.89643048329214e-14\\
37.5407194264671	1.15078085949349e-13\\
37.609970867782	7.6966211182139e-14\\
37.6796040887873	1.4407919302073e-13\\
37.7492373097926	4.89712437268253e-14\\
37.8188705307979	1.15064208161542e-13\\
37.8885037518033	7.6966211182139e-14\\
37.9585229632751	1.44072254126826e-13\\
38.0285421747469	4.89781826207292e-14\\
38.0985613862187	1.15053799820686e-13\\
38.1685805976905	7.69696806290909e-14\\
38.2389900798134	1.4406184578597e-13\\
38.3093995619363	4.90024687493928e-14\\
38.3798090440592	1.15005227563358e-13\\
38.4502185261821	7.69453945004273e-14\\
38.5209622984945	1.4347550925109e-13\\
38.591706070807	4.90267548780565e-14\\
38.6624498431194	1.14311338172968e-13\\
38.7331936154318	7.65464081009526e-14\\
38.8042656863826	1.42750394838131e-13\\
38.8753377573333	4.8770015803612e-14\\
38.9464098282841	1.13759696107607e-13\\
39.0174818992348	7.61821161709975e-14\\
39.0888909542047	1.4209119991726e-13\\
39.1603000091745	4.85375628578311e-14\\
39.2317090641443	1.13246217958718e-13\\
39.3031181191142	7.5845579816658e-14\\
39.3748725127202	1.41477107806764e-13\\
39.4466269063262	4.83224571468099e-14\\
39.5183812999322	1.12774373173252e-13\\
39.5901356935382	7.55367990379341e-14\\
39.662243568804	1.40908118506644e-13\\
39.7343514440698	4.81246986705486e-14\\
39.8064593193355	1.12340692304258e-13\\
39.8785671946013	7.52523043878739e-14\\
39.95103651206	1.40387701463851e-13\\
40.0235058295186	4.7944287429047e-14\\
40.0959751469772	1.11934767010879e-13\\
40.1684444644358	7.49886264195254e-14\\
40.2412830193083	1.39905448337529e-13\\
40.3141215741807	4.77777539753532e-14\\
40.3869601290532	1.11560066740068e-13\\
40.4597986839256	7.47457651328887e-14\\
40.5330141309314	1.39461359127679e-13\\
40.6062295779373	4.76250983094673e-14\\
40.6794450249431	1.11213122044873e-13\\
40.752660471949	7.45202510810117e-14\\
40.8262603410054	1.39048494940397e-13\\
40.8998602100619	4.74828509844372e-14\\
40.9734600791183	1.10897402372245e-13\\
41.0470599481748	7.43155537108464e-14\\
41.1210516636287	1.38666855775682e-13\\
41.1950433790826	4.73544814472149e-14\\
41.2690350945364	1.10599029934377e-13\\
41.3430268099903	7.4124734128489e-14\\
41.4174177047231	1.38316441633535e-13\\
41.4918085994559	4.72330508038965e-14\\
41.5661994941887	1.10324943625173e-13\\
41.6405903889215	7.39512617808913e-14\\
41.7153877210576	1.37986844173099e-13\\
41.7901850531936	4.712376322491e-14\\
41.8649823853296	1.10073408721156e-13\\
41.9397797174656	7.37916672211014e-14\\
42.0149906816255	1.37685002288279e-13\\
42.0902016457854	4.70214145398273e-14\\
42.1654126099454	1.09839221051899e-13\\
42.2406235741053	7.36442157256434e-14\\
42.316255315397	1.37402242361695e-13\\
42.3918870566888	4.69277394721246e-14\\
42.4675187979805	1.09622380617402e-13\\
42.5431505392722	7.35089072945172e-14\\
42.6192101664345	1.37142033840298e-13\\
42.6952697935968	4.68427380218017e-14\\
42.771329420759	1.09421152694189e-13\\
42.8473890479213	7.33840072042469e-14\\
42.9238836429179	1.36900907277138e-13\\
43.0003782379145	4.67629407419068e-14\\
43.0768728329111	1.09237272005736e-13\\
43.1533674279077	7.32695154548324e-14\\
43.2303040564086	1.36682332119165e-13\\
43.3072406849096	4.66918170793917e-14\\
43.3841773134105	1.09063799658138e-13\\
43.4611139419115	7.31654320462738e-14\\
43.5384996649489	1.36474165302047e-13\\
43.6158853879863	4.66258975873046e-14\\
43.6932711110238	1.08904205098348e-13\\
43.7706568340612	7.30682875316191e-14\\
43.8484987166093	1.3628334571969e-13\\
43.9263405991575	4.65651822656454e-14\\
44.0041824817057	1.08755018879414e-13\\
44.0820243642538	7.29798166343443e-14\\
44.1603294838729	1.3610640392514e-13\\
44.2386346034919	4.65096711144142e-14\\
44.3169397231109	1.08619710448288e-13\\
44.39524484273	7.28982846309734e-14\\
44.4740202988846	1.35941605194922e-13\\
44.5527957550393	4.64593641336108e-14\\
44.631571211194	1.08491340911065e-13\\
44.7103466673487	7.28236915215064e-14\\
44.7895995895968	1.35788949529037e-13\\
44.8688525118449	4.64125265997595e-14\\
44.948105434093	1.08373379714699e-13\\
45.0273583563411	7.27560373059433e-14\\
45.1070959123988	1.35648436927482e-13\\
45.1868334684566	4.6370893236336e-14\\
45.2665710245144	1.08264092135713e-13\\
45.3463085805721	7.26935872608081e-14\\
45.4265379845256	1.35518332666784e-13\\
45.5067673884792	4.63327293198645e-14\\
45.5869967924327	1.08160008727154e-13\\
45.6672261963862	7.26363413861009e-14\\
45.7479547148514	1.3539516729999e-13\\
45.8286832333165	4.6298034850345e-14\\
45.9094117517817	1.08068068382927e-13\\
45.9901402702469	7.25842996818216e-14\\
46.0713752310891	1.35280675550575e-13\\
46.1526101919314	4.62668098277774e-14\\
46.2338451527736	1.07979597485652e-13\\
46.3150801136159	7.25374621479702e-14\\
46.3968289125376	1.35176592142017e-13\\
46.4785777114593	4.62373195286858e-14\\
46.560326510381	1.07898065482281e-13\\
46.6420753093027	7.24958287845467e-14\\
46.7243454163926	1.35077712903886e-13\\
46.8066155234824	4.62112986765462e-14\\
46.8888856305723	1.07821737649338e-13\\
46.9711557376621	7.24576648680753e-14\\
47.0539547042701	1.34987507283135e-13\\
47.136753670878	4.61887472713585e-14\\
47.2195526374859	1.07750613986823e-13\\
47.3023516040939	7.24212356750797e-14\\
47.3856870701744	1.34902505832812e-13\\
47.4690225362549	4.61679305896467e-14\\
47.5523580023354	1.0768295977126e-13\\
47.6356934684159	7.23917453759881e-14\\
47.7195731680798	1.34822708552917e-13\\
47.8034528677438	4.6148848631411e-14\\
47.8873325674077	1.07623979173077e-13\\
47.9712122670716	7.23622550768965e-14\\
48.0556440359198	1.3474811544345e-13\\
48.1400758047679	4.61315013966512e-14\\
48.2245075736161	1.07564998574894e-13\\
48.3089393424642	7.23379689482329e-14\\
48.3939311234603	1.34676991780935e-13\\
48.4789229044563	4.61158888853674e-14\\
48.5639146854524	1.07512956870615e-13\\
48.6489064664485	7.23154175430452e-14\\
48.7344663172648	1.34612807012324e-13\\
48.820026168081	4.61028784592976e-14\\
48.9055860188973	1.07462649889811e-13\\
48.9911458697136	7.22954682230714e-14\\
49.0772819670679	1.34552959052403e-13\\
49.1634180644223	4.60916027567038e-14\\
49.2495541617767	1.07414944994222e-13\\
49.3356902591311	7.22763862648357e-14\\
49.4224109073406	1.34496580539434e-13\\
49.5091315555501	4.60829291393239e-14\\
49.5958522037596	1.07369842183846e-13\\
49.6825728519691	7.22599063918139e-14\\
49.769886487838	1.34443671473417e-13\\
49.857200123707	4.6075122883682e-14\\
49.944513759576	1.07327341458685e-13\\
50.0318273954449	7.22460286040061e-14\\
50.1197425956162	1.34392497130875e-13\\
50.2076577957875	4.60673166280401e-14\\
50.2955729959588	1.07289177542214e-13\\
50.3834881961302	7.22338855396742e-14\\
50.4720136825001	1.34345659597024e-13\\
50.5605391688701	4.60612450958742e-14\\
50.64906465524	1.0725188098748e-13\\
50.73759014161	7.22234771988184e-14\\
50.8267347882915	1.34300556786648e-13\\
50.915879434973	4.60560409254462e-14\\
51.0050240816545	1.07216319156223e-13\\
51.094168728336	7.22148035814385e-14\\
51.1839415683769	1.34258056061487e-13\\
51.2737144084178	4.60534388402323e-14\\
51.3634872484587	1.07182492048441e-13\\
51.4532600884996	7.22078646875346e-14\\
51.5436703203087	1.3421815742154e-13\\
51.6340805521178	4.60491020315423e-14\\
51.7244907839268	1.07151267025873e-13\\
51.8149010157359	7.22009257936307e-14\\
51.9059580087448	1.34179993505068e-13\\
51.9970150017537	4.60482346698043e-14\\
52.0880719947626	1.07120042003306e-13\\
52.1791289877716	7.21965889849407e-14\\
52.270842290771	1.34142696950335e-13\\
52.3625555937704	4.60464999463284e-14\\
52.4542688967698	1.0709228642769e-13\\
52.5459821997693	7.21931195379888e-14\\
52.6383615465513	1.34108002480815e-13\\
52.7307408933333	4.60464999463284e-14\\
52.8231202401153	1.07064530852075e-13\\
52.9154995868973	7.21905174527748e-14\\
53.0085549039533	1.34074175373033e-13\\
53.1016102210094	4.60464999463284e-14\\
53.1946655380654	1.07038509999935e-13\\
53.2877208551214	7.21887827292989e-14\\
53.3814622684928	1.34042082988728e-13\\
53.4752036818643	4.60464999463284e-14\\
53.5689450952358	1.07012489147795e-13\\
53.6626865086073	7.21879153675609e-14\\
53.7571243510415	1.34011725327898e-13\\
53.8515621934757	4.60491020315423e-14\\
53.9460000359099	1.06987335657394e-13\\
54.0404378783442	7.21879153675609e-14\\
54.1355826971555	1.33980500305331e-13\\
54.2307275159668	4.60499693932803e-14\\
54.325872334778	1.06964784252206e-13\\
54.4210171535893	7.21887827292989e-14\\
54.5168797176294	1.33951010006239e-13\\
54.6127422816695	4.60534388402323e-14\\
54.7086048457095	1.0694136548528e-13\\
54.8044674097496	7.21905174527748e-14\\
54.9010587178797	1.33922387068885e-13\\
54.9976500260098	4.60551735637083e-14\\
55.0942413341398	1.06918814080093e-13\\
55.1908326422699	7.21931195379888e-14\\
55.2881639308243	1.33895498855008e-13\\
55.3854952193786	4.60595103723982e-14\\
55.4828265079329	1.06896696355774e-13\\
55.5801577964873	7.21957216232028e-14\\
55.6782405480441	1.33868176960261e-13\\
55.776323299601	4.60629798193501e-14\\
55.8744060511579	1.06875012312324e-13\\
55.9724888027147	7.21991910701547e-14\\
56.0713347542828	1.33841288746384e-13\\
56.1701807058509	4.60668829471711e-14\\
56.2690266574189	1.06854195630612e-13\\
56.367872608987	7.22030941979757e-14\\
56.4674937609263	1.33815267894244e-13\\
56.5671149128656	4.607165343673e-14\\
56.6667360648049	1.06833378948901e-13\\
56.7663572167441	7.22078646875346e-14\\
56.8667658411891	1.33789247042104e-13\\
56.9671744656341	4.6076423926289e-14\\
57.0675830900791	1.06813429628927e-13\\
57.1679917145241	7.22135025388315e-14\\
57.2692003658787	1.33763659870834e-13\\
57.3704090172333	4.60811944158479e-14\\
57.4716176685879	1.06793046628084e-13\\
57.5728263199425	7.22191403901284e-14\\
57.6748478434472	1.33738072699563e-13\\
57.7768693669519	4.60863985862758e-14\\
57.8788908904565	1.06773530988979e-13\\
57.9809124139612	7.22252119222944e-14\\
58.0837599560178	1.33712919209161e-13\\
58.1866074980744	4.60924701184418e-14\\
58.2894550401309	1.06753581669006e-13\\
58.3923025821875	7.22312834544603e-14\\
58.4959896007833	1.3368776571876e-13\\
58.5996766193791	4.60981079697387e-14\\
58.7033636379749	1.06734066029901e-13\\
58.8070506565707	7.22382223483642e-14\\
58.911590931702	1.33663479590096e-13\\
59.0161312068333	4.61046131827736e-14\\
59.1206714819646	1.06714116709927e-13\\
59.2252117570959	7.22455949231371e-14\\
59.3306194021433	1.33638326099694e-13\\
59.4360270471907	4.61115520766775e-14\\
59.541434692238	1.06695034751692e-13\\
59.6468423372854	7.2252533817041e-14\\
59.7351317529641	5.34117021444569e-14\\
59.8234211686427	6.37320057839119e-14\\
59.9117105843214	2.23575498392581e-14\\
60	2.41126563160776e-16\\
};
\addlegendentry{Infetti}

\end{axis}
\end{tikzpicture}%}
}
\caption[Errori assoluti relativi al grafo~\ref{fig::lollipop}] {Errore assoluto tra la soluzione del problema esteso  e quello chiuso.\\
Per ottenere i grafici abbiamo risolto numericamente, con una tolleranza di $1e-9$ i due problemi di Cauchy con conduzioni iniziali di stati puri~\eqref{statipuri_lollipop}. Abbiamo inoltre supposto l'indipendenza statica delle condizioni iniziali.\\
Per la sperimentazione abbiamo usato come parametri $\tau = 0.3$ e $\gamma = 0.1$.}
\label{fig::lollipop_errori}

\end{figure}

%%% Local Variables:
%%% mode: latex
%%% TeX-master: "main"
%%% End:


 