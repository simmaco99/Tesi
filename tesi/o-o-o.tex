\chapter{Il modello SIR \textit{bottom-up} su una rete}
In questo capitolo ci occuperemo di descrivere il modello SIR su di una rete sfruttando il modello scalare ed i concetti introdotti nella Sezione~\ref{grafi}. Costruiremo un modello Markoviano SIR su una rete di $N$ individui. Ogni individuo \`e in ogni istante temporale, suscettibile, infetto o rimosso con una certa probabilit\`a $p$. Come nel modello scalare del capitolo precedente, un individuo infetto pu\`o trasmettere il suo stato con la differenza che questo processo avviene unicamente per i nodi che sono suoi vicini nella rete.\\
In particolare ci occuperemo di derivare le equazioni per lo stato di ogni singolo nodo delle rete ottenendo un modello ``\textit{bottom-up}" esatto dell'epidemia. Successivamente cercheremo di ridurre il numero di equazioni del modello con il fine di mantenere l'esattezza e semplificare, dal punto di vista computazionale, il problema da integrare.  \section{Un primo esempio}
Vediamo come \`e possibile applicare il modello $SIR$ ad un semplice grafo con 3 nodi (vedi Figura~\ref{fig::3nodi}).\\
\begin{figure}[ht]
\centering
\begin{tikzpicture}
	\Vertex[label=1]{1} \Vertex[x=2,label=2]{2} \Vertex[x=4,label=3]{3}
	\Edge(1)(2) \Edge(2)(3)
\end{tikzpicture}	
\caption{Grafo con 3 nodi.}
\label{fig::3nodi}
\end{figure}

Fissiamo la notazione:
\begin{itemize}
	\item  $\angol{S_i}(t)$  denota la probabilit\`a che il nodo $i$ sia suscettibile al tempo $t$;
	\item   $\angol{I_i}(t)$  denota la probabilit\`a che il nodo $i$ sia infetto al tempo $t$;
	\item  $\angol{R_i}(t)$  denota la probabilit\`a che il nodo $t$ sia rimosso al tempo $t$;
	\item $\tonde{ \angol{ I_i}+ \angol{ R_i}}(t)$ denota la probabilit\`a che il nodo $i$ sia infetto o rimosso al tempo $t$;
	\item $\angol{S_iI_j}(t)$ denota la probabilit\`a che il nodo $i$ sia sano e il nodo $j$ sia infetto al tempo $t$.
	\end{itemize}
In analogia a quanto fatto in~\ref{modellosir} si ha che:
\begin{itemize}
	\item ogni nodo $i$ pu\`o diventare infetto se \`e suscettibile e almeno uno dei suoi vicini \`e infetto. Per ogni suo vicino infetto $i$ diventa infetto con un tasso di~$\tau$;
	\item una volta infettato, ogni nodo diventa rimosso con un tasso di $\gamma$ indipendentemente dallo stato di qualsiasi altro nodo.
\end{itemize}

Passiamo ora allo studio del grafo~\ref{fig::3nodi}.
\begin{itemize}
	

\item Poich\`e il nodo $1$ ha come vicino solamente il nodo $2$ la probabbilit\`a che possa infettarsi \`e $\angol{S_1 I_2} $. $\angol{S_1}$ diminuisce con un tasso di $\tau \angol{S_1 I_2}$.
\item Il nodo $2$ ha due possibili fonti d'infezione: i nodi $1$ e $2$. Il suo tasso d'infezione \`e, dunque,  $\tau \tonde{ \angol{S_2I_1} + \angol{S_2 I_3}}$.
\item Il tasso d'infezione relativo al nodo $3$ \`e,  
in analogia a quanto visto per il nodo $1$, $\tau\angol{S_3I_2}$.
\end{itemize}
Mettendo insieme questi punti e ricordando il tasso di rimozione otteniamo il seguente sistema di equazioni differenziali:
\begin{equation*}
\begin{aligned}
	\dot{\angol {S_1}} = & -\tau \angol{ S_1 I_2}, 
\quad &
	\dot{\angol {I_1}} = & \tau \angol{S_1 I_2}-\gamma \angol{I_1}, 
\\
	\dot{\angol {S_2}} = & -\tau \tonde{ \angol{ I_1 S_2} + \angol{I_3S_2}},	
\quad & 
	\dot{\angol {I_2}} = & \tau \tonde{ \angol{ I_1 S_2} + \angol{I_3S_2}}-\gamma \angol{I_2},
\\
	\dot{\angol {S_3}} = & -\tau \angol{ I_2 S_3},
\quad & 
	\dot{\angol {I_3}} = & \tau \angol{ I_2 S_3}-\gamma \angol{I_3},
\end{aligned}
\end{equation*}
\begin{equation*} 
\begin{aligned}
	\dot{\angol {R_1}} = & \gamma \angol{ I_1},\\
	\dot{\angol {R_2}} = & \gamma \angol{ I_2},\\
 	\dot{\angol {R_3}} = & \gamma \angol{ I_3}.
 	\end{aligned}
\end{equation*}
Analizzando attentamente il sistema possiamo notare che le equazioni per $\angol{R_i}$ possono essere rimosse infatti 
$$ \forall i\in\{1, 2, 3\} \quad \forall t\geq 0 \quad \angol{R_i}(t) = 1 - \tonde{\angol{S_i} +\angol{I_i}}(t).$$
Osserviamo, inoltre, che il sistema precedente non \`e chiuso: abbiamo altre quantit\`a  (come ad esempio: $\angol{S_1I_2}$), dunque, dobbiamo conoscere come esse evolvono nel tempo. Servono altre equazioni.\\
Usando argomenti simili a quanto fatto per i nodi otteniamo  il sistema 
\begin{equation*}
\begin{aligned}
	\dot{\angol{S_1I_2}}=&\spa\tau\angol{S_1S_2I_3} - \tonde{ \tau + \gamma}\angol{S_1 I_2},
\\
	\dot{\angol{I_1S_2}}=&-\tau\angol{I_1S_2I_3} - \tonde{ \tau + \gamma}\angol{I_1 S_2},
\\
	\dot{\angol{S_2I_3}}=&-\tau\angol{I_1S_2I_3} - \tonde{ \tau + \gamma}\angol{S_2 I_3},
\\
	\dot{\angol{I_2S_3}}=&\spa\tau\angol{I_1S_2S_3} - \tonde{ \tau + \gamma}\angol{I_2 S_3}.
	\end{aligned}
\end{equation*}
Anche tale sistema non \`e chiuso: richiede  ulteriori informazioni sulle triple.\\
Aggiungendo le equazioni necessarie otteniamo il sistema 
\begin{equation}
\label{3nodi}
\begin{aligned}
	\dot{\angol {S_1}} =& -\tau \angol{ S_1 I_2},
\\ 
	\dot{\angol {I_1}} =&\spa \tau \angol{ S_1 I_2}-\gamma \angol{I_1}, 
\\ 
	\dot{\angol {S_2}} = & -\tau \tonde{ \angol{ I_1 S_2}+ \angol{I_3S_2}},
\\
	\dot{\angol {I_2}} = &\spa \tau \tonde{ \angol{ I_1 S_2} + \angol{I_3S_2}}-\gamma \angol{I_2}, 
\\ 
	\dot{\angol {S_3}} = & -\tau \angol{ I_2 S_3}
\\
	\dot{\angol {I_3}} = & \spa \tau  \angol{ I_2 S_3}-\gamma \angol{I_3},	
\\
	\dot{\angol{S_1I_2}}=& \spa\tau\angol{S_1S_2I_3} - \tonde{ \tau + \gamma}\angol{S_1 I_2},
\\
	\dot{\angol{I_1S_2}}=&-\tau\angol{I_1S_2I_3} - \tonde{ \tau + \gamma}\angol{I_1 S_2},
\\ 
	\dot{\angol{I_2S_3}}=&\spa\tau\angol{I_1S_2I_3} - \tonde{ \tau + \gamma}\angol{I_1 S_2},
\\
	\dot{\angol{S_2I_3}}=&-\tau\angol{I_1S_2I_3} - \tonde{ \tau + \gamma}\angol{I_1 S_2},
\\
	\dot{\angol{S_1S_2I_3}}=&-\tonde{\tau + \gamma}\angol{S_1S_2I_3},
\\
	\dot{\angol{I_1S_2I_3}}=&-\tonde{2\tau + 2\gamma}\angol{I_1S_2I_3},
\\
	\dot{\angol{I_1S_2S_3}}=&-\tonde{\tau + \gamma}\angol{I_1S_2S_3}.
 \end{aligned}
\end{equation}
Nella Figura~\ref{fig::spe3nodi} sono riportati i grafici dello stato dei nodi in funzione del tempo (a), (b), (c) e del grafico di prevalenza (d).\\
Per la sperimentazione, le condizioni iniziali sono state prese a partire da uno stato puro ($1$ infetto e $2,3$ sani):
\begin{equation}
\label{statipuri}
\begin{aligned}
\angol{S_1}(0)&=0, \\
\angol{I_1}(0) &=1, \\
\angol{S_2}(0)&=1,\\
\angol{I_2}(0)& = 0,\\ 
\angol{S_3}(0)&= 1,\\
\angol{I_3}(0) &= 0,\\
\end{aligned}
\end{equation}
sotto le ipotesi di indipendenza statistica delle condizioni iniziali (vedi~\cite{MR3340258}) ovvero
$$
\angol{A_i B_j}(0) =\angol{A_i}(0) \angol{B_j}(0)$$ 
per $A,B\in \{ S, I\}$ e  $i,j=1, 2,3$.\\

\begin{figure}[h]
	\centering
	\subfloat[][Nodo 1]{% This file was created by matlab2tikz.
%
\definecolor{mycolor1}{rgb}{0.00000,0.44700,0.74100}%
\definecolor{mycolor2}{rgb}{0.85000,0.32500,0.09800}%
\definecolor{mycolor3}{rgb}{0.92900,0.69400,0.12500}%
%
\begin{tikzpicture}

\begin{axis}[%
width=0.39\columnwidth,
height=1.7in,
at={(1.011in,0.642in)},
scale only axis,
xmin=0,
xmax=60,
xlabel style={font=\color{white!15!black}},
xlabel={T},
ymin=0,
ymax=1,
axis background/.style={fill=white},
legend style={legend cell align=left, align=left, draw=none,fill=none}
]
\addplot [color=mycolor1, line width=2.0pt]
  table[row sep=crcr]{%
0	0\\
0.000167459095433972	0\\
0.000334918190867944	0\\
0.000502377286301916	0\\
0.000669836381735888	0\\
0.00150713185890575	0\\
0.00234442733607561	0\\
0.00318172281324547	0\\
0.00401901829041533	0\\
0.00820549567626463	0\\
0.0123919730621139	0\\
0.0165784504479632	0\\
0.0207649278338125	0\\
0.041697314763059	0\\
0.0626297016923055	0\\
0.083562088621552	0\\
0.104494475550799	0\\
0.209156410197031	0\\
0.313818344843264	0\\
0.418480279489496	0\\
0.523142214135729	0\\
0.693500300195606	0\\
0.863858386255483	0\\
1.03421647231536	0\\
1.20457455837524	0\\
1.44051447786383	0\\
1.67645439735243	0\\
1.91239431684103	0\\
2.14833423632963	0\\
2.44722092525848	0\\
2.74610761418733	0\\
3.04499430311618	0\\
3.34388099204503	0\\
3.7078444355742	0\\
4.07180787910338	0\\
4.43577132263256	0\\
4.79973476616174	0\\
5.20292074647974	0\\
5.60610672679773	0\\
6.00929270711573	0\\
6.41247868743373	0\\
6.85758807700814	0\\
7.30269746658256	0\\
7.74780685615697	0\\
8.19291624573139	0\\
8.68220109096914	0\\
9.17148593620689	0\\
9.66077078144463	0\\
10.1500556266824	0\\
10.6416653734136	0\\
11.1332751201447	0\\
11.6248848668759	0\\
12.1164946136071	0\\
12.6079732666127	0\\
13.0994519196183	0\\
13.5909305726239	0\\
14.0824092256295	0\\
14.573895267778	0\\
15.0653813099266	0\\
15.5568673520752	0\\
16.0483533942238	0\\
16.5398390198718	0\\
17.0313246455198	0\\
17.5228102711679	0\\
18.0142958968159	0\\
18.5057815459406	0\\
18.9972671950653	0\\
19.48875284419	0\\
19.9802384933148	0\\
20.5017696853463	0\\
21.0233008773778	0\\
21.5448320694093	0\\
22.0663632614408	0\\
22.6746017709939	0\\
23.282840280547	0\\
23.8910787901001	0\\
24.4993172996532	0\\
25.0869400675273	0\\
25.6745628354014	0\\
26.2621856032755	0\\
26.8498083711495	0\\
27.5306427936611	0\\
28.2114772161728	0\\
28.8923116386844	0\\
29.573146061196	0\\
30.3571240725376	0\\
31.1411020838792	0\\
31.9250800952207	0\\
32.7090581065623	0\\
33.6393150845161	0\\
34.5695720624699	0\\
35.4998290404237	0\\
36.4300860183775	0\\
37.5675806203718	0\\
38.7050752223662	0\\
39.8425698243606	0\\
40.980064426355	0\\
42.4285780340909	0\\
43.8770916418269	0\\
45.3256052495628	0\\
46.7741188572988	0\\
48.2741188572988	0\\
49.7741188572988	0\\
51.2741188572988	0\\
52.7741188572988	0\\
54.2741188572988	0\\
55.7741188572988	0\\
57.2741188572988	0\\
58.7741188572988	0\\
59.0805891429741	0\\
59.3870594286494	0\\
59.6935297143247	0\\
60	0\\
};
\addlegendentry{$\langle\text{ S}_\text{1}\rangle\text{(t)}$}

\addplot [color=mycolor2, line width=2.0pt]
  table[row sep=crcr]{%
0	1\\
0.000167459095433972	0.999983254230669\\
0.000334918190867944	0.999966508741758\\
0.000502377286301916	0.999949763533263\\
0.000669836381735888	0.99993301860518\\
0.00150713185890575	0.999849298170771\\
0.00234442733607561	0.999765584745943\\
0.00318172281324547	0.999681878330108\\
0.00401901829041533	0.99959817892268\\
0.00820549567626463	0.999179786991109\\
0.0123919730621139	0.998761570181715\\
0.0165784504479632	0.998343528421197\\
0.0207649278338125	0.997925661636288\\
0.041697314763059	0.995838949783605\\
0.0626297016923055	0.993756601348492\\
0.083562088621552	0.991678607206836\\
0.104494475550799	0.989604958253573\\
0.209156410197031	0.979301573919075\\
0.313818344843264	0.969105464466204\\
0.418480279489496	0.959015513024293\\
0.523142214135729	0.949030614247059\\
0.693500300195606	0.933000044355593\\
0.863858386255483	0.917240255444454\\
1.03421647231536	0.901746674002992\\
1.20457455837524	0.886514802601982\\
1.44051447786383	0.865843198617418\\
1.67645439735243	0.845653612966956\\
1.91239431684103	0.82593480798817\\
2.14833423632963	0.806675802322813\\
2.44722092525848	0.782922080229594\\
2.74610761418733	0.759867824316432\\
3.04499430311618	0.737492443896854\\
3.34388099204503	0.715775937140886\\
3.7078444355742	0.690192682774706\\
4.07180787910338	0.665523833816803\\
4.43577132263256	0.641736722764718\\
4.79973476616174	0.618799807115624\\
5.20292074647974	0.594346903836115\\
5.60610672679773	0.570860310615108\\
6.00929270711573	0.548301864570512\\
6.41247868743373	0.526634848483406\\
6.85758807700814	0.503707833308423\\
7.30269746658256	0.481778962138025\\
7.74780685615697	0.460804813087566\\
8.19291624573139	0.440743766091925\\
8.68220109096914	0.419697857386959\\
9.17148593620689	0.399656933289053\\
9.66077078144463	0.380573049485789\\
10.1500556266824	0.362400430033594\\
10.6416653734136	0.345015262566064\\
11.1332751201447	0.32846412077008\\
11.6248848668759	0.312707032158958\\
12.1164946136071	0.297705839833527\\
12.6079732666127	0.283427933623719\\
13.0994519196183	0.269834808968427\\
13.5909305726239	0.25689365474737\\
14.0824092256295	0.244573149792937\\
14.573895267778	0.232843303587973\\
15.0653813099266	0.221676039881433\\
15.5568673520752	0.21104440229279\\
16.0483533942238	0.200922658547134\\
16.5398390198718	0.191286318919159\\
17.0313246455198	0.182112153427683\\
17.5228102711679	0.173378016819964\\
18.0142958968159	0.165062769471869\\
18.5057815459406	0.157146285625828\\
18.9972671950653	0.149609488922149\\
19.48875284419	0.142434186503514\\
19.9802384933148	0.135603011660711\\
20.5017696853463	0.128712120714768\\
21.0233008773778	0.122171411693478\\
21.5448320694093	0.115963108802182\\
22.0663632614408	0.110070287679395\\
22.6746017709939	0.103574886610203\\
23.282840280547	0.0974628051472713\\
23.8910787901001	0.0917114587174947\\
24.4993172996532	0.086299501584027\\
25.0869400675273	0.0813744396788814\\
25.6745628354014	0.0767304594123107\\
26.2621856032755	0.0723515428475588\\
26.8498083711495	0.0682225246959141\\
27.5306427936611	0.0637322443897468\\
28.2114772161728	0.0595375240597255\\
28.8923116386844	0.0556189510691986\\
29.573146061196	0.0519582853704699\\
30.3571240725376	0.0480403788323907\\
31.1411020838792	0.0444179283274167\\
31.9250800952207	0.0410687214707513\\
32.7090581065623	0.0379720524409922\\
33.6393150845161	0.0345988685610616\\
34.5695720624699	0.0315253797560828\\
35.4998290404237	0.028725086953008\\
36.4300860183775	0.0261735400119507\\
37.5675806203718	0.0233591425654702\\
38.7050752223662	0.0208474489027482\\
39.8425698243606	0.0186061695928961\\
40.980064426355	0.0166058694685394\\
42.4285780340909	0.0143659865246682\\
43.8770916418269	0.0124283702026242\\
45.3256052495628	0.0107528893267973\\
46.7741188572988	0.00930337062567787\\
48.2741188572988	0.00800708232919516\\
49.7741188572988	0.00689150362710516\\
51.2741188572988	0.00593189233082438\\
52.7741188572988	0.00510596666768183\\
54.2741188572988	0.00439452507303242\\
55.7741188572988	0.00378226228170326\\
57.2741188572988	0.00325559904427218\\
58.7741188572988	0.00280230646079182\\
59.0805891429741	0.00271772674151796\\
59.3870594286494	0.00263569983783545\\
59.6935297143247	0.00255614872433423\\
60	0.00247899863141475\\
};
\addlegendentry{$\langle\text{ I}_\text{1}\rangle\text{(t)}$}

\addplot [color=mycolor3, line width=2.0pt]
  table[row sep=crcr]{%
0	0\\
0.000167459095433972	1.67457693314166e-05\\
0.000334918190867944	3.34912582420355e-05\\
0.000502377286301916	5.02364667366306e-05\\
0.000669836381735888	6.6981394819754e-05\\
0.00150713185890575	0.000150701829228828\\
0.00234442733607561	0.000234415254057319\\
0.00318172281324547	0.000318121669892091\\
0.00401901829041533	0.000401821077319897\\
0.00820549567626463	0.00082021300889068\\
0.0123919730621139	0.00123842981828526\\
0.0165784504479632	0.00165647157880289\\
0.0207649278338125	0.00207433836371185\\
0.041697314763059	0.00416105021639457\\
0.0626297016923055	0.00624339865150814\\
0.083562088621552	0.00832139279316391\\
0.104494475550799	0.0103950417464269\\
0.209156410197031	0.0206984260809253\\
0.313818344843264	0.0308945355337955\\
0.418480279489496	0.0409844869757066\\
0.523142214135729	0.0509693857529409\\
0.693500300195606	0.066999955644407\\
0.863858386255483	0.0827597445555461\\
1.03421647231536	0.0982533259970083\\
1.20457455837524	0.113485197398018\\
1.44051447786383	0.134156801382582\\
1.67645439735243	0.154346387033044\\
1.91239431684103	0.17406519201183\\
2.14833423632963	0.193324197677187\\
2.44722092525848	0.217077919770406\\
2.74610761418733	0.240132175683568\\
3.04499430311618	0.262507556103146\\
3.34388099204503	0.284224062859114\\
3.7078444355742	0.309807317225294\\
4.07180787910338	0.334476166183197\\
4.43577132263256	0.358263277235282\\
4.79973476616174	0.381200192884376\\
5.20292074647974	0.405653096163885\\
5.60610672679773	0.429139689384892\\
6.00929270711573	0.451698135429488\\
6.41247868743373	0.473365151516594\\
6.85758807700814	0.496292166691577\\
7.30269746658256	0.518221037861975\\
7.74780685615697	0.539195186912434\\
8.19291624573139	0.559256233908075\\
8.68220109096914	0.580302142613041\\
9.17148593620689	0.600343066710948\\
9.66077078144463	0.619426950514211\\
10.1500556266824	0.637599569966406\\
10.6416653734136	0.654984737433936\\
11.1332751201447	0.67153587922992\\
11.6248848668759	0.687292967841042\\
12.1164946136071	0.702294160166473\\
12.6079732666127	0.716572066376281\\
13.0994519196183	0.730165191031573\\
13.5909305726239	0.74310634525263\\
14.0824092256295	0.755426850207063\\
14.573895267778	0.767156696412027\\
15.0653813099266	0.778323960118567\\
15.5568673520752	0.78895559770721\\
16.0483533942238	0.799077341452866\\
16.5398390198718	0.808713681080841\\
17.0313246455198	0.817887846572317\\
17.5228102711679	0.826621983180036\\
18.0142958968159	0.834937230528131\\
18.5057815459406	0.842853714374172\\
18.9972671950653	0.850390511077851\\
19.48875284419	0.857565813496486\\
19.9802384933148	0.864396988339289\\
20.5017696853463	0.871287879285232\\
21.0233008773778	0.877828588306522\\
21.5448320694093	0.884036891197818\\
22.0663632614408	0.889929712320605\\
22.6746017709939	0.896425113389797\\
23.282840280547	0.902537194852729\\
23.8910787901001	0.908288541282505\\
24.4993172996532	0.913700498415973\\
25.0869400675273	0.918625560321119\\
25.6745628354014	0.923269540587689\\
26.2621856032755	0.927648457152441\\
26.8498083711495	0.931777475304086\\
27.5306427936611	0.936267755610253\\
28.2114772161728	0.940462475940275\\
28.8923116386844	0.944381048930801\\
29.573146061196	0.94804171462953\\
30.3571240725376	0.951959621167609\\
31.1411020838792	0.955582071672583\\
31.9250800952207	0.958931278529249\\
32.7090581065623	0.962027947559008\\
33.6393150845161	0.965401131438938\\
34.5695720624699	0.968474620243917\\
35.4998290404237	0.971274913046992\\
36.4300860183775	0.973826459988049\\
37.5675806203718	0.97664085743453\\
38.7050752223662	0.979152551097252\\
39.8425698243606	0.981393830407104\\
40.980064426355	0.983394130531461\\
42.4285780340909	0.985634013475332\\
43.8770916418269	0.987571629797376\\
45.3256052495628	0.989247110673203\\
46.7741188572988	0.990696629374322\\
48.2741188572988	0.991992917670805\\
49.7741188572988	0.993108496372895\\
51.2741188572988	0.994068107669176\\
52.7741188572988	0.994894033332318\\
54.2741188572988	0.995605474926968\\
55.7741188572988	0.996217737718297\\
57.2741188572988	0.996744400955728\\
58.7741188572988	0.997197693539208\\
59.0805891429741	0.997282273258482\\
59.3870594286494	0.997364300162165\\
59.6935297143247	0.997443851275666\\
60	0.997521001368585\\
};
\addlegendentry{$\langle\text{ R}_\text{1}\rangle\text{(t)}$}

\end{axis}
\end{tikzpicture}%}
	\subfloat[][Nodo 2]{% This file was created by matlab2tikz.
%
%The latest updates can be retrieved from
%  http://www.mathworks.com/matlabcentral/fileexchange/22022-matlab2tikz-matlab2tikz
%where you can also make suggestions and rate matlab2tikz.
%
\begin{tikzpicture}

\begin{axis}[%
width=6.028in,
height=4.754in,
at={(1.011in,0.642in)},
scale only axis,
xmin=0,
xmax=5,
xlabel style={font=\color{white!15!black}},
xlabel={t},
ymin=0,
ymax=1,
axis background/.style={fill=white},
axis x line*=bottom,
axis y line*=left,
legend style={legend cell align=left, align=left, draw=white!15!black}
]
\addplot [color=blue, dashed, line width=2.0pt]
  table[row sep=crcr]{%
0	1\\
0.000100475457260383	0.999949766056924\\
0.000200950914520766	0.999899539684195\\
0.00030142637178115	0.999849320880672\\
0.000401901829041533	0.999799109645214\\
0.000904279115343449	0.999548166948998\\
0.00140665640164536	0.999297413283416\\
0.00190903368794728	0.999046848506074\\
0.0024114109742492	0.998796472474686\\
0.00492329740575878	0.997547418534447\\
0.00743518383726836	0.996303061961732\\
0.00994707026877794	0.995063385091027\\
0.0124589567002875	0.993828370323048\\
0.0250183888578354	0.987722616365095\\
0.0375778210153833	0.981730813042634\\
0.0501372531729312	0.975850833944071\\
0.0626966853304791	0.970080591669277\\
0.0991412937643315	0.953939046011196\\
0.135585902198184	0.938656254473793\\
0.172030510632036	0.924186587013759\\
0.208475119065889	0.910486692796938\\
0.257093885813804	0.893338017841503\\
0.305712652561719	0.877395555845\\
0.354331419309635	0.862574680234777\\
0.40295018605755	0.848796177141757\\
0.46627950019307	0.832290618164177\\
0.52960881432859	0.817281114199863\\
0.592938128464109	0.803632751202841\\
0.656267442599629	0.791221205689014\\
0.733658471248229	0.777568429104202\\
0.811049499896829	0.765412570030883\\
0.888440528545428	0.754590931818442\\
0.965831557194028	0.744955370973288\\
1.05793596976172	0.734850994744874\\
1.15004038232942	0.726051284757239\\
1.24214479489711	0.718390163309227\\
1.3342492074648	0.711717689094676\\
1.44286725998794	0.704941359996623\\
1.55148531251107	0.699184906067961\\
1.66010336503421	0.694298310044978\\
1.76872141755734	0.690146590836533\\
1.88893196521442	0.68626979781265\\
2.00914251287149	0.683033580994597\\
2.12935306052856	0.680335328566429\\
2.24956360818563	0.678082495710856\\
2.37456360818563	0.676129062260638\\
2.49956360818563	0.67451015902841\\
2.62456360818563	0.673170462694345\\
2.74956360818563	0.672059963058475\\
2.87456360818563	0.6711370825968\\
2.99956360818563	0.670372247653384\\
3.12456360818563	0.669739321268742\\
3.24956360818563	0.669214676614722\\
3.37456360818563	0.668778670798171\\
3.49956360818563	0.66841733203559\\
3.62456360818563	0.668118312176992\\
3.74956360818563	0.667870448959301\\
3.87456360818563	0.66766446229043\\
3.99956360818563	0.667493751338391\\
4.12456360818563	0.667352482340913\\
4.24956360818563	0.667235381796357\\
4.37456360818563	0.667138065416295\\
4.49956360818563	0.667057414702439\\
4.62456360818563	0.666990673551059\\
4.74956360818563	0.666935350548061\\
4.81217270613922	0.666911265128768\\
4.87478180409282	0.666889339104704\\
4.93739090204641	0.666869379826172\\
5	0.66685120964227\\
};
\addlegendentry{$\langle\text{ S}_\text{2}\rangle\text{(t)}$}

\addplot [color=green, dashdotted, line width=2.0pt]
  table[row sep=crcr]{%
0	0\\
0.000100475457260383	5.02314194582358e-05\\
0.000200950914520766	0.000100450222178378\\
0.00030142637178115	0.000150656410568844\\
0.000401901829041533	0.000200849987037635\\
0.000904279115343449	0.000451628774807697\\
0.00140665640164536	0.00070209262549841\\
0.00190903368794728	0.000952241839644842\\
0.0024114109742492	0.00120207671752405\\
0.00492329740575878	0.00244654655480787\\
0.00743518383726836	0.00368320287872335\\
0.00994707026877794	0.00491208293452641\\
0.0124589567002875	0.0061332238082172\\
0.0250183888578354	0.0121241281424412\\
0.0375778210153833	0.0179270245694221\\
0.0501372531729312	0.0235463722993485\\
0.0626966853304791	0.02898653721096\\
0.0991412937643315	0.0437975780064231\\
0.135585902198184	0.0572353494401202\\
0.172030510632036	0.0693937083334257\\
0.208475119065889	0.0803611576950339\\
0.257093885813804	0.0932814625263713\\
0.305712652561719	0.104411444050903\\
0.354331419309635	0.113918297020974\\
0.40295018605755	0.121956876117229\\
0.46627950019307	0.130459773511127\\
0.52960881432859	0.13699148818833\\
0.592938128464109	0.141802573766275\\
0.656267442599629	0.145120569742411\\
0.733658471248229	0.147443262414491\\
0.811049499896829	0.148152783459638\\
0.888440528545428	0.147523537737773\\
0.965831557194028	0.145800614492561\\
1.05793596976172	0.14261761324062\\
1.15004038232942	0.138468469319266\\
1.24214479489711	0.133593199535327\\
1.3342492074648	0.128203232510097\\
1.44286725998794	0.121423706047034\\
1.55148531251107	0.114375584460389\\
1.66010336503421	0.10722364787831\\
1.76872141755734	0.100112083509218\\
1.88893196521442	0.0924224291243308\\
2.00914251287149	0.0850002476399585\\
2.12935306052856	0.0779076868343403\\
2.24956360818563	0.0711991011319526\\
2.37456360818563	0.0646673812660732\\
2.49956360818563	0.0585888424369925\\
2.62456360818563	0.0529600282470089\\
2.74956360818563	0.0477773110138708\\
2.87456360818563	0.0430297403451633\\
2.99956360818563	0.0386916182635999\\
3.12456360818563	0.0347385923849993\\
3.24956360818563	0.0311483438435658\\
3.37456360818563	0.0278975046540668\\
3.49956360818563	0.0249585909052709\\
3.62456360818563	0.0223063173817782\\
3.74956360818563	0.0199176420689963\\
3.87456360818563	0.0177704893156483\\
3.99956360818563	0.0158425684352806\\
4.12456360818563	0.0141135755002288\\
4.24956360818563	0.0125650442349459\\
4.37456360818563	0.0111798516748117\\
4.49956360818563	0.00994182226490993\\
4.62456360818563	0.0088362940755339\\
4.74956360818563	0.00784994516657192\\
4.81217270613922	0.00739687025071491\\
4.87478180409282	0.00696920095174279\\
4.93739090204641	0.00656558347103006\\
5	0.00618472765868362\\
};
\addlegendentry{$\langle\text{ I}_\text{2}\rangle\text{(t)}$}

\addplot [color=red, line width=2.0pt]
  table[row sep=crcr]{%
0	0\\
0.000100475457260383	2.52361809227608e-09\\
0.000200950914520766	1.00936270452934e-08\\
0.00030142637178115	2.27087595394693e-08\\
0.000401901829041533	4.03677482552212e-08\\
0.000904279115343449	2.04276194182285e-07\\
0.00140665640164536	4.94091085712078e-07\\
0.00190903368794728	9.09654281233685e-07\\
0.0024114109742492	1.45080778990447e-06\\
0.00492329740575878	6.03491074524509e-06\\
0.00743518383726836	1.37351595448365e-05\\
0.00994707026877794	2.45319744467398e-05\\
0.0124589567002875	3.84058687347144e-05\\
0.0250183888578354	0.000153255492464255\\
0.0375778210153833	0.000342162387944045\\
0.0501372531729312	0.000602793756580233\\
0.0626966853304791	0.000932871119763479\\
0.0991412937643315	0.00226337598238135\\
0.135585902198184	0.00410839608608637\\
0.172030510632036	0.00641970465281561\\
0.208475119065889	0.00915214950802778\\
0.257093885813804	0.0133805196321258\\
0.305712652561719	0.0181930001040964\\
0.354331419309635	0.0235070227442485\\
0.40295018605755	0.0292469467410146\\
0.46627950019307	0.0372496083246957\\
0.52960881432859	0.0457273976118076\\
0.592938128464109	0.0545646750308842\\
0.656267442599629	0.0636582245685746\\
0.733658471248229	0.0749883084813072\\
0.811049499896829	0.0864346465094796\\
0.888440528545428	0.0978855304437856\\
0.965831557194028	0.109244014534151\\
1.05793596976172	0.122531392014506\\
1.15004038232942	0.135480245923496\\
1.24214479489711	0.148016637155446\\
1.3342492074648	0.160079078395227\\
1.44286725998794	0.173634933956343\\
1.55148531251107	0.18643950947165\\
1.66010336503421	0.198478042076712\\
1.76872141755734	0.209741325654249\\
1.88893196521442	0.221307773063019\\
2.00914251287149	0.231966171365444\\
2.12935306052856	0.241756984599231\\
2.24956360818563	0.250718403157192\\
2.37456360818563	0.259203556473288\\
2.49956360818563	0.266900998534597\\
2.62456360818563	0.273869509058646\\
2.74956360818563	0.280162725927655\\
2.87456360818563	0.285833177058036\\
2.99956360818563	0.290936134083016\\
3.12456360818563	0.295522086346259\\
3.24956360818563	0.299636979541713\\
3.37456360818563	0.303323824547763\\
3.49956360818563	0.306624077059139\\
3.62456360818563	0.309575370441229\\
3.74956360818563	0.312211908971703\\
3.87456360818563	0.314565048393922\\
3.99956360818563	0.316663680226328\\
4.12456360818563	0.318533942158858\\
4.24956360818563	0.320199573968697\\
4.37456360818563	0.321682082908893\\
4.49956360818563	0.323000763032651\\
4.62456360818563	0.324173032373408\\
4.74956360818563	0.325214704285367\\
4.81217270613922	0.325691864620517\\
4.87478180409282	0.326141459943553\\
4.93739090204641	0.326565036702798\\
5	0.326964062699047\\
};
\addlegendentry{$\langle\text{ R}_\text{2}\rangle\text{(t)}$}

\end{axis}
\end{tikzpicture}%}
	\\
	\subfloat[][Nodo 3]{% This file was created by matlab2tikz.
%
\definecolor{mycolor1}{rgb}{0.00000,0.44700,0.74100}%
\definecolor{mycolor2}{rgb}{0.85000,0.32500,0.09800}%
\definecolor{mycolor3}{rgb}{0.92900,0.69400,0.12500}%
%
\begin{tikzpicture}

\begin{axis}[%
width=0.39\columnwidth,
height=1.6in,
at={(1.011in,0.642in)},
scale only axis,
xmin=0,
xmax=60,
xlabel style={font=\color{white!15!black}},
xlabel={T},
ymin=0,
ymax=1,
axis background/.style={fill=white},
legend style={legend cell align=left, align=left, draw=none,fill=none}
]
\addplot [color=mycolor1, line width=2.0pt]
  table[row sep=crcr]{%
0	1\\
0.000167459095433972	0.999999998738142\\
0.000334918190867944	0.999999994952792\\
0.000502377286301916	0.999999988644289\\
0.000669836381735888	0.999999979812971\\
0.00150713185890575	0.999999897825981\\
0.00234442733607561	0.999999752819296\\
0.00318172281324547	0.99999954483513\\
0.00401901829041533	0.999999273915676\\
0.00820549567626463	0.999996976764418\\
0.0123919730621139	0.999993112547874\\
0.0165784504479632	0.999987686516463\\
0.0207649278338125	0.999980703907444\\
0.041697314763059	0.999922624618094\\
0.0626297016923055	0.999826408925982\\
0.083562088621552	0.99969269680636\\
0.104494475550799	0.999522120290137\\
0.209156410197031	0.998137960827395\\
0.313818344843264	0.995922465165433\\
0.418480279489496	0.992946027807028\\
0.523142214135729	0.989274937612289\\
0.693500300195606	0.981973940722739\\
0.863858386255483	0.973240038754978\\
1.03421647231536	0.963292530931682\\
1.20457455837524	0.952330841025336\\
1.44051447786383	0.935806342228975\\
1.67645439735243	0.918080997911094\\
1.91239431684103	0.899500127085142\\
2.14833423632963	0.880367859680095\\
2.44722092525848	0.85574171145015\\
2.74610761418733	0.83103916350853\\
3.04499430311618	0.806578409348801\\
3.34388099204503	0.782631978004113\\
3.7078444355742	0.75447211159862\\
4.07180787910338	0.727603684649032\\
4.43577132263256	0.702174223241644\\
4.79973476616174	0.678307316278175\\
5.20292074647974	0.653774724370522\\
5.60610672679773	0.631205361841917\\
6.00929270711573	0.610542558139064\\
6.41247868743373	0.591735043728731\\
6.85758807700814	0.57303680236404\\
7.30269746658256	0.556333540005082\\
7.74780685615697	0.541460456252127\\
8.19291624573139	0.528275695887157\\
8.68220109096914	0.51556125386222\\
9.17148593620689	0.504497996616038\\
9.66077078144463	0.494894174784991\\
10.1500556266824	0.486586354120359\\
10.6416653734136	0.479392347597474\\
11.1332751201447	0.473202420173063\\
11.6248848668759	0.46788601528397\\
12.1164946136071	0.463329681682318\\
12.6079732666127	0.459433513843843\\
13.0994519196183	0.456105361253136\\
13.5909305726239	0.453266757288285\\
14.0824092256295	0.450848775743308\\
14.573895267778	0.44879122631073\\
15.0653813099266	0.447042976012409\\
15.5568673520752	0.445559600087142\\
16.0483533942238	0.444301821295448\\
16.5398390198718	0.443235739886372\\
17.0313246455198	0.442333582646524\\
17.5228102711679	0.441571158188601\\
18.0142958968159	0.440926985076838\\
18.5057815459406	0.440382661675318\\
18.9972671950653	0.439923504076921\\
19.48875284419	0.439536690438452\\
19.9802384933148	0.439210799744039\\
20.5017696853463	0.438920754658941\\
21.0233008773778	0.438679302761171\\
21.5448320694093	0.438478627706432\\
22.0663632614408	0.43831175482993\\
22.6746017709939	0.438152012143916\\
23.282840280547	0.438023420740633\\
23.8910787901001	0.437920237430922\\
24.4993172996532	0.437837273729477\\
25.0869400675273	0.437772357070206\\
25.6745628354014	0.437719844553824\\
26.2621856032755	0.437677488337093\\
26.8498083711495	0.437643254426811\\
27.5306427936611	0.437611553445292\\
28.2114772161728	0.437586827964751\\
28.8923116386844	0.437567662580957\\
29.573146061196	0.437552729805634\\
30.3571240725376	0.437539406333623\\
31.1411020838792	0.43752943285221\\
31.9250800952207	0.437522069969169\\
32.7090581065623	0.437516567914547\\
33.6393150845161	0.437511657612987\\
34.5695720624699	0.437508192616622\\
35.4998290404237	0.437505840190485\\
36.4300860183775	0.437504189241777\\
37.5675806203718	0.437502680495585\\
38.7050752223662	0.437501704404973\\
39.8425698243606	0.437501154963458\\
40.980064426355	0.437500810027397\\
42.4285780340909	0.437500423216806\\
43.8770916418269	0.437500205882951\\
45.3256052495628	0.437500155093005\\
46.7741188572988	0.437500139033043\\
48.2741188572988	0.437500070492889\\
49.7741188572988	0.437500032582013\\
51.2741188572988	0.437500025442845\\
52.7741188572988	0.437500024043849\\
54.2741188572988	0.437500012593451\\
55.7741188572988	0.437500006149462\\
57.2741188572988	0.437500004486816\\
58.7741188572988	0.43750000393094\\
59.0805891429741	0.437500003525987\\
59.3870594286494	0.437500003162157\\
59.6935297143247	0.437500002835375\\
60	0.437500002541907\\
};
\addlegendentry{$\langle\text{ S}_\text{3}\rangle\text{(t)}$}

\addplot [color=mycolor2, line width=2.0pt]
  table[row sep=crcr]{%
0	0\\
0.000167459095433972	1.26185129491217e-09\\
0.000334918190867944	5.04715161691134e-09\\
0.000502377286301916	1.13555206521399e-08\\
0.000669836381735888	2.01865781270714e-08\\
0.00150713185890575	1.02168885342488e-07\\
0.00234442733607561	2.47161385709955e-07\\
0.00318172281324547	4.55116590395451e-07\\
0.00401901829041533	7.25987035736679e-07\\
0.00820549567626463	3.02240839189113e-06\\
0.0123919730621139	6.88460568276832e-06\\
0.0165784504479632	1.23066742198438e-05\\
0.0207649278338125	1.92827249369075e-05\\
0.041697314763059	7.7267642315786e-05\\
0.0626297016923055	0.000173227718853331\\
0.083562088621552	0.000306444247749785\\
0.104494475550799	0.000476207943859229\\
0.209156410197031	0.00184891650923723\\
0.313818344843264	0.00403428386881361\\
0.418480279489496	0.00695384907760906\\
0.523142214135729	0.0105339891355051\\
0.693500300195606	0.0175971136151213\\
0.863858386255483	0.0259614923743754\\
1.03421647231536	0.0353880052833885\\
1.20457455837524	0.0456606118346815\\
1.44051447786383	0.0609285917757482\\
1.67645439735243	0.0770270262868401\\
1.91239431684103	0.0935965284435017\\
2.14833423632963	0.110324114104919\\
2.44722092525848	0.13133553630602\\
2.74610761418733	0.151803393776639\\
3.04499430311618	0.17143301715889\\
3.34388099204503	0.189977555508\\
3.7078444355742	0.210833991668167\\
4.07180787910338	0.229677134951398\\
4.43577132263256	0.246438421785966\\
4.79973476616174	0.261066449116606\\
5.20292074647974	0.274785754990105\\
5.60610672679773	0.286039368883271\\
6.00929270711573	0.294983433116465\\
6.41247868743373	0.30175643196477\\
6.85758807700814	0.306899090107461\\
7.30269746658256	0.309867303731916\\
7.74780685615697	0.310918643619185\\
8.19291624573139	0.310273875840171\\
8.68220109096914	0.307859497934145\\
9.17148593620689	0.30394952537288\\
9.66077078144463	0.298802849798728\\
10.1500556266824	0.292638032935273\\
10.6416653734136	0.285615796252768\\
11.1332751201447	0.277950926739595\\
11.6248848668759	0.269801213188547\\
12.1164946136071	0.261301267061574\\
12.6079732666127	0.252569321908963\\
13.0994519196183	0.243702027756301\\
13.5909305726239	0.234781977080018\\
14.0824092256295	0.225879536525819\\
14.573895267778	0.217053038249217\\
15.0653813099266	0.208348121918181\\
15.5568673520752	0.199801976342069\\
16.0483533942238	0.191445697640121\\
16.5398390198718	0.183303671016761\\
17.0313246455198	0.175392222310533\\
17.5228102711679	0.167723699389534\\
18.0142958968159	0.160307635901842\\
18.5057815459406	0.153150116044107\\
18.9972671950653	0.146252792268445\\
19.48875284419	0.139615587093586\\
19.9802384933148	0.133237287300585\\
20.5017696853463	0.126749085509615\\
21.0233008773778	0.120543262208316\\
21.5448320694093	0.114613005008405\\
22.0663632614408	0.108951184514624\\
22.6746017709939	0.102676735601474\\
23.282840280547	0.0967423452585563\\
23.8910787901001	0.0911334321360161\\
24.4993172996532	0.0858358971254951\\
25.0869400675273	0.0810003106792662\\
25.6745628354014	0.0764286533064571\\
26.2621856032755	0.0721080243560162\\
26.8498083711495	0.0680260834605836\\
27.5306427936611	0.0635793731585896\\
28.2114772161728	0.0594186100567405\\
28.8923116386844	0.0555263353843065\\
29.573146061196	0.0518861460511328\\
30.3571240725376	0.0479865022042955\\
31.1411020838792	0.0443777124961245\\
31.9250800952207	0.0410385808014444\\
32.7090581065623	0.0379494355212916\\
33.6393150845161	0.0345829658368376\\
34.5695720624699	0.0315142114252991\\
35.4998290404237	0.0287171284645176\\
36.4300860183775	0.0261678325282954\\
37.5675806203718	0.0233554939459096\\
38.7050752223662	0.0208451311709022\\
39.8425698243606	0.0186045988786763\\
40.980064426355	0.0166047671655161\\
42.4285780340909	0.0143654118090185\\
43.8770916418269	0.0124280915770715\\
45.3256052495628	0.0107526786215321\\
46.7741188572988	0.00930318087347226\\
48.2741188572988	0.00800698643404652\\
49.7741188572988	0.00689145956002702\\
51.2741188572988	0.00593185767404391\\
52.7741188572988	0.00510593367583665\\
54.2741188572988	0.00439450787540565\\
55.7741188572988	0.00378225394920435\\
57.2741188572988	0.00325559290546576\\
58.7741188572988	0.00280230102039895\\
59.0805891429741	0.00271772186404062\\
59.3870594286494	0.00263569546579028\\
59.6935297143247	0.00255614480597557\\
60	0.00247899512024932\\
};
\addlegendentry{$\langle\text{ I}_\text{3}\rangle\text{(t)}$}

\addplot [color=mycolor3, line width=2.0pt]
  table[row sep=crcr]{%
0	0\\
0.000167459095433972	6.98961701446131e-15\\
0.000334918190867944	5.63679759927061e-14\\
0.000502377286301916	1.90163978948918e-13\\
0.000669836381735888	4.50699263636942e-13\\
0.00150713185890575	5.13323666107007e-12\\
0.00234442733607561	1.93184171486756e-11\\
0.00318172281324547	4.82799544605555e-11\\
0.00401901829041533	9.72878179088969e-11\\
0.00820549567626463	8.27190098121791e-10\\
0.0123919730621139	2.8464430794138e-09\\
0.0165784504479632	6.80931707495681e-09\\
0.0207649278338125	1.33676191129037e-08\\
0.041697314763059	1.07739590322502e-07\\
0.0626297016923055	3.63355164822163e-07\\
0.083562088621552	8.58945889800579e-07\\
0.104494475550799	1.67176600367631e-06\\
0.209156410197031	1.31226633678519e-05\\
0.313818344843264	4.32509657533504e-05\\
0.418480279489496	0.000100123115363044\\
0.523142214135729	0.000191073252205499\\
0.693500300195606	0.000428945662140103\\
0.863858386255483	0.000798468870646642\\
1.03421647231536	0.00131946378492968\\
1.20457455837524	0.002008547139982\\
1.44051447786383	0.00326506599527666\\
1.67645439735243	0.00489197580206599\\
1.91239431684103	0.00690334447135607\\
2.14833423632963	0.00930802621498689\\
2.44722092525848	0.0129227522438305\\
2.74610761418733	0.0171574427148301\\
3.04499430311618	0.0219885734923091\\
3.34388099204503	0.0273904664878871\\
3.7078444355742	0.0346938967332132\\
4.07180787910338	0.0427191803995696\\
4.43577132263256	0.0513873549723903\\
4.79973476616174	0.0606262346052192\\
5.20292074647974	0.0714395206393737\\
5.60610672679773	0.0827552692748121\\
6.00929270711573	0.0944740087444709\\
6.41247868743373	0.106508524306498\\
6.85758807700814	0.120064107528498\\
7.30269746658256	0.133799156263002\\
7.74780685615697	0.147620900128689\\
8.19291624573139	0.161450428272672\\
8.68220109096914	0.176579248203635\\
9.17148593620689	0.191552478011082\\
9.66077078144463	0.206302975416281\\
10.1500556266824	0.220775612944368\\
10.6416653734136	0.234991856149758\\
11.1332751201447	0.248846653087342\\
11.6248848668759	0.262312771527483\\
12.1164946136071	0.275369051256108\\
12.6079732666127	0.287997164247194\\
13.0994519196183	0.300192610990563\\
13.5909305726239	0.311951265631696\\
14.0824092256295	0.323271687730873\\
14.573895267778	0.334155735440053\\
15.0653813099266	0.344608902069409\\
15.5568673520752	0.354638423570789\\
16.0483533942238	0.364252481064431\\
16.5398390198718	0.373460589096867\\
17.0313246455198	0.382274195042944\\
17.5228102711679	0.390705142421865\\
18.0142958968159	0.398765379021319\\
18.5057815459406	0.406467222280575\\
18.9972671950653	0.413823703654635\\
19.48875284419	0.420847722467962\\
19.9802384933148	0.427551912955376\\
20.5017696853463	0.434330159831444\\
21.0233008773778	0.440777435030513\\
21.5448320694093	0.446908367285164\\
22.0663632614408	0.452737060655446\\
22.6746017709939	0.45917125225461\\
23.282840280547	0.465234234000811\\
23.8910787901001	0.470946330433061\\
24.4993172996532	0.476326829145028\\
25.0869400675273	0.481227332250528\\
25.6745628354014	0.485851502139719\\
26.2621856032755	0.490214487306891\\
26.8498083711495	0.494330662112605\\
27.5306427936611	0.498809073396118\\
28.2114772161728	0.502994561978508\\
28.8923116386844	0.506906002034737\\
29.573146061196	0.510561124143233\\
30.3571240725376	0.514474091462081\\
31.1411020838792	0.518092854651666\\
31.9250800952207	0.521439349229387\\
32.7090581065623	0.524533996564161\\
33.6393150845161	0.527905376550175\\
34.5695720624699	0.530977595958079\\
35.4998290404237	0.533777031344997\\
36.4300860183775	0.536327978229927\\
37.5675806203718	0.539141825558506\\
38.7050752223662	0.541653164424125\\
39.8425698243606	0.543894246157866\\
40.980064426355	0.545894422807086\\
42.4285780340909	0.548134164974176\\
43.8770916418269	0.550071702539977\\
45.3256052495628	0.551747166285463\\
46.7741188572988	0.553196680093485\\
48.2741188572988	0.554492943073064\\
49.7741188572988	0.55560850785796\\
51.2741188572988	0.556568116883111\\
52.7741188572988	0.557394042280314\\
54.2741188572988	0.558105479531143\\
55.7741188572988	0.558717739901334\\
57.2741188572988	0.559244402607718\\
58.7741188572988	0.559697695048661\\
59.0805891429741	0.559782274609972\\
59.3870594286494	0.559864301372052\\
59.6935297143247	0.559943852358649\\
60	0.560021002337844\\
};
\addlegendentry{$\langle\text{ R}_\text{3}\rangle\text{(t)}$}

\end{axis}
\end{tikzpicture}%}
	\subfloat[][Prevalenza]{% This file was created by matlab2tikz.
%
%The latest updates can be retrieved from
%  http://www.mathworks.com/matlabcentral/fileexchange/22022-matlab2tikz-matlab2tikz
%where you can also make suggestions and rate matlab2tikz.
%
\definecolor{mycolor1}{rgb}{0.00000,0.44700,0.74100}%
%
\begin{tikzpicture}

\begin{axis}[%
width=6.028in,
height=4.754in,
at={(1.011in,0.642in)},
scale only axis,
xmin=0,
xmax=60,
ymin=0,
ymax=0.6,
axis background/.style={fill=white},
axis x line*=bottom,
axis y line*=left
]
\addplot [color=mycolor1, line width=2.0pt, forget plot]
  table[row sep=crcr]{%
0	0.333333333333333\\
0.000167459095433972	0.333346357799813\\
0.000334918190867944	0.333359382006611\\
0.000502377286301916	0.333372405953696\\
0.000669836381735888	0.33338542964104\\
0.00150713185890575	0.333450544180612\\
0.00234442733607561	0.333515652222244\\
0.00318172281324547	0.333580753762272\\
0.00401901829041533	0.333645848797031\\
0.00820549567626463	0.333971226263809\\
0.0123919730621139	0.334296440551265\\
0.0165784504479632	0.334621491205169\\
0.0207649278338125	0.334946377773328\\
0.041697314763059	0.336568333634319\\
0.0626297016923055	0.33818612068879\\
0.083562088621552	0.339799684673782\\
0.104494475550799	0.341408972533459\\
0.209156410197031	0.34938946665612\\
0.313818344843264	0.357255789085086\\
0.418480279489496	0.365002602843887\\
0.523142214135729	0.372625132859308\\
0.693500300195606	0.38475568848826\\
0.863858386255483	0.39653142957701\\
1.03421647231536	0.407941031431478\\
1.20457455837524	0.418975299094782\\
1.44051447786383	0.433624794438073\\
1.67645439735243	0.447533637142379\\
1.91239431684103	0.460699895656536\\
2.14833423632963	0.473123944927263\\
2.44722092525848	0.487803848743562\\
2.74610761418733	0.501326422004098\\
3.04499430311618	0.513722744744762\\
3.34388099204503	0.525021947751027\\
3.7078444355742	0.53734651471454\\
4.07180787910338	0.548172156268076\\
4.43577132263256	0.55757753265428\\
4.79973476616174	0.565631517927584\\
5.20292074647974	0.573059119352328\\
5.60610672679773	0.579022490830576\\
6.00929270711573	0.583622708347856\\
6.41247868743373	0.586949280746428\\
6.85758807700814	0.589245817518974\\
7.30269746658256	0.59021550437947\\
7.74780685615697	0.589968512521601\\
8.19291624573139	0.588604520886292\\
8.68220109096914	0.585928312856786\\
9.17148593620689	0.582135329410503\\
9.66077078144463	0.577334567222661\\
10.1500556266824	0.571627548252987\\
10.6416653734136	0.565075967645812\\
11.1332751201447	0.557790677326464\\
11.6248848668759	0.54985109756196\\
12.1164946136071	0.541333503551458\\
12.6079732666127	0.532311188982125\\
13.0994519196183	0.522842721346191\\
13.5909305726239	0.51298544629886\\
14.0824092256295	0.502795214294804\\
14.573895267778	0.492323751703937\\
15.0653813099266	0.481615243092825\\
15.5568673520752	0.470711634011877\\
16.0483533942238	0.459653868755485\\
16.5398390198718	0.448479997446346\\
17.0313246455198	0.437222425317004\\
17.5228102711679	0.425912050817231\\
18.0142958968159	0.414578845656627\\
18.5057815459406	0.403250584696953\\
18.9972671950653	0.391951140591323\\
19.48875284419	0.380703112020529\\
19.9802384933148	0.369528133418102\\
20.5017696853463	0.357772098713114\\
21.0233008773778	0.346141209962941\\
21.5448320694093	0.334654425380947\\
22.0663632614408	0.323329522891469\\
22.6746017709939	0.310347602531035\\
23.282840280547	0.297628841593494\\
23.8910787901001	0.285191935057362\\
24.4993172996532	0.273053651900737\\
25.0869400675273	0.26162370596486\\
25.6745628354014	0.250495496367714\\
26.2621856032755	0.239677190944159\\
26.8498083711495	0.22917536145636\\
27.5306427936611	0.217409719439121\\
28.2114772161728	0.206079845103249\\
28.8923116386844	0.195187736872762\\
29.573146061196	0.184733168408286\\
30.3571240725376	0.173233522006174\\
31.1411020838792	0.16230395931065\\
31.9250800952207	0.151934828774627\\
32.7090581065623	0.142113892197481\\
33.6393150845161	0.131151668187394\\
34.5695720624699	0.120914545571232\\
35.4998290404237	0.111373434680736\\
36.4300860183775	0.102496709995189\\
37.5675806203718	0.0924969177596037\\
38.7050752223662	0.0833818295618134\\
39.8425698243606	0.075091428348473\\
40.980064426355	0.0675644223338172\\
42.4285780340909	0.0589891501052424\\
43.8770916418269	0.0514425847476294\\
45.3256052495628	0.0448184172275052\\
46.7741188572988	0.0390124577659239\\
48.2741188572988	0.0337602629542047\\
49.7741188572988	0.0291940254167815\\
51.2741188572988	0.0252314599893827\\
52.7741188572988	0.0217949251050357\\
54.2741188572988	0.0188152783318617\\
55.7741188572988	0.0162364315482906\\
57.2741188572988	0.0140072500712239\\
58.7741188572988	0.0120805138329808\\
59.0805891429741	0.0117201716201511\\
59.3870594286494	0.0113704541399994\\
59.6935297143247	0.0110310555073265\\
60	0.0107016779231368\\
};
\end{axis}
\end{tikzpicture}%}
	\caption[Sperimentazione in MATLAB relativo al grafo~\ref{fig::3nodi}]{Divisione in classi nei singoli nodi (a)(b)(c) e grafico della prevalenza (d) per il grafo~\ref{fig::3nodi}.   Per ottenere i grafici abbiamo risolto numericamente,  usando MATLAB,  il problema di Cauchy~\eqref{3nodi} con condizioni iniziali  di stati puri~\eqref{statipuri}. Abbiamo,  inoltre,  supposto l'indipendenza statistica delle condizioni iniziali.\\
		Per la sperimentazioni abbiamo usato come parametri $\tau= 0.3$ e $\gamma =0.1 $.}\label{fig::spe3nodi}
\end{figure}
Poich\`e $\angol{S_1}(0)=0$ si ha che $\angol{S_1}(t)=0$ \`e soluzione del problema di Cauchy.  Osserviamo,  inoltre, che $\angol{I_2}(t)\geq \angol{I_3}(t)$ infatti il nodo $2$ ha due strade per infettarsi (di cui una inizialmente chiusa $\angol{I_3}(0)=0$) mentre il nodo $3$ ne ha una sola (che inizialmente \`e chiusa). 



%%% Local Variables:
%%% mode: latex
%%% TeX-master: "main"
%%% End:


